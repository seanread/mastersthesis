\section{geometry}
\subsection{Introduction}

The reconstruction of nasal cavity model computer images - including discretisation for modeling with CFD - can be divided into 5 stages: image aquisition, data conversion, segmentation, refinement and meshing. Firstly medical imaging technologies are used to extract a series of pixelated slices, in which the various materials which make up the human body can be identified by variation in greyscale measurements. These slices are then interpolated to create a 3D structure of voxels (the 3D equivalent of a pixel).

The reconstruction of the nasal cavity geometries can be very time consuming and prone to human error. The improvement on and automation of the reconstruction process are areas of active research. This chapter presents an overview of the methods in current use for the various phases (mentioned above) by which a nasal cavity geometry is prepared for CFD analysis.

\subsection{Non-invasive Medical Imaging}

There are various forms of medical imaging that can be used for the identification of the nasal cavity geometry. Here a brief overview of Computed Tomography (CT), the form of medical imaging used for the nasal cavity geometries presented in this thesis.

CT scans use a series of x-rays taken at regular intervals within the required region. x-ray scans use photons sent in beams through the region of interest. These photons interact differently with the material that they encounter depending on its density. Electronic detectors feed the photon pattern emitted from the body in question to a computer, which uses this information to create images. These images are made up of pixels which are each assigned a grey scale value based on their x-ray attenuation coefficient. In medical imaging the Hounsfield scale is used, whereas in image processing the greyscale is numbered from black to white between 0 to 255. Figure \ref{fig:greyscale} shows an  

\begin{figure}
  \includegraphics[width=1\textwidth]{grayscale.png}
\caption{example of how greyscale values are assigned to pixels in a ct image} \label{fig:greyscale}
\centering
\end{figure}
 
\subsection{Image Segmentation} 

On ce the CT scan has been outputted as a series of voxels, the anatomical geometries relevant to the project at hand can be extracted. To achieve this goal, the voxels that make up the relevant chapter of the data need to be identified somehow. This can be done manually - by going through the many slices that make up the ct scan and identifying the relevant areas - however this process is extremely time consuming.
\ subsection{Segmentation Methods}

Numerous algorithms have been developed for the purpose of automating the process; all of which have certain advantages and disadvantages. For the most part these algorithms can be subsumed in to three categories, presented below (in order of increasing complexity):

\begin{itemize}

  \item \textbf{Thresholding:} regions are identified and separated according their  greyscale value. This is the simplest algorithm for defining regions.

  \item \textbf{Edge detection:} either local maxima of the first derivite of intensity or zero values of the second derivative are used to identify region edges. Edge detection tends to reduce the noise when compared with thresholding

  \item	\textbf{Region based:} regions are grown from the inside out. This produces more coherent regions, but is unable to detect regions that are segregated.

\end{itemize}

These methods and many more exist and are in active development. In addition, many of them are available for free online. The implementation of such algorithms and applying from first principles, however is quite a complex procedure. Fortunately many of these algorithms have already been implemented in software packages - of which commercial and open source variations are available - which are specifically compiled for working with medical imaging outputs. 

\subsection{Preparation of Model for Meshing}

Once the region in question has been extracted from the medical imaging output, it can be saved as some CAD format. There are certain criteria that need to be fulfilled by a CAD model before it can be read by 3d meshing software and used to create a mesh. It is often the case that when the data is exported from the medical imaging software is 'dirty'; that there are contradictory or arbitrary elements in the geometry. These elements need to be removed through the use of CAD software before the geometry is suitable for the creation of a useable 'clean' 3d mesh. Note that the geometry needs to contain also topological infortation in order to create a watertight mesh. The details of the process by which the geometry is prepared for meshing are described in Figure \ref{fig:segchart}.


\begin{figure}
  \includegraphics[width=1\textwidth]{flowchart.png}
\caption{Flow chart showing the process by which an anatomical geometry is prepared - with the help of medical imaging - for modeling with CFDs} \label{fig:segchart}
\centering
\end{figure}

\subsection{Summary}

Anatomical geometry reconstruction - including that of respiratory systems - begins with medical imaging of the area in question. The geometry is then extracted from the medical imaging data using one (or several) of the available extraction algorithms outlined above. After extraction, cleaning of the geometry with CAD software is generally necessary. It is necessary to ensure that the geometry is water-tight before it can be meshed. This chapter has given an overview of some of the more pertinent methods in current use for the reconstruction of anatomical geometries from medical imaging data. Figure \ref{fig:cavzamp} shows an example of the construction process.

\begin{figure}

  \caption{Example of nasal cavity geometry generation}\label{fig:cavzamp}
\end{figure}
 
\section{Meshing}
\subsection{Introduction}

The navier stokes equations, upon which Computational Fluid Dynamics is based (see chapter \ref{cfd}), is unsolvable (in most instances), except by approximation. A given geometry is approximated, or discretised, as a series of points. The process by which these points are defined and related to one another is known as meshing; this in itself is a complex discipline under active research. In this chapter current methods are reviewed, and guidelines are given for developing quality meshes.

\subsection{Mesh Types} \label{mtypes}

There are many types of mesh structure that can be employed; all of them have their own advantages and disadvantages. 

\begin{itemize}
  \item{Structured meshes} 
    by definition, are divided into segments of uniform size and shape. It is characterised by cells posessing either four nodal corner points in two dimensions, or eight three. The points are mutually orthogonal and cartesionally defined. Being defined in this relatively simple way facilitates a higher level of computational efficiency. It is, however, limited in the level of structural complexity that it can accomodate.

  \item{Unstructured Meshes}
    The geometries encountered in the respiratory system are generally too complex to be effectively discretised in a structured manner; in cases such as these unstructured meshes can be used to acommodate the complexities of the given geometry. Unstructured meshes - usually constructed from triangles or tetrahedra - do not fit a regular pattern, and they do not have coordinate lines corresponding to curvilinear directions. Because of this, the solving of compuations over unstructured domains is generally more computationally intensive, however with modern advances in computers this has become less significant an issue in many cases.

  \item{Hybrid meshes}
    One disadvantage to the use of unstructured meshes is that they tent to show less accuracy near the wall. One commonly applied solution is the use of hexahedral elements near the wall, with the rest of the volume filled with unstructured - usually tetrahedral - elements. This method tends to improve the accuracy of near wall computations. One draw back of this method is that the prism layers can break down in the vicinity of excessively contoured walls. In this study this is the mesh type that is employed.

\end{itemize}

\begin{figure}

  \caption{Examples of the mesh types described in section \ref{mtypes}} \label{fig:struct}
\end{figure}

\subsection{Meshing algorithms}
There are various meshing algorithms available, each with their own strengths and weaknesses. For the purpose of this study, an octree algorith was selected. Octree algorithms work by repeatedly dividing The volume in to smaller sections, until the given criteria, for example mesh size, is fulfilled. This method is generally considered to be a relatively simple but robust approach to mesh generation. One drawback, however is that it can cause irregular element distributions near the boundary.

\subsection{Quality}

The quality of a generated mesh is dependant on its warp angle, skewness and aspect ratio. For a quadrilateral cell, as shown in figure \ref{fig:mqual}, the aspect ratio of the cell is defined as $AR = \frac{\Delta y} {\Delta x}$. Within the interior region, the $AR$ should be maintained within the range $0.2 < AR < 5$. This can be somewhat relaxed, however, in the vicinity of the wall.

Mesh skewness is defined as the extent to which it deviates in shape from the ideal. This is a squate for quadrilateral cells, or an equilateral triangle for triangle or tetrahedral cells. It is simple quantified for tiangles and tetrahedrals as $\frac{\theta ideal - \theta actual} {\theta ideal}$ (see figure \ref{fig:mqual} for theta).

For unstructured mesh, the warp angle (shown in figure \ref{fig:unqual}) should be kept below \SI{75}{\degree}.

Many grid generation packages contain specific algorithms and/or functions for improving mesh quality. The gradient of mesh size variation should not exceed 1.2, as higher variations can cause problems in convergance.

\begin{figure}
  \includegraphics[width=1\textwidth]{mqual}
  \caption{Example of mesh cell with spacing $\Delta x$ , $\Delta y$ and angle $ \theta $ between the grid lines along with high $AR$ triangular and quadrilateral elements } \label{fig:mqual}
\end{figure}

\subsection{Mesh Independence}

A significant source of error in the solution of CFD problems is derived from the discretisation process; when a system is separated into a number of finite elements, for the purpose of Numerical solution, the solution that is obtained from its solution is an approximation. It is necessary, then, to ascertain the required resolution, or mesh size, required to calculate a result that approximates the exact solution to a satisfactory degree of accuracy. This is done by means of a mesh independence test. This entails the monitoring of one - or several - fluid flow parameters of interest to the study over a series of mesh sizes. Independence is said to have been achieved when the effect of mesh size on the selected flow variable(s) has become insignificant. Figure \ref{fig:mind} shows the results from the mesh independence test conducted for two of the five models presented in this thesis.

\begin{figure}

  \caption{mesh independence} \label{fig:mind}

\end{figure}
\subsection{Meshing of the Nasal Cavity}

In this section the relevant aspects of the theory behind mesh generation have been introduced. Figure \ref{fig:cavme} shows the meshing process applied to one of the nasal cavities geometries presented in this thesis. Note the use of hybrid mesh, with internal tetrahedral elements and close to wall prism layers, shown clearly in the cross sections. Also, note that the geometry has been previously cut into sections relevant to the planned analysis in the CAD refinement stage.

\begin{figure}
  \caption{mesh generation process} \label{fig:cavme}
\end{figure}
