% This file was created with JabRef 2.10.
% Encoding: UTF8


@Article{2000,
  Title                    = {Sublayer model for deposition of nano- and micro-particles in turbulent flows},
  Author                   = {sShams, M. and Ahmadi, G. and Rahimzadeh, H.},
  Journal                  = {Chemical Engineering Science},
  Year                     = {2000},
  Note                     = {Cited By (since 1996): 14
Export Date: 6 June 2011
Source: Scopus},
  Number                   = {24},
  Pages                    = {6097-6107},
  Volume                   = {55},

  Type                     = {Journal Article},
  Url                      = {http://www.scopus.com/inward/record.url?eid=2-s2.0-0034351953&partnerID=40&md5=5987248a9f0a15970047c6d05275ae7f}
}

@Article{A¼rge-Vorsatz2007,
  Title                    = {Appraisal of policy instruments for reducing buildings' CO2 emissions},
  Author                   = {ürge-Vorsatz, Diana and Koeppel, Sonja and Mirasgedis, Sebastian},
  Journal                  = {Building Research \& Information},
  Year                     = {2007},
  Number                   = {4},
  Pages                    = {458-477},
  Volume                   = {35},

  Doi                      = {10.1080/09613210701327384},
  ISSN                     = {0961-3218},
  Type                     = {Journal Article},
  Url                      = {http://dx.doi.org/10.1080/09613210701327384}
}

@Article{Abanto2004,
  Title                    = {Airflow modelling in a computer room},
  Author                   = {Abanto, Juan and Barrero, Daniel and Reggio, Marcelo and Ozell, Benoı̂t},
  Journal                  = {Building and Environment},
  Year                     = {2004},
  Number                   = {12},
  Pages                    = {1393-1402},
  Volume                   = {39},

  Doi                      = {10.1016/j.buildenv.2004.03.011},
  ISSN                     = {0360-1323},
  Keywords                 = {HVAC
Thermal comfort
CFD
Visualisation},
  Type                     = {Journal Article},
  Url                      = {http://www.sciencedirect.com/science/article/pii/S0360132304001167}
}

@Article{Abe2001,
  Title                    = {Direct Numerical Simulation of a Fully Developed Turbulent Channel Flow With Respect to the Reynolds Number Dependence},
  Author                   = {Abe, H. and Kawamura, H. and Matsuo, Y.},
  Journal                  = {Journal of Fluids Engineering},
  Year                     = {2001},
  Pages                    = {382-393},
  Volume                   = {123},

  Type                     = {Journal Article}
}

@Article{Abrahamsson1994,
  Title                    = {A turbulent plane two-dimensional wall jet in a quiescent surrounding},
  Author                   = {Abrahamsson, H. and Johansson, B. and Löfdahl, L.},
  Journal                  = {European Journal of Mechanics B. Fluids},
  Year                     = {1994},
  Pages                    = {533-556},
  Volume                   = {13},

  Type                     = {Journal Article}
}

@Article{Adams2012,
  Title                    = {Effects of Device and Formulation on In Vitro Performance of Dry Powder Inhalers},
  Author                   = {Adams, WallaceP and Lee, SauL and Plourde, Robert and Lionberger, RobertA and Bertha, CraigM and Doub, WilliamH and Bovet, Jean-Marc and Hickey, AnthonyJ},
  Journal                  = {The AAPS Journal},
  Year                     = {2012},
  Number                   = {3},
  Pages                    = {400-409},
  Volume                   = {14},

  Doi                      = {10.1208/s12248-012-9352-7},
  Keywords                 = {device modifications
dry powder inhaler
formulation
generic
in vitro performance},
  Type                     = {Journal Article},
  Url                      = {http://dx.doi.org/10.1208/s12248-012-9352-7}
}

@Article{Agnew1984,
  Title                    = {Sedimentational deposition of 5 um aerosol particles in the small airways of the human lung},
  Author                   = {Agnew, J. E. and Pavia, D. and Clarke, S. W.},
  Journal                  = {Journal of Aerosol Science},
  Year                     = {1984},
  Note                     = {doi: DOI: 10.1016/0021-8502(84)90006-5},
  Number                   = {6},
  Pages                    = {683-695},
  Volume                   = {15},

  ISSN                     = {0021-8502},
  Type                     = {Journal Article},
  Url                      = {http://www.sciencedirect.com/science/article/B6V6B-48BDPMW-84/2/b4ecf6d3b3f8dfc538044b1f6173df45}
}

@PhdThesis{Ahmadi1970,
  Title                    = {Analytical Prediction of Turbulent Dispersion of Finite Size Particle.},
  Author                   = {Ahmadi, G.},
  Year                     = {1970},
  Type                     = {Ph.D. Thesis},

  University               = {Purdue University}
}

@Article{Ahmadi1998,
  Title                    = {Dispersion and deposition of particles in a turbulent pipe flow with sudden expansion},
  Author                   = {Ahmadi, G. and Chen, Q.},
  Journal                  = {Journal of Aerosol Science},
  Year                     = {1998},
  Note                     = {Cited By (since 1996): 20
Export Date: 2 June 2011
Source: Scopus},
  Number                   = {9},
  Pages                    = {1097-1116},
  Volume                   = {29},

  Type                     = {Journal Article},
  Url                      = {http://www.scopus.com/inward/record.url?eid=2-s2.0-0031824772&partnerID=40&md5=f3256bf58a9d6546dc70f5618296c887}
}

@Article{Ahmadi1971,
  Title                    = {Motion of Particles in a Turbulent Fluid---The Basset History Term},
  Author                   = {Ahmadi, G. and Goldschmidt, V. W.},
  Journal                  = {Journal of Applied Mechanics},
  Year                     = {1971},
  Number                   = {2},
  Pages                    = {561-563},
  Volume                   = {38},

  Type                     = {Journal Article},
  Url                      = {http://link.aip.org/link/?AMJ/38/561/2}
}

@Article{Ahmadi1998a,
  Title                    = {Particle transport and deposition in a hot-gas cleanup pilot plant},
  Author                   = {Ahmadi, G. and Smith, D. H.},
  Journal                  = {Aerosol Science and Technology},
  Year                     = {1998},
  Note                     = {Cited By (since 1996): 27
Export Date: 2 June 2011
Source: Scopus},
  Number                   = {3},
  Pages                    = {183-205},
  Volume                   = {29},

  Type                     = {Journal Article},
  Url                      = {http://www.scopus.com/inward/record.url?eid=2-s2.0-0032167813&partnerID=40&md5=cacb9feee2be615b0fd2e2f9c23f07e2}
}

@Article{Ahmed2000,
  Title                    = {On the mechanisms of modifying the structure of turbulent homogeneous shear flows by dispersed particles},
  Author                   = {Ahmed, A. M. and Elghobashi, S.},
  Journal                  = {Physics of Fluids},
  Year                     = {2000},
  Note                     = {Cited By (since 1996): 46
Export Date: 2 June 2011
Source: Scopus},
  Number                   = {11},
  Pages                    = {2906-2930},
  Volume                   = {12},

  Type                     = {Journal Article},
  Url                      = {http://www.scopus.com/inward/record.url?eid=2-s2.0-0034327068&partnerID=40&md5=ee7796e37a4c5c33b65716e933f5efc6}
}

@TechReport{AIAA1998,
  Title                    = {Guide for the Verification and Validation of Computational Fluid Dynamics},
  Author                   = {AIAA},
  Institution              = {American Institute of Aeronautics and Astronautics},
  Year                     = {1998},
  Type                     = {Report}
}

@Article{Aidun2003,
  Title                    = {Dynamics of particle sedimentation in a vertical channel: Period-doubling bifurcation and chaotic state},
  Author                   = {Aidun, C. K. and Ding, E. J.},
  Journal                  = {Physics of Fluids},
  Year                     = {2003},
  Note                     = {Cited By (since 1996): 6
Export Date: 2 June 2011
Source: Scopus},
  Number                   = {6},
  Pages                    = {1612-1621},
  Volume                   = {15},

  Type                     = {Journal Article},
  Url                      = {http://www.scopus.com/inward/record.url?eid=2-s2.0-0038168790&partnerID=40&md5=e78554be9b1900c281717f23bf078fcb}
}

@Article{Aidun1998,
  Title                    = {A new method for analysis of the fluid interaction with a deformable membrane},
  Author                   = {Aidun, C. K. and Qi, D. W.},
  Journal                  = {Journal of Statistical Physics},
  Year                     = {1998},
  Note                     = {Cited By (since 1996): 15
Export Date: 2 June 2011
Source: Scopus},
  Number                   = {1-2},
  Pages                    = {145-158},
  Volume                   = {90},

  Type                     = {Journal Article},
  Url                      = {http://www.scopus.com/inward/record.url?eid=2-s2.0-0031623372&partnerID=40&md5=6e0d2d4204d5d8a8ec8afcb703d98373}
}

@Article{Akool2003,
  Title                    = {Nitric oxide increases the decay of matrix metalloproteinase 9 mRNA by inhibiting the expression of mRNA-stabilizing factor HuR},
  Author                   = {Akool, E. S. and Kleinert, H. and Hamada, F. M. A. and Abdelwahab, M. H. and Förstermann, U. and Pfeilschifter, J. and Eberhardt, W.},
  Journal                  = {Molecular and Cellular Biology},
  Year                     = {2003},
  Note                     = {Cited By (since 1996): 63
Export Date: 2 June 2011
Source: Scopus},
  Number                   = {14},
  Pages                    = {4901-4916},
  Volume                   = {23},

  Type                     = {Journal Article},
  Url                      = {http://www.scopus.com/inward/record.url?eid=2-s2.0-0038788884&partnerID=40&md5=82cb803b53a1a5d03a512d4a32c39dd2}
}

@Book{Alberts1989,
  Title                    = {Molecular Biology of the Cell},
  Author                   = {Alberts, B. and Bray, D. and Lewis, J. and Raff, M. and Roberts, K. and Watson, J.D. },
  Publisher                = {Garland Publishing, Inc.},
  Year                     = {1989},

  Address                  = {New York \& London},

  Type                     = {Book}
}

@Article{Alessandrini2006,
  Title                    = {Effects of ultrafine carbon particle inhalation on allergic inflammation of the lung},
  Author                   = {Alessandrini, Francesca and Schulz, Holger and Takenaka, Shinji and Lentner, Bernd and Karg, Erwin and Behrendt, Heidrun and Jakob, Thilo},
  Journal                  = {Journal of Allergy and Clinical Immunology},
  Year                     = {2006},
  Note                     = {doi: DOI: 10.1016/j.jaci.2005.11.046},
  Number                   = {4},
  Pages                    = {824-830},
  Volume                   = {117},

  ISSN                     = {0091-6749},
  Keywords                 = {Particulate matter
elemental carbon ultrafine particles
allergic inflammation},
  Type                     = {Journal Article},
  Url                      = {http://www.sciencedirect.com/science/article/B6WH4-4JD10YV-2/2/11e735e5616da6b129519998ec9d36b5}
}

@Article{AlfA¶ldy2009,
  Title                    = {Size-distribution dependent lung deposition of diesel exhaust particles},
  Author                   = {Alföldy, B. and Giechaskiel, B. and Hofmann, W. and Drossinos, Y.},
  Journal                  = {Journal of Aerosol Science},
  Year                     = {2009},
  Number                   = {8},
  Pages                    = {652-663},
  Volume                   = {40},

  Abstract                 = {Lung deposition fractions of nine experimental particle number distributions emitted by various light duty diesel vehicles with different after-treatment devices and fuels were calculated with a stochastic lung deposition model. The emitted volatile and non-volatile mass fractions were treated separately as the corresponding biological response in the human respiratory tract differs. The health related effects of the volatile mass fraction, referred to as the chemical effect, were associated with the deposited volatile mass. The deposited volatile mass depends on the total emitted volatile mass concentration, on its distribution (mass median diameter of nucleation mode and surface area median of the accumulation mode), and on the ratio of the volatile mass in the nucleation and accumulation modes. The effect of the non-volatile mass fraction, referred to as the physical effect, was associated with the surface area distribution of the accumulation mode. The deposited surface area of the non-volatile fraction depends on its emitted concentration and on the surface area median diameter of the size distribution. The calculations suggest the importance of selecting the appropriate distribution (surface, volatile or non-volatile mass) for an assessment of the health effects of diesel exhaust particles, and the importance of combined particulate mass and number distribution measurements.},
  ISSN                     = {0021-8502},
  Keywords                 = {Ultrafine particles
Nucleation mode
Accumulation mode
Active surface area
Volatile mass},
  Type                     = {Journal Article},
  Url                      = {http://www.sciencedirect.com/science/article/B6V6B-4W8VW5G-1/2/79e5de79610436008e4bafeef918c720}
}

@Article{Aliverti2006,
  Title                    = {Lung and chest wall dynamics during rest and exercise without and with expiratory flow limitation},
  Author                   = {Aliverti, A.},
  Journal                  = {Journal of Biomechanics},
  Year                     = {2006},
  Number                   = {Supplement 1},
  Pages                    = {S270-S270},
  Volume                   = {39},

  ISSN                     = {0021-9290},
  Type                     = {Journal Article},
  Url                      = {http://www.sciencedirect.com/science/article/B6T82-4KR88PB-1FV/2/fa221948356db7229114189ef478a16c}
}

@Article{Alonso2001,
  Title                    = {Dispersion of aerosol particles undergoing Brownian motion},
  Author                   = {Alonso, M. and Endo, M.},
  Journal                  = {Journal of Physics A: Mathematical and General},
  Year                     = {2001},
  Number                   = {49},
  Pages                    = {10745–10755},
  Volume                   = {34},

  Type                     = {Journal Article},
  Url                      = {http://dx.doi.org/10.1533/tepr.2006.0002}
}

@Article{Andersen2002,
  Title                    = {Physiologically Based Pharmacokinetic (PBPK) Models for Nasal Tissue Dosimetry of Organic Esters: Assessing the State-of-Knowledge and Risk Assessment Applications with Methyl Methacrylate and Vinyl Acetate},
  Author                   = {Andersen, M.E. and Green, T. and Clay, F.B. and Bogdanffy, M.S.},
  Journal                  = {Regulatory Toxicology and Pharmacology},
  Year                     = {2002},
  Pages                    = {234–245},
  Volume                   = {36},

  Type                     = {Journal Article}
}

@Article{Andersen2000,
  Title                    = {Application of a Hybrid CFD-PBPK Nasal Dosimetry Model in an Inhalation Risk Assessment: An Example with Acrylic Acid},
  Author                   = {Andersen, Melvin and Sarangapani, Ramesh and Gentry, Robinan and Clewell, Harvey and Covington, Tammie and Frederick, Clay B.},
  Journal                  = {Toxicological Sciences},
  Year                     = {2000},
  Number                   = {2},
  Pages                    = {312-325},
  Volume                   = {57},

  Abstract                 = {The available inhalation toxicity information for acrylic acid (AA) suggests that lesions to the nasal cavity, specifically olfactory degeneration, are the most sensitive end point for developing a reference concentration (RfC). Advances in physiologically based pharmacokinetic (PBPK) modeling, specifically the incorporation of computational fluid dynamic (CFD) models, now make it possible to estimate the flux of inhaled chemicals within the nasal cavity of experimental species, specifically rats. The focus of this investigation was to apply an existing CFD-PBPK hybrid model in the estimation of an RfC to determine the impact of incorporation of this new modeling technique into the risk assessment process. Information provided in the literature on the toxicity and mode of action for AA was used to determine the risk assessment approach. A comparison of the approach used for the current U.S. Environmental Protection Agency (U.S. EPA) RfC with the approach using the CFD-PBPK hybrid model was also conducted. The application of the CFD-PBPK hybrid model in a risk assessment for AA resulted in an RfC of 79 ppb, assuming a minute ventilation of 13.8 l/min (20 m3/day) in humans. This value differs substantially from the RfC of 0.37 ppb estimated for AA by the U.S. EPA before the PBPK modeling advances became available. The difference in these two RfCs arises from many factors, with the main difference being the species selected (mouse vs. rat). The choice to conduct the evaluation using the rat was based on the availability of dosimetry data in this species. Once these data are available in the mouse, an assessment should be conducted using this information. Additional differences included the methods used for estimating the target tissue concentration, the uncertainty factors (UFs) applied, and the application of duration and uncertainty adjustments to the internal target tissue dose rather than the external exposure concentration.},
  Doi                      = {10.1093/toxsci/57.2.312},
  Type                     = {Journal Article},
  Url                      = {http://toxsci.oxfordjournals.org/content/57/2/312.abstract}
}

@Article{Andersen2003,
  Title                    = {Toxicokinetic modeling and its applications in chemical risk assessment},
  Author                   = {Andersen, Melvin E.},
  Journal                  = {Toxicology Letters},
  Year                     = {2003},
  Number                   = {1-2},
  Pages                    = {9-27},
  Volume                   = {138},

  Abstract                 = {In recent years physiologically based pharmacokinetic (PBPK) modeling has found frequent application in risk assessments where PBPK models serve as important adjuncts to studies on modes of action of xenobiotics. In this regard, studies on mode of action provide insight into both the sites/mechanisms of action and the form of the xenobiotic associated with toxic responses. Validated PBPK models permit calculation of tissue doses of xenobiotics and metabolites for a variety of conditions, i.e. at low-doses, in different animal species, and in different members of a human population. In this manner, these PBPK models support the low-dose and interspecies extrapolations that are important components of current risk assessment methodologies. PBPK models are sometimes referred to as physiological toxicokinetic (PT) models to emphasize their application with compounds causing toxic responses. Pharmacokinetic (PK) modeling in general has a rich history. Data-based PK compartmental models were developed in the 1930's when only primitive tools were available for solving sets of differential equations. These models were expanded in the 1960's and 1970's to accommodate new observations on dose-dependent elimination and flow-limited metabolism. The application of clearance concepts brought many new insights about the disposition of drugs in the body. In the 1970's PBPK/PT models were developed to evaluate metabolism of volatile compounds of occupational importance, and, for the first time, dose-dependent processes in toxicology were included in PBPK models in order to assess the conditions under which saturation of metabolic and elimination processes lead to non-linear dose response relationships. In the 1980's insights from chemical engineers and occupational toxicology were combined to develop PBPK/PT models to support risk assessment with methylene chloride and other solvents. The 1990's witnessed explosive growth in risk assessment applications of PBPK/PT models and in applying sensitivity and variability methods to evaluate model performance. Some of the compounds examined in detail include butadiene, styrene, glycol ethers, dioxins and organic esters/aids. This paper outlines the history of PBPK/PT modeling, emphasizes more recent applications of PBPK/TK models in health risk assessment, and discusses the risk assessment perspective provided by modern uses of these modeling approaches.},
  ISSN                     = {0378-4274},
  Keywords                 = {Toxicokinetics
Physiologically based modeling
Styrene
Methylene chloride
Methotrexate
Organic esters
Dioxin
Retinoic acid
Vinyl chloride
Risk assessment},
  Type                     = {Journal Article},
  Url                      = {http://www.sciencedirect.com/science/article/B6TCR-47MKPSB-1/2/4fb828a6103ed4613dba02bccf999fc4}
}

@Article{Anderson2014,
  Title                    = {Computational Fluid Dynamics Investigation of Human Aspiration in Low Velocity Air: Orientation Effects on Nose-Breathing Simulations},
  Author                   = {Anderson, Kimberly R. and Anthony, T. Renee},
  Journal                  = {Annals of Occupational Hygiene},
  Year                     = {2014},
  Note                     = {Times Cited: 0
0},
  Number                   = {5},
  Pages                    = {625-645},
  Volume                   = {58},

  Doi                      = {10.1093/annhyg/meu018},
  ISSN                     = {0003-4878; 1475-3162},
  Type                     = {Journal Article},
  Url                      = {<Go to ISI>://WOS:000338107300009
http://annhyg.oxfordjournals.org/content/58/5/625.full.pdf}
}

@Article{Ansys2008,
  Title                    = {Gambit User Manual},
  Author                   = {Ansys},
  Journal                  = {Ansys Inc. USA},
  Year                     = {2008},

  Type                     = {Journal Article}
}

@Article{Ansys2007,
  Title                    = {Fluent User Manual},
  Author                   = {Ansys},
  Journal                  = {Ansys Inc. USA},
  Year                     = {2007},

  Type                     = {Journal Article}
}

@Article{Anthony2006,
  Title                    = {Computational fluid dynamics investigation of particle inhalability},
  Author                   = {Anthony, T.R. and Flynn, M.R.},
  Journal                  = {Aerosol Science},
  Year                     = {2006},
  Number                   = {6},
  Pages                    = {750-765},
  Volume                   = {37},

  Type                     = {Journal Article}
}

@Article{Anthony2005,
  Title                    = {Evaluation of facial features on particle inhalation},
  Author                   = {Anthony, T.R. and Flynn, M.R. and Eisner, A.},
  Journal                  = {Annals of Occupational Hygiene},
  Year                     = {2005},
  Number                   = {2},
  Pages                    = {179-193},
  Volume                   = {49},

  Type                     = {Journal Article}
}

@Article{Anthony2010,
  Title                    = {Contribution of Facial Feature Dimensions and Velocity Parameters on Particle Inhalability},
  Author                   = {Anthony, T. Renée},
  Journal                  = {Annals of Occupational Hygiene},
  Year                     = {2010},
  Number                   = {6},
  Pages                    = {710-725},
  Volume                   = {54},

  Abstract                 = {To examine whether the actual dimensions of human facial features are important to the development of a low-velocity inhalable particulate mass sampling criterion, this study evaluated the effect of facial feature dimensions (nose and lips) on estimates of aspiration efficiency of inhalable particles using computational fluid dynamics modeling over a range of indoor air and breathing velocities. Fluid flow and particle transport around four humanoid forms with different facial feature dimensions were simulated. All forms were facing the wind (0.2, 0.4 m s−1), and breathing was simulated with constant inhalation (1.81, 4.3, 12.11 m s−1). The fluid flow field was solved using standard k-epsilon turbulence equations, and laminar particle trajectories were used to determine critical areas defining inhaled particles. The critical areas were then used to compute the aspiration efficiency of the mouth-breathing humanoid. One-tailed t-tests indicated that models with larger nose and lip features resulted in significantly lower aspiration efficiencies than geometries with smaller features, but the shape of the orifice into the mouth (rounded rectangle versus elliptical) had no effect on aspiration efficiency. While statistically significant, the magnitudes of differences were small: on average, the large nose reduced aspiration efficiency by 6.5% and the large lips reduced aspiration efficiency by 3.2%. In comparison, a change in breathing velocity from at-rest to heavy increased aspiration efficiency by an average of 21% over all particle sizes, indicating a much greater impact of aspiration efficiency on breathing rate in the facing-the-wind orientation. Linear regression models confirmed that particle diameter and breathing velocity were significant predictors to the aspiration fraction, while the facial feature dimensions were not significant contributors to a unifying model. While these effects may be less pronounced as the orientation changes from facing-the-wind, their impact confirms the importance of breathing velocity and, to a lesser extent, facial feature dimensions on exposure estimates in low freestream velocities typical of occupational environments.},
  Doi                      = {10.1093/annhyg/meq040},
  Type                     = {Journal Article},
  Url                      = {http://annhyg.oxfordjournals.org/content/54/6/710.abstract}
}

@Article{Anthony2010a,
  Title                    = {Design and Computational Fluid Dynamics Investigation of a Personal, High Flow Inhalable Sampler},
  Author                   = {Anthony, T. Renée and Landázuri, Andrea C. and Van Dyke, Mike and Volckens, John},
  Journal                  = {Annals of Occupational Hygiene},
  Year                     = {2010},
  Number                   = {4},
  Pages                    = {427-442},
  Volume                   = {54},

  Abstract                 = {The objective of this research was to develop an inlet to meet the inhalable sampling criterion at 10 l min−1 flow using the standard, 37-mm cassette. We designed a porous head for this cassette and evaluated its performance using computational fluid dynamics (CFD) modeling. Particle aspiration efficiency was simulated in a wind tunnel environment at 0.4 m s−1 freestream velocity for a facing-the-wind orientation, with sampler oriented at both 0° (horizontal) and 30° down angles. The porous high-flow sampler oriented 30° downward showed reasonable agreement with published mannequin wind tunnel studies and humanoid CFD investigations for solid particle aspiration into the mouth, whereas the horizontal orientation resulted in oversampling. Liquid particles were under-aspirated in all cases, however, with 41–84% lower aspiration efficiencies relative to solid particles. A sampler with a single central 15-mm pore at 10 l min−1 was also investigated and was found to match the porous sampler’s aspiration efficiency for solid particles; the single-pore sampler is expected to be more suitable for liquid particle use.},
  Doi                      = {10.1093/annhyg/meq029},
  Type                     = {Journal Article},
  Url                      = {http://annhyg.oxfordjournals.org/content/54/4/427.abstract}
}

@Article{Antunes2010,
  Title                    = {Modelling PCC flocculation by bridging mechanism using population balances: Effect of polymer characteristics on flocculation},
  Author                   = {Antunes, E. and Garcia, F. A. P. and Ferreira, P. and Blanco, A. and Negro, C. and Rasteiro, M. G.},
  Journal                  = {Chemical Engineering Science},
  Year                     = {2010},
  Number                   = {12},
  Pages                    = {3798-3807},
  Volume                   = {65},

  Abstract                 = {A population balance model for flocculation of PCC particles with polyelectrolytes of very high molecular weight, medium charge density and different degree of branching is presented. The model considers simultaneously aggregation, breakage and flocs restructuring to describe the PCC flocculation by bridging mechanism. The maximum collision efficiency factor, a parameter related with the fragmentation rate and a time constant for flocs restructuring have been taken as fitting parameters. These fitting parameters are optimized to get the best fit between experimental data obtained by LDS in a previous study and the modelled results. The optimized parameters were correlated with flocculant concentration, flocs structure and polymer branching. The correlations obtained show well the effects of flocculant concentration, flocs structure and polymer structure on the flocculation kinetics and flocs restructuring which are translated in the model parameters. Moreover, the flocs break up due to polymer degradation was introduced in the model by decreasing, with time, the maximum collision efficiency factor. It was shown that this effect can be neglected since the improvement in the results is too small relatively to the high increase of the computational time required to perform the simulation.},
  Doi                      = {10.1016/j.ces.2010.03.020},
  ISSN                     = {0009-2509},
  Keywords                 = {Population balance
Flocculation dynamics
Branched polyelectrolytes
Flocs characteristics
Flocs restructuring
Papermaking},
  Type                     = {Journal Article},
  Url                      = {http://www.sciencedirect.com/science/article/pii/S0009250910001685}
}

@Article{AnwarulHasan2010,
  Title                    = {Effect of artificial mucus properties on the characteristics of airborne bioaerosol droplets generated during simulated coughing},
  Author                   = {Anwarul Hasan, M. D. and Lange, Carlos F. and King, Malcolm L.},
  Journal                  = {Journal of Non-Newtonian Fluid Mechanics},
  Year                     = {2010},
  Number                   = {21-22},
  Pages                    = {1431-1441},
  Volume                   = {165},

  Abstract                 = {The effect of viscoelastic properties and surface tension of artificial mucus simulant samples on the size distribution and volume concentration of bioaerosol droplets generated during simulated coughing was investigated through in vitro experiments. The mucus simulant samples had viscoelastic properties in a similar range as those of real human airway mucus. The mucus simulant gels were prepared by mixing various proportions of 0.5-1.7% locust bean gum solution and 0.1 M sodium tetraborate (XLB) solution. Surface tension of one set of samples was varied by adding different amounts of SDS (sodium dodecyl sulfate) surfactant while the measurement of surface tension was performed using ADSA (axisymmetric drop shape analysis) method. The viscoelastic properties of the samples were measured using a Bohlin Gemini 200 HR (Malvern, UK) nano-rheometer with peltier plate assembly. An artificial cough machine was used to simulate human cough, generating aerosol droplets in a model trachea attached to the front of the cough machine. The size distribution and volume concentration of the droplets generated through simulated cough were measured using a laser diffraction particle sizer (SprayTec, Malvern, USA). The surface tension was found to have negligible effect on the characteristic of generated droplets within the range of this investigation. The experimental results showed a decrease in particle size as the samples changed from a viscous fluid type to a viscoelastic to an elastic solid type sample. The volume concentration also changed significantly as the viscoelasticity of the samples was varied.},
  Doi                      = {10.1016/j.jnnfm.2010.07.005},
  ISSN                     = {0377-0257},
  Keywords                 = {Bioaerosol droplets
Lung mucus
Viscoelasticity
Surface tension
Cough machine},
  Type                     = {Journal Article},
  Url                      = {http://www.sciencedirect.com/science/article/pii/S0377025710002004}
}

@Article{Apodaca2002,
  Title                    = {Modulation of membrane traffic by mechanical stimuli},
  Author                   = {Apodaca, G. },
  Journal                  = {Am J Physiol Renal Physiol},
  Year                     = {2002},
  Pages                    = {F179-F190},
  Volume                   = {282},

  Type                     = {Journal Article}
}

@Book{Applications2004,
  Title                    = {Introduction to VidPIV user manual},
  Author                   = {Applications, Intelligent Laser},
  Year                     = {2004},

  Address                  = {Juelich, Germany},

  Type                     = {Book}
}

@Article{Arcen2006,
  Title                    = {On the influence of near-wall forces in particle-laden channel flows},
  Author                   = {Arcen, B. and Tanière, A. and Oesterlé, B.},
  Journal                  = {International Journal of Multiphase Flow},
  Year                     = {2006},
  Note                     = {doi: DOI: 10.1016/j.ijmultiphaseflow.2006.06.009},
  Number                   = {12},
  Pages                    = {1326-1339},
  Volume                   = {32},

  ISSN                     = {0301-9322},
  Keywords                 = {DNS
Gas-solid flow
Near-wall effects
Lift
Drag forces},
  Type                     = {Journal Article},
  Url                      = {http://www.sciencedirect.com/science/article/B6V45-4M6S045-2/2/1aadd29093fd31625a008f0f36c45090}
}

@InProceedings{Arcilla,
  Title                    = {Numerical Grid Generation in Computational Fluid Dynamics and Related Fields},
  Author                   = {Arcilla, A.S. and Häuser, J. and Eiseman, P.R. and Thompson, J.F.},

  Type                     = {Conference Proceedings}
}

@Article{Arif2002,
  Title                    = {Prevalence and risk factors of work related asthma by industry among United States workers: data from the third national health and nutrition examination survey (1988-94)},
  Author                   = {Arif, A.A. and Whitehead, L.W. and Delclos, G.L. and Tortolero, S.R. and Lee, E.S. },
  Journal                  = {Occ Environ Med},
  Year                     = {2002},
  Pages                    = {505-511},
  Volume                   = {59},

  Type                     = {Journal Article}
}

@Article{Armbruster1982,
  Title                    = {Investigations into defining inhalable dust},
  Author                   = {Armbruster, L. and Breuer, H.},
  Journal                  = {Annals of Occupational Hygiene},
  Year                     = {1982},
  Number                   = {1},
  Pages                    = {21-32},
  Volume                   = {26},

  Abstract                 = {A model human head was exposed in a wind tunnel to coal dust with a maximum particle size of about 100 μm. The wind speed varied between 1 and 8 m s−1. A pump giving a sinusoidal flow pattern simulated inhalation through nose or mouth, tidal volume and breathing frequency being adjusted to physiological values. The total airborne dust concentration was determined by isokinetic probes. The sampling efficiency of the model head under the various conditions in the wind tunnel was determined by comparing the size distribution of the total airborne dust with that of the dust collected on the filter in the nose or mouth. On the basis of these data and those of OGDEN and BIRKETT (1977, 1978), a general definition of inhalable dust is possible. An example of such a definition for conditions at a coal face is given.},
  Doi                      = {10.1093/annhyg/26.1.21},
  Type                     = {Journal Article},
  Url                      = {http://annhyg.oxfordjournals.org/content/26/1/21.abstract}
}

@Article{Armenio1999,
  Title                    = {Effect of the subgrid scales on particle motion},
  Author                   = {Armenio, Vincenzo and Piomelli, Ugo and Fiorotto, Virgilio},
  Journal                  = {Physics of Fluids},
  Year                     = {1999},
  Number                   = {10},
  Pages                    = {3030-3042},
  Volume                   = {11},

  Keywords                 = {turbulent diffusion
channel flow
fluctuations
two-phase flow},
  Type                     = {Journal Article},
  Url                      = {http://link.aip.org/link/?PHF/11/3030/1}
}

@PhdThesis{Arnason1982,
  Title                    = {Measurements of Particle Dispersion in Turbulent Pipe Flow. },
  Author                   = {Arnason, G.},
  Year                     = {1982},
  Type                     = { PhD Thesis},

  University               = {Washington State University.}
}

@Article{Arora2012,
  Title                    = {Nanotoxicology and in vitro studies: The need of the hour},
  Author                   = {Arora, Sumit and Rajwade, Jyutika M. and Paknikar, Kishore M.},
  Journal                  = {Toxicology and Applied Pharmacology},
  Year                     = {2012},
  Number                   = {2},
  Pages                    = {151-165},
  Volume                   = {258},

  Doi                      = {http://dx.doi.org/10.1016/j.taap.2011.11.010},
  ISSN                     = {0041-008X},
  Keywords                 = {Nanotoxicology
In vitro cytotoxicity
Bio-distribution of nanoparticles
Genotoxicity of nanoparticles
Molecular determinants of nanotoxicology},
  Type                     = {Journal Article},
  Url                      = {http://www.sciencedirect.com/science/article/pii/S0041008X11004467}
}

@Article{Artoli2006,
  Title                    = {Mesoscopic simulations of systolic flow in the human abdominal aorta},
  Author                   = {Artoli, A. M. and Hoekstra, A. G. and Sloot, P. M. A.},
  Journal                  = {Journal of Biomechanics},
  Year                     = {2006},
  Note                     = {Cited By (since 1996): 28
Export Date: 2 June 2011
Source: Scopus},
  Number                   = {5},
  Pages                    = {873-884},
  Volume                   = {39},

  Type                     = {Journal Article},
  Url                      = {http://www.scopus.com/inward/record.url?eid=2-s2.0-32644449824&partnerID=40&md5=95e641fdad6ab35067537b592219763d}
}

@Article{Asgharian1998,
  Title                    = {Effect of fiber geometry on deposition in small airways of the lung},
  Author                   = {Asgharian, B. and Ahmadi, G.},
  Journal                  = {Aerosol Science and Technology},
  Year                     = {1998},
  Note                     = {Export Date: 2 June 2011
Source: Scopus},
  Number                   = {6},
  Pages                    = {459-474},
  Volume                   = {29},

  Type                     = {Journal Article},
  Url                      = {http://www.scopus.com/inward/record.url?eid=2-s2.0-0032404084&partnerID=40&md5=2a29f71f5fc2e83b14e20542947d21b5}
}

@Article{Asgharian1995,
  Title                    = {The Effect of Fiber Inertia on Its Orientation in a Shear Flow with Application to Lung Dosimetry},
  Author                   = {Asgharian, Bahman and Anjilvel, Satish},
  Journal                  = {Aerosol Science and Technology},
  Year                     = {1995},
  Number                   = {3},
  Pages                    = {282-290},
  Volume                   = {23},

  Doi                      = {10.1080/02786829508965313},
  ISSN                     = {0278-6826},
  Type                     = {Journal Article},
  Url                      = {http://www.tandfonline.com/doi/abs/10.1080/02786829508965313}
}

@Article{Asgharian1994,
  Title                    = {Inertial and gravitational deposition of particles in a square cross section bifurcating airway},
  Author                   = {Asgharian, B. and Anjilvel, S.},
  Journal                  = {Aerosol Science and Technology},
  Year                     = {1994},
  Note                     = {Cited By (since 1996): 34
Export Date: 2 June 2011
Source: Scopus},
  Number                   = {2},
  Pages                    = {177-193},
  Volume                   = {20},

  Type                     = {Journal Article},
  Url                      = {http://www.scopus.com/inward/record.url?eid=2-s2.0-0028370903&partnerID=40&md5=027531ce6d6b15da4d1dc33838426859}
}

@Article{Asgharian2006,
  Title                    = {Prediction of particle deposition in the human lung using realistic models of lung ventilation},
  Author                   = {Asgharian, B. and Price, O. T. and Hofmann, W.},
  Journal                  = {Journal of Aerosol Science},
  Year                     = {2006},
  Note                     = {Cited By (since 1996): 11
Export Date: 2 June 2011
Source: Scopus},
  Number                   = {10},
  Pages                    = {1209-1221},
  Volume                   = {37},

  Type                     = {Journal Article},
  Url                      = {http://www.scopus.com/inward/record.url?eid=2-s2.0-33748970673&partnerID=40&md5=894af0fa29f0444dd9ebb84d0832b5af}
}

@Article{Asgharian1988,
  Title                    = {Deposition of Inhaled Fibrous Particles in the Human Lung},
  Author                   = {Asgharian, B. and Yu, C. P. },
  Journal                  = {Journal of Aerosol Medicine},
  Year                     = {1988
},
  Pages                    = {77-50},
  Volume                   = {1},

  Type                     = {Journal Article}
}

@Article{Asgharian1989,
  Title                    = {Deposition of fibers in the rat lung},
  Author                   = {Asgharian, B. and Yu, C. P.},
  Journal                  = {Journal of Aerosol Science},
  Year                     = {1989},
  Note                     = {Cited By (since 1996): 28
Export Date: 2 June 2011
Source: Scopus},
  Number                   = {3},
  Pages                    = {355-366},
  Volume                   = {20},

  Type                     = {Journal Article},
  Url                      = {http://www.scopus.com/inward/record.url?eid=2-s2.0-0024898984&partnerID=40&md5=e27bf127b6b5a21a60f364ea8d1904e3}
}

@InBook{ASHRAE2001,
  Title                    = {Chapter26 -Ventilation and Infiltration: Ventilation and Filtration},
  Author                   = {ASHRAE},
  Chapter                  = {26},
  Publisher                = {American Society of Heating, Refrigeration and Air Conditioning Engineers, Inc.},
  Year                     = {2001},

  Address                  = {Atlanta},
  Type                     = {Book Section},

  Booktitle                = {ASHRAE handbook 2001 Fundamentals}
}

@Article{Assanasen2001,
  Title                    = {Supine position decreases the ability of the nose to warm and humidify air},
  Author                   = {Assanasen, P. and Baroody, F.M. and Naureckas, E.N. and Solway, J. and Naclerio, R.M.},
  Journal                  = {Journal of Applied Physiology},
  Year                     = {2001},
  Number                   = {6},
  Pages                    = {2459-2465},
  Volume                   = {91},

  Type                     = {Journal Article}
}

@Article{Auffan2009,
  Title                    = {Towards a definition of inorganic nanoparticles from an environmental, health and safety perspective},
  Author                   = {Auffan, Melanie and Rose, Jerome and Bottero, Jean-Yves and Lowry, Gregory V. and Jolivet, Jean-Pierre and Wiesner, Mark R.},
  Journal                  = {Nat Nano},
  Year                     = {2009},
  Note                     = {10.1038/nnano.2009.242},
  Number                   = {10},
  Pages                    = {634-641},
  Volume                   = {4},

  ISSN                     = {1748-3387},
  Type                     = {Journal Article},
  Url                      = {http://dx.doi.org/10.1038/nnano.2009.242}
}

@Article{Avila2010,
  Title                    = {A Quantitative Method for Estimating Individual Lung Cancer Risk},
  Author                   = {Avila, Ricardo S. and Zulueta, Javier J. and Shara, Nawar M. and Jansen, Kenneth and Veronesi, Giulia and Wang, Hong and Mulshine, James L.},
  Journal                  = {Academic Radiology},
  Year                     = {2010},
  Number                   = {7},
  Pages                    = {830-840},
  Volume                   = {17},

  Abstract                 = {Rationale and Objectives Lung cancer is caused primarily by repeated exposure to carcinogenic particulate matter and noxious gasses with high particulate deposition localized to airway bifurcations and the lung periphery. The quantitative measurement and analysis of these sites has the potential to stratify lung cancer risk. The aim of this preliminary study was to assess the performance of a new method for estimating individual lung cancer risk based on the analysis of airway bifurcations on high-resolution (HR) computed tomographic (CT) scanning and spirometry.Materials and Methods One hundred eight subjects with spirometry and thin-slice CT data were selected from a CT screening study including 15 patients with early lung cancer and 93 age-matched and pack-year-matched controls. A subset of seven patients with cancer and 72 controls were scanned with 1-mm CT slice thickness, representing an HR case subset. A quantitative lung cancer risk index method was developed on the basis of airway bifurcation x-ray attenuation combined with the ratio of forced expiratory volume in 1 second to forced vital capacity. Cochran-Mantel-Haenszel and conditional logistic regression tests were used to analyze performance.Results Cochran-Mantel-Haenszel crude analysis revealed a cancer detection sensitivity and specificity of 67% and 72% for all cases and 100% and 73% for the HR case subset, respectively. Conditional logistic regression showed that a 0.0328 increase in lung cancer risk index was associated with odds ratios of 1.84 (95% confidence interval, 1.18-2.85) for the full data set (P = .0067) and 2.89 (95% confidence interval, 1.02-8.19) for the HR subset (P = .0467).Conclusions A preliminary evaluation of a new lung cancer risk estimation method based on thin slice CT and spirometry showed a statistically significant association with lung cancer.},
  Doi                      = {10.1016/j.acra.2010.03.012},
  ISSN                     = {1076-6332},
  Keywords                 = {Quantitative imaging
lung cancer risk
pulmonary function
computed tomography},
  Type                     = {Journal Article},
  Url                      = {http://www.sciencedirect.com/science/article/pii/S1076633210001674}
}

@Article{Aykac2003,
  Title                    = {Segmentation and analysis of the human airway tree from three-dimensional X-Ray CT Images},
  Author                   = {Aykac, D. and Hoffman, E.A. and McLennan, G. and Reinhardt, J.M.},
  Journal                  = {IEEE Transactions on Medical Imaging},
  Year                     = {2003},
  Pages                    = {940-950},
  Volume                   = {22},

  Type                     = {Journal Article}
}

@Article{BA©ghein2005,
  Title                    = {Using large eddy simulation to study particle motions in a room},
  Author                   = {Béghein, C. and Jiang, Y. and Chen, Q.},
  Journal                  = {Indoor Air},
  Year                     = {2005},
  Number                   = {4},
  Pages                    = {281-290},
  Volume                   = {15},

  Type                     = {Journal Article}
}

@InProceedings{Babu,
  Title                    = {Correlations for prediction of discharge rate, cone angle and aircore diameter of swirl spray atomisers},
  Author                   = {Babu, K.R. and Narasimhan, M.V. and Narayanaswamy, K.},
  Booktitle                = {Proceedings of the 2nd International Conference on Liquid Atomisation and Spray Systems},
  Pages                    = {91-97},

  Type                     = {Conference Proceedings}
}

@Article{Backer2008,
  Title                    = {Computational fluid dynamics can detect changes in airway resistance in asthmatics after acute bronchodilation},
  Author                   = {Backer, J. W. De and Vos, W. G. and Devolder, A. and Verhulst, S. L. and Germonpré, P. and Wuyts, F. L. and Parizel, P. M. and Backer, W. De},
  Journal                  = {Journal of biomechanics},
  Year                     = {2008},
  Number                   = {1},
  Pages                    = {106-113},
  Volume                   = {41},

  ISSN                     = {0021-9290},
  Keywords                 = {Functional imaging
Computational fluid dynamics (CFD)
Asthma
Bronchodilation},
  Type                     = {Journal Article},
  Url                      = {http://linkinghub.elsevier.com/retrieve/pii/S0021929007003181}
}

@Article{Badea2005,
  Title                    = {4-D Micro-CT of the Mouse Heart},
  Author                   = {Badea, Cristian T and Fubara, Boma and Hedlund, Laurence W and Johnson, G. Allan},
  Journal                  = {Molecular Imaging},
  Year                     = {2005},
  Number                   = {2},
  Pages                    = {110-116},
  Volume                   = {4},

  Type                     = {Journal Article}
}

@Misc{Badenhorst2005,
  Title                    = {Cold air distribution - Krantz Products \& Systems Australia},

  Author                   = {Badenhorst, S.},
  Year                     = {2005},

  Publisher                = {Australian Institute of Refrigeration Air Conditioning and Heating},
  Type                     = {Electronic Article}
}

@Article{Bai2010,
  Title                    = {Pulmonary responses to printer toner particles in mice after intratracheal instillation},
  Author                   = {Bai, Ru and Zhang, Lili and Liu, Ying and Meng, Li and Wang, Liming and Wu, Yan and Li, Wei and Ge, Cuicui and Le Guyader, Laurent and Chen, Chunying},
  Journal                  = {Toxicology Letters},
  Year                     = {2010},
  Number                   = {3},
  Pages                    = {288-300},
  Volume                   = {199},

  Doi                      = {10.1016/j.toxlet.2010.09.011},
  ISSN                     = {0378-4274},
  Keywords                 = {Printer toner
Pulmonary response
Proinflammatory cytokines
Intratracheal instillation
Long-term accumulation},
  Type                     = {Journal Article},
  Url                      = {http://www.sciencedirect.com/science/article/pii/S0378427410016991}
}

@InBook{Bailey1998,
  Title                    = {Nasal function and evaluation, nasal obstruction},
  Author                   = {Bailey, B.},
  Pages                    = {335-44, 376, 380-90},
  Publisher                = {Lippincott-Raven},
  Year                     = {1998},

  Address                  = {New York, NY},
  Type                     = {Book Section},

  Booktitle                = {Head and Neck Surgery: Otolaryngology. 2nd ed}
}

@Article{Bailie2006,
  Title                    = {An overview of numerical modelling of nasal airflow},
  Author                   = {Bailie, N. and Hanna, B. and Watterson, J. and Gallagher, G.},
  Journal                  = {Rhinology},
  Year                     = {2006},
  Pages                    = {53-57},
  Volume                   = {44},

  Type                     = {Journal Article}
}

@Article{BalA¡shA¡zy1994,
  Title                    = {Simulation of particle trajectories in bifurcating tubes},
  Author                   = {Balásházy, Imre},
  Journal                  = {Journal of Computational Physics},
  Year                     = {1994},
  Pages                    = {80-88},
  Volume                   = {110},

  Type                     = {Journal Article}
}

@Article{BalA¡shA¡zy1993,
  Title                    = {Particle deposition in airway bifurcations–II. Expiratory flow},
  Author                   = {Balásházy, Imre and Hofmann, Werner},
  Journal                  = {Journal of Aerosol Science},
  Year                     = {1993},
  Number                   = {6},
  Pages                    = {773-786},
  Volume                   = {24},

  Doi                      = {10.1016/0021-8502(93)90045-b},
  ISSN                     = {0021-8502},
  Type                     = {Journal Article},
  Url                      = {http://www.sciencedirect.com/science/article/pii/002185029390045B}
}

@Article{BalA¡shA¡zy2008,
  Title                    = {Three-Dimensional Model for Aerosol Transport and Deposition in Expanding and Contracting Alveoli},
  Author                   = {Balásházy, Imre and Hofmann, Werner and Farkas, �rpád and Madas, Balázs G.},
  Journal                  = {Inhalation Toxicology},
  Year                     = {2008},
  Number                   = {6},
  Pages                    = {611-621},
  Volume                   = {20},

  Doi                      = {doi:10.1080/08958370801915291},
  Type                     = {Journal Article},
  Url                      = {http://informahealthcare.com/doi/abs/10.1080/08958370801915291}
}

@Article{BalA¡shA¡zy1999,
  Title                    = {Computation of local enhancement factors for the quantification of particle deposition patterns in airway bifurcations},
  Author                   = {Balásházy, I. and Hofmann, W. and Heistracher, T.},
  Journal                  = {Journal of Aerosol Science},
  Year                     = {1999},
  Note                     = {doi: DOI: 10.1016/S0021-8502(98)00040-8},
  Number                   = {2},
  Pages                    = {185-203},
  Volume                   = {30},

  ISSN                     = {0021-8502},
  Type                     = {Journal Article},
  Url                      = {http://www.sciencedirect.com/science/article/B6V6B-3VFGVM6-6/2/0e94f55df0c61338652cba5208c71200}
}

@Article{BalA¡shA¡zy1990,
  Title                    = {Inertial impaction and gravitational deposition of aerosols in curved tube and airway bifurcations},
  Author                   = {Balásházy, I. and Hofmann, W. and Martonen, T.B.},
  Journal                  = {Aerosol Science andTechnology},
  Year                     = {1990},
  Number                   = {3},
  Pages                    = {308-321},
  Volume                   = {13},

  Type                     = {Journal Article}
}

@Article{BalA¡shA¡zy1990a,
  Title                    = {Inertial impaction and gravitational deposition of aerosols in curved tube and airway bifurcations},
  Author                   = {Balásházy, I., Hofmann, W. and Martonen, T.B.},
  Journal                  = {Aerosol Science and Technology},
  Year                     = {1990},
  Number                   = {3},
  Pages                    = {308-321},
  Volume                   = {13},

  Type                     = {Journal Article}
}

@Article{BalA¡shA¡zy1993a,
  Title                    = {Particle deposition in airway bifurcations—I. Inspiratory flow},
  Author                   = {Balásházy, Umre and Hofmann, Werner},
  Journal                  = {Journal of Aerosol Science},
  Year                     = {1993},
  Number                   = {6},
  Pages                    = {745-772},
  Volume                   = {24},

  Doi                      = {10.1016/0021-8502(93)90044-a},
  ISSN                     = {0021-8502},
  Type                     = {Journal Article},
  Url                      = {http://www.sciencedirect.com/science/article/pii/002185029390044A}
}

@Article{Baldwin1998,
  Title                    = {A survey of wind speeds in indoor workplaces},
  Author                   = {Baldwin, P.E.J. and Maynard, A.D.},
  Journal                  = {Ann. Occup. Hyg.},
  Year                     = {1998},
  Number                   = {5},
  Pages                    = {303-313},
  Volume                   = {42},

  Type                     = {Journal Article}
}

@Article{Ball2007,
  Title                    = {Mean Flow Structures Inside the Human Upper Airway},
  Author                   = {Ball, C.G. and Uddin, M. and Pollard, D.},
  Journal                  = {Flow Turbulence Combust},
  Year                     = {2007},
  Number                   = {1-2},
  Pages                    = {155-188},
  Volume                   = {81},

  Type                     = {Journal Article}
}

@Article{Ball2008,
  Title                    = {High resolution turbulence modelling of airflow in an idealised human extra-thoracic airway},
  Author                   = {Ball, C. G. and Uddin, M. and Pollard, A.},
  Journal                  = {Computers \& Fluids},
  Year                     = {2008},
  Number                   = {8},
  Pages                    = {943-964},
  Volume                   = {37},

  Abstract                 = {Computational fluid dynamic (CFD) studies of the flow inside a modelled human extra-thoracic airway (ETA) were conducted to evaluate the performance of several turbulence models in predicting flow inside this complex geometry. Veracity of the computational models is assessed for physiologically accurate flow rates of 10, 15, and 30 l/min by comparison of numerical results with hot-wire [Johnstone A, Uddin M, Pollard A, Heenan A, Finlay WH. The flow inside an idealised form of the human extra-thoracic airway. Exp Fluids 2004;37(5):673-89] and particle image velocimetry (PIV) [Heenan AF, Matida E, Pollard A, Finlay WH. Experimental measurements and computational modelling of the flow field in an idealised extra-thoracic airway. Exp Fluids 2003;35:70-84] mean velocity data for the central sagittal plane of the ETA. Furthermore, flow features predicted by numerical models are compared to those from experimental flow-visualisation studies [Johnstone et al., 2004]. The flow in the ETA is shown to be highly three-dimensional, having strong secondary flows.},
  ISSN                     = {0045-7930},
  Type                     = {Journal Article},
  Url                      = {http://www.sciencedirect.com/science/article/B6V26-4R2RMRM-1/2/858da90041e2bc8425006c92b5418f7f}
}

@Article{Ball1992,
  Title                    = {The deposition of micron-sized particles in bends of large diameter pipes},
  Author                   = {Ball, M. H. E. and Mitchell, J. P.},
  Journal                  = {Journal of Aerosol Science},
  Year                     = {1992},
  Note                     = {doi: DOI: 10.1016/0021-8502(92)90339-W},
  Number                   = {Supplement 1},
  Pages                    = {23-26},
  Volume                   = {23},

  ISSN                     = {0021-8502},
  Keywords                 = {Aerosol transport
particle deposition
gas-solid pipe flows},
  Type                     = {Journal Article},
  Url                      = {http://www.sciencedirect.com/science/article/B6V6B-487FB41-X/2/13f3f0eadae014a73b07872b581b5cb0}
}

@Article{Baraniuk2007,
  Title                    = {Nasonasal reflexes, the nasal cycle, and sneeze},
  Author                   = {Baraniuk, J.N. and Kim, D.},
  Journal                  = {Current Allergy and Asthma Reports},
  Year                     = {2007},
  Number                   = {2},
  Pages                    = {105-111},
  Volume                   = {7},

  Type                     = {Journal Article}
}

@Book{Barbee1997,
  Title                    = {Asthma in the elderly},
  Author                   = {Barbee, R.A. and Bloom, J.W.},
  Publisher                = {Marcel Dekker},
  Year                     = {1997},

  Address                  = {New York},
  Series                   = {Lung Biology in Health and Disease},
  Volume                   = {108},

  Type                     = {Edited Book}
}

@Article{Bardina1997,
  Title                    = {Turbulence Modeling Validation Testing and Development},
  Author                   = {Bardina, J.E. and Huang, P.G. and Coakley, T.J.},
  Journal                  = {NASA Technical Memorandum 110446},
  Year                     = {1997},

  Type                     = {Journal Article}
}

@InBook{Bardin-Monnier2008,
  Title                    = {Computational Fluid Dynamics: A tool to the formulation of therapeutic aerosols},
  Author                   = {Bardin-Monnier, Nathalie and Falk, Véronique and Marchal-Heussler, Laurent},
  Editor                   = {Bertrand, Braunschweig and Xavier, Joulia},
  Pages                    = {823-828},
  Publisher                = {Elsevier},
  Year                     = {2008},
  Type                     = {Book Section},
  Volume                   = {Volume 25},

  Booktitle                = {Computer Aided Chemical Engineering},
  ISBN                     = {1570-7946},
  Keywords                 = {Aerosols
Cascade Impactor
Computational Fluid Dynamics
Lagrangian Simulation},
  Url                      = {http://www.sciencedirect.com/science/article/B8G5G-4TK2DGX-53/2/35b3eff871ba658063ab2eab3dd2f64c}
}

@Article{Baron2001,
  Title                    = {Measurement of Airborne Fibers: a Review},
  Author                   = {Baron, P. A. },
  Journal                  = { Industrial Health},
  Year                     = {2001},
  Pages                    = {39-50},
  Volume                   = {39},

  Type                     = {Journal Article}
}

@Article{Baron1994,
  Title                    = {Length separation of fibers},
  Author                   = {Baron, P. A. and Deye, G. J. and Fernback, J.},
  Journal                  = {Aerosol Science and Technology},
  Year                     = {1994},
  Note                     = {Cited By (since 1996): 19
Export Date: 2 June 2011
Source: Scopus},
  Number                   = {2},
  Pages                    = {179-192},
  Volume                   = {21},

  Type                     = {Journal Article},
  Url                      = {http://www.scopus.com/inward/record.url?eid=2-s2.0-0028168248&partnerID=40&md5=2380b95a21c9ee02ffcc3adfb514b8a6}
}

@Article{Baron2002,
  Title                    = {Generation of Size-Selected Fibers for a Nose-Only Inhalation Toxicity Study},
  Author                   = {Baron, P. A. and Sorensen, C. M. and Brockmann, J. B. },
  Journal                  = {The Annals of Ocupational Hygiene},
  Year                     = {2002},
  Number                   = {Supplement 1},
  Pages                    = {186-190},
  Volume                   = {46},

  Type                     = {Journal Article}
}

@Book{Baron2001a,
  Title                    = {Nonspherical Particle Measurements: Shape Factors, Fractals, and Fibers. In Aerosol Measurement},
  Author                   = {Baron, P. A. and Sorensen, C. M. and Brockmann, J. B. },
  Publisher                = {Wiley-Interscience},
  Year                     = {2001},

  Address                  = {New York},

  Type                     = {Book}
}

@Article{Bassichis2002,
  Title                    = {Dry mouth and nose in the older patient: what every PCP should know. (ENT Series)},
  Author                   = {Bassichis, Benjamin A. and Marple, Bradley F.},
  Journal                  = {Geriatrics},
  Year                     = {2002},

  Month                    = {2002/10//},
  Number                   = {10},
  Pages                    = {22},
  Volume                   = {57},

  ISSN                     = {0016867X},
  Keywords                 = {Aging (Biology)
Nose
Xerostomia},
  Type                     = {Magazine Article},
  Url                      = {http://go.galegroup.com.ezproxy.lib.rmit.edu.au/ps/i.do?id=GALE%7CA94144359&v=2.1&u=rmit&it=r&p=PPNU&sw=w&asid=1e7137411858dc2f82c3b8b6d260ccc0}
}

@InBook{Bathe2003,
  Title                    = {Keyword Index},
  Author                   = {Bathe, K. J.},
  Editor                   = {Bathe, K. J.},
  Pages                    = {2429-2443},
  Publisher                = {Elsevier Science Ltd},
  Year                     = {2003},

  Address                  = {Oxford},
  Type                     = {Book Section},

  Booktitle                = {Computational Fluid and Solid Mechanics 2003},
  ISBN                     = {978-0-08-044046-0},
  Url                      = {http://www.sciencedirect.com/science/article/B856Y-4PCJMXX-FK/2/6beef19bc79aafc553988ba5b625a501}
}

@Article{Baughn1991,
  Title                    = {An Experimental Study of Entrainment Effects on the Heat Transfer From a Flat Surface to a Heated Circular Impinging Jet},
  Author                   = {Baughn, J. W. and Hechanova, A. E. and Yan, Xiaojun},
  Journal                  = {Journal of Heat Transfer},
  Year                     = {1991},
  Number                   = {4},
  Pages                    = {1023-1025},
  Volume                   = {113},

  Type                     = {Journal Article},
  Url                      = {http://dx.doi.org/10.1115/1.2911197}
}

@Article{Beare2006,
  Title                    = {A locally constrained watershed transform},
  Author                   = {Beare, R.},
  Journal                  = {IEEE transactions on pattern analysis and machine intelligence},
  Year                     = {2006},
  Number                   = {7},
  Pages                    = {1063-1074},
  Volume                   = {28},

  Type                     = {Journal Article}
}

@Article{Beekhuis1976,
  Title                    = {Nasal obstruction after rhinoplasty: etiology, and techniques for correction},
  Author                   = {Beekhuis, G.J.},
  Journal                  = {Laryngoscope},
  Year                     = {1976},
  Number                   = {4},
  Pages                    = {540-548},
  Volume                   = {86},

  Type                     = {Journal Article}
}

@Article{Beggs2003,
  Title                    = {The Airborne Transmission of Infection in Hospital Buildings: Fact or Fiction?},
  Author                   = {Beggs, C. B.},
  Journal                  = {Indoor and Built Environment},
  Year                     = {2003},
  Number                   = {1-2},
  Pages                    = {9-18},
  Volume                   = {12},

  Abstract                 = {Airborne transmission is known to be the route of infection for diseases such as tuberculosis and aspergillosis. It has also been implicated in nosocomial outbreaks of MRSA, Acinetobacter spp. and Pseudomonas spp. Despite this there is much scepticism about the role that airborne transmission plays in nosocomial outbreaks. This paper investigates the airborne spread of infection in hospital buildings, and evaluates the extent to which it is a problem. It is concluded that although contact-spread is the principle route of transmission for most infections, the contribution of airborne micro-organisms to the spread of infection is likely to be greater than is currently recognised. This is partly because many airborne micro-organisms remain viable while being non-culturable, with the result that they are not detected, and also because some infections arising from contact transmission involve the airborne transportation of micro-organisms onto inanimate surfaces.},
  Doi                      = {10.1177/1420326x03012001002},
  Type                     = {Journal Article},
  Url                      = {http://ibe.sagepub.com/cgi/content/abstract/12/1-2/9}
}

@Article{Behin2005,
  Title                    = {Surgical treatment of patients with refractory migraine headaches and intranasal contact points},
  Author                   = {Behin, F. and Behin, B. and Bigal, M.E. and Lipton, R.B. },
  Journal                  = {Cephalalgia},
  Year                     = {2005},
  Pages                    = {439-443},
  Volume                   = {25},

  Type                     = {Journal Article}
}

@InBook{Beidler1980,
  Title                    = {The chemical senses: gustation and olfaction},
  Author                   = {Beidler, L.M.},
  Editor                   = {Mountcastle, V.B.},
  Pages                    = {594-597},
  Publisher                = {C.V. Mosby Co},
  Year                     = {1980},

  Address                  = {St. Louis},
  Type                     = {Book Section},

  Booktitle                = {Medical Physiology}
}

@Article{Bennett2003,
  Title                    = {Nasal contribution to breathing with exercise: effect of race and gender},
  Author                   = {Bennett, William D. and Zeman, Kirby L. and Jarabek, Annie M.},
  Journal                  = {Journal of Applied Physiology},
  Year                     = {2003},
  Number                   = {2},
  Pages                    = {497-503},
  Volume                   = {95},

  Abstract                 = {Because the nose acts as a filter to prevent penetration of toxic particles and gases to the lower respiratory tract, the route of breathing, oral vs. nasal, may be an important determinant of toxicant dose to the lungs. Using respiratory inductance plethysmography and a nasal mask fitted with flowmeter, we measured the nasal contribution to breathing at rest and during exercise (to 60% maximum workload) in healthy young adults (men/women = 11/11 and Caucasian/African-American = 11/11). We found that the nasal contribution to breathing is less during submaximal exercise in the Caucasians vs. African-Americans (e.g., at 60% maximum workload, mean nasal-to-total ventilation ratio = 0.40 {+/-} 0.21 and 0.65 {+/-} 0.24, respectively, P < 0.05). This difference is likely due to the African-Americans' ability to achieve higher maximal inspiratory flows through their nose than the Caucasians. Men also had a lesser nasal contribution to breathing during exercise compared with women. This is likely due to greater minute ventilations at any given percentage of maximum workload in men vs. women.},
  Doi                      = {10.1152/japplphysiol.00718.2002},
  Type                     = {Journal Article},
  Url                      = {http://jap.physiology.org/cgi/content/abstract/95/2/497}
}

@Article{Berg,
  Title                    = {Flow field analysis in a compliant acinus replica model using particle image velocimetry (PIV)},
  Author                   = {Berg, Emily J. and Weisman, Jessica L. and Oldham, Michael J. and Robinson, Risa J.},
  Journal                  = {Journal of Biomechanics},
  Note                     = {doi: DOI: 10.1016/j.jbiomech.2009.12.019},
  Volume                   = {In Press, Corrected Proof},

  ISSN                     = {0021-9290},
  Keywords                 = {Pulmonary
Acinus
Expanding
Particle image velocimetry
PIV},
  Type                     = {Journal Article},
  Url                      = {http://www.sciencedirect.com/science/article/B6T82-4Y8G1RN-1/2/ff51d117cc1ef4a747881ffecd03ef43}
}

@Book{Berg2008,
  Title                    = {Computational Geometry: Algorithms and Applications},
  Author                   = {de Berg, M. and Ogtfried, C. and van Kreveld, M. and Overmars, M.},
  Publisher                = {Springer-Verlag},
  Year                     = {2008},

  Type                     = {Book}
}

@Article{Berger2000,
  Title                    = {Flows in Stenotic Vessels},
  Author                   = {Berger, S.A. and Jou, L-D. },
  Journal                  = {Annual Review of Fluid Mechanics},
  Year                     = {2000},
  Pages                    = {347-382},
  Volume                   = {32},

  Type                     = {Journal Article}
}

@InBook{Berglund1982,
  Title                    = {Olfaction},
  Author                   = {Berglund, B. and Lindvall, T.},
  Editor                   = {Proctor, D.F. and Andersen, I.},
  Pages                    = {279-306},
  Publisher                = {Elsevier Biomedical Press},
  Year                     = {1982},

  Address                  = {New York},
  Type                     = {Book Section},

  Booktitle                = {The Nose: Upper Airway Physiology and the Atmospheric Environment}
}

@Article{Bergstrom2002,
  Title                    = {The Effects of Surface Roughness on the Mean Velocity Profile in a Turbulent Boundary Layer},
  Author                   = {Bergstrom, Donald J. and Kotey, Nathan A. and Tachie, Mark F.},
  Journal                  = {Journal of Fluids Engineering},
  Year                     = {2002},
  Number                   = {3},
  Pages                    = {664-670},
  Volume                   = {124},

  Keywords                 = {boundary layer turbulence
rough surfaces
velocity measurement
drag},
  Type                     = {Journal Article},
  Url                      = {http://link.aip.org/link/?JFG/124/664/1}
}

@Article{Berlemont,
  Title                    = {Particle lagrangian simulation in turbulent flows},
  Author                   = {Berlemont, A. and Desjonqueres, P. and Gouesbet, G.},
  Journal                  = {International Journal of Multiphase Flow},
  Note                     = {doi: DOI: 10.1016/0301-9322(90)90034-G},
  Number                   = {1},
  Pages                    = {19-34},
  Volume                   = {16},

  ISSN                     = {0301-9322},
  Keywords                 = {turbulence
two-phase flows
Lagrangian simulation
dispersed flows
turbulent diffusion},
  Type                     = {Journal Article},
  Url                      = {http://www.sciencedirect.com/science/article/B6V45-47YSC0Y-94/2/6c0e74919185a0eb9aa466350f361b68}
}

@Article{Berrouk2008,
  Title                    = {Stochastic modelling of aerosol deposition for LES of 90° bend turbulent flow},
  Author                   = {Berrouk, Abdallah S. and Laurence, Dominique},
  Journal                  = {International Journal of Heat and Fluid Flow},
  Year                     = {2008},
  Note                     = {doi: DOI: 10.1016/j.ijheatfluidflow.2008.02.010},
  Number                   = {4},
  Pages                    = {1010-1028},
  Volume                   = {29},

  Abstract                 = {Aerosols deposition in turbulent bend flows is a major concern that is critical to many industrial, environmental and biomedical applications. In this work, a well-resolved LES was performed to compute the deposition efficiency of aerosols in turbulent circular cross-section bend flow of Dean number De=4,225. The numerical predictions were compared to the experimental work of Pui et al. [Pui, D.Y.H., Romay-Novas, F., Liu, B.Y.H., 1987. Experimental study of particle deposition in bend of circular cross-section. Aerosol Science Technol. 7, 301-315] and the fully-resolved LES of Breuer et al. [Breuer, M., Baytekin, H.T., Matida, E.A., 2006. Prediction of aerosol deposition in 90° bends using LES and an efficient Lagrangian tracking method. J. Aerosol Science 37, 1407-1428]. In the present LES, a slightly coarser but unstructured-grid numerical description was adopted, entailing that a portion of the small scales' contribution to particle dispersion to be discarded. Thus, a Langevin-type stochastic model was used to model the effect of the discarded sub-grid motion on aerosol deposition. This stochastic model was shown to perform well in previous studies [Berrouk, A.S., Laurence, D., Riley, J.J., Stock, D.E., 2007. Stochastic modelling of inertial particle dispersion by subgrid motion for LES of high Reynolds number pipe flow. J. Turbulence, 8, 50]. Good care was taken to ensure that the main dynamical features of the continuous phase were captured by the present LES. An estimation of the filtered-out kinetic energy was provided. Results of the present LES with SGS model for particles were found to compare well with the experimental work and the fully-resolved LES (near-wall DNS) of Breuer for all the range of the Stokes number considered, 0.001<St<1.5. Influence of the SGS model for particles was visible for the deposition efficiency of aerosols with Stokes number St<0.3.},
  ISSN                     = {0142-727X},
  Keywords                 = {LES
Aerosols deposition
Bend
Sub-grid
Stochastic},
  Type                     = {Journal Article},
  Url                      = {http://www.sciencedirect.com/science/article/B6V3G-4S7HWM9-1/2/caeba8820557931da5ebde81d4c3d40a}
}

@Article{Berrouk2007,
  Title                    = {Stochastic modelling of inertial particle dispersion by subgrid motion for LES of high Reynolds number pipe flow},
  Author                   = {Berrouk, A. S. and Laurence, D. and Riley, J. J. and Stock, D. E.},
  Journal                  = {Journal of Turbulence},
  Year                     = {2007},
  Pages                    = {N50},
  Volume                   = {8},

  Abstract                 = {Aiming at the better prediction of inertial particle dispersion in LES of turbulent shear flow, a stochastic diffusion process of the Langevin type is adopted to model the time increments of the fluid velocity seen by inertial particles. This modelling is particularly crucial for dispersion and deposition of inertial particles with small relaxation times compared to the smallest LES-resolved turbulence time scales. Simple closures for drift and diffusion terms are described. The effects of inertia and external forces on particle trajectories are modelled. To numerically validate the proposed model, LES calculations are performed to track solid particles (glass beads in air with different Stokes' numbers) in a high Reynolds number, equilibrium turbulent pipe flow (Re = 50 000 based on the maximum velocity and the pipe diameter). LES predictions are compared to RANS results and experimental observations. Simulation findings demonstrate the superiority of LES compared to RANS in predicting particle dispersion statistics. More importantly, the use of a stochastic approach to model the subgrid scale fluctuations has proven crucial for results concerning the small-Stokes-number particles. The influence of the model on particle deposition needs to be assessed and additional validation for non-equilibrium turbulent flows is required.},
  ISSN                     = {1468-5248},
  Type                     = {Journal Article},
  Url                      = {http://www.informaworld.com/10.1080/14685240701615952}
}

@Article{Bertram2008,
  Title                    = {Flow-induced oscillation of collapsed tubes and airway structures},
  Author                   = {Bertram, Christopher D.},
  Journal                  = {Respiratory Physiology \& Neurobiology},
  Year                     = {2008},
  Number                   = {1-3},
  Pages                    = {256-265},
  Volume                   = {163},

  Abstract                 = {The self-excited oscillation of airway structures and flexible tubes in response to flow is reviewed. The structures range from tiny airways deep in the lung causing wheezing at the end of a forced expiration, to the pursed lips of a brass musical instrument player. Other airway structures that vibrate include the vocal cords (and their avian equivalent, the syrinx) and the soft palate of a snorer. These biological cases are compared with experiments on and theories for the self-excited oscillation of flexible tubes conveying a flow on the laboratory bench, with particular reference to those observations dealing with the situation where the inertia of the tube wall is dominant. In each case an attempt is made to summarise the current state of understanding. Finally, some outstanding challenges are identified.},
  ISSN                     = {1569-9048},
  Keywords                 = {Self-excited oscillation
Forced expiration
Vocal folds
Fluid-structure interaction
Brass musical instruments
Lung sounds},
  Type                     = {Journal Article},
  Url                      = {http://www.sciencedirect.com/science/article/B6X16-4SCDB3B-2/2/313d235b3a4ae53c0b6651e095fb306b}
}

@Article{Beule2010,
  Title                    = {Physiology and pathophysiology of respiratory mucosa of the nose and the paranasal sinuses},
  Author                   = {Beule, Achim G.},
  Journal                  = {GMS Current Topics in Otorhinolaryngology, Head and Neck Surgery},
  Year                     = {2010},
  Note                     = {cto000071[PII]
22073111[pmid]
GMS Curr Top Otorhinolaryngol Head Neck Surg},
  Pages                    = {Doc07},
  Volume                   = {9},

  Abstract                 = {In this review, anatomy and physiology of the respiratory mucosa of nose and paranasal sinuses are summarized under the aspect of its clinical significance. Basics of endonasal cleaning including mucociliary clearance and nasal reflexes, as well as defence mechanisms are explained. Physiological wound healing, aspects of endonasal topical medical therapy and typical diagnostic procedures to evaluate the respiratory functions are presented. Finally, the pathophysiologies of different subtypes of non-allergic rhinitis are outlined together with treatment recommendations.},
  Doi                      = {10.3205/cto000071},
  ISSN                     = {1865-1011},
  Type                     = {Journal Article},
  Url                      = {http://www.ncbi.nlm.nih.gov/pmc/articles/PMC3199822/}
}

@Article{Bhasker2010,
  Title                    = {Flow simulation in industrial cyclone separator},
  Author                   = {Bhasker, C.},
  Journal                  = {Advances in Engineering Software},
  Year                     = {2010},
  Number                   = {2},
  Pages                    = {220-228},
  Volume                   = {41},

  ISSN                     = {0965-9978},
  Keywords                 = {Recycle cyclone collector
Circulating fluidized bed combustion
CAD model
Structured multi-block grids
CFD - finite volume technique
Pressure based algorithms
Flow recirculation - geometry modification
Partition plates
Pressure drop
Particle Trajectories},
  Type                     = {Journal Article},
  Url                      = {http://www.sciencedirect.com/science/article/B6V1P-4X9NC8F-1/2/8660a336019d0bdeb385d623559886e3}
}

@Article{Bickers1991,
  Title                    = {Danger: Toxic Aircraft},
  Author                   = {Bickers, C.},
  Journal                  = {Janes Defence Weekly},
  Year                     = {1991},
  Volume                   = {711},

  Type                     = {Journal Article}
}

@Article{Black1996,
  Title                    = {Laser-Based Techniques for Particle-Size Measurement: A Review of Sizing Methods and Their Industrial Applications},
  Author                   = {Black, D.L. and McQuay, M. and Bonin, M.P.},
  Journal                  = {Progress in Energy and Combustion Science},
  Year                     = {1996},
  Pages                    = {267-306},
  Volume                   = {22},

  Type                     = {Journal Article}
}

@Article{Blaisot2005,
  Title                    = {Droplet size and morphology characterization for dense sprays by image processing: application to the Diesel spray},
  Author                   = {Blaisot, J. and Yon, J.},
  Journal                  = {Experiments in Fluids},
  Year                     = {2005},
  Number                   = {6},
  Pages                    = {977-994},
  Volume                   = {39},

  Doi                      = {10.1007/s00348-005-0026-4},
  ISSN                     = {0723-4864},
  Keywords                 = {Physics and Astronomy},
  Type                     = {Journal Article},
  Url                      = {http://dx.doi.org/10.1007/s00348-005-0026-4}
}

@Article{Blake1998,
  Title                    = {Effect of fiber length on glass microfiber cytotoxicity},
  Author                   = {Blake, T. and Castranova, V. and Schwegler-Berry, D. and Baron, P. and Deye, G. J. and Li, C. and Jones, W.},
  Journal                  = {J Toxicol Environ Health A},
  Year                     = {1998},
  Note                     = {Blake, T
Castranova, V
Schwegler-Berry, D
Baron, P
Deye, G J
Li, C
Jones, W
United states
Journal of toxicology and environmental health. Part A
J Toxicol Environ Health A. 1998 Jun 26;54(4):243-59.},
  Number                   = {4},
  Pages                    = {243-59},
  Volume                   = {54},

  Abstract                 = {Fiber length has been implicated as a determinant of fiber toxicity. Fibers of narrowly defined length can be generated by dielectrophoretic classifiers. Since the quantities of fibers produced are very small, we developed a rat alveolar macrophage microculture system to study the toxicity of these samples. The objective of this study was to examine the role of fiber length on the cytotoxicity of Manville code 100 (JM-100) fibers. Rat alveolar macrophages were cultured with 0-500 microg/ml of 5 lengths of JM-100 fibers on 96-well plates. After 18 h, well supernatants were removed and lactate dehydrogenase (LDH) activity was measured to assess cell damage. Chemiluminescence (CL), an assessment of macrophage function, was measured by adding lucigenin with or without zymosan, a particulate stimulus, to appropriate wells. For each fiber length the effects were concentration dependent: CL declined and LDH rose with increasing fiber concentration. Comparing the effects of different lengths showed the greatest toxicity from a relatively long fiber sample (mean length = 17 microm). Microscopic examination of the interaction of fibers with macrophages revealed multiple macrophages attached along the length of the long fibers. This suggests that frustrated, or incomplete, phagocytosis may be a factor in the increased toxicity of longer fibers. Overall the results demonstrate that length is an important determinant of toxicity for JM-100 fibers.},
  ISSN                     = {1528-7394 (Print)
0098-4108 (Linking)},
  Keywords                 = {Acridines
Animals
Cell Death
Cells, Cultured
Glass/chemistry
L-Lactate Dehydrogenase/analysis
Luminescent Measurements
Macrophages, Alveolar/enzymology/metabolism/ pathology
Male
Microscopy, Electron, Scanning
Particle Size
Rats
Rats, Sprague-Dawley
Toxicity Tests
Zymosan},
  Type                     = {Journal Article}
}

@Article{Bogdanffy2003,
  Title                    = {Physiologically-based kinetic modeling of vapours toxic to the respiratory tract},
  Author                   = {Bogdanffy, Matthew S. and Sarangapani, Ramesh},
  Journal                  = {Toxicology Letters},
  Year                     = {2003},
  Number                   = {1-2},
  Pages                    = {103-117},
  Volume                   = {138},

  Abstract                 = {The respiratory tract is frequently identified as a site of toxicity for inhaled xenobiotic chemicals. Usually, these observations come from controlled animal studies. For these studies to be of quantitative value to human health risk assessment, species-specific factors governing dosimetry of inhaled substances must be taken into account. Toxicokinetics of vapours in the respiratory tract are defined by absorption, distribution, metabolism, and excretion, as they are in other tissues; however, these concepts take on new dimensions when considering respiratory tract toxicants, especially those that elicit portal of entry effects by directly interacting with the tissue lining the respiratory tract. Species-specific factors related to anatomy, physiology and biochemistry govern inter-species extrapolation of toxicokinetics. This article discusses critical factors of respiratory tract kinetics that should be considered when developing physiological-based toxicokinetic (PBTK) models for inhaled vapours. Important considerations such as impact of regional airflow-delivery, water solubility, reactivity, and rates of local biotransformation on respiratory tract tissue dosimetry are highlighted. These factors can be accounted for only to a limited extent when using default approaches to extrapolate dosimetry of inhaled substances across species. On the other hand, PBTK modeling has the flexibility to accommodate many of the critical determinants of respiratory tract toxicity. PBTK models can also help identify the most critical toxicokinetic data necessary to replace defaults. PBTK approaches have led to more informed estimates of human target tissue dose, and therefore human health risk, especially where these risk assessments have been based on extrapolation of animal dosimetry studies. Experience derived from the development of more intensive case studies have, in turn, enabled simplified approaches to the use of PBTK modeling for respiratory tract toxicants. Whether simplified or highly complex, PBTK modeling approaches are proven to be of great utility to risk assesors interested in applying quantitative information to informed risk assessment evaluations.},
  ISSN                     = {0378-4274},
  Keywords                 = {Respiratory tract
Physiological-based toxicokinetics
Toxicity
Risk assessment},
  Type                     = {Journal Article},
  Url                      = {http://www.sciencedirect.com/science/article/B6TCR-47S518T-4/2/d3f842460dddc56dab7074f6ed5d4106}
}

@Article{Bois2010,
  Title                    = {PBPK modelling of inter-individual variability in the pharmacokinetics of environmental chemicals},
  Author                   = {Bois, Frédéric Y. and Jamei, Masoud and Clewell, Harvey J.},
  Journal                  = {Toxicology},
  Year                     = {2010},
  Number                   = {3},
  Pages                    = {256-267},
  Volume                   = {278},

  Doi                      = {10.1016/j.tox.2010.06.007},
  ISSN                     = {0300-483X},
  Keywords                 = {Drug-drug interactions
Monte Carlo simulations
PBPK models
Population models
Susceptible populations
Toxicokinetics},
  Type                     = {Journal Article},
  Url                      = {http://www.sciencedirect.com/science/article/pii/S0300483X10002623}
}

@Article{Bokov2010,
  Title                    = {Lumen areas and homothety factor influence airway resistance in COPD},
  Author                   = {Bokov, Plamen and Mauroy, Benjamin and Revel, Marie-Pierre and Brun, Pierre-Amaury and Peiffer, Claudine and Daniel, Christel and Nay, Maria-Magdalena and Mahut, Bruno and Delclaux, Christophe},
  Journal                  = {Respiratory Physiology \& Neurobiology},
  Year                     = {2010},
  Number                   = {1},
  Pages                    = {1-10},
  Volume                   = {173},

  Abstract                 = {The remodelling process of COPD may affect both airway calibre and the homothety factor, which is a constant parameter describing the reduction of airway lumen (hd: diameter of child/parent bronchus) that might be critical because its reduction would induce a frank increase in airway resistance. Airway dimensions were obtained from CT scan images of smokers with (n = 22) and without COPD (n = 9), and airway resistance from plethysmography. Inspiratory airway resistance correlated to lumen area of the sixth bronchial generation of right lung, while peak expiratory flow correlated to the area of the third right generation (p = 0.0009, R = 0.57). A significant relationship was observed between hd and resistance (p = 0.036; R2 = 0.14). A modelling approach of central airways (5 generations) further described the latter relationship. In conclusion, a constant homothety factor can be described by CT scan analysis, which partially explains inspiratory resistance, as predicted by theoretical arguments. Airway resistance is related to lumen areas of less proximal airways than commonly admitted.},
  Doi                      = {10.1016/j.resp.2010.05.011},
  ISSN                     = {1569-9048},
  Keywords                 = {COPD
CT scan
Homothety
Lung model
Resistance
Airway},
  Type                     = {Journal Article},
  Url                      = {http://www.sciencedirect.com/science/article/pii/S1569904810001862}
}

@Article{Bousquet2001,
  Title                    = {Allergic rhinitis and its impact on asthma},
  Author                   = {Bousquet, J. and Van Cauwenberge, P. and Khaltaev, N. and W.H.O.},
  Journal                  = {Journal Allergy and Clin. Immun.},
  Year                     = {2001},
  Number                   = {5},
  Pages                    = {147-334},
  Volume                   = {108},

  Type                     = {Journal Article}
}

@Article{Bragg2005,
  Title                    = {Iced-airfoil aerodynamics},
  Author                   = {Bragg, M. B. and Broeren, A. P. and Blumenthal, L. A.},
  Journal                  = {Progress in Aerospace Sciences},
  Year                     = {2005},
  Number                   = {5},
  Pages                    = {323-362},
  Volume                   = {41},

  Abstract                 = {Past research on airfoil aerodynamics in icing are reviewed. This review emphasizes the time period after the 1978 NASA Lewis workshop that initiated the modern icing research program at NASA and the current period after the 1994 ATR accident where aerodynamics research has been more aircraft safety focused. Research pre-1978 is also briefly reviewed. Following this review, our current knowledge of iced airfoil aerodynamics is presented from a flowfield-physics perspective. This article identifies four classes of ice accretions: roughness, horn ice, streamwise ice, and spanwise-ridge ice. For each class, the key flowfield features such as flowfield separation and reattachment are discussed and how these contribute to the known aerodynamic effects of these ice shapes. Finally Reynolds number and Mach number effects on iced-airfoil aerodynamics are summarized.},
  ISSN                     = {0376-0421},
  Type                     = {Journal Article},
  Url                      = {http://www.sciencedirect.com/science/article/B6V3V-4H3Y9K0-1/2/960a7e412f1968a8552327a2c8a3ac9e}
}

@Article{Branka1999,
  Title                    = {Algorithms for Brownian dynamics computer simulations: Multivariable case},
  Author                   = {Branka, A.C. and Heyes, D.M.},
  Journal                  = {Physical Review E},
  Year                     = {1999},
  Note                     = {Copyright (C) 2010 The American Physical Society
Please report any problems to prola@aps.org
PRE},
  Number                   = {2},
  Pages                    = {2381},
  Volume                   = {60},

  Type                     = {Journal Article},
  Url                      = {http://link.aps.org/abstract/PRE/v60/p2381}
}

@Article{Brannon-Peppas2004,
  Title                    = {Nanoparticle and targeted systems for cancer therapy},
  Author                   = {Brannon-Peppas, Lisa and Blanchette, James O.},
  Journal                  = {Advanced Drug Delivery Reviews},
  Year                     = {2004},
  Number                   = {11},
  Pages                    = {1649-1659},
  Volume                   = {56},

  Doi                      = {10.1016/j.addr.2004.02.014},
  ISSN                     = {0169-409X},
  Keywords                 = {Targeted delivery
Nanoparticles
Cancer therapy
Angiogenesis
Antibodies},
  Type                     = {Journal Article},
  Url                      = {http://www.sciencedirect.com/science/article/pii/S0169409X04001450}
}

@Book{Brant2007,
  Title                    = {Fundamentals of diagnostic radiology},
  Author                   = {Brant, W. E. and Helms, C. A.},
  Publisher                = {Lippincott Williams \& Wilkins},
  Year                     = {2007},

  Type                     = {Book}
}

@Article{Breatnach1984,
  Title                    = {Dimensions of the normal human trachea},
  Author                   = {Breatnach, E. and Abbott, G. C. and Fraser, R. G.},
  Journal                  = {American Journal of Roentgenology},
  Year                     = {1984},
  Pages                    = {903-906},
  Volume                   = {142},

  Type                     = {Journal Article}
}

@Article{Bredberg2004,
  Title                    = {Low-Reynolds Number Turbulence Models: An Approach for Reducing Mesh Sensitivity},
  Author                   = {Bredberg, Jonas and Davidson, Lars},
  Journal                  = {Journal of Fluids Engineering},
  Year                     = {2004},
  Number                   = {1},
  Pages                    = {14-21},
  Volume                   = {126},

  Keywords                 = {turbulence
approximation theory
flow simulation
numerical stability
channel flow
heat transfer},
  Type                     = {Journal Article},
  Url                      = {http://link.aip.org/link/?JFG/126/14/1}
}

@Book{Brennen2005,
  Title                    = {Fundamentals of multiphase flow},
  Author                   = {Brennen, Christopher Earls},
  Publisher                = {Cambridge University Press},
  Year                     = {2005},

  Type                     = {Book}
}

@Article{Brenner1961,
  Title                    = {The slow motion of a sphere through a viscous fluid towards a plane surface},
  Author                   = {Brenner, H.},
  Journal                  = {Chemical Engineering Science},
  Year                     = {1961},
  Note                     = {Cited By (since 1996): 404
Export Date: 2 June 2011
Source: Scopus},
  Number                   = {3-4},
  Pages                    = {242-251},
  Volume                   = {16},

  Type                     = {Journal Article},
  Url                      = {http://www.scopus.com/inward/record.url?eid=2-s2.0-0001454828&partnerID=40&md5=67a87383be60d18a9144f7d64cc9eadb}
}

@Article{Breuer2006,
  Title                    = {Prediction of aerosol deposition in 90o bends using LES and an efficient Lagrangian tracking method},
  Author                   = {Breuer, M. and Baytekin, H. T. and Matida, E. A.},
  Journal                  = {Journal of Aerosol Science},
  Year                     = {2006},
  Note                     = {doi: DOI: 10.1016/j.jaerosci.2006.01.013},
  Number                   = {11},
  Pages                    = {1407-1428},
  Volume                   = {37},

  Abstract                 = {Aiming at the better prediction of pharmaceutical aerosol deposition in extrathoracic airways, a simpler test case, namely a 90o bend flow (tubular cross-section) laden with monodisperse particles, is adopted here and studied numerically. The continuous phase is calculated using a large-eddy simulation technique along with a finite-volume method for block-structured curvilinear grids. The particulate phase is simulated using a Lagrangian approach where hundred thousands of individual monodisperse particles with varying particle diameters are released and tracked throughout the computational domain. To allow such a large number of particles, a highly efficient tracking algorithm is applied, where particle paths are predicted in an orthogonal computational domain, avoiding time-consuming search algorithm, normally required when particles are tracked in the actual physical domain of a curvilinear body-fitted block-structured grid. Both simulation algorithms, for the continuous and particulate phases, are completely parallelized using domain decomposition. Additionally, the in-house code applied supports vector processing allowing efficient usage of nearly all kinds of high-performance architectures. Two different Reynolds numbers ReD are considered where ReD is based on the bend diameter and mean flow velocity. The first case is within the laminar regime at ReD=1000 and serves for the purpose of verification and validation. The second, more challenging case comprises the turbulent regime at ReD=10,000, which is the intrinsic objective of the present study. Depending on the Stokes number of the particles, 0.001[less-than-or-equals, slant]St[less-than-or-equals, slant]1.5, and the releasing locations at the entrance of the bend, the particles will either deposit on the wall or penetrate and exit the computational domain. Simulation results of aerosol deposition efficiency, over the entire range of particle diameters considered here, show an excellent agreement when compared to experimental values obtained by Pui, Romay-Novas, and Liu [(1987). Experimental study of particle deposition in bends of circular cross-section. Aerosol Science and Technology, 7, 301].},
  ISSN                     = {0021-8502},
  Keywords                 = {Monodisperse aerosols
Particle deposition
Large-eddy simulation
Eulerian-Lagrangian approach
Efficient tracking algorithm},
  Type                     = {Journal Article},
  Url                      = {http://www.sciencedirect.com/science/article/B6V6B-4JKYWH8-1/2/ca797d19d783c6bdf80e3f8d4c7e4914}
}

@Article{Brooke1992,
  Title                    = {Turbulent deposition and trapping of aerosols at a wall},
  Author                   = {Brooke, J. W. and Kontomaris, K. and Hanratty, T. J. and McLaughlin, J. B.},
  Journal                  = {Physics of Fluids A},
  Year                     = {1992},
  Note                     = {Cited By (since 1996): 75
Export Date: 2 June 2011
Source: Scopus},
  Number                   = {4},
  Pages                    = {825-834},
  Volume                   = {4},

  Type                     = {Journal Article},
  Url                      = {http://www.scopus.com/inward/record.url?eid=2-s2.0-0000588806&partnerID=40&md5=359ea5150fc4616596ad0836710a6183}
}

@Article{Brooks1973,
  Title                    = {The effect of neutral polymers on the electrokinetic potential of cells and other charged particles. II. A model for the effect of adsorbed polymer on the diffuse double layer},
  Author                   = {Brooks, D. E.},
  Journal                  = {Journal of Colloid And Interface Science},
  Year                     = {1973},
  Note                     = {Cited By (since 1996): 35
Export Date: 2 June 2011
Source: Scopus},
  Number                   = {3},
  Pages                    = {687-699},
  Volume                   = {43},

  Type                     = {Journal Article},
  Url                      = {http://www.scopus.com/inward/record.url?eid=2-s2.0-0001692110&partnerID=40&md5=3e671f92512714c6b9880af9b209affd}
}

@Article{Brooks1973a,
  Title                    = {The effect of neutral polymers on the electrokinetic potential of cells and other charged particles. I. Models for the zeta potential increase},
  Author                   = {Brooks, D. E. and Seaman, G. V. F.},
  Journal                  = {Journal of Colloid And Interface Science},
  Year                     = {1973},
  Note                     = {Cited By (since 1996): 25
Export Date: 2 June 2011
Source: Scopus},
  Number                   = {3},
  Pages                    = {670-686},
  Volume                   = {43},

  Type                     = {Journal Article},
  Url                      = {http://www.scopus.com/inward/record.url?eid=2-s2.0-49549164117&partnerID=40&md5=a6f28b15a90f4c4b0243346384d22965}
}

@Article{Broukal2011,
  Title                    = {Validation of an effervescent spray model with secondary atomization and its application to modeling of a large-scale furnace},
  Author                   = {Broukal, Jakub and Hájek, Jirí},
  Journal                  = {Applied Thermal Engineering},
  Year                     = {2011},
  Number                   = {13},
  Pages                    = {2153-2164},
  Volume                   = {31},

  Abstract                 = {The present work consists of a validation attempt of an effervescent spray model with secondary atomization. The objective is the simulation of a 1 MW industrial-type liquid fuel burner equipped with effervescent spray nozzle. The adopted approach is based on a double experimental validation. Firstly, the evolution of radial drop size distributions of an isothermal spray is investigated. Secondly, the spray model is tested in a swirling combustion simulation by means of measured wall heat flux profile along the flame. In the first part of the paper, both experiments are described along with the measuring techniques. Drop sizes and velocities measured using a Dantec phase/Doppler particle analyzer are analyzed in detail for six radial positions. Local heat fluxes are measured by a reliable technique along the furnace walls in a large-scale water-cooled laboratory furnace. In the second part Euler - Lagrange approach is applied for two-phase flow spray simulations. The adopted spray model is based on the latest industrially relevant (i.e. computationally manageable) primary and secondary breakup sub-models complemented with droplet collision model and a dynamic droplet drag model. Results show discrepancies in the prediction of radial evolution of Sauter mean diameter and exaggerated bimodality in drop size distributions. A partial qualitative agreement is found in radial evolution of drop size distributions. Difficulties in predicting the formation of small drops are highlighted. Comparison of the predicted wall heat fluxes and measured heat loads in swirling flame combustion simulation shows that the absence of the smallest droplets causes a significant elongation of the flame.},
  Doi                      = {10.1016/j.applthermaleng.2011.04.025},
  ISSN                     = {1359-4311},
  Keywords                 = {Drop size distribution
Effervescent atomizationModeling
Spray combustion},
  Type                     = {Journal Article},
  Url                      = {http://www.sciencedirect.com/science/article/pii/S1359431111002237}
}

@Article{Brown2003,
  Title                    = {Rapid Prototyping: The Future of Trauma Surgery?},
  Author                   = {Brown, George A. and Firoozbakhsh, Keikhosrow and DeCoster, Thomas A. and Reyna, Jose R., Jr. and Moneim, Moheb},
  Journal                  = {J Bone Joint Surg Am},
  Year                     = {2003},
  Number                   = {suppl_4},
  Pages                    = {49-55},
  Volume                   = {85},

  Type                     = {Journal Article},
  Url                      = {http://www.ejbjs.org}
}

@Article{Brown2006,
  Title                    = {The structural basis of airways hyperresponsiveness in asthma},
  Author                   = {Brown, R.H. and Pearse, D.B. and Pyrgos, G. and Liu, M.C. and Togias, A. and Permutt, S.},
  Journal                  = {Journal of Applied Physiology},
  Year                     = {2006},
  Pages                    = {30-39},
  Volume                   = {101},

  Type                     = {Journal Article}
}

@Article{BruchmA¼ller2011,
  Title                    = {Modelling discrete fragmentation of brittle particles},
  Author                   = {Bruchmüller, J. and van Wachem, B. G. M. and Gu, S. and Luo, K. H.},
  Journal                  = {Powder Technology},
  Year                     = {2011},
  Number                   = {3},
  Pages                    = {731-739},
  Volume                   = {208},

  Abstract                 = {A novel discrete fragmentation method (DFM) for spherical brittle particles using the discrete element method (DEM) has been developed, implemented and validated. Trajectories of individual fragments can be studied from the moment of breakage where the progeny might originate from incremental, simultaneous and/or repetitive fragmentation events. A particle breaks depending on the applied dynamic impact forces from collisions, the damage history, the particle size and material properties. This 3D model requires setting parameters solely dependent on the particle material and is consequently independent of any empirical value. Mass, momentum and energy is conserved during breakage. A theoretically consistent description from the onset of fragmentation to the cloud formation after breakage is provided and model outcomes have been compared to experimental results and other model predictions where very little deviation has been encountered. All material parameters have been varied independently to study the sensitivity of the model under dynamic fragmentation of numerous particles in a semi-autogenous mill.},
  Doi                      = {10.1016/j.powtec.2011.01.017},
  ISSN                     = {0032-5910},
  Keywords                 = {Fragmentation
Particle breakage
Brittle material
Discrete element method},
  Type                     = {Journal Article},
  Url                      = {http://www.sciencedirect.com/science/article/pii/S0032591011000362}
}

@Article{Brunekreef2005,
  Title                    = {Epidemiological evidence of effects of coarse airborne particles on health},
  Author                   = {Brunekreef, B. and Forsberg, B.},
  Journal                  = {Eur Respir J},
  Year                     = {2005},
  Number                   = {2},
  Pages                    = {309-318},
  Volume                   = {26},

  Abstract                 = {Studies on health effects of airborne particulate matter (PM) have traditionally focused on particles <10 {micro}m in diameter (PM10) or particles <2.5 {micro}m in diameter (PM2.5). The coarse fraction of PM10, particles >2.5 {micro}m, has only been studied recently. These particles have different sources and composition compared with PM2.5. This paper is based on a systematic review of studies that have analysed fine and coarse PM jointly and examines the epidemiological evidence for effects of coarse particles on health. Time series studies relating ambient PM to mortality have in some places provided evidence of an independent effect of coarse PM on daily mortality, but in most urban areas, the evidence is stronger for fine particles. The few long-term studies of effects of coarse PM on survival do not provide any evidence of association. In studies of chronic obstructive pulmonary disease, asthma and respiratory admissions, coarse PM has a stronger or as strong short-term effect as fine PM, suggesting that coarse PM may lead to adverse responses in the lungs triggering processes leading to hospital admissions. There is also support for an association between coarse PM and cardiovascular admissions. It is concluded that special consideration should be given to studying and regulating coarse particles separately from fine particles.},
  Doi                      = {10.1183/09031936.05.00001805},
  Type                     = {Journal Article},
  Url                      = {http://erj.ersjournals.com/cgi/content/abstract/26/2/309}
}

@Article{Buerk2001,
  Title                    = {Can We Model Nitric Oxide Biotransport? A Survey of Mathematical Models for a Simple Diatomic Molecule with Surprisingly Complex Biological Activities.},
  Author                   = {Buerk, D.E. },
  Journal                  = {Annual Review of Biomedical Engineering},
  Year                     = {2001},
  Pages                    = {109-143},
  Volume                   = {3},

  Type                     = {Journal Article}
}

@Article{Bur2009,
  Title                    = {A novel cell compatible impingement system to study in vitro drug absorption from dry powder aerosol formulations},
  Author                   = {Bur, Michael and Rothen-Rutishauser, Barbara and Huwer, Hanno and Lehr, Claus-Michael},
  Journal                  = {European Journal of Pharmaceutics and Biopharmaceutics},
  Year                     = {2009},
  Number                   = {2},
  Pages                    = {350-357},
  Volume                   = {72},

  Abstract                 = {A modified Astra type multistage liquid impinger (MSLI) with integrated bronchial cell monolayers was used to study deposition and subsequent drug absorption on in vitro models of the human airway epithelial barrier. Inverted cell culture of Calu-3 cells on the bottom side of cell culture filter inserts was integrated into a compendial MSLI. Upside down cultivation did not impair the barrier function, morphology and viability of Calu-3 cells. Size selective deposition with subsequent absorption was studied for three different commercially available dry powder formulations of salbutamol sulphate and budesonide. After deposition without size separation the absorption rates from the aerosol formulations differed but correlated with the size of the carrier lactose particles. However, after deposition in the MSLI, simulating relevant impaction and causing the separation of small drug crystals from the carrier lactose, the absorption rates of the three formulations were identical, confirming the bioequivalence of the three formulations.},
  Doi                      = {10.1016/j.ejpb.2008.07.019},
  ISSN                     = {0939-6411},
  Keywords                 = {Calu-3
Air interface deposition
Cell compatible impingement system
Multi stage liquid impinger
Transport experiment
Dry powder aerosol},
  Type                     = {Journal Article},
  Url                      = {http://www.sciencedirect.com/science/article/pii/S0939641108002920}
}

@Article{CNM:CNM2616,
  Title                    = {Effects of the ambient temperature on the airflow across a Caucasian nasal cavity},
  Author                   = {Burgos, M.A. and Sanmiguel-Rojas, E. and Martín-Alcántara, A. and Hidalgo-Martínez, M.},
  Journal                  = {International Journal for Numerical Methods in Biomedical Engineering},
  Year                     = {2014},
  Number                   = {3},
  Pages                    = {430--445},
  Volume                   = {30},

  Doi                      = {10.1002/cnm.2616},
  ISSN                     = {2040-7947},
  Keywords                 = {human nasal cavity, numerical simulations, ambient temperatures, natural heat exchanger},
  Url                      = {http://dx.doi.org/10.1002/cnm.2616}
}

@Article{Burrowes2008,
  Title                    = {Towards a virtual lung: multi-scale, multi-physics modelling of the pulmonary system},
  Author                   = {Burrowes, K.S and Swan, A.J and Warren, N.J and Tawhai, M.H},
  Journal                  = {Philosophical Transactions of the Royal Society A: Mathematical, Physical and Engineering Sciences},
  Year                     = {2008},
  Number                   = {1879},
  Pages                    = {3247-3263},
  Volume                   = {366},

  Abstract                 = {The essential function of the lung, gas exchange, is dependent on adequate matching of ventilation and perfusion, where air and blood are delivered through complex branching systems exposed to regionally varying transpulmonary and transmural pressures. Structure and function in the lung are intimately related, yet computational models in pulmonary physiology usually simplify or neglect structure. The geometries of the airway and vascular systems and their interaction with parenchymal tissue have an important bearing on regional distributions of air and blood, and therefore on whole lung gas exchange, but this has not yet been addressed by modelling studies. Models for gas exchange have typically incorporated considerable detail at the level of chemical reactions, with little thought for the influence of structure. To date, relatively little attention has been paid to modelling at the cellular or subcellular level in the lung, or to linking information from the protein structure/interaction and cellular levels to the operation of the whole lung. We review previous work in developing anatomically based models of the lung, airways, parenchyma and pulmonary vasculature, and some functional studies in which these models have been used. Models for gas exchange at several spatial scales are briefly reviewed, and the challenges and benefits from modelling cellular function in the lung are discussed.},
  Doi                      = {10.1098/rsta.2008.0073},
  Type                     = {Journal Article},
  Url                      = {http://rsta.royalsocietypublishing.org/content/366/1879/3247.abstract}
}

@Article{Burry1993,
  Title                    = {Dispersion of particles in anisotropic turbulent flows},
  Author                   = {Burry, D. and Bergeles, G.},
  Journal                  = {International Journal of Multiphase Flow},
  Year                     = {1993},
  Note                     = {doi: DOI: 10.1016/0301-9322(93)90093-A},
  Number                   = {4},
  Pages                    = {651-664},
  Volume                   = {19},

  ISSN                     = {0301-9322},
  Keywords                 = {turbulence
two-phase flows
Lagrangian simulation
anisotropy
turbulent velocity correlations},
  Type                     = {Journal Article},
  Url                      = {http://www.sciencedirect.com/science/article/B6V45-4806C1D-2B/2/a7ef03ef2165b12e609c6b367a7b6446}
}

@Article{Buschmann2009,
  Title                    = {Near-wall behavior of turbulent wall-bounded flows},
  Author                   = {Buschmann, Matthias H. and Indinger, Thomas and Gad-el-Hak, Mohamed},
  Journal                  = {International Journal of Heat and Fluid Flow},
  Year                     = {2009},
  Note                     = {doi: DOI: 10.1016/j.ijheatfluidflow.2009.06.004},
  Number                   = {5},
  Pages                    = {993-1006},
  Volume                   = {30},

  ISSN                     = {0142-727X},
  Keywords                 = {Turbulent boundary layer
Shear and normal stresses
Wall-behavior
Peak position},
  Type                     = {Journal Article},
  Url                      = {http://www.sciencedirect.com/science/article/B6V3G-4WW16V0-1/2/ae81b7009e9d44e943b9e294038e64e8}
}

@Article{Button2008,
  Title                    = {Role of mechanical stress in regulating airway surface hydration and mucus clearance rates},
  Author                   = {Button, Brian and Boucher, Richard C.},
  Journal                  = {Respiratory Physiology \& Neurobiology},
  Year                     = {2008},
  Note                     = {doi: DOI: 10.1016/j.resp.2008.04.020},
  Number                   = {1-3},
  Pages                    = {189-201},
  Volume                   = {163},

  ISSN                     = {1569-9048},
  Keywords                 = {Mechanical stress
Airway hydration
Mucociliary clearance},
  Type                     = {Journal Article},
  Url                      = {http://www.sciencedirect.com/science/article/B6X16-4SPJ1VH-1/2/a60eec452d11808dc2a6c06028cc83d4}
}

@Article{Buzea2007,
  Title                    = {Nanomaterials and nanoparticles: sources and toxicity},
  Author                   = {Buzea, C. and Pacheco, II and Robbie, K.},
  Journal                  = {Biointerphases},
  Year                     = {2007},
  Number                   = {4},
  Pages                    = {17-71},
  Volume                   = {2},

  Abstract                 = {This review is presented as a common foundation for scientists interested in nanoparticles, their origin,activity, and biological toxicity. It is written with the goal of rationalizing and informing public health concerns related to this sometimes-strange new science of "nano," while raising awareness of nanomaterials' toxicity among scientists and manufacturers handling them.We show that humans have always been exposed to tiny particles via dust storms, volcanic ash, and other natural processes, and that our bodily systems are well adapted to protect us from these potentially harmful intruders. There ticuloendothelial system, in particular, actively neutralizes and eliminates foreign matter in the body,including viruses and nonbiological particles. Particles originating from human activities have existed for millennia, e.g., smoke from combustion and lint from garments, but the recent development of industry and combustion-based engine transportation has profoundly increased an thropogenic particulate pollution. Significantly, technological advancement has also changed the character of particulate pollution, increasing the proportion of nanometer-sized particles--"nanoparticles"--and expanding the variety of chemical compositions. Recent epidemiological studies have shown a strong correlation between particulate air pollution levels, respiratory and cardiovascular diseases, various cancers, and mortality. Adverse effects of nanoparticles on human health depend on individual factors such as genetics and existing disease, as well as exposure, and nanoparticle chemistry, size, shape,agglomeration state, and electromagnetic properties. Animal and human studies show that inhaled nanoparticles are less efficiently removed than larger particles by the macrophage clearance mechanisms in the lungs, causing lung damage, and that nanoparticles can translocate through the circulatory, lymphatic, and nervous systems to many tissues and organs, including the brain. The key to understanding the toxicity of nanoparticles is that their minute size, smaller than cells and cellular organelles, allows them to penetrate these basic biological structures, disrupting their normal function.Examples of toxic effects include tissue inflammation, and altered cellular redox balance toward oxidation, causing abnormal function or cell death. The manipulation of matter at the scale of atoms,"nanotechnology," is creating many new materials with characteristics not always easily predicted from current knowledge. Within the nearly limitless diversity of these materials, some happen to be toxic to biological systems, others are relatively benign, while others confer health benefits. Some of these materials have desirable characteristics for industrial applications, as nanostructured materials often exhibit beneficial properties, from UV absorbance in sunscreen to oil-less lubrication of motors.A rational science-based approach is needed to minimize harm caused by these materials, while supporting continued study and appropriate industrial development. As current knowledge of the toxicology of "bulk" materials may not suffice in reliably predicting toxic forms of nanoparticles,ongoing and expanded study of "nanotoxicity" will be necessary. For nanotechnologies with clearly associated health risks, intelligent design of materials and devices is needed to derive the benefits of these new technologies while limiting adverse health impacts. Human exposure to toxic nanoparticles can be reduced through identifying creation-exposure pathways of toxins, a study that may someday soon unravel the mysteries of diseases such as Parkinson's and Alzheimer's. Reduction in fossil fuel combustion would have a large impact on global human exposure to nanoparticles, as would limiting deforestation and desertification.While nanotoxicity is a relatively new concept to science, this review reveals the result of life's long history of evolution in the presence of nanoparticles, and how the human body, in particular, has adapted to defend itself against nanoparticulate intruders.},
  ISSN                     = {1559-4106 (Electronic)
1559-4106 (Linking)},
  Type                     = {Journal Article}
}

@Article{ByanMendelson2012,
  Title                    = {changes in the facial skeleton with aging: implications and clinical applications in facial rejuvination},
  Author                   = {Byan Mendelson, Chin-Ho Wong},
  Journal                  = {aesthetic Plastic Surgery},
  Year                     = {2012},
  Pages                    = {753-760},
  Volume                   = {36},

  Owner                    = {sean},
  Timestamp                = {2015.04.05}
}

@Article{Cai1988,
  Title                    = {Inertial and interceptional deposition of spherical particles and fibres in a bifurcating airway.},
  Author                   = {Cai, F.S. and Yu, C.P.},
  Journal                  = {Journal of Aerosol Science},
  Year                     = {1988},
  Number                   = {6},
  Pages                    = {679-688},
  Volume                   = {19},

  Type                     = {Journal Article}
}

@Article{Cai2000,
  Title                    = {Endothelial Dysfunction in Cardiovascular Diseases: the Role of Oxidant Stress.},
  Author                   = {Cai, H. and Harrison, D.G. },
  Journal                  = {Circulation Research},
  Year                     = {2000},
  Pages                    = {840-844},
  Volume                   = {87},

  Type                     = {Journal Article}
}

@Article{Calabrese1979,
  Title                    = {The dispersion of discrete particles in a turbulent fluid field},
  Author                   = {Calabrese, Richard V. and Middleman, Stanley},
  Journal                  = {AIChE Journal},
  Year                     = {1979},
  Number                   = {6},
  Pages                    = {1025-1035},
  Volume                   = {25},

  Doi                      = {10.1002/aic.690250614},
  ISSN                     = {1547-5905},
  Type                     = {Journal Article},
  Url                      = {http://dx.doi.org/10.1002/aic.690250614}
}

@Article{Calay2002,
  Title                    = {Numerical simulation of respiratory flow patterns within human lung},
  Author                   = {Calay, R.K. and Kurujareon, J. and Holdo, A.E.},
  Journal                  = {Respiratory Physiology and Neurobiology},
  Year                     = {2002},
  Number                   = {201-221},
  Volume                   = {130},

  Type                     = {Journal Article}
}

@Article{Calay2002a,
  Title                    = {Numerical simulation of respiratory flow patterns within human lung},
  Author                   = {Calay, R. K. and Kurujareon, Jutarat and Holdø, Arne Erik},
  Journal                  = {Respiratory Physiology \& Neurobiology},
  Year                     = {2002},
  Number                   = {2},
  Pages                    = {201-221},
  Volume                   = {130},

  Abstract                 = {A computational fluid dynamics (CFD) modelling approach is used to study the unsteady respiratory airflow dynamics within a human lung. The three-dimensional asymmetric bifurcation model of the central airway based on the morphological data given by Horsfield et al. (J. Appl. Physiol. 67 (1971) 207) was used in the present study to simulate the oscillatory respiratory. The single bifurcation was found to be sufficient to give a number of results which both qualitatively and quantitatively agreed well with other published experimental and CFD results. Numerical simulation were made for two breathing conditions: (a) resting or normal breathing condition and (b) maximal exercise condition. The respiratory flow results for the both conditions are found strongly dependent on the convective effect and the viscous effect with some contribution of the unsteadiness effect. The secondary motions were stronger for the normal breathing condition as compared with the maximal exercise condition. The difference between the two cases is the flow separation regions found close to the carinal ridge for maximal exercise condition. For normal breathing condition no separation regions was observed in this region.},
  ISSN                     = {1569-9048},
  Keywords                 = {Airways, airflow, modeling
Flow, Respiratory air
Lung model, asymmetric bifurcation
Mammals, airflow dynamics},
  Type                     = {Journal Article},
  Url                      = {http://www.sciencedirect.com/science/article/B6X16-45RDXDJ-9/2/f4ae74fe4cadd0fcd1103485df1278db}
}

@Article{Calhoun1990,
  Title                    = {Normal nasal airway resistance in noses of different sizes and shapes},
  Author                   = {Calhoun, K.H. and House, W. and Hokanson, J.A. and Quinn, F.B.},
  Journal                  = {Otolaryngol Head Neck Surg},
  Year                     = {1990},
  Pages                    = {605-609},
  Volume                   = {103},

  Type                     = {Journal Article}
}

@Article{Campbell2002,
  Title                    = {Surface Roughness Visualization for Rapid Prototyping Models},
  Author                   = {Campbell, R.I. and Martorelli, M. and Lee, H.S.},
  Journal                  = {Computer-Aided Des.},
  Year                     = {2002},
  Pages                    = {717-725},
  Volume                   = {34},

  Type                     = {Journal Article}
}

@Article{Canny1986,
  Title                    = {A computational approach to edge detection},
  Author                   = {Canny, J.},
  Journal                  = {IEEE Transactions Pattern Analysis and Machine Intelligence},
  Year                     = {1986},
  Pages                    = {679-698},
  Volume                   = {8},

  Type                     = {Journal Article}
}

@Article{Cao2000,
  Title                    = {Gas-Particle Two-Phase Turbulent Flows in Horizontal and Vertical Ducts.},
  Author                   = {Cao, J. and Ahmadi, G. },
  Journal                  = {International Journal of Engineering Science},
  Year                     = {2000},
  Pages                    = {1961-1981},
  Volume                   = {38},

  Type                     = {Journal Article}
}

@Article{Cao1995,
  Title                    = {Gas-particle two-phase turbulent flow in a vertical duct},
  Author                   = {Cao, J. and Ahmadi, G.},
  Journal                  = {International Journal of Multiphase Flow},
  Year                     = {1995},
  Note                     = {Cited By (since 1996): 77
Export Date: 3 June 2011
Source: Scopus},
  Number                   = {6},
  Pages                    = {1203-1228},
  Volume                   = {21},

  Type                     = {Journal Article},
  Url                      = {http://www.scopus.com/inward/record.url?eid=2-s2.0-0029395191&partnerID=40&md5=2688e7e828545c039c59b57de9ad8099}
}

@Article{Cao1996,
  Title                    = {Gravity granular flows of slightly frictional particles down an inclined bumpy chute},
  Author                   = {Cao, J. and Ahmadi, G. and Massoudi, M.},
  Journal                  = {Journal of Fluid Mechanics},
  Year                     = {1996},
  Note                     = {Cited By (since 1996): 16
Export Date: 3 June 2011
Source: Scopus},
  Pages                    = {197-221},
  Volume                   = {316},

  Type                     = {Journal Article},
  Url                      = {http://www.scopus.com/inward/record.url?eid=2-s2.0-0030153047&partnerID=40&md5=3ee218c47539bf64b1ca6eb9c3810f90}
}

@Article{Capra1995,
  Title                    = {Photoaffinity labeling of Cys-Leukotriene binding sites in human lung parenchyma and bronchi},
  Author                   = {Capra, V. and Ragnini, D. and Galbiati, E. and Novarini, S. and Keppler, D. and Nicosia, S.},
  Journal                  = {Pharmacological Research},
  Year                     = {1995},
  Number                   = {Supplement 1},
  Pages                    = {55-55},
  Volume                   = {31},

  ISSN                     = {1043-6618},
  Type                     = {Journal Article},
  Url                      = {http://www.sciencedirect.com/science/article/B6WP9-49N90X9-7S/2/be0eade4deb4f5ccc81bce5e0301f43e}
}

@Article{Carey1981,
  Title                    = {Human nasal protrusion, latitude, and climate},
  Author                   = {Carey, J.W. and Steegmann, A.T.J.},
  Journal                  = {Am J Phys Anthropol},
  Year                     = {1981},
  Pages                    = {313-319},
  Volume                   = {56},

  Type                     = {Journal Article}
}

@Article{Carlier2005,
  Title                    = {An Improved Model for Anisotropic Dispersion of Small Particles in Turbulent Shear Flows},
  Author                   = {Carlier, J.P. and Khalij, M. and Oesterle, B.},
  Journal                  = {Aerosol Science and Technology},
  Year                     = {2005},
  Number                   = {3},
  Pages                    = {196-205},
  Volume                   = {39},

  Type                     = {Journal Article}
}

@Article{Caro1973,
  Title                    = {Common Carotid Artery Transport of 14C-4-Cholesterol between Serum and Wall in the Perfused Dog},
  Author                   = {Caro, C.G. and Nerem, R.M. },
  Journal                  = {Circulation Research},
  Year                     = {1973},
  Pages                    = {187-205},
  Volume                   = {32},

  Type                     = {Journal Article}
}

@Article{Carroll1993,
  Title                    = {The structure of large and small airways in nonfatal and fatal asthma},
  Author                   = {Carroll, N. and Elliot, J. and Morton, A. and James, A.},
  Journal                  = {Am Rev Respir Dis},
  Year                     = {1993},
  Pages                    = {405-410},
  Volume                   = {147},

  Type                     = {Journal Article}
}

@Article{Carson2010,
  Title                    = {High resolution lung airway cast segmentation with proper topology suitable for computational fluid dynamic simulations},
  Author                   = {Carson, James P. and Einstein, Daniel R. and Minard, Kevin R. and Fanucchi, Michelle V. and Wallis, Christopher D. and Corley, Richard A.},
  Journal                  = {Computerized Medical Imaging and Graphics},
  Year                     = {2010},
  Number                   = {7},
  Pages                    = {572-578},
  Volume                   = {34},

  Abstract                 = {Developing detailed lung airway models is an important step towards understanding the respiratory system. While modern imaging and airway casting approaches have dramatically improved the potential detail of such models, challenges have arisen in image processing as the demand for greater detail pushes the image processing approaches to their limits. Airway segmentations with proper topology have neither loops nor invalid voxel-to-voxel connections. Here we describe a new technique for segmenting airways with proper topology and apply the approach to an image volume generated by magnetic resonance imaging of a silicone cast created from an excised monkey lung.},
  Doi                      = {10.1016/j.compmedimag.2010.03.001},
  ISSN                     = {0895-6111},
  Keywords                 = {Image segmentation
Airway segmentation
Computational fluid dynamics
Lung cast
Monkey
Rat
Topology
Branching
MRI},
  Type                     = {Journal Article},
  Url                      = {http://www.sciencedirect.com/science/article/pii/S0895611110000285}
}

@Article{Cash1990,
  Title                    = {A variable order Runge-Kutta method for initial value problems with rapidly varying right-hand sides.},
  Author                   = {Cash, J.R. and Karp, A.H.},
  Journal                  = {ACM Transactions on Mathematical Software},
  Year                     = {1990},
  Pages                    = {201-222},
  Volume                   = {16},

  Type                     = {Journal Article}
}

@Article{Caughey2005,
  Title                    = {Anatomic risk factors for sinus disease: fact or fiction? },
  Author                   = {Caughey, R.J. and Jameson, M.J. and Gross, C.W. and Han, J.K. },
  Journal                  = {Am J Rhinol},
  Year                     = {2005},
  Number                   = {4},
  Pages                    = {334-339},
  Volume                   = {19},

  Type                     = {Journal Article}
}

@InBook{Cauna1982,
  Title                    = {Blood and nerve supply of the nasal lining},
  Author                   = {Cauna, N.},
  Editor                   = {Proctor, D.F. and Anderson, I.},
  Pages                    = {45-71},
  Publisher                = {Elsevier Biomedical Press},
  Year                     = {1982},

  Address                  = {New York},
  Type                     = {Book Section},

  Booktitle                = {The Nose }
}

@TechReport{Census2001,
  Title                    = {2001 Census Counts of Persons for Australia Classification Counts Occupation by Sex Catalogue No.2022.0 },
  Author                   = {ABS Census},
  Institution              = {Australian Bureau of Statistics},
  Year                     = {2001},
  Type                     = {Report}
}

@Article{Chaffey1965,
  Title                    = {Particle Motions in Sheared Suspensions Part 18. Wall Migration (Theoretical)},
  Author                   = {Chaffey, C. and Brenner, H. and Mason, S.G.},
  Journal                  = {Rheologica Acta},
  Year                     = {1965},
  Pages                    = {64-72},
  Volume                   = {4},

  Type                     = {Journal Article}
}

@Article{Chaffey1967,
  Title                    = {A second-order theory for shear deformation of drops},
  Author                   = {Chaffey, C. E. and Brenner, H.},
  Journal                  = {Journal of Colloid And Interface Science},
  Year                     = {1967},
  Note                     = {Cited By (since 1996): 51
Export Date: 3 June 2011
Source: Scopus},
  Number                   = {2},
  Pages                    = {258-269},
  Volume                   = {24},

  Type                     = {Journal Article},
  Url                      = {http://www.scopus.com/inward/record.url?eid=2-s2.0-0001365871&partnerID=40&md5=d70499042ef88b10e44db506a69d5ac2}
}

@Article{Chakraborty2007,
  Title                    = {Multiscale model for pulmonary oxygen uptake and its application to quantify hypoxemia in hepatopulmonary syndrome},
  Author                   = {Chakraborty, Saikat and Balakotaiah, Vemuri and Bidani, Akhil},
  Journal                  = {Journal of Theoretical Biology},
  Year                     = {2007},
  Number                   = {2},
  Pages                    = {190-207},
  Volume                   = {244},

  Doi                      = {10.1016/j.jtbi.2006.07.030},
  ISSN                     = {0022-5193},
  Keywords                 = {Pulmonary oxygen uptake
Multiscale method
Red blood cell
Ventilation-perfusion heterogeneity
Capillary dilatation
Hepatopulmonary syndrome},
  Type                     = {Journal Article},
  Url                      = {http://www.sciencedirect.com/science/article/pii/S0022519306003067}
}

@Article{Chakroun2011,
  Title                    = {Air quality in rooms conditioned by chilled ceiling and mixed displacement ventilation for energy saving},
  Author                   = {Chakroun, W. and Ghali, K. and Ghaddar, N.},
  Journal                  = {Energy and Buildings},
  Year                     = {2011},
  Number                   = {10},
  Pages                    = {2684-2695},
  Volume                   = {43},

  Doi                      = {10.1016/j.enbuild.2011.06.019},
  ISSN                     = {0378-7788},
  Keywords                 = {Contaminant transport in plumes
Indoor air quality with mixed return air
Chilled ceiling displacement ventilation},
  Type                     = {Journal Article},
  Url                      = {http://www.sciencedirect.com/science/article/pii/S0378778811002684}
}

@Article{Chalupa2004,
  Title                    = {Ultrafine particle deposition in subjects with asthma},
  Author                   = {Chalupa, D.C. and Morrow, P.E. and Oberdorster, G. and Utell, M.J. and Frampton, M.W.},
  Journal                  = {Environmental Health Perspectives},
  Year                     = {2004},
  Number                   = {8},
  Pages                    = {879-882},
  Volume                   = {112},

  Type                     = {Journal Article}
}

@Article{Chan1980,
  Title                    = {Experimental measurements and empirical modeling of the regional deposition of inhaled particles in humans},
  Author                   = {Chan, T.L. and Lippmann, M. },
  Journal                  = {Journal American Industrial Hygiene Association},
  Year                     = {1980},
  Pages                    = {399-409},
  Volume                   = {41},

  Type                     = {Journal Article}
}

@Article{Chan1980a,
  Title                    = {Effect of the laryngeal jet on particle deposition in the human trachea and upper bronchial airways},
  Author                   = {Chan, T.L. and Schreck, R.M. and Lippman, M.},
  Journal                  = {Journal of Aerosol Science},
  Year                     = {1980},
  Number                   = {5-6},
  Pages                    = {447-459},
  Volume                   = {11},

  Type                     = {Journal Article}
}

@InBook{Chang1989,
  Title                    = {Flow dynamics in the respiratory tract},
  Author                   = {Chang, H.K.},
  Editor                   = {Chang, H.K. and Paiva, M.},
  Pages                    = {54-138},
  Publisher                = {Dekker},
  Year                     = {1989},

  Address                  = {New York},
  Type                     = {Book Section},

  Booktitle                = {Respiratory Physiology, an Analytical Approach}
}

@Article{Chang1982,
  Title                    = {A model study of flow dynamics in human central airways},
  Author                   = {Chang, H.K. and El Masry, O.A.},
  Journal                  = {Respiration Physiology},
  Year                     = {1982},
  Number                   = {1},
  Pages                    = {75-95},
  Volume                   = {49},

  Type                     = {Journal Article}
}

@Article{Chang1982a,
  Title                    = {A model study of flow dynamics in human central airways. Part I: Axial velocity profiles},
  Author                   = {Chang, H. K. and El Masry, Osama A.},
  Journal                  = {Respiration Physiology},
  Year                     = {1982},
  Note                     = {doi: DOI: 10.1016/0034-5687(82)90104-9},
  Number                   = {1},
  Pages                    = {75-95},
  Volume                   = {49},

  ISSN                     = {0034-5687},
  Keywords                 = {Air flow patterns
Hot-wire anemometry
Airway model
Quasi-steady flow
Bronchial tree
Velocity profiles
Fluid mechanics},
  Type                     = {Journal Article},
  Url                      = {http://www.sciencedirect.com/science/article/B6T3J-47MKKNH-1T/2/7da68e976f3006a650f01919b668d47b}
}

@Article{Chang2007,
  Title                    = {Pneumonic Alveolar Cavity Transport and Deposition during Inhalation},
  Author                   = {Chang, I-S., and Ahmadi, G. },
  Journal                  = {Computers in Biology and Medicine },
  Year                     = {2007},

  Type                     = {Journal Article}
}

@Article{Chen1996,
  Title                    = {Design and use of a virtual impactor and an electrical classifier for generation of test fiber aerosols with narrow size distributions},
  Author                   = {Chen, B. T. and Yeh, H. C. and Johnson, N. F.},
  Journal                  = {Journal of Aerosol Science},
  Year                     = {1996},
  Note                     = {Cited By (since 1996): 6
Export Date: 3 June 2011
Source: Scopus},
  Number                   = {1},
  Pages                    = {83-94},
  Volume                   = {27},

  Type                     = {Journal Article},
  Url                      = {http://www.scopus.com/inward/record.url?eid=2-s2.0-0029874115&partnerID=40&md5=6fb63efb237eba3041e45602c841d776}
}

@Article{Chen1985,
  Title                    = {TURBULENCE CLOSURE MODEL FOR DILUTE GAS-PARTICLE FLOWS},
  Author                   = {Chen, C. P. and Wood, P. E.},
  Journal                  = {Canadian Journal of Chemical Engineering},
  Year                     = {1985},
  Note                     = {Cited By (since 1996): 71
Export Date: 3 June 2011
Source: Scopus},
  Number                   = {3},
  Pages                    = {349-360},
  Volume                   = {63},

  Type                     = {Journal Article},
  Url                      = {http://www.scopus.com/inward/record.url?eid=2-s2.0-0022077669&partnerID=40&md5=160d62c63da73b5a1289786b5918c4da}
}

@Article{Chen2006,
  Title                    = {Active and passive cilia motion: a computational fluid mechanics model},
  Author                   = {Chen, D. and Graf, S. and Norris, D. and Sundaram, S. and Ventikos, Y.},
  Journal                  = {Journal of Biomechanics},
  Year                     = {2006},
  Number                   = {Supplement 1},
  Pages                    = {S265-S265},
  Volume                   = {39},

  ISSN                     = {0021-9290},
  Type                     = {Journal Article},
  Url                      = {http://www.sciencedirect.com/science/article/B6T82-4KR88PB-1F1/2/dbe4daa12c06003cd4f82e5c8912856c}
}

@Article{Chen2006a,
  Title                    = {Modeling particle distribution and deposition in indoor environments with a new drift-flux model},
  Author                   = {Chen, Fangzhi and Yu, Simon C. M. and Lai, Alvin C. K.},
  Journal                  = {Atmospheric Environment},
  Year                     = {2006},
  Number                   = {2},
  Pages                    = {357-367},
  Volume                   = {40},

  Abstract                 = {A new drift-flux model for particle distribution and deposition in indoor environments is developed. Gravitational settling and deposition are examined thoroughly and the model is applied to simulate particle distribution and deposition in a ventilated model room. The turbulent airflow field is modeled with the renormalization group (RNG) k-[epsilon] turbulence model. For the particulate phase, the concentration field is divided into two regions, the core region and the concentration boundary layer. The concentration distribution in the core region is obtained by solving a three-dimensional particle transport equation. Deposition flux towards the wall is determined with a semi-empirical particle deposition model. With the feasibility of the proposed Eulerian model, the influences of the other deposition mechanisms can be captured easily into the model. The model is validated experimentally and a good agreement between numerical and experimental results is found. Inferring from the result, it shows that the well-mixed assumption cannot hold for coarse particles. The model presented in the current work provides an efficient and reliable tool for investigation of spatial and temporal particle concentration in enclosures and the result will be very useful for improving our current understanding of human exposure assessment.},
  ISSN                     = {1352-2310},
  Keywords                 = {Computational fluid dynamics
Drift-flux model
Ventilation
Indoor particles
Mixing},
  Type                     = {Journal Article},
  Url                      = {http://www.sciencedirect.com/science/article/B6VH3-4HMFJJN-1/2/f1c43170f1b86ef485efb832122fdbcd}
}

@Article{Chen2004,
  Title                    = {Nitric oxide inhibits matrix metalloproteinase-2 expression via the induction of activating transcription factor 3 in endothelial cells},
  Author                   = {Chen, H. H. and Wang, D. L.},
  Journal                  = {Molecular Pharmacology},
  Year                     = {2004},
  Note                     = {Cited By (since 1996): 38
Export Date: 3 June 2011
Source: Scopus},
  Number                   = {5},
  Pages                    = {1130-1140},
  Volume                   = {65},

  Type                     = {Journal Article},
  Url                      = {http://www.scopus.com/inward/record.url?eid=2-s2.0-2142761052&partnerID=40&md5=9a658f9d57dabad99382cd1f3049b951}
}

@Article{Chen2007,
  Title                    = {Computer simulations and experimental measurements of air distributions in buildings: past, present, and future},
  Author                   = {Chen, Q.},
  Journal                  = {HVAC\&R Research},
  Year                     = {2007},
  Number                   = {6},
  Pages                    = {849-997},
  Volume                   = {13},

  Type                     = {Journal Article}
}

@Article{Chen1996a,
  Title                    = {Prediction of room air motion by Reynolds-Stress models},
  Author                   = {Chen, Q.},
  Journal                  = {Building and Environment},
  Year                     = {1996},
  Number                   = {3},
  Pages                    = {233-244},
  Volume                   = {31},

  Type                     = {Journal Article}
}

@Article{Chen1995,
  Title                    = {Comparison of different k-e models for indoor airflow computations},
  Author                   = {Chen, Q.},
  Journal                  = {Numerical Heat Transfer, Part B, Fundamentals},
  Year                     = {1995},
  Pages                    = {353-369},
  Volume                   = {28},

  Type                     = {Journal Article}
}

@Article{Chen1995a,
  Title                    = {COMPARISON OF DIFFERENT k-ε MODELS FOR INDOOR AIR FLOW COMPUTATIONS},
  Author                   = {Chen, Q.},
  Journal                  = {Numerical Heat Transfer, Part B: Fundamentals},
  Year                     = {1995},
  Number                   = {3},
  Pages                    = {353-369},
  Volume                   = {28},

  Doi                      = {10.1080/10407799508928838},
  ISSN                     = {1040-7790},
  Type                     = {Journal Article},
  Url                      = {http://dx.doi.org/10.1080/10407799508928838}
}

@Article{Chen1997,
  Title                    = {Deposition of particles in a turbulent pipe flow},
  Author                   = {Chen, Qian and Ahmadi, Goodarz},
  Journal                  = {Journal of Aerosol Science},
  Year                     = {1997},
  Note                     = {doi: DOI: 10.1016/S0021-8502(96)00474-0},
  Number                   = {5},
  Pages                    = {789-796},
  Volume                   = {28},

  ISSN                     = {0021-8502},
  Type                     = {Journal Article},
  Url                      = {http://www.sciencedirect.com/science/article/B6V6B-3SWJHXC-5/2/2bc03bffc80a7eca765d9055f37fc40a}
}

@Article{Chen2009,
  Title                    = {Experimental study on displacement and mixing ventilation systems for a patient ward},
  Author                   = {Chen, Qingyan and Guity, Arash and Gulick, Bob and Gupta, Jitendra K. and Manning, Andy and Marmion, Paul and Xu, Weiran and Yin, Yonggao and Zhang, Xiaosong},
  Journal                  = {HVAC \& R Research},
  Year                     = {2009},
  Pages                    = {1175+},
  Volume                   = {15},

  ISSN                     = {10789669},
  Keywords                 = {Fluid dynamics
Hospitals
Tracers (Chemistry)
Ventilation
Ventilation equipment},
  Type                     = {Journal Article},
  Url                      = {http://go.galegroup.com/ps/i.do?id=GALE%7CA215609912&v=2.1&u=rmit&it=r&p=AONE&sw=w&asid=32d903bcfda828b327735f5b4eff5c46}
}

@Article{Chen2009a,
  Title                    = {Experimental study on displacement and mixing ventilation systems for a patient ward},
  Author                   = {Chen, Qingyan and Guity, Arash and Gulick, Bob and Gupta, Jitendra K. and Manning, Andy and Marmion, Paul and Xu, Weiran and Yin, Yonggao and Zhang, Xiaosong},
  Journal                  = {HVAC \& R Research},
  Year                     = {2009},
  Pages                    = {1175+},
  Volume                   = {15},

  ISSN                     = {10789669},
  Keywords                 = {(Chemistry)
dynamics
equipment
Fluid
Hospitals
Tracers
Ventilation},
  Type                     = {Journal Article}
}

@InBook{Chen2004a,
  Title                    = {The use of CFD tools for indoor environmental design},
  Author                   = {Chen, Q. and Zhai, Z.},
  Editor                   = {Malkawi, A. and Augenbroe, G.},
  Pages                    = {119-140},
  Publisher                = {Spon Press},
  Year                     = {2004},

  Address                  = {New York},
  Type                     = {Book Section},

  Booktitle                = {Advanced Building Simulation}
}

@Article{Chen1998,
  Title                    = {Lattice Boltzmann Method for Fluid Flows},
  Author                   = {Chen, S. and Doolen, G.D. },
  Journal                  = {Annual Review of Fluid Mechanics},
  Year                     = {1998},
  Pages                    = {329-364},
  Volume                   = {30},

  Type                     = {Journal Article}
}

@Article{Chen2010,
  Title                    = {A Computational Fluid Dynamics Model for Drug Delivery in a Nasal Cavity with Inferior Turbinate Hypertrophy},
  Author                   = {Chen,Xiao B. and Lee,Heow P. and Chong,Vincent F. H. and Wang,De Y.},
  Journal                  = {Journal of Aerosol Medicine and Pulmonary Drug Delivery},
  Year                     = {2010},

  Month                    = {10},
  Note                     = {Copyright - (©) Copyright 2010, Mary Ann Liebert, Inc; Last updated - 2014-04-01},
  Number                   = {5},
  Pages                    = {329-38},
  Volume                   = {23},

  Abstract                 = {Background: Intranasal medications are commonly used in treating nasal diseases. However, technical details of the correct usage of these medications for nasal cavity with obstruction are unclear. Methods: A three-dimensional model of nasal cavity was constructed from MRI scans of a healthy human subject. Nasal cavities corresponding to healthy, moderate, and severe nasal obstruction (NO) were simulated by enlarging the inferior turbinate geometrically, which was documented by approximately one-third reduction of the minimum cross-sectional area for the moderate and two-thirds for the severe obstruction. The discrete phase model based on steady-state computational fluid dynamics was used to study the gas-particle flow. The results were presented with drug particle (from 7 × 10sup]-5 to 10sup]-7 m) deposition distribution along the lateral walls inside these three nasal cavities, and comparisons of the particle ratio escaping from the cavity were also presented and discussed. Results: Nasal patency is an essential condition that had the most impact on particle deposition of the factors studied; the particle percentage escaping the nasal cavity decreased to less than a half and one-tenth for the moderately and severely blocked noses. Decreasing of flow rate and particle diameter increased the escaping ratio; however, zero escaping percentage was detected with the absence of air flow and the effect was less noticeable when the particle diameter was very small (<10sup]-6 m). The existence of inspiratory flow and head tilt angle helped to improve the particle escaping ratio for the healthy nose; however, such changes were not significant for the moderately and severely blocked noses. Conclusion: When using an intranasal medication, it is advisable to have a moderate inspiratory air-flow rate and small size particles to improve particle escaping ratio. Various head positions suggested by clinicians do not seem to improve the drug escaping ratio significantly for the nasal cavities with inferior turbinate hypertrophy.},
  ISBN                     = {19412711},
  Keywords                 = {Medical Sciences--Respiratory Diseases; Severity of Illness Index; Magnetic Resonance Imaging; Computer Simulation; Particle Size; Humans; Administration, Intranasal; Computational Biology -- methods; Drug Delivery Systems; Nasal Cavity -- anatomy & histology; Nasal Obstruction -- pathology; Models, Anatomic},
  Language                 = {English},
  Url                      = {http://search.proquest.com/docview/757381944?accountid=13552}
}

@Article{Chen2010a,
  Title                    = {A Computational Fluid Dynamics Model for Drug Delivery in a Nasal Cavity with Inferior Turbinate Hypertrophy},
  Author                   = {Chen,Xiao B. and Lee,Heow P. and Chong,Vincent F. H. and Wang,De Y.},
  Journal                  = {Journal of Aerosol Medicine and Pulmonary Drug Delivery},
  Year                     = {2010},

  Month                    = {10},
  Note                     = {Copyright - (©) Copyright 2010, Mary Ann Liebert, Inc; Last updated - 2014-04-01},
  Number                   = {5},
  Pages                    = {329-38},
  Volume                   = {23},

  Abstract                 = {Background: Intranasal medications are commonly used in treating nasal diseases. However, technical details of the correct usage of these medications for nasal cavity with obstruction are unclear. Methods: A three-dimensional model of nasal cavity was constructed from MRI scans of a healthy human subject. Nasal cavities corresponding to healthy, moderate, and severe nasal obstruction (NO) were simulated by enlarging the inferior turbinate geometrically, which was documented by approximately one-third reduction of the minimum cross-sectional area for the moderate and two-thirds for the severe obstruction. The discrete phase model based on steady-state computational fluid dynamics was used to study the gas-particle flow. The results were presented with drug particle (from 7 × 10sup]-5 to 10sup]-7 m) deposition distribution along the lateral walls inside these three nasal cavities, and comparisons of the particle ratio escaping from the cavity were also presented and discussed. Results: Nasal patency is an essential condition that had the most impact on particle deposition of the factors studied; the particle percentage escaping the nasal cavity decreased to less than a half and one-tenth for the moderately and severely blocked noses. Decreasing of flow rate and particle diameter increased the escaping ratio; however, zero escaping percentage was detected with the absence of air flow and the effect was less noticeable when the particle diameter was very small (<10sup]-6 m). The existence of inspiratory flow and head tilt angle helped to improve the particle escaping ratio for the healthy nose; however, such changes were not significant for the moderately and severely blocked noses. Conclusion: When using an intranasal medication, it is advisable to have a moderate inspiratory air-flow rate and small size particles to improve particle escaping ratio. Various head positions suggested by clinicians do not seem to improve the drug escaping ratio significantly for the nasal cavities with inferior turbinate hypertrophy.},
  ISBN                     = {19412711},
  Keywords                 = {Medical Sciences--Respiratory Diseases; Severity of Illness Index; Magnetic Resonance Imaging; Computer Simulation; Particle Size; Humans; Administration, Intranasal; Computational Biology -- methods; Drug Delivery Systems; Nasal Cavity -- anatomy & histology; Nasal Obstruction -- pathology; Models, Anatomic},
  Language                 = {English},
  Url                      = {http://search.proquest.com/docview/757381944?accountid=13552}
}

@Article{Chen2009b,
  Title                    = {Assessment of septal deviation effects on nasal air flow: A computational fluid dynamics model},
  Author                   = {Chen, Xiao Bing and Lee, Heow Pueh and Chong, Vincent Fook Hin and Wang, De Yun},
  Journal                  = {The Laryngoscope},
  Year                     = {2009},
  Number                   = {9},
  Pages                    = {1730-1736},
  Volume                   = {119},

  ISSN                     = {1531-4995},
  Type                     = {Journal Article},
  Url                      = {http://dx.doi.org/10.1002/lary.20585}
}

@Article{Chen2006b,
  Title                    = {Acute toxicological effects of copper nanoparticles in vivo},
  Author                   = {Chen, Zhen and Meng, Huan and Xing, Gengmei and Chen, Chunying and Zhao, Yuliang and Jia, Guang and Wang, Tiancheng and Yuan, Hui and Ye, Chang and Zhao, Feng and Chai, Zhifang and Zhu, Chuanfeng and Fang, Xiaohong and Ma, Baocheng and Wan, Lijun},
  Journal                  = {Toxicology Letters},
  Year                     = {2006},
  Number                   = {2},
  Pages                    = {109-120},
  Volume                   = {163},

  Doi                      = {10.1016/j.toxlet.2005.10.003},
  ISSN                     = {0378-4274},
  Keywords                 = {Nanotoxicity
LD50
Target organs
Copper nanoparticles
In vivo},
  Type                     = {Journal Article},
  Url                      = {http://www.sciencedirect.com/science/article/pii/S0378427405003176}
}

@Article{Cheng1996,
  Title                    = {In-vivo measurements of nasal airway dimensions and ultrafine aerosol deposition in the human nasal and oral airways},
  Author                   = {Cheng, K.H. and Cheng, Y.S. and Yeh, H.C. and Guilmette, A. and Simpson, S.Q. and Yang, Y.H. and Swift, D.L.},
  Journal                  = {Journal of Aerosol Science},
  Year                     = {1996},
  Number                   = {5},
  Pages                    = {785-801},
  Volume                   = {27},

  Type                     = {Journal Article}
}

@Article{Cheng1997,
  Title                    = {Measurements of Airway Dimensions and Calculation of Mass Transfer Characteristics of the Human Oral Passage},
  Author                   = {Cheng, K.H. and Cheng, Y.S. and Yeh, H.C. and Swift, D. L.},
  Journal                  = {Journal of Biomechanical Engineering},
  Year                     = {1997},
  Number                   = {Nov},
  Pages                    = {476-482},
  Volume                   = {119},

  Type                     = {Journal Article}
}

@Article{Cheng1995,
  Title                    = {Deposition of Ultrafine Aerosols in the Head Airways During Natural Breathing and During Simulated Breath Holding Using Replicate Human Upper Airway Casts},
  Author                   = {Cheng, Kuo-Hsi and Cheng, Yung-Sung and Yeh, Hsu-Chi and Swift, David L.},
  Journal                  = {Aerosol Science and Technology},
  Year                     = {1995},
  Number                   = {3},
  Pages                    = {465-474},
  Volume                   = {23},

  ISSN                     = {0278-6826},
  Type                     = {Journal Article},
  Url                      = {http://www.informaworld.com/10.1080/02786829508965329}
}

@Article{Cheng2009,
  Title                    = {Comparison of formulas for drag coefficient and settling velocity of spherical particles},
  Author                   = {Cheng, Nian-Sheng},
  Journal                  = {Powder Technology},
  Year                     = {2009},
  Note                     = {doi: DOI: 10.1016/j.powtec.2008.07.006},
  Number                   = {3},
  Pages                    = {395-398},
  Volume                   = {189},

  ISSN                     = {0032-5910},
  Keywords                 = {Drag coefficient
Reynolds number
Settling velocity
Sphere},
  Type                     = {Journal Article},
  Url                      = {http://www.sciencedirect.com/science/article/B6TH9-4T4HP6G-2/2/bd254f8b40965e51eeab0706221f2ea9}
}

@Article{Cheng2003,
  Title                    = {Aerosol deposition in the extrathoracic region},
  Author                   = {Cheng, Y.S.},
  Journal                  = {Aerosol Science and Technology},
  Year                     = {2003},
  Number                   = {8},
  Pages                    = {659-671},
  Volume                   = {37},

  Type                     = {Journal Article}
}

@Article{Cheng2001,
  Title                    = {Characterization of nasal spray pumps and deposition pattern in a replica of the human nasal airway},
  Author                   = {Cheng, Y.S. and Holmes, T.D. and Gao, J. and Guilmette, R.A. and Li, S. and Surakitbanharn, Y. and Rowlings, C.},
  Journal                  = {Journal of Aerosol Medicine},
  Year                     = {2001},
  Number                   = {2},
  Pages                    = {267-280},
  Volume                   = {14},

  Type                     = {Journal Article}
}

@Article{Cheng1993,
  Title                    = {Deposition of thoron progeny in human head airways},
  Author                   = {Cheng, Y.S. and Su, Y.F. and Yeh, H.C. and Swift, D.L.},
  Journal                  = {Aerosol Science and Technology},
  Year                     = {1993},
  Number                   = {4},
  Pages                    = {359-357},
  Volume                   = {18},

  Type                     = {Journal Article}
}

@Article{Cheng1988,
  Title                    = {Diffusional deposition of ultrafine aerosols in a human nasal cast.},
  Author                   = {Cheng, Y.S. and Yamada, Y. and Yeh, H.C. and Swift, D.L.},
  Journal                  = {Journal of Aerosol Science},
  Year                     = {1988},
  Number                   = {6},
  Pages                    = {741-751},
  Volume                   = {19},

  Type                     = {Journal Article}
}

@Article{Cheng1998,
  Title                    = {Deposition of ultrafine particles in the nasal and tracheo-bronchial airways},
  Author                   = {Cheng, Y.S. and Yeh, H.C. and Smith, S.M. and Cheng, K.H. and Swift, D.L.},
  Journal                  = {Journal of Aerosol Science},
  Year                     = {1998},
  Number                   = {1},
  Pages                    = {S941-S942},
  Volume                   = {29},

  Type                     = {Journal Article}
}

@Article{Cheng1999,
  Title                    = {Particle deposition in a cast of human oral airways},
  Author                   = {Cheng, Y.S. and Zhou, Y. and Chen, B.T.},
  Journal                  = {Aerosol Science and Technology},
  Year                     = {1999},
  Number                   = {4},
  Pages                    = {286-300},
  Volume                   = {31},

  Type                     = {Journal Article}
}

@Article{Cheng1996a,
  Title                    = {Nasal Deposition of Ultrafine Particles in Human Volunteers and Its Relationship to Airway Geometry},
  Author                   = {Cheng, Y. S. and Yeh, H. C. and Guilmette, R. A. and Simpson, S. Q. and Cheng, K. H. and Swift, D. L.},
  Journal                  = {Aerosol Science and Technology},
  Year                     = {1996},
  Number                   = {3},
  Pages                    = {274-291},
  Volume                   = {25},

  ISSN                     = {0278-6826},
  Type                     = {Journal Article},
  Url                      = {http://www.informaworld.com/10.1080/02786829608965396}
}

@Article{Cheng2008,
  Title                    = {Comparison of the TSI Model 8520 and Grimm Series 1.108 Portable Aerosol Instruments Used to Monitor Particulate Matter in an Iron Foundry},
  Author                   = {Cheng, Yu-Hsiang},
  Journal                  = {Journal of Occupational and Environmental Hygiene},
  Year                     = {2008},
  Number                   = {3},
  Pages                    = {157-168},
  Volume                   = {5},

  Doi                      = {10.1080/15459620701860867},
  ISSN                     = {1545-9624},
  Type                     = {Journal Article},
  Url                      = {http://dx.doi.org/10.1080/15459620701860867}
}

@Article{Cheng1995a,
  Title                    = {Deposition of ultrafine aerosols and thoron progeny in replicas of nasal airways of young children},
  Author                   = {Cheng, Yung-Sung and Smith, Shawna M and Yeh, Hsu-Chi and Kim, Dai-Byung and Cheng, Kuo-Hsi and Swift, David L},
  Journal                  = {Aerosol science and technology},
  Year                     = {1995},
  Note                     = {C:\Users\sean\AppData\Roaming\Zotero\Zotero\Profiles\16a4oype.default\zotero\storage\WGBZ8VWU\Deposition of ultrafine aerosols and thoron progeny in replicas of nasal airways of young children.pdf},
  Pages                    = {541–552},
  Volume                   = {23},

  Type                     = {Journal Article}
}

@Article{Cheng1988a,
  Title                    = {Diffusional deposition of ultrafine aerosols in a human nasal cast},
  Author                   = {Cheng, Yung-Sung and Yamada, Yuji and Yeh, Hsu-Chi and Swift, David L.},
  Journal                  = {Journal of Aerosol Science},
  Year                     = {1988},
  Note                     = {C:\Users\sean\AppData\Roaming\Zotero\Zotero\Profiles\16a4oype.default\zotero\storage\DZNIZM3A\Cheng et al. - 1988 - Diffusional deposition of ultrafine aerosols in a .pdf},
  Pages                    = {741-751},
  Volume                   = {19},

  Abstract                 = {This report describes an experimental study of aerosol deposition in a human nasal cast. A clear polyester resin cast of the upper airways of a normal human adult was used. This life-sized model included nasal hairs, the nasal airways, oral cavity, nasopharynx, larynx and entrance to the trachea. For nasal air flow rates, the measured pressure drop in the cast was similar to the in vivo measurements reported in the literature. Thus, the cast appeared to be a good representation of the nasal airways of living humans. The total deposition was measured in the cast for monodisperse NaCl aerosols between 0.2 and 0.0046 μm at flow rates between 4 and 50 l min−1 of inspiratory flow. The deposition efficiency increased with decreasing particle size and flow rate, indicating that diffusion was the dominant mechanism for deposition. At 20 1 min−1 flow (comparable to that for normal breating at rest), inspiratory deposition efficiencies were 16 and 40% for 0.01 and 0.005 μm particles, respectively. A theoretical equation relating the deposition efficiency to the flow rate and diffusion coefficient was derived based on a turbulent diffusion process. The agreement achieved between the theoretical and measured deposition indicated that the turbulent diffusion was the dominant mechanism of deposition in the nasal cast. This information will be incorporated into future models of respiratory tract dosimetry.},
  Doi                      = {10.1016/0021-8502(88)90009-2},
  ISSN                     = {0021-8502},
  Type                     = {Journal Article},
  Url                      = {http://ac.els-cdn.com/0021850288900092/1-s2.0-0021850288900092-main.pdf?_tid=abdb577a-421f-11e4-8e3e-00000aab0f26&acdnat=1411366632_70979a7bdcfeb25ebf892855655ff277}
}

@Article{Cheng1991,
  Title                    = {Aerosol Deposition in Human Nasal Airway for Particles 1nm to 20 µm: A Model Study},
  Author                   = {Cheng, Y.-S. and Yeh, H.C. and Swift, D.L.},
  Journal                  = {Radiat Prot Dosimetry},
  Year                     = {1991},
  Number                   = {1-3},
  Pages                    = {41-47},
  Volume                   = {38},

  Abstract                 = {Substantial deposition of either very large or very small particles in the nasal airways prevents many inhaled particles from reaching the tracheobronchial and pulmonary regions. Deposition efficiency in the head airways is a function of particle size, breathing pattern and airway geometry. The results of experiments with human volunteers and physical replica models provide the data necessary for calculating deposition fractions in the head airways. All of the earlier studies focused on particles larger than 0.5 µm, inertial deposition is the dominant mechanism; for particles smaller than 0.5 µm, diffusional deposition is dominant. By assuming turbulent flow and complete mixing of aerosol in the head airways, we derived the following equation: di=1exp (-a dar2Q-b D1/2Q-1/8). Constants a and b are estimated from experimental data and are a function of breathing mode (inspiration or expiration). The combination of turbulent inertial and diffusional deposition mechanisms results in increased deposition for particles either greater than 1 µm or smaller than 0.1 µm, with minimal deposition for particles in the size range of 0.1 to 1 µm. These expressions provide estimates for nasal airway deposition of particles ranging from 1 nm to over 20 µm in diameter and are useful for dosimetry calculations.},
  Type                     = {Journal Article},
  Url                      = {http://rpd.oxfordjournals.org/cgi/content/abstract/38/1-3/41}
}

@Article{Cheong2003,
  Title                    = {Measurements and computations of contaminant's distribution in an office environment},
  Author                   = {Cheong, K.W.D. and Djunaedy, E. and Poh, T.K. and Tham, K.W. and Sekhar, S.C.},
  Journal                  = {Building and Environment},
  Year                     = {2003},
  Number                   = {1},
  Pages                    = {135-145},
  Volume                   = {38},

  Type                     = {Journal Article}
}

@Article{Cheong2001,
  Title                    = {The influence of furniture and equipment layouts on airflow pattern in a clean room},
  Author                   = {Cheong, K W and Djunaedy, E},
  Journal                  = {Building Service Engineering},
  Year                     = {2001},
  Number                   = {4},
  Pages                    = {261-266},
  Volume                   = {22},

  Abstract                 = {The layout of the production line in any clean rooms will change according to the production process and this posed a problem for post clean room maintenance. Air velocity is one of the many problematic issues commonly found in clean room environment. It is important to address this on-going problem whenever there is an upgrading of the production process in which more advanced equipment were being introduced to the clean room to automate the process. This paper examines the characteristics of airflow distribution within the clean room due to changes in the production process. This study was conducted for a class 100 clean room at a disk-drive manufacturing firm. The airflow distribution in the clean room was simulated using computational fluid dynamics (CFD) software. This model provides a valuable insight into the processes that are difficult and time-consuming to study experimentally. The results showed that furniture such as a workbench and a laminar flow hood have great impact on the airflow pattern in the clean room. The predicted airflow results offered substantial benefits in the design and development of a better furniture layout plan that eradicates any of the possible turbulence problems.},
  Doi                      = {10.1177/014362440102200405},
  Type                     = {Journal Article},
  Url                      = {http://bse.sagepub.com/cgi/content/abstract/22/4/261}
}

@Article{Cheong2003a,
  Title                    = {Measurements and computations of contaminant's distribution in an office environment},
  Author                   = {Cheong, K. W. D. and Djunaedy, E. and Poh, T. K. and Tham, K. W. and Sekhar, S. C. and Wong, N. H. and Ullah, M. B.},
  Journal                  = {Building and Environment},
  Year                     = {2003},
  Note                     = {doi: DOI: 10.1016/S0360-1323(02)00031-8},
  Number                   = {1},
  Pages                    = {135-145},
  Volume                   = {38},

  ISSN                     = {0360-1323},
  Keywords                 = {Contaminant
CFD
Measurement
Computation
Office
Tracer-gas},
  Type                     = {Journal Article},
  Url                      = {http://www.sciencedirect.com/science/article/B6V23-458P7DH-5/2/34c70040b3007871af174e7675be5102}
}

@Article{Cherukat,
  Title                    = {Wall-induced lift on a sphere},
  Author                   = {Cherukat, P. and McLaughlin, J. B.},
  Journal                  = {International Journal of Multiphase Flow},
  Note                     = {doi: DOI: 10.1016/0301-9322(90)90011-7},
  Number                   = {5},
  Pages                    = {899-907},
  Volume                   = {16},

  ISSN                     = {0301-9322},
  Keywords                 = {inertia
lateral
lift
migration
sedimentation
wall},
  Type                     = {Journal Article},
  Url                      = {http://www.sciencedirect.com/science/article/B6V45-47YJFKP-CD/2/d3823f2469ea0eb833cc4888c4b0ee29}
}

@Article{Cherukat1990,
  Title                    = {Wall-induced lift on a sphere},
  Author                   = {Cherukat, P. and McLaughlin, J. B.},
  Journal                  = {International Journal of Multiphase Flow},
  Year                     = {1990},
  Note                     = {doi: DOI: 10.1016/0301-9322(90)90011-7},
  Number                   = {5},
  Pages                    = {899-907},
  Volume                   = {16},

  ISSN                     = {0301-9322},
  Keywords                 = {inertia
lateral
lift
migration
sedimentation
wall},
  Type                     = {Journal Article},
  Url                      = {http://www.sciencedirect.com/science/article/B6V45-47YJFKP-CD/2/d3823f2469ea0eb833cc4888c4b0ee29}
}

@Article{Cherukat1994,
  Title                    = {The inertial lift on a rigid sphere translating in a linear shear flow field},
  Author                   = {Cherukat, P. and McLaughlin, J. B. and Graham, A. L.},
  Journal                  = {International Journal of Multiphase Flow},
  Year                     = {1994},
  Note                     = {Cited By (since 1996): 22
Export Date: 3 June 2011
Source: Scopus},
  Number                   = {2},
  Pages                    = {339-353},
  Volume                   = {20},

  Type                     = {Journal Article},
  Url                      = {http://www.scopus.com/inward/record.url?eid=2-s2.0-0028410663&partnerID=40&md5=b850a99a3095013aeb85dbbc3f8c9264}
}

@Article{Chibbaro2008,
  Title                    = {Langevin PDF simulation of particle deposition in a turbulent pipe flow},
  Author                   = {Chibbaro, Sergio and Minier, Jean-Pierre},
  Journal                  = {Journal of Aerosol Science},
  Year                     = {2008},
  Note                     = {doi: DOI: 10.1016/j.jaerosci.2008.03.002},
  Number                   = {7},
  Pages                    = {555-571},
  Volume                   = {39},

  ISSN                     = {0021-8502},
  Keywords                 = {Deposition
Stochastic modeling
Numerical simulations},
  Type                     = {Journal Article},
  Url                      = {http://www.sciencedirect.com/science/article/B6V6B-4S26680-1/2/d648902895dad0e1d8d3bfb662e8a3f4}
}

@Article{Cho2010,
  Title                    = {Nrf2 protects against airway disorders},
  Author                   = {Cho, Hye-Youn and Kleeberger, Steven R.},
  Journal                  = {Toxicology and Applied Pharmacology},
  Year                     = {2010},
  Number                   = {1},
  Pages                    = {43-56},
  Volume                   = {244},

  Abstract                 = {Nuclear factor-erythroid 2 related factor 2 (Nrf2) is a ubiquitous master transcription factor that regulates antioxidant response elements (AREs)-mediated expression of antioxidant enzyme and cytoprotective proteins. In the unstressed condition, Kelch-like ECH-associated protein 1 (Keap1) suppresses cellular Nrf2 in cytoplasm and drives its proteasomal degradation. Nrf2 can be activated by diverse stimuli including oxidants, pro-oxidants, antioxidants, and chemopreventive agents. Nrf2 induces cellular rescue pathways against oxidative injury, abnormal inflammatory and immune responses, apoptosis, and carcinogenesis. Application of Nrf2 germ-line mutant mice has identified an extensive range of protective roles for Nrf2 in experimental models of human disorders in the liver, gastrointestinal tract, airway, kidney, brain, circulation, and immune or nerve system. In the lung, lack of Nrf2 exacerbated toxicity caused by multiple oxidative insults including supplemental respiratory therapy (e.g., hyperoxia, mechanical ventilation), cigarette smoke, allergen, virus, bacterial endotoxin and other inflammatory agents (e.g., carrageenin), environmental pollution (e.g., particles), and a fibrotic agent bleomycin. Microarray analyses and bioinformatic studies elucidated functional AREs and Nrf2-directed genes that are critical components of signaling mechanisms in pulmonary protection by Nrf2. Association of loss of function with promoter polymorphisms in NRF2 or somatic and epigenetic mutations in KEAP1 and NRF2 has been found in cohorts of patients with acute lung injury/acute respiratory distress syndrome or lung cancer, which further supports the role for NRF2 in these lung diseases. In the current review, we address the role of Nrf2 in airways based on emerging evidence from experimental oxidative disease models and human studies.},
  Doi                      = {10.1016/j.taap.2009.07.024},
  ISSN                     = {0041-008X},
  Keywords                 = {ARE
Keap1
Lung
Knockout mice
Oxidant
Polymorphism
Mutation},
  Type                     = {Journal Article},
  Url                      = {http://www.sciencedirect.com/science/article/pii/S0041008X09003068}
}

@Article{Choi2007,
  Title                    = {A Brownian dynamics simulation method for analyzing particle behavior in nonuniform and alternating electric fields},
  Author                   = {Choi, Jinyoung and Park, Seokjoo and Kim, Sangsoo},
  Journal                  = {Journal of Aerosol Science},
  Year                     = {2007},
  Note                     = {doi: DOI: 10.1016/j.jaerosci.2006.11.003},
  Number                   = {2},
  Pages                    = {192-210},
  Volume                   = {38},

  ISSN                     = {0021-8502},
  Keywords                 = {Brownian dynamics simulation (BDS)
Diffusion
Alternating electric field
Quadrupole electric field},
  Type                     = {Journal Article},
  Url                      = {http://www.sciencedirect.com/science/article/B6V6B-4MV1H5X-1/2/7271c9d34776e59bf9c7eb0d7533b633}
}

@Article{Chouly2008,
  Title                    = {Numerical and experimental study of expiratory flow in the case of major upper airway obstructions with fluid-structure interaction},
  Author                   = {Chouly, F. and Van Hirtum, A. and Lagrée, P. Y. and Pelorson, X. and Payan, Y.},
  Journal                  = {Journal of Fluids and Structures},
  Year                     = {2008},
  Number                   = {2},
  Pages                    = {250-269},
  Volume                   = {24},

  Abstract                 = {This study deals with the numerical prediction and experimental description of the flow-induced deformation in a rapidly convergent-divergent geometry which stands for a simplified tongue, in interaction with an expiratory airflow. An original in vitro experimental model is proposed, which allows measurement of the deformation of the artificial tongue, in condition of major initial airway obstruction. The experimental model accounts for asymmetries in geometry and tissue properties which are two major physiological upper airway characteristics. The numerical method for prediction of the fluid-structure interaction is described. The theory of linear elasticity in small deformations has been chosen to compute the mechanical behaviour of the tongue. The main features of the flow are taken into account using a boundary layer theory. The overall numerical method entails finite element solving of the solid problem and finite differences solving of the fluid problem. First, the numerical method predicts the deformation of the tongue with an overall error of the order of 20%, which can be seen as a preliminary successful validation of the theory and simulations. Moreover, expiratory flow limitation is predicted in this configuration. As a result, both the physical and numerical models could be useful to understand this phenomenon reported in heavy snorers and apneic patients during sleep.},
  ISSN                     = {0889-9746},
  Keywords                 = {Upper airway
Fluid-structure interaction
Finite element modelling
Boundary layer theory
Numerical simulation
In vitro experiment
Expiratory flow limitation
Snoring
Obstructive sleep apnea},
  Type                     = {Journal Article},
  Url                      = {http://www.sciencedirect.com/science/article/B6WJG-4PXNHGH-1/2/966027d544213647f6473ba63d0dff90}
}

@Article{Chouly2006,
  Title                    = {Fluid-structure interaction in obstructive sleep apnea: Validation of numerical simulations using in-vitro measurements},
  Author                   = {Chouly, F. and Van Hirtum, A. and Lagrée, P. Y. and Pelorson, X. and Payan, Y.},
  Journal                  = {Journal of Biomechanics},
  Year                     = {2006},
  Number                   = {Supplement 1},
  Pages                    = {S441-S441},
  Volume                   = {39},

  ISSN                     = {0021-9290},
  Type                     = {Journal Article},
  Url                      = {http://www.sciencedirect.com/science/article/B6T82-4KR88PB-2F5/2/a23bdbf544c4c7e7a7e06aba534809cf}
}

@Article{Chouly2009,
  Title                    = {Modelling the human pharyngeal airway: validation of numerical simulations using in vitro experiments},
  Author                   = {Chouly, Franz and Van Hirtum, Annemie and Lagrée, Pierre-Yves and Pelorson, Xavier and Payan, Yohan},
  Journal                  = {Medical and Biological Engineering and Computing},
  Year                     = {2009},
  Note                     = {10.1007/s11517-008-0412-1},
  Number                   = {1},
  Pages                    = {49-58},
  Volume                   = {47},

  Abstract                 = {Abstract&nbsp;&nbsp;In the presented study, a numerical model which predicts the flow-induced collapse within the pharyngeal airway is validated using in vitro measurements. Theoretical simplifications were considered to limit the computation time. Systematic comparisons between simulations and measurements were performed on an in vitro replica, which reflects asymmetries of the geometry and of the tissue properties at the base of the tongue and in pathological conditions (strong initial obstruction). First, partial obstruction is observed and predicted. Moreover, the prediction accuracy of the numerical model is of 4.2% concerning the deformation (mean quadratic error on the constriction area). It shows the ability of the assumptions and method to predict accurately and quickly a fluid–structure interaction.},
  Type                     = {Journal Article},
  Url                      = {http://dx.doi.org/10.1007/s11517-008-0412-1}
}

@Article{Chowdhury1992,
  Title                    = {A thermodynamically consistent rate-dependent model for turbulence-part II. Computational results},
  Author                   = {Chowdhury, S. J. and Ahmadi, G.},
  Journal                  = {International Journal of Non-Linear Mechanics},
  Year                     = {1992},
  Note                     = {Cited By (since 1996): 7
Export Date: 3 June 2011
Source: Scopus},
  Number                   = {4},
  Pages                    = {705-718},
  Volume                   = {27},

  Type                     = {Journal Article},
  Url                      = {http://www.scopus.com/inward/record.url?eid=2-s2.0-0026898567&partnerID=40&md5=e7415e6e005798204bae6cc5edcff564}
}

@Article{Chu2006,
  Title                    = {An experimental investigation on airflow in human airway},
  Author                   = {Chu, S. K. and Kim, S. K.},
  Journal                  = {Journal of Biomechanics},
  Year                     = {2006},
  Number                   = {Supplement 1},
  Pages                    = {S272-S272},
  Volume                   = {39},

  ISSN                     = {0021-9290},
  Type                     = {Journal Article},
  Url                      = {http://www.sciencedirect.com/science/article/B6T82-4KR88PB-1G4/2/6b4f0baaeb2fb51cd89b6e52a4b72e49}
}

@Article{ChunXua2009,
  Title                    = {Modeling upper airway collapse by a finite element model with regional tissue properties},
  Author                   = {Chun Xua, Michael J. Brennickb, Lawrence Doughertyc, David M. Woottond},
  Journal                  = {Medical Engineering \& Physics},
  Year                     = {2009},
  Pages                    = {5},
  Volume                   = {31},

  Type                     = {Journal Article}
}

@Article{Chung1997,
  Title                    = { Experimental investigation of a two-dimensional cylindrical sampler},
  Author                   = {Chung, I.P. and Dunn-Rankin, D.},
  Journal                  = {Journal of Aerosol Science},
  Year                     = {1997},
  Number                   = {5},
  Pages                    = {935-955},
  Volume                   = {25},

  Type                     = {Journal Article}
}

@Article{Chung2000,
  Title                    = {A study on dust emission, particle size distribution and formaldehyde concentration during machining of medium density fibreboard},
  Author                   = {Chung, K.Y.K and Cuthber, R.J. and Revell, G.S. and Wassel, S.G. and Summer, N.},
  Journal                  = {Annals of Occ. Hyg.},
  Year                     = {2000},
  Pages                    = {455-466},
  Volume                   = {44},

  Type                     = {Journal Article}
}

@Article{Chung1999,
  Title                    = {Three-dimensional analysis of airflow and contaminant particle transport in a partitioned enclosure},
  Author                   = {Chung, K. C.},
  Journal                  = {Building and Environment},
  Year                     = {1999},
  Number                   = {1},
  Pages                    = {7-17},
  Volume                   = {34},

  Doi                      = {10.1016/s0360-1323(97)00073-5},
  ISSN                     = {0360-1323},
  Type                     = {Journal Article},
  Url                      = {http://www.sciencedirect.com/science/article/pii/S0360132397000735}
}

@Article{Chung2008,
  Title                    = {Digital particle image velocimetry studies of nasal airflow},
  Author                   = {Chung, S.K. and Kim, S.K.},
  Journal                  = {Respiratory Physiology \& Neurobiology},
  Year                     = {2008},
  Note                     = {doi: DOI: 10.1016/j.resp.2008.07.023},
  Number                   = {1-3},
  Pages                    = {111-120},
  Volume                   = {163},

  Abstract                 = {Understanding the properties of airflow in the nasal cavity is essential to understanding physiologic and pathologic aspects of nasal breathing. Many attempts have been made to evaluate nasal airflow patterns using the best possible analytical methods available at the time. Recently, digital particle image velocimetry (DPIV) and computational fluid dynamic methods have been applied to this area. Digital PIV is an experimental method used to evaluate airflow in an accurately reproduced transparent model of the nasal cavity. In this review, use of the DPIV procedure in the study of nasal airflow, airflow patterns in quiet respiration, and changes to airflow after modification of the nasal turbinates are reviewed, along with aspects of the DPIV technique and the future role of DPIV in this field of research.},
  ISSN                     = {1569-9048},
  Keywords                 = {Nasal airflow
Particle image velocimetry
Human nasal cavity
Turbinectomy
3D model},
  Type                     = {Journal Article},
  Url                      = {http://www.sciencedirect.com/science/article/B6X16-4T4Y5X6-2/2/6071f0c038b4ccd96d6c5d3734c7b39f}
}

@Article{Chung2008a,
  Title                    = {Digital particle image velocimetry studies of nasal airflow},
  Author                   = {Chung, Seung-Kyu and Kim, Sung Kyun},
  Journal                  = {Respiratory Physiology \& Neurobiology},
  Year                     = {2008},
  Number                   = {1-3},
  Pages                    = {111-120},
  Volume                   = {163},

  Abstract                 = {Understanding the properties of airflow in the nasal cavity is essential to understanding physiologic and pathologic aspects of nasal breathing. Many attempts have been made to evaluate nasal airflow patterns using the best possible analytical methods available at the time. Recently, digital particle image velocimetry (DPIV) and computational fluid dynamic methods have been applied to this area. Digital PIV is an experimental method used to evaluate airflow in an accurately reproduced transparent model of the nasal cavity. In this review, use of the DPIV procedure in the study of nasal airflow, airflow patterns in quiet respiration, and changes to airflow after modification of the nasal turbinates are reviewed, along with aspects of the DPIV technique and the future role of DPIV in this field of research.},
  ISSN                     = {1569-9048},
  Keywords                 = {Nasal airflow
Particle image velocimetry
Human nasal cavity
Turbinectomy
3D model},
  Type                     = {Journal Article},
  Url                      = {http://www.sciencedirect.com/science/article/B6X16-4T4Y5X6-2/2/6071f0c038b4ccd96d6c5d3734c7b39f}
}

@Article{Churchill2004,
  Title                    = {Morphological variation and airflow dynamics in the human nose},
  Author                   = {Churchill, S. and Shackelford, L.L. and Georgi, N. and Black, M.},
  Journal                  = {Am J Human Biol},
  Year                     = {2004},
  Pages                    = {625-638},
  Volume                   = {16},

  Type                     = {Journal Article}
}

@Article{Cisonni,
  Title                    = {Effect of the velopharynx on intraluminal pressures in reconstructed pharynges derived from individuals with and without sleep apnea},
  Author                   = {Cisonni, Julien and Lucey, Anthony D. and Walsh, Jennifer H. and King, Andrew J. C. and Elliott, Novak S. J. and Sampson, David D. and Eastwood, Peter R. and Hillman, David R.},
  Journal                  = {Journal of Biomechanics},
  Number                   = {0},

  Doi                      = {http://dx.doi.org/10.1016/j.jbiomech.2013.07.007},
  ISSN                     = {0021-9290},
  Keywords                 = {Airway resistance
CFD
Obstructive sleep apnea
Pharyngeal wall pressure
Velopharynx},
  Type                     = {Journal Article},
  Url                      = {http://www.sciencedirect.com/science/article/pii/S0021929013003187}
}

@Article{Clark1995,
  Title                    = {Medical aerosol inhaler: Past, present and future},
  Author                   = {Clark, A.R.},
  Journal                  = {Aerosol Science and Technology},
  Year                     = {1995},
  Number                   = {4},
  Pages                    = {374-391},
  Volume                   = {22},

  Type                     = {Journal Article}
}

@Article{Clark1994,
  Title                    = {Modelling the deposition of inhaled powdered drug aerosols},
  Author                   = {Clark, Andrew R. and Egan, Mike},
  Journal                  = {Journal of Aerosol Science},
  Year                     = {1994},
  Note                     = {doi: DOI: 10.1016/0021-8502(94)90189-9},
  Number                   = {1},
  Pages                    = {175-186},
  Volume                   = {25},

  ISSN                     = {0021-8502},
  Type                     = {Journal Article},
  Url                      = {http://www.sciencedirect.com/science/article/B6V6B-488G4W7-J/2/2d8d97f66fa8c7037d00b29436c74f64}
}

@Article{Clark2009,
  Title                    = {An improved model for particle deposition in porous foams},
  Author                   = {Clark, Phillip and Koehler, Kirsten A. and Volckens, John},
  Journal                  = {Journal of Aerosol Science},
  Year                     = {2009},
  Note                     = {doi: DOI: 10.1016/j.jaerosci.2009.02.005},
  Number                   = {7},
  Pages                    = {563-572},
  Volume                   = {40},

  Abstract                 = {Porous foam provides an inexpensive, light-weight and effective medium to capture physiologically-relevant aerosol fractions. It can be manufactured to have a wide range of properties relevant to aerosol deposition. A series of laboratory experiments were conducted to measure particle penetration though porous foam media of varying pore size and foam length. Both solid and liquid aerosols (0.01-10 [mu]m diameter) were tested using a Sequenzial Mobility Particle Sizer or Aerodynamic Particle Sizer to count and size particles penetrating the foam. With this data, an existing semi-empirical model was improved upon to predict particle penetration through a foam of a given fiber diameter, and thickness. The model is based on three dimensionless parameters (St, Ng, Pe) that account for inertial, gravitational, and diffusive modes of deposition, respectively.},
  ISSN                     = {0021-8502},
  Keywords                 = {Porous foam
Aerosol sampling
Aerosol penetration model},
  Type                     = {Journal Article},
  Url                      = {http://www.sciencedirect.com/science/article/B6V6B-4VWB1CR-4/2/96aef3c430b8fd513294909b1a16d43e}
}

@Article{Clarke1992,
  Title                    = {The role of mucosal receptors in the nasal sensation of airflow},
  Author                   = {Clarke, R.W. and Jones, A.S. and Charters, P. and Sherman, I.},
  Journal                  = {Clinical Otolaryngology \& Allied Sciences},
  Year                     = {1992},
  Note                     = {10.1111/j.1365-2273.1992.tb01679.x},
  Number                   = {5},
  Pages                    = {383-387},
  Volume                   = {17},

  Abstract                 = {50 subjects were admitted into a randomized double-blind placebo controlled cross-over trial with 4% lignocaine as the active drug and normal saline as the placebo. Each subject had 2 ml of solution sprayed into each nasal cavity and all subjects had both sprays but on different occasions. The order in which the sprays were administered was randomized. The subjective sensation of nasal airflow was measured using a visual anologue scale before and after the spray. These measurements were made under conditions of the same airflow rate, which was monitored throughout the experiment using a reprogrammed NR6 rhinomanometer. Objective nasal patency was measured as peak nasal inspiratory flow rate. It was found that the nasal sensation of airflow decreased slightly after both lignocaine (difference between medians 5.0; 95% confidence interval 20142.91 to 6.11) and normal saline (difference between medians 6.0; 95% confidence interval 2014 1.02 to 7.21). Nonparametric analysis of variance showed this difference to be non-significant (P = 0.73). In addition there was no significant change in objective nasal patency. The results suggest that nerve endings in the nasal mucosa play no part in sensing nasal airflow during respiration.},
  ISSN                     = {1365-2273},
  Type                     = {Journal Article},
  Url                      = {http://dx.doi.org/10.1111/j.1365-2273.1992.tb01679.x}
}

@Article{Cleaver1975,
  Title                    = {A sub layer model for the deposition of particles from a turbulent flow},
  Author                   = {Cleaver, J. W. and Yates, B.},
  Journal                  = {Chemical Engineering Science},
  Year                     = {1975},
  Number                   = {8},
  Pages                    = {983-992},
  Volume                   = {30},

  ISSN                     = {0009-2509},
  Type                     = {Journal Article},
  Url                      = {http://www.sciencedirect.com/science/article/pii/0009250975800650}
}

@Book{Clift1978,
  Title                    = {Bubbles, Drops, and Particles},
  Author                   = {Clift, R. and Grace, J.R. and Weber, M.E. },
  Publisher                = {Academic Press Inc. London, Ltd.},
  Year                     = {1978},

  Address                  = {London, UK},

  Type                     = {Book}
}

@Book{Clift1978a,
  Title                    = {Drops and Particles},
  Author                   = {Clift, R. and Grace, J. R. and Weber, M. E. },
  Publisher                = {Academic Press},
  Year                     = {1978},

  Address                  = {New York. },

  Type                     = {Book}
}

@Article{Coates2004,
  Title                    = {Effect of design on the performance of a dry powder inhaler using computational fluid dynamics. Part 1: Grid structure and mouthpiece length},
  Author                   = {Coates, M. S. and Fletcher, D. F. and Chan, H. K. and Raper, J. A.},
  Journal                  = {J Pharm Sci},
  Year                     = {2004},
  Note                     = {Using Smart Source Parsing
Nov},
  Number                   = {11},
  Pages                    = {2863-76},
  Volume                   = {93},

  Abstract                 = {This study investigates (1) the effect of modifying the design of a dry powder inhaler on the device performance, and (2) which design features significantly contribute to overall inhaler performance. Computational Fluid Dynamics (CFD) analysis was performed to determine how the flowfield generated in an Aerolizer at 60 l min(-1) varied when the inhaler grid and mouthpiece were modified. The computational models were validated by Laser Doppler Velocimetry (LDV). Dispersion performance of the modified inhalers was measured with a mannitol powder using a multistage liquid impinger at 60 l min(-1). The inhaler grid was found to significantly affect the performance of the Aerolizer. As the grid voidage was increased, the amount of powder retained in the device doubled (due to increased tangential flow of particles in the inhaler mouthpiece) and the FPF(Loaded) was reduced from 57 to 44% (due to increased mouthpiece retention). The length of the mouthpiece played a lesser role on the inhaler performance, having no significant effect on the flowfield generated in the devices. In summary, the performance of a dry powder inhaler can be affected by simple design changes. CFD, coupled with experimental results, provides a rational basis for understanding the performance difference.},
  ISSN                     = {0022-3549 (Print)
0022-3549 (Linking)},
  Type                     = {Journal Article}
}

@Article{Coates2004a,
  Title                    = {Effect of design on the performance of a dry powder inhaler using computational fluid dynamics. Part 1: Grid structure and mouthpiece length},
  Author                   = {Coates, Matthew S. and Fletcher, David F. and Chan, Hak-Kim and Raper, Judy A.},
  Journal                  = {Journal of Pharmaceutical Sciences},
  Year                     = {2004},
  Number                   = {11},
  Pages                    = {2863-2876},
  Volume                   = {93},

  ISSN                     = {1520-6017},
  Type                     = {Journal Article},
  Url                      = {http://dx.doi.org/10.1002/jps.20201}
}

@Article{Coffey2002,
  Title                    = {Langevin equation method for the rotational Brownian motion and orientational relaxation in liquids},
  Author                   = {Coffey, W. T. and et al.},
  Journal                  = {Journal of Physics A: Mathematical and General},
  Year                     = {2002},
  Number                   = {32},
  Pages                    = {6789},
  Volume                   = {35},

  Abstract                 = {A theory of orientational relaxation for the inertial rotational Brownian motion of a linear molecule (rotator in space) is developed in the context of the Langevin equation method without recourse to the Fokker–Planck equation. The general term in the time-dependent infinite hierarchy of differential-recurrence relations for the orientational correlation functions describing the relaxation behaviour of the system is derived by averaging the corresponding Euler–Langevin equation. The solution of this hierarchy is obtained in terms of continued fractions. The correlation times and the spectra of the orientational correlation functions are calculated for typical values of the model parameters.},
  ISSN                     = {0305-4470},
  Type                     = {Journal Article},
  Url                      = {http://stacks.iop.org/0305-4470/35/i=32/a=302}
}

@Article{CohenHubal1996,
  Title                    = {Incorporation of Nasal-Lining Mass-Transfer Resistance into Acfd Model for Prediction of Ozone Dosimetry in the Upper Respiratory Tract},
  Author                   = {Cohen Hubal, Elaine A. and Kimbell, Julia S. and Fedkiw, Peter S.},
  Journal                  = {Inhalation Toxicology},
  Year                     = {1996},
  Number                   = {9},
  Pages                    = {831-857},
  Volume                   = {8},

  Doi                      = {doi:10.3109/08958379609034267},
  Type                     = {Journal Article},
  Url                      = {http://informahealthcare.com/doi/abs/10.3109/08958379609034267}
}

@Article{Cohen1990,
  Title                    = {Deposition of ultrafine particles in the upper airways: An empirical analysis},
  Author                   = {Cohen, B. S. and Asgharian, B.},
  Journal                  = {Journal of Aerosol Science},
  Year                     = {1990},
  Note                     = {doi: DOI: 10.1016/0021-8502(90)90044-X},
  Number                   = {6},
  Pages                    = {789-797},
  Volume                   = {21},

  ISSN                     = {0021-8502},
  Type                     = {Journal Article},
  Url                      = {http://www.sciencedirect.com/science/article/B6V6B-488Y3YT-R/2/0418de110ce66162b7af8cd9df4fef5c}
}

@Article{Cohen1990a,
  Title                    = {Ultrafine particle deposition in a human tracheobronchial cast},
  Author                   = {Cohen, B. S. and Sussman, R. G. and Lippmann, M.},
  Journal                  = {Aerosol Science and Technology},
  Year                     = {1990},
  Note                     = {Cited By (since 1996): 68
Export Date: 3 June 2011
Source: Scopus},
  Number                   = {4},
  Pages                    = {1082-1091},
  Volume                   = {12},

  Type                     = {Journal Article},
  Url                      = {http://www.scopus.com/inward/record.url?eid=2-s2.0-0025428328&partnerID=40&md5=84f2b0749996232bbb7780404f463d55}
}

@Article{Cokelet2005,
  Title                    = {Magnetic resonance microscopy determined velocity and hematocrit distributions in a Couette viscometer},
  Author                   = {Cokelet, G. R. and Brown, J. R. and Codd, S. L. and Seymour, J. D.},
  Journal                  = {Biorheology},
  Year                     = {2005},
  Note                     = {Cited By (since 1996): 4
Export Date: 3 June 2011
Source: Scopus},
  Number                   = {5},
  Pages                    = {385-399},
  Volume                   = {42},

  Type                     = {Journal Article},
  Url                      = {http://www.scopus.com/inward/record.url?eid=2-s2.0-32444444893&partnerID=40&md5=c6dd007c9817876ed82c6cc2755d8462}
}

@Article{Cokelet1991,
  Title                    = {Decreased hydrodynamic resistance in the two-phase flow of blood through small vertical tubes at low flow rates},
  Author                   = {Cokelet, G. R. and Goldsmith, H. L.},
  Journal                  = {Circulation Research},
  Year                     = {1991},
  Note                     = {Cited By (since 1996): 82
Export Date: 3 June 2011
Source: Scopus},
  Number                   = {1},
  Pages                    = {1-17},
  Volume                   = {68},

  Type                     = {Journal Article},
  Url                      = {http://www.scopus.com/inward/record.url?eid=2-s2.0-0026098613&partnerID=40&md5=1aa4a7c1b314a0898adf5ae3bcbf6a3d}
}

@InBook{Cole1982,
  Title                    = {Upper Respiratory Airflow},
  Author                   = {Cole, P.},
  Editor                   = {Proctor, D.F. and Andersen, I.},
  Pages                    = {163-185},
  Publisher                = {Elsevier Biomedical Press},
  Year                     = {1982},

  Address                  = {Amsterdam},
  Type                     = {Book Section},

  Booktitle                = {The Nose: Upper Airway Physiology and the Atmospheric Environment}
}

@Misc{Collins2000,
  Title                    = {Reynolds number scaling of preferential concentration of particles in isotropic turbulence},

  Author                   = {Collins, L.R.},
  Month                    = {Nov. 13-17},
  Year                     = {2000},

  Type                     = {Conference Paper}
}

@Article{Collins2007,
  Title                    = {A computational fluid dynamics study of inspiratory flow in orotracheal geometries},
  Author                   = {Collins, T.P. and Tabor, G.R. and Young, P.G.},
  Journal                  = {Med Bio Eng Comput},
  Year                     = {2007},
  Number                   = {9},
  Pages                    = {829-836},
  Volume                   = {45},

  Type                     = {Journal Article}
}

@Article{Colonius2004,
  Title                    = {Computational aeroacoustics: progress on nonlinear problems of sound generation},
  Author                   = {Colonius, Tim and Lele, Sanjiva K.},
  Journal                  = {Progress in Aerospace Sciences},
  Year                     = {2004},
  Number                   = {6},
  Pages                    = {345-416},
  Volume                   = {40},

  Abstract                 = {Computational approaches are being developed to study a range of problems in aeroacoustics. These aeroacoustic problems may be classified based on the physical processes responsible for the sound radiation, and range from linear problems of radiation, refraction, and scattering in known base flows or by solid bodies, to sound generation by turbulence. In this article, we focus mainly on the challenges and successes associated with numerically simulating sound generation by turbulent flows. We discuss a hierarchy of computational approaches that range from semi-empirical schemes that estimate the noise sources using mean-flow and turbulence statistics, to high-fidelity unsteady flow simulations that resolve the sound generation process by direct application of the fundamental conservation principles. We stress that high-fidelity methods such as Direct Numerical Simulation (DNS) and Large Eddy Simulation (LES) have their merits in helping to unravel the flow physics and the mechanisms of sound generation. They also provide rich databases for modeling activities that will ultimately be needed to improve existing predictive capabilities. Spatial and temporal discretization schemes that are well-suited for aeroacoustic calculations are analyzed, including the effects of artificial dispersion and dissipation on uniform and nonuniform grids. We stress the importance of the resolving power of the discretization as well as computational efficiency of the overall scheme. Boundary conditions to treat the flow of disturbances in and out of the computational domain, as well as methods to mimic anechoic domain extension are discussed. Test cases on some benchmark problems are included to provide a realistic assessment of several boundary condition treatments. Finally, highlights of recent progress are given using selected model problems. These include subsonic cavity noise and jet noise. In the end, the current challenges in aeroacoustic modeling and in simulation algorithms are revisited with a look toward the future developments.},
  ISSN                     = {0376-0421},
  Type                     = {Journal Article},
  Url                      = {http://www.sciencedirect.com/science/article/B6V3V-4DW3NDY-1/2/bac23911d5ba98084ffc5b7bd4be9b9b}
}

@Article{Comer2001,
  Title                    = {Flow structures and particle deposition patterns in double-bifurcation airway models. Part 1. Air flow field},
  Author                   = {Comer, J.K. and Kleinstreuer, C. and Zhang, Z.},
  Journal                  = { Journal Fluid Mechanics},
  Year                     = {2001},
  Pages                    = {25–54},
  Volume                   = {435},

  Type                     = {Journal Article}
}

@Article{Comer2001a,
  Title                    = {Flow structures and particle deposition patterns in double-bifurcation airway models. Part 2. Aerosol transport and deposition},
  Author                   = {Comer, J.K. and Kleinstreuer, C. and Zhang, Z.},
  Journal                  = { Journal Fluid Mechanics},
  Year                     = {2001},
  Pages                    = {55-80},
  Volume                   = {435},

  Type                     = {Journal Article}
}

@Article{Comer2000,
  Title                    = {Aerosol transport and deposition in sequentially bifurcating airways},
  Author                   = {Comer, J. K. and Kleinstreuer, C. and Hyun, S. and Kim, C. S.},
  Journal                  = {Journal of Biomechanical Engineering},
  Year                     = {2000},
  Note                     = {Cited By (since 1996): 51
Export Date: 3 June 2011
Source: Scopus},
  Number                   = {2},
  Pages                    = {152-158},
  Volume                   = {122},

  Type                     = {Journal Article},
  Url                      = {http://www.scopus.com/inward/record.url?eid=2-s2.0-0034177941&partnerID=40&md5=d38d749210bf2cc165279645d4545762}
}

@Article{Comte2008,
  Title                    = {Simulation of the reduction of unsteadiness in a passively controlled transonic cavity flow},
  Author                   = {Comte, P. and Daude, F. and Mary, I.},
  Journal                  = {Journal of Fluids and Structures},
  Year                     = {2008},
  Number                   = {8},
  Pages                    = {1252-1261},
  Volume                   = {24},

  Abstract                 = {A 30 dB reduction of the peak pressure tone and a reduction by 6 dB of the background pressure found in an experiment of high-subsonic cavity flow controlled by a spanwise rod are retrieved numerically. The injection of deterministic upstream fluctuations in the large-eddy simulation (LES) domain is found to be of crucial importance, in contrast with the baseflow case. Reduction of the vortex impingement onto the aft edge of the cavity is confirmed, together with reduction of mass flow rate breathing through the grazing plane. Visual evidence of merging between the Kelvin-Helmholtz-type vortices shed downstream of the fore edge of the cavity and the von Kármán vortices shed behind the cylinder is provided. Shocklets downstream of the cylinder are also observed.},
  ISSN                     = {0889-9746},
  Type                     = {Journal Article},
  Url                      = {http://www.sciencedirect.com/science/article/B6WJG-4V2H751-4/2/081ec5fdb9c03ca4b9000b4d9ee37dbd}
}

@Article{Cooper1986,
  Title                    = {Particulate contamination and microelectronics manufacturing: An introduction},
  Author                   = {Cooper, D. W.},
  Journal                  = {Aerosol Science and Technology},
  Year                     = {1986},
  Note                     = {Cited By (since 1996): 19
Export Date: 3 June 2011
Source: Scopus},
  Number                   = {3},
  Pages                    = {287-299},
  Volume                   = {5},

  Type                     = {Journal Article},
  Url                      = {http://www.scopus.com/inward/record.url?eid=2-s2.0-0000893718&partnerID=40&md5=a867a03eb399264d0ef8565f45234c3b}
}

@Article{Corey1997,
  Title                    = {A comparison of the nasal cross-sectional areas and volumes obtained with acoustic rhinometry and magnetic resonance imaging},
  Author                   = {Corey, J.P. and Gungor, A. and Nelson, R. and Fredberg, J. and Lai, V.},
  Journal                  = {Otolaryngology - Head and Neck Surgery },
  Year                     = {1997},
  Number                   = {4},
  Pages                    = {349-354},
  Volume                   = {117},

  Type                     = {Journal Article}
}

@Article{Corley2005,
  Title                    = {Development of a physiologically based pharmacokinetic model for propylene glycol monomethyl ether and its acetate in rats and humans},
  Author                   = {Corley, R. A. and Gies, R. A. and Wu, H. and Weitz, K. K.},
  Journal                  = {Toxicology Letters},
  Year                     = {2005},
  Number                   = {1},
  Pages                    = {193-213},
  Volume                   = {156},

  Abstract                 = {Propylene glycol monomethyl ether (PM), along with its acetate, is the most widely used of the propylene glycol ether family of solvents. The most common toxic effects of PM observed in animal studies include sedation, very slight alpha2u-globulin mediated nephropathy (male rats only) and hepatomegally at high exposures (typically > 1000 ppm). Sedation in animal studies usually resolves within a few exposures to 3000 ppm (the highest concentration used in subchronic and chronic inhalation studies) due to the induction of metabolizing enzymes. Data from a variety of pharmacokinetic and mechanistic studies have been incorporated into a PBPK model for PM and its acetate in rats and mice. Published controlled exposure and workplace biomonitoring studies have also been included for comparisons of the internal dosimetry of PM and its acetate between laboratory animals and humans. PM acetate is rapidly hydrolyzed to PM, which is further metabolized to either glucuronide or sulfate conjugates (minor pathways) or propylene glycol (major pathway). In vitro half-lives for PM acetate range from 14 to 36 min depending upon the tissue and species. In vivo half-lives are considerably faster, reflecting the total contributions of esterases in the blood and tissues of the body, and are on the order of just a few minutes. Thus, very little PM acetate is found in vivo and, other than potential portal of entry irritation, the toxicity of PM acetate is related to PM. Regardless of the source for PM (either PM or its acetate), rats were predicted to have a higher Cmax and AUC for PM in blood than humans, especially at concentrations greater than the current ACGIH TLV of 100 ppm. This would indicate that the major systemic effects of PM would be expected to be less severe in humans than rats at comparable inhalation exposures.},
  ISSN                     = {0378-4274},
  Keywords                 = {Propylene glycol monomethyl ether
Propylene glycol monomethyl ether acetate
PBPK modeling},
  Type                     = {Journal Article},
  Url                      = {http://www.sciencedirect.com/science/article/B6TCR-4DWH3JT-2/2/43903b7c8b67bef6d480002bb8f46c79}
}

@Book{Corn1976,
  Title                    = {In "Air Pollution," },
  Author                   = {Corn, M. },
  Publisher                = {Ed. Stren, A.C., Academic Press},
  Year                     = {1976},

  Address                  = {New York},

  Type                     = {Book}
}

@Article{Corrsin1956,
  Title                    = {On the equation of motion for a particle in turbulent fluid},
  Author                   = {Corrsin, S. and Lumley, J.},
  Journal                  = {Applied Scientific Research},
  Year                     = {1956},
  Number                   = {2},
  Pages                    = {114-116},
  Volume                   = {6},

  Doi                      = {10.1007/bf03185030},
  ISSN                     = {0003-6994},
  Keywords                 = {Physics},
  Type                     = {Journal Article},
  Url                      = {http://dx.doi.org/10.1007/BF03185030}
}

@Article{Costa,
  Title                    = {Nanoparticle Processes Modelling: The Role of Key Parameters for Population Balances for On-Line Crystallization Processes Applications},
  Author                   = {Costa, Caliane Bastos Borba and Filho, Rubens Maciel},
  Journal                  = {Powder Technology},
  Volume                   = {In Press, Accepted Manuscript},

  Abstract                 = {The nanoparticle production with well defined properties is of great scientific and technological interest due to the increasing number of applications of such material. Crystallization may be a suitable unit operation for producing nanoparticles, since it may be conceptually designed to be an intensified process. The mean particle size, the particle morphology, the modality and the broadness of the particle size distribution (PSD) are important indications of the quality of a particulate product. By synthesizing nanocrystals with narrow PSD, it is possible to produce advanced nanomaterials with tailored properties. There are possible interaction effects that may be unique to nanoparticles due to the size dependence of properties. These effects must be accounted for in predictive models, which are useful either for exploratory investigations or as a tool for computer operated procedures. For online process control, time calculations should not be greater than a few seconds, and, therefore, a population balance model should be more appropriate than stochastic methods. However, at present there is no consensus and complete knowledge by the scientific community of all needed parameters and which are the computer tools for a workable framework for dealing with nanoparticles reliable process design and on-line applications. This paper proposes to discuss important aspects of nanoparticle crystallization processes in order to bring light to nanoparticle modelling. In order to do so, the key factors influencing nanoparticulate processes and how they can be accounted for, going from a molecular to a process engineering approach, are discussed.},
  ISSN                     = {0032-5910},
  Keywords                 = {Nanoparticle
Crystallization
Agglomeration Inhibition
Modelling
Solvation},
  Type                     = {Journal Article},
  Url                      = {http://www.sciencedirect.com/science/article/B6TH9-4YXT60F-2/2/a30a3746a219f5c0d7e6f4ce5d47b185}
}

@Article{Cottle1955,
  Title                    = {The structure and function of the nasal vestibule},
  Author                   = {Cottle, M.H.},
  Journal                  = {Arch Otolaryngol},
  Year                     = {1955},
  Pages                    = {173-181},
  Volume                   = {62},

  Type                     = {Journal Article}
}

@Article{Courtiss1983,
  Title                    = {The effects of nasal surgery on airflow},
  Author                   = {Courtiss, E.H. and Goldwyn, R.M.},
  Journal                  = {Plastic Reconstr Surg},
  Year                     = {1983},
  Pages                    = {9-19},
  Volume                   = {72},

  Type                     = {Journal Article}
}

@Article{Cox1967,
  Title                    = {The slow motion of a sphere through a viscous fluid towards a plane surface--II Small gap widths, including inertial effects},
  Author                   = {Cox, Raymond G. and Brenner, Howard},
  Journal                  = {Chemical Engineering Science},
  Year                     = {1967},
  Number                   = {12},
  Pages                    = {1753-1777},
  Volume                   = {22},

  ISSN                     = {0009-2509},
  Type                     = {Journal Article},
  Url                      = {http://www.sciencedirect.com/science/article/pii/0009250967802082}
}

@Article{CrA¼ts2008,
  Title                    = {Exposure to diesel exhaust induces changes in EEG in human volunteers.},
  Author                   = {Crüts, B. and van Etten, L. and Törnqvist, H. and Blomberg, A. and Sandström, T. and Mills, N.L. and Borm, P.J.},
  Journal                  = {Part Fibre Toxicol.},
  Year                     = {2008},
  Number                   = {5},
  Pages                    = {4},
  Volume                   = {11},

  Type                     = {Journal Article}
}

@Article{Craven2006,
  Title                    = {A Computational and Experimental Investigation of the Human Thermal Plume},
  Author                   = {Craven, Brent A. and Settles, Gary S.},
  Journal                  = {Journal of Fluids Engineering},
  Year                     = {2006},
  Number                   = {6},
  Pages                    = {1251-1258},
  Volume                   = {128},

  Keywords                 = {biothermics
stratified flow
flow visualisation
computational fluid dynamics
Navier-Stokes equations},
  Type                     = {Journal Article},
  Url                      = {http://link.aip.org/link/?JFG/128/1251/1}
}

@Article{Crawford1982,
  Title                    = {Identifying critical human subpopulations by age groups: radioactivity and the lung},
  Author                   = {Crawford, DJ},
  Journal                  = {Physics in medicine and biology},
  Year                     = {1982},
  Note                     = {C:\Users\sean\AppData\Roaming\Zotero\Zotero\Profiles\16a4oype.default\zotero\storage\SZ93KI4C\identifying critical human subpopulations by agegroups radioactivity and the lung.pdf},
  Pages                    = {539},
  Volume                   = {27},

  Type                     = {Journal Article},
  Url                      = {http://iopscience.iop.org/0031-9155/27/4/005/pdf/0031-9155_27_4_005.pdf}
}

@Article{Crawford1949,
  Title                    = {Determination of the specific gravity of ragweed pollen (Ambrosia elatior) and conversion of gravity sample counts to volumetric incidence},
  Author                   = {Crawford, J.H.},
  Journal                  = {Publ. Health Rep.},
  Year                     = {1949},
  Pages                    = {1195-1200},
  Volume                   = {64},

  Type                     = {Journal Article}
}

@Article{Croce2006,
  Title                    = {In vitro experiments and numerical simulations of airflow in realistic nasal airway geometry},
  Author                   = {Croce, C. and Fodil, R. and Durand, M. and Sbirlea-Apiou, G. and Caillibotte, G. and Papon, J.F. and Blondeau, J.R. and Coste, A. and Isabey, D. and Louis, B.},
  Journal                  = {Annals of Biomedical Engineering},
  Year                     = {2006},
  Number                   = {6},
  Pages                    = {997-1007},
  Volume                   = {34},

  Type                     = {Journal Article}
}

@TechReport{Crockett2002,
  Title                    = {Economic Case Statement. Chronic Obstructive Pulmonary Disease},
  Author                   = {Crockett, Alan J and Cranstoni, Josephine M and Moss, John R},
  Institution              = {The Australian Lung Foundation},
  Year                     = {2002},
  Type                     = {Report}
}

@Article{Crosland2009,
  Title                    = {Characterization of the spray velocities from a pressurized metered-dose inhaler},
  Author                   = {Crosland, B.M. and Johnson, M.R. and Matida, E.A.},
  Journal                  = {Journal of Aerosol Medicine and Pulmonary Drug Delivery},
  Year                     = {2009},
  Number                   = {2},
  Pages                    = {85-97},
  Volume                   = {22},

  Type                     = {Journal Article}
}

@Article{Crouse1999,
  Title                    = {Effects of Age, Body Mass Index, and Gender on Nasal Airflow Rate and Pressures},
  Author                   = {Crouse, Ulla and Laine-Alava, M. T.},
  Journal                  = {The Laryngoscope},
  Year                     = {1999},
  Number                   = {9},
  Pages                    = {1503--1508},
  Volume                   = {109},

  Doi                      = {10.1097/00005537-199909000-00027},
  ISSN                     = {1531-4995},
  Keywords                 = {Nasal airflow rate, oral, nasal pressure, nasal resistance, nasal cross, sectional area},
  Publisher                = {John Wiley \& Sons, Inc.},
  Url                      = {http://dx.doi.org/10.1097/00005537-199909000-00027}
}

@Article{Crowder2002,
  Title                    = {Fundamental effects of particle morphology on lung delivery: Predictions of Stokes law and the particular relevance to dry powder inhaler formulation and development},
  Author                   = {Crowder, T.M. and Rosati, J.A. and Schroeter, J.D. and Hickey, A.J. and Martonen, T.B.},
  Journal                  = {Pharama. Res.},
  Year                     = {2002},
  Number                   = {3},
  Pages                    = {239-245},
  Volume                   = {19},

  Type                     = {Journal Article}
}

@Article{Crowe1986,
  Title                    = {Gas-Solid Flows -},
  Author                   = {Crowe, C.T. },
  Journal                  = {ASME FED },
  Year                     = {1986},
  Pages                    = {91-96},
  Volume                   = {35},

  Type                     = {Journal Article}
}

@Book{Crowe1998,
  Title                    = {Multiphase flows with droplets and particles},
  Author                   = {Crowe, C.T. and Sommerfeld, M. and Tsuji, Y.},
  Publisher                = {CRC Press},
  Year                     = {1998},

  Address                  = {Boca Raton, Fla.},

  Type                     = {Book}
}

@Article{Crowe2000,
  Title                    = {On models for turbulence modulation in fluid-particle flows},
  Author                   = {Crowe, Clayton T.},
  Journal                  = {International Journal of Multiphase Flow},
  Year                     = {2000},
  Number                   = {5},
  Pages                    = {719-727},
  Volume                   = {26},

  Doi                      = {10.1016/s0301-9322(99)00050-6},
  ISSN                     = {0301-9322},
  Keywords                 = {Fluid-particle flows
Turbulence
Turbulence modulation},
  Type                     = {Journal Article},
  Url                      = {http://www.sciencedirect.com/science/article/pii/S0301932299000506}
}

@Article{Crowe1982,
  Title                    = {REVIEW - NUMERICAL MODELS FOR DILUTE GAS-PARTICLE FLOWS},
  Author                   = {Crowe, C. T.},
  Journal                  = {J FLUIDS ENG TRANS ASME},
  Year                     = {1982},
  Note                     = {Cited By (since 1996): 108
Export Date: 3 June 2011
Source: Scopus},
  Number                   = {N 3},
  Pages                    = {297-303},
  Volume                   = {V 104},

  Type                     = {Journal Article},
  Url                      = {http://www.scopus.com/inward/record.url?eid=2-s2.0-0020186123&partnerID=40&md5=bb310d5530e5c96bae5c5a60ed48fbdd}
}

@Article{Cryan2007,
  Title                    = {In vivo animal models for drug delivery across the lung mucosal barrier},
  Author                   = {Cryan, Sally-Ann and Sivadas, Neeraj and Garcia-Contreras, Lucila},
  Journal                  = {Advanced Drug Delivery Reviews},
  Year                     = {2007},
  Number                   = {11},
  Pages                    = {1133-1151},
  Volume                   = {59},

  Abstract                 = {Over recent years the research focus within the field of respiratory drug delivery has broadened to include a wide range of potential applications for inhalation by delivering drugs not just onto the lung mucosa but across it. The range of drugs being assessed is broad and includes both current and novel therapies and there are a growing number of additives that appear capable of enhancing systemic absorption. Comprehensive characterisation of drug delivery to the lungs is a complex task involving the determination of delivered, deposited and (for systemically-targeted drugs) absorbed dose. As it is difficult to simulate in vitro, in vivo whole animal models are still key to inhaled drug development. Because of the anatomical complexities and interspecies differences in the lungs, the appropriate choice of species and drug delivery method is vital during study design. New delivery devices designed specifically for animal studies as well as more sophisticated methods to determine drug deposition and absorption after inhalation are improving the information derived from these studies.},
  ISSN                     = {0169-409X},
  Keywords                 = {Systemic drug delivery
Respiratory delivery
Inhalation
Aerosol
Pharmacokinetics
Toxicity
Drug deposition
Drug absorption},
  Type                     = {Journal Article},
  Url                      = {http://www.sciencedirect.com/science/article/B6T3R-4PJCY5K-1/2/beb8aa8eb4b7d4ff9c0a812d569a0e4b}
}

@Article{Csanady1963,
  Title                    = {Turbulent Diffusion of Heavy Particles in the Atmosphere},
  Author                   = {Csanady, G.T.},
  Journal                  = {Journal of Atmospheric Sciences},
  Year                     = {1963},
  Number                   = {3},
  Pages                    = {201-208},
  Volume                   = {20},

  Type                     = {Journal Article}
}

@Article{Cu2009,
  Title                    = {Mathematical modeling of molecular diffusion through mucus},
  Author                   = {Cu, Yen and Saltzman, W. Mark},
  Journal                  = {Advanced Drug Delivery Reviews},
  Year                     = {2009},
  Number                   = {2},
  Pages                    = {101-114},
  Volume                   = {61},

  Abstract                 = {The rate of molecular transport through the mucus gel can be an important determinant of efficacy for therapeutic agents delivered by oral, intranasal, intravaginal/rectal, and intraocular routes. Transport through mucus can be described by mathematical models based on principles of physical chemistry and known characteristics of the mucus gel, its constituents, and of the drug itself. In this paper, we review mathematical models of molecular diffusion in mucus, as well as the techniques commonly used to measure diffusion of solutes in the mucus gel, mucus gel mimics, and mucosal epithelia.},
  ISSN                     = {0169-409X},
  Keywords                 = {Mass transport
Diffusion in gel
Imaging
Mucosa
Drug delivery
Particles},
  Type                     = {Journal Article},
  Url                      = {http://www.sciencedirect.com/science/article/B6T3R-4V59TPH-3/2/504041c040df249bbb22020b4a7a3fe1}
}

@Article{Cui2003,
  Title                    = {Large-eddy simulation of turbulent flow in a channel with rib roughness},
  Author                   = {Cui, Jie and Patel, Virendra C. and Lin, Ching-Long},
  Journal                  = {International Journal of Heat and Fluid Flow},
  Year                     = {2003},
  Number                   = {3},
  Pages                    = {372-388},
  Volume                   = {24},

  Abstract                 = {Turbulent flow in a channel with transverse rib roughness on one wall is investigated by large-eddy simulation (LES). The spacing of the roughness elements is varied to reproduce the so-called d- and k-type roughness, and an intermediate roughness between the two. The time-mean and instantaneous flows are analyzed. The LES results agree with the rather limited laboratory observations in flows with rib roughness, and provide new insights into the effects of roughness on the mean flow as well as the turbulence structure. In d-type roughness, the outer flow almost rides over the roughness layer, in the mean, with the separation eddies confined to the gaps between the ribs. For intermediate roughness, between the d-type and k-type, the mean separation region is about the same size as the cavity between the ribs, but the outer flow is affected by large turbulent eddies emanating from the roughness layer. For k-type roughness, separation and reattachment occur between two adjoining ribs, much larger and more frequent eddies are thrown into the outer flow, resulting in strong interaction between the roughness layer and the outer flow. Time and space averaged velocity profiles show the well-known downward shift of the semi-logarithmic portion of the law of the wall. This result is quite surprising in view of the grossly varying spatial structure of the instantaneous flow.},
  ISSN                     = {0142-727X},
  Keywords                 = {Channel flow
Turbulence
LES
Surface roughness},
  Type                     = {Journal Article},
  Url                      = {http://www.sciencedirect.com/science/article/B6V3G-4817M0F-1/2/cbd081784bceb5d6c346cf3002bbdbff}
}

@Article{Cunningham1910,
  Title                    = {On the velocity of steady fall of spherical particles through fluid medium},
  Author                   = {Cunningham, E.},
  Journal                  = {Proc. Roy. Soc. A },
  Year                     = {1910},
  Volume                   = {83},

  Type                     = {Journal Article}
}

@Article{Dai2006,
  Title                    = {In vivo measurements of inhalability of ultralarge aerosol particles in calm air by humans},
  Author                   = {Dai, Y.T. and Juang, Y.J. and WU, Y.Y. and Breysse, P.N. and Hsu, D.J },
  Journal                  = {Journal of Aerosol Science},
  Year                     = {2006},
  Number                   = {8},
  Pages                    = {967-973},
  Volume                   = {37},

  Type                     = {Journal Article}
}

@Article{Daigle2003,
  Title                    = {Ultrafine particle deposition in humans during rest and exercise},
  Author                   = {Daigle, C.C. and Chalupa, D.C. and Gibb, F.R. and Morrow, P.E. and Oberdörster, G. and Utell, M.J. and Frampton, M.W.},
  Journal                  = {Inhalation Toxicology},
  Year                     = {2003},
  Number                   = {6},
  Pages                    = {539-552},
  Volume                   = {15},

  Type                     = {Journal Article}
}

@Article{Dailey2007,
  Title                    = {Fluid-structure analysis of microparticle transport in deformable pulmonary alveoli},
  Author                   = {Dailey, H. L. and Ghadiali, S. N.},
  Journal                  = {Journal of Aerosol Science},
  Year                     = {2007},
  Number                   = {3},
  Pages                    = {269-288},
  Volume                   = {38},

  Doi                      = {10.1016/j.jaerosci.2007.01.001},
  ISSN                     = {0021-8502},
  Keywords                 = {Brownian diffusion
Deep-lung deformation
Alveolar mechanics
Tissue mechanics
Biofluid mechanics
Microgravity
Drug delivery
Viscoelasticity
Fluid-structure interactions
ADINA},
  Type                     = {Journal Article},
  Url                      = {http://www.sciencedirect.com/science/article/pii/S0021850207000183}
}

@Article{Dalby2003,
  Title                    = {Inhalation therapy: technological milestones in asthma treatment},
  Author                   = {Dalby, Richard and Suman, Julie},
  Journal                  = {Advanced Drug Delivery Reviews},
  Year                     = {2003},
  Note                     = {doi: DOI: 10.1016/S0169-409X(03)00077-2},
  Number                   = {7},
  Pages                    = {779-791},
  Volume                   = {55},

  ISSN                     = {0169-409X},
  Keywords                 = {Aerosol technology
Nebulizers
Metered dose inhalers
Dry powder inhalers
Spacer},
  Type                     = {Journal Article},
  Url                      = {http://www.sciencedirect.com/science/article/B6T3R-48PDF92-1/2/f277f62cc6250cf051cedb76425fdd4f}
}

@Article{Dandy1990,
  Title                    = {Sphere in shear flow at finite Reynolds number. Effect of shear on particle lift, drag, and heat transfer},
  Author                   = {Dandy, David S. and Dwyer, Harry A.},
  Journal                  = {Journal of Fluid Mechanics},
  Year                     = {1990},
  Note                     = {Cited By (since 1996): 95
Export Date: 3 June 2011
Source: Scopus},
  Pages                    = {381-410},
  Volume                   = {216},

  Type                     = {Journal Article},
  Url                      = {http://www.scopus.com/inward/record.url?eid=2-s2.0-0025461731&partnerID=40&md5=a1afe7926f012822af20d54795d2bb42}
}

@Article{Darquenne2002,
  Title                    = {Heterogeneity of aerosol deposition in a two-dimensional model of human alveolated ducts},
  Author                   = {Darquenne, Chantal},
  Journal                  = {Journal of Aerosol Science},
  Year                     = {2002},
  Number                   = {9},
  Pages                    = {1261-1278},
  Volume                   = {33},

  Abstract                 = {Transport of 0.5-5 [mu]m diameter particles was simulated within a symmetric two-dimensional (2D) six-generation structure representative of the human acinus. Aerosol boluses were introduced at the beginning of a breathing cycle (2-s inspiration, 2-s expiration). Airflow corresponded to a flow rate at the mouth of 500 ml/s. Deposition increased from 1% to 100% for 0.5-5 [mu]m diameter particles. Deposition agreed with the upper limit of alveolar deposition predicted by the one-dimensional (1D) ICRP66 model but only after accounting for multibreath. The 2D model showed that, for each particle size and structure orientation, there was a large heterogeneity in deposition among ducts. Differences of more than one order of magnitude were found between deposition in a single duct of a given generation and the average deposition in that generation. The presence of "hot spots" may have important implications in human health as several studies suggest a strong correlation between airborne pollutants and the onset of pulmonary or cardiovascular diseases.},
  ISSN                     = {0021-8502},
  Keywords                 = {Computational fluid dynamics
Gravitational sedimentation
Acinus},
  Type                     = {Journal Article},
  Url                      = {http://www.sciencedirect.com/science/article/B6V6B-4603N3S-1/2/95c922f7a4ba47560a55858978e5951b}
}

@Article{Darquenne2001,
  Title                    = {A realistic two-dimensional model of aerosol transport and deposition in the alveolar zone of the human lung},
  Author                   = {Darquenne, Chantal},
  Journal                  = {Journal of Aerosol Science},
  Year                     = {2001},
  Number                   = {10},
  Pages                    = {1161-1174},
  Volume                   = {32},

  Abstract                 = {A realistic two-dimensional model of aerosol transport and deposition in the alveolar zone of the human lung was developed, using the FIDAP software package (Fluent, Inc). Gas flow corresponding to a mouth flow rate of 500 mls-1 was computed in a symmetric 6-generation structure of alveolated ducts located in a vertical plane. A bolus of 2 [mu]m-diameter particles was introduced in the model at the beginning of inspiration. Particle trajectories were predicted for one breath cycle (2 s inspiration, 2 s expiration). There were large non-uniformities in deposition between generations, between ducts of a given generation and within each alveolated duct, suggesting that local aerosol concentrations can be much larger than the mean acinar concentration. A significant number of particles failed to exit the structure during expiration. Most of these particles were located deep in the structure (distal to the 3rd generation), leaving them in a position to penetrate deeper in the lung during the subsequent inspiration and eventually deposit.},
  ISSN                     = {0021-8502},
  Keywords                 = {Aerosol deposition
Human lung
Modeling
Gravitational sedimentation},
  Type                     = {Journal Article},
  Url                      = {http://www.sciencedirect.com/science/article/B6V6B-43MJ93P-3/2/fe2757b807465fe3e987d22106877aab}
}

@Article{Darquenne2006,
  Title                    = {The use of aerosols to study convective mixing in the lung},
  Author                   = {Darquenne, C. and Prisk, G. K.},
  Journal                  = {Journal of Biomechanics},
  Year                     = {2006},
  Number                   = {Supplement 1},
  Pages                    = {S265-S265},
  Volume                   = {39},

  ISSN                     = {0021-9290},
  Type                     = {Journal Article},
  Url                      = {http://www.sciencedirect.com/science/article/B6T82-4KR88PB-1DY/2/62c3c4d030e833249fd47dbd199633de}
}

@Article{Darquenne2003,
  Title                    = {Effect of gravitational sedimentation on simulated aerosol dispersion in the human acinus},
  Author                   = {Darquenne, Chantal and Prisk, G. Kim},
  Journal                  = {Journal of Aerosol Science},
  Year                     = {2003},
  Number                   = {4},
  Pages                    = {405-418},
  Volume                   = {34},

  Abstract                 = {We studied the effect of gravitational sedimentation on the dispersion of 0.5 and 1 [mu]m-diameter particle boluses within a two-dimensional symmetric six-generation model of the human acinus. Boluses were introduced at the beginning of a 2-s inspiration immediately followed by a 4-s expiration, in normal gravity (1G) and in the absence of gravity (0G). The flow corresponded to a flow rate at the mouth of 500 ml/s. In 0G, simulated dispersion (Hsim) was 16 ml for both particle sizes. In 1G, Hsim was 71 and 242 ml for 0.5 and 1 [mu]m-diameter particles, respectively, showing the effect of gravitational sedimentation. The difference between experimental data (J. Appl. Physiol. 86 (1999) 1402) and simulations was independent of particle size. This suggests that the residual dispersion was independent of the intrinsic properties of the particles and was more likely due to other mechanisms such as ventilation inhomogeneities, cardiogenic oscillations and alveolar wall motion.},
  ISSN                     = {0021-8502},
  Keywords                 = {Computational fluid dynamics
Aerosol bolus
Human lung},
  Type                     = {Journal Article},
  Url                      = {http://www.sciencedirect.com/science/article/B6V6B-481MSJ3-1/2/3be61732c5ccb30f59b20d44bf944219}
}

@Article{Dastan2014,
  Title                    = {CFD simulation of total and regional fiber deposition in human nasal cavities},
  Author                   = {Dastan, Alireza and Abouali, Omid and Ahmadi, Goodarz},
  Journal                  = {Journal of Aerosol Science},
  Year                     = {2014},
  Note                     = {C:\Users\sean\AppData\Roaming\Zotero\Zotero\Profiles\16a4oype.default\zotero\storage\KKSG2MWT\Dastan et al. - 2014 - CFD simulation of total and regional fiber deposit.pdf},
  Pages                    = {132-149},
  Volume                   = {69},

  Abstract                 = {In this study, CFD simulations of fibrous particle deposition in different realistic human nasal cavities were performed. The airflow field in the cavity was evaluated by solving the Navier–Stokes and continuity equations using commercial software, while a Lagrangian trajectory analysis approach for solving the coupled translational and rotational equations of motion of ellipsoids was developed and used to investigate fiber transport and deposition in the nasal passages. Different breathing rates in the laminar flow regime in the nose and a range of fiber lengths and diameters were used in these simulations. It was shown that the aerodynamic diameter based on the Stokes equivalent diameter is an appropriate parameter for correlating the fiber deposition rate. Presenting the deposition fraction results versus the Stokes-based and pressure-based impaction parameters collapsed the results of different cases for various nose models roughly to a single curve. The simulated regional fiber deposition results were also presented for different nasal cavities. A simple approach developed earlier for modeling non-spherical particles using the shape factor in the drag force was also studied, and the resulting deposition fraction was compared with the present coupled translational–rotational trajectory analysis approach.},
  Doi                      = {10.1016/j.jaerosci.2013.12.008},
  ISSN                     = {0021-8502},
  Keywords                 = {Deposition fraction
Ellipsoidal fiber
Human nasal cavity
Numerical simulation
Regional deposition},
  Type                     = {Journal Article},
  Url                      = {http://ac.els-cdn.com/S0021850213002541/1-s2.0-S0021850213002541-main.pdf?_tid=1eb53fb8-4220-11e4-82ca-00000aacb35d&acdnat=1411366825_6662befebc76bc95644e2aa4f33efc2d}
}

@Article{Davidson2003,
  Title                    = {Modification of the V2F model for computing the flow in a 3d wall jet},
  Author                   = {Davidson, L. and Nielsen, P.V. and Sveningsson, A.},
  Journal                  = {Turbulence Heat and Mass Transfer},
  Year                     = {2003},
  Pages                    = {577-84},
  Volume                   = {4},

  Type                     = {Journal Article}
}

@Book{Davies1966,
  Title                    = {Aerosol Science},
  Author                   = {Davies, C. N. },
  Publisher                = {Academic Press},
  Year                     = {1966},

  Address                  = {London.},

  Type                     = {Book}
}

@Article{Davies1995,
  Title                    = {Flow-Mediated Endothelial Mechanotransduction},
  Author                   = {Davies, P.F. },
  Journal                  = {Physiology Review},
  Year                     = {1995},
  Pages                    = {519-560},
  Volume                   = {75},

  Type                     = {Journal Article}
}

@Book{Davies1988,
  Title                    = {Endothelial Cells, Hemodynamic Stress, and the Localization of Atherosclerosis},
  Author                   = {Davies, P.F. },
  Publisher                = {CRC Press},
  Year                     = {1988},

  Address                  = {Boca Raton},

  Type                     = {Book}
}

@Article{Davis2008,
  Title                    = {Nanoparticle therapeutics: an emerging treatment modality for cancer},
  Author                   = {Davis, Mark E. and Chen, Zhuo and Shin, Dong M.},
  Journal                  = {Nat Rev Drug Discov},
  Year                     = {2008},
  Note                     = {(Georgia)
10.1038/nrd2614},
  Number                   = {9},
  Pages                    = {771-782},
  Volume                   = {7},

  ISSN                     = {1474-1776},
  Type                     = {Journal Article},
  Url                      = {http://dx.doi.org/10.1038/nrd2614}
}

@Article{Davis2003,
  Title                    = {Absorption Enhancers for Nasal Drug Delivery},
  Author                   = {Davis, S.S. and Illum, L.},
  Journal                  = {Clinical Pharmacokinetics},
  Year                     = {2003},
  Pages                    = {1107-1128},
  Volume                   = {42},

  Type                     = {Journal Article},
  Url                      = {http://www.ingentaconnect.com/content/adis/cpk/2003/00000042/00000013/art00003}
}

@Article{Dayal2004,
  Title                    = {Evaluation of different parameters that affect droplet-size distribution from nasal sprays using the Malvern Spraytec},
  Author                   = {Dayal, P. and Shaik, M.S. and Singh, M.},
  Journal                  = {Journal of Pharmaceutical Sciences},
  Year                     = {2004},
  Note                     = {10.1002/jps.20090},
  Number                   = {7},
  Pages                    = {1725-1742},
  Volume                   = {93},

  ISSN                     = {1520-6017},
  Type                     = {Journal Article},
  Url                      = {http://dx.doi.org/10.1002/jps.20090}
}

@Article{DeBacker2007,
  Title                    = {Functional imaging using computational fluid dynamics to predict treatment success of mandibular advancement devices in sleep-disordered breathing},
  Author                   = {De Backer, J. W. and Vanderveken, O. M. and Vos, W. G. and Devolder, A. and Verhulst, S. L. and Verbraecken, J. A. and Parizel, P. M. and Braem, M. J. and Van de Heyning, P. H. and De Backer, W. A.},
  Journal                  = {Journal of Biomechanics},
  Year                     = {2007},
  Number                   = {16},
  Pages                    = {3708-3714},
  Volume                   = {40},

  Abstract                 = {Mandibular advancement devices (MADs) have emerged as a popular alternative for the treatment of sleep-disordered breathing. These devices bring the mandibula forward in order to increase upper airway (UA) volume and prevent total UA collapse during sleep. However, the precise mechanism of action appears to be quite complex and is not yet completely understood; this might explain interindividual variation in treatment success. We examined whether an UA model, that combines imaging techniques and computational fluid dynamics (CFD), allows for a prediction of the treatment outcome with MADs. Ten patients that were treated with a custom-made mandibular advancement device (MAD), underwent split-night polysomnography. The morning after the sleep study, a low radiation dose CT scan was scheduled with and without the MAD. The CT examinations allowed for a comparison between the change in UA volume and the anatomical characteristics through the conversion to three-dimensional computer models. Furthermore, the change in UA resistance could be calculated through flow simulations with CFD. Boundary conditions for the model such as mass flow rate and pressure distributions were obtained during the split-night polysomnography. Therefore, the flow modeling was based on a patient specific geometry and patient specific boundary conditions. The results indicated that a decrease in UA resistance and an increase in UA volume correlate with both a clinical and an objective improvement. The results of this pilot study suggest that the outcome of MAD treatment can be predicted using the described UA model.},
  ISSN                     = {0021-9290},
  Keywords                 = {Imaging
Oral appliances
Sleep apnea
Snoring
Upper airway
Computational fluid dynamics},
  Type                     = {Journal Article},
  Url                      = {http://www.sciencedirect.com/science/article/B6T82-4P9KD0W-1/2/fb7300aa5d8b52dce160ec41df470bb3}
}

@Article{DeBacker2008,
  Title                    = {Computational fluid dynamics can detect changes in airway resistance in asthmatics after acute bronchodilation},
  Author                   = {De Backer, J. W. and Vos, W. G. and Devolder, A. and Verhulst, S. L. and Germonpré, P. and Wuyts, F. L. and Parizel, P. M. and De Backer, W.},
  Journal                  = {Journal of Biomechanics},
  Year                     = {2008},
  Note                     = {doi: DOI: 10.1016/j.jbiomech.2007.07.009},
  Number                   = {1},
  Pages                    = {106-113},
  Volume                   = {41},

  ISSN                     = {0021-9290},
  Keywords                 = {Functional imaging
Computational fluid dynamics (CFD)
Asthma
Bronchodilation},
  Type                     = {Journal Article},
  Url                      = {http://www.sciencedirect.com/science/article/B6T82-4PDT00V-1/2/b0e02b0011c21019bdc84e8dbf431708}
}

@Article{DeBacker2008a,
  Title                    = {Flow analyses in the lower airways: Patient-specific model and boundary conditions},
  Author                   = {De Backer, J. W. and Vos, W. G. and Gorlé, C. D. and Germonpré, P. and Partoens, B. and F.L.Wuyts and Parizel, P. M. and De Backer, W.},
  Journal                  = {Medical Engineering \& Physics},
  Year                     = {2008},
  Note                     = {doi: DOI: 10.1016/j.medengphy.2007.11.002},
  Number                   = {7},
  Pages                    = {872-879},
  Volume                   = {30},

  ISSN                     = {1350-4533},
  Keywords                 = {Airway modeling
Computational fluid dynamics
Boundary conditions
Patient specific
Functional imaging},
  Type                     = {Journal Article},
  Url                      = {http://www.sciencedirect.com/science/article/B6T9K-4RDBFCG-1/2/2d2d871e6ed7348d8d393b85dd80028c}
}

@Article{DeBacker2008b,
  Title                    = {Novel imaging techniques using computer methods for the evaluation of the upper airway in patients with sleep-disordered breathing: A comprehensive review},
  Author                   = {De Backer, Jan W. and Vos, Wim G. and Verhulst, Stijn L. and De Backer, Wilfried},
  Journal                  = {Sleep Medicine Reviews},
  Year                     = {2008},
  Number                   = {6},
  Pages                    = {437-447},
  Volume                   = {12},

  ISSN                     = {1087-0792},
  Keywords                 = {Obstructive sleep apnea
Sleep related breathing disorders
Upper airway imaging
Computational fluid dynamics
Mandibular advancement devices
UPPP},
  Type                     = {Journal Article},
  Url                      = {http://www.sciencedirect.com/science/article/B6WX7-4TP1FGD-1/2/4f89eb924b14deaee0f8ecddf5415a76}
}

@Article{DeBoeck1999,
  Title                    = {Is the correct use of a dry powder inhaler (Turbohaler) age dependent?},
  Author                   = {De Boeck, Kris and Alifier, Marek and Warnier, Gerd},
  Journal                  = {Journal of Allergy and Clinical Immunology},
  Year                     = {1999},
  Number                   = {5},
  Pages                    = {763-767},
  Volume                   = {103},

  ISSN                     = {0091-6749},
  Keywords                 = {Asthma
dry powder inhaler
children
Turbohaler},
  Type                     = {Journal Article},
  Url                      = {http://www.sciencedirect.com/science/article/B6WH4-4HGRV9X-7/2/35d099a2002f5c53307f31d7ced5c353}
}

@Article{DeJong2008,
  Title                    = {Drug delivery and nanoparticles: Applications and hazards},
  Author                   = {De Jong, W.H and Norm, P.JA},
  Journal                  = {International Journal of Nanomedicine},
  Year                     = {2008},
  Number                   = {2},
  Pages                    = {133-149},
  Volume                   = {3},

  Type                     = {Journal Article}
}

@Article{DeKeulenaer1998,
  Title                    = {Oscillatory and steady laminar shear stress differentially affect human endothelial redox state: Role of a superoxide-producing NADH oxidase},
  Author                   = {De Keulenaer, G. W. and Chappell, D. C. and Ishizaka, N. and Nerem, R. M. and Wayne Alexander, R. and Griendling, K. K.},
  Journal                  = {Circulation Research},
  Year                     = {1998},
  Note                     = {Cited By (since 1996): 333
Export Date: 3 June 2011
Source: Scopus},
  Number                   = {10},
  Pages                    = {1094-1101},
  Volume                   = {82},

  Type                     = {Journal Article},
  Url                      = {http://www.scopus.com/inward/record.url?eid=2-s2.0-0032104153&partnerID=40&md5=86d010c52c815e9dada8187d50e24be8}
}

@Article{Deevey2007,
  Title                    = {Modelling the effect of an occupant on displacement ventilation with computational fluid dynamics.},
  Author                   = {Deevey, M. and Sinai, Y. and Everitt, P. and Voigt, L. and Gobeau, N.},
  Journal                  = {Energy and Buildings},
  Year                     = {2007},
  Pages                    = {doi:10.1016/j.enbuild.2007.02.021},

  Type                     = {Journal Article}
}

@Article{DeHaan2004,
  Title                    = {Predicting extrathoracic deposition from dry powder inhalers},
  Author                   = {DeHaan, W. H. and Finlay, W. H.},
  Journal                  = {Journal of Aerosol Science},
  Year                     = {2004},
  Number                   = {3},
  Pages                    = {309-331},
  Volume                   = {35},

  Abstract                 = {The deposition of monodisperse aerosols entering an idealized oral cavity geometry through a variety of inlets was experimentally measured. Aerosol particles with diameters of 2.5, 3.8 and 5.0 [mu]m were investigated at flow rates ranging from 15 to 90 L/min. The tested inlets ranged in diameter from 3 to 17 mm and included contraction nozzles, straight tubes, a turbulence generator and six commercially available dry powder inhalers (DPIs). A model for predicting the oral cavity deposition was derived from the data based on the particle Stokes number near the primary impaction location modified to incorporate the turbulent kinetic energy at the inlet. The model predicted similar (but slightly underestimated) deposition for monodisperse aerosols entering through DPIs, with increasing deposition for decreasing inlet diameter. The model was then extended to predict extrathoracic deposition for polydisperse aerosol formulations in vivo. Improved agreement was found between the in vitro predictions and the in vivo measurements compared to previous attempts.},
  ISSN                     = {0021-8502},
  Keywords                 = {In vitro
Oral cavity
Mouth
Stokes number
Impaction
Dry powder inhaler},
  Type                     = {Journal Article},
  Url                      = {http://www.sciencedirect.com/science/article/B6V6B-49Y3X5X-1/2/ca612af294f610bdd169e8b8e34e8d7c}
}

@Article{Dehbi2008,
  Title                    = {A CFD model for particle dispersion in turbulent boundary layer flows},
  Author                   = {Dehbi, A.},
  Journal                  = {Nuclear Engineering and Design},
  Year                     = {2008},
  Number                   = {3},
  Pages                    = {707-715},
  Volume                   = {238},

  ISSN                     = {0029-5493},
  Type                     = {Journal Article},
  Url                      = {http://www.sciencedirect.com/science/article/B6V4D-4P0N2BW-4/2/4e1ac139cb59e41d2b407bf989c08a60}
}

@Article{Dehbi2008a,
  Title                    = {Turbulent particle dispersion in arbitrary wall-bounded geometries: A coupled CFD-Langevin-equation based approach},
  Author                   = {Dehbi, A.},
  Journal                  = {Int J Multiphase Flow},
  Year                     = {2008},
  Number                   = {9},
  Pages                    = {819-828},
  Volume                   = {34},

  Type                     = {Journal Article}
}

@Article{Demers1998,
  Title                    = {Cancer risk from occupational exposure to wood dust: a pooled analysis of epidemiological studies.},
  Author                   = {Demers, P. and Boffette, P.},
  Journal                  = { IARC Technical Report No. 30.},
  Year                     = {1998},
  Volume                   = {30},

  Type                     = {Journal Article}
}

@Book{Demirkaya2009,
  Title                    = {Image processing with Matlab - Application in medicine and biology},
  Author                   = {Demirkaya, O., Asyali, M. H., Sahoo, P. K.},
  Publisher                = {CRC Press, Taylor Francis Group},
  Year                     = {2009},

  Address                  = {Broken Sound Parkway, NW},

  Type                     = {Book}
}

@Article{Denison,
  Title                    = {Forensic implications of respiratory derived blood spatter distributions},
  Author                   = {Denison, David and Porter, Alice and Mills, Matthew and Schroter, Robert C.},
  Journal                  = {Forensic Science International},
  Volume                   = {In Press, Corrected Proof},

  Abstract                 = {The nature of blood aerosols produced in physiological studies of an upright subject expiring small volumes through straws, spitting and mouthing sounds, and a semi-prone subject spitting through a bloody mouth or snorting through a single nasal orifice and by a simplified physical model of the respiratory system were investigated. Each manoeuvre produced many hundreds of droplets of a range of size, the vast majority being less than 1 mm diameter. Droplets under 1 mm dia. travelled over 1 m - much further than could be expected if their flight was ballistic, like that of impact spatter. Respired blood aerosol properties are explained in terms of established mechanics of airflow shear induced aerosol production and the fluid mechanics of exhaled air movement.},
  ISSN                     = {0379-0738},
  Keywords                 = {Respired blood spatter
Respirated blood droplet size
Blood pattern analysis
Exhaled blood aerosol transport
Chest stab wounds},
  Type                     = {Journal Article},
  Url                      = {http://www.sciencedirect.com/science/article/B6T6W-50BBSW6-1/2/2c50fb9948cec056f30daf13d019629d}
}

@PhdThesis{Deo2005,
  Title                    = {Experimental Investigations of the Influence of Reynolds Number and Boundary Conditions on a Plane Air Jet},
  Author                   = {Deo, R.C.},
  Year                     = {2005},
  Type                     = {Thesis},

  University               = {University of Adelaide, South Australia}
}

@Article{Derjaguin1975,
  Title                    = {Effect of contact deformations on the adhesion of particles},
  Author                   = {Derjaguin, B. V. and Muller, V. M. and Toporov, Yu P.},
  Journal                  = {Journal of Colloid And Interface Science},
  Year                     = {1975},
  Number                   = {2},
  Pages                    = {314-326},
  Volume                   = {53},

  ISSN                     = {0021-9797},
  Type                     = {Journal Article},
  Url                      = {http://www.sciencedirect.com/science/article/pii/0021979775900181}
}

@Article{Dewey1981,
  Title                    = {The Dynamic Response of Vascular Endothelial Cells to Fluid Shear Stress},
  Author                   = {Dewey, Jr C. F. and Bussolari, S. R. and Gimbrone, Jr M. A. and Davies, P. F.},
  Journal                  = {Journal of Biomechanical Engineering},
  Year                     = {1981},
  Number                   = {3},
  Pages                    = {177-185},
  Volume                   = {103},

  Type                     = {Journal Article},
  Url                      = {http://link.aip.org/link/?JBY/103/177/1}
}

@Article{Dilworth2000,
  Title                    = {Wood Dust Survey 1999/2000 Final Report},
  Author                   = {Dilworth, M.},
  Journal                  = {Health and Safety Laboratory, Crown},
  Year                     = {2000},

  Type                     = {Journal Article}
}

@Article{Ding2006,
  Title                    = {Cluster Size Distribution and Scaling for Spherical Particles and Red Blood Cells in Pressure-Driven Flows at Small Reynolds Number},
  Author                   = {Ding, E. Jiang and Aidun, Cyrus K.},
  Journal                  = {Physical Review Letters},
  Year                     = {2006},
  Note                     = {Copyright (C) 2011 The American Physical Society
Please report any problems to prola@aps.org
PRL},
  Number                   = {20},
  Pages                    = {204502},
  Volume                   = {96},

  Type                     = {Journal Article},
  Url                      = {http://link.aps.org/doi/10.1103/PhysRevLett.96.204502}
}

@Article{Ding2003,
  Title                    = {Extension of the Lattice-Boltzmann Method for Direct Simulation of Suspended Particles Near Contact},
  Author                   = {Ding, E. Jiang and Aidun, Cyrus K.},
  Journal                  = {Journal of Statistical Physics},
  Year                     = {2003},
  Number                   = {3},
  Pages                    = {685-708},
  Volume                   = {112},

  Doi                      = {10.1023/a:1023880126272},
  ISSN                     = {0022-4715},
  Keywords                 = {Physics and Astronomy},
  Type                     = {Journal Article},
  Url                      = {http://dx.doi.org/10.1023/A:1023880126272}
}

@Article{Djupesland2003,
  Title                    = {Bi-directional nasal delivery of aerosols can prevent lung deposition},
  Author                   = {Djupesland, P.G. and Sketting, A. and Winderen, M. and Holand, T.},
  Journal                  = {Journal of Aerosol Medicine},
  Year                     = {2003},
  Number                   = {3},
  Pages                    = {249-259},
  Volume                   = {17},

  Type                     = {Journal Article}
}

@Article{Dombrowski1963,
  Title                    = {The aerodynamic instability and disintegration of viscous liquid sheets},
  Author                   = {Dombrowski, N. and Johns, W. R.},
  Journal                  = {Chemical Engineering Science},
  Year                     = {1963},
  Number                   = {3},
  Pages                    = {203-214},
  Volume                   = {18},

  Doi                      = {10.1016/0009-2509(63)85005-8},
  ISSN                     = {0009-2509},
  Type                     = {Journal Article},
  Url                      = {http://www.sciencedirect.com/science/article/pii/0009250963850058}
}

@Article{Donaldson2006,
  Title                    = {Carbon Nanotubes: A Review of Their Properties in Relation to Pulmonary Toxicology and Workplace Safety},
  Author                   = {Donaldson, Ken and Aitken, Robert and Tran, Lang and Stone, Vicki and Duffin, Rodger and Forrest, Gavin and Alexander, Andrew},
  Journal                  = {Toxicol. Sci.},
  Year                     = {2006},
  Number                   = {1},
  Pages                    = {5-22},
  Volume                   = {92},

  Abstract                 = {Carbon nanotubes (CNT) are an important new class of technological materials that have numerous novel and useful properties. The forecast increase in manufacture makes it likely that increasing human exposure will occur, and as a result, CNT are beginning to come under toxicological scrutiny. This review seeks to set out the toxicological paradigms applicable to the toxicity of inhaled CNT, building on the toxicological database on nanoparticles (NP) and fibers. Relevant workplace regulation regarding exposure is also considered in the light of our knowledge of CNT. CNT could have features of both NP and conventional fibers, and so the current paradigm for fiber toxicology, which is based on mineral fibers and synthetic vitreous fibers, is discussed. The NP toxicology paradigm is also discussed in relation to CNT. The available peer-reviewed literature suggests that CNT may have unusual toxicity properties. In particular, CNT seem to have a special ability to stimulate mesenchymal cell growth and to cause granuloma formation and fibrogenesis. In several studies, CNT have more adverse effects than the same mass of NP carbon and quartz, the latter a commonly used benchmark of particle toxicity. There is, however, no definitive inhalation study available that would avoid the potential for artifactual effects due to large mats and aggregates forming during instillation exposure procedures. Studies also show that CNT may exhibit some of their effects through oxidative stress and inflammation. CNT represent a group of particles that are growing in production and use, and therefore, research into their toxicology and safe use is warranted.},
  Doi                      = {10.1093/toxsci/kfj130},
  Type                     = {Journal Article},
  Url                      = {http://toxsci.oxfordjournals.org/cgi/content/abstract/92/1/5}
}

@Article{Donaldson2012,
  Title                    = {a short history of the toxicology of inhaled particles},
  Author                   = {Donaldson, K and Anthony, S},
  Journal                  = {particle and fibre toxicology},
  Year                     = {2012},
  Number                   = {13},
  Pages                    = {12},
  Volume                   = {9},

  Type                     = {Journal Article}
}

@Article{Dondeti1995,
  Title                    = {In vivo evaluation of spray formulations of human insulin for nasal delivery},
  Author                   = {Dondeti, Polireddy and Zia, Hossein and Needham, Thomas E.},
  Journal                  = {International Journal of Pharmaceutics},
  Year                     = {1995},
  Number                   = {1–2},
  Pages                    = {91-105},
  Volume                   = {122},

  Doi                      = {http://dx.doi.org/10.1016/0378-5173(95)00045-K},
  ISSN                     = {0378-5173},
  Keywords                 = {Insulin, human
Nasal administration
Spray formulation
Bioadhesive polymer
Microcrystalline cellulose
Plastoid
Ammonium glycyrrhizinate
Glcyrrhetinic acid
Sodium taurocholate},
  Type                     = {Journal Article},
  Url                      = {http://www.sciencedirect.com/science/article/pii/037851739500045K}
}

@Article{Dong2014,
  Title                    = {Fluid–structure interaction analysis of the left coronary artery with variable angulation},
  Author                   = {Dong, Jingliang and Sun, Zhonghua and Inthavong, Kiao and Tu, Jiyuan},
  Journal                  = {Computer Methods in Biomechanics and Biomedical Engineering},
  Year                     = {2014},
  Pages                    = {1-9},

  Doi                      = {10.1080/10255842.2014.921682},
  ISSN                     = {1025-5842},
  Type                     = {Journal Article},
  Url                      = {http://dx.doi.org/10.1080/10255842.2014.921682}
}

@Article{Doorly2008,
  Title                    = {Experimental investigation of nasal flow},
  Author                   = {Doorly, D. and Taylor, D.J. and Franke, P. and Schroter, R.C.},
  Journal                  = {Proc. of the Institution of Mechanical Engineers Part H - J Engineering in Medicine},
  Year                     = {2008},
  Number                   = {H4},
  Pages                    = {439-453},
  Volume                   = {222},

  Type                     = {Journal Article}
}

@Article{Doorly2008a,
  Title                    = {Nasal architecture: form and flow},
  Author                   = {Doorly, D.J and Taylor, D.J and Gambaruto, A.M and Schroter, R.C and Tolley, N},
  Journal                  = {Philosophical Transactions of the Royal Society A: Mathematical, Physical and Engineering Sciences},
  Year                     = {2008},
  Number                   = {1879},
  Pages                    = {3225-3246},
  Volume                   = {366},

  Abstract                 = {Current approaches to model nasal airflow are reviewed in this study, and new findings presented. These new results make use of improvements to computational and experimental techniques and resources, which now allow key dynamical features to be investigated, and offer rational procedures to relate variations in anatomical form. Specifically, both replica and simplified airways of a single subject were investigated and compared with the replica airways of two other individuals with overtly differing geometries. Procedures to characterize and compare complex nasal airway geometry are first outlined. It is then shown that coupled computational and experimental studies, capable of obtaining highly resolved data, reveal internal flow structures in both intrinsically steady and unsteady situations. The results presented demonstrate that the intimate relation between nasal form and flow can be explored in greater detail than hitherto possible. By outlining means to compare complex airway geometries and demonstrating the effects of rational geometric simplification on the flow structure, this work offers a fresh approach to studies of how natural conduits guide and control flow. The concepts and tools address issues that are thus generic to flow studies in other physiological systems.},
  Doi                      = {10.1098/rsta.2008.0083},
  Type                     = {Journal Article},
  Url                      = {http://rsta.royalsocietypublishing.org/content/366/1879/3225.abstract}
}

@Article{Doorly2006,
  Title                    = {Nasal airflow: computational and experimental modelling},
  Author                   = {Doorly, D. J. and Franke, V. and Gambaruto, A. and Taylor, D. J. and Schroter, R. C.},
  Journal                  = {Journal of Biomechanics},
  Year                     = {2006},
  Number                   = {Supplement 1},
  Pages                    = {S270-S270},
  Volume                   = {39},

  ISSN                     = {0021-9290},
  Type                     = {Journal Article},
  Url                      = {http://www.sciencedirect.com/science/article/B6T82-4KR88PB-1FW/2/6512f9e86f220c400d8161959afea3b2}
}

@Article{Doorly2008b,
  Title                    = {Nasal architecture: form and flow},
  Author                   = {Doorly, D. J. and Taylor, D. J. and Gambaruto, A. M. and Schroter, R. C. and Tolley, N.},
  Journal                  = {Philosophical Transactions of the Royal Society A: Mathematical, Physical and Engineering Sciences},
  Year                     = {2008},
  Note                     = {C:\Users\sean\AppData\Roaming\Zotero\Zotero\Profiles\16a4oype.default\zotero\storage\VGGMZ2PW\Doorly et al. - 2008 - Nasal architecture form and flow.pdf
C:\Users\sean\AppData\Roaming\Zotero\Zotero\Profiles\16a4oype.default\zotero\storage\T3J5M5C7\Doorly et al. - 2008 - Nasal architecture form and flow.pdf
C:\Users\sean\AppData\Roaming\Zotero\Zotero\Profiles\16a4oype.default\zotero\storage\3Z6VDHHP\Doorly et al. - 2008 - Nasal architecture form and flow.pdf
C:\Users\sean\AppData\Roaming\Zotero\Zotero\Profiles\16a4oype.default\zotero\storage\FGDHE8QD\Doorly et al. - 2008 - Nasal architecture form and flow.pdf
C:\Users\sean\AppData\Roaming\Zotero\Zotero\Profiles\16a4oype.default\zotero\storage\U225NCDU\Doorly et al. - 2008 - Nasal architecture form and flow.pdf
C:\Users\sean\AppData\Roaming\Zotero\Zotero\Profiles\16a4oype.default\zotero\storage\WBXBEWW8\Doorly et al. - 2008 - Nasal architecture form and flow.pdf},
  Pages                    = {3225-3246},
  Volume                   = {366},

  Abstract                 = {Current approaches to model nasal airflow are reviewed in this study, and new findings presented. These new results make use of improvements to computational and experimental techniques and resources, which now allow key dynamical features to be investigated, and offer rational procedures to relate variations in anatomical form. Specifically, both replica and simplified airways of a single subject were investigated and compared with the replica airways of two other individuals with overtly differing geometries. Procedures to characterize and compare complex nasal airway geometry are first outlined. It is then shown that coupled computational and experimental studies, capable of obtaining highly resolved data, reveal internal flow structures in both intrinsically steady and unsteady situations. The results presented demonstrate that the intimate relation between nasal form and flow can be explored in greater detail than hitherto possible. By outlining means to compare complex airway geometries and demonstrating the effects of rational geometric simplification on the flow structure, this work offers a fresh approach to studies of how natural conduits guide and control flow. The concepts and tools address issues that are thus generic to flow studies in other physiological systems.},
  Doi                      = {10.1098/rsta.2008.0083},
  ISSN                     = {1364-503X, 1471-2962},
  Keywords                 = {airflow
computation
flow visualization
nasal
particle image velocimetry
shape modelling},
  Type                     = {Journal Article},
  Url                      = {http://rsta.royalsocietypublishing.org/content/366/1879/3225.full.pdf}
}

@Article{Doorly2008c,
  Title                    = {Mechanics of airflow in the human nasal airways},
  Author                   = {Doorly, D. J. and Taylor, D. J. and Schroter, R. C.},
  Journal                  = {Respiratory Physiology \& Neurobiology},
  Year                     = {2008},
  Number                   = {13},
  Pages                    = {100-110},
  Volume                   = {163},

  Abstract                 = {The mechanics of airflow in the human nasal airways is reviewed, drawing on the findings of experimental and computational model studies. Modelling inevitably requires simplifications and assumptions, particularly given the complexity of the nasal airways. The processes entailed in modelling the nasal airways (from defining the model, to its production and, finally, validating the results) is critically examined, both for physical models and for computational simulations. Uncertainty still surrounds the appropriateness of the various assumptions made in modelling, particularly with regard to the nature of flow. New results are presented in which high-speed particle image velocimetry (PIV) and direct numerical simulation are applied to investigate the development of flow instability in the nasal cavity. These illustrate some of the improved capabilities afforded by technological developments for future model studies. The need for further improvements in characterising airway geometry and flow together with promising new methods are briefly discussed.},
  Doi                      = {http://dx.doi.org/10.1016/j.resp.2008.07.027},
  ISSN                     = {1569-9048},
  Keywords                 = {Nasal airways
Computational biomechanics
Physiological flows
Particle image velocimetry},
  Type                     = {Journal Article},
  Url                      = {http://www.sciencedirect.com/science/article/pii/S1569904808002140}
}

@Article{Doty2014,
  Title                    = {The Influences of Age on Olfaction: A Review},
  Author                   = {Doty, Richard L and Kamath, Vidyulata},
  Journal                  = {Frontiers in Psychology},
  Year                     = {2014},
  Number                   = {20},
  Volume                   = {5},

  Abstract                 = {Decreased olfactory function is very common in the older population, being present in over half of those between the ages of 65 and 80 years and in over three quarters of those over the age of 80 years. Such dysfunction significantly influences physical well-being and quality of life, nutrition, the enjoyment of food, as well as everyday safety. Indeed a disproportionate number of the elderly die in accident gas poisonings each year. As described in this review, multiple factors contribute to such age-related loss, including altered nasal engorgement, increased propensity for nasal disease, cumulative damage to the olfactory epithelium from viral and other environmental insults, decrements in mucosal metabolizing enzymes, ossification of cribriform plate foramina, loss of selectivity of receptor cells to odorants, changes in neurotransmitter and neuromodulator systems, and neuronal expression of aberrant proteins associated with neurodegenerative disease. It is now well established that decreased smell loss can be an early sign of such neurodegenerative diseases as Alzheimer's disease and sporadic Parkinson's disease. In this review we provide an overview of the anatomy and physiology of the aging olfactory system, how this system is clinically evaluated, and the multiple pathophysiological factors that are associated with its dysfunction.},
  Doi                      = {10.3389/fpsyg.2014.00020},
  ISSN                     = {1664-1078},
  Url                      = {http://www.frontiersin.org/cognitive_science/10.3389/fpsyg.2014.00020/abstract}
}

@Article{Doughty2011,
  Title                    = {Automated actuation of nasal spray products: determination and comparison of adult and pediatric settings},
  Author                   = {Doughty, Diane V. and Vibbert, Carrie and Kewalramani, Anupama and Bollinger, Mary E. and Dalby, Richard N.},
  Journal                  = {Drug Development and Industrial Pharmacy},
  Year                     = {2011},
  Number                   = {3},
  Pages                    = {359-366},
  Volume                   = {37},

  Doi                      = {doi:10.3109/03639045.2010.520321},
  Type                     = {Journal Article},
  Url                      = {http://informahealthcare.com/doi/abs/10.3109/03639045.2010.520321}
}

@Article{Dreeben1997,
  Title                    = {Probability density function and Reynolds-stress modeling of near-wall turbulent flows},
  Author                   = {Dreeben, T.D. and Pope, S.B.},
  Journal                  = {Physics of Fluids},
  Year                     = {1997},
  Number                   = {1},
  Pages                    = {154-163},
  Volume                   = {9},

  Type                     = {Journal Article}
}

@Article{Drolet1998,
  Title                    = {Fluid Flow Induced by a Random Acceleration Field},
  Author                   = {Drolet, F. and Vinals, J.},
  Year                     = {1998},
  Number                   = {2},
  Pages                    = {64-68},
  Volume                   = {11},

  Type                     = {Journal Article}
}

@Article{Druzhinin2001,
  Title                    = {Direct numerical simulation of a three-dimensional spatially developing bubble-laden mixing layer with two-way coupling},
  Author                   = {Druzhinin, O. A. and Elghobashi, S. E.},
  Journal                  = {Journal of Fluid Mechanics},
  Year                     = {2001},
  Note                     = {Cited By (since 1996): 27
Export Date: 3 June 2011
Source: Scopus},
  Pages                    = {23-61},
  Volume                   = {429},

  Type                     = {Journal Article},
  Url                      = {http://www.scopus.com/inward/record.url?eid=2-s2.0-0035945914&partnerID=40&md5=495867b930f3c79d9bdd25e914000f6d}
}

@Article{Druzhinin1999,
  Title                    = {A Lagrangian-Eulerian Mapping Solver for Direct Numerical Simulation of Bubble-Laden Turbulent Shear Flows Using the Two-Fluid Formulation},
  Author                   = {Druzhinin, O. A. and Elghobashi, S. E.},
  Journal                  = {Journal of Computational Physics},
  Year                     = {1999},
  Number                   = {1},
  Pages                    = {174-196},
  Volume                   = {154},

  ISSN                     = {0021-9991},
  Type                     = {Journal Article},
  Url                      = {http://www.sciencedirect.com/science/article/pii/S0021999199963118}
}

@Article{Druzhinin1999a,
  Title                    = {On the decay rate of isotropic turbulence laden with microparticles},
  Author                   = {Druzhinin, O.A. and Elghobashi, S.E.},
  Journal                  = {Physics of Fluids},
  Year                     = {1999},
  Pages                    = {602-610},
  Volume                   = {11},

  Type                     = {Journal Article}
}

@Article{Duci2004,
  Title                    = {Numerical approach of carbon monoxide concentration dispersion in an enclosed garage},
  Author                   = {Duci, A. and Papakonstantinou, K. and Chaloulakou, A. and Markatos, N.},
  Journal                  = {Building and Environment},
  Year                     = {2004},
  Note                     = {doi: DOI: 10.1016/j.buildenv.2003.11.005},
  Number                   = {9},
  Pages                    = {1043-1048},
  Volume                   = {39},

  ISSN                     = {0360-1323},
  Keywords                 = {Carbon monoxide
Numerical methods
Air quality
Ventilation
Enclosed garages},
  Type                     = {Journal Article},
  Url                      = {http://www.sciencedirect.com/science/article/B6V23-4CC7B73-3/2/d85d2ebccae986a1d1a8fb983cb22fa7}
}

@Article{Dufort1953,
  Title                    = {Stability Conditions in the Numerical Treatment of Parabolic Differential Equations},
  Author                   = {Dufort, E. C. and Frankel, S. P.},
  Journal                  = {Math. Tables and Other Aids to Computation},
  Year                     = {1953},
  Pages                    = {559-673},
  Volume                   = {7},

  Type                     = {Journal Article}
}

@Article{Dufour,
  Title                    = {A Review of CFD Applications to Pulmonary Drug Delivery: Pros, Cons and Gaps},
  Author                   = {Dufour, F. and Nichols, S. and Downie, S. and Davies, G.},

  Type                     = {Journal Article}
}

@Article{Dulay2002,
  Title                    = {Olfactory acuity and cognitive function converge in older adulthood: Support for the common cause hypothesis},
  Author                   = {Dulay,Mario F. and Murphy,Claire},
  Journal                  = {Psychology and aging},
  Year                     = {2002},
  Number                   = {3},
  Pages                    = {392-404},
  Volume                   = {17},

  ISBN                     = {0882-7974},
  Language                 = {English},
  Url                      = {http://search.proquest.com/docview/614375627?accountid=13552}
}

@Article{Dulay2002a,
  Title                    = {Olfactory acuity and cognitive function converge in older adulthood: Support for the common cause hypothesis},
  Author                   = {Dulay,Mario F. and Murphy,Claire},
  Journal                  = {Psychology and aging},
  Year                     = {2002},
  Number                   = {3},
  Pages                    = {392-404},
  Volume                   = {17},

  ISBN                     = {0882-7974},
  Language                 = {English},
  Url                      = {http://search.proquest.com/docview/614375627?accountid=13552}
}

@Article{Dunnett1986,
  Title                    = {A mathematical theory to two-dimensional blunt body sampling},
  Author                   = {Dunnett, S.J. and Ingham, D.B.},
  Journal                  = {Journal of Aerosol Science},
  Year                     = {1986},
  Number                   = {5},
  Pages                    = {839-853},
  Volume                   = {17},

  Type                     = {Journal Article}
}

@Article{Dupin2006,
  Title                    = {A multi-component lattice Boltzmann scheme: Towards the mesoscale simulation of blood flow},
  Author                   = {Dupin, M. M. and Halliday, I. and Care, C. M.},
  Journal                  = {Medical engineering \& physics},
  Year                     = {2006},
  Number                   = {1},
  Pages                    = {13-18},
  Volume                   = {28},

  ISSN                     = {1350-4533},
  Keywords                 = {Lattice Boltzmann
Multiphase flow
Blood flow
Blunting
Chaining},
  Type                     = {Journal Article},
  Url                      = {http://linkinghub.elsevier.com/retrieve/pii/S1350453305000895?showall=true}
}

@Article{Durbin2009,
  Title                    = {Limiters and wall treatments in applied turbulence modeling},
  Author                   = {Durbin, P. A.},
  Journal                  = {Fluid Dynamics Research},
  Year                     = {2009},
  Number                   = {1},
  Pages                    = {012203},
  Volume                   = {41},

  Abstract                 = {Issues in practical use of turbulence models include large strain limiters, wall functions and roughness corrections. Motivations for studying these topics are provided and various recent developments in modeling are reviewed. Inequalities and consistency between wall and outer region models are discussed. New elliptic relaxation models are also surveyed. These include elliptic blending and changes of the dependent variable in scalar eddy viscosity models.},
  ISSN                     = {1873-7005},
  Type                     = {Journal Article},
  Url                      = {http://stacks.iop.org/1873-7005/41/i=1/a=012203}
}

@Article{Durbin1991,
  Title                    = {Near-wall turbulence closure modeling without “damping functions�},
  Author                   = {Durbin, P. A.},
  Journal                  = {Theoretical and Computational Fluid Dynamics},
  Year                     = {1991},
  Note                     = {10.1007/BF00271513},
  Number                   = {1},
  Pages                    = {1-13},
  Volume                   = {3},

  Abstract                 = {An elliptic relaxation model is proposed for the strongly inhomogeneous region near the wall in wall-bounded turbulent shear flow. This model enables the correct kinematic boundary condition to be imposed on the normal component of turbulent intensity. Hence, wall blocking is represented. Means for enforcing the correct boundary conditions on the other components of intensity and on the k — ? equations are discussed. The present model agrees quite well with direct numerical simulation (DNS) data. The virtue of the present approach is that arbitrary “damping functions� are not required.},
  Type                     = {Journal Article},
  Url                      = {http://dx.doi.org/10.1007/BF00271513}
}

@Article{Eastes1995,
  Title                    = {Dissolution of fibres inhaled by rats},
  Author                   = {Eastes, Walter and Hadiey, John G.},
  Journal                  = {Inhalation Toxicology},
  Year                     = {1995},
  Number                   = {2},
  Pages                    = {179-196},
  Volume                   = {7},

  Doi                      = {doi:10.3109/08958379509029092},
  Type                     = {Journal Article},
  Url                      = {http://informahealthcare.com/doi/abs/10.3109/08958379509029092}
}

@InProceedings{Eaton,
  Title                    = {Attenuation of gas turbulence by a nearly stationary dispersion of solid particles},
  Author                   = {Eaton, J.K.},
  Booktitle                = {5th Microgravity Fluid Physics and Transport Phenomena Conference},

  Type                     = {Conference Proceedings}
}

@Article{Ebina1993,
  Title                    = {Cellular hypertrophy and hyperplasia of airway smooth muscles underlying bronchial asthma. A 3-D morphometric study},
  Author                   = {Ebina, M. and Takahashi, T. and Chiba, T. and Motomiya, M.},
  Journal                  = {Am Rev Respir Dis},
  Year                     = {1993},
  Pages                    = {720–726},
  Volume                   = {148},

  Type                     = {Journal Article}
}

@Article{Eccles2000,
  Title                    = {Nasal airflow in health and disease},
  Author                   = {Eccles, R.},
  Journal                  = {Acta Oto-Laryngologica},
  Year                     = {2000},
  Note                     = {Biomedical; Continental Europe; Double Blind Peer Reviewed; Editorial Board Reviewed; Europe; Expert Peer Reviewed; Peer Reviewed. No. of Refs: 145 ref. NLM UID: 0370354.
PMID: 11039867},
  Number                   = {5},
  Pages                    = {580-595},
  Volume                   = {120},

  Abstract                 = {This review examines our present understanding of the physiology, pathophysiology and pharmacology of nasal airflow. The main aim of the review is to discuss the basic scientific and clinical knowledge that is essential for a proper understanding of the usefulness of measurements of nasal airflow in the clinical practice of rhinology. The review concludes with a discussion of the measurement of nasal airflow to assess the efficacy of surgery in the treatment of nasal obstruction. Areas covered by the review include: influence of nasal blood vessels on nasal airflow; nasal valve and control of nasal airflow; autonomic control of nasal airflow; normal nasal airflow; nasal cycle; central control of nasal airflow; effect of changes in posture on nasal airflow; effect of exercise on nasal airflow; effect of hyperventilation and rebreathing on nasal airflow; nasal airflow in animals; cerebral effects of nasal airflow; sensation of nasal airflow; sympathomimetics and sympatholytics; histamine and antihistamines; bradykinin; and corticosteroids.},
  ISSN                     = {0001-6489},
  Keywords                 = {Air
Health Status
Manometry -- Methods
Nasal Cavity -- Physiology
Administration, Intranasal
Antiinflammatory Agents -- Administration and Dosage
Antiinflammatory Agents -- Pharmacokinetics
Autonomic Nervous System -- Physiology
Exercise
Hyperventilation
Nasal Cavity -- Blood Supply
Nasal Cavity -- Metabolism
Nasal Septum -- Blood Supply
Nasal Septum -- Metabolism
Nasal Septum -- Physiology
Steroids
Sympathomimetics -- Pharmacokinetics
Turbinates -- Blood Supply
Turbinates -- Metabolism
Turbinates -- Physiology
Vasodilator Agents -- Metabolism},
  Type                     = {Journal Article},
  Url                      = {http://search.ebscohost.com/login.aspx?direct=true&db=c8h&AN=2009434274&site=ehost-live&scope=site}
}

@Article{Eccles2000a,
  Title                    = {Nasal airflow in health and disease},
  Author                   = {Eccles, R.},
  Journal                  = {Acta Oto-Laryngologica},
  Year                     = {2000},
  Pages                    = {580-595},
  Volume                   = {120},

  Abstract                 = {This review examines our present understanding of the physiology, pathophysiology and pharmacology of nasal airflow. The main aim of the review is to discuss the basic scientific and clinical knowledge that is essential for a proper understanding of the usefulness of measurements of nasal airflow in the clinical practice of rhinology. The review concludes with a discussion of the measurement of nasal airflow to assess the efficacy of surgery in the treatment of nasal obstruction. Areas covered by the review include: influence of nasal blood vessels on nasal airflow; nasal valve and control of nasal airflow; autonomic control of nasal airflow; normal nasal airflow; nasal cycle; central control of nasal airflow; effect of changes in posture on nasal airflow; effect of exercise on nasal airflow; effect of hyperventilation and rebreathing on nasal airflow; nasal airflow in animals; cerebral effects of nasal airflow; sensation of nasal airflow; sympathomimetics and sympatholytics; histamine and antihistamines; bradykinin; and corticosteroids.},
  ISSN                     = {0001-6489},
  Keywords                 = {Air Health Status Manometry – Methods Nasal Cavity – Physiology Administration
Intranasal Antiinflammatory Agents – Administration and Dosage Antiinflammatory Agents – Pharmacokinetics Autonomic Nervous System – Physiology Exercise Hyperventilation Nasal Cavity – Blood Supply Nasal Cavity – Metabolism Nasal Septum – Blood Supply Nasal Septum – Metabolism Nasal Septum – Physiology Steroids Sympathomimetics – Pharmacokinetics Turbinates – Blood Supply Turbinates – Metabolism Turbinates – Physiology Vasodilator Agents – Metabolism},
  Type                     = {Journal Article}
}

@Article{Eccles1996,
  Title                    = {A role for the nasal cycle in respiratory defence},
  Author                   = {Eccles, R},
  Journal                  = {Eur Respir J},
  Year                     = {1996},
  Number                   = {2},
  Pages                    = {371-376},
  Volume                   = {9},

  Abstract                 = {This review describes the phenomenon of the nasal cycle, which consists of periodic congestion and decongestion of the nasal venous sinusoids. The hypothesis is put forward that the nasal venous sinusoids participate in respiratory defence by generation of plasma exudate. This hypothesis is based on recent studies, which have shown that the nasal venous sinusoids have a fenestrated endothelium and that the nasal cycle is increased during periods of nasal infection; and also on a series of older observations in the literature, which link the generation of nasal fluid to the decongestion of nasal venous sinusoids. It is proposed that the periodic congestion and decongestion of nasal venous sinusoids may provide a pump mechanism for the generation of plasma exudate, and that this mechanism is an important component of respiratory defence.},
  Type                     = {Journal Article},
  Url                      = {http://erj.ersjournals.com/cgi/content/abstract/9/2/371}
}

@Article{Eck2000,
  Title                    = {Droplet size distributions in a solution nasal spray},
  Author                   = {Eck, C.R. and McGrath, T.F. and Perlwitz, A.G.},
  Journal                  = {Respir. Drug Deliv.},
  Year                     = {2000},
  Pages                    = {475-478},
  Volume                   = {7},

  Type                     = {Journal Article}
}

@Article{Edelstein1996,
  Title                    = {Aging of the Normal Nose in Adults},
  Author                   = {Edelstein, David R.},
  Journal                  = {The Laryngoscope},
  Year                     = {1996},
  Number                   = {S81},
  Pages                    = {1--25},
  Volume                   = {106},

  Doi                      = {10.1097/00005537-199609001-00001},
  ISSN                     = {1531-4995},
  Publisher                = {John Wiley \& Sons, Inc.},
  Url                      = {http://dx.doi.org/10.1097/00005537-199609001-00001}
}

@TechReport{Edwards2007,
  Title                    = {Respiratory infectious disease burden in Australia},
  Author                   = {Edwards, B. and Jenkins, C.},
  Institution              = {The Australian Lung Foundation},
  Year                     = {2007},
  Type                     = {Report}
}

@Article{Edwards2002,
  Title                    = {Delivery of biological agents by aerosols},
  Author                   = {Edwards, D.A.},
  Journal                  = {AIChE Journal},
  Year                     = {2002},
  Pages                    = {2-6},
  Volume                   = {48},

  Type                     = {Journal Article}
}

@Article{Edwards1997,
  Title                    = {Large porous particles for pulmonary drug delivery},
  Author                   = {Edwards, D.A. and Hanes, J. and Caponetti, G. and Hrkach, J. and Ben-Jebria, A. and Eskew, M.L. and Mintzes, J. and Deaver, D. and Lotan, N. and Langer, R.},
  Journal                  = {Science},
  Year                     = {1997},
  Pages                    = {1868-1872},
  Volume                   = {276},

  Type                     = {Journal Article}
}

@Article{Einstein1905,
  Title                    = {On the Movement of Small Particles Suspended in Stationary Liquids Required by the Molecular-Kinetic Theory of Heat},
  Author                   = {Einstein, A.},
  Year                     = {1905},
  Pages                    = {549-560},
  Volume                   = {17},

  Type                     = {Journal Article}
}

@Article{Eisele1992,
  Title                    = {Individuality of breathing patterns during hypoxia and exercise},
  Author                   = {Eisele, J. H. and Wuyam, B. and Savourey, G. and Eterradossi, J. and Bittel, J. H. and Benchetrit, G.},
  Journal                  = {Journal of Applied Physiology},
  Year                     = {1992},
  Number                   = {6},
  Pages                    = {2446-2453},
  Volume                   = {72},

  Abstract                 = {Breathing was recorded via a pulsed ultrasonic flowmeter in 11 healthy subjects, at rest and during steady-state exercise (at 50% of their maximal O2 consumption) at both sea level (200 m) and simulated altitude (4,500 m in a hypobaric chamber). The pattern of breathing was quantified breath by breath in terms of classical respiratory variables (tidal volume and inspiratory and expiratory times), and the shape of the entire airflow profile was quantified by harmonic analysis. Statistical tests were used to compare the within-individual with the between-individual variations. In comparing the sea level vs. altitude rest (16% increase in ventilation) and sea level vs. altitude exercise (40% increase in ventilation) airflow profiles, we found a significantly greater resemblance within the individual than between individuals. Comparisons of sea level rest and exercise (295% increase in ventilation) and altitude rest and exercise (375% increase in ventilation) revealed no similarity within individuals. Despite airflow profile changes between rest and exercise, it is still possible to attest to a diversity of flow profile between individuals during exercise. Hypoxia at rest or during exercise does not alter the phenomenon of the individuality of breathing patterns.},
  Type                     = {Journal Article},
  Url                      = {http://jap.physiology.org/content/72/6/2446.abstract}
}

@Article{ElGharbi2011,
  Title                    = {An improved near-wall treatment for turbulent channel flows},
  Author                   = {El Gharbi, Najla and Absi, Rafik and Benzaoui, Ahmed and Bennacer, Rachid},
  Journal                  = {International Journal of Computational Fluid Dynamics},
  Year                     = {2011},
  Number                   = {1},
  Pages                    = {41 - 46},
  Volume                   = {25},

  ISSN                     = {1061-8562},
  Type                     = {Journal Article},
  Url                      = {http://www.informaworld.com/10.1080/10618562.2011.554832}
}

@Article{Elad2006,
  Title                    = {Physical stresses at the air-wall interface of the human nasal cavity during breathing},
  Author                   = {Elad, David and Naftali, Sara and Rosenfeld, Moshe and Wolf, Michael},
  Journal                  = {Journal of Applied Physiology},
  Year                     = {2006},
  Number                   = {3},
  Pages                    = {1003-1010},
  Volume                   = {100},

  Abstract                 = {The nose is the front line defender of the respiratory system and is rich with mechanoreceptors, thermoreceptors, and nerve endings. A time-dependent computational model of transport through nasal models of a healthy human has been used to analyze the fields of physical stresses that may develop at the air-wall interface of the nasal mucosa. Simulations during quiet breathing revealed wall shear stresses as high as 0.3 Pa in the noselike model and 1.5 Pa in the anatomical model. These values are of the same order of those known to exist in uniform large arteries. The distribution of temperature near the nasal wall at peak inspiration is similar to that of wall shear stresses. The lowest temperatures occur in the vicinity of high stresses due to the narrow passageway in these locations. Time and spatial gradients of these stresses may have functional effects on nasal sensation of airflow and may play a role in the well-being of nasal breathing.},
  Doi                      = {10.1152/japplphysiol.01049.2005},
  Type                     = {Journal Article},
  Url                      = {http://jap.physiology.org/cgi/content/abstract/100/3/1003}
}

@Article{Elad2006a,
  Title                    = {Wall shear stresses in the normal and septal-deviated nose},
  Author                   = {Elad, D. and Naftali, S. and Rosenfeld, M. and Wolf, M.},
  Journal                  = {Journal of Biomechanics},
  Year                     = {2006},
  Number                   = {Supplement 1},
  Pages                    = {S270-S270},
  Volume                   = {39},

  ISSN                     = {0021-9290},
  Type                     = {Journal Article},
  Url                      = {http://www.sciencedirect.com/science/article/B6T82-4KR88PB-1FX/2/b42e92906d7c75defc4661cc51634b50}
}

@Article{Elad2008,
  Title                    = {Air-conditioning in the human nasal cavity },
  Author                   = {David Elad and Michael Wolf and Tilman Keck},
  Journal                  = {Respiratory Physiology \& Neurobiology },
  Year                     = {2008},
  Note                     = {Respiratory Biomechanics },
  Number                   = {1-3},
  Pages                    = {121 - 127},
  Volume                   = {163},

  Abstract                 = {Healthy humans normally breathe through their nose even though its complex geometry imposes a significantly higher resistance in comparison with mouth breathing. The major functional roles of nasal breathing are defense against infiltrating particles and conditioning of the inspired air to nearly alveolar conditions in order to maintain the internal milieu of the lung. The state-of-the-art of the existing knowledge on nasal air-conditioning will be discussed in this review, including in vivo measurements in humans and computational studies on nasal air-conditioning capacity. Areas where further studies will improve our understanding and may help medical diagnosis and intervention in pathological states will be introduced. },
  Doi                      = {http://dx.doi.org/10.1016/j.resp.2008.05.002},
  ISSN                     = {1569-9048},
  Keywords                 = {Nasal function},
  Url                      = {http://www.sciencedirect.com/science/article/pii/S1569904808001262}
}

@Article{Elder2006,
  Title                    = {Translocation of Inhaled Ultrafine Manganese Oxide Particles to the Central Nervous System},
  Author                   = {Elder, Alison and Gelein, Robert and Silva, Vanessa and Feikert, Tessa and Opanashuk, Lisa and Carter, Janet and Potter, Russell and Maynard, Andrew and Ito, Yasuo and Finkelstein, Jacob and Oberdörster, Günter},
  Journal                  = {Environment Health Perspective},
  Year                     = {2006},
  Number                   = {8},
  Pages                    = {1172-1178},
  Volume                   = {114},

  Type                     = {Journal Article}
}

@Article{Elghobashi1983,
  Title                    = {A Two-Equation Turbulence Model for Two-Phase Flows},
  Author                   = {Elghobashi, S. and Abou- Arab, T.W. },
  Journal                  = {Physics of Fluids},
  Year                     = {1983},
  Pages                    = {931-938},
  Volume                   = {26},

  Type                     = {Journal Article}
}

@Article{Elghobashi1992,
  Title                    = {Direct simulation of particle dispersion in a decaying isotropic turbulence},
  Author                   = {Elghobashi, S. and Truesdell, G. C.},
  Journal                  = {Journal of Fluid Mechanics},
  Year                     = {1992},
  Note                     = {Cited By (since 1996): 185
Export Date: 3 June 2011
Source: Scopus},
  Pages                    = {655-700},
  Volume                   = {242},

  Type                     = {Journal Article},
  Url                      = {http://www.scopus.com/inward/record.url?eid=2-s2.0-0027009842&partnerID=40&md5=afc08121346aba2f925ab9181284d73f}
}

@Article{,
  Title                    = {Rhinosinusitis: Establishing definitions for clinical research and patient care},
  Author                   = {Eli O. Meltzer, MD, a Daniel L. Hamilos, MD,b James A. Hadley, MD,c Donald C. Lanza, and MD, d Bradley F. Marple, MD,e Richard A. Nicklas, MD,f Claus Bachert, MD, PhD,g and James Baraniuk, MD, h Fuad M. Baroody, MD,i Michael S. Benninger, MD,j and Itzhak Brook, MD, k Badrul A. Chowdhury, MD, PhD,l Howard M. Druce, MD,m Stephen and Durham, MD, n Berrylin Ferguson, MD,o Jack M. Gwaltney, Jr, MD,p Michael Kaliner, and MD, q David W. Kennedy, MD,r Valerie Lund, MD,s Robert Naclerio, MD,t and Ruby Pawankar, MD, PhD,u Jay F. Piccirillo, MD,v Patricia Rohane, MD,w Ronald Simon, and MD, x Raymond G. Slavin, MD, MS,y Alkis Togias, MD,z Ellen R. Wald, MD,aa and and S. James Zinreich, MDbb},
  Journal                  = {The Journal Of Allergy and Clinical Immunology},
  Year                     = {2004},

  Type                     = {Journal Article}
}

@Article{Ellison1997,
  Title                    = {Stochastic response of passive vibration control systems to g-jitter excitation},
  Author                   = {Ellison, J. and Ahmadi, G. and Grodsinsky, C.},
  Journal                  = {Microgravity Science and Technology},
  Year                     = {1997},
  Note                     = {Cited By (since 1996): 2
Export Date: 3 June 2011
Source: Scopus},
  Number                   = {1},
  Pages                    = {2-12},
  Volume                   = {10},

  Type                     = {Journal Article},
  Url                      = {http://www.scopus.com/inward/record.url?eid=2-s2.0-0031380777&partnerID=40&md5=1aa9031bc782fc00424f0205f9f1d856}
}

@Article{Ellison1995,
  Title                    = {Stochastic model for microgravity excitation},
  Author                   = {Ellison, Joseph and Ahmadi, Goodarz and Grodsinsky, Carlos},
  Journal                  = {Journal of Aerospace Engineering},
  Year                     = {1995},
  Note                     = {Cited By (since 1996): 1
Export Date: 3 June 2011
Source: Scopus},
  Number                   = {2},
  Pages                    = {100-106},
  Volume                   = {8},

  Type                     = {Journal Article},
  Url                      = {http://www.scopus.com/inward/record.url?eid=2-s2.0-0029289854&partnerID=40&md5=77a453ba2ad45953dad5b6820829055a}
}

@Article{Ellison1995a,
  Title                    = {Particle motion in a liquid under g-jitter excitation},
  Author                   = {Ellison, J. and Ahmadi, G. and Regel, L. and Wilcox, W.},
  Journal                  = {Microgravity Science and Technology},
  Year                     = {1995},
  Note                     = {Cited By (since 1996): 3
Export Date: 3 June 2011
Source: Scopus},
  Number                   = {3},
  Pages                    = {140-147},
  Volume                   = {8},

  Type                     = {Journal Article},
  Url                      = {http://www.scopus.com/inward/record.url?eid=2-s2.0-0029405495&partnerID=40&md5=7418926b46cdc11791cff5b85d086adb}
}

@Article{Elsaesser2012,
  Title                    = {Toxicology of nanoparticles},
  Author                   = {Elsaesser, Andreas and Howard, C. Vyvyan},
  Journal                  = {Advanced Drug Delivery Reviews},
  Year                     = {2012},
  Number                   = {2},
  Pages                    = {129-137},
  Volume                   = {64},

  Doi                      = {http://dx.doi.org/10.1016/j.addr.2011.09.001},
  ISSN                     = {0169-409X},
  Keywords                 = {Nanotoxicology
Nanoparticle characterisation
Surface
Dose
Agglomeration
Reactive oxygen
Inflammation
Entry routes
Risk assessment
Hazard identification},
  Type                     = {Journal Article},
  Url                      = {http://www.sciencedirect.com/science/article/pii/S0169409X11002328}
}

@Article{Ely2007,
  Title                    = {Effervescent dry powder for respiratory drug delivery, },
  Author                   = {Ely, L. and Roa, W. and Finlay, W.H. and Löbenberg, R.},
  Journal                  = {Eur. J. Pharm. Biopharm. 65 (3) (2007) },
  Year                     = {2007},
  Number                   = {3},
  Pages                    = {346-353},
  Volume                   = {65},

  Type                     = {Journal Article}
}

@Article{Elyyan2008,
  Title                    = {Investigation of dimpled fins for heat transfer enhancement in compact heat exchangers},
  Author                   = {Elyyan, Mohammad A. and Rozati, Ali and Tafti, Danesh K.},
  Journal                  = {International Journal of Heat and Mass Transfer},
  Year                     = {2008},
  Number                   = {11-12},
  Pages                    = {2950-2966},
  Volume                   = {51},

  Abstract                 = {Direct and Large-Eddy simulations are conducted in a fin bank with dimples and protrusions over a Reynolds number range of ReH = 200 to 15,000, encompassing laminar, transitional and fully turbulent regimes. Two dimple-protrusion geometries are studied in which the same imprint pattern is investigated for two different channel heights or fin pitches, Case 1 with twice the fin pitch of Case 2. The smaller fin pitch configuration (Case 2) develops flow instabilities at ReH = 450, whereas Case 1 undergoes transition at ReH = 900. Case 2, exhibits higher Nusselt numbers and friction coefficients in the low Reynolds number regime before Case 1 transitions to turbulence, after which, the differences between the two decreases considerably in the fully turbulent regime. Vorticity generated within the dimple cavity and at the dimple rim contribute substantially to heat transfer augmentation on the dimple side, whereas flow impingement and acceleration between protrusions contribute substantially on the protrusion side. While friction drag dominates losses in Case 1 at low Reynolds numbers, both form and friction drag contributed equally in Case 2. As the Reynolds number increases to fully turbulent flow, form drag dominates in both cases, contributing about 80% to the total losses. While both geometries are viable and competitive with other augmentation surfaces in the turbulent regime, Case 2 with larger feature sizes with respect to the fin pitch is more appropriate in the low Reynolds number regime ReH < 2000, which makes up most of the operating range of typical compact heat exchangers.},
  ISSN                     = {0017-9310},
  Keywords                 = {Dimples
Compact heat exchangers
LES},
  Type                     = {Journal Article},
  Url                      = {http://www.sciencedirect.com/science/article/B6V3H-4R2H6NY-2/2/f0b00b762d78c98cd10fe6ab19b1676e}
}

@Article{Enarson1990,
  Title                    = {Characterization of health effects of wood dust exposures},
  Author                   = {Enarson, Donald A. and Chan-Yeung, Moira},
  Journal                  = {American Journal of Industrial Medicine},
  Year                     = {1990},
  Number                   = {1},
  Pages                    = {33-38},
  Volume                   = {17},

  Doi                      = {10.1002/ajim.4700170107},
  ISSN                     = {1097-0274},
  Keywords                 = {wood dusts
ODTS
chronic bronchitis
fungicide exposures
allergic alveolitis},
  Type                     = {Journal Article},
  Url                      = {http://dx.doi.org/10.1002/ajim.4700170107}
}

@Article{Enarson1990a,
  Title                    = {Characterization of health effects of wood dust exposures},
  Author                   = {Enarson, Donald A. and Chan-Yeung, Moira},
  Journal                  = {American Journal of Industrial Medicine},
  Year                     = {1990},
  Pages                    = {33-38},
  Volume                   = {17},

  Doi                      = {10.1002/ajim.4700170107},
  ISSN                     = {1097-0274},
  Keywords                 = {allergic
alveolitis
bronchitis
chronic
dusts
exposures
fungicide
ODTS
wood},
  Type                     = {Journal Article},
  Url                      = {http://onlinelibrary.wiley.com/doi/10.1002/ajim.4700170107/abstract}
}

@Article{Enarson1990b,
  Title                    = {Characterization of Health Effects of Wood Dust Exposures.},
  Author                   = {Enarson, D.A. and Chan-Yeung, M.},
  Journal                  = {Am. Journal Ind. Medicine},
  Year                     = {1990},
  Pages                    = {33-38},
  Volume                   = {17},

  Type                     = {Journal Article}
}

@Article{Eric2011,
  Title                    = {Measurement of lung airways in three dimensions using hyperpolarized helium-3 MRI},
  Author                   = {Eric, T. Peterson and Jionghan, Dai and James, H. Holmes and Sean, B. Fain},
  Journal                  = {Physics in Medicine and Biology},
  Year                     = {2011},
  Number                   = {10},
  Pages                    = {3107},
  Volume                   = {56},

  Abstract                 = {Large airway measurement is clinically important in cases of airway disease and trauma. The gold standard is computed tomography (CT), which allows for airway measurement. However, the ionizing radiation dose associated with CT is a major limitation in longitudinal studies and trauma. To avoid ionizing radiation from CT, we present a method for measuring the large airway diameter in humans using hyperpolarized helium-3 (HPHe) MRI in conjunction with a dynamic 3D radial acquisition. An algorithm is introduced which utilizes the significant airway contrast for semi-automated segmentation and skeletonization which is used to derive the airway lumen diameter. The HPHe MRI method was validated with quantitative CT in an excised and desiccated porcine lung (linear regression R 2 = 0.974 and slope = 0.966 over 32 airway segments). The airway lumen diameters were then compared in 24 human subjects (22 asthmatics and 2 normals; linear regression R 2 value of 0.799 and slope = 0.768 over 309 airway segments). The feasibility for airway path analysis to areas of ventilation defect is also demonstrated.},
  ISSN                     = {0031-9155},
  Type                     = {Journal Article},
  Url                      = {http://stacks.iop.org/0031-9155/56/i=10/a=014}
}

@Article{Ermak1980,
  Title                    = {Numerical integration of the Langevin equation: Monte Carlo simulation},
  Author                   = {Ermak, Donald L. and Buckholz, Helen},
  Journal                  = {Journal of Computational Physics},
  Year                     = {1980},
  Note                     = {doi: DOI: 10.1016/0021-9991(80)90084-4},
  Number                   = {2},
  Pages                    = {169-182},
  Volume                   = {35},

  ISSN                     = {0021-9991},
  Type                     = {Journal Article},
  Url                      = {http://www.sciencedirect.com/science/article/B6WHY-4DDR2JN-3Y/2/922b25ce57eff6f336737b2b5e0aa77a}
}

@Article{Ermak1978,
  Title                    = {Brownian dynamics with hydrodynamic interactions},
  Author                   = {Ermak, Donald L. and McCammon, J. A.},
  Journal                  = {The Journal of Chemical Physics},
  Year                     = {1978},
  Number                   = {4},
  Pages                    = {1352-1360},
  Volume                   = {69},

  Keywords                 = {BROWNIAN MOVEMENT
DIFFUSION
LANGEVIN EQUATION
SOLUTIONS
HYDRODYNAMICS},
  Type                     = {Journal Article},
  Url                      = {http://link.aip.org/link/?JCP/69/1352/1}
}

@Article{Ertbruggen2008,
  Title                    = {Validation of CFD predictions of flow in a 3D alveolated bend with experimental data},
  Author                   = {van Ertbruggen, C. and Corieri, P. and Theunissen, R. and Riethmuller, M. L. and Darquenne, C.},
  Journal                  = {Journal of Biomechanics},
  Year                     = {2008},
  Number                   = {2},
  Pages                    = {399-405},
  Volume                   = {41},

  ISSN                     = {0021-9290},
  Keywords                 = {Alveolar flow
CFD
PIV},
  Type                     = {Journal Article},
  Url                      = {http://www.sciencedirect.com/science/article/B6T82-4PTF9CB-1/2/0d92e15dbbcd0bebe5ff990e929e02b2}
}

@Article{Ertbruggen2005,
  Title                    = {Anatomically based three-dimensional model of airways to simulate flow and particle transport using computational fluid dynamics},
  Author                   = {van Ertbruggen, C. and Hirsch, C. and Paiva, M.},
  Journal                  = {Journal Appl. Physiol.},
  Year                     = {2005},
  Pages                    = {970-980},
  Volume                   = {98},

  Type                     = {Journal Article}
}

@Article{EscotBocanegra2010,
  Title                    = {Experimental and numerical studies on the burning of aluminum micro and nanoparticle clouds in air},
  Author                   = {Escot Bocanegra, P. and Davidenko, D. and Sarou-Kanian, V. and Chauveau, C. and Gökalp, I.},
  Journal                  = {Experimental Thermal and Fluid Science},
  Year                     = {2010},
  Number                   = {3},
  Pages                    = {299-307},
  Volume                   = {34},

  Abstract                 = {An experimental study has been conducted to determine flame propagation velocities in clouds of micro- (4.8 [mu]m) and nano- (187 nm) aluminum particles in air at various concentrations. The experimental results show faster flame propagation in nanoparticle cloud with respect to the case of microparticles. Maximum flame temperature has been measured using a high-resolution spectrometer operating in the visible range. Analysis of combustion residual shows that nanoparticles combustion is realized via the gas-phase mechanism. A three-stage particle combustion model has been proposed based on these observations. Model parameters have been fitted to match the experimental results on the flame velocity and maximum temperature. Particle burning time is estimated from the flame simulations.},
  ISSN                     = {0894-1777},
  Keywords                 = {Combustion
Nanoparticles
Aluminum
Propagation
Temperature
Modeling},
  Type                     = {Journal Article},
  Url                      = {http://www.sciencedirect.com/science/article/B6V34-4XHCHYV-1/2/d0e166dcdda550f8edd5099a4dd10652}
}

@Article{Escudier1988,
  Title                    = {Vortex breakdown: Observations and explanations},
  Author                   = {Escudier, M.},
  Journal                  = {Prog. Aerospace Sci.},
  Year                     = {1988},
  Number                   = {25},
  Pages                    = {189-229},
  Volume                   = {422},

  Type                     = {Journal Article}
}

@Article{Eskens1995,
  Title                    = {Septic shock caused by group G [beta]-haemolytic streptococci as presenting symptom of acute myeloid leukaemia},
  Author                   = {Eskens, F. A. L. M. and Verweij, P. E. and Meis, J. F. G. M. and Soomers, A.},
  Journal                  = {The Netherlands Journal of Medicine},
  Year                     = {1995},
  Number                   = {3},
  Pages                    = {153-155},
  Volume                   = {46},

  Abstract                 = {A patient with rapidly fatal septic shock caused by group G [beta]-haemolytic streptococci as presenting symptom of acute myeloid leukaemia is presented. Although the association of septic shock due to Group G [beta]-haemolytic streptococci and different kinds of malignancy is known, presentation of acute myeloid leukaemia in this form is rare.},
  ISSN                     = {0300-2977},
  Keywords                 = {Group G [beta]-haemolytic streptococci
Septic shock
Acute myeloid leukaemia},
  Type                     = {Journal Article},
  Url                      = {http://www.sciencedirect.com/science/article/B6TCK-3YXBCTP-W/2/0a34eab5395b7b7e0e033ee8eeb1818f}
}

@Article{Etherington1998,
  Title                    = {Deposition and clearance of inhaled particles in the human nasal passage: Implications for dose assessment},
  Author                   = {Etherington, G. and Smith, J.R.H. and Bailey, M.R. and Dorrian, M.D. and Shutt, A.L. and Youngman, M.J.},
  Journal                  = {Radiation Protection Dosimetry},
  Year                     = {1998},
  Number                   = {1-4},
  Pages                    = {249-252},
  Volume                   = {79},

  Type                     = {Journal Article}
}

@Article{Evans1980,
  Title                    = {Mechanics and Thermodynamics of Biomembranes},
  Author                   = {Evans, E. A. and Skalak, R. and Weinbaum, S.},
  Journal                  = {Journal of Biomechanical Engineering},
  Year                     = {1980},
  Number                   = {4},
  Pages                    = {345-345},
  Volume                   = {102},

  Type                     = {Journal Article},
  Url                      = {http://link.aip.org/link/?JBY/102/345/1}
}

@Article{Everard2003,
  Title                    = {Inhalation therapy for infants},
  Author                   = {Everard, Mark L},
  Journal                  = {Advanced drug delivery reviews},
  Year                     = {2003},
  Note                     = {C:\Users\sean\AppData\Roaming\Zotero\Zotero\Profiles\16a4oype.default\zotero\storage\5SI5U8GH\inhalation therapy for infants.pdf},
  Pages                    = {869–878},
  Volume                   = {55},

  Type                     = {Journal Article},
  Url                      = {http://ac.els-cdn.com/S0169409X03000826/1-s2.0-S0169409X03000826-main.pdf?_tid=ba36791c-421f-11e4-b9f9-00000aacb360&acdnat=1411366656_1917f27c8a1cdedb2706f2d53d640eb9}
}

@Article{Ezzouhri2009,
  Title                    = {Large Eddy simulation of turbulent mixed convection in a 3D ventilated cavity: Comparison with existing data},
  Author                   = {Ezzouhri, Ridouane and Joubert, Patrice and Penot, François and Mergui, Sophie},
  Journal                  = {International Journal of Thermal Sciences},
  Year                     = {2009},
  Number                   = {11},
  Pages                    = {2017-2024},
  Volume                   = {48},

  Abstract                 = {We consider in this study the mixed convection airflow encountered in a 3D anisothermal cavity ventilated with supply and exhaust slots under stable thermal stratification. The flow in this cavity has been experimentally studied in the past (S. Mergui, Caractérisation expérimentale des écoulements d'air de convection naturelle et mixte dans une cavité fermée, thèse de l'Université de Poitiers, France, 1993) and was subject to a jet deflection and to a sudden change in the flow pattern between a general clockwise rotation and a counter-clockwise rotation when varying the inlet jet velocity. This phenomenon also exhibits a hysteresis effect depending on the way the velocity is changed, which made us to think that this flow bifurcation is of subcritical nature. Numerical studies have been yet devoted to this configuration, using RANS simulations or Large Eddy Simulation (LES), but this phenomenon has not been reported. So, we chose to numerically study this challenging flow with an LES approach associated with a subgrid diffusivity model previously developed for natural convection airflows. The comparison with the available experimental data and with other LES results using a classical dynamic model proves that the present LES not only correctly predicts the mean characteristics of the flow but is also able to correctly reproduce the flow bifurcation and the hysterisis effect.},
  ISSN                     = {1290-0729},
  Keywords                 = {Turbulent mixed convection
Indoor airflow
Large Eddy Simulation
Dynamic subgrid-scale model
Flow bifurcation
Hysteresis cycle},
  Type                     = {Journal Article},
  Url                      = {http://www.sciencedirect.com/science/article/B6VT1-4W7B56W-1/2/76acf80cdfe9596eb64efe58a1eeb96e}
}

@Article{FA¥hraeus1929,
  Title                    = {The Suspension Stability of the Blood. Physiological Reviews},
  Author                   = {FÃ¥hraeus, R. },
  Journal                  = {Physiological Reviews},
  Year                     = {1929},
  Pages                    = {241-274},
  Volume                   = {9},

  Type                     = {Journal Article}
}

@Article{FA¥hraeus1931,
  Title                    = {The Viscosity of Blood in Narrow Capillary Tubes},
  Author                   = {FÃ¥hraeus, R. and Lindqvist, T. },
  Journal                  = {American Journal of Physiology},
  Year                     = {1931},
  Pages                    = {562-568},
  Volume                   = {96},

  Type                     = {Journal Article}
}

@Article{Fabian2008,
  Title                    = {Influenza Virus in Human Exhaled Breath: An Observational Study},
  Author                   = {Fabian, Patricia and McDevitt, James J. and DeHaan, Wesley H. and Fung, Rita O. P. and Cowling, Benjamin J. and Chan, Kwok Hung and Leung, Gabriel M. and Milton, Donald K.},
  Journal                  = {PLoS ONE},
  Year                     = {2008},
  Number                   = {7},
  Pages                    = {e2691},
  Volume                   = {3},

  Abstract                 = {<sec><title>Background</title><p>Recent studies suggest that humans exhale fine particles during tidal breathing but little is known of their composition, particularly during infection.</p></sec><sec><title>Methodology/Principal Findings</title><p>We conducted a study of influenza infected patients to characterize influenza virus and particle concentrations in their exhaled breath. Patients presenting with influenza-like-illness, confirmed influenza A or B virus by rapid test, and onset within 3 days were recruited at three clinics in Hong Kong, China. We collected exhaled breath from each subject onto Teflon filters and measured exhaled particle concentrations using an optical particle counter. Filters were analyzed for influenza A and B viruses by quantitative polymerase chain reaction (qPCR). Twelve out of thirteen rapid test positive patients provided exhaled breath filter samples (7 subjects infected with influenza B virus and 5 subjects infected with influenza A virus). We detected influenza virus RNA in the exhaled breath of 4 (33%) subjects–three (60%) of the five patients infected with influenza A virus and one (14%) of the seven infected with influenza B virus. Exhaled influenza virus RNA generation rates ranged from &lt;3.2 to 20 influenza virus RNA particles per minute. Over 87% of particles exhaled were under 1 µm in diameter.</p></sec><sec><title>Conclusions</title><p>These findings regarding influenza virus RNA suggest that influenza virus may be contained in fine particles generated during tidal breathing, and add to the body of literature suggesting that fine particle aerosols may play a role in influenza transmission.</p></sec>},
  Type                     = {Journal Article},
  Url                      = {http://dx.plos.org/10.1371%2Fjournal.pone.0002691}
}

@InProceedings{Fabio,
  Title                    = {From point cloud to surface: the modelling and visualisation problem},
  Author                   = {Fabio, R.},
  Booktitle                = {International Workshop on Visualisation and Animation of Reality-based 3D Models},
  Volume                   = {XXXIV},

  Type                     = {Conference Proceedings}
}

@Article{Fan1995,
  Title                    = {A sublayer model for wall deposition of ellipsoidal particles in turbulent streams},
  Author                   = {Fan, F.G. and Ahmadi, G.},
  Journal                  = {Journal of Aerosol Science},
  Year                     = {1995},
  Pages                    = {813-840},
  Volume                   = {26},

  Type                     = {Journal Article}
}

@Article{Fan2000,
  Title                    = {Wall depositions of small ellipsoids from turbulent air flow - A Brownian dynamics simulation},
  Author                   = {Fan, F. G. and Ahmadi, G.},
  Journal                  = {Journal of Aerosol Science},
  Year                     = {2000},
  Note                     = {Cited By (since 1996): 12
Export Date: 3 June 2011
Source: Scopus},
  Number                   = {10},
  Pages                    = {1205-1229},
  Volume                   = {31},

  Type                     = {Journal Article},
  Url                      = {http://www.scopus.com/inward/record.url?eid=2-s2.0-0342656201&partnerID=40&md5=7ca729f1dca1a6f96f08d0389fa4437f}
}

@Article{Fan1994,
  Title                    = {On the sublayer model for turbulent deposition of aerosol particles in the presence of gravity and electric fields},
  Author                   = {Fan, F. G. and Ahmadi, G.},
  Journal                  = {Aerosol Science and Technology},
  Year                     = {1994},
  Note                     = {Cited By (since 1996): 18
Export Date: 3 June 2011
Source: Scopus},
  Number                   = {1},
  Pages                    = {49-71},
  Volume                   = {21},

  Type                     = {Journal Article},
  Url                      = {http://www.scopus.com/inward/record.url?eid=2-s2.0-0028278440&partnerID=40&md5=91aa63f0e71e02ef0aa16a7b20d97fcd}
}

@Article{Fan1993,
  Title                    = {A sublayer model for turbulent deposition of particles in vertical ducts with smooth and rough surfaces},
  Author                   = {Fan, F. G. and Ahmadi, G.},
  Journal                  = {Journal of Aerosol Science},
  Year                     = {1993},
  Note                     = {Cited By (since 1996): 57
Export Date: 3 June 2011
Source: Scopus},
  Number                   = {1},
  Pages                    = {45-64},
  Volume                   = {24},

  Type                     = {Journal Article},
  Url                      = {http://www.scopus.com/inward/record.url?eid=2-s2.0-0002804515&partnerID=40&md5=9fa551700dbd4be831e322be02151e8a}
}

@Article{Fan1995a,
  Title                    = {Dispersion of Ellipsoidal Particles in an Isotropic Pseudo-Turbulent Flow Field},
  Author                   = {Fan, Fa-Gung and Ahmadi, Goodarz},
  Journal                  = {Journal of Fluids Engineering},
  Year                     = {1995},
  Number                   = {1},
  Pages                    = {154-161},
  Volume                   = {117},

  Type                     = {Journal Article},
  Url                      = {http://link.aip.org/link/?JFG/117/154/1}
}

@Article{Fan2010,
  Title                    = {Demonstration of pulmonary perfusion heterogeneity induced by gravity and lung inflation using arterial spin labeling},
  Author                   = {Fan, Li and Liu, Shi-yuan and Xiao, Xiang-sheng and Sun, Fei},
  Journal                  = {European Journal of Radiology},
  Year                     = {2010},
  Number                   = {2},
  Pages                    = {249-254},
  Volume                   = {73},

  ISSN                     = {0720-048X},
  Keywords                 = {Magnetic resonance imaging
Arterial spin labeling
Pulmonary perfusion
Lung inflation
Gravity},
  Type                     = {Journal Article},
  Url                      = {http://www.sciencedirect.com/science/article/B6T6F-4V936M8-2/2/634d02bdb313b63040193ec68657b2a0}
}

@Article{Fantini1990,
  Title                    = {Drop size distribution in sprays by image processing},
  Author                   = {Fantini, E. and Tognotti, L. and Tonazzini, A.},
  Journal                  = {Computers \& Chemical Engineering},
  Year                     = {1990},
  Number                   = {11},
  Pages                    = {1201-1211},
  Volume                   = {14},

  Doi                      = {10.1016/0098-1354(90)80002-s},
  ISSN                     = {0098-1354},
  Type                     = {Journal Article},
  Url                      = {http://www.sciencedirect.com/science/article/pii/009813549080002S}
}

@Article{Farina2008,
  Title                    = {Advancing the science of in vitro testing and laboratory data management for nasal sprays},
  Author                   = {Farina, D.},
  Journal                  = {Drug Delivery Technology},
  Year                     = {2008},
  Number                   = {4},
  Pages                    = {1},
  Volume                   = {4},

  Type                     = {Journal Article}
}

@Article{Farkas2007,
  Title                    = {Simulation of the effect of local obstructions and blockage on airflow and aerosol deposition in central human airways},
  Author                   = {Farkas, A. and Balásházy, I.},
  Journal                  = {Aerosol Science},
  Year                     = {2007},
  Pages                    = {865-884},
  Volume                   = {38},

  Type                     = {Journal Article}
}

@Article{Farkas2008,
  Title                    = {Quantification of particle deposition in asymmetrical tracheobronchial model geometry},
  Author                   = {Farkas, �rpád and Balásházy, Imre},
  Journal                  = {Computers in Biology and Medicine},
  Year                     = {2008},
  Number                   = {4},
  Pages                    = {508-518},
  Volume                   = {38},

  Abstract                 = {The primary objective of this study was to quantify the local inspiratory and expiratory aerosol deposition in a highly asymmetric five-generation tracheobronchial tree. User-enhanced commercial codes and self-developed software was used to compute the air velocity field as well as particle deposition distributions for a large size range of inhalable particles. The numerical model was validated by comparison of our results with experimental flow measurements and particle deposition data available in the open literature. Our simulations show highly localised deposition patterns for all particle sizes, but mainly for the larger particles. As expected, deposition efficiencies and deposition fractions proved to be very sensitive to the particle size. The deposition density in the hot spots can be hundreds and even thousand times higher than the mean deposition density. Present results can be of interest to researchers involved in the assessment of adverse health effects of inhaled aerosols or optimising the drug aerosol delivery into the lung.},
  ISSN                     = {0010-4825},
  Keywords                 = {Computational fluid and particle dynamics
Central airways
Airflow
Particle deposition patterns
Deposition efficiency
Deposition fraction
Deposition enhancement factor},
  Type                     = {Journal Article},
  Url                      = {http://www.sciencedirect.com/science/article/B6T5N-4S1JK4H-2/2/db0b336af91ffb6670584862bef5e9ca}
}

@Article{Farkas2007a,
  Title                    = {Simulation of the effect of local obstructions and blockage on airflow and aerosol deposition in central human airways},
  Author                   = {Farkas, �rpád and Balásházy, Imre},
  Journal                  = {Journal of Aerosol Science},
  Year                     = {2007},
  Number                   = {8},
  Pages                    = {865-884},
  Volume                   = {38},

  Abstract                 = {Investigation of the effect of sidewall and carinal tumours, airway constrictions and airway blockage on the inspiratory airflow and particle deposition in the large central human airways was the primary objective of this study. A computational fluid and particle dynamics model was implemented, validated and applied in order to simulate the air and particle transport and to quantify the aerosol deposition in double airway bifurcation models. Our investigations revealed that surface abnormalities and tubular constrictions can significantly alter the airstreams and the related local aerosol deposition distributions. Sidewall tumours have lead to an enhanced deposition of large particles and caused lower deposition efficiency values of nano-particles compared to the deposition efficiency in healthy airways. Central tumours multiplied the deposition efficiency of large particles but hardly affected the deposition efficiency of nano-particles. Airway blockage caused a significant redistribution of particle deposition sites. The deposition efficiency of the inhaled aerosols in constricted airways was much higher than the same deposition efficiency in healthy airways. Current results might help in the understanding of the adverse health effects of the inhaled air-pollutants in patients with lung disease and might be integrated into future aerosol therapy protocols.},
  ISSN                     = {0021-8502},
  Keywords                 = {Airflow and particle deposition
Computational fluid and particle dynamics
Diseased airways},
  Type                     = {Journal Article},
  Url                      = {http://www.sciencedirect.com/science/article/B6V6B-4P0DJTM-1/2/d4189dfc4b3cbde75d06aab943f88089}
}

@Article{FarrA©2006,
  Title                    = {Assessment of airway function during sleep},
  Author                   = {Farré, R.},
  Journal                  = {Journal of Biomechanics},
  Year                     = {2006},
  Number                   = {Supplement 1},
  Pages                    = {S270-S270},
  Volume                   = {39},

  ISSN                     = {0021-9290},
  Type                     = {Journal Article},
  Url                      = {http://www.sciencedirect.com/science/article/B6T82-4KR88PB-1FT/2/67a95d3553d5a0330be814a5e981adf8}
}

@Article{Faxen1923,
  Title                    = {Die Bewegung einer starren Kugel l¨angs der Achse eines mit z¨aher Fl¨ussigkeit gef¨ullten Rohres. Arkiv},
  Author                   = {Faxen, H. },
  Journal                  = {Arkiv. Mat. Astron. Fys},
  Year                     = {1923},
  Number                   = {17},
  Volume                   = {17},

  Type                     = {Journal Article}
}

@Article{Feih2012,
  Title                    = {Tensile properties of carbon fibres and carbon fibre–polymer composites in fire},
  Author                   = {Feih, S. and Mouritz, A. P.},
  Journal                  = {Composites Part A: Applied Science and Manufacturing},
  Year                     = {2012},
  Number                   = {5},
  Pages                    = {765-772},
  Volume                   = {43},

  Doi                      = {http://dx.doi.org/10.1016/j.compositesa.2011.06.016},
  ISSN                     = {1359-835X},
  Keywords                 = {A. Carbon fibre
A. Polymer–matrix composites (PMCs)
B. High-temperature properties
B. Thermomechanical},
  Type                     = {Journal Article},
  Url                      = {http://www.sciencedirect.com/science/article/pii/S1359835X11002053}
}

@Article{Fenske1986,
  Title                    = {Structural and functional changes of normal aging skin},
  Author                   = {Fenske, Neil A. and Lober, Clifford W.},
  Journal                  = {Journal of the American Academy of Dermatology},
  Year                     = {1986},
  Number                   = {4, Part 1},
  Pages                    = {571-585},
  Volume                   = {15},

  Abstract                 = {Solar-induced cutaneous changes are more prevalent and profound in older persons and, thus, are often inappropriately attributed to the aging process, per se. Structural and functional alterations caused by intrinsic aging and independent of environmental insults are now recognized in the skin of elderly individuals. Structurally the aged epidermis likely becomes thinner, the corneocytes become less adherent to one another, and there is flattening of the dermoepidermal interface. The number of melanocytes and Langerhans cells is decreased. The dermis becomes atrophic and it is relatively acellular and avascular. Dermal collagen, elastin, and glycosaminoglycans are altered. The subcutaneous tissue is diminished in some areas, especially the face, shins, hands, and feet, while in others, particularly the abdomen in men and the thighs in women, it is increased. The number of eccrine glands is reduced and both the eccrine and apocrine glands undergo attenuation. Sebaceous glands tend to increase in size but paradoxically their secretory output is lessened, The nail plate is generally thinned, the surface ridged and lusterless, and the lunula decreased in size. There is a progressive reduction in the density of hair follicles per unit area on the face and scalp, independent of male-pattern alopecia. The hair shaft diameter is generally reduced but in some areas, especially the ears, nose, and eyebrows of men and the upper lip and chin in women, it is increased as vellus hairs convert to cosmetically compromising terminal hairs. Functional alterations noted in the skin of elderly persons include a decreased growth rate of the epidermis, hair, and nails, delayed wound healing, reduced dermal clearance of fluids and foreign materials, and compromised vascular responsiveness. Eccrine and apocrine secretions are diminished. The cutaneous immune and inflammatory responses are impaired, particularly cell-mediated immunity. Clinical correlates of these intrinsic aging changes of the skin include alopecia, pallor, xerosis, an increased number of benign and malignant epidermal neoplasms, increased susceptibility to blister formation, predisposition to injury of the dermis and underlying tissues, delayed onset and resolution of blisters and wheals, persistent contact dermatitis, impaired tanning response to ultraviolet light, increased risk for wound infections, prolongation of therapy necessary for onychomycosis, and thermoregulatory disturbances.},
  Doi                      = {http://dx.doi.org/10.1016/S0190-9622(86)70208-9},
  ISSN                     = {0190-9622},
  Type                     = {Journal Article},
  Url                      = {http://www.sciencedirect.com/science/article/pii/S0190962286702089}
}

@Article{Ferron1991,
  Title                    = {Airflow simulation in two-dimensional bifurcations},
  Author                   = {Ferron, G. A. and Hillebrecht, A. and Peter, J. and Priesack, E. and Thoma, M. and Künzer, I. and Mederer, R. and Klump, U. G.},
  Journal                  = {Journal of Aerosol Science},
  Year                     = {1991},
  Number                   = {Supplement 1},
  Pages                    = {S809-S812},
  Volume                   = {22},

  Abstract                 = {This contribution describes the tested possibilities of a computational fluid dynamics(CFD) software package (FIDAP) to determine airflows in combination with heat and water vapour transport in two dimensional bifurcations simulating human lung bifurcations.},
  ISSN                     = {0021-8502},
  Keywords                 = {Deposition and clearance
particle inhalation},
  Type                     = {Journal Article},
  Url                      = {http://www.sciencedirect.com/science/article/B6V6B-4FXXR3S-7B/2/2ac525801b1e40d55dc8c6d18b8535ae}
}

@Book{Ferziger1999,
  Title                    = {Computational Methods for Fluid Dynamics},
  Author                   = {Ferziger, J. H. and Perić, M.},
  Publisher                = {Springer-Verlag},
  Year                     = {1999},

  Address                  = {Belrin},

  Type                     = {Book}
}

@Article{Fichman1988,
  Title                    = {A model for turbulent deposition of aerosols},
  Author                   = {Fichman, M. and Gutfinger, C. and Pnueli, D.},
  Journal                  = {Journal of Aerosol Science},
  Year                     = {1988},
  Number                   = {1},
  Pages                    = {123-136},
  Volume                   = {19},

  ISSN                     = {0021-8502},
  Type                     = {Journal Article},
  Url                      = {http://www.sciencedirect.com/science/article/pii/0021850288902613}
}

@Article{Fiedler1990,
  Title                    = {On management and control of turbulent shear flows},
  Author                   = {Fiedler, H. E. and Fernholz, H. H.},
  Journal                  = {Progress in Aerospace Sciences},
  Year                     = {1990},
  Number                   = {4},
  Pages                    = {305-387},
  Volume                   = {27},

  Abstract                 = {Concepts of turbulent flow control have become of growing importance during the last few years, following increased interest in the detailed structural scenario of turbulence--in particular our improved understanding of coherent structures on the one hand (the prerequisite), and a need for improvement of technological processes on the other (the goal). These considerations have mainly been followed by engineers and physicists concerned with problems in aerodynamics. It is our aim to draw the attention of a wider group of engineers to turbulent flow control in order to speed up the transfer of knowledge from aerodynamics to applications in other fields of engineering. In this paper an attempt is made to compile a major body of the available knowledge on flow control in separated and wall bounded turbulent flows. After a brief introduction of the basics of control theory (Section 2) and of the major flow structures and their stability characteristics (Section 3) free and wall bounded turbulent shear flows are discussed (Sections 4 and 5). This discussion summarizes the main relationships between structure and flow behaviour and shows possibilities of influencing properties of these flows such as increasing mixing or avoiding separation.},
  ISSN                     = {0376-0421},
  Type                     = {Journal Article},
  Url                      = {http://www.sciencedirect.com/science/article/B6V3V-47YRHS9-17/2/ca065a793133a09f55ab9a6657eb16d9}
}

@Article{Finck2007,
  Title                    = {Simulation of nasal flow by lattice Boltzmann methods},
  Author                   = {Finck, M. and Hänel, D. and Wlokas, I.},
  Journal                  = {Computers in Biology and Medicine},
  Year                     = {2007},
  Number                   = {6},
  Pages                    = {739-749},
  Volume                   = {37},

  Abstract                 = {The lattice Boltzmann method is used to calculate the incompressible, viscous flow of air through a model of a nasal cavity, used in experiments. Computations are performed for steady flows at the inspiration and expiration phase of nose breathing. Computed pressure distributions and friction coefficients compare well with Navier-Stokes solutions from a finite-volume method on structured, curvilinear grids. The comparison with conventional Navier-Stokes solvers shows several advantages of the lattice Boltzmann method in particular for bio-medical flow problems. These are the fast grid generation, the simple, granular algorithm, suited for efficient parallelization and the high flexibility for implementing complex boundary conditions and additional transport equations. Lattice Boltzmann methods are therefore efficient candidates for fast flow predictions in the frame of computer-aided rhino-surgery.},
  ISSN                     = {0010-4825},
  Keywords                 = {Nasal flow simulation
Nasal cavity
Computer-aided rhino-surgery
Lattice Boltzmann methods
Cartesian-like grids
Grid refinement
Parallelization
Comparisons with finite-volume method
Pressure distributions
Pressure losses},
  Type                     = {Journal Article},
  Url                      = {http://www.sciencedirect.com/science/article/B6T5N-4KV3Y1H-1/2/06b5cb62fac4feb8c26e7949b79072bb}
}

@Article{Finck2006,
  Title                    = {Simulation of nasal flow by lattice Boltzmann methods},
  Author                   = {Finck, M. and Hänel, D. and Wlokas, I.},
  Journal                  = {Computers Biology Medicine},
  Year                     = {2006},
  Number                   = {6},
  Pages                    = {739-749},
  Volume                   = {37},

  Type                     = {Journal Article}
}

@Article{Finlay1998,
  Title                    = {Estimating the type of hygroscopic behavior exhibited by aqueous droplets},
  Author                   = {Finlay, W. H.},
  Journal                  = {Journal of Aerosol Medicine},
  Year                     = {1998},
  Number                   = {4},
  Pages                    = {221-229},
  Volume                   = {11},

  Type                     = {Journal Article}
}

@Article{Fleming2011,
  Title                    = {Normal ranges of heart rate and respiratory rate in children from birth to 18 years of age: a systematic review of observational studies},
  Author                   = {Fleming, Susannah and Thompson, Matthew and Stevens, Richard and Heneghan, Carl and Plüddemann, Annette and Maconochie, Ian and Tarassenko, Lionel and Mant, David},
  Journal                  = {The Lancet},
  Year                     = {2011},
  Note                     = {C:\Users\sean\AppData\Roaming\Zotero\Zotero\Profiles\16a4oype.default\zotero\storage\3M54H748\Normal ranges of heart rate and respiratory rate in children.pdf},
  Pages                    = {1011–1018},
  Volume                   = {377},

  Type                     = {Journal Article}
}

@Article{Flemmer1986,
  Title                    = {On the drag coefficient of a sphere},
  Author                   = {Flemmer, R.L.C. and Banks, C.L.},
  Journal                  = {Powder Technology},
  Year                     = {1986},
  Pages                    = {217-221},
  Volume                   = {48},

  Type                     = {Journal Article}
}

@Book{Fletcher1991,
  Title                    = {Computational Techniques for Fluid Dynamics},
  Author                   = {Fletcher, C. A. J.},
  Publisher                = {Springer-Verlag},
  Year                     = {1991},

  Address                  = { Berlin},
  Volume                   = {I and II},

  Type                     = {Book}
}

@Article{Flower1950,
  Title                    = {Empirical formulae for the terminal velocity of water drops falling through the atmosphere},
  Author                   = {Flower, W.D.},
  Journal                  = {Quarterly Journal of the Royal Meteorological Society},
  Year                     = {1950},
  Number                   = {329},
  Pages                    = {302},
  Volume                   = {76},

  Type                     = {Journal Article},
  Url                      = {http://adsabs.harvard.edu/cgi-bin/bib_query?bibcode=1950QJRMS..76..302B}
}

@Article{Flower1927,
  Title                    = {The terminal velocity of drops},
  Author                   = {Flower, W. D.},
  Journal                  = {Proceedings of the Physical Society},
  Year                     = {1927},
  Number                   = {1},
  Pages                    = {167},
  Volume                   = {40},

  Abstract                 = {The distance a drop of given volume has to fall through in order that it may attain a constant terminal velocity has been determined for both water and methyl salicylate. A relation connecting the distance of fall for constant velocity and the volume of the drop has been obtained. The terminal velocities of drops of water and of methyl salicylate have been determined for drops 0.2 cm. to 0.55 cm. in diameter. A relation connecting the terminal velocity of a drop and its volume has been obtained.},
  ISSN                     = {0959-5309},
  Type                     = {Journal Article},
  Url                      = {http://stacks.iop.org/0959-5309/40/i=1/a=322}
}

@Book{Fodil2005,
  Title                    = {Inspiratory flow in the nose: a model coupling flow and vasoerectile tissue distensibility},
  Author                   = {Fodil, Redouane and Brugel-Ribere, Lydia and Croce, Céline and Sbirlea-Apiou, Gabriela and Larger, Christian and Papon, Jean-François and Delclaux, Christophe and Coste, André and Isabey, Daniel and Louis, Bruno},
  Year                     = {2005},
  Volume                   = {98},

  Abstract                 = {Redouane Fodil1, Lydia Brugel-Ribere1,2, Céline Croce1, Gabriela Sbirlea-Apiou3, Christian Larger4, Jean-François Papon1,2, Christophe Delclaux1,4, André Coste1,2, Daniel Isabey1, and Bruno Louis11Physiopathologie et Thérapeutique Respiratoires INSERM UMR 492, and 2Service d'ORL et de Chirurgie Cervico-Faciale, Centre Hospitalier Inter-Communal de Créteil, Créteil; 3Centre de Recherche Claude Delorme, Air Liquide, Jouy en Josas; and 4Service de Physiologie-Explorations Fonctionnelles, Hôpital Henri Mondor Assistance Publique Hôpitaux de Paris, Créteil, FranceAddress for reprint requests and other correspondence: B. Louis, Inserm U492, Faculté de Médecine, 8 Rue du Général Sarrail, 94010 Créteil Cedex, France (E-mail: bruno.louis{at}creteil.inserm.fr)Submitted 18 June 2004.Accepted 26 August 2004.Abstract We have developed a discrete multisegmental model describing the coupling between inspiratory flow and nasal wall distensibility. This model is composed of 14 individualized compliant elements, each with its own relationship between cross-sectional area and transmural pressure. Conceptually, this model is based on flow limitation induced by the narrowing of duct due to collapsing pressure. For a given inspiratory pressure and for a given compliance distribution, this model predicts the area profile and inspiratory flow. Acoustic rhinometry and posterior rhinomanometry were used to determine the initial geometric area and mechanical characteristics of each element. The proposed model, used under steady-state conditions, is able to simulate the pressure-flow relationship observed in vivo under normal conditions (4 subjects) and under pathological conditions (4 vasomotor rhinitis and 3 valve syndrome subjects). Our results suggest that nasal wall compliance is an essential parameter to understand the nasal inspiratory flow limitation phenomenon and the associated increase of resistance that is well known to physiologists. By predicting the functional pressure-flow relationship, this model could be a useful tool for the clinician to evaluate the potential effects of treatments. acoustic rhinometrynasal physiologynasal wall compliancefluid-structure couplingCopyright © 2005 the American Physiological Society},
  Doi                      = {10.1152/japplphysiol.00625.2004},
  Pages                    = {288-295},
  Type                     = {Book},
  Url                      = {http://jap.physiology.org/jap/98/1/288.full.pdf}
}

@Article{Fodil2005a,
  Title                    = {Inspiratory flow in the nose: a model coupling flow and vasoerectile tissue distensibility},
  Author                   = {Fodil, Redouane and Brugel-Ribere, Lydia and Croce, Céline and Sbirlea-Apiou, Gabriela and Larger, Christian and Papon, Jean-François and Delclaux, Christophe and Coste, André and Isabey, Daniel and Louis, Bruno},
  Year                     = {2005},
  Pages                    = {288-295},
  Volume                   = {98},

  Abstract                 = {Redouane Fodil1, Lydia Brugel-Ribere1,2, Céline Croce1, Gabriela Sbirlea-Apiou3, Christian Larger4, Jean-François Papon1,2, Christophe Delclaux1,4, André Coste1,2, Daniel Isabey1, and Bruno Louis11Physiopathologie et Thérapeutique Respiratoires INSERM UMR 492, and 2Service d'ORL et de Chirurgie Cervico-Faciale, Centre Hospitalier Inter-Communal de Créteil, Créteil; 3Centre de Recherche Claude Delorme, Air Liquide, Jouy en Josas; and 4Service de Physiologie-Explorations Fonctionnelles, Hôpital Henri Mondor Assistance Publique Hôpitaux de Paris, Créteil, FranceAddress for reprint requests and other correspondence: B. Louis, Inserm U492, Faculté de Médecine, 8 Rue du Général Sarrail, 94010 Créteil Cedex, France (E-mail: bruno.louisatcreteil.inserm.fr)Submitted 18 June 2004.Accepted 26 August 2004.Abstract We have developed a discrete multisegmental model describing the coupling between inspiratory flow and nasal wall distensibility. This model is composed of 14 individualized compliant elements, each with its own relationship between cross-sectional area and transmural pressure. Conceptually, this model is based on flow limitation induced by the narrowing of duct due to collapsing pressure. For a given inspiratory pressure and for a given compliance distribution, this model predicts the area profile and inspiratory flow. Acoustic rhinometry and posterior rhinomanometry were used to determine the initial geometric area and mechanical characteristics of each element. The proposed model, used under steady-state conditions, is able to simulate the pressure-flow relationship observed in vivo under normal conditions (4 subjects) and under pathological conditions (4 vasomotor rhinitis and 3 valve syndrome subjects). Our results suggest that nasal wall compliance is an essential parameter to understand the nasal inspiratory flow limitation phenomenon and the associated increase of resistance that is well known to physiologists. By predicting the functional pressure-flow relationship, this model could be a useful tool for the clinician to evaluate the potential effects of treatments. acoustic rhinometrynasal physiologynasal wall compliancefluid-structure couplingCopyright © 2005 the American Physiological Society},
  Doi                      = {10.1152/japplphysiol.00625.2004},
  Type                     = {Journal Article},
  Url                      = {http://jap.physiology.org/content/jap/98/1/288.full.pdf}
}

@Article{Fodil2006,
  Title                    = {Mechanisms of inspiratory and expiratory flow limitations during obstructive sleep apnoea},
  Author                   = {Fodil, R. and Croce, C. and Lofaso, F. and Louis, B. and Isabey, D.},
  Journal                  = {Journal of Biomechanics},
  Year                     = {2006},
  Number                   = {Supplement 1},
  Pages                    = {S442-S442},
  Volume                   = {39},

  ISSN                     = {0021-9290},
  Type                     = {Journal Article},
  Url                      = {http://www.sciencedirect.com/science/article/B6T82-4KR88PB-2F7/2/00cb58530c743b4a1ea08eb80860195b}
}

@Article{Foo2007,
  Title                    = {The Influence of Spray Properties on Intranasal Deposition},
  Author                   = {Foo, M. and Cheng, Y. and Su, W. and Donovan, M.},
  Journal                  = {Journal of Aerosol Medicine},
  Year                     = {2007},
  Number                   = {4 (December 1)},
  Pages                    = {495-508},
  Volume                   = {20},

  Type                     = {Journal Article}
}

@Article{Franciscus1991,
  Title                    = {Variation in human nasal height and breadth},
  Author                   = {Franciscus, R.G. and Long, J.C.},
  Journal                  = {Am J Phys Anthropol},
  Year                     = {1991},
  Pages                    = {419-427},
  Volume                   = {85},

  Type                     = {Journal Article}
}

@Article{Frank2012,
  Title                    = {Computed intranasal spray penetration: comparisons before and after nasal surgery},
  Author                   = {Frank, D. O. and Kimbell, J. S. and Cannon, D. and Rhee, J. S.},
  Journal                  = {Int Forum Allergy Rhinol},
  Year                     = {2012},
  Note                     = {Frank, Dennis O
Kimbell, Julia S
Cannon, Daniel
Rhee, John S
R01 EB009557/EB/NIBIB NIH HHS/United States
International forum of allergy \& rhinology
Int Forum Allergy Rhinol. 2012 Aug 27. doi: 10.1002/alr.21070.},

  Abstract                 = {BACKGROUND: Quantitative methods for comparing intranasal drug delivery efficiencies pre- and postoperatively have not been fully utilized. The objective of this study is to use computational fluid dynamics techniques to evaluate aqueous nasal spray penetration efficiencies before and after surgical correction of intranasal anatomic deformities. METHODS: Ten three-dimensional models of the nasal cavities were created from pre- and postoperative computed tomography scans in 5 subjects. Spray simulations were conducted using a particle size distribution ranging from 10 mum to 110 mum, a spray speed of 3 m/second, plume angle of 68 degrees, and with steady state, resting inspiratory airflow present. Two different nozzle positions were compared. Statistical analysis was conducted using Student t test for matched pairs. RESULTS: On the obstructed side, posterior particle deposition after surgery increased by 118% and was statistically significant (p = 0.036), while anterior particle deposition decreased by 13% and was also statistically significant (p = 0.020). The fraction of particles that bypassed the airways either pre- or postoperatively was less than 5%. Posterior particle deposition differences between obstructed and contralateral sides of the airways were 113% and 30% for pre- and postsurgery, respectively. Results showed that nozzle positions can influence spray delivery. CONCLUSION: Simulations predicted that surgical correction of nasal anatomic deformities can improve spray penetration to areas where medications can have greater effect. Particle deposition patterns between both sides of the airways are more evenly distributed after surgery. These findings suggest that correcting anatomic deformities may improve intranasal medication delivery. For enhanced particle penetration, patients with nasal deformities may explore different nozzle positions. (c) 2012 ARS-AAOA, LLC.},
  Doi                      = {10.1002/alr.21070},
  ISSN                     = {2042-6984 (Electronic)},
  Type                     = {Journal Article}
}

@Article{Frank2012a,
  Title                    = {Effects of Anatomy and Particle Size on Nasal Sprays and Nebulizers},
  Author                   = {Frank, Dennis O. and Kimbell, Julia S. and Pawar, Sachin and Rhee, John S.},
  Journal                  = {Otolaryngology -- Head and Neck Surgery},
  Year                     = {2012},
  Number                   = {2},
  Pages                    = {313-319},
  Volume                   = {146},

  Abstract                 = {Objective. To study the effects of nasal deformity on aerosol penetration past the nasal valve (NV) for varying particle sizes using sprays or nebulizers. Study Design. Computed mathematical nasal airway model. Setting. Department computer lab. Subjects and Methods. Particle deposition was analyzed using a computational fluid dynamics model of the human nose with leftward septal deviation and compensatory right inferior turbinate hypertrophy. Sprays were simulated for 10 µm, 20 µm, 50 µm, or particle sizes following a Rosin Rammler particle size distribution (10-110 µm), at speeds of 1, 3, or 10 meters per second. Nebulization was simulated for 1, 3.2, 6.42, or 10 µm particles. Steady state inspiratory airflow was simulated at 15.7 liters per minute. Results. Sprays predicted higher NV penetration on the right side for particle sizes >10 µm, with comparable penetration on both sides at 10 µm. Nearly 100% deposited in the nasal passages for all spray characteristics. Nebulizer predictions showed nearly 100% of particles <6.42 µm and more than 50% of 6.42 µm bypassing both sides of the nose without depositing. Of the nebulized particles that deposited, penetration was higher on the right at 10 µm, with comparable penetration on both sides at 6.42 µm. Spray penetration was highest at 10 µm, with more than 96% penetrating on both sides at 1 and 3 meters per second. Nebulization penetration was also highest at 10 µm (40% on the left, >90% on the right). Conclusion. In the presence of a septal deviation, sprays or nebulizers containing 10-µm particles may have good penetration beyond the NV. Nebulized particles <10 µm are likely to be respirable. Additionally, spray speeds above 3 meters per second may limit penetration.},
  Doi                      = {10.1177/0194599811427519},
  Type                     = {Journal Article},
  Url                      = {http://oto.sagepub.com/content/146/2/313.abstract}
}

@Article{Franks2005,
  Title                    = {A mathematical model for the absorption and metabolism of formaldehyde vapour by humans},
  Author                   = {Franks, S. J.},
  Journal                  = {Toxicology and Applied Pharmacology},
  Year                     = {2005},
  Number                   = {3},
  Pages                    = {309-320},
  Volume                   = {206},

  Abstract                 = {Epidemiological studies of occupational exposure to formaldehyde gas (HCHO) have suggested possible links between concentration and duration of exposure, and elevated risks of leukaemia and other cancers at sites distant from the site of contact. Formaldehyde is a highly water soluble gas which, when inhaled, reacts rapidly at the site of contact and is quickly metabolised by enzymes in the respiratory tissue. Inhaled formaldehyde is almost entirely absorbed in the respiratory tract and, for formaldehyde induced toxicity to occur at distant sites, HCHO must enter the blood and be transported to systemic tissues via the circulatory system. A mathematical model describing the absorption and removal of inhaled formaldehyde in the nasal tissue is therefore formulated to predict the proportion of formaldehyde entering into the blood. Accounting for the spatial distribution of the formaldehyde concentration and the metabolic activity within the mucosa, the concentration of formaldehyde in the mucus, the epithelium and the blood has been determined and was found to attain a steady-state profile within a few seconds of exposure. The increase of the formaldehyde concentration in the blood was predicted to be insignificant compared with the existing pre-exposure levels in the body, indicating that formaldehyde is rapidly removed in the nasal tissue. The results of the model thus suggest that it is highly unlikely that following inhalation by the nose, formaldehyde itself will cause toxicity at sites other than the initial site of contact in the respiratory tract.},
  ISSN                     = {0041-008X},
  Keywords                 = {Formaldehyde vapour
Mathematical modelling
Nasal tissue
Toxicity
Circulatory system},
  Type                     = {Journal Article},
  Url                      = {http://www.sciencedirect.com/science/article/B6WXH-4F7B42G-2/2/ba7759ec1531c484efee5b300e5d0946}
}

@Article{FranzChouly2009,
  Title                    = {Modelling the human pharyngeal airway: validation of numerical simulations using in vitro experiments},
  Author                   = {Franz Chouly, Annemie Van Hirtum, Pierre-Yves Lagre´e, Xavier Pelorson,Yohan Payan},
  Journal                  = {Med Biol Eng Comput},
  Year                     = {2009},
  Pages                    = {9},
  Volume                   = {47},

  Type                     = {Journal Article}
}

@Book{Fraser1977,
  Title                    = {Organ physiology: Structure and function of the lung},
  Author                   = {Fraser, R.G. and Pare, J.A.P},
  Publisher                = {W.G.Saunders},
  Year                     = {1977},

  Address                  = {Philadelphia},
  Edition                  = {2nd},

  Type                     = {Book}
}

@Article{Frasnelli2003,
  Title                    = {Age-related decline of intranasal trigeminal sensitivity: is it a peripheral event? },
  Author                   = {Johannes Frasnelli and Thomas Hummel},
  Journal                  = {Brain Research },
  Year                     = {2003},
  Number                   = {2},
  Pages                    = {201 - 206},
  Volume                   = {987},

  Abstract                 = {Compared to younger subjects, older people have a reduced sensitivity of the intranasal trigeminal system which responds to irritation of the nasal cavity. It is unclear whether the cause of this difference relates to age-dependent changes in the periphery of the system. The aim of the present study was the comparison of intranasal trigeminal thresholds assessed through electrophysiological measurements in eight young (four women, four men; mean age 25 years) and eight older subjects (four women, four men; mean age 62 years). The negative mucosa potential (NMP), a peripheral correlate of trigeminal activation, was recorded from the nasal mucosa in response to stimulation with varying concentrations of the mixed olfactory/trigeminal stimulants menthol and linalool. Thresholds were estimated as the strongest concentration which did not elicit a \{NMP\} response. Older subjects were found to have higher thresholds for menthol when compared to younger subjects. Furthermore, an explorative analysis indicated that the increase of response amplitudes to increasing stimulus concentrations was shallower in older subjects. These findings indicate that age related loss of intranasal trigeminal sensitivity seems to take place, at least to some degree, in the periphery of the intranasal trigeminal system. },
  Doi                      = {http://dx.doi.org/10.1016/S0006-8993(03)03336-5},
  ISSN                     = {0006-8993},
  Keywords                 = {Age},
  Url                      = {http://www.sciencedirect.com/science/article/pii/S0006899303033365}
}

@Article{Frauenfelder2005,
  Title                    = {Pulmonary spread of recurrent respiratory papillomatosis with malignant transformation: CT-findings and airflow simulation},
  Author                   = {Frauenfelder, Thomas and Marincek, Borut and Wildermuth, Simon},
  Journal                  = {European Journal of Radiology Extra},
  Year                     = {2005},
  Number                   = {1},
  Pages                    = {11-16},
  Volume                   = {56},

  Abstract                 = {Malignant transformation of pulmonary lesions of recurrent respiratory papillomatosis is a rare finding. Only a few cases are described in the radiological literature. This case report presents the radiological findings of the disease and for the first time introduces computational fluid dynamics (CFD), to describe the flow pattern of the airflow inside the scarred tracheobronchial tree. CFD simulates fluids in different kind of geometric models based on numerical calculations. It is well known in aerospace and automobile industries, but can also be adapted to simulate the airflow in anatomical correct models of the tracheobronchial tree.},
  ISSN                     = {1571-4675},
  Keywords                 = {Papillomas
Lung
Squamous cell carcinoma
Computed tomography
3D imaging
Infinite element analysis},
  Type                     = {Journal Article},
  Url                      = {http://www.sciencedirect.com/science/article/B7583-4GX0CFF-1/2/f846c7c86e6dd3b1a4ae08ae6103137f}
}

@Article{Frederick2002,
  Title                    = {Use of a Hybrid Computational Fluid Dynamics and Physiologically Based Inhalation Model for Interspecies Dosimetry Comparisons of Ester Vapors},
  Author                   = {Frederick, Clay B. and Lomax, Larry G. and Black, Kurt A. and Finch, Lavorgie and Scribner, Harvey E. and Kimbell, Julia S. and Morgan, Kevin T. and Subramaniam, Ravi P. and Morris, John B.},
  Journal                  = {Toxicology and Applied Pharmacology},
  Year                     = {2002},
  Number                   = {1},
  Pages                    = {23-40},
  Volume                   = {183},

  Abstract                 = {Numerous inhalation studies have demonstrated that exposure to high concentrations of a wide range of volatile acids and esters results in cytotoxicity to the nasal olfactory epithelium. Previously, a hybrid computational fluid dynamics (CFD) and physiologically based pharmacokinetic (PBPK) dosimetry model was constructed to estimate the regional tissue dose of organic acids in the rodent and human nasal cavity. This study extends this methodology to a representative volatile organic ester, ethyl acrylate (EA). An in vitro exposure of explants of rat olfactory epithelium to EA with and without an esterase inhibitor demonstrated that the organic acid, acrylic acid, released by nasal esterases is primarily responsible for the olfactory cytotoxicity. Estimates of the steady-state concentration of acrylic acid in olfactory tissue were made for the rat nasal cavity by using data from a series of short-term in vivo studies and from the results of CFD-PBPK computer modeling. Appropriate parameterization of the CFD-PBPK model for the human nasal cavity and to accommodate human systemic anatomy, metabolism, and physiology allowed interspecies dose comparisons. The CFD-PBPK model simulations indicate that the olfactory epithelium of the human nasal cavity is exposed to at least 18-fold lower tissue concentrations of acid released from EA than the olfactory epithelium of the rat nasal cavity under the same exposure conditions. The magnitude of this difference varies with the specific exposure scenario that is simulated and with the specific dataset of human esterase activity used for the simulations. The increased olfactory tissue dose in rats relative to humans may be attributed to both the vulnerable location of the rodent olfactory tissue (comprising greater than 50% of the nasal cavity) and the high concentration of rat olfactory esterase activity (comparable to liver esterase activity) relative to human olfactory tissue. These studies suggest that the human olfactory epithelium is protected from vapors of organic esters significantly better than rat olfactory epithelium due to substantive differences in nasal anatomy, nasal and systemic metabolism, systemic physiology, and air flow. Although the accumulation of acrylic acid in the nasal tissues may be a primary concern for nasal irritation and human risk assessment, acute animal inhalation studies to evaluate lethality (LD50-type studies) conducted at very high vapor concentrations of ethyl acrylate indicated that a different mechanism is primarily responsible for mortality. The rodent studies demonstrated that systemic tissue nonprotein sulfhydryl depletion is a primary cause of death at exposure concentrations more than two orders of magnitude above the concentrations that induce nasal irritation. The CFD-PBPK model adequately simulated the severe depletion of glutathione in systemic tissues (e.g., liver and lung) associated with acute inhalation exposures in the 500-1000 ppm range. These results indicate that the CFD-PBPK model can simulate both the low-dose nasal tissue dosimetry associated with irritation and the high-dose systemic tissue dosimetry associated with mortality. In addition, the comparison of simulation results for ethyl acetate and acetone to nasal deposition data suggests that the CFD-PBPK model has general utility as a tool for dosimetry estimates for a wide range of other esters and slowly metabolized vapors.},
  ISSN                     = {0041-008X},
  Keywords                 = {physiologically based pharmacokinetic (PBPK)
computational fluid dynamics (CFD)
ethyl acrylate
acrylic acid
nasal cavity
olfactory
esterase
interspecies extrapolation
risk assessment},
  Type                     = {Journal Article},
  Url                      = {http://www.sciencedirect.com/science/article/B6WXH-46P4164-3/2/c3acad1e771b6401ef8815432bdafd26}
}

@Article{Freitag1989,
  Title                    = {Removal of excessive bronchial secretions by asymmetric high-frequency oscillations},
  Author                   = {Freitag, L. and Long, W. M. and Kim, C. S. and Wanner, A.},
  Journal                  = {Journal of Applied Physiology},
  Year                     = {1989},
  Number                   = {2},
  Pages                    = {614-619},
  Volume                   = {67},

  Type                     = {Journal Article},
  Url                      = {http://jap.physiology.org/cgi/content/abstract/67/2/614}
}

@Article{Freitas2008,
  Title                    = {Numerical investigation of the three-dimensional flow in a human lung model},
  Author                   = {Freitas, Rainhill K. and Schröder, Wolfgang},
  Journal                  = {Journal of Biomechanics},
  Year                     = {2008},
  Note                     = {doi: DOI: 10.1016/j.jbiomech.2008.05.016},
  Number                   = {11},
  Pages                    = {2446-2457},
  Volume                   = {41},

  ISSN                     = {0021-9290},
  Keywords                 = {Lattice Boltzmann
Human lung cast
Human airways
Respiratory flow simulation},
  Type                     = {Journal Article},
  Url                      = {http://www.sciencedirect.com/science/article/B6T82-4SY5X1T-2/2/7633d97b28726fe05b6d4d9d86553a1d}
}

@Book{Friedlander1977,
  Title                    = {Smoke, Dust and Haze - Fundamentals of Aerosol Behaviour},
  Author                   = {Friedlander, S.K. },
  Publisher                = {John Wiley \& Sons},
  Year                     = {1977},

  Address                  = {New York},

  Type                     = {Book}
}

@Article{Friedlander1957,
  Title                    = {Deposition of Suspended Particles from Turbulent Gas Streams},
  Author                   = {Friedlander, S. K. and Johnstone, H. F.},
  Journal                  = {Industrial \& Engineering Chemistry},
  Year                     = {1957},
  Note                     = {doi: 10.1021/ie50571a039},
  Number                   = {7},
  Pages                    = {1151-1156},
  Volume                   = {49},

  Doi                      = {10.1021/ie50571a039},
  ISSN                     = {0019-7866},
  Type                     = {Journal Article},
  Url                      = {http://dx.doi.org/10.1021/ie50571a039}
}

@Article{Frisch1986,
  Title                    = {Lattice-Gas Automata for the Navier-Stokes Equation},
  Author                   = {Frisch, U. and Hasslacher, B. and Pomeau, Y.},
  Journal                  = {Physical Review Letters},
  Year                     = {1986},
  Note                     = {Copyright (C) 2011 The American Physical Society
Please report any problems to prola@aps.org
PRL},
  Number                   = {14},
  Pages                    = {1505},
  Volume                   = {56},

  Type                     = {Journal Article},
  Url                      = {http://link.aps.org/doi/10.1103/PhysRevLett.56.1505}
}

@TechReport{Fritschi2006,
  Title                    = {Occupational Cancer in Australia Report, },
  Author                   = {Fritschi, L.},
  Institution              = {Australia Safety and Compensation Council},
  Year                     = {2006},
  Type                     = {Report}
}

@Article{Fry1973,
  Title                    = {Regional deposition and clearance of particles in the human nose},
  Author                   = {Fry, F.A. and Black, A. },
  Journal                  = {Aerosol Science},
  Year                     = {1973},
  Number                   = {2},
  Pages                    = {113-124},
  Volume                   = {4},

  Type                     = {Journal Article}
}

@Book{Fuchs1964,
  Title                    = {The Mechanics of Aerosols},
  Author                   = {Fuchs, N.A. },
  Publisher                = {Pergamon Press},
  Year                     = {1964 },

  Address                  = {Oxford},

  Type                     = {Book}
}

@Article{Fung2013,
  Title                    = {External characteristics of unsteady spray atomization from a nasal spray device},
  Author                   = {Fung, Man Chiu and Inthavong, Kiao and Yang, William and Lappas, Petros and Tu, Jiyuan},
  Journal                  = {Journal of Pharmaceutical Sciences},
  Year                     = {2013},
  Number                   = {3},
  Pages                    = {1024-35},
  Volume                   = {102},

  Doi                      = {10.1002/jps.23449},
  ISSN                     = {1520-6017},
  Keywords                 = {nasal drug delivery
targeted drug delivery
computer aided drug design
aerosol
high-speed camera
imaging methods
spray atomization},
  Type                     = {Journal Article},
  Url                      = {http://dx.doi.org/10.1002/jps.23449}
}

@Article{Fung2012,
  Title                    = {CFD Modeling of Spray Atomization for a Nasal Spray Device},
  Author                   = {Fung, Man Chiu and Inthavong, Kiao and Yang, William and Tu, Jiyuan},
  Journal                  = {Aerosol Science and Technology},
  Year                     = {2012},
  Number                   = {11},
  Pages                    = {1219-1226},
  Volume                   = {46},

  Doi                      = {10.1080/02786826.2012.704098},
  ISSN                     = {0278-6826},
  Type                     = {Journal Article},
  Url                      = {http://dx.doi.org/10.1080/02786826.2012.704098}
}

@Book{Fung1993,
  Title                    = {Biomechanics, Mechanical Properties of Living Tissues},
  Author                   = {Fung, Y.C. },
  Publisher                = {Springer-Verlag},
  Year                     = {1993},

  Address                  = {Berlin},
  Edition                  = {2nd Edition},

  Type                     = {Book}
}

@Article{Fung1988,
  Title                    = {A model of the lung structure and its validation},
  Author                   = {Fung, Y. C.},
  Journal                  = {J Appl Physiol},
  Year                     = {1988},
  Number                   = {5},
  Pages                    = {2132-2141},
  Volume                   = {64},

  Type                     = {Journal Article},
  Url                      = {http://jap.physiology.org/cgi/content/abstract/64/5/2132}
}

@Article{GA¼nther1998,
  Title                    = {Turbulent flow in a channel at a low Reynolds number},
  Author                   = {Günther, A. and Papavassiliou, D. V. and Warholic, M. D. and Hanratty, T. J.},
  Journal                  = {Experiments in Fluids},
  Year                     = {1998},
  Note                     = {10.1007/s003480050256},
  Number                   = {5},
  Pages                    = {503-511},
  Volume                   = {25},

  Abstract                 = {Abstract&nbsp;&nbsp; A two-component laser Doppler velocimeter with high spatial and temporal resolution was used to obtain measurements for fully developed turbulent flow of water through a channel with an aspect ratio of 12 : 1 at Re=5700 (based on the centerline velocity and the half-height of the channel). Statistical quantities that were determined are the mean streamwise velocity, the root-mean-square of the fluctuations of the streamwise and the normal velocities, the Reynolds shear stress and higher order moments. Turbulence production is calculated from these quantities. Turbulence statistics obtained from experiments are compared with results from a direct numerical simulation at the same Reynolds number. The good agreement validates a recent DNS, at Re=5700, which is approximately twice as large as used in most previous studies.},
  Type                     = {Journal Article},
  Url                      = {http://dx.doi.org/10.1007/s003480050256}
}

@Article{Gabitto2008,
  Title                    = {Drag coefficient and settling velocity for particles of cylindrical shape},
  Author                   = {Gabitto, Jorge and Tsouris, Costas},
  Journal                  = {Powder Technology},
  Year                     = {2008},
  Number                   = {2},
  Pages                    = {314-322},
  Volume                   = {183},

  ISSN                     = {0032-5910},
  Keywords                 = {Drag coefficient
Drag force
Settling velocity
Terminal velocity
Cylindrical particles},
  Type                     = {Journal Article},
  Url                      = {http://www.sciencedirect.com/science/article/B6TH9-4PC3SGX-2/2/15e1e4d88d257b1cd54c818d2df0801b}
}

@Article{Gabitto2007,
  Title                    = {Drag coefficient and settling velocity for particles of cylindrical shape},
  Author                   = {Gabitto, J. and Tsouris, C.},
  Journal                  = {Powder Technology},
  Year                     = {2007},
  Pages                    = {doi:10,1016/j.powtec.2007.07.031},

  Type                     = {Journal Article}
}

@Article{Gabrio1999,
  Title                    = {A new method to evaluate plume characteristics of hydrofluoroalkane and chlorofluorocarbon metered dose inhalers},
  Author                   = {Gabrio, Brian J. and Stein, Stephen W. and Velasquez, David J.},
  Journal                  = {International Journal of Pharmaceutics},
  Year                     = {1999},
  Note                     = {doi: DOI: 10.1016/S0378-5173(99)00133-7},
  Number                   = {1},
  Pages                    = {3-12},
  Volume                   = {186},

  ISSN                     = {0378-5173},
  Keywords                 = {Cold-Freon effect
Deposition
Metered dose inhaler
Plume force
Plume temperature
pMDI},
  Type                     = {Journal Article},
  Url                      = {http://www.sciencedirect.com/science/article/B6T7W-3X94260-2/2/7bbbda6076ea5fa660cef12b52ff091f}
}

@Article{Gadgil2003,
  Title                    = {Indoor pollutant mixing time in an isothermal closed room: an investigation using CFD},
  Author                   = {Gadgil, A. J. and Lobscheid, C. and Abadie, M. O. and Finlayson, E. U.},
  Journal                  = {Atmospheric Environment},
  Year                     = {2003},
  Number                   = {39–40},
  Pages                    = {5577-5586},
  Volume                   = {37},

  Doi                      = {10.1016/j.atmosenv.2003.09.032},
  ISSN                     = {1352-2310},
  Keywords                 = {Mixing time
Pollutant dispersion
CFD modeling
Short-term exposure},
  Type                     = {Journal Article},
  Url                      = {http://www.sciencedirect.com/science/article/pii/S135223100300774X}
}

@Article{Galea2009,
  Title                    = {Trends in Wood Dust Inhalation Exposure in the UK, 1985–2005},
  Author                   = {Galea, Karen S. and Van Tongeren, Martie and Sleeuwenhoek, Anne J. and While, David and Graham, Mairi and Bolton, Annette and Kromhout, Hans and Cherrie, John W.},
  Journal                  = {Annals of Occupational Hygiene},
  Year                     = {2009},
  Number                   = {7},
  Pages                    = {657-667},
  Volume                   = {53},

  Abstract                 = {Objectives: Wood dust data held in the Health and Safety Executive (HSE) National Exposure DataBase (NEDB) were reviewed to investigate the long-term changes in inhalation exposure from 1985 to 2005. In addition, follow-up sampling measurements were obtained from selected companies where exposure measurements had been collected prior to 1994, thereby providing a follow-up period of at least 10 years, to determine whether changes in exposure levels had occurred, with key staff being interviewed to identify factors that might be responsible for any changes observed.Methods: Analysis of the temporal trend in exposure concentrations was performed using Linear Mixed Effect Models on the log-transformed NEDB data set and expressed as the relative annual change in concentration.Results: For the NEDB wood dust data, an annual decline of geometric mean (GM) exposure of 8.1% per year was found based on 1459 exposure measurements collected between 1985 and 2003. This trend was predominantly observed in data from inspection visits (measurements collected on a mandatory basis by a Specialist HSE Inspector) (n = 1009), while data from representative surveys (measurements collected on a voluntary basis to provide information on current practices and exposures) remained relatively stable. Ten follow-up surveys in individual workplaces in 2004–2005 resulted in 70 new measurements and for each of the companies resurveyed, the GM of the wood dust exposure decreased between sampling surveys.Conclusion: Analysis of the temporal trend in UK wood dust exposure concentrations revealed declines of 8% per annum. Interviews with key long-serving employees and management suggest that factors such as technological changes in production processes, response to new legislation, and enforcement agency inspections, together with global economic trends, could be linked to the downward trends observed.},
  Doi                      = {10.1093/annhyg/mep044},
  Type                     = {Journal Article},
  Url                      = {http://annhyg.oxfordjournals.org/content/53/7/657.abstract}
}

@Article{Galimov2010,
  Title                    = {Parallel adaptive simulation of a plunging liquid jet},
  Author                   = {Galimov, Azat Yu and Sahni, Onkar and Lahey Jr, Richard T. and Shephard, Mark S. and Drew, Donald A. and Jansen, Kenneth E.},
  Journal                  = {Acta Mathematica Scientia},
  Year                     = {2010},
  Number                   = {2},
  Pages                    = {522-538},
  Volume                   = {30},

  Abstract                 = {This paper is concerned with three-dimensional numerical simulation of a plunging liquid jet. The transient processes of forming an air cavity around the jet, capturing an initially large air bubble, and the break-up of this large toroidal-shaped bubble into smaller bubbles were analyzed. A stabilized finite element method (FEM) was employed under parallel numerical simulations based on adaptive, unstructured grid and coupled with a level-set method to track the interface between air and liquid. These simulations show that the inertia of the liquid jet initially depresses the pool's surface, forming an annular air cavity which surrounds the liquid jet. A toroidal liquid eddy which is subsequently formed in the liquid pool results in air cavity collapse, and in turn entrains air into the liquid pool from the unstable annular air gap region around the liquid jet.},
  ISSN                     = {0252-9602},
  Keywords                 = {plunging liquid jet
air entrainment
two-phase ows
level set method
parallel adaptive simulation
76B10
76M10
65Y05},
  Type                     = {Journal Article},
  Url                      = {http://www.sciencedirect.com/science/article/B82XD-4YRYTJ7-8/2/de185e16ddd27d39966a8504199917d0}
}

@Article{Gallily1979,
  Title                    = {On the orderly nature of the motion of nonspherical aerosol particles. I. Deposition from a laminar flow},
  Author                   = {Gallily, Isaiah and Eisner, Alfred D.},
  Journal                  = {Journal of Colloid And Interface Science},
  Year                     = {1979},
  Number                   = {2},
  Pages                    = {320-337},
  Volume                   = {68},

  ISSN                     = {0021-9797},
  Type                     = {Journal Article},
  Url                      = {http://www.sciencedirect.com/science/article/pii/0021979779902868}
}

@Article{Gambaruto2006,
  Title                    = {Characterisation of nasal geometry and flow},
  Author                   = {Gambaruto, A. and Doorly, D.},
  Journal                  = {Journal of Biomechanics},
  Year                     = {2006},
  Number                   = {Supplement 1},
  Pages                    = {S272-S272},
  Volume                   = {39},

  ISSN                     = {0021-9290},
  Type                     = {Journal Article},
  Url                      = {http://www.sciencedirect.com/science/article/B6T82-4KR88PB-1G5/2/3a53e0e3bcab77ef77671ddb06b25eee}
}

@Article{Gambaruto2008,
  Title                    = {Reconstruction of shape and its effect on flow in arterial conduits},
  Author                   = {Gambaruto, A. M. and Peiró, J. and Doorly, Denis J. and Radaelli, A. G.},
  Journal                  = {International Journal for Numerical Methods in Fluids},
  Year                     = {2008},
  Note                     = {C:\Users\sean\AppData\Roaming\Zotero\Zotero\Profiles\16a4oype.default\zotero\storage\Q445D785\Gambaruto et al. - 2008 - Reconstruction of shape and its effect on flow in .pdf},
  Pages                    = {495-517},
  Volume                   = {57},

  Abstract                 = {The geometry of arterial conduits derived from in vivo image data is subject to acquisition and reconstruction errors. This results in a degree of uncertainty in defining the bounding geometry for a patient-specific anatomical conduit. In applying computational fluid dynamics to model the flow in specific anatomical configurations, the effect of the uncertainty in boundary definition should be considered, particularly if the objective is to extract quantitative measures of the local haemodynamics. Taking an example of a bypass graft configuration, we examine the effects of image threshold, surface smoothing and semi-idealization on the modelled geometry and the resulting flow. Procedures for reconstruction from medical images are outlined and applied with different parameter values within the image uncertainty range to create alternative models from the same data set. Methods to characterize the flow structure and wall shear stress (WSS) are introduced and used to provide quantitative comparison of the different haemodynamic environments associated with the varying model geometries. Comparable effects on the WSS distribution are found to occur with progressively increased surface smoothing and semi-idealization of the geometry by elliptical section fitting. Significant differences in WSS correspond to different threshold choices. Copyright © 2007 John Wiley & Sons, Ltd.},
  Doi                      = {10.1002/fld.1642},
  ISSN                     = {1097-0363},
  Keywords                 = {medical image processing
segmentation uncertainty
shape reconstruction
surface smoothing
topology idealization, sensitivity and confidence bounds},
  Type                     = {Journal Article},
  Url                      = {http://onlinelibrary.wiley.com/store/10.1002/fld.1642/asset/1642_ftp.pdf?v=1&t=i0df7wwh&s=63be74e44aae4f9aaee30974a36d3d0b683d6eb3}
}

@Article{Gambaruto2012,
  Title                    = {Decomposition and Description of the Nasal Cavity Form},
  Author                   = {Gambaruto, A. M. and Taylor, D. J. and Doorly, D. J.},
  Journal                  = {Annals of Biomedical Engineering},
  Year                     = {2012},
  Pages                    = {1142-1159},
  Volume                   = {40},

  Abstract                 = {Patient-specific studies of physiological flows rely on anatomically realistic or idealized models. Objective comparison of datasets or the relation of specific to idealized geometries has largely been performed in an ad hoc manner. Here, two rational procedures (based respectively on Fourier descriptors and medial axis (MA) transforms) are presented; each provides a compact representation of a complex anatomical region, specifically the nasal airways. The techniques are extended to furnish average geometries. These retain a sensible anatomical form, facilitating the identification of a specific anatomy as a set of weighted perturbations about the average. Both representations enable a rapid translation of the surface description into a virtual model for computation of airflow, enabling future work to comprehensively investigate the relation between anatomic form and flow-associated function, for the airways or for other complex biological conduits. The methodology based on MA transforms is shown to allow flexible geometric modeling, as illustrated by a local alteration in airway patency. Computational simulations of steady inspiratory flow are used to explore the relation between the flow in individual vs. averaged anatomical geometries. Results show characteristic flow measures of the averaged geometries to be within the range obtained from the original three subjects, irrespective of averaging procedure. However the effective regularization of anatomic form resulting from the shape averaging was found to significantly reduce trans-nasal pressure loss and the mean shear stress in the cavity. It is suggested that this may have implications in attempts to relate model geometries and flow patterns that are broadly representative.},
  Doi                      = {http://dx.doi.org.ezproxy.lib.rmit.edu.au/10.1007/s10439-011-0485-0},
  ISSN                     = {0090-6964},
  Keywords                 = {Biochemistry, general
Biomedical Engineering
Biomedicine general
Biophysics and Biological Physics
Fourier descriptors
Geometry average
Geometry characterization and deconstruction
Mechanics
Medial axis
Modal analysis
nasal airflow
Radial basis function},
  Type                     = {Journal Article},
  Url                      = {http://download.springer.com/static/pdf/937/art%253A10.1007%252Fs10439-011-0485-0.pdf?auth66=1411539477_b76697edfe30be3bf5c0e73ca9d74c2b&ext=.pdf}
}

@Article{Gambaruto2009,
  Title                    = {Modelling nasal airflow using a Fourier descriptor representation of geometry},
  Author                   = {Gambaruto, A. M. and Taylor, D. J. and Doorly, D. J.},
  Journal                  = {International Journal for Numerical Methods in Fluids},
  Year                     = {2009},
  Pages                    = {1259-1283},
  Volume                   = {59},

  Abstract                 = {Procedures capable of providing both compact representations and rational simplifications of complex anatomical flow conduits are essential to explore how form and function are related in the respiratory, cardiovascular and other physiological flow systems. This study focuses on flow in the human nasal cavity. Methods to derive the cavity wall boundary from medical images are first outlined. Anisotropic smoothing of the boundary surface is shown to provide less geometric distortion in regions of high curvature, such as at the ends of the narrow nasal passages. A reversible decomposition of the surface into a stack of planar contours is then effected using an implicit function formulation. Fourier descriptors provide a continuous representation of each contour as a modal expansion and permit simplification of the geometry by retaining only dominant modes via filtering. Computations of the steady inspiratory flow field are performed for replica and reduced geometries, where reduced geometry is derived by retaining only the first 15 modes in the expansion of each slice contour. The overall pressure drop and integrated wall shear are shown to be virtually unaffected by simplification. More sensitive measures, such as the Lagrangian particle trajectories and residence time distributions, show slight changes as discussed. Comparison of the Fourier descriptor method applied to three different patient data sets indicates the potential of the technique as a means to characterize complex flow conduit geometry, and the scope for further work is outlined. Copyright © 2008 John Wiley & Sons, Ltd.},
  Doi                      = {10.1002/fld.1866},
  ISSN                     = {1097-0363},
  Keywords                 = {anisotropic surface smoothing
anisotropic surface smoothing
Fourier descriptors
Fourier descriptors
geometry characterization and decomposition
geometry characterization and decomposition
implicit function
implicit function
nasal airflow
nasal airflow
nasal passage
nasal passage},
  Type                     = {Journal Article},
  Url                      = {http://onlinelibrary.wiley.com/store/10.1002/fld.1866/asset/1866_ftp.pdf?v=1&t=i0df7pg2&s=9035651bed67d406a9bd32156ba14b8457ff619a}
}

@Article{Gandhi1999,
  Title                    = {Potential Health Hazards from Burning Aircraft Composites},
  Author                   = {Gandhi, Sanjeev and Lyon, Richard and Speitel, Louise},
  Journal                  = {Journal of Fire Sciences},
  Year                     = {1999},
  Number                   = {1},
  Pages                    = {20-41},
  Volume                   = {17},

  Abstract                 = {Burning of polymer matrix composites in postcrash aircraft fires generates a complex mixture of combustion products comprised of gases, organic vapors, and particulate matter including airborne carbon fibers. There is concern among the fire fighting, investigative, and mishap response communities that an unusual health hazard is posed by this combination of combustion products. This paper presents an overview of the nature and potential hazards of acute exposure to airborne carbon fibers from fire and explosion involving advanced composites ma terials. Data from fire tests and crash-site investigations suggest that a small frac tion of the fibers released in fires are respirable and can be inhaled deep into the lung. Most of the carbon fibers produced in fires are 2-10 times larger than the criti cal fiber size associated with asbestos toxicity, and their concentration is well below OSHA recommended levels for chronic exposure. At issue however are the toxico logical effects of adsorbed combustion products. Chemical extraction shows that a large number of toxic organic compounds are adsorbed on these fibers, several of which are known carcinogens and mutagens in animals. At the present time there is no conclusive evidence linking airborne fibers from burning composites to any un usual health hazard. However, no toxicological studies have been conducted to as sess the long-term health effects from exposure to a single high dose of fibrous particulates and any synergistic interactions with the organic chemicals.},
  Doi                      = {10.1177/073490419901700102},
  Type                     = {Journal Article},
  Url                      = {http://jfs.sagepub.com/content/17/1/20.abstract}
}

@Article{Ganser1993,
  Title                    = {A rational approach to drag prediction of spherical and non-spherical particles},
  Author                   = {Ganser, G.H.},
  Journal                  = {Powder Technology},
  Year                     = {1993},
  Pages                    = {143-152},
  Volume                   = {77},

  Type                     = {Journal Article}
}

@Article{Gao2005,
  Title                    = {Experimental and numerical study of high-pressure-swirl injector sprays in a direct injection gasoline engine. Proceedings of the Institution of Mechanical Engineers. Part A.},
  Author                   = {Gao, J. and Jiang, D. and Huang, Z. and Wane, X.},
  Journal                  = {Journal Power Energy},
  Year                     = {2005},
  Number                   = {8},
  Pages                    = {617-629},
  Volume                   = {219},

  Type                     = {Journal Article}
}

@Article{Gao2006,
  Title                    = {Transient CFD simulation of the respiration process and inter-person exposure assessment},
  Author                   = {Gao, Naiping and Niu, Jianlei},
  Journal                  = {Building and Environment},
  Year                     = {2006},
  Number                   = {9},
  Pages                    = {1214-1222},
  Volume                   = {41},

  Doi                      = {10.1016/j.buildenv.2005.05.014},
  ISSN                     = {0360-1323},
  Keywords                 = {Human body
Computational thermal manikin
Inhalation
Exhalation
Human exposure},
  Type                     = {Journal Article},
  Url                      = {http://www.sciencedirect.com/science/article/pii/S0360132305001873}
}

@Article{Gao2009,
  Title                    = {The airborne transmission of infection between flats in high-rise residential buildings: Particle simulation},
  Author                   = {Gao, N. P. and Niu, J. L. and Perino, M. and Heiselberg, P.},
  Journal                  = {Building and Environment},
  Year                     = {2009},
  Number                   = {2},
  Pages                    = {402-410},
  Volume                   = {44},

  ISSN                     = {0360-1323},
  Keywords                 = {Particles
Lagrangian method
Eulerian method
Transport
Deposition},
  Type                     = {Journal Article},
  Url                      = {http://www.sciencedirect.com/science/article/B6V23-4S73RBY-2/2/2549d3e81cabc6034d0172cd109f626a}
}

@Article{Garcia2007,
  Title                    = {Atrophic rhinitis: a CFD study of air conditioning in the nasal cavity},
  Author                   = {Garcia, Guilherme J. M. and Bailie, Neil and Martins, Dario A. and Kimbell, Julia S.},
  Journal                  = {Journal of Applied Physiology},
  Year                     = {2007},
  Number                   = {3},
  Pages                    = {1082-1092},
  Volume                   = {103},

  Abstract                 = {Atrophic rhinitis is a chronic disease of the nasal mucosa. The disease is characterized by abnormally wide nasal cavities, and its main symptoms are dryness, crusting, atrophy, fetor, and a paradoxical sensation of nasal congestion. The etiology of the disease remains unknown. Here, we propose that excessive evaporation of the mucous layer is the basis for the relentless nature of this disease. Airflow and water and heat transport were simulated using computational fluid dynamics (CFD) techniques. The nasal geometry of an atrophic rhinitis patient was acquired from computed tomography scans before and after a procedure to narrow the nasal cavity. Simulations of air conditioning in the atrophic nose were compared with similar computations performed within the nasal geometries of four healthy humans. The excessively wide cavity of the patient generated abnormal flow patterns, which led to abnormal patterns of water fluxes across the wall. Geometrically, the atrophic nose had a much lower surface area than the healthy nasal passages, which increased water fluxes per unit area. Nevertheless, the simulations indicated that the atrophic nose did not condition inspired air as effectively as the healthy geometries. These simulations of water transport in the nasal cavity are consistent with the hypothesis that excessive evaporation of mucus plays a key role in the pathophysiology of atrophic rhinitis. We conclude that the main goals of a surgery to treat atrophic rhinitis should be 1) to restore the original surface area of the nose, 2) to restore the physiological airflow distribution, and 3) to create symmetric cavities.},
  Doi                      = {10.1152/japplphysiol.01118.2006},
  Type                     = {Journal Article},
  Url                      = {http://jap.physiology.org/cgi/content/abstract/103/3/1082}
}

@Article{Garcia2009,
  Title                    = {Interindividual Variability in Nasal Filtration as a Function of Nasal Cavity Geometry},
  Author                   = {Garcia, Guilherme J. M. and Tewksbury, Earl W. and Wong, Brian A. and Kimbell, Julia S.},
  Journal                  = {Journal of Aerosol Medicine and Pulmonary Drug Delivery},
  Year                     = {2009},
  Note                     = {Copyright - (©) Mary Ann Liebert, Inc.
Last updated - 2013-05-29},
  Number                   = {2},
  Pages                    = {139-55},
  Volume                   = {22},

  Abstract                 = {Background: Interindividual variability in nasal filtration is significant due to interindividual differences in nasal anatomy and breathing rate. Two important consequences arise from this variation among humans. First, devices for nasal drug delivery may furnish quite different doses in the nasal passages of different individuals, leading to different responses to therapeutic treatment. Second, people with poor nasal filtration may be more susceptible to adverse health effects when exposed to airborne particulate matter (PM) due to greater lung deposition. Although interindividual variability of nasal filtration has been reported by several authors, a relationship for predicting filtration efficiency from nasal anatomy and ventilation is still lacking. Such a relationship is needed to (1) devise nasal drug delivery systems and (2) define limits of exposure to PM that are effective for the human population at large. Methods: Anatomically correct nasal replicas of five adults (four healthy individuals and one atrophic rhinitis patient) were used in aerosol experiments to measure nasal deposition of 1-12-μm particles. The dependence of nasal filtration on nasal anatomy and breathing rate was investigated using various definitions of the Stokes number as well as phenomenological Impaction Parameters proposed in the literature. Results: Interindividual variability among the healthy adults was nearly eliminated when nasal filtration was plotted against a specific definition of the Stokes number or against a pressure-based Impaction Parameter. Nasal filtration in the atrophic rhinitis patient was lower than in the healthy subjects. Conclusions: The new definition of the Stokes number introduced in this study, which is based on a new definition of the characteristic diameter of the nasal passages, nearly eliminated interindividual differences in nasal filtration. Our results suggest that it is possible to estimate nasal filtering efficiency using measurements of transnasal pressure drop.},
  Doi                      = {http://dx.doi.org/10.1089/jamp.2008.0713},
  ISSN                     = {19412711},
  Keywords                 = {Medical Sciences--Respiratory Diseases
Aerosols
Decanoic Acids
di-2-ethylhexyl sebacate
Rhinitis, Atrophic -- physiopathology
Reference Values
Humans
Particle Size
Nasal Cavity -- metabolism
Decanoic Acids -- metabolism
Models, Biological
Rhinitis, Atrophic -- metabolism
Administration, Intranasal
Decanoic Acids -- chemistry
Adult
Middle Aged
Female
Male
Decanoic Acids -- administration & dosage
Respiratory Mechanics
Nasal Cavity -- anatomy & histology
Rhinitis, Atrophic -- pathology
Models, Anatomic},
  Type                     = {Journal Article},
  Url                      = {http://search.proquest.com/docview/219287478?accountid=13552
http://primoapac01.hosted.exlibrisgroup.com/openurl/RMITU/RMIT_SERVICES_PAGE??url_ver=Z39.88-2004&rft_val_fmt=info:ofi/fmt:kev:mtx:journal&genre=article&sid=ProQ:ProQ%3Ahealthcompleteshell&atitle=Interindividual+Variability+in+Nasal+Filtration+as+a+Function+of+Nasal+Cavity+Geometry&title=Journal+of+Aerosol+Medicine+and+Pulmonary+Drug+Delivery&issn=19412711&date=2009-06-01&volume=22&issue=2&spage=139&au=Garcia%2C+Guilherme+JM%3BTewksbury%2C+Earl+W%3BWong%2C+Brian+A%3BKimbell%2C+Julia+S&isbn=&jtitle=Journal+of+Aerosol+Medicine+and+Pulmonary+Drug+Delivery&btitle=&rft_id=info:eric/}
}

@Book{Gatski1992,
  Title                    = {On explicit algebraic stress models for complex turbulent flows [microform] / T.B. Gatski, C.G. Speziale},
  Author                   = {Gatski, T. B.},
  Publisher                = {National Aeronautics and Space Administration, Langley Research Center ; National Technical Information Service, distributor},
  Year                     = {1992},

  Address                  = {Hampton, Va. : [Springfield, Va. :},
  Note                     = {(Charles G.)
Distributed to depository libraries in microfiche.
Shipping list no.: 93-454-M.},
  Series                   = {ICASE report ; no. 189725.},

  Keywords                 = {Strains and stresses - Mathematical models.},
  Type                     = {Book}
}

@Article{Gaver2006,
  Title                    = {Surfactant biophysical properties and their impact on pulmonary atelectrauma},
  Author                   = {Gaver, D. P. and Krueger, M. A.},
  Journal                  = {Journal of Biomechanics},
  Year                     = {2006},
  Number                   = {Supplement 1},
  Pages                    = {S265-S265},
  Volume                   = {39},

  ISSN                     = {0021-9290},
  Type                     = {Journal Article},
  Url                      = {http://www.sciencedirect.com/science/article/B6T82-4KR88PB-1F2/2/2f914bf9d44f30490bd20258810cdbbe}
}

@Article{GA³mez2007,
  Title                    = {Modelling the influence of nanoparticles in the phase behaviour of an epoxy/polystyrene mixture},
  Author                   = {Gómez, Clara M. and Porcar, Iolanda and Monzo, Isidro S. and Abad, Concepción and Campos, Agustín},
  Journal                  = {European Polymer Journal},
  Year                     = {2007},
  Number                   = {2},
  Pages                    = {360-373},
  Volume                   = {43},

  Abstract                 = {The cloud point temperatures of four series of epoxy/polystyrene blends have been experimentally determined as a function of polymer mass and system composition. The phase diagrams show an UCST behaviour, increasing incompatibility as the molar mass increases. The Flory-Huggins theory with a concentration-dependent interaction parameter has been developed to study the compatibility of two polymers in presence of spherical nanoparticles. This theory has been first compared with the experimental cloud point curve in absence of nanoparticles, and secondly it has been used to predict the thermodynamic behaviour in presence of different volume fraction of nanoparticles. Nanoparticles coated with two types of functional groups have been tested. It can be concluded that the inclusion of nanoparticles increases compatibility. Moreover concentration-dependent interaction parameters have been obtained in these systems for the first time.},
  ISSN                     = {0014-3057},
  Keywords                 = {Phase separation
Polymer blend
Nanoparticle},
  Type                     = {Journal Article},
  Url                      = {http://www.sciencedirect.com/science/article/B6TWW-4MCWB12-1/2/a1515b2ce9a75845d219e1729e01f0a7}
}

@Article{Ge2013,
  Title                    = {Numerical study of the effects of human body heat on particle transport and inhalation in indoor environment},
  Author                   = {Ge, Qinjiang and Li, Xiangdong and Inthavong, Kiao and Tu, Jiyuan},
  Journal                  = {Building and Environment},
  Year                     = {2013},
  Pages                    = {1-9},
  Volume                   = {59},

  Doi                      = {http://dx.doi.org/10.1016/j.buildenv.2012.08.002},
  ISSN                     = {0360-1323},
  Keywords                 = {Airflow
Body
CFD
field
heat
inhalation
Particle
speed
Wind},
  Type                     = {Journal Article},
  Url                      = {http://ac.els-cdn.com/S0360132312002041/1-s2.0-S0360132312002041-main.pdf?_tid=bd28aa78-421f-11e4-9dbc-00000aab0f02&acdnat=1411366661_8fa870a8df7aea62513d8d9c99836590}
}

@Article{Ge2013a,
  Title                    = {Numerical study of the effects of human body heat on particle transport and inhalation in indoor environment},
  Author                   = {Ge, Qinjiang and Li, Xiangdong and Inthavong, Kiao and Tu, Jiyuan},
  Journal                  = {Building and Environment},
  Year                     = {2013},
  Number                   = {0},
  Pages                    = {1-9},
  Volume                   = {59},

  Doi                      = {http://dx.doi.org/10.1016/j.buildenv.2012.08.002},
  ISSN                     = {0360-1323},
  Keywords                 = {Body heat
Airflow field
Particle inhalation
Wind speed
CFD},
  Type                     = {Journal Article},
  Url                      = {http://www.sciencedirect.com/science/article/pii/S0360132312002041}
}

@Article{Ge2013b,
  Title                    = {Numerical study of the effects of human body heat on particle transport and inhalation in indoor environment},
  Author                   = {Ge, Qinjiang and Li, Xiangdong and Inthavong, Kiao and Tu, Jiyuan},
  Journal                  = {Building and Environment},
  Year                     = {2013},
  Number                   = {0},
  Pages                    = {1-9},
  Volume                   = {59},

  Doi                      = {10.1016/j.buildenv.2012.08.002},
  ISSN                     = {0360-1323},
  Keywords                 = {Body heat
Airflow field
Particle inhalation
Wind speed
CFD},
  Type                     = {Journal Article},
  Url                      = {http://www.sciencedirect.com/science/article/pii/S0360132312002041}
}

@Article{Ge2012,
  Title                    = {Local deposition fractions of ultrafine particles in a human nasal-sinus cavity CFD model},
  Author                   = {Ge, Qin Jiang and Inthavong, Kiao and Tu, Ji Yuan},
  Journal                  = {Inhalation Toxicology},
  Year                     = {2012},
  Number                   = {8},
  Pages                    = {492-505},
  Volume                   = {24},

  Doi                      = {doi:10.3109/08958378.2012.694494},
  Type                     = {Journal Article},
  Url                      = {http://informahealthcare.com/doi/abs/10.3109/08958378.2012.694494}
}

@Article{Gebremedhin2003,
  Title                    = {Characterization of flow field in a ventilated space and simulation of heat exchange between cows and their environment},
  Author                   = {Gebremedhin, K. G. and Wu, B. X.},
  Journal                  = {Journal of Thermal Biology},
  Year                     = {2003},
  Number                   = {4},
  Pages                    = {301-319},
  Volume                   = {28},

  Doi                      = {10.1016/s0306-4565(03)00007-x},
  ISSN                     = {0306-4565},
  Keywords                 = {Turbulence model
Flow field
Ventilated space
Simulation
Heat exchange
Coupled heat and mass transfer},
  Type                     = {Journal Article},
  Url                      = {http://www.sciencedirect.com/science/article/pii/S030645650300007X}
}

@Article{Gelperina2005,
  Title                    = {The Potential Advantages of Nanoparticle Drug Delivery Systems in Chemotherapy of Tuberculosis},
  Author                   = {Gelperina, Svetlana and Kisich, Kevin and Iseman, Michael D. and Heifets, Leonid},
  Journal                  = {Am. Journal Respir. Crit. Care Medicine},
  Year                     = {2005},
  Number                   = {12},
  Pages                    = {1487-1490},
  Volume                   = {172},

  Abstract                 = {Nanoparticle-based drug delivery systems have considerable potential for treatment of tuberculosis (TB). The important technological advantages of nanoparticles used as drug carriers are high stability, high carrier capacity, feasibility of incorporation of both hydrophilic and hydrophobic substances, and feasibility of variable routes of administration, including oral application and inhalation. Nanoparticles can also be designed to allow controlled (sustained) drug release from the matrix. These properties of nanoparticles enable improvement of drug bioavailability and reduction of the dosing frequency, and may resolve the problem of nonadherence to prescribed therapy, which is one of the major obstacles in the control of TB epidemics. This article highlights some of the issues of nanotechnology relevant to the anti-TB drugs.},
  Doi                      = {10.1164/rccm.200504-613PP},
  Type                     = {Journal Article},
  Url                      = {http://ajrccm.atsjournals.org/cgi/content/abstract/172/12/1487}
}

@Article{Gemci2002,
  Title                    = {A Numerical and Experimental Study of Spray Dynamics in a Simple Throat Model},
  Author                   = {Gemci, T. and Corcoran, T.E. and Chigier, N.},
  Journal                  = {Aerosol Science and Technology},
  Year                     = {2002},
  Number                   = {1},
  Pages                    = {18-38},
  Volume                   = {36},

  Type                     = {Journal Article}
}

@Article{Gemci2008,
  Title                    = {Computational model of airflow in upper 17 generations of human respiratory tract},
  Author                   = {Gemci, T. and Ponyavin, V. and Chen, Y. and Chen, H. and Collins, R.},
  Journal                  = {Journal of Biomechanics},
  Year                     = {2008},
  Number                   = {9},
  Pages                    = {2047-2054},
  Volume                   = {41},

  Abstract                 = {Computational fluid dynamics (CFD) studies of airflow in a digital reference model of the 17-generation airway (bronchial tree) were accomplished using the FLUENT® computational code, based on the anatomical model by Schmidt et al. [2004. A digital reference model of the human bronchial tree. Computerized Medical Imaging and Graphics 28, 203-211]. The lung model consists of 6.744×106 unstructured tetrahedral computational cells. A steady-state airflow rate of 28.3 L/min was used to simulate the transient turbulent flow regime using a large eddy simulation (LES) turbulence model. This CFD mesh represents the anatomically realistic asymmetrical branching pattern of the larger airways. It is demonstrated that the nature of the secondary vortical flows, which develop in such asymmetric airways, varies with the specific anatomical characteristics of the branching conduits.},
  ISSN                     = {0021-9290},
  Keywords                 = {Tracheobronchial airways
Human tracheobronchial airway model
Asymmetric bronchial tree
Computational
LES model},
  Type                     = {Journal Article},
  Url                      = {http://www.sciencedirect.com/science/article/B6T82-4SK0C4J-1/2/ae6c29bde377fdaa5f3c67ca7a86ad19}
}

@Article{Gemci2003,
  Title                    = {A CFD study of the throat during aerosol drug delivery using heliox and air},
  Author                   = {Gemci, T. and Shortall, B. and Allen, G. M. and Corcoran, T. E. and Chigier, N.},
  Journal                  = {Journal of Aerosol Science},
  Year                     = {2003},
  Number                   = {9},
  Pages                    = {1175-1192},
  Volume                   = {34},

  ISSN                     = {0021-8502},
  Keywords                 = {Inhalation therapy spray
Computational fluid dynamics
Human throat
KIVA-3V},
  Type                     = {Journal Article},
  Url                      = {http://www.sciencedirect.com/science/article/B6V6B-48SBV1F-1/2/d85eee8d3b0f17ae70c670fa81b2642a}
}

@Article{GeoMagic2005,
  Title                    = {Geomagic Documentation},
  Author                   = {GeoMagic},
  Journal                  = {Geomagic Inc. NC, USA},
  Year                     = {2005},

  Type                     = {Journal Article}
}

@Article{Gerz2002,
  Title                    = {Commercial aircraft wake vortices},
  Author                   = {Gerz, Thomas and Holzäpfel, Frank and Darracq, Denis},
  Journal                  = {Progress in Aerospace Sciences},
  Year                     = {2002},
  Number                   = {3},
  Pages                    = {181-208},
  Volume                   = {38},

  Abstract                 = {This paper discusses the problem of wake vortices shed by commercial aircraft. It presents a consolidated European view on the current status of knowledge of the nature and characteristics of aircraft wakes and of technical and operational procedures of minimizing and predicting the vortex strength and avoiding wake encounters. Methodological aspects of data evaluation and interpretation, like the description of wake ages, the characterization of wake vortices, and the proper evaluation of wake data from measurement and simulation, are addressed in the first part. In the second part an inventory of our knowledge is given on vortex characterization and control, prediction and monitoring of vortex decay, vortex detection and warning, vortex encounter models, and wake-vortex safety assessment. Each section is concluded by a list of questions and required actions which may help to guide further research activities. The primary objective of the joint international efforts in wake-vortex research is to avoid potentially hazardous wake encounters for aircraft. Shortened aircraft separations under appropriate meteorological conditions, whilst keeping or even increasing the safety level, is the ultimate goal. Reduced time delays on the tactical side and increased airport capacities on the strategic side will be the benefits of these ambitious ventures for the air transportation industry and services.},
  ISSN                     = {0376-0421},
  Type                     = {Journal Article},
  Url                      = {http://www.sciencedirect.com/science/article/B6V3V-45NGHXY-1/2/55b701827ff6de230fe11e9c08c7b196}
}

@Article{Ghaboussi1990,
  Title                    = {Three-dimensional discrete element method for granular materials},
  Author                   = {Ghaboussi, Jamshid and Barbosa, Ricardo},
  Journal                  = {International Journal for Numerical and Analytical Methods in Geomechanics},
  Year                     = {1990},
  Note                     = {10.1002/nag.1610140702},
  Number                   = {7},
  Pages                    = {451-472},
  Volume                   = {14},

  ISSN                     = {1096-9853},
  Type                     = {Journal Article},
  Url                      = {http://dx.doi.org/10.1002/nag.1610140702}
}

@Article{Ghaemi2010,
  Title                    = {Evaluation of Digital Image Discretization Error in Droplet Shape Measurement Using Simulation},
  Author                   = {Ghaemi, S. and Rahimi, P. and Nobes, D.S.},
  Journal                  = {Particle \& Particle Systems Characterization},
  Year                     = {2010},
  Number                   = {5},
  Pages                    = {243-255},
  Volume                   = {26},

  Type                     = {Journal Article}
}

@Article{Ghaemi2010a,
  Title                    = {Evaluation of StereoPIV measurement of droplet velocity in an effervescent spray},
  Author                   = {Ghaemi, S. and Rahimi, P. and Nobes, D.S.},
  Journal                  = {The
International Journal of Spray and Combustion Dynamics},
  Year                     = {2010},
  Number                   = {2},
  Pages                    = {103-124},
  Volume                   = {2},

  Type                     = {Journal Article}
}

@Article{Ghidossi2006,
  Title                    = {Computational fluid dynamics applied to membranes: State of the art and opportunities},
  Author                   = {Ghidossi, R. and Veyret, D. and Moulin, P.},
  Journal                  = {Chemical Engineering and Processing},
  Year                     = {2006},
  Number                   = {6},
  Pages                    = {437-454},
  Volume                   = {45},

  ISSN                     = {0255-2701},
  Keywords                 = {Computational fluid dynamics
Membrane
Review
Opportunities
Flow},
  Type                     = {Journal Article},
  Url                      = {http://www.sciencedirect.com/science/article/B6TFH-4HSY4CD-1/2/aaaf484472a7b1a9d4891d5f801e0bb5}
}

@Article{Giddens1993,
  Title                    = {The Role of Fluid Mechanics in the Localization and Detection of Atherosclerosis},
  Author                   = {Giddens, D. P. and Zarins, C. K. and Glagov, S.},
  Journal                  = {Journal of Biomechanical Engineering},
  Year                     = {1993},
  Number                   = {4B},
  Pages                    = {588-594},
  Volume                   = {115},

  Type                     = {Journal Article},
  Url                      = {http://link.aip.org/link/?JBY/115/588/1}
}

@Article{Gill2007,
  Title                    = {Nanoparticles: characteristics, mechanisms of action and toxicity in pulmonary drug delivery—a review},
  Author                   = {Gill, S. and Löbenberg, R. and Ku, T. and Azarmi, S. and Roa, W. and Prenner, E.J.},
  Journal                  = {J. Biomed. Nanotechnol},
  Year                     = {2007},
  Volume                   = {3},

  Type                     = {Journal Article}
}

@Article{Girardin1984,
  Title                    = {Experimental study of velocity fields in a human nasal fossa by laser anemometry},
  Author                   = {Girardin, M. and Bilgen, E. and Arbour, P.},
  Journal                  = {Prog. Neurobiol.},
  Year                     = {1984},
  Pages                    = {317-345},
  Volume                   = {23},

  Type                     = {Journal Article}
}

@Article{Girardin1983,
  Title                    = {Experimental study of velocity fields in a human nasal fossa by laser anemometry},
  Author                   = {Girardin, M. and Bilgen, E. and Arbour, P.},
  Journal                  = {Ann. Otol. Rhinol. Laryngol.},
  Year                     = {1983},
  Pages                    = {231-236},
  Volume                   = {92},

  Type                     = {Journal Article}
}

@Article{Goldman1967,
  Title                    = {Slow viscous motion of a sphere parallel to a plane wall--I Motion through a quiescent fluid},
  Author                   = {Goldman, A. J. and Cox, R. G. and Brenner, H.},
  Journal                  = {Chemical Engineering Science},
  Year                     = {1967},
  Number                   = {4},
  Pages                    = {637-651},
  Volume                   = {22},

  ISSN                     = {0009-2509},
  Type                     = {Journal Article},
  Url                      = {http://www.sciencedirect.com/science/article/pii/0009250967800472}
}

@Article{Goldman1967a,
  Title                    = {Slow viscous motion of a sphere parallel to a plane wall--II Couette flow},
  Author                   = {Goldman, A. J. and Cox, R. G. and Brenner, H.},
  Journal                  = {Chemical Engineering Science},
  Year                     = {1967},
  Number                   = {4},
  Pages                    = {653-660},
  Volume                   = {22},

  ISSN                     = {0009-2509},
  Type                     = {Journal Article},
  Url                      = {http://www.sciencedirect.com/science/article/pii/0009250967800484}
}

@Article{Goldschmidt1972,
  Title                    = {Turbulent Diffusion of Small Particles Suspended in Turbulent Jets},
  Author                   = {Goldschmidt, V. and Householder, M. K. and Ahmadi, G. and Chuang, S. C. },
  Journal                  = {Progress in Heat and Mass Transfer},
  Year                     = {1972},
  Pages                    = {487-508},
  Volume                   = {6},

  Type                     = {Journal Article}
}

@Article{Goldsmith1972,
  Title                    = {Flow Behavior of Erythrocytes I. Rotation and Deformation in Dilute Suspensions},
  Author                   = {Goldsmith, H.L. and Marlow, J. },
  Journal                  = {Proceedings of the Royal Society of London Series B},
  Year                     = {1972},
  Pages                    = {351-384},
  Volume                   = {182},

  Type                     = {Journal Article}
}

@Article{Goldsmith1975,
  Title                    = {Hemodynamics},
  Author                   = {Goldsmith, H.L. and Skalak, R. },
  Journal                  = {Annual Review of Fluid Mechanics},
  Year                     = {1975},
  Pages                    = {213-247},
  Volume                   = {7},

  Type                     = {Journal Article}
}

@Article{Goldsmith1971,
  Title                    = {Deformation of human red cells in tube flow},
  Author                   = {Goldsmith, H. L.},
  Journal                  = {Biorheology},
  Year                     = {1971},
  Note                     = {Cited By (since 1996): 21
Export Date: 4 June 2011
Source: Scopus},
  Number                   = {4},
  Pages                    = {235-242},
  Volume                   = {7},

  Type                     = {Journal Article},
  Url                      = {http://www.scopus.com/inward/record.url?eid=2-s2.0-0015061255&partnerID=40&md5=cf3667d174e70007b7d038b874b38cc5}
}

@Article{Goldsmith1962,
  Title                    = {The flow of suspensions through tubes. I. Single spheres, rods, and discs},
  Author                   = {Goldsmith, H. L. and Mason, S. G.},
  Journal                  = {Journal of Colloid Science},
  Year                     = {1962},
  Number                   = {5},
  Pages                    = {448-476},
  Volume                   = {17},

  ISSN                     = {0095-8522},
  Type                     = {Journal Article},
  Url                      = {http://www.sciencedirect.com/science/article/pii/0095852262900569}
}

@Article{Goldstein2002,
  Title                    = {Heat transfer - a review of 2000 literature},
  Author                   = {Goldstein, R. J. and Eckert, E. R. G. and Ibele, W. E. and Patankar, S. V. and Simon, T. W. and Kuehn, T. H. and Strykowski, P. J. and Tamma, K. K. and Bar-Cohen, A. and Heberlein, J. V. R. and Davidson, J. H. and Bischof, J. and Kulacki, F. A. and Kortshagen, U. and Garrick, S.},
  Journal                  = {International Journal of Heat and Mass Transfer},
  Year                     = {2002},
  Number                   = {14},
  Pages                    = {2853-2957},
  Volume                   = {45},

  ISSN                     = {0017-9310},
  Type                     = {Journal Article},
  Url                      = {http://www.sciencedirect.com/science/article/B6V3H-45MDBP7-1/2/64ed723a876fd580ed8cecf9dd696906}
}

@Article{Goldstein2006,
  Title                    = {Heat transfer--A review of 2003 literature},
  Author                   = {Goldstein, R. J. and Ibele, W. E. and Patankar, S. V. and Simon, T. W. and Kuehn, T. H. and Strykowski, P. J. and Tamma, K. K. and Heberlein, J. V. R. and Davidson, J. H. and Bischof, J. and Kulacki, F. A. and Kortshagen, U. and Garrick, S. and Srinivasan, V.},
  Journal                  = {International Journal of Heat and Mass Transfer},
  Year                     = {2006},
  Number                   = {3-4},
  Pages                    = {451-534},
  Volume                   = {49},

  Abstract                 = {The present paper is intended to encompass the English language heat transfer papers published in 2003, including some translations of foreign language papers. This survey, although extensive cannot include every paper; some selection is necessary. Many papers reviewed herein relate to the science of heat transfer, including numerical, analytical and experimental works. Others relate to applications where heat transfer plays a major role not only in man-made devices, but in natural systems as well. The papers are grouped into categories and then into sub-fields within these categories. We restrict ourselves to papers published in reviewed archival journals. Besides reviewing the journal articles in the body of this paper, we also mention important conferences and meetings on heat transfer and related fields, major awards presented in 2003, and books on heat transfer published during the year.},
  ISSN                     = {0017-9310},
  Keywords                 = {Conduction
Boundary layers
Internal flows
Porous media
Heat transfer
Experimental methods
Natural convection
Rotating flows
Mass transfer
Bio-heat transfer
Melting
Freezing
Boiling
Condensation
Radiative heat transfer
Numerical methods
Transport properties
Heat exchangers
Solar energy
Thermal plasmas},
  Type                     = {Journal Article},
  Url                      = {http://www.sciencedirect.com/science/article/B6V3H-4J0WRJX-2/2/97279250f0adb4a732c3d6e65735355b}
}

@Article{Golshahi2010,
  Title                    = {Deposition of inhaled ultrafine aerosols in replicas of nasal airways of infants},
  Author                   = {Golshahi, Laleh and Finlay, WH and Olfert, JS and Thompson, RB and Noga, ML},
  Journal                  = {Aerosol Science and Technology},
  Year                     = {2010},
  Note                     = {C:\Users\sean\AppData\Roaming\Zotero\Zotero\Profiles\16a4oype.default\zotero\storage\KM9FM4F3\deposition of inhaled ultrafine aerosols in replicas of nasal airways of infants.pdf},
  Pages                    = {741–752},
  Volume                   = {44},

  Type                     = {Journal Article}
}

@Article{Golshahi2010a,
  Title                    = {Deposition of Inhaled Ultrafine Aerosols in Replicas of Nasal Airways of Infants},
  Author                   = {Golshahi, L. and Finlay, W. H. and Olfert, J. S. and Thompson, R. B. and Noga, M. L.},
  Journal                  = {Aerosol Science and Technology},
  Year                     = {2010},
  Number                   = {9},
  Pages                    = {741 - 752},
  Volume                   = {44},

  ISSN                     = {0278-6826},
  Type                     = {Journal Article},
  Url                      = {http://www.informaworld.com/10.1080/02786826.2010.488256}
}

@Article{Gonda1990,
  Title                    = {Model of disposition of drugs administered into the human nasal cavity},
  Author                   = {Gonda, I. and Gipps, E.},
  Journal                  = {Pharm. Res.},
  Year                     = {1990},
  Number                   = {1},
  Pages                    = {69-75},
  Volume                   = {7},

  Type                     = {Journal Article},
  Url                      = {http://link.springer.com.ezproxy.lib.rmit.edu.au/article/10.1023/A:1015891727080}
}

@Article{raey1,
  Title                    = {Model of Disposition of Drugs Administered into the Human Nasal Cavity},
  Author                   = {Gonda, Igor and Gipps, Elizabeth},
  Journal                  = {Pharmaceutical Research},
  Year                     = {1990},
  Number                   = {1},
  Pages                    = {69-75},
  Volume                   = {7},

  Doi                      = {10.1023/A:1015891727080},
  ISSN                     = {0724-8741},
  Keywords                 = {intranasal delivery; nasal absorption; pharmacokinetic models; bioavailability},
  Language                 = {English},
  Publisher                = {Kluwer Academic Publishers-Plenum Publishers},
  Url                      = {http://dx.doi.org/10.1023/A%3A1015891727080}
}

@Book{GonzA¡lez2008,
  Title                    = {Digital image processing},
  Author                   = {González, R. and Woods, R.},
  Publisher                = {Pearson/Prentice Hall},
  Year                     = {2008},

  Type                     = {Book}
}

@Article{Gore1989,
  Title                    = {Effect of particle size on modulating turbulent intensity},
  Author                   = {Gore, R. A. and Crowe, C. T.},
  Journal                  = {International Journal of Multiphase Flow},
  Year                     = {1989},
  Number                   = {2},
  Pages                    = {279-285},
  Volume                   = {15},

  Doi                      = {10.1016/0301-9322(89)90076-1},
  ISSN                     = {0301-9322},
  Type                     = {Journal Article},
  Url                      = {http://www.sciencedirect.com/science/article/pii/0301932289900761}
}

@InProceedings{Gosman,
  Title                    = {Aspects of computer simulation of liquid-fuelled combustors.},
  Author                   = {Gosman, A.D. and Ioannides, E.},
  Booktitle                = {AIAA 19th Aerospace Sciences Meeting},
  Pages                    = {Paper AIAA-81-0323},

  Type                     = {Conference Proceedings}
}

@Article{Gosman1983,
  Title                    = {Aspects of Computer Simulation of Liquid-Fueled Combustors},
  Author                   = {Gosman, A. D. and Ioannides, E. },
  Journal                  = {Journal of Energy},
  Year                     = {1983},
  Pages                    = {482-490},
  Volume                   = {7},

  Type                     = {Journal Article}
}

@TechReport{GovernmentalIndustrialHygienists2004,
  Title                    = {TLVs and BEIs based on the documentation of the threshold limit values for chemical substances and physical agents and biological exposure indices},
  Author                   = {American Conference of Governmental Industrial Hygienists},
  Year                     = {2004},
  Type                     = {Report}
}

@Article{Gowadia,
  Title                    = {A transport model for nicotine in the tracheobronchial and pulmonary region of the lung},
  Author                   = {Gowadia, Neha and Dunn-Rankin, Derek},
  Journal                  = {Inhalation Toxicology},
  Number                   = {1},
  Pages                    = {42-48},
  Volume                   = {22},

  Doi                      = {doi:10.3109/08958370902862442},
  Type                     = {Journal Article},
  Url                      = {http://informahealthcare.com/doi/abs/10.3109/08958370902862442}
}

@Article{GradoAa„1992,
  Title                    = {Experimental study on fibrous particle deposition in the human nasal cast},
  Author                   = {Gradoń, Leon and Podgórski, Albert},
  Journal                  = {Journal of Aerosol Science},
  Year                     = {1992},
  Number                   = {0},
  Pages                    = {469-472},
  Volume                   = {23, Supplement 1},

  Doi                      = {http://dx.doi.org/10.1016/0021-8502(92)90451-Z},
  ISSN                     = {0021-8502},
  Type                     = {Journal Article},
  Url                      = {http://www.sciencedirect.com/science/article/pii/002185029290451Z}
}

@Article{Gradon1988,
  Title                    = {Analysis of motion and deposition of fibrous particles on a single filter element},
  Author                   = {Gradon, Leon and Grzybowski, Piotr and Pilacinski, Wlodzimierz},
  Journal                  = {Chemical Engineering Science},
  Year                     = {1988},
  Number                   = {6},
  Pages                    = {1253-1259},
  Volume                   = {43},

  ISSN                     = {0009-2509},
  Type                     = {Journal Article},
  Url                      = {http://www.sciencedirect.com/science/article/pii/0009250988850978}
}

@Article{Gradon1992,
  Title                    = {Experimental study on fibrous particle deposition in the human nasal cast},
  Author                   = {Gradon, Leon and Podgórski, Albert},
  Journal                  = {Journal of Aerosol Science},
  Year                     = {1992},
  Number                   = {Supplement 1},
  Pages                    = {469-472},
  Volume                   = {23},

  ISSN                     = {0021-8502},
  Type                     = {Journal Article},
  Url                      = {http://www.sciencedirect.com/science/article/B6V6B-487FB41-4X/2/cab33fa5f9e32517b2c8e32d97cb03cb}
}

@Article{Gradon1992a,
  Title                    = {Experimental study on fibrous particle deposition in the human nasal cast},
  Author                   = {Gradon, L. and Podgorski, A.},
  Journal                  = {Journal of Aerosol Science},
  Year                     = {1992},
  Number                   = {1},
  Pages                    = {469-472},
  Volume                   = {23},

  Type                     = {Journal Article}
}

@Article{Graham1996,
  Title                    = {Dispersion of solid particles in turbulent flow through pipe expansions},
  Author                   = {Graham, D. I. and James, P. W.},
  Journal                  = {Journal of Mathematics Applied in Business and Industry},
  Year                     = {1996},
  Pages                    = {149-179},
  Volume                   = {7},

  Type                     = {Journal Article}
}

@Article{Graham1996a,
  Title                    = {Turbulent dispersion of particles using eddy interaction models},
  Author                   = {Graham, D. I. and James, P. W.},
  Journal                  = {International Journal of Multiphase Flow},
  Year                     = {1996},
  Note                     = {doi: DOI: 10.1016/0301-9322(95)00061-5},
  Number                   = {1},
  Pages                    = {157-175},
  Volume                   = {22},

  ISSN                     = {0301-9322},
  Keywords                 = {particle dispersion
eddy interaction model
stationarity
homogeneity
integral scales
numerical simulation
Lagrangian particle tracking},
  Type                     = {Journal Article},
  Url                      = {http://www.sciencedirect.com/science/article/B6V45-3VV70T0-B/2/a40d9587c82989dcf12be167c599c353}
}

@Article{Gravemeier2010,
  Title                    = {An algebraic variational multiscale-multigrid method for large eddy simulation of turbulent flow},
  Author                   = {Gravemeier, Volker and Gee, Michael W. and Kronbichler, Martin and Wall, Wolfgang A.},
  Journal                  = {Computer Methods in Applied Mechanics and Engineering},
  Year                     = {2010},
  Number                   = {13-16},
  Pages                    = {853-864},
  Volume                   = {199},

  Abstract                 = {An algebraic variational multiscale-multigrid method is proposed for large eddy simulation of turbulent flow. Level-transfer operators from plain aggregation algebraic multigrid methods are employed for scale separation. In contrast to earlier approaches based on geometric multigrid methods, this purely algebraic strategy for scale separation obviates any coarse discretization besides the basic one. Operators based on plain aggregation algebraic multigrid provide a projective scale separation, enabling an efficient implementation of the proposed method. The application of the algebraic variational multiscale-multigrid method to turbulent flow in a channel produces results notably closer to reference (direct numerical simulation) results than other state-of-the-art methods both for mean streamwise and root-mean-square velocities. For predicting highly sensitive components of the Reynolds-stress tensor in the context of turbulent recirculating flow in a lid-driven cavity, the algebraic variational multiscale-multigrid method also shows a remarkably good performance in predicting reference results from experiment and direct numerical simulation compared to other methods.},
  ISSN                     = {0045-7825},
  Keywords                 = {Turbulent flow
Large eddy simulation
Variational multiscale method
Scale separation
Algebraic multigrid},
  Type                     = {Journal Article},
  Url                      = {http://www.sciencedirect.com/science/article/B6V29-4WFG5PG-5/2/6e50d42fc6019fa7dd3470338f8b6d8b}
}

@Article{Graves2009,
  Title                    = {Economic rationale for infection control in Australian hospitals},
  Author                   = {Graves, Nicholas and Halton, Kate and Paterson, David and Whitby, Michael},
  Journal                  = {Healthcare Infection},
  Year                     = {2009},
  Number                   = {3},
  Pages                    = {81-88},
  Volume                   = {14},

  Type                     = {Journal Article}
}

@Article{Green2004,
  Title                    = {Modelling of peak-flow wall shear stress in major airways of the lung},
  Author                   = {Green, A. S.},
  Journal                  = {Journal of Biomechanics},
  Year                     = {2004},
  Number                   = {5},
  Pages                    = {661-667},
  Volume                   = {37},

  ISSN                     = {0021-9290},
  Keywords                 = {Wall shear stress
Lung airways
Expiration
Cough
CFD modelling},
  Type                     = {Journal Article},
  Url                      = {http://www.sciencedirect.com/science/article/B6T82-4B1X4J5-1/2/442b7821a35b12f89ecc9bfdb2f60f3f}
}

@Article{Grenha2005,
  Title                    = {Microencapsulated chitosan nanoparticles for lung protein delivery},
  Author                   = {Grenha, A. and Seijo, B. and Remuñán-López, C.},
  Journal                  = {Eur. J. Pharm. Sci.},
  Year                     = {2005},
  Number                   = {4},
  Pages                    = {427-437},
  Volume                   = {25},

  Type                     = {Journal Article}
}

@Article{Grenier1996,
  Title                    = {Abnormalities of the airways and lung parenchyma in asthmatics: CT observations in 50 patients and inter- and intraobserver variability.},
  Author                   = {Grenier, P. and Mourey-Gerosa, I. and Benali, K.},
  Journal                  = {Eur Radiol},
  Year                     = {1996},
  Pages                    = {199-206},
  Volume                   = {6},

  Type                     = {Journal Article}
}

@Article{Grgic2006,
  Title                    = {The effect of unsteady flow rate increase on in vitro mouth-throat deposition of inhaled boluses},
  Author                   = {Grgic, B. and Martin, A. R. and Finlay, W. H.},
  Journal                  = {Journal of Aerosol Science},
  Year                     = {2006},
  Number                   = {10},
  Pages                    = {1222-1233},
  Volume                   = {37},

  Abstract                 = {A procedure for measuring the effects of flow increase rate (FIR) on mouth-throat and downstream filter ("lung") deposition was developed. A respiratory aerosol probe (RAP) was used for generation and delivery of small aerosol boluses. This device employs computer controlled, fast opening and closing valves, which allow aerosol injection at a predetermined time during inhalation. Flow was driven by a pulmonary waveform generator (PWG) breathing machine. Monodisperse, diameter dioctylphthalate (DOP) particles were used. The following experimental flow conditions were tested: bolus injection into a steady-state flow of 30 l/min, and bolus injection into unsteady flow accelerating through 30 l/min at either 2 or . In addition, numerical simulation of the effects of FIR on mouth-throat deposition of aerosol boluses was performed. Our results show that unsteady flow resulted in enhanced mouth-throat deposition. This difference can be explained by the higher velocity reached by the particles at their primary deposition site within the mouth-throat due to the accelerating flow rate (which reaches approximately 40 l/min by the time the particles deposit in the area of the larynx). This hypothesis is supported by data with bolus injection into a steady flow rate of 40 l/min, which yields mouth-throat deposition that does not differ significantly from the unsteady case.},
  ISSN                     = {0021-8502},
  Keywords                 = {Transient
Mouth-throat
Aerosol
Particle
Inhaler
Accelerating},
  Type                     = {Journal Article},
  Url                      = {http://www.sciencedirect.com/science/article/B6V6B-4JD0YFT-1/2/7672419dad54dc95c66bab772ed949fb}
}

@Article{Griendling2001,
  Title                    = {Out, Damned Dot: Studies of the NADPH Oxidase in Atherosclerosis},
  Author                   = {Griendling, K.K. and Harrison, D.G. },
  Journal                  = {Journal of Clinical Investigation},
  Year                     = {2001},
  Pages                    = {1423-1424},
  Volume                   = {108},

  Type                     = {Journal Article}
}

@Article{Grigoriadis2009,
  Title                    = {Lagrangian particle dispersion in turbulent flow over a wall mounted obstacle},
  Author                   = {Grigoriadis, D. G. E. and Kassinos, S. C.},
  Journal                  = {International Journal of Heat and Fluid Flow},
  Year                     = {2009},
  Note                     = {doi: DOI: 10.1016/j.ijheatfluidflow.2009.01.010},
  Number                   = {3},
  Pages                    = {462-470},
  Volume                   = {30},

  ISSN                     = {0142-727X},
  Keywords                 = {Incompressible flow
Large-eddy simulation
Immersed boundary method
Particle dispersion
Wake flow},
  Type                     = {Journal Article},
  Url                      = {http://www.sciencedirect.com/science/article/B6V3G-4VP5XFN-1/2/0159aba1774ce999bf6f7757f9c77d0f}
}

@Article{GroAYe2007,
  Title                    = {Time resolved analysis of steady and oscillating flow in the upper human airways},
  Author                   = {Große, S. and Schröder, W. and Klaas, M. and Klöckner, A. and Roggenkamp, J.},
  Journal                  = {Experiments in Fluids},
  Year                     = {2007},
  Note                     = {10.1007/s00348-007-0318-y},
  Number                   = {6},
  Pages                    = {955-970},
  Volume                   = {42},

  Abstract                 = {Abstract&nbsp;&nbsp;In this experimental study a thorough analysis of the steady and unsteady flow field in a realistic transparent silicone lung model of the first bifurcation of the upper human airways will be presented. To determine the temporal evolution of the flow time-resolved particle-image velocimetry recordings were performed for a Womersley number range 3.3 ≤ α ≤ 5.8 and Reynolds numbers of Re D &nbsp;= 1,050, 1,400, and 2,100. The results evidence a highly three-dimensional and asymmetric character of the velocity field in the upper human airways, in which the influence of the asymmetric geometry of the realistic lung model plays a significant role for the development of the flow field in the respiratory system. At steady inspiration, the flow shows independent of the Reynolds number a large zone with embedded counter-rotating vortices in the left bronchia ensuring a continuous streamwise transport into the lung. At unsteady flow the critical Reynolds number, which describes the onset of vortices in the first bifurcation, is increased at higher Womersley number and decreased at higher Reynolds number. At expiration the unsteady and steady flows are almost alike.},
  Type                     = {Journal Article},
  Url                      = {http://dx.doi.org/10.1007/s00348-007-0318-y}
}

@InProceedings{Groszmann,
  Title                    = {Decoupling the Roles of Inertia and Gravity on Particle Dispersion},
  Author                   = {Groszmann},
  Booktitle                = {Proceeding of the Fourth Microgravity Fluid Physics \& Transport Phenomena Conference},
  Pages                    = {117-118},

  Type                     = {Conference Proceedings}
}

@Article{Guan2000,
  Title                    = {FLOW TRANSITION IN BENDS AND APPLICATIONS TO AIRWAYS},
  Author                   = {Guan, X. and Martonen, T. B.},
  Journal                  = {Journal of Aerosol Science},
  Year                     = {2000},
  Note                     = {doi: DOI: 10.1016/S0021-8502(99)00559-5},
  Number                   = {7},
  Pages                    = {833-847},
  Volume                   = {31},

  ISSN                     = {0021-8502},
  Type                     = {Journal Article},
  Url                      = {http://www.sciencedirect.com/science/article/B6V6B-4066RTM-6/2/7e9bbb2fa9a092367d09fb3fe2cf294c}
}

@Article{Guha2008,
  Title                    = {Transport and Deposition of Particles in Turbulent and Laminar Flow},
  Author                   = {Guha, Abhijit},
  Journal                  = {Annual Review of Fluid Mechanics},
  Year                     = {2008},
  Number                   = {1},
  Pages                    = {311-341},
  Volume                   = {40},

  Doi                      = {doi:10.1146/annurev.fluid.40.111406.102220},
  Type                     = {Journal Article},
  Url                      = {http://arjournals.annualreviews.org/doi/abs/10.1146/annurev.fluid.40.111406.102220}
}

@Book{Guide2001,
  Title                    = {Computational Fluid Dynamic Software, Version 6.0.12},
  Author                   = {Guide, FLUENT User’s},
  Publisher                = {FLUENT Inc.},
  Year                     = {2001},

  Type                     = {Book}
}

@Article{Guilmette1994,
  Title                    = {Deposition of 0.005-12um monodisperse particles in a computer-milled, MRI-based nasal airway replica. },
  Author                   = {Guilmette, R.A. and Cheng, Y.S. and Yeh, H.C. and Swift, D.L.},
  Journal                  = {Inhalation Toxicology},
  Year                     = {1994},
  Number                   = {S1},
  Pages                    = {395-399},
  Volume                   = {6},

  Type                     = {Journal Article}
}

@Article{Gungor1999,
  Title                    = {Detection of the nasal cycle with acoustic rhinometry: Techniques and applications},
  Author                   = {Gungor, Anil and Moinuddin, Rizwan and Nelson, Robert H. and Corey, Jacquelynne P.},
  Journal                  = {Otolaryngology - Head and Neck Surgery},
  Year                     = {1999},
  Note                     = {doi: DOI: 10.1016/S0194-5998(99)70413-4},
  Number                   = {2},
  Pages                    = {238-247},
  Volume                   = {120},

  ISSN                     = {0194-5998},
  Type                     = {Journal Article},
  Url                      = {http://www.sciencedirect.com/science/article/B6WP4-4HGNM0J-K/2/be595523c3a79ce9621824927cb452e2}
}

@Article{Guo2006,
  Title                    = {The influence of actuation parameters on in vitro testing of nasal spray products},
  Author                   = {Guo, C. and Doub, W.H.},
  Journal                  = {Journal of Pharmaceutical Sciences},
  Year                     = {2006},
  Number                   = {9},
  Pages                    = {2029-2040},
  Volume                   = {95},

  Type                     = {Journal Article}
}

@Article{Guo2008,
  Title                    = {Assessment of the influence factors on in vitro testing of nasal sprays using Box-Behnken experimental design},
  Author                   = {Guo, Changning and Stine, Keith J. and Kauffman, John F. and Doub, William H.},
  Journal                  = {European Journal of Pharmaceutical Sciences},
  Year                     = {2008},
  Note                     = {doi: DOI: 10.1016/j.ejps.2008.09.001},
  Number                   = {5},
  Pages                    = {417-426},
  Volume                   = {35},

  ISSN                     = {0928-0987},
  Keywords                 = {Nasal drug delivery
In vitro testing
Automated actuation
Shot weight
Spray pattern
Plume geometry
Droplet size distribution
Viscosity
Surface tension
Box-Behnken experimental design},
  Type                     = {Journal Article},
  Url                      = {http://www.sciencedirect.com/science/article/B6T25-4TDVMB8-2/2/632390fcb422c41583d8dffa4df9b46a}
}

@Article{Gupta2005,
  Title                    = {Synthesis and surface engineering of iron oxide nanoparticles for biomedical applications},
  Author                   = {Gupta, A.K. and Gupta, M.},
  Journal                  = {Biomaterials},
  Year                     = {2005},
  Number                   = {18},
  Pages                    = {3995-},
  Volume                   = {25},

  Type                     = {Journal Article}
}

@Article{Gupta2010,
  Title                    = {Characterizing exhaled airflow from breathing and talking},
  Author                   = {Gupta, Jitendra K. and Lin, Chao-Hsin and Chen, Qingyan},
  Journal                  = {Indoor Air},
  Year                     = {2010},
  Number                   = {1},
  Pages                    = {31-39},
  Volume                   = {20},

  ISSN                     = {1600-0668},
  Type                     = {Journal Article},
  Url                      = {http://dx.doi.org/10.1111/j.1600-0668.2009.00623.x}
}

@Article{Gutmark1976,
  Title                    = {The planar turbulent jet},
  Author                   = {Gutmark, E. and Wygnanski, I.},
  Journal                  = {Journal of Fluid Mechanics},
  Year                     = {1976},
  Number                   = {03},
  Pages                    = {465-495},
  Volume                   = {73},

  Type                     = {Journal Article}
}

@Article{HA¶lzer2008,
  Title                    = {New simple correlation formula for the drag coefficient of non-spherical particles},
  Author                   = {Hölzer, Andreas and Sommerfeld, Martin},
  Journal                  = {Powder Technology},
  Year                     = {2008},
  Note                     = {doi: DOI: 10.1016/j.powtec.2007.08.021},
  Number                   = {3},
  Pages                    = {361-365},
  Volume                   = {184},

  ISSN                     = {0032-5910},
  Keywords                 = {Drag coefficient
Correlation
Orientation
Non-spherical particles},
  Type                     = {Journal Article},
  Url                      = {http://www.sciencedirect.com/science/article/B6TH9-4PKFH5K-1/2/407406017d786a741ac4e2fbd1da7112}
}

@Article{HA¶rschler2006,
  Title                    = {Investigation of the impact of the geometry on the nose flow},
  Author                   = {Hörschler, I. and Brücker, Ch and Schröder, W. and Meinke, M.},
  Journal                  = {European Journal of Mechanics - B/Fluids},
  Year                     = {2006},
  Note                     = {doi: DOI: 10.1016/j.euromechflu.2005.11.006},
  Number                   = {4},
  Pages                    = {471-490},
  Volume                   = {25},

  Abstract                 = {Results for flow simulations and experiments of different models of the human nasal cavity with and without turbinates and/or spurs are presented. The flow is investigated for normal inspiration and expiration at Reynolds numbers based on the throat diameter of Re=500 and Re=400. The numerical method is second-order accurate on a multi-block structured grid. Flow measurements are based on the method of Digital Particle-Image Velocimetry (DPIV) in transparent nose models. The experimental results corroborate the numerical flow structure thereby evidencing that the nose flow can be considered laminar in the Reynolds number range investigated. Moreover, the analysis of the flow field indicates overall, the higher susceptibility to geometric changes at inspiration and in particular, the lower turbinate to have the major impact on the flow structure especially when air is inhaled.},
  ISSN                     = {0997-7546},
  Keywords                 = {Nasal cavity flow
Steady and laminar flow field
Numerical and experimental investigation
Geometry variations},
  Type                     = {Journal Article},
  Url                      = {http://www.sciencedirect.com/science/article/B6VKX-4J8KVXC-1/2/b93ca9a86e30c7aefe8ecb9272379954}
}

@Article{HA¶rschler2006a,
  Title                    = {Impact of the geometry on the nose flow},
  Author                   = {Hörschler, I. and Schröder, W.},
  Journal                  = {Journal of Biomechanics},
  Year                     = {2006},
  Number                   = {Supplement 1},
  Pages                    = {S271-S271},
  Volume                   = {39},

  ISSN                     = {0021-9290},
  Type                     = {Journal Article},
  Url                      = {http://www.sciencedirect.com/science/article/B6T82-4KR88PB-1G0/2/9eac34a7f3861a4c50ef341aeb86951d}
}

@Article{HA¶rschler2010,
  Title                    = {On the assumption of steadiness of nasal cavity flow},
  Author                   = {Hörschler, I. and Schröder, W. and Meinke, M.},
  Journal                  = {Journal of Biomechanics},
  Year                     = {2010},
  Number                   = {6},
  Pages                    = {1081-1085},
  Volume                   = {43},

  Abstract                 = {The unsteady flow through a model of the human nasal cavity is analyzed at a Strouhal number of Sr=0.791 for the complete respiration cycle. A comparison of the essential flow structures in the model geometry and a real nasal cavity shows the relevance of the model data. The analysis of the steady and unsteady solutions indicate that at Reynolds numbers Re>=1500 the differences of the solutions of the unsteady and steady flow field can be neglegted. To be more precise, the comparison of the total pressure loss distribution as a function of mass flux for the steady state and unsteady solutions shows the major differences to occur at increasing mass flux. At transition from inspiration to expiration the unsteady results differ the most from the steady state solutions. At high mass fluxes the total pressure loss of the nasal cavity flow almost matches that of the steady state solutions. The comparison with rhinomanometry measurements confirms the present numerical findings.},
  ISSN                     = {0021-9290},
  Keywords                 = {Human nasal cavity
Unsteady inspiration and expiration},
  Type                     = {Journal Article},
  Url                      = {http://www.sciencedirect.com/science/article/B6T82-4Y64DJY-1/2/734349a4d6e7bf3ce011e278dd546e4b}
}

@Article{HA¶rschler2010a,
  Title                    = {On the assumption of steadiness of nasal cavity flow},
  Author                   = {Hörschler, I. and Schröder, W. and Meinke, M.},
  Journal                  = {Journal of Biomechanics},
  Year                     = {2010},
  Number                   = {6},
  Pages                    = {1081-1085},
  Volume                   = {43},

  Abstract                 = {The unsteady flow through a model of the human nasal cavity is analyzed at a S trouhal number of Sr = 0.791 for the complete respiration cycle. A comparison of the essential flow structures in the model geometry and a real nasal cavity shows the relevance of the model data. The analysis of the steady and unsteady solutions indicate that at R eynolds numbers Re ≥ 1500 the differences of the solutions of the unsteady and steady flow field can be neglegted. To be more precise, the comparison of the total pressure loss distribution as a function of mass flux for the steady state and unsteady solutions shows the major differences to occur at increasing mass flux. At transition from inspiration to expiration the unsteady results differ the most from the steady state solutions. At high mass fluxes the total pressure loss of the nasal cavity flow almost matches that of the steady state solutions. The comparison with rhinomanometry measurements confirms the present numerical findings.},
  Doi                      = {10.1016/j.jbiomech.2009.12.008},
  ISSN                     = {0021-9290},
  Keywords                 = {Human nasal cavity
Unsteady inspiration and expiration},
  Type                     = {Journal Article}
}

@Article{HA¶rschler2010b,
  Title                    = {On the assumption of steadiness of nasal cavity flow},
  Author                   = {Hörschler, I. and Schröder, W. and Meinke, M.},
  Journal                  = {Journal of Biomechanics},
  Year                     = {2010},
  Note                     = {C:\Users\sean\AppData\Roaming\Zotero\Zotero\Profiles\16a4oype.default\zotero\storage\JS5A4EGT\Hörschler et al. - 2010 - On the assumption of steadiness of nasal cavity fl.pdf},
  Pages                    = {1081-1085},
  Volume                   = {43},

  Abstract                 = {The unsteady flow through a model of the human nasal cavity is analyzed at a Strouhal number of Sr = 0.791 for the complete respiration cycle. A comparison of the essential flow structures in the model geometry and a real nasal cavity shows the relevance of the model data. The analysis of the steady and unsteady solutions indicate that at Reynolds numbers Re ≥ 1500 the differences of the solutions of the unsteady and steady flow field can be neglegted. To be more precise, the comparison of the total pressure loss distribution as a function of mass flux for the steady state and unsteady solutions shows the major differences to occur at increasing mass flux. At transition from inspiration to expiration the unsteady results differ the most from the steady state solutions. At high mass fluxes the total pressure loss of the nasal cavity flow almost matches that of the steady state solutions. The comparison with rhinomanometry measurements confirms the present numerical findings.},
  Doi                      = {10.1016/j.jbiomech.2009.12.008},
  ISSN                     = {0021-9290},
  Keywords                 = {Human nasal cavity
Unsteady inspiration and expiration},
  Type                     = {Journal Article},
  Url                      = {http://ac.els-cdn.com/S0021929009007179/1-s2.0-S0021929009007179-main.pdf?_tid=eb4119f4-421f-11e4-84ac-00000aacb362&acdnat=1411366738_0421be5e20bfb6328c304e29c7625063}
}

@Article{HA¤uAYermann2001,
  Title                    = {The influence of breathing patterns on particle deposition in a nasal replicate cast.},
  Author                   = {Häußermann, S. and Bailey, A.G. and Bailey, M.R. and Etherington, G. and Youngman, M.J.},
  Journal                  = {Journal of Aerosol Science},
  Year                     = {2001},
  Number                   = {6},
  Pages                    = {923-933},
  Volume                   = {33},

  Type                     = {Journal Article}
}

@Article{Haber2000,
  Title                    = {Shear flow over a self-similar expanding pulmonary alveolus during rhythmical breathing},
  Author                   = {Haber, S., Butler, J. P., Brenner, H., Emanuel, I., Tsuda, A.},
  Journal                  = {Journal Fluid Mechanics},
  Year                     = {2000},
  Pages                    = {243-268},
  Volume                   = {405},

  Type                     = {Journal Article}
}

@Article{,
  Title                    = {A coupled approximate deconvolution and dynamic mixed scale model for large-eddy simulation},
  Author                   = {Habisreutinger, Marc A. and Bouffanais, Roland and Leriche, Emmanuel and Deville, Michel O.},
  Journal                  = {Journal of Computational Physics},
  Year                     = {2007},
  Number                   = {1},
  Pages                    = {241-266},
  Volume                   = {224},

  Abstract                 = {Large-eddy simulations of incompressible Newtonian fluid flows with approximate deconvolution models based on the van Cittert method are reported. The Legendre spectral element method is used for the spatial discretization to solve the filtered Navier-Stokes equations. A novel variant of approximate deconvolution models blended with a mixed scale model using a dynamic evaluation of the subgrid-viscosity constant is proposed. This model is validated by comparing the large-eddy simulation with the direct numerical simulation of the flow in a lid-driven cubical cavity, performed at a Reynolds number of 12,000. Subgrid modeling in the case of a flow with coexisting laminar, transitional and turbulent zones such as the lid-driven cubical cavity flow represents a challenging problem. Moreover, the coupling with the spectral element method having very low numerical dissipation and dispersion builds a well suited framework to analyze the efficiency of a subgrid model. First- and second-order statistics obtained using this new model are showing very good agreement with the direct numerical simulation. Filtering operations rely on an invertible filter applied in a modal basis and preserving the C0-continuity across elements. No clipping on dynamic parameters was needed to preserve numerical stability.},
  ISSN                     = {0021-9991},
  Keywords                 = {Large-eddy simulation
Approximate deconvolution models
Dynamic mixed scales model
Lid-driven cavity
Spectral element methods},
  Type                     = {Journal Article},
  Url                      = {http://www.sciencedirect.com/science/article/B6WHY-4N3P0C3-1/2/b7fe9ecaf916455611bc55d26e7662a1}
}

@InBook{Hadfield1972,
  Title                    = {Damage to the human nasal mucosa by wood dust},
  Author                   = {Hadfield, E.M.},
  Editor                   = {Walton, W.H.},
  Publisher                = {Unwin Bros. Old Working},
  Year                     = {1972},
  Type                     = {Book Section},
  Volume                   = {2},

  Booktitle                = {Inhaled Particles III}
}

@Article{Haefeli-Bleuer1988,
  Title                    = {Morphometry of the human pulmonary acinus. },
  Author                   = {Haefeli-Bleuer, B. and Weibel, E. R.},
  Journal                  = {The Anatomical Record},
  Year                     = {1988},
  Pages                    = {401-414},
  Volume                   = {220},

  Type                     = {Journal Article}
}

@Article{Hahn1993,
  Title                    = {Velocity profiles measured for airflow through a large-scale model of the human nasal cavity.},
  Author                   = {Hahn, I. and Scherer, P.W. and Mozell, M.M.},
  Journal                  = {J Appl. Physiol.},
  Year                     = {1993},
  Number                   = {5},
  Pages                    = {2273-2287},
  Volume                   = {75},

  Type                     = {Journal Article},
  Url                      = {http://jap.physiology.org.ezproxy.lib.rmit.edu.au/content/75/5/2273}
}

@Article{Hahn1993a,
  Title                    = {Velocity profiles measured for airflow through a large-scale model of the human nasal cavity},
  Author                   = {Hahn, I. and Scherer, P. W. and Mozell, M. M.},
  Journal                  = {Journal of Applied Physiology},
  Year                     = {1993},
  Number                   = {5},
  Pages                    = {2273--2287},
  Volume                   = {75},

  Abstract                 = {An anatomically accurate, x20 enlarged scale model of a healthy right human adult nasal cavity was constructed from computerized axial tomography scans for the study of nasal airflow patterns. Detailed velocity profiles for inspiratory and expiratory flow through the model and turbulence intensity were measured with a hot-film anemometer probe with 1 mm spatial resolution. Steady flow rates equivalent to 1,100, 560, and 180 ml/s through one side of the real human nose were studied. Airflows were determined to be moderately turbulent, but changes in the velocity profiles between the highest and lowest flow rates suggest that for normal breathing laminar flow may be present in much of the nasal cavity. The velocity measurements closest to the model wall were estimated to be inside the laminar sublayer, such that the slopes of the velocity profiles are reasonably good estimates of the velocity gradients at the walls. The overall longitudinal pressure drop inside the nasal cavity for the three inspiratory flow rates was estimated from the average total shear stress measured at the central nasal wall and showed good agreement with literature values measured in human subjects.},
  ISBN                     = {1522-1601},
  ISSN                     = {8750-7587},
  Publisher                = {American Physiological Society}
}

@Article{Haider1989,
  Title                    = {Drag coefficient and terminal velocity of spherical and nonspherical particles},
  Author                   = {Haider, A. and Levenspiel, O.},
  Journal                  = {Powder Technology},
  Year                     = {1989},
  Pages                    = {63-70},
  Volume                   = {58},

  Type                     = {Journal Article}
}

@Article{Hall1988,
  Title                    = {Measurements of the Mean Force on a Particle Near a Boundary in Turbulent Flow},
  Author                   = {Hall, D. },
  Journal                  = {Journal of Fluid Mechanics},
  Year                     = {1988},
  Pages                    = {451-466},
  Volume                   = {187},

  Type                     = {Journal Article}
}

@Article{Hamaker1937,
  Title                    = {The London-van der Waals attraction between spherical particles},
  Author                   = {Hamaker, H. C.},
  Journal                  = {Physica},
  Year                     = {1937},
  Note                     = {Cited By (since 1996): 886
Export Date: 4 June 2011
Source: Scopus},
  Number                   = {10},
  Pages                    = {1058-1072},
  Volume                   = {4},

  Type                     = {Journal Article},
  Url                      = {http://www.scopus.com/inward/record.url?eid=2-s2.0-0043116551&partnerID=40&md5=6f82b207ce6db3c0efa02887115f0f80}
}

@Article{Hamed2007,
  Title                    = {Assessment of multiscale resolution for hybrid turbulence model in unsteady separated transonic flows},
  Author                   = {Hamed, Awatef and Basu, Debashis and Das, Kaushik},
  Journal                  = {Computers \& Fluids},
  Year                     = {2007},
  Number                   = {5},
  Pages                    = {924-934},
  Volume                   = {36},

  Abstract                 = {This paper presents results of a computational study conducted to assess the multi-scale resolution capabilities of a hybrid two-equation turbulence model in predicting unsteady separated high speed flows. Numerical solutions are obtained using a third order Roe scheme and the SST (shear-stress-transport) two-equation-based hybrid turbulence model for three-dimensional transonic flow over an open cavity. A detailed assessment of the effects of the computational grid and the hybrid turbulence model coefficient is presented for the unsteady flow field. Computed results are presented for both the resolved and the modeled turbulent kinetic energy (TKE) and for the predicted sound pressure level (SPL) spectra, which are compared to available experimental data and large Eddy simulation (LES) results. The comparison shows that the predicted SPL spectra agree well with the experimental results over a frequency range up to 2500 Hz, and that hybrid turbulence effectively models the shorter wavelengths. The results demonstrate improved agreement with experimental SPL spectra with increased grid resolution and a reduced hybrid turbulence model coefficient. In addition, they show that energy dissipation of the unresolved scales is over-predicted at low resolutions and that the hybrid coefficient influences the grid resolution requirements.},
  ISSN                     = {0045-7930},
  Type                     = {Journal Article},
  Url                      = {http://www.sciencedirect.com/science/article/B6V26-4MX4VN8-1/2/40ff90d686c99409f2c4d4a23af0342d}
}

@Article{Hanna1986,
  Title                    = {Measurement of local mass transfer coefficient in a cast model of the human upper respiratory tract.},
  Author                   = {Hanna, L.M. and Scherer, P.W.},
  Journal                  = {Journal Biomech. Eng.},
  Year                     = {1986},
  Pages                    = {12-18},
  Volume                   = {108},

  Type                     = {Journal Article}
}

@Article{Hao2003,
  Title                    = {Changes in the human vocal tract due to aging and the acoustic correlates of speech production: a pilot study},
  Author                   = {Hao, Grace Jianping and Xue, Steve An},
  Journal                  = {Journal of Speech, Language, and Hearing Research},
  Year                     = {2003},
  Pages                    = {689+},
  Volume                   = {46},

  ISSN                     = {10924388},
  Keywords                 = {Elderly
Speech
Speech disorders
Acoustical Society of America},
  Type                     = {Journal Article},
  Url                      = {http://go.galegroup.com/ps/i.do?id=GALE%7CA104650268&v=2.1&u=rmit&it=r&p=AONE&sw=w&asid=2d3e87095b613e8a8696f5537fef3cee}
}

@Article{Hao2003a,
  Title                    = {Changes in the human vocal tract due to aging and the acoustic correlates of speech production: a pilot study},
  Author                   = {Hao, Grace Jianping and Xue, Steve An},
  Journal                  = {Journal of Speech, Language, and Hearing Research},
  Year                     = {2003},
  Pages                    = {689+},
  Volume                   = {46},

  ISSN                     = {10924388},
  Keywords                 = {Acoustical
America
disorders
Elderly
of
Society
Speech},
  Type                     = {Journal Article}
}

@Article{Harper2002,
  Title                    = {Determining particle size distributions in the inhalable size range for wood dust collected by air samplers},
  Author                   = {Harper, M. and Muller, B.S. and Bartolucci, A.},
  Journal                  = {Journal Environ. Monit.},
  Year                     = {2002},
  Pages                    = {642-647},
  Volume                   = {4},

  Type                     = {Journal Article}
}

@Article{Harrington1963,
  Title                    = {Ragweed pollen density},
  Author                   = {Harrington, J.B. and Metzger, K.},
  Journal                  = {American Journal Botany},
  Year                     = {1963},
  Pages                    = {532-539},
  Volume                   = {50},

  Type                     = {Journal Article}
}

@Article{Harrington2006,
  Title                    = {Importance of the bifurcation zone and branch orientation in simulated aerosol deposition in the alveolar zone of the human lung},
  Author                   = {Harrington, Liam and Kim Prisk, G. and Darquenne, Chantal},
  Journal                  = {Journal of Aerosol Science},
  Year                     = {2006},
  Number                   = {1},
  Pages                    = {37-62},
  Volume                   = {37},

  Abstract                 = {We developed two new three-dimensional models of aerosol transport and deposition representative of two generations in the alveolar zone of the human lung, one with the bifurcation area, and one without. The models were used to simulate particle trajectories of 1- particles in generations 18-22 of the human lung in different orientations with respect to gravity. Orientation had a significant effect on the overall deposition predicted, which varied by a factor of approximately 3 between the orientations with highest and lowest deposition. Total deposition was a function of orientation and the ratio of terminal settling velocity to mean lumen flow velocity. Predicted deposition was significantly greater in the model without the bifurcation than in the model with the bifurcation. We conclude that modeling the bifurcation is important due to the complex relationship between aerodynamic drag and gravitational sedimentation within the bifurcation zone.},
  ISSN                     = {0021-8502},
  Keywords                 = {Gravitational sedimentation
Aerodynamic drag
3D model
Computational fluid dynamics},
  Type                     = {Journal Article},
  Url                      = {http://www.sciencedirect.com/science/article/B6V6B-4G1GFJ7-1/2/1739ff8cb26f034c0e4780f92395b28f}
}

@Article{Harris2005,
  Title                    = {Pressure-Volume curves of the respiratory system},
  Author                   = {Harris, R. S.},
  Journal                  = {Respiratory Care},
  Year                     = {2005},
  Pages                    = {78-99},
  Volume                   = {50},

  Type                     = {Journal Article}
}

@Article{Harris1959,
  Title                    = {Tracheal extension in respiration},
  Author                   = {Harris, R. S.},
  Journal                  = {Thorax},
  Year                     = {1959},
  Pages                    = {201-210},
  Volume                   = {14},

  Type                     = {Journal Article}
}

@Article{Harrison2003,
  Title                    = {Role of oxidative stress in atherosclerosis},
  Author                   = {Harrison, David and Griendling, Kathy K. and Landmesser, Ulf and Hornig, Burkhard and Drexler, Helmut},
  Journal                  = {The American journal of cardiology},
  Year                     = {2003},
  Number                   = {3},
  Pages                    = {7-11},
  Volume                   = {91},

  ISSN                     = {0002-9149},
  Type                     = {Journal Article},
  Url                      = {http://linkinghub.elsevier.com/retrieve/pii/S0002914902031442?showall=true}
}

@Article{Hashish,
  Title                    = {Lung deposition of particles by airway generation in healthy subjects: Three-dimensional radionuclide imaging and numerical model prediction},
  Author                   = {Hashish, Adel H. and Fleming, John S. and Conway, Joy and Halson, Peter and Moore, Elizabeth and Williams, Trevor J. and Bailey, Adrian G. and Nassim, Michael and Holgate, Stephen T.},
  Journal                  = {Journal of Aerosol Science},
  Note                     = {doi: DOI: 10.1016/S0021-8502(97)00023-2},
  Number                   = {1-2},
  Pages                    = {205-215},
  Volume                   = {29},

  ISSN                     = {0021-8502},
  Type                     = {Journal Article},
  Url                      = {http://www.sciencedirect.com/science/article/B6V6B-3SX6XPK-H/2/d8139622e21e427c17be83edd6220411}
}

@Article{Haskin1982,
  Title                    = {Normal tracheal bifurcation angle: A reassessment},
  Author                   = {Haskin, P.H. and Goodman, L.R.},
  Journal                  = {American Journal Roentgen},
  Year                     = {1982},
  Pages                    = {879-882},
  Volume                   = {139},

  Type                     = {Journal Article}
}

@Article{Haslbeck2010,
  Title                    = {Submicron droplet formation in the human lung},
  Author                   = {Haslbeck, Karsten and Schwarz, Katharina and Hohlfeld, Jens M. and Seume, Jörg R. and Koch, Wolfgang},
  Journal                  = {Journal of Aerosol Science},
  Year                     = {2010},
  Number                   = {5},
  Pages                    = {429-438},
  Volume                   = {41},

  Abstract                 = {The exhaled breath of humans contains droplets originating from the lung lining fluid. An analysis of these droplets for non-volatile proteinaceous biomarkers holds potential as a non-invasive diagnosis of lung diseases. To ease the interpretation of the diagnostic results, the source strength of the particles should be known und therefore an understanding of the particle generation process is required. It is assumed that during reopening of a collapsed terminal airway a liquid bridge of the lung lining fluid ruptures and droplets are generated. The objective of our experimental and theoretical study was to clarify the mechanisms of droplet generation for quiet breathing patterns by investigating in detail the number flux and the particle size distribution in the exhaled breath. The process of liquid film rupture is modelled by computational fluid dynamics analysis from which the droplet size distribution is calculated. In addition the number emission flux and the droplet size distribution are systematically measured in the exhaled breath of healthy volunteers. The strong increase of the particle emission flux with tidal volume and the good agreement between measured and calculated droplet number distribution both showing droplets primarily in the submicron range confirm the present hypothesis that reopening of collapsed airway structures associated with the rupture of a surfactant film is the physical mechanism of droplet generation. This was hypothesized previously in the literature.},
  Doi                      = {10.1016/j.jaerosci.2010.02.010},
  ISSN                     = {0021-8502},
  Keywords                 = {Exhaled breath analysis
Exhaled droplets
Droplet formation
Particle/lung interaction
Free surface flow},
  Type                     = {Journal Article},
  Url                      = {http://www.sciencedirect.com/science/article/pii/S0021850210000431}
}

@Article{Hassan2001,
  Title                    = {New-wall modeling for complex flows using the large eddy simulation technique in curvilinear coordinates},
  Author                   = {Hassan, Y. A. and Barsamian, H. R.},
  Journal                  = {International Journal of Heat and Mass Transfer},
  Year                     = {2001},
  Number                   = {21},
  Pages                    = {4009-4026},
  Volume                   = {44},

  ISSN                     = {0017-9310},
  Type                     = {Journal Article},
  Url                      = {http://www.sciencedirect.com/science/article/B6V3H-437XPDK-3/2/518bdc9ed5169a54934e314a790de19a}
}

@Article{Hathway2011,
  Title                    = {CFD simulation of airborne pathogen transport due to human activities},
  Author                   = {Hathway, E. A. and Noakes, C. J. and Sleigh, P. A. and Fletcher, L. A.},
  Journal                  = {Building and Environment},
  Year                     = {2011},
  Number                   = {12},
  Pages                    = {2500-2511},
  Volume                   = {46},

  Doi                      = {10.1016/j.buildenv.2011.06.001},
  ISSN                     = {0360-1323},
  Keywords                 = {CFD
Bioaerosols
MRSA
Health-care associated infection},
  Type                     = {Journal Article},
  Url                      = {http://www.sciencedirect.com/science/article/pii/S0360132311001727}
}

@Article{Haverkamp2007,
  Title                    = {Commentary on “The role of the large airways on smooth muscle contraction in asthma�},
  Author                   = {Haverkamp, H.C. and Lundblad, L.K.F.},
  Journal                  = {Journal of Applied Physiology},
  Year                     = {2007},
  Pages                    = {1463 - doi:10.1152/japplphysiol.00703.2007},
  Volume                   = {103},

  Type                     = {Journal Article}
}

@Article{Hayashi2002,
  Title                    = {CFD analysis on characteristics of contaminated indoor air ventilation and its application in the evaluation of the effects of contaminant inhalation by a human occupant},
  Author                   = {Hayashi, Tatsuya and Ishizu, Yoshiaki and Kato, Shinsuke and Murakami, Shuzo},
  Journal                  = {Building and Environment},
  Year                     = {2002},
  Number                   = {3},
  Pages                    = {219-230},
  Volume                   = {37},

  Doi                      = {10.1016/s0360-1323(01)00029-4},
  ISSN                     = {0360-1323},
  Keywords                 = {Computational fluid dynamics (CFD)
κ-ε model
Indoor air pollution
Ventilation effectiveness
Contaminant inhalation},
  Type                     = {Journal Article},
  Url                      = {http://www.sciencedirect.com/science/article/pii/S0360132301000294}
}

@Article{Hazare2007,
  Title                    = { In vivo performance of nasal spray pumps in human volunteers by SPECT-CT imaging},
  Author                   = {Hazare, S.A. and Menon, M.D. and Soni, P.S. and Williams, G. and Brouet, G.},
  Journal                  = {Indian Journal of Pharmaceutical Sciences},
  Year                     = {2007},
  Pages                    = {728-729},
  Volume                   = {69},

  Type                     = {Journal Article}
}

@Article{He1999,
  Title                    = {Particle deposition in a nearly developed turbulent duct flow with electrophoresis},
  Author                   = {He, C. and Ahmadi, G.},
  Journal                  = {Journal of Aerosol Science},
  Year                     = {1999},
  Note                     = {Cited By (since 1996): 42
Export Date: 4 June 2011
Source: Scopus},
  Number                   = {6},
  Pages                    = {739-758},
  Volume                   = {30},

  Type                     = {Journal Article},
  Url                      = {http://www.scopus.com/inward/record.url?eid=2-s2.0-0033167845&partnerID=40&md5=f47296e57fcd053e18f6a3a94061da86}
}

@Article{He1998,
  Title                    = {Particle Deposition with Themophoresis in Laminar and Turbulent Duct Flows},
  Author                   = {He, C. and Ahmadi, G. },
  Journal                  = {Aerosol Science and Technology},
  Year                     = {1998},
  Pages                    = {525-546},
  Volume                   = {29},

  Type                     = {Journal Article}
}

@Article{He2005,
  Title                    = {Removal of contaminants released from room surfaces by displacement and mixing ventilation: modeling and validation},
  Author                   = {He, G. and Yang, X. and Srebric, J.},
  Journal                  = {Indoor Air},
  Year                     = {2005},
  Number                   = {5},
  Pages                    = {367-380},
  Volume                   = {15},

  Doi                      = {10.1111/j.1600-0668.2005.00383.x},
  ISSN                     = {1600-0668},
  Keywords                 = {Area Source
Indoor Air Quality
Validation
Displacement Ventilation
Mixing Ventilation},
  Type                     = {Journal Article},
  Url                      = {http://dx.doi.org/10.1111/j.1600-0668.2005.00383.x}
}

@Article{He1988,
  Title                    = {Mortality and apnea index in obstructive sleep apnea. Experience in 385 male patients.},
  Author                   = {He, J. and Kryger, M.H. and Zorick, F.J. and Conway, W. and Roth, T.},
  Journal                  = {Chest},
  Year                     = {1988},
  Number                   = {1},
  Pages                    = {9-14},
  Volume                   = {94},

  Type                     = {Journal Article}
}

@TechReport{HealthWelfare&AustralasianAssoc.ofCancerRegistries2007,
  Title                    = {Cancer in Australia: an overview, 2006},
  Author                   = {Australian Institute of Health Welfare \& Australasian Assoc. of Cancer Registries},
  Year                     = {2007},
  Type                     = {Report}
}

@TechReport{Health1995,
  Title                    = {Adopted national exposure standards for atmospheric contaminants in the occupational environment},
  Author                   = {National Occupational Health and Safety Commission},
  Year                     = {1995},
  Type                     = {Report}
}

@Article{Hegedus2004,
  Title                    = {Detailed mathematical description of the geometry of airway bifurcations},
  Author                   = {Hegedus, Cs J. and Balásházy, I. and Farkas, �},
  Journal                  = {Respiratory Physiology \& Neurobiology},
  Year                     = {2004},
  Note                     = {doi: DOI: 10.1016/j.resp.2004.03.004},
  Number                   = {1},
  Pages                    = {99-114},
  Volume                   = {141},

  ISSN                     = {1569-9048},
  Keywords                 = {Airway geometry
Computational fluid dynamics},
  Type                     = {Journal Article},
  Url                      = {http://www.sciencedirect.com/science/article/B6X16-4CB0HX9-1/2/ef14f02c1c6d0c720e13be4a6bc2c0ef}
}

@Article{Heil2011,
  Title                    = {Fluid-Structure Interaction in Internal Physiological Flows},
  Author                   = {Heil, Matthias and Hazel, Andrew L.},
  Journal                  = {Annual Review of Fluid Mechanics},
  Year                     = {2011},
  Number                   = {1},
  Pages                    = {141-162},
  Volume                   = {43},

  Doi                      = {doi:10.1146/annurev-fluid-122109-160703},
  Type                     = {Journal Article},
  Url                      = {http://www.annualreviews.org/doi/abs/10.1146/annurev-fluid-122109-160703}
}

@Article{Heil2006,
  Title                    = {Three-dimensional flows in rapidly oscillating vessels},
  Author                   = {Heil, M. and Waters, S. L.},
  Journal                  = {Journal of Biomechanics},
  Year                     = {2006},
  Number                   = {Supplement 1},
  Pages                    = {S442-S442},
  Volume                   = {39},

  ISSN                     = {0021-9290},
  Type                     = {Journal Article},
  Url                      = {http://www.sciencedirect.com/science/article/B6T82-4KR88PB-2F8/2/6cb013420460230ae4d3c5d60ce4bf2b}
}

@Article{Heist2003,
  Title                    = {Airflow Around a Child-Size Manikin in a Low-Speed Wind Environment},
  Author                   = { David K. Heist and Alfred D. Eisner and William Mitchell and Russell Wiener },
  Journal                  = {Aerosol Science and Technology},
  Year                     = {2003},
  Number                   = {4},
  Pages                    = {303-314},
  Volume                   = {37},

  Doi                      = {10.1080/02786820300960},
  Eprint                   = { 
 http://dx.doi.org/10.1080/02786820300960
 
},
  Url                      = { 
 http://dx.doi.org/10.1080/02786820300960
 
}
}

@Article{Henry2002,
  Title                    = {Kinematically irreversible acinar flow: a departure from classical dispersive aerosol transport theories},
  Author                   = {Henry, F. S., Butler, J. B., Tsuda, A.},
  Journal                  = {Journal Applied Physiology},
  Year                     = {2002},
  Pages                    = {835-845},
  Volume                   = {92},

  Type                     = {Journal Article}
}

@Article{Heschl2014,
  Title                    = {Nonlinear eddy viscosity modeling and experimental study of jet spreading rates},
  Author                   = {Heschl, C. and Inthavong, K. and Sanz, W. and Tu, J.},
  Journal                  = {Indoor Air},
  Year                     = {2014},
  Number                   = {1},
  Pages                    = {93-102},
  Volume                   = {24},

  Doi                      = {10.1111/ina.12050},
  ISSN                     = {1600-0668},
  Keywords                 = {Linear diffuser
Jet flow
Turbulent models
Reynolds Averaged Navier Stokes equations
Nonlinear RANS
Spreading rate},
  Type                     = {Journal Article},
  Url                      = {http://dx.doi.org/10.1111/ina.12050}
}

@Article{Heschl2014a,
  Title                    = {Nonlinear eddy viscosity modeling and experimental study of jet spreading rates},
  Author                   = {Heschl, C. and Inthavong, K. and Sanz, W. and Tu, J.},
  Journal                  = {Indoor Air},
  Year                     = {2014},
  Number                   = {1},
  Pages                    = {93-102},
  Volume                   = {24},

  Doi                      = {10.1111/ina.12050},
  ISSN                     = {1600-0668},
  Keywords                 = {Linear diffuser
Jet flow
Turbulent models
Reynolds Averaged Navier Stokes equations
Nonlinear RANS
Spreading rate},
  Type                     = {Journal Article},
  Url                      = {http://dx.doi.org/10.1111/ina.12050}
}

@Article{Heschl2014b,
  Title                    = {Nonlinear eddy viscosity modeling and experimental study of jet spreading rates},
  Author                   = {Heschl, C. and Inthavong, K. and Sanz, W. and Tu, J.},
  Journal                  = {Indoor Air},
  Year                     = {2014},
  Pages                    = {93-102},
  Volume                   = {24},

  Doi                      = {10.1111/ina.12050},
  ISSN                     = {1600-0668},
  Keywords                 = {Averaged
diffuser
equations
flow
Jet
Linear
models
Navier
Nonlinear
RANS
rate
Reynolds
Spreading
Stokes
Turbulent},
  Type                     = {Journal Article},
  Url                      = {http://onlinelibrary.wiley.com/store/10.1111/ina.12050/asset/ina12050.pdf?v=1&t=i0df60an&s=71f6d141da4fbec82cf782dd6653813941ab10af}
}

@Article{Heschl2013,
  Title                    = {Evaluation and improvements of RANS turbulence models for linear diffuser flows},
  Author                   = {Heschl, Christian and Inthavong, Kiao and Sanz, Wolfgang and Tu, Jiyuan},
  Journal                  = {Computers and Fluids},
  Year                     = {2013},
  Number                   = {0},
  Pages                    = {272-282},
  Volume                   = {71},

  Doi                      = {10.1016/j.compfluid.2012.10.015},
  ISSN                     = {0045-7930},
  Keywords                 = {CFD
Mixed ventilation
Indoor airflow
Anisotropic
RANS},
  Type                     = {Journal Article},
  Url                      = {http://www.sciencedirect.com/science/article/pii/S0045793012004045}
}

@InProceedings{Heschl,
  Title                    = {Demands on turbulence modelling for ventilated room airflows},
  Author                   = {Heschl, Christian and Sanz, Wolfgang and Lindmeier, Ines },
  Booktitle                = {Seventh International Conference on CFD in the Minerals and Process Industries},
  Editor                   = {Schwarz, P. and Witt, P.},
  Publisher                = {CSIRO, Australia},

  Type                     = {Conference Proceedings}
}

@Article{Hesterberg1993,
  Title                    = {Chronic inhalation toxicity of size-separated glass fibers in Fischer 344 rats},
  Author                   = {Hesterberg, T. W. and Miiller, W. C. and McConnell, E. E. and Chevalier, J. and Hadley, J. G. and Bernstein, D. M. and Thevenaz, P. and Anderson, R.},
  Journal                  = {Fundam Appl Toxicol},
  Year                     = {1993},
  Number                   = {4},
  Pages                    = {464-76},
  Volume                   = {20},

  ISSN                     = {0272-0590 (Print)
0272-0590 (Linking)},
  Keywords                 = {Aerosols
Animals
Asbestos/toxicity
Asbestos, Serpentine
Body Burden
Ceramics/toxicity
Glass
Lung Diseases/ chemically induced/pathology
Lung Neoplasms/chemically induced
Male
Particle Size
Pulmonary Fibrosis/pathology
Rats
Rats, Inbred F344},
  Type                     = {Journal Article}
}

@Book{Hetsroni1971,
  Title                    = {A Second-Order Theory for a Deformable Drop Suspended in a Long Conduit},
  Author                   = {Hetsroni and Haber, G.S. and Brenner, H. and Greenstein, T. },
  Publisher                = {Pergamon},
  Year                     = {1971},

  Address                  = {Oxford \& New York, },
  Series                   = {In Progress in Heat and Mass Transfer, edited by G. Hetsroni, },
  Volume                   = {6},

  Type                     = {Book}
}

@Article{Heyder2004,
  Title                    = {Deposition of Inhaled Particles in the Human Respiratory Tract and Consequences for Regional Targeting in Respiratory Drug Delivery},
  Author                   = {Heyder, Joachim},
  Journal                  = {Proc Am Thorac Soc},
  Year                     = {2004},
  Number                   = {4},
  Pages                    = {315-320},
  Volume                   = {1},

  Abstract                 = {Particle behavior in the human respiratory tract is well understood and can be used to (1) estimate particle deposition in all regions of the respiratory tract for any aerosol respired at any pattern, and (2) optimize targeting of all regions of the respiratory tract in respiratory drug delivery. Extrathoracic and alveolar regions can effectively be targeted with mono- and polydisperse aerosols respired steadily. Effective targeting of the bronchial region can only be achieved with bolus inhalations. When particles are suspended in a gas heavier than air, targeting the alveolar region can be enhanced.},
  Doi                      = {10.1513/pats.200409-046TA},
  Type                     = {Journal Article},
  Url                      = {http://pats.atsjournals.org/cgi/content/abstract/1/4/315}
}

@Article{Heyder1986,
  Title                    = {Deposition of particles in the human respiratory tract in the size range 0.005-15 [mu]m},
  Author                   = {Heyder, J. and Gebhart, J. and Rudolf, G. and Schiller, C. F. and Stahlhofen, W.},
  Journal                  = {Journal of Aerosol Science},
  Year                     = {1986},
  Number                   = {5},
  Pages                    = {811-825},
  Volume                   = {17},

  ISSN                     = {0021-8502},
  Type                     = {Journal Article},
  Url                      = {http://www.sciencedirect.com/science/article/pii/0021850286900352}
}

@InBook{Heyder1977,
  Title                    = {Depostion of aerosol particles in the human nose},
  Author                   = {Heyder, J. and Rudolf, G.},
  Editor                   = {Walton, W.H.},
  Pages                    = {107-125},
  Publisher                = {Pergamon Press},
  Year                     = {1977},

  Address                  = {Oxford UK},
  Type                     = {Book Section},
  Volume                   = {IV},

  Booktitle                = {Inhaled particles}
}

@Book{Hidy1984,
  Title                    = {Aerosols, an Industrial and Environmental Science},
  Author                   = {Hidy, G. M. },
  Publisher                = {Academic Press},
  Year                     = {1984},

  Address                  = {New York. },

  Type                     = {Book}
}

@Book{Hilberg1989,
  Title                    = {Acoustic rhinometry: evaluation of nasal cavity geometry by acoustic reflection},
  Author                   = {Hilberg, O. and Jackson, A. C. and Swift, D. L. and Pedersen, O. F.},
  Year                     = {1989},
  Volume                   = {66},

  Abstract                 = {O. Hilberg, A. C. Jackson, D. L. Swift, and O. F. PedersenInstitute of Environmental and Occupational Medicine, University of Aarhus, Denmark.AbstractTo study the geometry of the nasal cavity we applied an acoustic method (J. Appl. Physiol. 43: 523–536, 1977) providing an estimate of cross-sectional area as a function of distance. Acoustic areas in a model constructed from a human nasal cast, in the nasal cavity of a cadaver and in 10 normal subjects and two patients with well-defined afflictions of the nasal cavity, were compared with similar areas obtained by computerized tomography (CT) scans, a specially developed water displacement method, and anterior rhinomanometry. We found a coefficient of variation of the areas of less than 2% by the acoustic method compared with 15% for the rhinomanometric measurements. Acoustic areas correlated highly to similar areas obtained by CT scanning (r = 0.94) and by water displacement (r = 0.96). In two patients the acoustic method accurately outlined, respectively, a tumor in the nose and a septum deviation. It is concluded that this method provides an accurate method for measuring the geometry of the nasal cavity. It is easy to perform and is potentially useful for investigation of physiological and pathological changes in the nose. Copyright © 1989 the American Physiological Society},
  Pages                    = {295-303},
  Type                     = {Book},
  Url                      = {http://jap.physiology.org/jap/66/1/295.full.pdf}
}

@Article{Hilberg1993,
  Title                    = {Nasal airway geometry: comparison between acoustic reflections and magnetic resonance scanning},
  Author                   = {Hilberg, O. and Jensen, F.T. and Pedersen, O.F.},
  Journal                  = {Journal Appl. Physiol.},
  Year                     = {1993},
  Number                   = {6},
  Pages                    = {2811-2819},
  Volume                   = { 75},

  Type                     = {Journal Article}
}

@Article{Hinchcliffe1999,
  Title                    = {Intranasal insulin delivery and therapy},
  Author                   = {Hinchcliffe, M. and Illum, L.},
  Journal                  = {Adv. Drug Delivery Rev.},
  Year                     = {1999},
  Number                   = {199-234},
  Volume                   = {35},

  Type                     = {Journal Article}
}

@Book{Hinds1982,
  Title                    = {Aerosol Technology, Properties, Behavior, and Measurement of Airborne Particles},
  Author                   = {Hinds, W.C. },
  Publisher                = {John Wiley and Sons},
  Year                     = {1982},

  Address                  = {New York},

  Type                     = {Book}
}

@Book{Hinze1975,
  Title                    = {Turbulence},
  Author                   = {Hinze, J.O.},
  Publisher                = {McGraw-Hill Publishing Co.},
  Year                     = {1975},

  Address                  = {New York},

  Type                     = {Book}
}

@Article{Hofemeier2014,
  Title                    = {Role of Alveolar Topology on Acinar Flows and Convective Mixing},
  Author                   = {Hofemeier, Philipp and Sznitman, Josue},
  Journal                  = {Journal of Biomechanical Engineering-Transactions of the Asme},
  Year                     = {2014},
  Note                     = {Times Cited: 0
0},
  Number                   = {6},
  Volume                   = {136},

  Doi                      = {10.1115/1.4027328},
  ISSN                     = {0148-0731; 1528-8951},
  Type                     = {Journal Article},
  Url                      = {<Go to ISI>://WOS:000335894800008
http://biomechanical.asmedigitalcollection.asme.org/article.aspx?articleid=1857503}
}

@Article{Hofmann,
  Title                    = {Modelling inhaled particle deposition in the human lung},
  Author                   = {Hofmann, Werner},
  Journal                  = {Journal of Aerosol Science},
  Volume                   = {In Press, Accepted Manuscript},

  Abstract                 = {Particle deposition in the human respiratory tract is determined by biological factors such as lung morphology and breathing patterns, and physical factors such as fluid dynamics, particle properties, and deposition mechanisms. Current particle deposition models may be grouped into two categories referring to the region of interest in the lung, i.e. either deposition in the whole lung (whole lung models), or deposition in a localized region of the lung (local scale models). In whole lung models, particle deposition in individual airways is computed by analytical equations for particle deposition efficiencies and specific flow conditions (analytical models). The present review focuses upon the philosophy of different conceptual whole lung models to determine deposition in bronchial and acinar airway generations, and to compare the deposition patterns predicted by these models. Since any modeling approach requires validation by comparison with the available experimental evidence, predicted deposition data are compared with published experimental data in human subjects. This comparison indicates that, at least during the writing of this review, deposition models can be validated only for total and, to some extent, for regional deposition. In local scale models, particle transport and deposition equations are solved by Computational Fluid and Particle Dynamics (CFPD) methods (numerical models), providing information on particle deposition patterns within selected structural elements of the lung, e.g. bronchial bifurcations. In this review, however, only their potential contribution to improve upon current analytical whole lung models will be considered.},
  Doi                      = {10.1016/j.jaerosci.2011.05.007},
  ISSN                     = {0021-8502},
  Keywords                 = {Human lung
Inhalation
Aerosol
Deposition
Modelling},
  Type                     = {Journal Article},
  Url                      = {http://www.sciencedirect.com/science/article/pii/S0021850211000875}
}

@Article{Hofmann2002,
  Title                    = {Modeling intersubject variability of particle deposition in human lungs},
  Author                   = {Hofmann, W. and Asgharian, B. and Winkler-Heil, R.},
  Journal                  = {Journal of Aerosol Science},
  Year                     = {2002},
  Note                     = {doi: DOI: 10.1016/S0021-8502(01)00167-7},
  Number                   = {2},
  Pages                    = {219-235},
  Volume                   = {33},

  ISSN                     = {0021-8502},
  Type                     = {Journal Article},
  Url                      = {http://www.sciencedirect.com/science/article/B6V6B-44HS8SG-2/2/d0f4d798a0f650c193cee878714f5515}
}

@Article{Holden1999,
  Title                    = {Temperature conditioning of nasal air: effects of vasoactive agents and involvement of nitric oxide},
  Author                   = {Holden, W.E. and Wilkins, J.P. and Harris, M. and Milczuk, H.A. and Giraud, G.D.},
  Journal                  = {Journal of Applied Physiology},
  Year                     = {1999},
  Number                   = {4},
  Pages                    = {1260-1265},
  Volume                   = {87},

  Type                     = {Journal Article}
}

@Article{Holmberg1998,
  Title                    = {Modelling of the indoor environment - particle dispersion and deposition.},
  Author                   = {Holmberg, S. and Li, Y.},
  Journal                  = {Indoor Air},
  Year                     = {1998},
  Number                   = {2},
  Pages                    = {113-122},
  Volume                   = {8},

  Type                     = {Journal Article}
}

@Article{Holton2013,
  Title                    = {The Morphological Interaction Between the Nasal Cavity and Maxillary Sinuses in Living Humans},
  Author                   = {Holton, Nathan and Yokley, Todd and Butaric, Lauren},
  Journal                  = {The Anatomical Record},
  Year                     = {2013},
  Number                   = {3},
  Pages                    = {414--426},
  Volume                   = {296},

  Doi                      = {10.1002/ar.22655},
  ISSN                     = {1932-8494},
  Keywords                 = {climate, computed tomography, human variation, pneumatization},
  Url                      = {http://dx.doi.org/10.1002/ar.22655}
}

@Article{Hood2005,
  Title                    = {Quantitative Analysis of the Low Molecular Weight Serum Proteome Using 18O Stable Isotope Labeling in a Lung Tumor Xenograft Mouse Model},
  Author                   = {Hood, Brian L. and Lucas, David A. and Kim, Grace and Chan, King C. and Blonder, Josip and Issaq, Haleem J. and Veenstra, Timothy D. and Conrads, Thomas P. and Pollet, Ingrid and Karsan, Aly},
  Journal                  = {Journal of the American Society for Mass Spectrometry},
  Year                     = {2005},
  Number                   = {8},
  Pages                    = {1221-1230},
  Volume                   = {16},

  ISSN                     = {1044-0305},
  Type                     = {Journal Article},
  Url                      = {http://www.sciencedirect.com/science/article/B6TH2-4GG8VYJ-1/2/3d7e2e57a44c59b33fd859c1eb1af5e0}
}

@Article{Hood2009,
  Title                    = {Computational modeling of flow and gas exchange in models of the human maxillary sinus},
  Author                   = {Hood, C. M. and Schroter, R. C. and Doorly, D. J. and Blenke, E. J. S. M. and Tolley, N. S.},
  Journal                  = {Journal of Applied Physiology},
  Year                     = {2009},
  Number                   = {4},
  Pages                    = {1195-1203},
  Volume                   = {107},

  Abstract                 = {The present study uses numerical modeling to increase the understanding of sinus gas exchange, which is thought to be a factor in sinus disease. Order-of-magnitude estimates and computational fluid dynamics simulations were used to investigate convective and diffusive transport between the nose and the sinus in a range of simplified geometries. The interaction between mucociliary transport and gas exchange was modeled and found to be negligible. Diffusion was the dominant transport mechanism for small ostia and large concentration differences between the sinus and the nose, whereas convection was important for larger ostia or smaller concentration differences. The presence of one or more accessory ostia can increase the sinus ventilation rate by several orders of magnitude, because it allows a net flow through the sinus. Estimates of nitric oxide (NO) transport through the ostium based on measured sinus and nasal NO concentrations suggest that the sinuses cannot supply all the NO in nasally exhaled air.},
  Doi                      = {10.1152/japplphysiol.91615.2008},
  Type                     = {Journal Article},
  Url                      = {http://jap.physiology.org/cgi/content/abstract/107/4/1195}
}

@Article{Hood2009a,
  Title                    = {Computational modeling of flow and gas exchange in models of the human maxillary sinus},
  Author                   = {Hood, C. M. and Schroter, R. C. and Doorly, D. J. and Blenke, E. J. S. M. and Tolley, N. S.},
  Journal                  = {JOURNAL OF APPLIED PHYSIOLOGY},
  Year                     = {2009},
  Pages                    = {1195-1203},
  Volume                   = {107},

  Abstract                 = {Hood CM, Schroter RC, Doorly DJ, Blenke EJ, Tolley NS. Computational modeling of flow and gas exchange in models of the human maxillary sinus. J Appl Physiol 107: 1195-1203, 2009. First published July 16, 2009; doi:10.1152/japplphysiol.91615.2008.-The present study uses numerical modeling to increase the understanding of sinus gas exchange, which is thought to be a factor in sinus disease. Order-of-magnitude estimates and computational fluid dynamics simulations were used to investigate convective and diffusive transport between the nose and the sinus in a range of simplified geometries. The interaction between mucociliary transport and gas exchange was modeled and found to be negligible. Diffusion was the dominant transport mechanism for small ostia and large concentration differences between the sinus and the nose, whereas convection was important for larger ostia or smaller concentration differences. The presence of one or more accessory ostia can increase the sinus ventilation rate by several orders of magnitude, because it allows a net flow through the sinus. Estimates of nitric oxide (NO) transport through the ostium based on measured sinus and nasal NO concentrations suggest that the sinuses cannot supply all the NO in nasally exhaled air.},
  Doi                      = {10.1152/japplphysiol.91615.2008},
  ISSN                     = {8750-7587},
  Keywords                 = {Airflow
computational fluid dynamics
nitric oxide
Nose},
  Type                     = {Journal Article},
  Url                      = {http://jap.physiology.org/content/jap/107/4/1195.full.pdf}
}

@Article{Hooff2013,
  Title                    = {On the suitability of steady RANS CFD for forced mixing ventilation at transitional slot Reynolds numbers},
  Author                   = {van Hooff, T. and Blocken, B. and van Heijst, G. J.},
  Journal                  = {Indoor Air},
  Year                     = {2013},
  Note                     = {van Hooff, T
Blocken, B
van Heijst, G J F
England
Indoor Air. 2013 Jun;23(3):236-49. doi: 10.1111/ina.12010. Epub 2012 Nov 29.},
  Number                   = {3},
  Pages                    = {236-49},
  Volume                   = {23},

  Abstract                 = {Accurate prediction of ventilation flow is of primary importance for designing a healthy, comfortable, and energy-efficient indoor environment. Since the 1970s, the use of computational fluid dynamics (CFD) has increased tremendously, and nowadays, it is one of the primary methods to assess ventilation flow in buildings. The most commonly used numerical approach consists of solving the steady Reynolds-averaged Navier-Stokes (RANS) equations with a turbulence model to provide closure. This article presents a detailed validation study of steady RANS for isothermal forced mixing ventilation of a cubical enclosure driven by a transitional wall jet. The validation is performed using particle image velocimetry (PIV) measurements for slot Reynolds numbers of 1000 and 2500. Results obtained with the renormalization group (RNG) k-epsilon model, a low-Reynolds k-epsilon model, the shear stress transport (SST) k-omega model, and a Reynolds stress model (RSM) are compared with detailed experimental data. In general, the RNG k-epsilon model shows the weakest performance, whereas the low-Re k-epsilon model shows the best agreement with the measurements. In addition, the influence of the turbulence model on the predicted air exchange efficiency in the cubical enclosure is analyzed, indicating differences up to 44% for this particular case.},
  Doi                      = {10.1111/ina.12010},
  ISSN                     = {1600-0668 (Electronic)
0905-6947 (Linking)},
  Type                     = {Journal Article}
}

@Article{Hopkins2014,
  Title                    = {Nose-to-brain transport of aerosolised quantum dots following acute exposure},
  Author                   = {Hopkins, Laurie E. and Patchin, Esther S. and Chiu, Po-Lin and Brandenberger, Christina and Smiley-Jewell, Suzette and Pinkerton, Kent E.},
  Journal                  = {Nanotoxicology},
  Year                     = {2014},
  Note                     = {C:\Users\sean\AppData\Roaming\Zotero\Zotero\Profiles\16a4oype.default\zotero\storage\EJHTBJN8\Hopkins et al. - 2013 - Nose-to-brain transport of aerosolised quantum dot.pdf},
  Pages                    = {885-893},
  Volume                   = {8},

  Abstract                 = {Nanoparticles are of wide interest due to their potential use for diverse commercial applications. Quantum dots (QDs) are semiconductor nanocrystals possessing unique optical and electrical properties. Although QDs are commonly made of cadmium, a metal known to have neurological effects, potential transport of QDs directly to the brain has not been assessed. This study evaluated whether QDs (CdSe/ZnS nanocrystals) could be transported from the olfactory tract to the brain via inhalation. Adult C57BL/6 mice were exposed to an aerosol of QDs for 1 h via nasal inhalation, and nanoparticles were detected 3 h post-exposure within the olfactory tract and olfactory bulb by a wide range of techniques, including visualisation via fluorescent and transmission electron microscopy. We conclude that, following short-term inhalation of solid QD nanoparticles, there is rapid olfactory uptake and axonal transport to the brain/olfactory bulb with observed activation of microglial cells, indicating a pro-inflammatory response. To our knowledge, this is the first study to clearly demonstrate that QDs can be rapidly transported from the nose to the brain by olfactory uptake via axonal transport following inhalation.},
  Doi                      = {10.3109/17435390.2013.842267},
  ISSN                     = {1743-5390},
  Keywords                 = {activation
cadmium
central-nervous-system
inhalation
manganese
microglia
mouse-brain
olfactory bulb
olfactory epithelium
particles
qdots
translocation},
  Type                     = {Journal Article},
  Url                      = {http://informahealthcare.com/doi/pdfplus/10.3109/17435390.2013.842267}
}

@Article{Hoppel1986,
  Title                    = {Ion—Aerosol Attachment Coefficients and the Steady-State Charge Distribution on Aerosols in a Bipolar Ion Environment},
  Author                   = {Hoppel, William A. and Frick, Glendon M.},
  Journal                  = {Aerosol Science and Technology},
  Year                     = {1986},
  Number                   = {1},
  Pages                    = {1-21},
  Volume                   = {5},

  ISSN                     = {0278-6826},
  Type                     = {Journal Article},
  Url                      = {http://www.informaworld.com/10.1080/02786828608959073}
}

@Article{Horn2008,
  Title                    = {A comprehensive approach in modeling Lagrangian particle deposition in turbulent boundary layers},
  Author                   = {Horn, M. and Schmid, H. J.},
  Journal                  = {Powder Technology},
  Year                     = {2008},
  Note                     = {doi: DOI: 10.1016/j.powtec.2007.11.048},
  Number                   = {3},
  Pages                    = {189-198},
  Volume                   = {186},

  ISSN                     = {0032-5910},
  Keywords                 = {Euler
Lagrange
Modeling
Turbulence
Boundary layer
Deposition},
  Type                     = {Journal Article},
  Url                      = {http://www.sciencedirect.com/science/article/B6TH9-4R98K8D-1/2/8bafd58885ed5f1ff78f2178f79b0555}
}

@Article{Hornung1987,
  Title                    = {Airflow patterns in a human nasal model},
  Author                   = {Hornung, D.E. and Leopold, D.A. and Youngentob, S.L. and Sheehe, P.R. and Gagne, G.M. and Thomas, F.D. and Mozell, M.M. },
  Journal                  = {Arch Otolaryngol Head Neck Surg},
  Year                     = {1987},
  Pages                    = {169-172},
  Volume                   = {113},

  Type                     = {Journal Article}
}

@Article{Horschler2003,
  Title                    = {Numerical simulation of the flow field in a model of the nasal cavity},
  Author                   = {Horschler, I. and Meinke, M. and Schraeder, W.},
  Journal                  = {Comp. Fluids},
  Year                     = {2003},
  Number                   = {1},
  Pages                    = {39-45},
  Volume                   = {32},

  Type                     = {Journal Article}
}

@Article{Horvat2001,
  Title                    = {Two-dimensional large-eddy simulation of turbulent natural convection due to internal heat generation},
  Author                   = {Horvat, Andrej and Kljenak, Ivo and Marn, Jure},
  Journal                  = {International Journal of Heat and Mass Transfer},
  Year                     = {2001},
  Number                   = {21},
  Pages                    = {3985-3995},
  Volume                   = {44},

  ISSN                     = {0017-9310},
  Keywords                 = {Heat transfer
Natural convection
Turbulence},
  Type                     = {Journal Article},
  Url                      = {http://www.sciencedirect.com/science/article/B6V3H-437XPDK-1/2/7abbd935f31e4197a2c58a8f1e1866be}
}

@Misc{Hossain2007,
  Title                    = {CFD investigation of particle deposition around bends in a turbulent flow},

  Author                   = {Hossain, A. and Naser, J.},
  Year                     = {2007},

  ISBN                     = {978-1-864998-94-8 },
  Publisher                = {School of Engineering, The University of Queensland },
  Type                     = {Conference Paper}
}

@Article{Hounam1971,
  Title                    = {Deposition of aerosol particles in the nasopharyngeal region of the human respiratory tract},
  Author                   = {Hounam, R.F. and Black, A. and Walsh, M.},
  Journal                  = {Journal of Aerosol Science},
  Year                     = {1971},
  Pages                    = {341-352},
  Volume                   = {2},

  Type                     = {Journal Article}
}

@Article{Hounam1969,
  Title                    = {Deposition of Aerosol Particles in the Nasopharyngeal Region of the Human Respiratory Tract},
  Author                   = {Hounam, R. F. and Black, A. and Walsh, M.},
  Journal                  = {Nature},
  Year                     = {1969},
  Note                     = {10.1038/2211254a0},
  Number                   = {5187},
  Pages                    = {1254-1255},
  Volume                   = {221},

  Type                     = {Journal Article},
  Url                      = {http://dx.doi.org/10.1038/2211254a0}
}

@Article{Hout2004,
  Title                    = {A method for measuring the density of irregularly shaped biological aerosols such as pollen},
  Author                   = {van Hout, R. and Katz, J.},
  Journal                  = {Journal Aeros. Sci.},
  Year                     = {2004},
  Pages                    = {1369-1384},
  Volume                   = {35},

  Type                     = {Journal Article}
}

@Article{How2007,
  Title                    = {Acute and chronic responses of the upper airway to inspiratory loading in healthy awake humans: An MRI study},
  Author                   = {How, Stephen C. and McConnell, Alison K. and Taylor, Bryan J. and Romer, Lee M.},
  Journal                  = {Respiratory Physiology \& Neurobiology},
  Year                     = {2007},
  Note                     = {doi: DOI: 10.1016/j.resp.2007.01.008},
  Number                   = {2-3},
  Pages                    = {270-280},
  Volume                   = {157},

  ISSN                     = {1569-9048},
  Keywords                 = {Upper airway
Inspiratory muscle training
MRI},
  Type                     = {Journal Article},
  Url                      = {http://www.sciencedirect.com/science/article/B6X16-4MVVSRX-2/2/55c754f1263a3e2702de1a2f2370b305}
}

@Book{Howarth2001,
  Title                    = {Airway Remodeling},
  Author                   = {Howarth, J. and Wilson, J. and Bousquet, S. and Rak, R. and Pauwels, R. },
  Publisher                = {Marcel Dekker},
  Year                     = {2001},

  Address                  = {New York},
  Series                   = {Lung Biology in Health and Disease},

  Type                     = {Edited Book}
}

@Article{Hsieh2014,
  Title                    = {In vitro measurement and dynamic modeling-based approaches for deposition risk assessment of inhaled aerosols in human respiratory system},
  Author                   = {Hsieh, Nan-Hung and Liao, Chung-Min},
  Journal                  = {Atmospheric Environment},
  Year                     = {2014},
  Note                     = {Times Cited: 0
0},
  Pages                    = {268-276},
  Volume                   = {95},

  ISSN                     = {1352-2310},
  Type                     = {Journal Article},
  Url                      = {<Go to ISI>://CCC:000340977400029}
}

@Article{Hsu1999,
  Title                    = {The measurement of human inhalability of ultralarge aerosols in calm air using manikins},
  Author                   = {Hsu, D.J. and Swift, D.L.},
  Journal                  = {Journal Aerosol Science},
  Year                     = {1999},
  Pages                    = {1331-1343},
  Volume                   = {30},

  Type                     = {Journal Article}
}

@Article{Huang2007,
  Title                    = {Field comparison of real-time pm2.5 readings from a beta gauge monitor and a light scattering method},
  Author                   = {Huang, C.H.},
  Journal                  = {Aerosol and Air Quality Research},
  Year                     = {2007},
  Number                   = {2},
  Pages                    = {239-250},
  Volume                   = {7},

  Type                     = {Journal Article}
}

@Article{Huang2013,
  Title                    = {Moving boundary simulation of airflow and micro-particle deposition in the human extra-thoracic airway under steady inspiration. Part I: Airflow},
  Author                   = {Huang, Jianhua and Sun, Hui and Liu, Canjie and Zhang, Lianzhong},
  Journal                  = {European Journal of Mechanics - B/Fluids},
  Year                     = {2013},
  Number                   = {0},
  Pages                    = {29-41},
  Volume                   = {37},

  Doi                      = {http://dx.doi.org/10.1016/j.euromechflu.2012.05.003},
  ISSN                     = {0997-7546},
  Keywords                 = {Extra-thoracic airway (ETA) model
Mesh deformation
Moving wall boundary
Secondary intensity
Recirculation flow},
  Type                     = {Journal Article},
  Url                      = {http://www.sciencedirect.com/science/article/pii/S0997754612000817}
}

@Article{Huang,
  Title                    = {Numerical simulation of micro-particle deposition in a realistic human upper respiratory tract model during transient breathing cycle},
  Author                   = {Huang, Jianhua and Zhang, Lianzhong},
  Journal                  = {Particuology},
  Volume                   = {In Press, Corrected Proof},

  Abstract                 = {An more reliable human upper respiratory tract model that consisted of an oropharynx and four generations of asymmetric tracheo-bronchial (TB) airways has been constructed to investigate the micro-particle deposition pattern and mass distribution in five lobes under steady inspiratory condition in former work by Huang and Zhang (2011). In the present work, transient airflow patterns and particle deposition during both inspiratory and expiratory processes were numerically simulated in the realistic human upper respiratory tract model with 14 cartilaginous rings (CRs) in the tracheal tube. The present model was validated under steady inspiratory flow rates by comparing current results with the theoretical models and published experimental data. The transient deposition fraction was found to strongly depend on breathing flow rate and particle diameter but slightly on turbulence intensity. Particles were mainly distributed in the high axial speed zones and traveled basically following the secondary flow. "Hot spots" of deposition were found in the lower portion of mouth cavity and posterior wall of pharynx/larynx during inspiration, but transferred to upper portion of mouth and interior wall of pharynx/larynx during expiration. The deposition fraction in the trachea during expiration was found to be much higher than that during inspiration because of the stronger secondary flow.},
  Doi                      = {10.1016/j.partic.2011.02.004},
  ISSN                     = {1674-2001},
  Keywords                 = {Human upper respiratory tract
Transient breathing cycle
Airflow pattern
Micro-particle deposition},
  Type                     = {Journal Article},
  Url                      = {http://www.sciencedirect.com/science/article/pii/S1674200111000666}
}

@Article{Huang2006,
  Title                    = {An acoustic study on the wave speed mechanism of flow limitation through a flexible channel},
  Author                   = {Huang, L.},
  Journal                  = {Journal of Biomechanics},
  Year                     = {2006},
  Number                   = {Supplement 1},
  Pages                    = {S442-S442},
  Volume                   = {39},

  ISSN                     = {0021-9290},
  Type                     = {Journal Article},
  Url                      = {http://www.sciencedirect.com/science/article/B6T82-4KR88PB-2F6/2/b2693ba23c1fcb478427d4116dc8b05d}
}

@Article{Huang2003,
  Title                    = {Assessment of nasal cycle by acoustic rhinometry and rhinomanometry},
  Author                   = {Huang, Zhi L. i and Ong, Kee Leong and Goh, Sze Y. i and Liew, Han Lim and Yeoh, Kian Hian and Wang, D. e Yun},
  Journal                  = {Otolaryngology - Head and Neck Surgery},
  Year                     = {2003},
  Note                     = {doi: DOI: 10.1016/S0194-5998(03)00123-2},
  Number                   = {4},
  Pages                    = {510-516},
  Volume                   = {128},

  ISSN                     = {0194-5998},
  Type                     = {Journal Article},
  Url                      = {http://www.sciencedirect.com/science/article/B6WP4-48BKW18-C/2/2aeabfa76d0b9348149eb187842ce190}
}

@Article{HuaZhua2010,
  Title                    = {Evaluation and comparison of nasal airway flow pattern samong three subjects from Caucasian,Chinese and Indian ethnic groups using computational fluid dynamics simulation},
  Author                   = {Jian HuaZhua and HeowPuehLeea, KianMengLima, ShuJinLeeb, DeYunWangc},
  Journal                  = {Respiratory Physiology \& Neurobiology},
  Year                     = {2010},

  Type                     = {Journal Article}
}

@Article{Hubbard1994,
  Title                    = {A Chimera Scheme for Incompressible Viscous Flows with Applications to Submarine Hydrodynamics},
  Author                   = {Hubbard, B.J. and Chen, H.C.},
  Journal                  = {AIAA },
  Year                     = {1994},
  Pages                    = {Paper 94-2210},

  Type                     = {Journal Article}
}

@Misc{Hubbard1995,
  Title                    = {Calculation of Unsteady Flows around Bodies with Relative Motion Using a Chimera RANS Method},

  Author                   = {Hubbard, B.J. and Chen, H. C.},
  Month                    = {May 21-24},
  Year                     = {1995},

  Pages                    = {7832-785},
  Type                     = {Conference Paper}
}

@Article{Hursthouse2004,
  Title                    = {A pilot study of personal exposure to respirable and inhalable dust during the sanding and sawing of medium density fibreboard (MDF) and soft wood},
  Author                   = {Hursthouse, Andrew and Allan, Fraser and Rowley, Louise and Smith, Frank},
  Journal                  = {International Journal of Environmental Health Research},
  Year                     = {2004},
  Number                   = {4},
  Pages                    = {323 - 326},
  Volume                   = {14},

  ISSN                     = {0960-3123},
  Type                     = {Journal Article},
  Url                      = {http://www.informaworld.com/10.1080/09603120410001725667}
}

@Book{Huston2009,
  Title                    = {Principles of biomechanics},
  Author                   = {Huston, R.L.},
  Publisher                = {CRC Press},
  Year                     = {2009},

  ISBN                     = {9780849334948},
  Type                     = {Book},
  Url                      = {http://books.google.com.au/books?id=H-bw7OY7TokC}
}

@Article{Huupponen2009,
  Title                    = {Improved computational fronto-central sleep depth parameters show differences between apnea patients and control subjects},
  Author                   = {Huupponen, E. and Saunamäki, T. and Saastamoinen, A. and Kulkas, A. and Tenhunen, M. and Himanen, S. L.},
  Journal                  = {Medical and Biological Engineering and Computing},
  Year                     = {2009},
  Note                     = {10.1007/s11517-008-0374-3},
  Number                   = {1},
  Pages                    = {3-10},
  Volume                   = {47},

  Abstract                 = {Abstract&nbsp;&nbsp;All-night EEG recordings from 12 male apnea patients and 12 age-matched healthy control subjects were studied in the present work. The spectral mean frequency was used to provide computational sleep depth curves from two frontopolar and two central EEG channels. Our previously presented computational parameters quantifying the properties of the sleep depth curves were improved. The resulting light sleep percentage (LS%) values were higher in apnea patients than in control subjects in the right central brain position (P&nbsp;=&nbsp;0.028), in concordance to our previous work. Moreover, apnea patients showed higher LS% values in the right frontopolar position (P&nbsp;=&nbsp;0.008). Also, apnea patients showed a smaller anteroposterior sleep depth difference than control subjects on the right hemisphere (P&nbsp;=&nbsp;0.002). These are interesting new findings, achieved by the present methodology. Thus, the developed computational parameters were able to quantify, at least to some degree, the disruption of sleep process caused by the recurrent apneic events.},
  Type                     = {Journal Article},
  Url                      = {http://dx.doi.org/10.1007/s11517-008-0374-3}
}

@Misc{Hwang1978,
  Title                    = {Quantitative Cardiovascular Studies: Clinical and Research Applications of Engineering Principles},

  Author                   = {Hwang, N.H.C. and Gross, D.R. and Patel, D.J.},
  Year                     = {1978},

  Pages                    = {289-351},
  Publisher                = {University Park Press},
  Type                     = {Generic}
}

@TechReport{IARC/WHO1995,
  Title                    = {I.A.R.C Monographs on the Evaluation of Carcinogenic Risks to Humans. Vol. 62: Wood Dust and Formaldehyde. International Agency for Research on Cancer/World Health Organization },
  Author                   = {IARC/WHO},
  Year                     = {1995},
  Type                     = {Report}
}

@Article{Ibrahim2003,
  Title                    = {Microparticle detachment from surfaces exposed to turbulent air flow: controlled experiments and modeling},
  Author                   = {Ibrahim, A. H. and Dunn, P. F. and Brach, R. M.},
  Journal                  = {Journal of Aerosol Science},
  Year                     = {2003},
  Number                   = {6},
  Pages                    = {765-782},
  Volume                   = {34},

  ISSN                     = {0021-8502},
  Keywords                 = {Adhesion to surfaces
Detachment from surfaces
Particle adhesion
Surface forces},
  Type                     = {Journal Article},
  Url                      = {http://www.sciencedirect.com/science/article/pii/S0021850203000314}
}

@TechReport{ICRP1994,
  Title                    = {Human respiratory tract model for radiological protection},
  Author                   = {ICRP},
  Institution              = {Elsevier Science},
  Year                     = {1994},
  Type                     = {Report},

  Url                      = {http://iopscience.iop.org.ezproxy.lib.rmit.edu.au/0952-4746/16/1/013/}
}

@Article{ICRP1960,
  Title                    = {Report of Committee II on permissible dose for internal radiation},
  Author                   = {ICRP},
  Journal                  = {Annals of the ICRP/ICRP Publication},
  Year                     = {1960},
  Pages                    = {1-40},
  Volume                   = {2},

  Type                     = {Journal Article}
}

@Article{Ijzerman1989,
  Title                    = {A refined method for the photoaffinity labelling of the nitrobenzylthioinosine-sensitive nucleoside transport protein: Application to cell membranes of calf lung tissue},
  Author                   = {Ijzerman, Adriaan P. and Menkveld, Gerda J. and Thedinga, Karen H.},
  Journal                  = {Biochimica et Biophysica Acta (BBA) - Biomembranes},
  Year                     = {1989},
  Number                   = {2},
  Pages                    = {153-156},
  Volume                   = {979},

  ISSN                     = {0005-2736},
  Keywords                 = {Nucleoside transport protein
Adenosine
Photoaffinity labeling
Nitrobenzylthioinosine
(Calf lung)},
  Type                     = {Journal Article},
  Url                      = {http://www.sciencedirect.com/science/article/B6T1T-48891KJ-J1/2/a310a90b5f275ca0afed1075032201cc}
}

@Misc{Ikeda1998,
  Title                    = {PIV Application for Spray Characteristic Measurement},

  Author                   = {Ikeda, Y. and Yamada, N. and Kawahara, N.},
  Year                     = {1998},

  Type                     = {Conference Paper}
}

@Article{Illum2012254,
  Title                    = {Nasal drug delivery Recent developments and future prospects },
  Author                   = {Lisbeth Illum},
  Journal                  = {Journal of Controlled Release },
  Year                     = {2012},
  Note                     = {Drug Delivery Research in Europe },
  Number                   = {2},
  Pages                    = {254 - 263},
  Volume                   = {161},

  Abstract                 = {The present review sets out to discuss recent developments and prospects of absorption promoters and absorption modulator systems being developed commercially by companies specialising in nasal drug delivery of normal small molecular weight drugs and biological drugs such as peptide and proteins. The absorption promoter systems selected for discussion in this review are those with the most promising preclinical and/or clinical data and sufficient toxicology data and/or company development efforts to warrant use in marketed products i.e. CPE-215 (cyclopenta decalactone (azone)) developed by \{CPEX\} Pharma, Intravail (alkylsaccharides) developed by Aegis Therapeutics, ChiSysTM (chitosan) and PecSysTM (low methylated pectin) in development by Archimedes Pharma and CriticalSorbTM (polyglycol mono- and diesters of 12-hydroxystearate (70%), polyethylene glycol (30%)) developed by Critical Pharmaceuticals. },
  Doi                      = {http://dx.doi.org/10.1016/j.jconrel.2012.01.024},
  ISSN                     = {0168-3659},
  Keywords                 = {Nasal absorption promoters},
  Url                      = {http://www.sciencedirect.com/science/article/pii/S0168365912000296}
}

@Article{Illum2002,
  Title                    = {Nasal drug delviery: new developments and strategies},
  Author                   = {Illum, L.},
  Journal                  = {Drug Discovery Today},
  Year                     = {2002},
  Number                   = {23},
  Pages                    = {1184-1189},
  Volume                   = {7},

  Type                     = {Journal Article}
}

@Article{Im2013,
  Title                    = {Tomographic PIV measurements of flow patterns in a nasal cavity with geometry acquisition},
  Author                   = {Im, Sunghyuk and Heo, GoEun and Jeon, YoungJin and Sung, HyungJin and Kim, SungKyun},
  Journal                  = {Experiments in Fluids},
  Year                     = {2013},
  Number                   = {1},
  Pages                    = {1-18},
  Volume                   = {55},

  Doi                      = {10.1007/s00348-013-1644-x},
  ISSN                     = {0723-4864},
  Type                     = {Journal Article},
  Url                      = {http://dx.doi.org/10.1007/s00348-013-1644-x}
}

@Article{Im2013a,
  Title                    = {Tomographic PIV measurements of flow patterns in a nasal cavity with geometry acquisition},
  Author                   = {Im, Sunghyuk and Heo, GoEun and Jeon, YoungJin and Sung, HyungJin and Kim, SungKyun},
  Journal                  = {Experiments in Fluids},
  Year                     = {2013},
  Note                     = {C:\Users\sean\AppData\Roaming\Zotero\Zotero\Profiles\16a4oype.default\zotero\storage\S3VV4I2V\Im et al. - 2014 - Tomographic PIV measurements of flow patterns in a.pdf
C:\Users\sean\AppData\Roaming\Zotero\Zotero\Profiles\16a4oype.default\zotero\storage\CS6ADSTW\Im et al. - 2014 - Tomographic PIV measurements of flow patterns in a.pdf},
  Pages                    = {1-18},
  Volume                   = {55},

  Doi                      = {10.1007/s00348-013-1644-x},
  ISSN                     = {0723-4864},
  Keywords                 = {Engineering Fluid Dynamics
Engineering Thermodynamics, Heat and Mass Transfer
Fluid- and Aerodynamics},
  Type                     = {Journal Article},
  Url                      = {http://download.springer.com/static/pdf/645/art%253A10.1007%252Fs00348-013-1644-x.pdf?auth66=1411539290_dc37b1e02371e7b14b65451b0be01ee9&ext=.pdf}
}

@Article{Ingham1975,
  Title                    = {Diffusion of aerosols from a stream flowing through a cylindrical tube},
  Author                   = {Ingham, D.B.},
  Journal                  = {Journal of Aerosol Science},
  Year                     = {1975},
  Number                   = {2},
  Pages                    = {125-132},
  Volume                   = {6},

  Type                     = {Journal Article}
}

@Article{Ingham1991,
  Title                    = {The fluid-flow into a blunt aerosol sampler oriented at an angle to the oncoming flow},
  Author                   = {Ingham, D.B. and Hildyard, M.L.},
  Journal                  = {Journal of Aerosol Science},
  Year                     = {1991},
  Number                   = {3},
  Pages                    = {235-252},
  Volume                   = {22},

  Type                     = {Journal Article}
}

@TechReport{Initiative2003,
  Title                    = {The 2002 non-melanocytic skin cancers survey, 1-51.2003 },
  Author                   = {National Cancer Control Initiative},
  Institution              = {National Cancer Control Initiative},
  Year                     = {2003},
  Type                     = {Report}
}

@Article{Inthavong2010,
  Title                    = {Micron particle deposition in a tracheobronchial airway model under different breathing conditions},
  Author                   = {Inthavong, K and Choi, L.T and Tu, J. Y and Ding, S and Thien, F},
  Journal                  = {Medical Engineering \& Physics},
  Year                     = {2010},
  Number                   = {10},
  Pages                    = {1198-1212},
  Volume                   = {32},

  Abstract                 = {Effective management of asthma is dependent on achieving adequate delivery of the drugs into the lung. Inhalers come in the form of dry powder inhalers (DPIs) and metered dose inhalers (pMDIs) with the former requiring a deep fast breath for activation while there are no restrictions on inhalation rates for the latter. This study investigates two aerosol medication delivery methods (i) an idealised case for drug particle delivery under a normal breathing cycle (inhalation-exhalation) and (ii) for an increased effort during the inhalation with a breath hold. A computational model of a human tracheobronchial airway was reconstructed from computerised tomography (CT) scans. The model's geometry and lobar flow distribution were compared with experimental and empirical models to verify the current model. Velocity contours and secondary flow vectors showed vortex formation downstream of the bifurcations which enhanced particle deposition. The velocity contour profiles served as a predictive tool for the final deposition patterns. Different spherical aerosol particle sizes (3-10 [mu]m, 1.55 g/cm3) were introduced into the airway for comparison over a range of Stokes number. It was found that a deep inhalation with a breath hold of 2 s did not necessarily increase later deposition up to the sixth branch generation, but rather there was an increase in the deposition in the first few airway generations was found. In addition the breath hold allows deposition by sedimentation which assists in locally targeted deposition. Visualisation of particle deposition showed local "hot-spots" where particle deposition was concentrated in the lung airway.},
  Doi                      = {10.1016/j.medengphy.2010.08.012},
  ISSN                     = {1350-4533},
  Keywords                 = {Airway
Particle deposition
Breathing
CFD},
  Type                     = {Journal Article},
  Url                      = {http://www.sciencedirect.com/science/article/pii/S1350453310001815}
}

@Article{Inthavong2014,
  Title                    = {High Resolution Visualization and Analysis of Nasal Spray Drug Delivery},
  Author                   = {Inthavong, Kiao and Fung, Man Chiu and Tong, Xuwen and Yang, William and Tu, Jiyuan},
  Journal                  = {Pharmaceutical research},
  Year                     = {2014},
  Pages                    = {1-8},

  ISSN                     = {0724-8741},
  Type                     = {Journal Article}
}

@Article{Inthavong2014a,
  Title                    = {Measurements of Droplet Size Distribution and Analysis of Nasal Spray Atomization from Different Actuation Pressure},
  Author                   = {Inthavong, Kiao and Fung, Man Chiu and Yang, William and Tu, Jiyuan},
  Journal                  = {Journal of aerosol medicine and pulmonary drug delivery},
  Year                     = {2014},

  ISSN                     = {1941-2711},
  Type                     = {Journal Article}
}

@Article{Inthavong2013,
  Title                    = {Source and trajectories of inhaled particles from a surrounding environment and its deposition in the respiratory airway},
  Author                   = {Inthavong, K. and Ge, Q.J. and Li, A. and Tu, J.Y.},
  Journal                  = {Inhalation Toxicology},
  Year                     = {2013},
  Number                   = {5},
  Pages                    = {280-291},
  Volume                   = {25},

  Doi                      = {doi:10.3109/08958378.2013.781250},
  Type                     = {Journal Article},
  Url                      = {http://informahealthcare.com/doi/abs/10.3109/08958378.2013.781250}
}

@Article{Inthavong2012,
  Title                    = {Detailed predictions of particle aspiration affected by respiratory inhalation and airflow},
  Author                   = {Inthavong, K and Ge, Q.J and Li, X.D and Tu, J.Y},
  Journal                  = {Atmospheric Environment},
  Year                     = {2012},
  Number                   = {0},
  Pages                    = {107-117},
  Volume                   = {62},

  Doi                      = {10.1016/j.atmosenv.2012.07.071},
  ISSN                     = {1352-2310},
  Keywords                 = {Air pollution
Particle
Deposition
Nasal cavity
Inhalation
CFD},
  Type                     = {Journal Article},
  Url                      = {http://www.sciencedirect.com/science/article/pii/S1352231012007571}
}

@Article{Inthavong2011,
  Title                    = {Simulation of sprayed particle deposition in a human nasal cavity including a nasal spray device},
  Author                   = {Inthavong, K. and Ge, Qinjiang and Se, Camby M. K. and Yang, W. and Tu, J. Y.},
  Journal                  = {Journal of Aerosol Science},
  Year                     = {2011},
  Number                   = {2},
  Pages                    = {100-113},
  Volume                   = {42},

  Doi                      = {DOI: 10.1016/j.jaerosci.2010.11.008},
  ISSN                     = {0021-8502},
  Keywords                 = {Nasal cavity
Spray
Deposition
CFD
Modelling
Simulation},
  Type                     = {Journal Article},
  Url                      = {http://www.sciencedirect.com/science/article/B6V6B-51N22D6-1/2/c10233d24d60c6f73ee29e3508b8d9db}
}

@Misc{Inthavong2012a,
  Title                    = {A computational platform for predicting nanoparticle deposition in the human respiratory system},

  Author                   = {Inthavong, K. and Ge, Q.J and Tu, J.Y.},
  Year                     = {2012},

  Pages                    = {Paper
O85},
  Type                     = {Conference Paper}
}

@Article{Inthavong2013a,
  Title                    = {Detailed predictions of particle aspiration affected by respiratory inhalation and airflow},
  Author                   = {Inthavong, Kiao and Ge, Qin Jiang and Li, Xiang Dong and Tu, Ji Yuan},
  Journal                  = {Atmospheric Environment},
  Year                     = {2013},
  Number                   = {0},
  Pages                    = {107-117},
  Volume                   = {62},

  Doi                      = {10.1016/j.atmosenv.2012.07.071},
  ISSN                     = {1352-2310},
  Keywords                 = {Air pollution
Particle
Deposition
Nasal cavity
Inhalation
CFD},
  Type                     = {Journal Article},
  Url                      = {http://www.sciencedirect.com/science/article/pii/S1352231012007571}
}

@Article{Inthavong2012b,
  Title                    = {Detailed predictions of particle aspiration affected by respiratory inhalation and airflow},
  Author                   = {Inthavong, Kiao and Ge, Qin Jiang and Li, Xiang Dong and Tu, Ji Yuan},
  Journal                  = {Atmospheric Environment},
  Year                     = {2012},
  Number                   = {0},
  Pages                    = {107-117},
  Volume                   = {62},

  Doi                      = {http://dx.doi.org/10.1016/j.atmosenv.2012.07.071},
  ISSN                     = {1352-2310},
  Keywords                 = {Air pollution
Particle
Deposition
Nasal cavity
Inhalation
CFD},
  Type                     = {Journal Article},
  Url                      = {http://www.sciencedirect.com/science/article/pii/S1352231012007571}
}

@Article{Inthavong2012c,
  Title                    = {Detailed predictions of particle aspiration affected by respiratory inhalation and airflow},
  Author                   = {Inthavong, Kiao and Ge, Qin Jiang and Li, Xiang Dong and Tu, Ji Yuan},
  Journal                  = {Atmospheric Environment},
  Year                     = {2012},
  Pages                    = {107-117},
  Volume                   = {62},

  Doi                      = {http://dx.doi.org/10.1016/j.atmosenv.2012.07.071},
  ISSN                     = {1352-2310},
  Keywords                 = {Air
cavity
CFD
Deposition
inhalation
Nasal
Particle
pollution},
  Type                     = {Journal Article},
  Url                      = {http://ac.els-cdn.com/S1352231012007571/1-s2.0-S1352231012007571-main.pdf?_tid=c3558e84-421f-11e4-8212-00000aab0f02&acdnat=1411366671_533884b6bae7a6f3851bde836461566b}
}

@Article{Inthavong2013b,
  Title                    = {Inhalation and deposition of carbon and glass composite fibre in the respiratory airway},
  Author                   = {Inthavong, Kiao and Mouritz, Adrian P. and Dong, Jingliang and Tu, Ji Yuan},
  Journal                  = {Journal of Aerosol Science},
  Year                     = {2013},
  Note                     = {C:\Users\sean\AppData\Roaming\Zotero\Zotero\Profiles\16a4oype.default\zotero\storage\UIJ99MST\Inthavong et al. - 2013 - Inhalation and deposition of carbon and glass comp.pdf
C:\Users\sean\AppData\Roaming\Zotero\Zotero\Profiles\16a4oype.default\zotero\storage\ICDCWRVP\Inthavong et al. - 2013 - Inhalation and deposition of carbon and glass comp.pdf},
  Pages                    = {58-68},
  Volume                   = {65},

  Abstract                 = {This paper presents a comparative study into the inhalation by the human respiratory system of airborne carbon and glass fibres released from burning composite materials used in aircraft. Using experimental data on the release of fibres from composites produced from fires, carbon and glass fibres of varying lengths were evaluated for their likely deposition in the human respiratory airway during inhalation. An anatomical model of the respiratory system coupled with a computational fluid dynamics (CFD) model was used to analyse the fibre trajectories from the outside air, through the nostrils, and into the respiratory system (including the larynx, trachea and lungs). Deposition of carbon fibres were more influenced by their length than glass fibres, which is attributed to their lower density and smaller diameter. Local deposition fractions showed that all fibres either deposited mainly in the nasal cavity or penetrated through to the lungs.},
  Doi                      = {10.1016/j.jaerosci.2013.07.003},
  ISSN                     = {0021-8502},
  Keywords                 = {Composite fibre
Deposition patterns
Fibre drag
Nasal deposition
Realistic nasal model},
  Type                     = {Journal Article},
  Url                      = {http://ac.els-cdn.com/S0021850213001559/1-s2.0-S0021850213001559-main.pdf?_tid=2e88b9ce-4220-11e4-b90d-00000aacb35d&acdnat=1411366851_6c8815b3fa974b6898b0eae7b134aaaf}
}

@Article{Inthavong2014b,
  Title                    = {Surface mapping for visualization of wall stresses during inhalation in a human nasal cavity},
  Author                   = {Inthavong, Kiao and Shang, Yidan and Tu, Jiyuan},
  Journal                  = {Respiratory Physiology \& Neurobiology},
  Year                     = {2014},
  Number                   = {0},
  Pages                    = {54-61},
  Volume                   = {190},

  Doi                      = {http://dx.doi.org/10.1016/j.resp.2013.09.004},
  ISSN                     = {1569-9048},
  Keywords                 = {Surface map
Wall shear stress
Nasal cavity
CFD
Inhalation},
  Type                     = {Journal Article},
  Url                      = {http://www.sciencedirect.com/science/article/pii/S1569904813003108}
}

@Article{Inthavong2014c,
  Title                    = {Surface mapping for visualization of wall stresses during inhalation in a human nasal cavity},
  Author                   = {Inthavong, Kiao and Shang, Yidan and Tu, Jiyuan},
  Journal                  = {Respiratory Physiology \& Neurobiology},
  Year                     = {2014},
  Note                     = {C:\Users\sean\AppData\Roaming\Zotero\Zotero\Profiles\16a4oype.default\zotero\storage\8N39PVS7\Inthavong et al. - 2014 - Surface mapping for visualization of wall stresses.pdf},
  Pages                    = {54-61},
  Volume                   = {190},

  Abstract                 = {Airflow analysis can assist in better understanding the physiology however the human nasal cavity is an extremely complicated geometry that is difficult to visualize in 3D space, let alone in 2D space. In this paper, an anatomically accurate 3D surface of the nasal passages derived from CT data was unwrapped and transformed into a 2D space, into a UV-domain (where u and v are the coordinates) to allow a complete view of the entire wrapped surface. This visualization technique allows surface flow parameters to be analyzed with greater precision. A UV-unwrapping tool is developed and a strategy is presented to allow deeper analysis to be performed. This includes (i) the ability to present instant comparisons of geometry and flow variables between any number of different nasal cavity models through normalization of the 2D unwrapped surface; (ii) visualization of an entire surface in one view and; (iii) a planar surface that allows direct 1D and 2D analytical solutions of diffusion of inhaled vapors and particles through the nasal walls. This work lays a foundation for future investigations that correlates adverse and therapeutic health responses to local inhalation of gases and particles.},
  Doi                      = {10.1016/j.resp.2013.09.004},
  ISSN                     = {1569-9048},
  Keywords                 = {CFD
inhalation
Nasal cavity
Surface map
Wall shear stress},
  Type                     = {Journal Article},
  Url                      = {http://ac.els-cdn.com/S1569904813003108/1-s2.0-S1569904813003108-main.pdf?_tid=293ea2bc-4220-11e4-8bc9-00000aab0f6b&acdnat=1411366842_2625421060112678c5da87139aaae0af}
}

@InProceedings{Inthavong,
  Title                    = {Local deposition sites of drug particles in a human nasal cavity},
  Author                   = {Inthavong, K. and Tian, Z.F. and Li, H.F. and Tu, J.Y. and Yang, W. and Xue, C.L. and Li, C.G.},
  Booktitle                = {Fifth International Conference on CFD in the Process Industries},
  Editor                   = {Scharwz, P. and Witt, P.},
  Publisher                = {CSIRO},

  Type                     = {Conference Proceedings}
}

@Article{Inthavong2008,
  Title                    = {Optimising nasal spray parameters for efficient drug delivery using computational fluid dynamics},
  Author                   = {Inthavong, K. and Tian, Z.F. and Tu, J.Y. and Yang, W. and Xue, C.},
  Journal                  = {Computers in Biology and Medicine},
  Year                     = {2008},
  Number                   = {6},
  Pages                    = {713-726},
  Volume                   = {38},

  Type                     = {Journal Article}
}

@Misc{Inthavong2007,
  Title                    = {CFD Simulations on the Heating Capability in a Human Nasal Cavity},

  Author                   = {Inthavong, K. and Tian, Z.F. and Tu, J. Y.},
  Year                     = {2007},

  ISBN                     = {978-1-864998-94-8 },
  Publisher                = {School of Engineering, The University of Queensland },
  Type                     = {Conference Paper}
}

@Article{Inthavong2006,
  Title                    = {A Numerical Study of Spray Particle Deposition in a Human Nasal Cavity},
  Author                   = {Inthavong, K. and Tian, Z. F. and Li, H. F. and Tu, J. Y. and Yang, W. and Xue, C. L. and Li, C. G.},
  Journal                  = {Aerosol Science and Technology},
  Year                     = {2006},
  Number                   = {11},
  Pages                    = {1034-1045},
  Volume                   = {40},

  Abstract                 = {Particle depositional studies from nasal sprays are important for efficient drug delivery. The main influences on deposition involve the nasal cavity geometry and the nasal spray device of which its parameters are controlled by the product design. It is known that larger particle sizes (? 10 ?m) at a flow rate of 333 ml/s impact in the anterior portion of the nose, leaving a significant portion of the nasal cavity unexposed to the drugs. Studies have found correlations for the spray cone angles and particle sizes with deposition efficiencies. This study extends these ideas to incorporate other parameters such as the insertion angle of the nasal spray and the injected particle velocity to observe its effect on deposition. A numerical method utilizing a particle tracking procedure found that the most important parameter was the particle's Stokes number which affected all other parameters on the deposition efficiency.},
  Doi                      = {10.1080/02786820600924978},
  ISSN                     = {0278-6826},
  Type                     = {Journal Article},
  Url                      = {http://dx.doi.org/10.1080/02786820600924978}
}

@Article{Inthavong2009,
  Title                    = {Effect of ventilation design on removal of particles in woodturning workstations},
  Author                   = {Inthavong, K. and Tian, Z. F. and Tu, J. Y.},
  Journal                  = {Building and Environment},
  Year                     = {2009},
  Number                   = {1},
  Pages                    = {125-136},
  Volume                   = {44},

  Abstract                 = {Wood processing tasks such as circular sawing and turning that is associated with woodturning operators produce particularly high exposure levels. A computational model including a humanoid and lathe within a test chamber was simulated with monodisperse particles under five different ventilation designs with the aim of reducing the particle suspension within the breathing zone. A commercial CFD code was used to solve the governing equations of motion with a k–ε RNG turbulence model. A discrete Lagrangian model was used to track the particles individually. Measurements to evaluate the efficiency of each ventilation design included total particle clearance and the percentage of particles crossing through the breathing plane. It was concluded that the percentage of particles that cross the breathing plane is of greater significance than the other measurements as it provides a better determination of exposure levels. Ventilation that emanated from the roof and had an angled outlet provided greatest total particle clearance and a low number of particles in the breathing plane. It was also found that the obstruction from the local roof ventilations caused separation of the air that flowed along the ceiling to produce a complex flow region. This study provides a basis for further investigation into the effects of particle size and density on the particle flow patterns and potential inhalation conditions for a given ventilation design.},
  Doi                      = {http://dx.doi.org/10.1016/j.buildenv.2008.02.002},
  ISSN                     = {0360-1323},
  Keywords                 = {Wood dust
Particle clearance
Simulation
CFD
Building
Woodturning},
  Type                     = {Journal Article},
  Url                      = {http://www.sciencedirect.com/science/article/pii/S0360132308000255}
}

@Article{Inthavong2009a,
  Title                    = {Effect of ventilation design on removal of particles in woodturning workstations},
  Author                   = {Inthavong, K. and Tian, Z. F. and Tu, J. Y.},
  Journal                  = {Building and Environment},
  Year                     = {2009},
  Number                   = {1},
  Pages                    = {125-136},
  Volume                   = {44},

  Keywords                 = {Wood dust
Particle clearance
Simulation
CFD
Building
Woodturning},
  Type                     = {Journal Article},
  Url                      = {http://www.sciencedirect.com/science/article/B6V23-4S1C87G-2/2/401e2bd0758f781be46f0a6c8c2bf279 }
}

@Article{Inthavong2009b,
  Title                    = {Effect of ventilation design on removal of particles in woodturning workstations},
  Author                   = {Inthavong, K. and Tian, Z. F. and Tu, J. Y.},
  Journal                  = {Building and Environment},
  Year                     = {2009},
  Pages                    = {125-136},
  Volume                   = {44},

  Abstract                 = {Wood processing tasks such as circular sawing and turning that is associated with woodturning operators produce particularly high exposure levels. A computational model including a humanoid and lathe within a test chamber was simulated with monodisperse particles under five different ventilation designs with the aim of reducing the particle suspension within the breathing zone. A commercial CFD code was used to solve the governing equations of motion with a k–ε RNG turbulence model. A discrete Lagrangian model was used to track the particles individually. Measurements to evaluate the efficiency of each ventilation design included total particle clearance and the percentage of particles crossing through the breathing plane. It was concluded that the percentage of particles that cross the breathing plane is of greater significance than the other measurements as it provides a better determination of exposure levels. Ventilation that emanated from the roof and had an angled outlet provided greatest total particle clearance and a low number of particles in the breathing plane. It was also found that the obstruction from the local roof ventilations caused separation of the air that flowed along the ceiling to produce a complex flow region. This study provides a basis for further investigation into the effects of particle size and density on the particle flow patterns and potential inhalation conditions for a given ventilation design.},
  Doi                      = {http://dx.doi.org/10.1016/j.buildenv.2008.02.002},
  ISSN                     = {0360-1323},
  Keywords                 = {Building
CFD
clearance
dust
Particle
Simulation
wood
Woodturning},
  Type                     = {Journal Article},
  Url                      = {http://ac.els-cdn.com/S0360132308000255/1-s2.0-S0360132308000255-main.pdf?_tid=cbce9f42-421f-11e4-82ca-00000aacb35d&acdnat=1411366685_c1d1d4725f3fb2ec81d06df88d85d5a3}
}

@Article{Inthavong2011a,
  Title                    = {Micron particle deposition in the nasal cavity using the v2–f model},
  Author                   = {Inthavong, Kiao and Tu, Jiyuan and Heschl, Christian},
  Journal                  = {Computers \& Fluids},
  Year                     = {2011},
  Number                   = {1},
  Pages                    = {184-188},
  Volume                   = {51},

  Doi                      = {10.1016/j.compfluid.2011.08.013},
  ISSN                     = {0045-7930},
  Keywords                 = {Computational fluid dynamics
Eddy interaction model
Particle deposition
Near wall turbulence
Nasal cavity},
  Type                     = {Journal Article},
  Url                      = {http://www.sciencedirect.com/science/article/pii/S0045793011002568}
}

@InBook{Inthavong2010a,
  Title                    = {Discrete Phase Modelling of Particles and its Deposition in the Nasal Cavity Using Computational Fluid Dynamics},
  Author                   = {Inthavong, Kiao and Tu, J. Y.},
  Editor                   = {Campbell, Jeremy M.},
  Publisher                = {Nova Science Publishers},
  Year                     = {2010},
  Type                     = {Book Section},
  Volume                   = {1},

  Booktitle                = {Advances in Mechanics Research},
  ISBN                     = {978-1-61728-105-1}
}

@Article{Inthavong2009c,
  Title                    = {Computational Modelling of Gas-Particle Flows with Different Particle Morphology in the Human Nasal Cavity},
  Author                   = {Inthavong, K. and Tu, J. Y. and Ahmadi, G.},
  Journal                  = {Journal of Computational Multiphase Flows},
  Year                     = {2009},
  Number                   = {1},
  Pages                    = {57-82},
  Volume                   = {1},

  Type                     = {Journal Article}
}

@Article{Inthavong2010b,
  Title                    = {Effects of airway obstruction induced by asthma attack on particle deposition},
  Author                   = {Inthavong, K and Tu, J. Y. and Ye, Y and Ding, S and Subic, A and Thien, F},
  Journal                  = {Journal of Aerosol Science},
  Year                     = {2010},
  Number                   = {6},
  Pages                    = {587-601},
  Volume                   = {41},

  ISSN                     = {0021-8502},
  Keywords                 = {Asthmatic airways
Airflow patterns
Particle deposition
Modeling},
  Type                     = {Journal Article},
  Url                      = {http://www.sciencedirect.com/science/article/B6V6B-4YMK1PY-1/2/2a8c1c009331e1909e4ace20d2215cab}
}

@Article{Inthavong2008a,
  Title                    = {Numerical study of fibre deposition in a human nasal cavity},
  Author                   = {Inthavong, Kiao and Wen, Jian and Tian, Zhaofeng and Tu, Jiyuan},
  Journal                  = {Journal of Aerosol Science},
  Year                     = {2008},
  Number                   = {3},
  Pages                    = {253-265},
  Volume                   = {39},

  Abstract                 = {The inhalation of toxic particles such as asbestos fibres through the nasal airway has been found to cause harmful damage to the respiratory system. This study made use of CFD techniques to investigate deposition of fibrous particles in a human nasal cavity. A 3D computational model was created from CT scans which provided the framework to study the flow and deposition of fibres at a constant flow rate of 7.5 L/min. The effects of the fibres’ elongated shape, density and size were incorporated into empirical drag correlations and the fibre trajectories were recorded through a Lagrangian tracking scheme. In general, good agreement was found in the right cavity and an overprediction in the left cavity. The major cause of deposition differences was in the geometrical variations between subjects as well as the left and right cavities. The dominant mechanism of deposition was by inertial impaction, with a majority of the particles depositing in the anterior region. It was found that asbestos had a very low deposition, ≈ 14 % , and was independent of fibre length. In comparison, the carbon fibre exhibited increases in deposition as the fibre length increased. A parameter � A cross which represents the mass per unit length was used to equate the d ae for different fibre lengths.},
  Doi                      = {http://dx.doi.org/10.1016/j.jaerosci.2007.11.007},
  ISSN                     = {0021-8502},
  Keywords                 = {Nasal airway
Fibre
Morphology
CFD
Deposition},
  Type                     = {Journal Article},
  Url                      = {http://www.sciencedirect.com/science/article/pii/S0021850207002078}
}

@Article{Inthavong2008b,
  Title                    = {Numerical study of fibre deposition in a human nasal cavity},
  Author                   = {Inthavong, Kiao and Wen, Jian and Tian, Zhaofeng and Tu, Jiyuan},
  Journal                  = {Journal of Aerosol Science},
  Year                     = {2008},
  Pages                    = {253-265},
  Volume                   = {39},

  Abstract                 = {The inhalation of toxic particles such as asbestos fibres through the nasal airway has been found to cause harmful damage to the respiratory system. This study made use of CFD techniques to investigate deposition of fibrous particles in a human nasal cavity. A 3D computational model was created from CT scans which provided the framework to study the flow and deposition of fibres at a constant flow rate of 7.5 L/min. The effects of the fibres’ elongated shape, density and size were incorporated into empirical drag correlations and the fibre trajectories were recorded through a Lagrangian tracking scheme. In general, good agreement was found in the right cavity and an overprediction in the left cavity. The major cause of deposition differences was in the geometrical variations between subjects as well as the left and right cavities. The dominant mechanism of deposition was by inertial impaction, with a majority of the particles depositing in the anterior region. It was found that asbestos had a very low deposition, ≈ 14 % , and was independent of fibre length. In comparison, the carbon fibre exhibited increases in deposition as the fibre length increased. A parameter �? A cross which represents the mass per unit length was used to equate the d ae for different fibre lengths.},
  Doi                      = {http://dx.doi.org/10.1016/j.jaerosci.2007.11.007},
  ISSN                     = {0021-8502},
  Keywords                 = {airway
CFD
Deposition
Fibre
Morphology
Nasal},
  Type                     = {Journal Article},
  Url                      = {http://ac.els-cdn.com/S0021850207002078/1-s2.0-S0021850207002078-main.pdf?_tid=c80522fa-421f-11e4-9fad-00000aab0f27&acdnat=1411366679_ed3fc82b7985611b4c9575a248a761a1}
}

@Article{Inthavong2008c,
  Title                    = {Numerical study of fibre deposition in a human nasal cavity},
  Author                   = {Inthavong, K. and Wen, J. and Tian, Z.F. and Tu, J. Y.},
  Journal                  = {Journal of Aerosol Science},
  Year                     = {2008},
  Number                   = {3},
  Pages                    = {253-265},
  Volume                   = {39},

  Abstract                 = {The inhalation of toxic particles such as asbestos fibres through the nasal airway has been found to cause harmful damage to the respiratory system. This study made use of CFD techniques to investigate deposition of fibrous particles in a human nasal cavity. A 3D computational model was created from CT scans which provided the framework to study the flow and deposition of fibres at a constant flow rate of 7.5 L/min. The effects of the fibres' elongated shape, density and size were incorporated into empirical drag correlations and the fibre trajectories were recorded through a Lagrangian tracking scheme. In general, good agreement was found in the right cavity and an overprediction in the left cavity. The major cause of deposition differences was in the geometrical variations between subjects as well as the left and right cavities. The dominant mechanism of deposition was by inertial impaction, with a majority of the particles depositing in the anterior region. It was found that asbestos had a very low deposition, [approximate]14%, and was independent of fibre length. In comparison, the carbon fibre exhibited increases in deposition as the fibre length increased. A parameter [rho]Across which represents the mass per unit length was used to equate the dae for different fibre lengths.},
  Keywords                 = {Nasal airway
Fibre
Morphology
CFD
Deposition},
  Type                     = {Journal Article},
  Url                      = {http://www.sciencedirect.com/science/article/B6V6B-4R68NKG-1/2/1ecfe25221d1688bc4304b0de3f25158 }
}

@Article{Inthavong2010c,
  Title                    = {Modelling the inhalation of drug particles in a human nasal cavity},
  Author                   = {Inthavong, K. and Wen, J. and Tu, J.Y.},
  Journal                  = {Journal of Biomedical Science and Engineering},
  Year                     = {2010},
  Number                   = {1},
  Pages                    = {52-58},
  Volume                   = {3},

  Type                     = {Journal Article}
}

@InProceedings{Inthavonga,
  Title                    = {Inhalation of Toxic and Therapeutic Particles in a Human Nasal Cavity},
  Author                   = {Inthavong, K. and Wen, J. and Tu, J. Y.},
  Booktitle                = {The 2nd International Conference on Bioinformatics and Biomedical Engineering, 2008. ICBBE 2008.},
  Pages                    = {1707-1711},
  Publisher                = {IEEE Transactions},

  Keywords                 = {asbestos
carbon fibres
computational fluid dynamics
computerised tomography
hazardous materials
health hazards
image segmentation
laminar flow
lung
medical image processing
pneumodynamics
toxicology
CATIA program
GAMBIT program
MegaWave2
air flow field
asbestos fibres
carbon fibres
computational fluid dynamics
computed tomography
flow patterns
human nasal cavity
inhalation
laminar flow
particle dynamics
therapeutic particles
toxic particles},
  Type                     = {Conference Proceedings}
}

@Article{Inthavong2009d,
  Title                    = {From CT Scans to CFD Modelling – Fluid and Heat Transfer in a Realistic Human Nasal Cavity},
  Author                   = {Inthavong, K. and Wen, J. and Tu, J. Y. and Tian, Z. F.},
  Journal                  = {Engineering Applications of Computational Fluid Mechanics},
  Year                     = {2009},
  Number                   = {3},
  Pages                    = {321-335},
  Volume                   = {3},

  Type                     = {Journal Article}
}

@Article{Inthavong2012d,
  Title                    = {External and Near-Nozzle Spray Characteristics of a Continuous Spray Atomized from a Nasal Spray Device},
  Author                   = {Inthavong, K. and Yang, W. and Fung, M. C. and Tu, J. Y.},
  Journal                  = {Aerosol Science and Technology},
  Year                     = {2012},
  Number                   = {2},
  Pages                    = {165-177},
  Volume                   = {46},

  Doi                      = {10.1080/02786826.2011.617793},
  ISSN                     = {0278-6826},
  Type                     = {Journal Article},
  Url                      = {http://dx.doi.org/10.1080/02786826.2011.617793}
}

@InProceedings{Inthavongb,
  Title                    = {Comparison of Micron and Nano Particle Deposition Patterns in a Realistic Human Nasal Cavity},
  Author                   = {Inthavong, K. and Yong, Y. and Ding, S. and Tu, J.Y. and Subic, A. and Thien, F.},
  Booktitle                = {13th International Conference on Biomedical Engineering (ICBME2008)},
  Editor                   = {Lim, Chwee Teck. and Goh, Cho Hong James.},
  Publisher                = {Springer},
  Volume                   = {IFMBE Proceedings, Vol. 23},

  Type                     = {Conference Proceedings}
}

@Misc{Inthavong2008d,
  Title                    = {Airway geometry and inhalation effort comparisons in a narrowed and recovered airway caused by acute asthma. },

  Author                   = {Inthavong, K. and Yong, Y. and Ding, S. and Tu, J.Y. and Subic, A. and Thien, F.},
  Year                     = {2008},

  Publisher                = {IEEE},
  Type                     = {Conference Paper}
}

@Misc{Inthavong2008e,
  Title                    = {Comparative study of the effects of acute asthma in relation to a recovered airway tree on airflow patterns},

  Author                   = {Inthavong, K. and Yong, Y. and Ding, S. and Tu, J.Y. and Subic, A. and Thien, F.},
  Month                    = {3-6 December 2008},
  Year                     = {2008},

  Type                     = {Conference Paper}
}

@Misc{Inthavong2008f,
  Title                    = {Comparison of Airway Geometries – Effect on Airflow and Particle Deposition. },

  Author                   = {Inthavong, K. and Yong, Y. and Ding, S. and Tu, J.Y. and Subic, A. and Thien, F.},
  Month                    = {3-6 December 2008},
  Year                     = {2008},

  Type                     = {Conference Paper}
}

@Article{Inthavong2011b,
  Title                    = {Numerical modelling of nanoparticle deposition in the nasal cavity and the tracheobronchial airway},
  Author                   = {Inthavong, Kiao and Zhang, Kai and Tu, Jiyuan},
  Journal                  = {Computer Methods in Biomechanics and Biomedical Engineering},
  Year                     = {2011},
  Number                   = {7},
  Pages                    = {633-643},
  Volume                   = {14},

  Doi                      = {10.1080/10255842.2010.493510},
  ISSN                     = {1025-5842},
  Type                     = {Journal Article},
  Url                      = {http://dx.doi.org/10.1080/10255842.2010.493510}
}

@Misc{Inthavong2009e,
  Title                    = {Modelling submicron and micron particle deposition in a human nasal cavity},

  Author                   = {Inthavong, K. and Zhang, K. and Tu, J.Y.},
  Month                    = {9-11 December 2008},
  Year                     = {2009},

  Publisher                = {CSIRO Minerals, Australia},
  Type                     = {Conference Paper}
}

@Article{Irvine2011,
  Title                    = {Drug delivery: One nanoparticle, one kill},
  Author                   = {Irvine, Darrell J.},
  Journal                  = {Nat Mater},
  Year                     = {2011},
  Note                     = {10.1038/nmat3014},
  Number                   = {5},
  Pages                    = {342-343},
  Volume                   = {10},

  ISSN                     = {1476-1122},
  Type                     = {Journal Article},
  Url                      = {http://dx.doi.org/10.1038/nmat3014}
}

@Article{Isabela2004,
  Title                    = {Asthma and Associated Conditions: High-Resolution CT and Pathologic Findings},
  Author                   = {Isabela, C. and Silva, S. and Colby, T.V. and Müller , N.L.},
  Journal                  = {AJR},
  Year                     = {2004},
  Volume                   = {183},

  Type                     = {Journal Article}
}

@Article{Isabey1981,
  Title                    = {Steady and unsteady pressure-flow relationships in central airways.},
  Author                   = {Isabey, D. and Chang, H.K.},
  Journal                  = {Journal Appl. Physiol.},
  Year                     = {1981},
  Pages                    = {1338-1348},
  Volume                   = {51},

  Type                     = {Journal Article}
}

@Article{Isabey,
  Title                    = {Dependence of central airway resistance on frequency and tidal volume: a model study},
  Author                   = {Isabey, D. and Chang, H.K. and Delpuech, C. and Harf, A. and Hatzfeld, C.},

  Type                     = {Journal Article}
}

@Article{Isabey1982,
  Title                    = {A model study of flow dynamics in human central airways. Part II: Secondary flow velocities},
  Author                   = {Isabey, D. and Chang, H. K.},
  Journal                  = {Respiration Physiology},
  Year                     = {1982},
  Note                     = {doi: DOI: 10.1016/0034-5687(82)90105-0},
  Number                   = {1},
  Pages                    = {97-113},
  Volume                   = {49},

  ISSN                     = {0034-5687},
  Keywords                 = {Air flow patterns
Secondary flow
Airway model
Tracheo-bronchial tree
Hot-wire anemometry},
  Type                     = {Journal Article},
  Url                      = {http://www.sciencedirect.com/science/article/B6T3J-47MKKNH-1V/2/e8d2d0e846ef75cbe7ac0b1794ed3149}
}

@Article{Ishikawa2009,
  Title                    = {Flow Mechanisms in the Human Olfactory Groove: Numerical Simulation of Nasal Physiological Respiration During Inspiration, Expiration, and Sniffing},
  Author                   = {Ishikawa, Shigeru and Nakayama, Toshio and Watanabe, Masahiro and Matsuzawa, Teruo},
  Journal                  = {Arch Otolaryngol Head Neck Surg},
  Year                     = {2009},
  Number                   = {2},
  Pages                    = {156-162},
  Volume                   = {135},

  Abstract                 = {Objectives To visualize the velocity and the streamline of physiological unsteady nasal flow and sniffing using the computational fluid dynamics method and to compare the inspiratory phase, expiratory phase, and sniffing flow patterns of the olfactory area. Design An anatomically correct 3-dimensional nasal and pharyngeal cavity was constructed from computed tomographic images of a healthy adult nose and pharynx. The unsteady state Navier-Stokes and continuity equations were solved numerically on inspiratory and expiratory nasal flow and sniffing. Setting Numerical simulation application. Main Outcome Measures The detailed velocity distribution and streamline distribution of nasal airflow were visualized using the computational fluid dynamics method (an imaging technology for regional flow factors [velocity and streamline]). Results The inspiratory flow passes through a wider olfactory area than the expiratory flow, and the sniffing flow passes through the widest olfactory area without increasing the velocity of the airflow. In addition, a recirculating flow strongly promotes olfactory function. Conclusion The computational fluid dynamics model allows for the investigation of the flow mechanisms in the human olfactory groove.},
  Doi                      = {10.1001/archoto.2008.530},
  Type                     = {Journal Article},
  Url                      = {http://archotol.ama-assn.org/cgi/content/abstract/135/2/156}
}

@Article{Ishikawa2006,
  Title                    = {Visualization of Flow Resistance in Physiological Nasal Respiration: Analysis of Velocity and Vorticities Using Numerical Simulation},
  Author                   = {Ishikawa, Shigeru and Nakayama, Toshio and Watanabe, Masahiro and Matsuzawa, Teruo},
  Journal                  = {Arch Otolaryngol Head Neck Surg},
  Year                     = {2006},
  Number                   = {11},
  Pages                    = {1203-1209},
  Volume                   = {132},

  Abstract                 = {Objectives To visualize the velocity gradients and the vorticities of physiological unsteady nasal flow using the computational fluid dynamics method and to compare the inspiratory phase and expiratory phase flow patterns. Design An anatomically correct 3-dimensional nasal and pharyngeal cavity was constructed from computed tomographic images of a healthy adult nose and pharynx. The unsteady state Navier-Stokes and continuity equations were solved numerically on inspiratory and expiratory nasal flow. Setting Numerical simulation application. Participants Coronary and axial computed tomographic images from a healthy adult were used. Main Outcome Measures The detailed velocity distribution and vorticity (resistance) distribution of nasal airflow were visualized using the computational fluid dynamics method (an imaging technology for regional flow factors [velocity, vector, streamline, and vortex]). Results In the inspiratory phase, a high-velocity area was prominent in the middle meatus, and the highest vorticity area had good agreement with this region. In the expiratory phase, the distributions of velocity and vorticities were flatter than those in the inspiratory phase. Conclusion The computational fluid dynamics model allows the investigation of airflow elements under physiological conditions, as well as the examination of the effect of nasal structure.},
  Doi                      = {10.1001/archotol.132.11.1203},
  Type                     = {Journal Article},
  Url                      = {http://archotol.ama-assn.org/cgi/content/abstract/132/11/1203}
}

@Article{Itoh1985,
  Title                    = {Mechanisms of aerosol deposition in a nasal model},
  Author                   = {Itoh, H. and Smaldone, G.C. and Swift, D. L. and Wagner, H.N.J.},
  Journal                  = {Journal of Aerosol Science},
  Year                     = {1985},
  Pages                    = {529–534},
  Volume                   = {16},

  Type                     = {Journal Article}
}

@Article{Jaakkola2007,
  Title                    = {Office work exposures and respiratory and sick building syndrome symptoms},
  Author                   = {Jaakkola, Maritta S and Yang, Liyan and Ieromnimon, Antonia and Jaakkola, Jouni J K},
  Journal                  = {Occupational and Environmental Medicine},
  Year                     = {2007},
  Number                   = {3},
  Pages                    = {178-184},
  Volume                   = {64},

  Abstract                 = {To assess the relation between exposure to carbonless copy paper (CCP), paper dust, and fumes from photocopiers and printers (FPP), and the occurrence of sick building syndrome (SBS)-related symptoms, chronic respiratory symptoms and respiratory infections. A population-based cross-sectional study with a random sample of 1016 adults, 21–63 years old, living in Pirkanmaa District in South Finland was conducted. This study focused on 342 office workers classified as professionals, clerks or administrative personnel according to their current occupation by the International Standard Classification of Occupations-88. They answered a questionnaire about personal information, health, smoking, occupation, and exposures in the work environment and at home. In logistic regression analyses adjusting for age, sex and a set of other confounders, all three exposures were related to a significantly increased risk of general symptoms (headache and fatigue). Exposure to paper dust and to FPP was associated with upper respiratory and skin symptoms, breathlessness, tonsillitis and middle ear infections. Exposure to CCP increased the risk of eye symptoms, chronic bronchitis and breathlessness. It was also associated with increased occurrence of sinus and middle ear infections and diarrhoea. A dose–response relations was observed between the number of exposures and occurrence of headache. The risk of tonsillitis and sinus infections also increased with increasing number of exposures. All chronic respiratory symptoms, apart from cough, were increased in the highest exposure category (including all three exposures). This study provides new evidence that exposure to paper dust and to FPP is related to the risk of SBS symptoms, breathlessness and upper respiratory infections. It strengthens the evidence that exposure to CCP increases the risk of eye symptoms, general symptoms, chronic respiratory symptoms and some respiratory infections. Reduction of these exposures could improve the health of office workers.},
  Doi                      = {10.1136/oem.2005.024596},
  Type                     = {Journal Article},
  Url                      = {http://oem.bmj.com/content/64/3/178.abstract}
}

@Article{Jaber2001,
  Title                    = {Helium-oxygen in the postextubation period decreases inspiratory effort},
  Author                   = {Jaber, S., Carlucci, A., Boussarsar, M., Fodil, R., Pigeot, J., Maggiore, S., Harf, A., Isabey, D., Brochard, L.},
  Journal                  = {American Journal of Respiratory and Critical Care Medicine},
  Year                     = {2001},
  Pages                    = {633-637},
  Volume                   = {164},

  Type                     = {Journal Article}
}

@Article{Jachowicz2009,
  Title                    = {A MEMS-based super fast dew point hygrometer—construction and medical applications},
  Author                   = {Ryszard S Jachowicz and Jerzy Weremczuk and Daniel Paczesny and Grzegorz Tarapata},
  Journal                  = {Measurement Science and Technology},
  Year                     = {2009},
  Number                   = {12},
  Pages                    = {124008},
  Volume                   = {20},

  Abstract                 = {The paper shows how MEMS (micro-electro-mechanical system) technology and a modified principle of fast temperature control (by heat injection instead of careful control of cooling) can considerably improve the dynamic parameters of dew point hygrometers. Some aspects of MEMS-type integrated sensor construction and technology, whole measurement system design, the control algorithm to run the system as well as empirical dynamic parameters from the tests are discussed too. The hygrometer can easily obtain five to six measurements per second with an uncertainty of less than 0.3 K. The meter range is between −10 °C and 40 °C dew point. In the second part of the paper (section 2), two different successful applications in medicine based on fast humidity measurements have been discussed. Some specific constructions of these super fast dew point hygrometers based on a MEMS sensor as well as limited empirical results from clinical tests have been reported too.},
  Url                      = {http://stacks.iop.org/0957-0233/20/i=12/a=124008}
}

@Article{Jackson1995,
  Title                    = {Optimizing inhaled drug delivery in patients with asthma},
  Author                   = {Jackson, C. and Lipworth, B.},
  Journal                  = {British Journal of General Practice},
  Year                     = {1995},
  Pages                    = {683-687},
  Volume                   = {45},

  Type                     = {Journal Article}
}

@Article{Jackson1943,
  Title                    = {Correlated applied anatomy of the bronchial tree and lungs with a system of nomenclature. },
  Author                   = {Jackson, C. L. and Huber, J. F.},
  Journal                  = {Chest},
  Year                     = {1943},
  Pages                    = {319-326},
  Volume                   = {9},

  Type                     = {Journal Article}
}

@Article{Jaffrin1974,
  Title                    = {Airway resistance: a fluid mechanical approach},
  Author                   = {Jaffrin, M.Y. and Kesic, P.},
  Journal                  = {Journal of Applied Physiology},
  Year                     = {1974},
  Number                   = {3},
  Pages                    = {354-361},
  Volume                   = {36},

  Type                     = {Journal Article}
}

@Article{Jan1973,
  Title                    = {Role of surface electric charge in red blood cell interactions},
  Author                   = {Jan, K. M. and Chien, S.},
  Journal                  = {Journal of General Physiology},
  Year                     = {1973},
  Note                     = {Cited By (since 1996): 33
Export Date: 4 June 2011
Source: Scopus},
  Number                   = {5},
  Pages                    = {638-654},
  Volume                   = {61},

  Type                     = {Journal Article},
  Url                      = {http://www.scopus.com/inward/record.url?eid=2-s2.0-0015620609&partnerID=40&md5=a72d443018791993197cd8c579e7e78f}
}

@Article{Jan1973a,
  Title                    = {Influence of Ionic Composition of Fluid Medium on Red-Cell Aggregration},
  Author                   = {Jan, K-M. and Chien, S. },
  Journal                  = {Journal of General Physiology},
  Year                     = {1973},
  Pages                    = {655-668},
  Volume                   = {61},

  Type                     = {Journal Article}
}

@Article{Janela2010,
  Title                    = {Absorbing boundary conditions for a 3D non-Newtonian fluid-structure interaction model for blood flow in arteries},
  Author                   = {Janela, João and Moura, Alexandra and Sequeira, Adélia},
  Journal                  = {International Journal of Engineering Science},
  Year                     = {2010},
  Number                   = {11},
  Pages                    = {1332-1349},
  Volume                   = {48},

  Abstract                 = {Blood flow in arteries is characterized by pulse pressure waves due to the interaction with the vessel walls. A 3D fluid-structure interaction (FSI) model in a compliant vessel is used to represent the pressure wave propagation. The 3D fluid is described through a shear-thinning generalized Newtonian model and the structure by a nonlinear hyperelastic model. In order to cope with the spurious reflections due to the truncation of the computational domain, several absorbing boundary conditions are analyzed. First, a 1D hyperbolic model that effectively captures the wave propagation nature of blood flow in arteries is coupled with the 3D FSI model. Extending previous results, an energy estimate is derived for the 3D FSI-1D coupling in the case of generalized Newtonian models. Secondly, absorbing boundary conditions obtained from the 1D model are imposed directly on the outflow sections of the 3D FSI model, and numerical results comparing the different absorbing conditions in an idealized vessel are presented. Results in a human carotid bifurcation reconstructed from medical images are also provided in order to show that the proposed methodology can be applied to anatomically realistic geometries.},
  Doi                      = {10.1016/j.ijengsci.2010.08.004},
  ISSN                     = {0020-7225},
  Keywords                 = {Fluid-structure interaction (FSI)
Absorbing boundary conditions
Non-Newtonian fluids
1D model
Blood flow},
  Type                     = {Journal Article},
  Url                      = {http://www.sciencedirect.com/science/article/pii/S0020722510001722}
}

@TechReport{Jankewicz2008,
  Title                    = {Benchmarking of exposures to wood dust and formaldehyde in selected industries in Australia},
  Author                   = {Jankewicz, G. and Lee, S.G. and Pisaniello, D. and Tkaczuk, M.},
  Institution              = {Australian Safety and Compensation Council},
  Year                     = {2008},
  Type                     = {Report}
}

@Article{Jarrin2009,
  Title                    = {Reconstruction of turbulent fluctuations for hybrid RANS/LES simulations using a Synthetic-Eddy Method},
  Author                   = {Jarrin, N. and Prosser, R. and Uribe, J. C. and Benhamadouche, S. and Laurence, D.},
  Journal                  = {International Journal of Heat and Fluid Flow},
  Year                     = {2009},
  Note                     = {doi: DOI: 10.1016/j.ijheatfluidflow.2009.02.016},
  Number                   = {3},
  Pages                    = {435-442},
  Volume                   = {30},

  ISSN                     = {0142-727X},
  Keywords                 = {Hybrid RANS-LES methods
Embedded LES
Synthetic turbulence
Channel flow
Square duct flow
Trailing edge flow},
  Type                     = {Journal Article},
  Url                      = {http://www.sciencedirect.com/science/article/B6V3G-4W0R3FP-1/2/2f63d4c41fcf48eeac3eb7a7f7228af5}
}

@Article{Jasuja1979,
  Title                    = {Atomization of Crude and Residual Fuel Oils},
  Author                   = {Jasuja, A.K.},
  Journal                  = {Trans. ASME J. Eng. Power},
  Year                     = {1979},
  Number                   = {2},
  Pages                    = {250-258},
  Volume                   = {101},

  Type                     = {Journal Article}
}

@Article{Jayaraju2008,
  Title                    = {Large eddy and detached eddy simulations of fluid flow and particle deposition in a human mouth-throat},
  Author                   = {Jayaraju, S. T. and Brouns, M. and Lacor, C. and Belkassem, B. and Verbanck, S.},
  Journal                  = {Journal of Aerosol Science},
  Year                     = {2008},
  Number                   = {10},
  Pages                    = {862-875},
  Volume                   = {39},

  Abstract                 = {Fluid flow was simulated in a human mouth-throat model under normal breathing conditions (30 l/min) alternatively employing RANS k-[omega] (without near-wall corrections), DES and LES methods. To test the validity of the fluid phase simulations, PIV measurements were carried out in a central sagittal plane of an identical model cast. Velocity and kinetic-energy profiles showed good quantitative agreement of experiments with LES/DES, and less so with RANS k-[omega]. Mouth-throat deposition was simulated for particle diameters 2, 4, 6, 8 and . By comparison with existing experimental data, LES/DES showed considerable improvement over the RANS k-[omega] model in predicting deposition for particle sizes below . For the bigger particles, RANS k-[omega] and LES/DES methods produced similarly good predictions. It is concluded that for the simulation of medication aerosols inhaled at a steady flow rate of 30 l/min, LES and DES provide more accurate results than the RANS k-[omega] model tested.},
  ISSN                     = {0021-8502},
  Keywords                 = {Mouth-throat model
Particle image velocimetry (PIV)
Large eddy simulation (LES)
Detached eddy simulation (DES)
Particle deposition},
  Type                     = {Journal Article},
  Url                      = {http://www.sciencedirect.com/science/article/B6V6B-4SPC0TC-1/2/e3020754da6a94597d59a9216b28a1e3}
}

@Article{Jayaraju2007,
  Title                    = {Fluid flow and particle deposition analysis in a realistic extrathoracic airway model using unstructured grids},
  Author                   = {Jayaraju, S. T. and Brouns, M. and Verbanck, S. and Lacor, C.},
  Journal                  = {Journal of Aerosol Science},
  Year                     = {2007},
  Note                     = {doi: DOI: 10.1016/j.jaerosci.2007.03.003},
  Number                   = {5},
  Pages                    = {494-508},
  Volume                   = {38},

  ISSN                     = {0021-8502},
  Keywords                 = {Realistic extrathoracic airway model
Unstructured grids
Transitional airflow
Lagrangian stochastic trajectory method},
  Type                     = {Journal Article},
  Url                      = {http://www.sciencedirect.com/science/article/B6V6B-4N8BN2N-1/2/1778650e18cab1370947515a09938f9b}
}

@Article{Jeffery1922,
  Title                    = {The Motion of Ellipsoidal Particles Immersed in a Viscous Fluid},
  Author                   = {Jeffery, G. B. },
  Journal                  = {Proceeding of Royal Society of London, Series A},
  Year                     = {1922},
  Pages                    = {161-179},
  Volume                   = {102},

  Type                     = {Journal Article}
}

@Article{Jeffery1998,
  Title                    = {Investigation and assessment of airway and lung inflammation: We now have the tools, what are the questions? },
  Author                   = {Jeffery, P.K.},
  Journal                  = {Eur Respir J},
  Year                     = {1998},
  Pages                    = {524-528},
  Volume                   = {11},

  Type                     = {Journal Article}
}

@Article{Jeon2012,
  Title                    = {Evaluation of newly developed nose-only inhalation exposure chamber for nanoparticles},
  Author                   = {Jeon, KiSoo and Yu, Il Je and Ahn, Kang-Ho},
  Journal                  = {Inhalation Toxicology},
  Year                     = {2012},
  Number                   = {9},
  Pages                    = {550-556},
  Volume                   = {24},

  Doi                      = {doi:10.3109/08958378.2012.696742},
  Type                     = {Journal Article},
  Url                      = {http://informahealthcare.com/doi/abs/10.3109/08958378.2012.696742}
}

@Article{Jeong2007,
  Title                    = {Numerical investigation on the flow characteristics and aerodynamic force of the upper airway of patient with obstructive sleep apnea using computational fluid dynamics},
  Author                   = {Jeong, S.J. and Kim, W.S. and Sung, S.J.},
  Journal                  = {Medical Engineering \& Physics},
  Year                     = {2007},
  Pages                    = {637-651},
  Volume                   = {29},

  Type                     = {Journal Article}
}

@Article{Jeong2007a,
  Title                    = {Numerical investigation on the flow characteristics and aerodynamic force of the upper airway of patient with obstructive sleep apnea using computational fluid dynamics},
  Author                   = {Jeong, S-J. and Kim, W-S. and Sung, S-J.},
  Journal                  = {Medical Engineering \& Physics},
  Year                     = {2007},
  Pages                    = {637-651},
  Volume                   = {29},

  Type                     = {Journal Article}
}

@Article{Jiang2010,
  Title                    = {Airflow and nanoparticle deposition in rat nose under various breathing and sniffing conditions--A computational evaluation of the unsteady and turbulent effect},
  Author                   = {Jiang, Jianbo and Zhao, Kai},
  Journal                  = {Journal of Aerosol Science},
  Year                     = {2010},
  Number                   = {11},
  Pages                    = {1030-1043},
  Volume                   = {41},

  Abstract                 = {Accurate prediction of nanoparticle (1-100 nm) deposition in the rat nasal cavity is important for assessing the toxicological impact of inhaled nanoparticles as well as for potential therapeutic applications. A quasi-steady assumption has been widely adopted in the past investigations on this topic, yet the validity of such simplification under various breathing and sniffing conditions has not been carefully examined. In this study, both steady and unsteady computational fluid dynamics (CFD) simulations were conducted in a published rat nasal model under various physiologically realistic breathing and sniffing flow rates. The transient airflow structures, nanoparticle transport and deposition patterns in the whole nasal cavity and the olfactory region were investigated and compared with steady state simulation of equivalent flow rate. The results showed that (1) the quasi-steady flow assumption for cyclic flow was valid for over 70% of the cycle period during all simulated breathing and sniffing conditions in the rat nasal cavity, or the unsteady effect was only significant during the transition between the respiratory phases; (2) yet the quasi-steady assumption for nanoparticle transport was not valid, except in the vicinity of peak respiration. In general, the total deposition efficiency of nanoparticle during cyclic breathing would be lower than that of steady state due to the unsteady effect on particle transport and deposition, and further decreased with the increase of particle size, sniffing frequency, and flow rate. In the contrary, previous study indicated that for micro-scale particles (0.5-4 [mu]m), the unsteady effect would increase deposition efficiencies in rat nasal cavity. Combined, these results suggest that the quasi-steady assumption of nasal particle transport during cycling breathing should be used with caution for an accurate assessment of the toxicological and therapeutic impact of particle inhalation. Empirical equations and effective steady state approximation derived in this study are thus valuable to estimate such unsteady effects in future applications. Finally, through large eddy simulations, turbulent effects were found to be negligible for all breathing/sniff conditions in rat nasal cavity.},
  ISSN                     = {0021-8502},
  Keywords                 = {Computational fluid dynamics
Rat nasal cavity
Nanoparticle deposition
Olfactory region
Cycling breathing},
  Type                     = {Journal Article},
  Url                      = {http://www.sciencedirect.com/science/article/B6V6B-50GJ2RM-1/2/cd211a82a35b4f96726ae6e5918964af}
}

@Article{Jin2007,
  Title                    = {Large eddy simulation of inhaled particle deposition within the human upper respiratory tract},
  Author                   = {Jin, H. H. and Fan, J. R. and Zeng, M. J. and Cen, K. F.},
  Journal                  = {Journal of Aerosol Science},
  Year                     = {2007},
  Number                   = {3},
  Pages                    = {257-268},
  Volume                   = {38},

  Abstract                 = {A three-dimensional (3-D) computer geometrical model of the human upper respiratory tract (URT), including the airway from mouth, pharynx, larynx and trachea to triple bifurcation was created. A large eddy simulation (LES) based on this model investigated the deposition of inhaled particles within the human URT. Steady respiration mode was first considered with three kinds of breathing intensity Q=30, 60 and 90 L/min and then compared with an unsteady respiration mode at the breathing intensity of . The deposition efficiencies (DEs) of different particles with density of [rho]=600, and particle diameter of d=1, 5, was studied, respectively. The results showed that the increases of particle diameter, particle density and breathing intensity improves the particle DEs in the human URT; and the particle DEs in unsteady respiration mode are higher than those in steady mode. The results agree well with corresponding experimental data.},
  ISSN                     = {0021-8502},
  Keywords                 = {Large eddy simulation (LES)
Upper respiratory tract (URT)
Particle deposition},
  Type                     = {Journal Article},
  Url                      = {http://www.sciencedirect.com/science/article/B6V6B-4MY1158-1/2/f84fe8c03e917e19879ca9df0771bb66}
}

@Article{Jobe1980,
  Title                    = {The in vivo labeling with acetate and palmitate of lung phospholipids from developing and adult rabbits},
  Author                   = {Jobe, Alan and Ikegami, Machiko and Sarton-Miller, Isabelle},
  Journal                  = {Biochimica et Biophysica Acta (BBA) - Lipids and Lipid Metabolism},
  Year                     = {1980},
  Number                   = {1},
  Pages                    = {65-75},
  Volume                   = {617},

  ISSN                     = {0005-2760},
  Keywords                 = {Pulmonary surfactant
Phosphatidylcholine
Phosphatidylglycerol
Palmitic acid
Acetate labeling
(Rabbit)},
  Type                     = {Journal Article},
  Url                      = {http://www.sciencedirect.com/science/article/B6T1X-47F7GG7-XR/2/109b357da74bdc4b0059d9a92af1ae6e}
}

@Article{Johnson1979,
  Title                    = {Deposition of Particles in Model Airways},
  Author                   = {Johnson, J.R. and Schroter, R.C. },
  Journal                  = {Journal of Applied Physiology},
  Year                     = {1979},
  Pages                    = {947-953},
  Volume                   = {47},

  Type                     = {Journal Article}
}

@Article{Johnson1971,
  Title                    = {Surface Energy and the Contact of Elastic Solids},
  Author                   = {Johnson, K.L. and Kendall, K. and Roberts, A.D. },
  Journal                  = {Proceedings of the Royal Society of London, Series A},
  Year                     = {1971},
  Pages                    = {301-313},
  Volume                   = {324},

  Type                     = {Journal Article}
}

@Article{Johnstone2004,
  Title                    = {The flow inside an idealised form of the human extra-thoracic airway},
  Author                   = {Johnstone, A. and Uddin, M. and Pollard, A. and Heenan, A. and Finlay, W. H.},
  Journal                  = {Experiments in Fluids},
  Year                     = {2004},
  Number                   = {5},
  Pages                    = {673-689},
  Volume                   = {37},

  Doi                      = {10.1007/s00348-004-0857-4},
  ISSN                     = {0723-4864},
  Type                     = {Journal Article},
  Url                      = {http://dx.doi.org/10.1007/s00348-004-0857-4}
}

@Article{JonathanRichard2002,
  Title                    = {Delaunay refinement algorithms for triangular mesh generation},
  Author                   = {Jonathan Richard, Shewchuk},
  Journal                  = {Computational Geometry},
  Year                     = {2002},
  Number                   = {1–3},
  Pages                    = {21-74},
  Volume                   = {22},

  Doi                      = {10.1016/s0925-7721(01)00047-5},
  ISSN                     = {0925-7721},
  Keywords                 = {Triangular mesh generation
Delaunay triangulation
Constrained Delaunay triangulation
Delaunay refinement
Computational geometry},
  Type                     = {Journal Article},
  Url                      = {http://www.sciencedirect.com/science/article/pii/S0925772101000475}
}

@Article{Jorissen1998,
  Title                    = {Nasal ciliary beat frequency is age independent},
  Author                   = {Jorissen, Mark and Willems, Tom and Van Der Schueren, Bernadette},
  Journal                  = {The Laryngoscope},
  Year                     = {1998},
  Number                   = {7},
  Pages                    = {1042--1047},
  Volume                   = {108},

  Doi                      = {10.1097/00005537-199807000-00017},
  ISSN                     = {1531-4995},
  Keywords                 = {ciliary beat frequency, age, ultrastructure, biopsy, ciliogenesis},
  Publisher                = {John Wiley \& Sons, Inc.},
  Url                      = {http://dx.doi.org/10.1097/00005537-199807000-00017}
}

@Article{Jouvray2007,
  Title                    = {On nonlinear RANS models when predicting more complex geometry room air flows},
  Author                   = {Jouvray, A. and Tucker, P. G. and Liu, Y.},
  Journal                  = {International Journal of Heat and Fluid Flow},
  Year                     = {2007},
  Number                   = {2},
  Pages                    = {275-288},
  Volume                   = {28},

  Doi                      = {10.1016/j.ijheatfluidflow.2006.02.029},
  ISSN                     = {0142-727X},
  Keywords                 = {Room ventilation
RANS
LES
Hybrid LES–RANS},
  Type                     = {Journal Article},
  Url                      = {http://www.sciencedirect.com/science/article/pii/S0142727X06000543}
}

@Book{Junqueira2005,
  Title                    = {Basic histology : text \& atlas},
  Author                   = {Junqueira, Luiz Carlos UchoÌ‚a and Carneiro, JoseÌ},
  Publisher                = {McGraw-Hill},
  Year                     = {2005},

  Address                  = {New York},
  Edition                  = {11th ed.},
  Note                     = {Includes bibliographical references and index.},

  Keywords                 = {Histology
Tissues},
  Type                     = {Book}
}

@Article{Jurewicz1976,
  Title                    = {ASME WAM, Paper No. 76-WA/PE-33.},
  Author                   = {Jurewicz, J.T. and Stock, D.E.},
  Year                     = {1976},

  Type                     = {Journal Article}
}

@Article{Kaazempur-Mofrad2004,
  Title                    = {Characterization of the atherosclerotic carotid bifurcation using MRI, finite element modeling, and histology},
  Author                   = {Kaazempur-Mofrad, M. R. and Isasi, A. G. and Younis, H. F. and Chan, R. C. and Hinton, D. P. and Sukhova, G. and LaMuraglia, G. M. and Lee, R. T. and Kamm, R. D.},
  Journal                  = {Annals of Biomedical Engineering},
  Year                     = {2004},
  Note                     = {Cited By (since 1996): 48
Export Date: 5 June 2011
Source: Scopus},
  Number                   = {7},
  Pages                    = {932-946},
  Volume                   = {32},

  Type                     = {Journal Article},
  Url                      = {http://www.scopus.com/inward/record.url?eid=2-s2.0-3843084986&partnerID=40&md5=88cbdd2d3886123ac4651c04f4b1ebd3}
}

@Article{Kai1998,
  Title                    = {Rapid prototyping assisted surgery planning},
  Author                   = {Kai, Chua Chee and Meng, Chou Siaw and Ching, Lin Sing and Hoe, Eu Kee and Fah, Lew Kok},
  Journal                  = {The International Journal of Advanced Manufacturing Technology},
  Year                     = {1998},
  Note                     = {10.1007/BF01192281},
  Number                   = {9},
  Pages                    = {624-630},
  Volume                   = {14},

  Abstract                 = {In recent years, new surgical techniques have been developed to improve the quality of operations, reduce the risk to patients and reduce the pain experienced by patients. Prominent developments include minimally invasive surgery, robot-assisted hip operations, computer-assisted surgery (CAS) and virtual reality (VR). These developments have helped surgeons operate under difficult visual conditions.},
  Type                     = {Journal Article},
  Url                      = {http://dx.doi.org/10.1007/BF01192281}
}

@Article{Kalberlah2002,
  Title                    = {Time Extrapolation and Interspecies Extrapolation for Locally Acting Substances in case of Limited Toxicological Data},
  Author                   = {Kalberlah, F. and Fost, U. and Schneider, K.},
  Journal                  = {Annals of Occupational Hygiene},
  Year                     = {2002},
  Number                   = {2},
  Pages                    = {175-185},
  Volume                   = {46},

  Abstract                 = {In the case of substances with a limited toxicological data base there is often (i) a lack of qualified human toxicological data; and (ii) a paucity of studies with adequate exposure duration. Hence, several extrapolations have to be performed to arrive at appropriate risk assessments or derive occupational exposure limits. The present paper deals with the possibilities for extrapolating the change in effect concentrations over time (time extrapolation, e.g. from subacute to chronic exposure) and for interspecies extrapolation (from animal to human) in connection with locally acting substances (respiratory toxicants). To justify the time extrapolation factors, 46 technical reports produced by the US National Toxicology Program (NTP) involving studies with subacute, subchronic and chronic exposure duration were evaluated. On the basis of geometric mean values, decreases in effect concentrations by factors of 3.2 (subacute → subchronic), 2.7 (subchronic → chronic) and 6.6 (subacute → chronic) were found. Differentiation according to animal species (mouse, rat), sex or substance properties did not result in any relevant changes of the mean value. NTP studies with less than lifetime exposure periods (subacute, subchronic) in many cases showed different locations of respiratory effects compared with chronic studies, and thus offered limited possibilities for qualitative prediction of long-term respiratory effects (occurrence of effects in certain regions of the respiratory tract). With regard to interspecies extrapolation, gaseous and particulate substances were evaluated separately. With some modifications (e.g. consideration of the clearance of particles of low solubility), the 1994 US Environmental Protection Agency (EPA) model for deriving reference concentrations for humans on the basis of experimental data in animals is proposed for inhalable particulate substances. In the case of gaseous substances, the assumptions of the EPA model do not seem to consider sufficiently the local inhomogeneity in substance distribution and anatomical and histological differences between the upper respiratory tracts of rodents and humans. Considerable uncertainty would attach to a default factor for interspecies extrapolation for gaseous substances.},
  Doi                      = {10.1093/annhyg/mef014},
  Type                     = {Journal Article},
  Url                      = {http://annhyg.oxfordjournals.org/content/46/2/175.abstract}
}

@Article{Kallio1989,
  Title                    = {A numerical simulation of particle deposition in turbulent boundary layers},
  Author                   = {Kallio, G. A. and Reeks, M. W.},
  Journal                  = {International Journal of Multiphase Flow},
  Year                     = {1989},
  Note                     = {doi: DOI: 10.1016/0301-9322(89)90012-8},
  Number                   = {3},
  Pages                    = {433-446},
  Volume                   = {15},

  ISSN                     = {0301-9322},
  Keywords                 = {particles
deposition
turbulence},
  Type                     = {Journal Article},
  Url                      = {http://www.sciencedirect.com/science/article/B6V45-47YJFDM-9C/2/af5d93ef2e7ce563b4cce2636d3dc302}
}

@Article{Kalmovich2005,
  Title                    = {Endonasal geometry changes in elderly people: Acoustic rhinometry measurements},
  Author                   = {Kalmovich, L. M. and Elad, D. and Zaretsky, U. and Adunsky, A. and Chetrit, A. and Sadetzki, S. and Segal, S. and Wolf, M.},
  Journal                  = {Journals of Gerontology Series a-Biological Sciences and Medical Sciences},
  Year                     = {2005},
  Pages                    = {396-398},
  Volume                   = {60},

  Abstract                 = {Background. Skeletal nasal changes in elderly people have been extensively investigated, but data on variation of the endonasal architecture with age do not exist. We evaluated endonasal parameters in an elderly population as computed with those in a young group. Methods. Acoustic rhinometry measurements were performed on 165 participants in the age range of 20-93 years, The rhinograms provided the endonasal volume from the nostril entrance to a 7.0 cm cephalic point (VO-7) and the minimal cross-sectional areas (MCA1 and MCA2). Statistical analysis was performed by Pearson correlation and one-way analysis of variance, using age as a continuous or categorical variable. Results. There was no statistical difference in gender distribution within each age group. The results obtained for the left and fight nostrils were similar. Endonasal volume VO-7 and the narrowing areas MCA1 anti MCA2 significantly increase with age, except for men over 80 years in which a relative decrease was observed, Conclusion. Acoustic rhinometry examination of the endonasal architecture in a healthy Young and elderly population demonstrated a gradual increase of endonasal volumes and minimal cross-sectional areas with age,},
  ISSN                     = {1079-5006},
  Keywords                 = {Abridged Index Medicus
Index Medicus
nasal valve
nose},
  Type                     = {Journal Article},
  Url                      = {http://biomedgerontology.oxfordjournals.org/content/60/3/396.full.pdf}
}

@Article{Kalmovich2005a,
  Title                    = {Endonasal geometry changes in elderly people: acoustic rhinometry measurements},
  Author                   = {Kalmovich, Limor Muallem and Elad, David and Zaretsky, Uri and Adunsky, Abraham and Chetrit, Angela and Sadetzki, Siegal and Segal, Samuel and Wolf, Michael},
  Journal                  = {The journals of gerontology. Series A, Biological sciences and medical sciences},
  Year                     = {2005},
  Note                     = {Date completed - 2005-06-24
Date created - 2005-04-29
Date revised - 2014-01-13},
  Number                   = {3},
  Pages                    = {396-398},
  Volume                   = {60},

  Abstract                 = {Skeletal nasal changes in elderly people have been extensively investigated, but data on variation of the endonasal architecture with age do not exist. We evaluated endonasal parameters in an elderly population as compared with those in a young group. Acoustic rhinometry measurements were performed on 165 participants in the age range of 20-93 years. The rhinograms provided the endonasal volume from the nostril entrance to a 7.0 cm cephalic point (V0-7) and the minimal cross-sectional areas (MCA1 and MCA2). Statistical analysis was performed by Pearson correlation and one-way analysis of variance, using age as a continuous or categorical variable. There was no statistical difference in gender distribution within each age group. The results obtained for the left and right nostrils were similar. Endonasal volume V0-7 and the narrowing areas MCA1 and MCA2 significantly increase with age, except for men over 80 years in which a relative decrease was observed. Acoustic rhinometry examination of the endonasal architecture in a healthy young and elderly population demonstrated a gradual increase of endonasal volumes and minimal cross-sectional areas with age.},
  ISSN                     = {1079-5006, 1079-5006},
  Keywords                 = {Abridged Index Medicus
Index Medicus
Sensitivity and Specificity
Probability
Analysis of Variance
Nasal Obstruction -- epidemiology
Humans
Aged
Israel
Geriatric Assessment
Risk Assessment
Cross-Sectional Studies
Aged, 80 and over
Nasal Obstruction -- diagnosis
Middle Aged
Nasal Mucosa -- anatomy & histology
Female
Male
Prevalence
Aging -- physiology
Nasal Cavity -- anatomy & histology
Rhinometry, Acoustic -- methods
Nasal Cavity -- physiology},
  Type                     = {Journal Article},
  Url                      = {http://search.proquest.com/docview/67799244?accountid=13552
http://primoapac01.hosted.exlibrisgroup.com/openurl/RMITU/RMIT_SERVICES_PAGE??url_ver=Z39.88-2004&rft_val_fmt=info:ofi/fmt:kev:mtx:journal&genre=article&sid=ProQ:ProQ%3Amedlineshell&atitle=Endonasal+geometry+changes+in+elderly+people%3A+acoustic+rhinometry+measurements.&title=The+journals+of+gerontology.+Series+A%2C+Biological+sciences+and+medical+sciences&issn=10795006&date=2005-03-01&volume=60&issue=3&spage=396&au=Kalmovich%2C+Limor+Muallem%3BElad%2C+David%3BZaretsky%2C+Uri%3BAdunsky%2C+Abraham%3BChetrit%2C+Angela%3BSadetzki%2C+Siegal%3BSegal%2C+Samuel%3BWolf%2C+Michael&isbn=&jtitle=The+journals+of+gerontology.+Series+A%2C+Biological+sciences+and+medical+sciences&btitle=&rft_id=info:eric/&rft_id=info:doi/}
}

@Article{Kaminsky2004,
  Title                    = {Oscillation mechanics of the human lung periphery in asthma},
  Author                   = {Kaminsky, D.A. and Irvin, C.G. and Lundblad, L. and Moriya, H.T. and Lang, S. and Allen, J. and Viola, T. and Lynn, M. and Bates, J.H.T.},
  Journal                  = {J Appl. Physiol.},
  Year                     = {2004},
  Pages                    = {1849-1858},
  Volume                   = {97},

  Type                     = {Journal Article}
}

@Article{Kamm2002,
  Title                    = {Cellular Fluid Mechanics},
  Author                   = {Kamm, R.D.},
  Journal                  = {Annual Review of Fluid Mechanics},
  Year                     = {2002},
  Pages                    = {211-232},
  Volume                   = {34},

  Type                     = {Journal Article}
}

@Article{Kang2011,
  Title                    = {Effect of geometric variations on pressure loss for a model bifurcation of the human lung airway},
  Author                   = {Kang, Min-Yeong and Hwang, Jeongeun and Lee, Jin-Won},
  Journal                  = {Journal of Biomechanics},
  Year                     = {2011},
  Number                   = {6},
  Pages                    = {1196-1199},
  Volume                   = {44},

  Abstract                 = {Characteristics of pressure loss ([Delta]P) in human lung airways were numerically investigated using a realistic model bifurcation. Flow equations were numerically solved for the steady inspiratory condition with the tube length, the branching angle and flow velocity being varied over a wide range. In general, the [Delta]P coefficient K showed a power-law dependence on Reynolds number (Re) and length-to-diameter ratio with a different exponent for Re>=100 than for Re<100. The effect of different branching angles on pressure loss was very weak in the smooth-branching airways.},
  Doi                      = {10.1016/j.jbiomech.2011.02.011},
  ISSN                     = {0021-9290},
  Keywords                 = {Bifurcation model
Length to diameter
Branching angle
Developing flow
Computational fluid dynamics (CFD)},
  Type                     = {Journal Article},
  Url                      = {http://www.sciencedirect.com/science/article/pii/S0021929011000923}
}

@Article{Kang,
  Title                    = {A direct-forcing immersed boundary method for the thermal lattice Boltzmann method},
  Author                   = {Kang, Shin K. and Hassan, Yassin A.},
  Journal                  = {Computers \& Fluids},
  Volume                   = {In Press, Corrected Proof},

  Abstract                 = {In this study, a direct-forcing immersed boundary method (IBM) for thermal lattice Boltzmann method (TLBM) is proposed to simulate the non-isothermal flows. The direct-forcing IBM formulas for thermal equations are derived based on two TLBM models: a double-population model with a simplified thermal lattice Boltzmann equation (Model 1) and a hybrid model with an advection-diffusion equation of temperature (Model 2). As an interface scheme, which is required due to a mismatch between boundary and computational grids in the IBM, the sharp interface scheme based on second-order bilinear and linear interpolations (instead of the diffuse interface scheme, which uses discrete delta functions) is adopted to obtain the more accurate results. The proposed methods are validated through convective heat transfer problems with not only stationary but also moving boundaries - the natural convection in a square cavity with an eccentrically located cylinder and a cold particle sedimentation in an infinite channel. In terms of accuracy, the results from the IBM based on both models are comparable and show a good agreement with those from other numerical methods. In contrast, the IBM based on Model 2 is more numerically efficient than the IBM based on Model 1.},
  Doi                      = {10.1016/j.compfluid.2011.04.016},
  ISSN                     = {0045-7930},
  Keywords                 = {Direct-forcing immersed boundary method
Thermal lattice Boltzmann method
Double-population model
Hybrid thermal model
Natural convection
Moving particle},
  Type                     = {Journal Article},
  Url                      = {http://www.sciencedirect.com/science/article/pii/S0045793011001563}
}

@Article{Karabulut2005,
  Title                    = {CT Assessment of tracheal carinal angle and its determinants},
  Author                   = {Karabulut, N.},
  Journal                  = {British Journal of Radiology},
  Year                     = {2005},
  Pages                    = {787-790},
  Volume                   = {78},

  Type                     = {Journal Article}
}

@Article{Karino1988,
  Title                    = {Role of Hemodynamic Factors in Atherogenesis},
  Author                   = {Karino, T. and Asakura, T. and Mabuchi, S. },
  Journal                  = {Advances in Experimental Medicine and Biology},
  Year                     = {1988},
  Pages                    = {51-57},
  Volume                   = {242},

  Type                     = {Journal Article}
}

@Book{Karino1984,
  Title                    = {Role of blood cell-wall interactions in thrombogenesis and atherogenesis: a microrheological study},
  Author                   = {Karino, T and Goldsmith, H L},
  Year                     = {1984},
  Volume                   = {21},

  ISBN                     = {0006-355X},
  Pages                    = {587-601},
  Type                     = {Book},
  Url                      = {http://www.biomedsearch.com/nih/Role-blood-cell-wall-interactions/6487769.html}
}

@Article{Karino1980,
  Title                    = {Disturbed Flow in Models of Branching Vessels},
  Author                   = {Karino, T. and Goldsmith, H. L.},
  Journal                  = {ASAIO Journal},
  Year                     = {1980},
  Volume                   = {26},

  ISSN                     = {1058-2916},
  Type                     = {Journal Article},
  Url                      = {http://journals.lww.com/asaiojournal/Fulltext/1980/26000/Disturbed_Flow_in_Models_of_Branching_Vessels.100.aspx}
}

@Article{Karnis1967,
  Title                    = {Particle Motions in Sheared Suspensions},
  Author                   = {Karnis, A. and Mason, S.G. },
  Journal                  = {Journal of Colloid and Interface Science},
  Year                     = {1967},
  Pages                    = {161-169},
  Volume                   = {24},

  Type                     = {Journal Article}
}

@Misc{Kashdan2002,
  Title                    = {Automated quantitative interrogation of volumes to size high speed sprays},

  Author                   = {Kashdan, J.T. and Parasram, N.T. and Shrimpton, J.S. and Whybrew, A.},
  Month                    = {8-11 July 2002},
  Year                     = {2002},

  Publisher                = {Calouste Gulbenkian Foundation},
  Type                     = {Conference Paper}
}

@Article{Kashdan2004,
  Title                    = {Two-Phase Flow Characterization by Automated Digital Image Analysis. Part 2: Application ofPDIA for Sizing Sprays},
  Author                   = {Kashdan, J.T. and Shrimpton, J.S. and Whybrew, A.},
  Journal                  = {Particle \& Particle Systems Characterization},
  Year                     = {2004},
  Pages                    = {15-23},
  Volume                   = {21},

  Type                     = {Journal Article}
}

@Article{Kashdan2003,
  Title                    = {Two-Phase Flow Characterization by Automated Digital Image Analysis. Part 1: Fundamental Principles and Calibration of the Technique},
  Author                   = {Kashdan, J.T. and Shrimpton, J.S. and Whybrew, A.},
  Journal                  = {Particle \& Particle Systems Characterization},
  Year                     = {2003},
  Pages                    = {387-397},
  Volume                   = {20},

  Type                     = {Journal Article}
}

@Article{Katz1999,
  Title                    = {A numerical study of particle motion within the human larynx and trachea},
  Author                   = {Katz, I. M. and Davis, B. M. and Martonen, T. B.},
  Journal                  = {Journal of Aerosol Science},
  Year                     = {1999},
  Note                     = {doi: DOI: 10.1016/S0021-8502(98)00043-3},
  Number                   = {2},
  Pages                    = {173-183},
  Volume                   = {30},

  ISSN                     = {0021-8502},
  Type                     = {Journal Article},
  Url                      = {http://www.sciencedirect.com/science/article/B6V6B-3VFGVM6-5/2/bf06e2f4ca809844c72c98d179960c99}
}

@Article{Katz2011,
  Title                    = {The ventilation distribution of helium-oxygen mixtures and the role of inertial losses in the presence of heterogeneous airway obstructions},
  Author                   = {Katz, Ira M. and Martin, Andrew R. and Muller, Pierre-Antoine and Terzibachi, Karine and Feng, Chia-Hsiang and Caillibotte, Georges and Sandeau, Julien and Texereau, Joëlle},
  Journal                  = {Journal of Biomechanics},
  Year                     = {2011},
  Number                   = {6},
  Pages                    = {1137-1143},
  Volume                   = {44},

  Abstract                 = {The regional distribution of inhaled gas within the lung is affected in part by normal variations in airway geometry or by obstructions resulting from disease. In the present work, the effects of heterogeneous airway obstructions on the distribution of air and helium-oxygen were examined using an in vitro model, the two compartments of a dual adult test lung. Breathing helium-oxygen resulted in a consistently more uniform distribution, with the gas volume delivered to a severely obstructed compartment increased by almost 80%. An engineering approach to pipe flow was used to analyze the test lung and was extrapolated to a human lung model to show that the in vitro experimental parameters are relevant to the observed in vivo conditions. The engineering analysis also showed that helium-oxygen can decrease the relative weight of the flow resistance due to obstructions if they are inertial in nature (i.e., density dependent) due to either turbulence or laminar convective losses.},
  Doi                      = {10.1016/j.jbiomech.2011.01.022},
  ISSN                     = {0021-9290},
  Keywords                 = {Helium-oxygen
Turbulence
Respiratory mechanics
Mathematical model
Two-compartment test lung
Heterogeneous obstruction},
  Type                     = {Journal Article},
  Url                      = {http://www.sciencedirect.com/science/article/pii/S0021929011000534}
}

@Article{,
  Title                    = {Occupational Exposure to Inhalable Wood Dust in the Member States of the European Union},
  Author                   = {Kauppinen, Timo and Vincent, Raymond and Liukkonen, Tuula and Grzebyk, Michel and Kauppinen, Antti and Welling, Irma and Arezes, Pedro and Black, Nigel and Bochmann, Frank and Campelo, Filipe and Costa, Manuel and Elsigan, Gerhard and Goerens, Robert and Kikemenis, Anastasia and Kromhout, Hans and Miguel, Sergio and Mirabelli, Dario and Mceneany, Roisin and Pesch, Beate and Plato, Nils and Schlunssen, Vivi and Schulze, Johannes and Sonntag, Roland and Verougstraete, Violaine and De Vicente, Maria Angeles and Wolf, Joachim and Zimmermann, Marta and Husgafvel-Pursiainen, Kirsti and Savolainen, Kai},
  Journal                  = {Ann Occup Hyg},
  Year                     = {2006},
  Number                   = {6},
  Pages                    = {549-561},
  Volume                   = {50},

  Doi                      = {10.1093/annhyg/mel013},
  Type                     = {Journal Article},
  Url                      = {http://annhyg.oxfordjournals.org/cgi/content/abstract/50/6/549}
}

@Article{Kawaguchi1998,
  Title                    = {Numerical simulation of two-dimensional fluidized beds using the discrete element method (comparison between the two- and three-dimensional models)},
  Author                   = {Kawaguchi, T. and Tanaka, T. and Tsuji, Y.},
  Journal                  = {Powder Technology, 96(2):129–138, 1998.},
  Year                     = {1998},
  Number                   = {2},
  Pages                    = {129-138},
  Volume                   = {96},

  Type                     = {Journal Article}
}

@Article{Kaye1997,
  Title                    = {The influence of the branching pattern of the conducting airways on flow and aerosol deposition parameters in the human, dog, rat and hamster},
  Author                   = {Kaye, S. R. and Phillips, C. G.},
  Journal                  = {Journal of Aerosol Science},
  Year                     = {1997},
  Note                     = {doi: DOI: 10.1016/S0021-8502(97)00024-4},
  Number                   = {7},
  Pages                    = {1291-1300},
  Volume                   = {28},

  ISSN                     = {0021-8502},
  Type                     = {Journal Article},
  Url                      = {http://www.sciencedirect.com/science/article/B6V6B-3SWPY6N-D/2/e65974f0be46459a503ab1bc0adf9467}
}

@Article{Kazerooni2009,
  Title                    = {Simulation of turbulent flow through porous media employing a v2f model},
  Author                   = {Kazerooni, R.B. and Hannani, S.K.},
  Journal                  = {Transcation B: Mechanical Engineering},
  Year                     = {2009},
  Number                   = {2},
  Pages                    = {19-167},
  Volume                   = {16},

  Type                     = {Journal Article}
}

@Article{Keay1987,
  Title                    = {The nasal cycle and clinical examination of the nose},
  Author                   = {Keay, D. and Smith, I. and White, A. and Hardcastle, P.F.},
  Journal                  = {Clinical Otolaryngology},
  Year                     = {1987},
  Pages                    = {345-348},
  Volume                   = {12},

  Type                     = {Journal Article}
}

@Article{Keck2000,
  Title                    = {Humidity and temperature profiles in the nasal cavity},
  Author                   = {Keck, T. and Leiacker, R. and Heinrich, A. and Kuhneman, S. and Rettinger, G.},
  Journal                  = {Rhinology},
  Year                     = {2000},
  Number                   = {167-171},
  Volume                   = {38},

  Type                     = {Journal Article},
  Url                      = {http://www.rhinologyjournal.com/abstract.php?id=55}
}

@Article{Keck2000a,
  Title                    = {Detection of particles within the nasal airways during respiration.},
  Author                   = {Keck, T. and Leiacker, R. and Klotz, M. and Lindemann, J.},
  Journal                  = {Eur Arch Oto-Rhino-L},
  Year                     = {2000},
  Pages                    = {493-497},
  Volume                   = {257},

  Type                     = {Journal Article},
  Url                      = {http://download.springer.com/static/pdf/429/art%253A10.1007%252Fs004050000283.pdf?auth66=1412669337_ef83667c989bd713e8b02a8b274eb19d&ext=.pdf }
}

@Article{Keck2001,
  Title                    = {Heating of air in the nasal airways},
  Author                   = {Keck, T. and Leiacker, R. and Meixner, D.},
  Journal                  = {HNO},
  Year                     = {2001},
  Pages                    = {36-40},
  Volume                   = {49},

  Type                     = {Journal Article}
}

@Article{Keck2000b,
  Title                    = {Temperature profile in the nasal cavity},
  Author                   = {Keck, T. and Leiacker, R. and Riechelmann, H. and Rettinger, G.},
  Journal                  = { Laryngoscope},
  Year                     = {2000},
  Pages                    = {651-654},
  Volume                   = {110},

  Type                     = {Journal Article},
  Url                      = {http://onlinelibrary.wiley.com.ezproxy.lib.rmit.edu.au/doi/10.1097/00005537-200004000-00021/pdf}
}

@Article{Keck2006,
  Title                    = {Upper airway function: objective measures and computer simulation},
  Author                   = {Keck, T. and Rozsasi, A.},
  Journal                  = {Journal of Biomechanics},
  Year                     = {2006},
  Number                   = {Supplement 1},
  Pages                    = {S271-S271},
  Volume                   = {39},

  ISSN                     = {0021-9290},
  Type                     = {Journal Article},
  Url                      = {http://www.sciencedirect.com/science/article/B6T82-4KR88PB-1G2/2/e9945144379ffb00b854f0ef7c3ac099}
}

@Article{Keck2011,
  Title                    = {Nasal air-conditioning},
  Author                   = {Keck, T. and Rozsasi, A. and Gruen, P. M.},
  Journal                  = {HNO},
  Year                     = {2011},
  Pages                    = {38+},
  Volume                   = {59},

  Abstract                 = {We present a review of nasal-air conditioning, a process essential to undisturbed gas exchange and cleansing of the respiratory mucosa in the nose. A selective literature review was made on the basis of in vivo measurements and computer simulation of the upper airways as well as the authors' own clinical and experimental data. Healthy subjects normally breathe through the nose, although the nasal airways have significantly higher airway resistance compared to the oral cavity, which is opened for breathing during exercise, in the case of nasal airway blockage, or in allergic rhinitis. In addition to olfaction, the main tasks of nasal breathing include: cleansing, defense, and conditioning (i.e., humidification and heating). The current knowledge of nasal conditioning processes will be discussed. In addition, research activities of particular relevance for diagnosis and intervention in various pathologies of the upper airways will be presented.},
  Doi                      = {10.1007/s00106-010-2219-2},
  ISSN                     = {0017-6192},
  Keywords                 = {Airway obstruction
Nasal cavity
Nasal obstruction
Respiratory physiological phenomena
Turbinates},
  Type                     = {Journal Article},
  Url                      = {http://download.springer.com/static/pdf/251/art%253A10.1007%252Fs00106-010-2219-2.pdf?auth66=1411539381_1500837c4f5287f8d718c479de05214c&ext=.pdf}
}

@Article{Keir2009,
  Title                    = {Why do we have paranasal sinuses?},
  Author                   = {Keir, J.},
  Journal                  = {The Journal of Laryngology \& Otology},
  Year                     = {2009},
  Pages                    = {4-8},
  Volume                   = {123},

  Type                     = {Journal Article}
}

@Article{Kelly2004,
  Title                    = {Particle deposition in human nasal airway replicas manufactured by different methods. Part 1: Inertial regime particles},
  Author                   = {Kelly, J.T. and Asgharian, B. and Kimbell, J.S. and B.A., Wong.},
  Journal                  = {Aerosol Science and Technology},
  Year                     = {2004},
  Number                   = {11},
  Pages                    = {1063-1071},
  Volume                   = {38},

  Type                     = {Journal Article}
}

@Article{Kelly2004a,
  Title                    = {Particle Deposition in Human Nasal Airway Replicas Manufactured by Different Methods. Part II: Ultrafine Particles},
  Author                   = {Kelly, James and Asgharian, Bahman and Kimbell, Julia and Wong, Brian},
  Journal                  = {Aerosol Science and Technology},
  Year                     = {2004},
  Number                   = {11},
  Pages                    = {1072 - 1079},
  Volume                   = {38},

  ISSN                     = {0278-6826},
  Type                     = {Journal Article},
  Url                      = {http://www.informaworld.com/10.1080/027868290883432}
}

@Article{Kelly2000,
  Title                    = {Detailed flow patterns in the nasal cavity.},
  Author                   = {Kelly, J.T. and Prasad, A.K. and Wexler, A.S.},
  Journal                  = {J Appl. Physiol.},
  Year                     = {2000},
  Pages                    = {323-337},
  Volume                   = {89},

  Type                     = {Journal Article}
}

@Article{Kemmotsu2006,
  Title                    = {Restrained eaters show altered brain response to food odor},
  Author                   = {Kemmotsu, Nobuko and Murphy, Claire},
  Journal                  = {Physiology \& Behavior},
  Year                     = {2006},
  Number                   = {2},
  Pages                    = {323-329},
  Volume                   = {87},

  Abstract                 = {Do restrained and unrestrained eaters differ in their brain response to food odor? We addressed this question by examining restrained eaters' brain response to food (chocolate) and non-food (geraniol, floral) odors, both when odor was attended to and when ignored. Using olfactory event-related potentials (OERPs), we found that restrained eaters and controls responded similarly to the non-food odor; however, unlike controls, restrained eaters showed no increase in brain response to the food odor when they focused attention on it. Rather, restrained eaters showed attenuated OERP amplitudes to the food odor in both attended and ignored conditions, suggesting that the brain's response to attended food odor was abnormally suppressed.},
  ISSN                     = {0031-9384},
  Keywords                 = {Dietary restraint
Olfactory
Event-related potential
ERP
Attention
Odor},
  Type                     = {Journal Article},
  Url                      = {http://www.sciencedirect.com/science/article/B6T0P-4HYD9N0-2/2/9cc83aa38cc570adce333ba1f8f5df17}
}

@Article{Kennedy2002,
  Title                    = {Inhalability of large solid particles},
  Author                   = {Kennedy, N.J. and Hinds, W.C.},
  Journal                  = {Journal of Aerosol Science},
  Year                     = {2002},
  Pages                    = {237-255},
  Volume                   = {33},

  Type                     = {Journal Article}
}

@InBook{Kerr1997,
  Title                    = {Rhinology},
  Author                   = {Kerr, A.},
  Publisher                = {Butterworth-Heinemann},
  Year                     = {1997},

  Address                  = {Oxford},
  Type                     = {Book Section},

  Booktitle                = {Scott-Brown's Otolaryngology 6th Ed.}
}

@Article{Kesavanathan1998,
  Title                    = {The effect of nasal passage characteristics on particle deposition},
  Author                   = {Kesavanathan, J. and Bascom, R. and Swift, D.L.},
  Journal                  = {Journal of Aerosol Medicine},
  Year                     = {1998},
  Number                   = {27-39},
  Volume                   = {11},

  Type                     = {Journal Article}
}

@Article{Kesavanathan1998a,
  Title                    = {Human nasal passage particle deposition: The effect of particle size, flow rate and anatomical factors.},
  Author                   = {Kesavanathan, J. and Swift, D.L.},
  Journal                  = {Aerosol Science and Technology},
  Year                     = {1998},
  Number                   = {457-463},
  Volume                   = {28},

  Type                     = {Journal Article}
}

@Article{Keyhani1997,
  Title                    = {A numerical model of nasal odorant transport for the analysis of human olfaction},
  Author                   = {Keyhani, K. and Scherer, P.W. and Mozell, M.M},
  Journal                  = {JournalTheoretical Biology},
  Year                     = {1997},
  Pages                    = {279-301},
  Volume                   = {186},

  Type                     = {Journal Article}
}

@Article{Keyhani1997a,
  Title                    = {A Numerical Model of Nasal Odorant Transport for the Analysis of Human Olfaction},
  Author                   = {Keyhani, Keyvan and Scherer, Peter W. and Mozell, Maxwell M.},
  Journal                  = {Journal of Theoretical Biology},
  Year                     = {1997},
  Number                   = {3},
  Pages                    = {279-301},
  Volume                   = {186},

  Abstract                 = {The transport and uptake of inspired odorant molecules in the human nasal cavity were determined using an anatomically correct three-dimensional finite element model. The steady-state equations of motion and continuity were first solved to determine laminar flow patterns of odorous air at quiet breathing flow rates. The air stream entering the ventral tip of the naris travelled to the olfactory slit, and then passed through the slit in nearly a straight path without forming separated recirculating zones. The fraction of volumetric flow passing through the olfactory airway was about 10%, and remained nearly constant with variations in flow rate. The three-dimensional inspiratory velocity field was used in the solution of the uncoupled steady convective-diffusion equation to determine the concentration field in the airways and odorant mass flux at the nasal walls. The mass-transfer boundary condition used at the nasal cavity wall included the effects of solubility and diffusivity of odorants in the mucosal lining, and the thickness of the mucus layer. The total olfactory flux of odorants, that is highly correlated with perceived odor intensity, was determined as a function of all transport parameters in our model. Increase in nasal flow rate at a constant inlet concentration resulted in an increase in total olfactory uptake for all odorants. However, with increase in flow rate, the fractional uptake, i.e., total olfactory flux normalized by convective flux at the inlet, decreased for poorly soluble odorants, while it increased for highly soluble odorants. The pattern of flux (or imposed patterning) across the olfactory mucosa, that carries information concerning odor identity, was also determined as a function of transport parameters. There was an overall decrease in odorant flux as the location on the olfactory surface was varied from the anterior towards the posterior and from the inferior towards the superior ends. The flux pattern became more uniform, i.e., the steepness of the flux gradients across the olfactory surface decreased, as the mucus solubility of the odorants decreased. Different odorants generated discernibly different flux patterns across the olfactory mucosa that may contribute to the encoding of odor quality. Variation of total olfactory flux with time after cessation of airflow was determined by solving the unsteady diffusion equation in the air-phase. The flux decreased approximately exponentially with time. The rate of decay decreased as solubility and diffusivity decreased, but was very rapid over a wide range of the parameters, with time constants of less that 0.5 s for most odorants, implying a rapid decrease in perceived odor intensity with cessation of nasal airflow.},
  Doi                      = {http://dx.doi.org/10.1006/jtbi.1996.0347},
  ISSN                     = {0022-5193},
  Type                     = {Journal Article},
  Url                      = {http://www.sciencedirect.com/science/article/pii/S0022519396903471}
}

@Article{Keyhani1997b,
  Title                    = {A Numerical Model of Nasal Odorant Transport for the Analysis of Human Olfaction},
  Author                   = {Keyhani, Keyvan and Scherer, Peter W. and Mozell, Maxwell M.},
  Journal                  = {Journal of Theoretical Biology},
  Year                     = {1997},
  Number                   = {3},
  Pages                    = {279-301},
  Volume                   = {186},

  Abstract                 = {The transport and uptake of inspired odorant molecules in the human nasal cavity were determined using an anatomically correct three-dimensional finite element model. The steady-state equations of motion and continuity were first solved to determine laminar flow patterns of odorous air at quiet breathing flow rates. The air stream entering the ventral tip of the naris travelled to the olfactory slit, and then passed through the slit in nearly a straight path without forming separated recirculating zones. The fraction of volumetric flow passing through the olfactory airway was about 10%, and remained nearly constant with variations in flow rate. The three-dimensional inspiratory velocity field was used in the solution of the uncoupled steady convective-diffusion equation to determine the concentration field in the airways and odorant mass flux at the nasal walls. The mass-transfer boundary condition used at the nasal cavity wall included the effects of solubility and diffusivity of odorants in the mucosal lining, and the thickness of the mucus layer. The total olfactory flux of odorants, that is highly correlated with perceived odor intensity, was determined as a function of all transport parameters in our model. Increase in nasal flow rate at a constant inlet concentration resulted in an increase in total olfactory uptake for all odorants. However, with increase in flow rate, the fractional uptake, i.e., total olfactory flux normalized by convective flux at the inlet, decreased for poorly soluble odorants, while it increased for highly soluble odorants. The pattern of flux (or imposed patterning) across the olfactory mucosa, that carries information concerning odor identity, was also determined as a function of transport parameters. There was an overall decrease in odorant flux as the location on the olfactory surface was varied from the anterior towards the posterior and from the inferior towards the superior ends. The flux pattern became more uniform, i.e., the steepness of the flux gradients across the olfactory surface decreased, as the mucus solubility of the odorants decreased. Different odorants generated discernibly different flux patterns across the olfactory mucosa that may contribute to the encoding of odor quality. Variation of total olfactory flux with time after cessation of airflow was determined by solving the unsteady diffusion equation in the air-phase. The flux decreased approximately exponentially with time. The rate of decay decreased as solubility and diffusivity decreased, but was very rapid over a wide range of the parameters, with time constants of less that 0.5 s for most odorants, implying a rapid decrease in perceived odor intensity with cessation of nasal airflow.},
  Doi                      = {http://dx.doi.org/10.1006/jtbi.1996.0347},
  ISSN                     = {0022-5193},
  Type                     = {Journal Article},
  Url                      = {http://www.sciencedirect.com/science/article/pii/S0022519396903471}
}

@Article{Keyhani1997c,
  Title                    = {A Numerical Model of Nasal Odorant Transport for the Analysis of Human Olfaction},
  Author                   = {Keyhani, Keyvan and Scherer, Peter W. and Mozell, Maxwell M.},
  Journal                  = {Journal of Theoretical Biology},
  Year                     = {1997},
  Pages                    = {279-301},
  Volume                   = {186},

  Abstract                 = {The transport and uptake of inspired odorant molecules in the human nasal cavity were determined using an anatomically correct three-dimensional finite element model. The steady-state equations of motion and continuity were first solved to determine laminar flow patterns of odorous air at quiet breathing flow rates. The air stream entering the ventral tip of the naris travelled to the olfactory slit, and then passed through the slit in nearly a straight path without forming separated recirculating zones. The fraction of volumetric flow passing through the olfactory airway was about 10%, and remained nearly constant with variations in flow rate. The three-dimensional inspiratory velocity field was used in the solution of the uncoupled steady convective-diffusion equation to determine the concentration field in the airways and odorant mass flux at the nasal walls. The mass-transfer boundary condition used at the nasal cavity wall included the effects of solubility and diffusivity of odorants in the mucosal lining, and the thickness of the mucus layer. The total olfactory flux of odorants, that is highly correlated with perceived odor intensity, was determined as a function of all transport parameters in our model. Increase in nasal flow rate at a constant inlet concentration resulted in an increase in total olfactory uptake for all odorants. However, with increase in flow rate, the fractional uptake, i.e., total olfactory flux normalized by convective flux at the inlet, decreased for poorly soluble odorants, while it increased for highly soluble odorants. The pattern of flux (or imposed patterning) across the olfactory mucosa, that carries information concerning odor identity, was also determined as a function of transport parameters. There was an overall decrease in odorant flux as the location on the olfactory surface was varied from the anterior towards the posterior and from the inferior towards the superior ends. The flux pattern became more uniform, i.e., the steepness of the flux gradients across the olfactory surface decreased, as the mucus solubility of the odorants decreased. Different odorants generated discernibly different flux patterns across the olfactory mucosa that may contribute to the encoding of odor quality. Variation of total olfactory flux with time after cessation of airflow was determined by solving the unsteady diffusion equation in the air-phase. The flux decreased approximately exponentially with time. The rate of decay decreased as solubility and diffusivity decreased, but was very rapid over a wide range of the parameters, with time constants of less that 0.5 s for most odorants, implying a rapid decrease in perceived odor intensity with cessation of nasal airflow.},
  Doi                      = {http://dx.doi.org/10.1006/jtbi.1996.0347},
  ISSN                     = {0022-5193},
  Type                     = {Journal Article},
  Url                      = {http://ac.els-cdn.com/S0022519396903471/1-s2.0-S0022519396903471-main.pdf?_tid=cf104a5c-421f-11e4-83b9-00000aacb360&acdnat=1411366691_94757c671967573ce9c4cad9fba32cb7}
}

@Article{Khan2009,
  Title                    = {Spread of a Novel Influenza A (H1N1) Virus via Global Airline Transportation},
  Author                   = {Khan, Kamran M. D. M. P. H. and Arino, Julien PhD and Hu, Wei BSc and Raposo, Paulo BSc and Sears, Jennifer BSc and Calderon, Felipe MSc and Heidebrecht, Christine MSc and Macdonald, Michael M. S. A. and Liauw, Jessica BHSc and Chan, Angie M. P. A. and Gardam, Michael M. D. MSc},
  Journal                  = {The New England Journal of Medicine},
  Year                     = {2009},
  Note                     = {版� - Copyright © 2009 Massachusetts Medical Society. All rights reserved.
最近更新 - 2014-03-22
CODEN - NEJMAG
1. Outbreak of swine-origin influenza A (H1N1) virus infection--Mexico, March-April 2009. MMWR Morb Mortal Wkly Rep 2009;58:467-470
2. The Bio.Diaspora Project. Toronto:St. Michael's Hospital. (Accessed June 18, 2009, at http://www.biodiaspora.com.)
3. Global Public Health Intelligence Network (GPHIN). Ottawa:Public Health Agency of Canada. (Accessed June 18, 2009, at http://www.phac-aspc.gc.ca/media/nr-rp/2004/2004_gphin-rmispbk-eng.php.)
4. International Health Regulations. Geneva:World Health Organization. (Accessed June 18, 2009, at http://www.who.int/topics/international_health_regulations/en/.)
5. The World Health Report 2007 -- a safer future:global public health security in the 21st century. Geneva:World Health Organization. (Accessed June 18, 2009, at http://www.who.int/whr/2007/en/index.html.)},
  Number                   = {2},
  Pages                    = {212-4},
  Volume                   = {361},

  Abstract                 = {In response to the outbreak of the severe acute respiratory syndrome in Toronto, an initiative called the Bio.Diaspora Project2 was developed to conduct rapid assessments of global infectious-disease alerts3 or confirmed epidemics by evaluating the probable pathways of international dissemination from any commercial airport worldwide at any point in time.\n},
  Doi                      = {http://dx.doi.org/10.1056/NEJMc0904559},
  ISSN                     = {00284793},
  Keywords                 = {Medical Sciences
Passengers
International
Travel
Mexico
ROC Curve
Humans
Aerospace Medicine
Influenza, Human -- virology
Aircraft
Influenza, Human -- transmission
Influenza A Virus, H1N1 Subtype},
  Type                     = {Journal Article},
  Url                      = {http://search.proquest.com/docview/223911518?accountid=13552
http://primoapac01.hosted.exlibrisgroup.com/openurl/RMITU/RMIT_SERVICES_PAGE??url_ver=Z39.88-2004&rft_val_fmt=info:ofi/fmt:kev:mtx:journal&genre=unknown&sid=ProQ:ProQ%3Ahealthcompleteshell&atitle=Spread+of+a+Novel+Influenza+A+%28H1N1%29+Virus+via+Global+Airline+Transportation&title=The+New+England+Journal+of+Medicine&issn=00284793&date=2009-07-09&volume=361&issue=2&spage=212&au=Khan%2C+Kamran%2C+MD%2C+MPH%3BArino%2C+Julien%2C+PhD%3BHu%2C+Wei%2C+BSc%3BRaposo%2C+Paulo%2C+BSc%3BSears%2C+Jennifer%2C+BSc%3BCalderon%2C+Felipe%2C+MSc%3BHeidebrecht%2C+Christine%2C+MSc%3BMacdonald%2C+Michael%2C+MSA%3BLiauw%2C+Jessica%2C+BHSc%3BChan%2C+Angie%2C+MPA%3BGardam%2C+Michael%2C+MD%2C+MSc&isbn=&jtitle=The+New+England+Journal+of+Medicine&btitle=&rft_id=info:eric/&rft_id=info:doi/10.1056%2FNEJMc0904559}
}

@Article{Khan2009a,
  Title                    = {Spread of a Novel Influenza A (H1N1) Virus via Global Airline Transportation},
  Author                   = {Khan, Kamran M. D. M. P. H. and Arino, Julien PhD and Hu, Wei BSc and Raposo, Paulo BSc and Sears, Jennifer BSc and Calderon, Felipe MSc and Heidebrecht, Christine MSc and Macdonald, Michael M. S. A. and Liauw, Jessica BHSc and Chan, Angie M. P. A. and Gardam, Michael M. D. MSc},
  Journal                  = {The New England Journal of Medicine},
  Year                     = {2009},
  Pages                    = {212-4},
  Volume                   = {361},

  Abstract                 = {In response to the outbreak of the severe acute respiratory syndrome in Toronto, an initiative called the Bio.Diaspora Project2 was developed to conduct rapid assessments of global infectious-disease alerts3 or confirmed epidemics by evaluating the probable pathways of international dissemination from any commercial airport worldwide at any point in time.\n},
  Doi                      = {http://dx.doi.org/10.1056/NEJMc0904559},
  ISSN                     = {00284793},
  Keywords                 = {H1N1 Subtype
Human – transmission Influenza A Virus
Human – virology Aircraft Influenza
Medical Sciences Passengers International Travel Mexico ROC Curve Humans Aerospace Medicine Influenza},
  Type                     = {Journal Article},
  Url                      = {http://www.nejm.org/doi/pdf/10.1056/NEJMc0904559}
}

@Article{Kiely2000,
  Title                    = {Cardiovascular risk factors in patients with obstructive sleep apnoea syndrome},
  Author                   = {Kiely, JL and McNicholas, WT},
  Journal                  = {Eur Respir J},
  Year                     = {2000},
  Number                   = {1},
  Pages                    = {128-133},
  Volume                   = {16},

  Abstract                 = {Cardiovascular disorders are common in patients with obstructive sleep apnoea syndrome (OSAS) but there is debate as to whether OSAS is an independent risk factor for their development, since OSAS may be associated with other disorders and risk factors that predispose to cardiovascular disease. In an effort to quantify the risk of OSAS patients for cardiovascular disease arising from these other factors, the authors assessed the future risk for cardiovascular disease among a group of 114 consecutive patients with established OSAS prior to nasal continuous positive airway pressure therapy, using an established method of risk prediction employed in the Framingham studies. Patients were 100 males, aged (mean+/-SD) 52+/-9.0 yrs, and 14 females, aged 51+/-10.4 yrs, with an apnoea/hypopnoea index of 45+/-22 x h(-1). Based on either a prior diagnosis, or a mean of three resting blood pressure recordings >140 mmHg systolic and/or 90 diastolic, 68% of patients were hypertensive. Only 18% were current smokers, while 16% had either diabetes mellitus or impaired glucose tolerance, and 63% had elevated fasting cholesterol and/or triglyceride levels. The estimated 10-yr risk of a coronary heart disease (CHD) event in males was (mean+/-SEM) 13.9+/-0.9%, 95% confidence interval (95% CI) 12.1-16.0, and for a stroke was 12.3+/-1.4%; 95% CI 9.4-15.1, with a combined 10 yr risk for stroke and CHD events of 32.9+/-2.7%; 95% CI 27.8-38.5 in males aged >53 yrs. These findings indicate that obstructive sleep apnoea syndrome patients are at high risk of future cardiovascular disease from factors other than obstructive sleep apnoea syndrome, and may help explain the difficulties in identifying a potential independent risk from obstructive sleep apnoea syndrome.},
  Type                     = {Journal Article},
  Url                      = {http://erj.ersjournals.com/cgi/content/abstract/16/1/128}
}

@Article{Kim1999,
  Title                    = {Deposition characteristics of aerosol particles in sequentially bifurcating airway models.},
  Author                   = {Kim, C.S. and Fisher, D.M.},
  Journal                  = {Aerosol Science and Technology},
  Year                     = {1999},
  Pages                    = {198-220},
  Volume                   = {31},

  Type                     = {Journal Article}
}

@Article{Kim1996,
  Title                    = {Assessment of regional deposition of inhaled particles in human lungs by serial bolus delivery method},
  Author                   = {Kim, C.S. and Hu, S.C. and Dewitt, P. and Gerrity, T.R.},
  Journal                  = {Journal Appl. Physiol.},
  Year                     = {1996},
  Pages                    = {2203-2213},
  Volume                   = {81},

  Type                     = {Journal Article}
}

@Article{Kim1989,
  Title                    = {Deposition pf Inhaled Particles in Bifurcating Airways Models: I. Inspiratory Deposition},
  Author                   = {Kim, C.S. and Iglesias, A.J. },
  Year                     = {1989},
  Pages                    = {1-14},
  Volume                   = {2},

  Type                     = {Journal Article}
}

@Article{Kim1997,
  Title                    = {Comparative measurement of lung deposition of inhaled fine particles in normal subjects and patients with obstructive airway disease},
  Author                   = {Kim, CS and Kang, TC},
  Journal                  = {Am. Journal Respir. Crit. Care Medicine},
  Year                     = {1997},
  Number                   = {3},
  Pages                    = {899-905},
  Volume                   = {155},

  Type                     = {Journal Article},
  Url                      = {http://ajrccm.atsjournals.org/cgi/content/abstract/155/3/899}
}

@Article{Kim1997a,
  Title                    = {Comparative measurement of lung deposition of inhaled fine particles in normal subjects and patients with obstructive airway disease},
  Author                   = {Kim, C.S. and Kang, T.C.},
  Journal                  = {American Journal of Respiratory and Critical Care Medicine},
  Year                     = {1997},
  Pages                    = {899-905},
  Volume                   = {155},

  Type                     = {Journal Article}
}

@Article{Kim1989a,
  Title                    = {Aerosol Deposition in the Lung with Obstructed Airways},
  Author                   = {Kim, Chong S.},
  Journal                  = {Journal of Aerosol Medicine},
  Year                     = {1989},
  Number                   = {2},
  Pages                    = {111-120},
  Volume                   = {2},

  Doi                      = {doi:10.1089/jam.1989.2.111},
  Type                     = {Journal Article},
  Url                      = {http://www.liebertonline.com/doi/abs/10.1089/jam.1989.2.111}
}

@Article{Kim1983,
  Title                    = {Deposition of aerosol particles and flow resistance in mathematical and experimental airway models},
  Author                   = {Kim, C. S. and Brown, L. K. and Lewars, G. G. and Sackner, M. A.},
  Journal                  = {Journal of Applied Physiology},
  Year                     = {1983},
  Number                   = {1},
  Pages                    = {154-163},
  Volume                   = {55},

  Type                     = {Journal Article},
  Url                      = {http://jap.physiology.org/cgi/content/abstract/55/1/154}
}

@Article{Kim1989b,
  Title                    = {Aerosol deposition in the lung with asymmetric airways obstruction: in vivo observation},
  Author                   = {Kim, C. S. and Eldridge, M. A. and Garcia, L. and Wanner, A.},
  Journal                  = {Journal of Applied Physiology},
  Year                     = {1989},
  Number                   = {6},
  Pages                    = {2579-2585},
  Volume                   = {67},

  Type                     = {Journal Article},
  Url                      = {http://jap.physiology.org/cgi/content/abstract/67/6/2579}
}

@Article{Kim1994,
  Title                    = {Particle deposition in bifurcating airway models with varying airway geometry},
  Author                   = {Kim, Chong S. and Fisher, Donald M. and Lutz, David J. and Gerrity, Timothy R.},
  Journal                  = {Journal of Aerosol Science},
  Year                     = {1994},
  Note                     = {doi: DOI: 10.1016/0021-8502(94)90072-8},
  Number                   = {3},
  Pages                    = {567-581},
  Volume                   = {25},

  ISSN                     = {0021-8502},
  Type                     = {Journal Article},
  Url                      = {http://www.sciencedirect.com/science/article/B6V6B-4893VYW-3N/2/8b459ac458e5efe1f112f6b03ab34759}
}

@Article{Kim2006,
  Title                    = {Deposition and coagulation of polydisperse nanoparticles by Brownian motion and turbulence},
  Author                   = {Kim, D. S. and Hong, S. B. and Kim, Y. J. and Lee, K. W.},
  Journal                  = {Journal of Aerosol Science},
  Year                     = {2006},
  Note                     = {doi: DOI: 10.1016/j.jaerosci.2006.07.001},
  Number                   = {12},
  Pages                    = {1781-1787},
  Volume                   = {37},

  ISSN                     = {0021-8502},
  Keywords                 = {Deposition
Turbulent coagulation
Brownian coagulation
Polydisperse aerosols
Coagulation rate},
  Type                     = {Journal Article},
  Url                      = {http://www.sciencedirect.com/science/article/B6V6B-4KV2YBJ-1/2/95b6ce3b28328cd97ea328b24db96281}
}

@Article{Kim1987,
  Title                    = {Turbulence statistics in fully developed channel flow at low Reynolds number},
  Author                   = {Kim, J. and Moin, P. and Moser, R.},
  Journal                  = {Journal Fluid Mechanics},
  Year                     = {1987},
  Pages                    = {133-166},
  Volume                   = {177},

  Type                     = {Journal Article}
}

@Article{Kim2006a,
  Title                    = {Particle image velocimetry measurements for the study of nasal airflow},
  Author                   = {Kim, Jin Kook and Yoon, Joo-Heon and Kim, Chang Hoon and Nam, Tae Wook and Shim, Dae Bo and Shin, Hyang Ae},
  Journal                  = {Acta Oto-Laryngologica},
  Year                     = {2006},
  Number                   = {3},
  Pages                    = {282 - 287},
  Volume                   = {126},

  Abstract                 = {<i>Conclusions</i>. Particle image velocimetry (PIV) permits investigation of the distribution and velocity of the airflow in the nasal cavity. During breathing, the main laminar flow stream passes through the middle meatus and turbulent flow can be detected under physiologic conditions. <i>Objectives</i>. Physical models or casts of the nasal cavity have been utilized in several studies in an effort to understand its aerodynamics. PIV is a new technique for measuring the aerodynamic properties of tubular structures. In this article we evaluate nasal airflow characteristics during physiologic breathing under normal conditions and the usefulness of PIV. <i>Material and methods</i>. A nasal model cast obtained by a combination of rapid prototyping and solidification of clear silicone was connected to a pump which simulated the physiological pressure in the upper airway system. A glycerol–water mixture was used as the flow material. The airstream was marked with spherical polyvinyl particles, observed through solidified clear silicone and analyzed using PIV. <i>Results</i>. The main flow within the cavity, which was mostly laminar, passed through the middle meatus. Turbulence was clearly visible in the anteroinferior part of the middle turbinate. The flow rate was highest at the middle meatus during inspiration and expiration.},
  ISSN                     = {0001-6489},
  Type                     = {Journal Article},
  Url                      = {http://www.informaworld.com/10.1080/00016480500361320}
}

@Article{Kim1997b,
  Title                    = {Airway goblet cell mucin: its structure and regulation of secretion},
  Author                   = {Kim, KC and McCracken, K and Lee, BC and Shin, CY and Jo, MJ and Lee, CJ and Ko, KH},
  Journal                  = {Eur Respir J},
  Year                     = {1997},
  Number                   = {11},
  Pages                    = {2644-2649},
  Volume                   = {10},

  Abstract                 = {Mucociliary clearance is a major function of the airway epithelium. This important function depends both on the physicochemical properties of the airway mucus and on the activity of the cilia. The former, in turn, is dependent mainly on the quality and quantity of mucous glycoproteins or mucins, which are produced by two different cell types, namely, goblet cells of the epithelium and mucous cells of the submucosal gland. Neither the structural nor the functional differences of mucins produced by these two cell types are yet known. The availability of primary airway epithelial cell culture systems, however, has made it possible to study the structure and regulation of airway goblet cells to some extent. The epithelial mucins are extremely hydrophobic and are associated with various macromolecules, the quality and quantity of which may also affect the physicochemical properties of the mucus. Secretion of epithelial mucins is stimulated by various factors, including a number of inflammatory agents. The recent progress in mucin molecular biological research will allow us to identify different mucin core proteins produced by those different cell types, and, hopefully, the differential functions of these mucins in health and disease.},
  Type                     = {Journal Article},
  Url                      = {http://erj.ersjournals.com/cgi/content/abstract/10/11/2644}
}

@Article{Kim2013,
  Title                    = {Patient specific CFD models of nasal airflow: Overview of methods and challenges},
  Author                   = {Kim, Sung Kyun and Na, Yang and Kim, Jee-In and Chung, Seung-Kyu},
  Journal                  = {Journal of Biomechanics},
  Year                     = {2013},
  Pages                    = {299-306},
  Volume                   = {46},

  Abstract                 = {Respiratory physiology and pathology are strongly dependent on the airflow inside the nasal cavity. However, the nasal anatomy, which is characterized by complex airway channels and significant individual differences, is difficult to analyze. Thus, commonly adopted diagnostic tools have yielded limited success. Nevertheless, with the rapid advances in computer resources, there have been more elaborate attempts to correlate airflow characteristics in human nasal airways with the symptoms and functions of the nose by computational fluid dynamics study. Furthermore, the computed nasal geometry can be virtually modified to reflect predicted results of the proposed surgical technique. In this article, several computational fluid mechanics (CFD) issues on patient-specific three dimensional (3D) modeling of nasal cavity and clinical applications were reviewed in relation to the cases of deviated nasal septum (decision for surgery), turbinectomy, and maxillary sinus ventilation (simulated- and post-surgery). Clinical relevance of fluid mechanical parameters, such as nasal resistance, flow allocation, wall shear stress, heat/humidity/NO gas distributions, to the symptoms and surgical outcome were discussed. Absolute values of such parameters reported by many research groups were different each other due to individual difference of nasal anatomy, the methodology for 3D modeling and numerical grid, laminar/turbulent flow model in CFD code. But, the correlation of these parameters to symptoms and surgery outcome seems to be obvious in each research group with subject-specific models and its variations (virtual- and post-surgery models). For the more reliable, patient-specific, and objective tools for diagnosis and outcomes of nasal surgery by using CFD, the future challenges will be the standardizations on the methodology for creating 3D airway models and the CFD procedures.},
  Doi                      = {10.1016/j.jbiomech.2012.11.022},
  ISSN                     = {0021-9290},
  Keywords                 = {3D modeling
Biomedical flow
CFD
Nasal cavity
Physiology},
  Type                     = {Journal Article},
  Url                      = {http://ac.els-cdn.com/S0021929012006744/1-s2.0-S0021929012006744-main.pdf?_tid=f857851a-421f-11e4-a11a-00000aab0f01&acdnat=1411366760_c8f8c6f2d0a39388fdf5eaa6e555ce2c}
}

@Article{Kim1990,
  Title                    = {NEAR-WALL TURBULENCE MODEL AND ITS APPLICATION TO FULLY DEVELOPED TURBULENT CHANNEL AND PIPE FLOWS},
  Author                   = {Kim, S.-W.},
  Journal                  = {Numerical Heat Transfer, Part B: Fundamentals: An International Journal of Computation and Methodology},
  Year                     = {1990},
  Number                   = {1},
  Pages                    = {101 - 122},
  Volume                   = {17},

  ISSN                     = {1040-7790},
  Type                     = {Journal Article},
  Url                      = {http://www.informaworld.com/10.1080/10407799008961735}
}

@Book{Kimbell2001,
  Title                    = {Computational Fluid Dynamics of the Extrathoracic Airways, Medical Applications of Computer Modelling: The Respiratory System},
  Author                   = {Kimbell, J.S.},
  Publisher                = {WIT Press },
  Year                     = {2001},

  Address                  = {UK},

  Type                     = {Book}
}

@Article{Kimbell2007,
  Title                    = {Characterization of deposition from nasal spray devices using a computational fluid dynamics model of the human nasal passages},
  Author                   = {Kimbell, J.S. and Segal, R.A. and Asgharian, B. and Wong, B.A. and Schroeter, J.D. and Southall, J.P. and Dickens, C.J. and Brace, G. and Miller, F.J.},
  Journal                  = {Journal Aeros. Medicine},
  Year                     = {2007},
  Number                   = {1},
  Pages                    = {59-74},
  Volume                   = {20},

  Type                     = {Journal Article}
}

@Article{Kimbell2004,
  Title                    = {Optimisation of nasal delivery devices using computational models.},
  Author                   = {Kimbell, J. and Shroeter, J.D. and Asgharian, B. and Wong, B.A. and Segal, R.A. and Dickens, C.J. and Southall, J.P. and Miller, F.J.},
  Journal                  = {Respiratory Drug Delivery 9},
  Year                     = {2004},
  Number                   = {233-238},
  Volume                   = {1},

  Type                     = {Journal Article}
}

@Article{Kimbell2006,
  Title                    = {Nasal Dosimetry of Inhaled Gases and Particles: Where Do Inhaled Agents Go in the Nose?},
  Author                   = {Kimbell, J. S.},
  Journal                  = {Toxicol Pathol},
  Year                     = {2006},
  Number                   = {3},
  Pages                    = {270-273},
  Volume                   = {34},

  Abstract                 = {The anatomical structure of the nasal passages differs significantly among species, affecting airflow and the transport of inhaled gases and particles throughout the respiratory tract. Since direct measurement of local nasal dose is often difficult, 3-dimensional, anatomically accurate, computational models of the rat, monkey, and human nasal passages were developed to estimate regional transport and dosimetry of inhaled material. The computational models predicted that during resting breathing, a larger portion of inspired air passed through olfactory-lined regions in the rat than in the monkey or human. The models also predicted that maximum wall mass flux (mass per surface area per time) of inhaled formaldehyde in the nonsquamous epithelium was highest in monkeys (anterior middle turbinate) and similar in rats and humans (dorsal medial meatus in the rat and mid-septum in the human, near the squamous/nonsquamous epithelial boundary in both species). For particles that are 5 {micro}m in aerodynamic diameter, preliminary simulations at minute volume flow rates predicted nasal deposition efficiencies of 92%, 11% and 25% in the rat, monkey, and human, respectively, with more vestibular deposition in the rat than in the monkey or human. Estimates such as these can be used to test hypotheses about mechanisms of toxicity and supply species-specific information for risk assessment, thus reducing uncertainty in extrapolating animal data to humans.},
  Doi                      = {10.1080/01926230600695607},
  Type                     = {Journal Article},
  Url                      = {http://tpx.sagepub.com/cgi/content/abstract/34/3/270}
}

@Article{Kimbell2006a,
  Title                    = {Life beyond the air phase: adding tissue disposition to models of nasal airway transport},
  Author                   = {Kimbell, J. S. and Schroeter, J. D. and Dorman, D. C. and Andersen, M. E.},
  Journal                  = {Journal of Biomechanics},
  Year                     = {2006},
  Number                   = {Supplement 1},
  Pages                    = {S270-S271},
  Volume                   = {39},

  ISSN                     = {0021-9290},
  Type                     = {Journal Article},
  Url                      = {http://www.sciencedirect.com/science/article/B6T82-4KR88PB-1FY/2/074522a8add8bd34f7162530d48e829e}
}

@Article{KingSe2010,
  Title                    = {Inhalability of micron particles through the nose and mouth},
  Author                   = {King Se, Camby Mei and Inthavong, Kiao and Tu, Jiyuan},
  Journal                  = {Inhalation Toxicology},
  Year                     = {2010},
  Number                   = {4},
  Pages                    = {287-300},
  Volume                   = {22},

  Doi                      = {10.3109/08958370903295204},
  ISSN                     = {0895-8378},
  Type                     = {Journal Article},
  Url                      = {http://dx.doi.org/10.3109/08958370903295204}
}

@Article{KingSe2010a,
  Title                    = {Inhalability of micron particles through the nose and mouth},
  Author                   = {King Se, Camby Mei and Inthavong, Kiao and Tu, Jiyuan},
  Journal                  = {Inhalation Toxicology},
  Year                     = {2010},
  Number                   = {4},
  Pages                    = {287-300},
  Volume                   = {22},

  Doi                      = {doi:10.3109/08958370903295204},
  Type                     = {Journal Article},
  Url                      = {http://informahealthcare.com/doi/abs/10.3109/08958370903295204}
}

@Article{KingSe2010b,
  Title                    = {Inhalability of micron particles through the nose and mouth},
  Author                   = {King Se, Camby Mei and Inthavong, Kiao and Tu, Jiyuan},
  Journal                  = {Inhalation Toxicology},
  Year                     = {2010},
  Pages                    = {287-300},
  Volume                   = {22},

  Doi                      = {10.3109/08958370903295204},
  ISSN                     = {0895-8378},
  Type                     = {Journal Article},
  Url                      = {http://informahealthcare.com/doi/pdfplus/10.3109/08958370903295204}
}

@Article{Kino2007,
  Title                    = {Optimal breathing protocol for dynamic contrast-enhanced MRI of solitary pulmonary nodules at 3T},
  Author                   = {Kino, A. and Takahashi, M. and Ashiku, S. K. and Decamp, M. M. and Lenkinski, R. E. and Hatabu, H.},
  Journal                  = {European Journal Radiology},
  Year                     = {2007},
  Pages                    = {397-400},
  Volume                   = {64},

  Type                     = {Journal Article}
}

@Article{Kippax2000,
  Title                    = {Applications for droplet sizing - manual versus automated actuation of nasal sprays},
  Author                   = {Kippax, P.G. and Krarup, H. and Suman, J.D.},
  Journal                  = {Pharmaceutical Technology},
  Year                     = {2000},
  Pages                    = {30-39},
  Volume                   = {Jan},

  Type                     = {Journal Article}
}

@Article{Kiraly2002,
  Title                    = {Three-Dimensional Human Airway Segmentation Methods for Clinical Virtual Bronchoscopy},
  Author                   = {Kiraly, Atilla P. and Higgins, William E. and McLennan, Geoffrey and Hoffman, Eric A. and Reinhardt, Joseph M.},
  Journal                  = {Academic Radiology},
  Year                     = {2002},
  Number                   = {10},
  Pages                    = {1153-1168},
  Volume                   = {9},

  Keywords                 = {Bronchi, CT
bronchoscopy
computed tomography (CT), image processing
computed tomography (CT), three-dimensional
trachea, CT},
  Type                     = {Journal Article},
  Url                      = {http://www.sciencedirect.com/science/article/B75BK-4B1WYM0-2H/2/70f0c99ca60d8e06d18a059a1ac286ae }
}

@Article{Kirkpatrick2005,
  Title                    = {Experimental and large eddy simulation results for the purging of salt water from a cavity by an overflow of fresh water},
  Author                   = {Kirkpatrick, M. P. and Armfield, S. W.},
  Journal                  = {International Journal of Heat and Mass Transfer},
  Year                     = {2005},
  Number                   = {2},
  Pages                    = {341-359},
  Volume                   = {48},

  Abstract                 = {This paper presents the results of an experimental and numerical investigation of a flow in which salt water is purged from a square cavity by an overflow of fresh water. Two numerical simulations are presented, one two-dimensional simulation and one three-dimensional large eddy simulation. The results are used to describe the important transport mechanisms that occur during the purging process. In particular, we propose a mechanism for the formation of the streamers observed in the experiment. We also discuss the performance of the numerical models for flows of this type.},
  ISSN                     = {0017-9310},
  Keywords                 = {Purging
Stratified flow
Large eddy simulation
Density interface, Turbulent entrainment},
  Type                     = {Journal Article},
  Url                      = {http://www.sciencedirect.com/science/article/B6V3H-4DR868J-1/2/df400aa439a72becdca4587ebaa08340}
}

@Article{Kleinstreuer2010a,
  Title                    = {Airflow and Particle Transport in the Human Respiratory System},
  Author                   = {Kleinstreuer, C. and Zhang, Z.},
  Journal                  = {annual review of fluid mechanics},
  Year                     = {2010},
  Pages                    = {301-334},
  Volume                   = {42},

  Abstract                 = {Airflows in the nasal cavities and oral airways are rather complex, possibly featuring a transition to turbulent jet-like flow, recirculating flow, Deans flow, vortical flows, large pressure drops, prevailing secondary flows, and merging streams in the case of exhalation. Such complex flows propagate subsequendy into the tracheobronchial airways. The underlying assumptions for particle transport and deposition are that the aerosols are spherical, noninteracting, and monodisperse and deposit upon contact with the airway surface. Such dilute particle suspensions are typically modeled with the Euler-Lagrange approach for micron particles and in the Euler-Euler framework for nanoparticles. Micron particles deposit nonuniformly with very high concentrations at some local sites (e.g., carinal ridges of large bronchial airways). In contrast, nanomaterial almost coats the airway surfaces, which has implications of detrimental health effects in the case of inhaled toxic nanoparticles. Geometric airway features, as well as histories of airflow fields and particle distributions, may significantly affect particle deposition.},
  Doi                      = {10.1146/annurev-fluid-121108-145453},
  ISSN                     = {0066-4189},
  Keywords                 = {dilute suspensions of nanomaterial
experimental observations and computer simulations
human respiratory air-way models
inhaled/exhaled aerosol transport and deposition
laminar/turbulent airflow structures
solid particles or droplets},
  Type                     = {Journal Article},
  Url                      = {http://www.annualreviews.org/doi/pdf/10.1146/annurev-fluid-121108-145453}
}

@Article{Kleinstreuer2003,
  Title                    = {Laminar-to-turbulent fluid-particle flows in a human airway model},
  Author                   = {Kleinstreuer, C. and Zhang, Z.},
  Journal                  = {International Journal of Multiphase Flow},
  Year                     = {2003},
  Pages                    = {271-289},
  Volume                   = {29},

  Type                     = {Journal Article}
}

@Article{Kleinstreuer2003a,
  Title                    = {Targeted drug aerosol deposition analysis for a four-generation lung airway model with hemispherical tumours},
  Author                   = {Kleinstreuer, C. and Zhang, Z.},
  Journal                  = {Journal of Biomechanical Engineering},
  Year                     = {2003},
  Pages                    = {197-206},
  Volume                   = {125},

  Type                     = {Journal Article}
}

@Article{Kleinstreuer2007,
  Title                    = {Combined inertial and gravitational deposition of microparticles in small model airways of a human respiratory system},
  Author                   = {Kleinstreuer, Clement and Zhang, Zhe and Kim, Chong S.},
  Journal                  = {Journal of Aerosol Science},
  Year                     = {2007},
  Note                     = {doi: DOI: 10.1016/j.jaerosci.2007.08.010},
  Number                   = {10},
  Pages                    = {1047-1061},
  Volume                   = {38},

  ISSN                     = {0021-8502},
  Keywords                 = {Inertial impaction
Gravitational sedimentation
Micron particle deposition
Small human airways
Computational analysis
Deposition efficiency
Stokes number
Froude number
Sedimentation parameter
Deposition correlations},
  Type                     = {Journal Article},
  Url                      = {http://www.sciencedirect.com/science/article/B6V6B-4PN05WR-1/2/9beaf7259293016ffd32e42e1d6b4768}
}

@Article{Kleinstreuer2008,
  Title                    = {Modeling airflow and particle transport/deposition in pulmonary airways},
  Author                   = {Kleinstreuer, Clement and Zhang, Zhe and Li, Zheng},
  Journal                  = {Respiratory Physiology \& Neurobiology},
  Year                     = {2008},
  Note                     = {doi: DOI: 10.1016/j.resp.2008.07.002},
  Number                   = {1-3},
  Pages                    = {128-138},
  Volume                   = {163},

  ISSN                     = {1569-9048},
  Keywords                 = {Pulmonary airways
Deposition enhancement factor
Tracheobronchial airways},
  Type                     = {Journal Article},
  Url                      = {http://www.sciencedirect.com/science/article/B6X16-4SYTC99-1/2/f6c032db407915685fa9132b14e7013f}
}

@Article{Kleven2006,
  Title                    = {Computational modelling of nasal aerodynamics},
  Author                   = {Kleven, M. and Melaaen, M. C. and Reimers, M. and Djupesland, P. G.},
  Journal                  = {Journal of Biomechanics},
  Year                     = {2006},
  Number                   = {Supplement 1},
  Pages                    = {S271-S271},
  Volume                   = {39},

  ISSN                     = {0021-9290},
  Type                     = {Journal Article},
  Url                      = {http://www.sciencedirect.com/science/article/B6T82-4KR88PB-1G1/2/3e6a1d826f8fc98e3bc8248c6c4c9e83}
}

@Article{Kleven2005,
  Title                    = {Using Computational Fluid Dynamics (CFD) to Improve the Bi-Directional Nasal Drug Delivery Concept},
  Author                   = {Kleven, M. and Melaaen, M. C. and Reimers, M. and Røtnes, J. S. and Aurdal, L. and Djupesland, P. G.},
  Journal                  = {Food and Bioproducts Processing},
  Year                     = {2005},
  Number                   = {2},
  Pages                    = {107-117},
  Volume                   = {83},

  Abstract                 = {Nasal delivery is considered for an increasing number of existing and new drugs and vaccines, but current nasal delivery devices have major disadvantages. The Norwegian company OptiNose AS is developing a novel concept that challenges traditional delivery systems. The patended bi-directional delivery system improves drug and vaccine distribution to the nasal mucous membrane while at the same time preventing lung deposition. It takes advantage of the posterior connection between the nasal passages persisting when the velum automatically closes during oral exhalation. It is the exhalation into the delivery device that triggers the release of particles into an airflow, which enters one nostril via a sealing nozzle and exits through the other nostril.This paper describes how OptiNose is using Computational Fluid Dynamics (CFD) during the development process for their drug delivery concept. The simulations are used to visualize and demonstrate the basic features of the bi-directional technique and discuss how its design and function could be further optimized. CFD computations thus increase the efficiency of device development and reduce the need for expensive and time consuming laboratory experiments. To perform successful CFD calculations on the nose, construction of a proper surface grid of the nasal cavity is important. The process of building the surface grid is presented in the paper. The final surface grid was next imported into Tgrid, a volume grid generator, and finally the simulations were carried out by use of the commercial CFD code FLUENT. These steps are described in the paper. Testing of the cell quality, both during surface grid and volume grid generation, is mandatory. The testing procedures are briefly presented in the paper. Finally, to be able to rely on the CFD computations done, one needs thorough validation. This article presents some comparison of the CFD computation results against physical experiments. The comparison analysis shows promising results.},
  ISSN                     = {0960-3085},
  Keywords                 = {computational fluid dynamics (CFD)
nasal cavity
nasal drug delivery},
  Type                     = {Journal Article},
  Url                      = {http://www.sciencedirect.com/science/article/B8JGD-4RV2DF2-5/2/b78143ef121b7605541a8a0650bff6b8}
}

@Article{Knaapen2004,
  Title                    = {Inhaled particles and lung cancer. Part A: Mechanisms},
  Author                   = {Knaapen, Ad M. and Borm, Paul J.A. and Albrecht, Catrin and Schins, Roel P.F.},
  Journal                  = {International Journal of Cancer},
  Year                     = {2004},
  Note                     = {10.1002/ijc.11708},
  Number                   = {6},
  Pages                    = {799-809},
  Volume                   = {109},

  ISSN                     = {1097-0215},
  Type                     = {Journal Article},
  Url                      = {http://dx.doi.org/10.1002/ijc.11708}
}

@Article{Kohl2008,
  Title                    = {Untitled},
  Author                   = {Kohl, Peter and Coveney, Peter and Clapworthy, Gordon and Viceconti, Marco},
  Journal                  = {PHILOSOPHICAL TRANSACTIONS OF THE ROYAL SOCIETY A-MATHEMATICAL PHYSICAL AND ENGINEERING SCIENCES},
  Year                     = {2008},
  Pages                    = {3223-3224},
  Volume                   = {366},

  Doi                      = {10.1098/rsta.2008.0102},
  ISSN                     = {1364-503X},
  Type                     = {Journal Article},
  Url                      = {http://rsta.royalsocietypublishing.org/content/366/1879/3223.full.pdf}
}

@Article{Kongerud2006,
  Title                    = {Nasal Responses in Asthmatic and Nonasthmatic Subjects Following Exposure to Diesel Exhaust Particles},
  Author                   = {Kongerud, Johny and Madden, Michael C. and Hazucha, Milan and Peden, David},
  Journal                  = {Inhalation Toxicology},
  Year                     = {2006},
  Number                   = {9},
  Pages                    = {589-594},
  Volume                   = {18},

  ISSN                     = {0895-8378},
  Type                     = {Journal Article},
  Url                      = {http://www.informaworld.com/10.1080/08958370600743027}
}

@InBook{Kornhaas2008,
  Title                    = {Influence of Time Step Size and Convergence Criteria on Large Eddy Simulations with Implicit Time Discretization},
  Author                   = {Kornhaas, M. and Sternel, D.C. and Schäfer, M.},
  Editor                   = {Meyers, J. and Geurts, B.J. and Sagaut, P.},
  Pages                    = {331-342},
  Publisher                = {Springer},
  Year                     = {2008},

  Address                  = {Netherlands},
  Type                     = {Book Section},
  Volume                   = {12},

  Booktitle                = {Quality and Reliability of Large-Eddy Simulations}
}

@Article{Kovacs2004,
  Title                    = {Mechanisms of olfactory dysfunction in aging and neurodegenerative disorders },
  Author                   = {Tibor Kovács},
  Journal                  = {Ageing Research Reviews },
  Year                     = {2004},
  Number                   = {2},
  Pages                    = {215 - 232},
  Volume                   = {3},

  Abstract                 = {Although olfaction is the primal sense in animals, its importance in humans is underappreciated. Extensive literature demonstrates that aging is accompanied by olfactory loss and hyposmia/anosmia which is also a feature of several neurodegenerative disorders. Alzheimer’s and Parkinson’s diseases are characterized by severe olfactory deficits, while problems of olfactory discrimination are less prominent features in several other disorders. Olfactory loss is accompanied by structural abnormalities of the olfactory epithelium, the olfactory bulb and the central olfactory cortices. This review summarizes our present knowledge about the pathological changes in the olfactory system during aging and in various neurodegenerative diseases. },
  Doi                      = {http://dx.doi.org/10.1016/j.arr.2003.10.003},
  ISSN                     = {1568-1637},
  Keywords                 = {Olfaction},
  Url                      = {http://www.sciencedirect.com/science/article/pii/S1568163703000631}
}

@Article{KrA¼ger2011,
  Title                    = {Efficient and accurate simulations of deformable particles immersed in a fluid using a combined immersed boundary lattice Boltzmann finite element method},
  Author                   = {Krüger, T. and Varnik, F. and Raabe, D.},
  Journal                  = {Computers \& Mathematics with Applications},
  Year                     = {2011},
  Number                   = {12},
  Pages                    = {3485-3505},
  Volume                   = {61},

  Abstract                 = {The deformation of an initially spherical capsule, freely suspended in simple shear flow, can be computed analytically in the limit of small deformations [D. Barthés-Biesel, J.M. Rallison, The time-dependent deformation of a capsule freely suspended in a linear shear flow, J. Fluid Mech. 113 (1981) 251-267]. Those analytic approximations are used to study the influence of the mesh tessellation method, the spatial resolution, and the discrete delta function of the immersed boundary method on the numerical results obtained by a coupled immersed boundary lattice Boltzmann finite element method. For the description of the capsule membrane, a finite element method and the Skalak constitutive model [R. Skalak, A. Tozeren, R.P. Zarda, S. Chien, Strain energy function of red blood cell membranes, Biophys. J. 13 (1973) 245-264] have been employed. Our primary goal is the investigation of the presented model for small resolutions to provide a sound basis for efficient but accurate simulations of multiple deformable particles immersed in a fluid. We come to the conclusion that details of the membrane mesh, as tessellation method and resolution, play only a minor role. The hydrodynamic resolution, i.e., the width of the discrete delta function, can significantly influence the accuracy of the simulations. The discretization of the delta function introduces an artificial length scale, which effectively changes the radius and the deformability of the capsule. We discuss possibilities of reducing the computing time of simulations of deformable objects immersed in a fluid while maintaining high accuracy.},
  Doi                      = {10.1016/j.camwa.2010.03.057},
  ISSN                     = {0898-1221},
  Keywords                 = {Lattice Boltzmann method
Immersed boundary method
Finite element method
Capsule
Simple shear flow
Small deformations},
  Type                     = {Journal Article},
  Url                      = {http://www.sciencedirect.com/science/article/pii/S0898122110002476}
}

@Article{Krane2006,
  Title                    = {Glottal jet structure measured in a scaled-up model},
  Author                   = {Krane, M. and Barry, M. and Wei, T.},
  Journal                  = {Journal of Biomechanics},
  Year                     = {2006},
  Number                   = {Supplement 1},
  Pages                    = {S442-S442},
  Volume                   = {39},

  ISSN                     = {0021-9290},
  Type                     = {Journal Article},
  Url                      = {http://www.sciencedirect.com/science/article/B6T82-4KR88PB-2F9/2/5f42d0018c37aae6de962f60e53cd84c}
}

@Article{Kreplin1978,
  Title                    = {Behaviour of the three fluctuating velocity components in the wall region of a turbulent channel flow.},
  Author                   = {Kreplin, H-P. and Eckelmann, H.},
  Journal                  = {Phys. Fluids},
  Year                     = {1978},
  Number                   = {7},
  Pages                    = {1233-1239},
  Volume                   = {22},

  Type                     = {Journal Article}
}

@Article{Kreutzer2005,
  Title                    = {Multiphase monolith reactors: Chemical reaction engineering of segmented flow in microchannels},
  Author                   = {Kreutzer, Michiel T. and Kapteijn, Freek and Moulijn, Jacob A. and Heiszwolf, Johan J.},
  Journal                  = {Chemical Engineering Science},
  Year                     = {2005},
  Number                   = {22},
  Pages                    = {5895-5916},
  Volume                   = {60},

  Abstract                 = {The use of segmented flow in capillaries, also known as Taylor flow, for reaction engineering purposes has soared in recent years. On the one hand, Taylor flow has been used in honeycomb monolith catalyst supports. On the other hand, Taylor flow is the common flow pattern in multiphase microchannel reactors. This contribution reviews the fluid mechanical aspects of this flow pattern in quite general terms, with an emphasis on the underlying principles. From very simple analysis, design estimates for mass transfer, pressure drop and residence time distribution may be obtained with relative ease and--for multiphase reactors--surprising accuracy.},
  ISSN                     = {0009-2509},
  Keywords                 = {Monoliths
Honeycombs
Taylor flow
Segmented flow
Microreactors
Bretherton's problem},
  Type                     = {Journal Article},
  Url                      = {http://www.sciencedirect.com/science/article/B6TFK-4G5BJS0-5/2/7e35e8f8bdae920abf8dd4fa6dd99064}
}

@Article{Krieger1992,
  Title                    = {Long-term compliance with nasal continuous positive airway pressure (CPAP) in obstructive sleep apnea patients and nonapneic snorers},
  Author                   = {Krieger, J.},
  Journal                  = {Sleep},
  Year                     = {1992},
  Number                   = {6 Suppl},
  Pages                    = {S42-6},
  Volume                   = {15},

  Type                     = {Journal Article}
}

@Article{Krushkal1984,
  Title                    = {On the orientation distribution function of nonspherical aerosol particles in a general shear flow : I. The laminar case},
  Author                   = {Krushkal, E. M. and Gallily, Isaiah},
  Journal                  = {Journal of Colloid And Interface Science},
  Year                     = {1984},
  Number                   = {1},
  Pages                    = {141-152},
  Volume                   = {99},

  ISSN                     = {0021-9797},
  Type                     = {Journal Article},
  Url                      = {http://www.sciencedirect.com/science/article/pii/0021979784900948}
}

@Article{Ku1997,
  Title                    = {Blood Flow in Arteries. },
  Author                   = {Ku, D.N. },
  Journal                  = {Annual Review of Fluid Mechanics},
  Year                     = {1997},
  Pages                    = {399-434},
  Volume                   = {29},

  Type                     = {Journal Article}
}

@Article{Ku1985,
  Title                    = {Hemodynamics of the Normal Human Carotid Bifurcation: in Vitro and in Vivo Studies},
  Author                   = {Ku, D.N. and Giddens, D.P. and Phillips, D.J. and Strandress Jr., D.E. },
  Journal                  = {Ultrasound Medical Biology},
  Year                     = {1985},
  Pages                    = {13-26},
  Volume                   = {11},

  Type                     = {Journal Article}
}

@Article{Ku1985a,
  Title                    = {Pulsatile Flow and Atherosclerosis in the Human Carotid Bifurcation. Positive Correlation between Plaque Location and Low Oscillating Shear Stress},
  Author                   = {Ku, D.N. and Giddens, D.P. and Zarins, C.K. and Glagov, S. },
  Journal                  = {Arteriosclerosis, Thrombosis, and Vascular Biology},
  Year                     = {1985},
  Pages                    = {293-302},
  Volume                   = {5},

  Type                     = {Journal Article}
}

@Article{Kublik1998,
  Title                    = {Nasal delivery systems and their effect on deposition and absorption},
  Author                   = {Kublik, H. and Vidgren, M.T.},
  Journal                  = {Adv. Drug Delivery Rev. },
  Year                     = {1998},
  Pages                    = {157-177},
  Volume                   = {29},

  Type                     = {Journal Article}
}

@Article{Kubo2001,
  Title                    = {Targeted systemic chemotherapy using magnetic liposomes with incorporated adriamycin for osteosarcoma in hamsters},
  Author                   = {Kubo, T and Sugita, T and Shimose, S and Nitta, Y and Ikuta, Y and Murakami, T},
  Journal                  = {Int J Oncol},
  Year                     = {2001},
  Number                   = {1},
  Pages                    = {121 - 125},
  Volume                   = {18},

  Type                     = {Journal Article}
}

@Article{Kubo2000,
  Title                    = {Targeted delivery of anticancer drugs with intravenously administered magnetic liposomes in osteosarcoma-bearing hamsters},
  Author                   = {Kubo, T and Sugita, T and Shimose, S and Nitta, Y and Ikuta, Y and Murakami, T},
  Journal                  = {Int J Oncol},
  Year                     = {2000},
  Number                   = {2},
  Pages                    = {309 - 15},
  Volume                   = {17},

  Type                     = {Journal Article}
}

@Article{Kuipers2000,
  Title                    = {Multilevel Modelling of Dispersed Multiphase Flows},
  Author                   = {Kuipers, J.A.M.},
  Journal                  = {Oil \& Gas Science and Technology – Rev. IFP},
  Year                     = {2000},
  Number                   = {4},
  Pages                    = {427-435},
  Volume                   = {55},

  Type                     = {Journal Article}
}

@Article{Kumahata2010,
  Title                    = {Nasal Flow Simulation Using Heat and Humidity Models},
  Author                   = {Kumahata, Kiyoshi and Mori, Futoshi and Ishikawa, Shigeru and Matsuzawa, Teruo},
  Journal                  = {Journal of Biomechanical Science and Engineering},
  Year                     = {2010},
  Number                   = {5},
  Pages                    = {565-577},
  Volume                   = {5},

  Type                     = {Journal Article}
}

@Article{Kumar2009,
  Title                    = {The effects of geometry on airflow in the acinar region of the human lung},
  Author                   = {Kumar, Haribalan and Tawhai, Merryn H. and Hoffman, Eric A. and Lin, Ching-Long},
  Journal                  = {Journal of Biomechanics},
  Year                     = {2009},
  Number                   = {11},
  Pages                    = {1635-1642},
  Volume                   = {42},

  ISSN                     = {0021-9290},
  Keywords                 = {Acinar fluid dynamics
Low Reynolds number flow
Finite Element
Open cavity flow
Recirculation},
  Type                     = {Journal Article},
  Url                      = {http://www.sciencedirect.com/science/article/B6T82-4WDNBWN-3/2/5d8f2078f36ed7daf7a1a86e4a1f24d4}
}

@Article{Kundoor2011,
  Title                    = {Effect of formulation- and administration-related variables on deposition pattern of nasal spray pumps evaluated using a nasal cast},
  Author                   = {Kundoor, V. and Dalby, R. N.},
  Journal                  = {Pharm Res},
  Year                     = {2011},
  Note                     = {1573-904x
Kundoor, Vipra
Dalby, Richard N
Journal Article
United States
Pharm Res. 2011 Aug;28(8):1895-904. doi: 10.1007/s11095-011-0417-6. Epub 2011 Apr 16.},
  Number                   = {8},
  Pages                    = {1895-904},
  Volume                   = {28},

  Abstract                 = {PURPOSE: To systematically evaluate the effect of formulation- and administration-related variables on nasal spray deposition using a nasal cast. METHODS: Deposition pattern was assessed by uniformly coating a transparent nose model with Sar-Gel(R), which changes from white to purple on contact with water. Sprays were subsequently discharged into the cast, which was then digitally photographed. Images were quantified using Adobe(R) Photoshop. The effects of formulation viscosity (which influences droplet size), simulated administration techniques (head orientation, spray administration angle, spray nozzle insertion depth), spray pump design and metering volume on nasal deposition pattern were investigated. RESULTS: There was a significant decrease in the deposition area associated with sprays of increasing viscosity. This appeared to be mediated by an increase in droplet size and a narrowing of the spray plume. Administration techniques and nasal spray pump design also had a significant effect on the deposition pattern. CONCLUSIONS: This simple color-based method provides quantitative estimates of the effects that different formulation and administration variables may have on the nasal deposition area, and provides a rational basis on which manufacturers of nasal sprays can base their patient instructions or post approval changes when it is impractical to optimize these using a clinical study.},
  Doi                      = {10.1007/s11095-011-0417-6},
  ISSN                     = {0724-8741},
  Keywords                 = {Administration, Intranasal
Chemistry, Pharmaceutical
Humans
*Models, Anatomic
*Nasal Sprays
*Nebulizers and Vaporizers
Particle Size
Tissue Distribution
Viscosity},
  Type                     = {Journal Article}
}

@Article{Kundoor2010,
  Title                    = {Assessment of nasal spray deposition pattern in a silicone human nose model using a color-based method},
  Author                   = {Kundoor, Vipra and Dalby, Richard N},
  Journal                  = {Pharmaceutical research},
  Year                     = {2010},
  Number                   = {1},
  Pages                    = {30-36},
  Volume                   = {27},

  ISSN                     = {0724-8741},
  Type                     = {Journal Article}
}

@Article{Kundu1990,
  Title                    = {Local segmentation of biomedical images},
  Author                   = {Kundu, A.},
  Journal                  = {Computerized Medical Imaging and Graphics},
  Year                     = {1990},
  Pages                    = {173-183},
  Volume                   = {14},

  Type                     = {Journal Article}
}

@Article{Kundu1991,
  Title                    = {Affinity labeling of endothelin receptors in bovine and rat lung membranes by N[epsilon]9-azidobenzoyl-125I-endothelin-1},
  Author                   = {Kundu, Gopal C. and Misono, Kunio S.},
  Journal                  = {Molecular and Cellular Endocrinology},
  Year                     = {1991},
  Number                   = {1-3},
  Pages                    = {85-92},
  Volume                   = {79},

  ISSN                     = {0303-7207},
  Keywords                 = {Endothelin
Receptor
Photoaffinity labeling
Lung membranes, bovine, rat},
  Type                     = {Journal Article},
  Url                      = {http://www.sciencedirect.com/science/article/B6T3G-47N6MCG-GT/2/1cfa4801711625f4959cfd8db79d7261}
}

@Article{Kuo-HsiCheng1996,
  Title                    = {IN VIVO MEASUREMENTS OF NASAL AIRWAY DIMENSIONS AND ULTRAFINE AEROSOL DEPOSITION IN THE HUMAN NASAL AND ORAL AIRWAYS},
  Author                   = {Kuo-Hsi Cheng, * Yung-Sung Cheng, * Hsu-Chi Yeh, ** Raymond A. Guilmette,t Steven Q. Simpson, ~ Yi-Hsin Yang ~ and David L. Swift*},
  Journal                  = {J. Aerosol Sci},
  Year                     = {1996},

  Type                     = {Journal Article}
}

@Article{Kuwana2005,
  Title                    = {Modeling CVD synthesis of carbon nanotubes: Nanoparticle formation from ferrocene},
  Author                   = {Kuwana, Kazunori and Saito, Kozo},
  Journal                  = {Carbon},
  Year                     = {2005},
  Number                   = {10},
  Pages                    = {2088-2095},
  Volume                   = {43},

  Abstract                 = {Catalyst nanoparticles play an important role in the synthesis of carbon nanotubes. In this paper, we present a two-equation model that can predict the formation process of iron nanoparticles from ferrocene fed into a CVD reactor. The model, combined with an axisymmetric two-dimensional computational fluid dynamics (CFD) simulation, includes the mechanism of nucleation and surface growth of an iron particle and bi-particle collision. The model predicts that the diameter of a particle will increase with an increase in the reaction temperature or the radial distance from the center of the reactor. Iron particles may deposit on the reactor wall; our model predicts that the thickness of the layer consisting of deposited iron particles will decrease with an increase in the axial distance from the entrance. The first prediction was validated by experimental observations reported by other researchers. In addition to the CFD simulation, a dimensional analysis was conducted to find pi-numbers that govern the process of particle formation; three pi-numbers were identified. Furthermore, one-dimensional governing equations were obtained under the assumptions of constant diffusion coefficient and collision frequency function, and solutions for particle diameter were obtained in qualitative agreement with the earlier CFD simulations.},
  ISSN                     = {0008-6223},
  Keywords                 = {Carbon nanotubes
Chemical vapor deposition
Modeling
Particle size},
  Type                     = {Journal Article},
  Url                      = {http://www.sciencedirect.com/science/article/B6TWD-4G3CX0M-2/2/64fd8ddf355685e728baf3a3b3c218f3}
}

@Article{Kuwano1993,
  Title                    = {Small airways dimensions in asthma and in chronic obstructive pulmonary disease},
  Author                   = {Kuwano, K. and Bosken, C.H. and Pare, P.D. and Bai, T.R. and Wiggs, B.R. and Hogg, J.C.},
  Journal                  = {Am Rev Respir Dis},
  Year                     = {1993},
  Pages                    = {1220-1225},
  Volume                   = {148},

  Type                     = {Journal Article}
}

@Article{Kvasnak1996,
  Title                    = {Deposition of ellipsoidal particles in turbulent duct flows},
  Author                   = {Kvasnak, William and Ahmadi, Goodarz},
  Journal                  = {Chemical Engineering Science},
  Year                     = {1996},
  Number                   = {23},
  Pages                    = {5137-5148},
  Volume                   = {51},

  ISSN                     = {0009-2509},
  Keywords                 = {Fiber deposition
ellipsoidal particles
turbulent deposition
elongated particles
particle transport},
  Type                     = {Journal Article},
  Url                      = {http://www.sciencedirect.com/science/article/pii/S0009250996003570}
}

@Article{Kvasnak1995,
  Title                    = {Fibrous particle deposition in a turbulent channel flow - An experimental study},
  Author                   = {Kvasnak, W. and Ahmadi, G.},
  Journal                  = {Aerosol Science and Technology},
  Year                     = {1995},
  Number                   = {641-652},
  Volume                   = {23},

  Type                     = {Journal Article}
}

@Article{Kvasnak1995a,
  Title                    = {Fibrous Particle Deposition in a Turbulent Channel Flow—An Experimental Study},
  Author                   = {Kvasnak, William and Ahmadi, Goodarz},
  Journal                  = {Aerosol Science and Technology},
  Year                     = {1995},
  Number                   = {4},
  Pages                    = {641 - 652},
  Volume                   = {23},

  ISSN                     = {0278-6826},
  Type                     = {Journal Article},
  Url                      = {http://www.informaworld.com/10.1080/02786829508965344}
}

@Article{Kvasnak1993,
  Title                    = {Experimental investigation of dust particle deposition in a turbulent channel flow},
  Author                   = {Kvasnak, William and Ahmadi, Goodarz and Bayer, Raymond and Gaynes, Michael},
  Journal                  = {Journal of Aerosol Science},
  Year                     = {1993},
  Number                   = {6},
  Pages                    = {795-815},
  Volume                   = {24},

  ISSN                     = {0021-8502},
  Type                     = {Journal Article},
  Url                      = {http://www.sciencedirect.com/science/article/pii/002185029390047D}
}

@Article{LabovskA½,
  Title                    = {Cfd Simulations of Ammonia Dispersion Using "Dynamic" Boundary Conditions},
  Author                   = {Labovský, J. and Jelemenský, L.},
  Journal                  = {Process Safety and Environmental Protection},
  Volume                   = {In Press, Accepted Manuscript},

  Abstract                 = {Ammonia is stored in liquid form at ambient temperature and under high pressure. During an accident, ammonia will flash out of the vessel and disperse in the surrounding area. This paper provides a comparison of the results obtained by the FLADIS field experiments and those of CFD modeling by Fluent 6.3. FLADIS experiments were carried out by the Risø National Laboratory using pressure liquefied ammonia. Time series of meteorological conditions as wind speed, wind direction and source strength were determined from the experimentally measured data and used as the inflow boundary conditions. Furthermore, for more realistic simulation of air flow in the computation domain for the desired atmospheric stability, periodic boundary conditions were used on both side boundaries. The initial two-phase flow of the released ammonia was also included. The liquid phase was modeled as droplets using discrete particle modeling, i.e. the Euler-Lagrangian approach for continuous and discrete phases.},
  ISSN                     = {0957-5820},
  Keywords                 = {CFD modeling
gas dispersion
ammonia release
emergency preparedness},
  Type                     = {Journal Article},
  Url                      = {http://www.sciencedirect.com/science/article/B8JGG-4YJCM0P-1/2/4d16df5cfc657c35bcdbb31176dc802d}
}

@Article{Lai2008,
  Title                    = {Experimental and numerical study on particle distribution in a two-zone chamber},
  Author                   = {Lai, Alvin C. K. and Wang, K. and Chen, F. Z.},
  Journal                  = {Atmospheric Environment},
  Year                     = {2008},
  Number                   = {8},
  Pages                    = {1717-1726},
  Volume                   = {42},

  Abstract                 = {Better understanding of aerosol dynamics is an important step for improving personal exposure assessments in indoor environments. Although the limitation of the assumptions in a well-mixed model is well known, there has been very little research reported in the published literature on the discrepancy of exposure assessments between numerical models which take account of gravitational effects and the well-mixed model. A new Eulerian-type drift-flux model has been developed to simulate particle dispersion and personal exposure in a two-zone geometry, which accounts for the drift velocity resulting from gravitational settling and diffusion. To validate the numerical model, a small-scale chamber was fabricated. The airflow characteristics and particle concentrations were measured by a phase Doppler Anemometer. Both simulated airflow and concentration profiles agree well with the experimental results. A strong inhomogeneous concentration was observed experimentally for 10 [mu]m aerosols. The computational model was further applied to study a simple hypothetical, yet more realistic scenario. The aim was to explore different levels of exposure predicted by the new model and the well-mixed model. Aerosols are initially uniformly distributed in one zone and subsequently transported and dispersed to an adjacent zone through an opening. Owing to the significant difference in the rates of transport and dispersion between aerosols and gases, inferred from the results, the well-mixed model tends to overpredict the concentration in the source zone, and under-predict the concentration in the exposed zone. The results are very useful to illustrate that the well-mixed assumption must be applied cautiously for exposure assessments as such an ideal condition may not be applied for coarse particles.},
  ISSN                     = {1352-2310},
  Keywords                 = {Dispersion
Exposure
Eulerian model
Well-mixed model
Multi-zone},
  Type                     = {Journal Article},
  Url                      = {http://www.sciencedirect.com/science/article/B6VH3-4R6B2VV-4/2/84b23c6a21eb36f1ad5be4454e746983}
}

@Article{Lai2009,
  Title                    = {Mucus-penetrating nanoparticles for drug and gene delivery to mucosal tissues},
  Author                   = {Lai, Samuel K. and Wang, Ying-Ying and Hanes, Justin},
  Journal                  = {Advanced Drug Delivery Reviews},
  Year                     = {2009},
  Number                   = {2},
  Pages                    = {158-171},
  Volume                   = {61},

  Abstract                 = {Mucus is a viscoelastic and adhesive gel that protects the lung airways, gastrointestinal (GI) tract, vagina, eye and other mucosal surfaces. Most foreign particulates, including conventional particle-based drug delivery systems, are efficiently trapped in human mucus layers by steric obstruction and/or adhesion. Trapped particles are typically removed from the mucosal tissue within seconds to a few hours depending on anatomical location, thereby strongly limiting the duration of sustained drug delivery locally. A number of debilitating diseases could be treated more effectively and with fewer side effects if drugs and genes could be more efficiently delivered to the underlying mucosal tissues in a controlled manner. This review first describes the tenacious mucus barrier properties that have precluded the efficient penetration of therapeutic particles. It then reviews the design and development of new mucus-penetrating particles that may avoid rapid mucus clearance mechanisms, and thereby provide targeted or sustained drug delivery for localized therapies in mucosal tissues.},
  ISSN                     = {0169-409X},
  Keywords                 = {Mucus
Mucus barrier properties
Mucus-penetrating particles
Therapeutic particles},
  Type                     = {Journal Article},
  Url                      = {http://www.sciencedirect.com/science/article/B6T3R-4V4KPJ9-3/2/db24e0644c63955f411ff51460bcde38}
}

@Article{Lambert1993,
  Title                    = {Functional significance of increased airway smooth muscle in asthma and COPD},
  Author                   = {Lambert, R.K. and Wiggs, B.R. and Kuwano, K. and Hogg, J.C. and Pare, P.D.},
  Journal                  = {Journal of Applied Physiology},
  Year                     = {1993},
  Pages                    = {2771-2781},
  Volume                   = {74},

  Type                     = {Journal Article}
}

@Article{Lane1978,
  Title                    = {Aerosol Deposition on a Flat Plate},
  Author                   = {Lane, D.D. and Stukel, J.J. },
  Journal                  = {Journal of Aerosol Science},
  Year                     = {1978},
  Pages                    = {191-197},
  Volume                   = {9},

  Type                     = {Journal Article}
}

@Article{Lang2003,
  Title                    = {Investigating the Nasal Cycle Using Endoscopy, Rhinoresistometry, and Acoustic Rhinometry},
  Author                   = {Lang, Christian and Grützenmacher, Stefan and Mlynski, Barbara and Plontke, Stefan and Mlynski, Gunter},
  Journal                  = {The Laryngoscope},
  Year                     = {2003},
  Note                     = {10.1097/00005537-200302000-00016},
  Number                   = {2},
  Pages                    = {284-289},
  Volume                   = {113},

  ISSN                     = {1531-4995},
  Type                     = {Journal Article},
  Url                      = {http://dx.doi.org/10.1097/00005537-200302000-00016}
}

@Article{LARY:LARY5541130216,
  Title                    = {Investigating the Nasal Cycle Using Endoscopy, Rhinoresistometry, and Acoustic Rhinometry},
  Author                   = {Lang, Christian and Grützenmacher, Stefan and Mlynski, Barbara and Plontke, Stefan and Mlynski, Gunter},
  Journal                  = {The Laryngoscope},
  Year                     = {2003},
  Number                   = {2},
  Pages                    = {284--289},
  Volume                   = {113},

  Doi                      = {10.1097/00005537-200302000-00016},
  ISSN                     = {1531-4995},
  Keywords                 = {Nasal cycle, turbulent behavior, acoustic rhinometry, rhinoresistometry, nasal airflow},
  Publisher                = {John Wiley \& Sons, Inc.},
  Url                      = {http://dx.doi.org/10.1097/00005537-200302000-00016}
}

@Article{Langtry2006,
  Title                    = {A Correlation-Based Transition Model Using Local Variables---Part II: Test Cases and Industrial Applications},
  Author                   = {Langtry, R. B. and Menter, F. R. and Likki, S. R. and Suzen, Y. B. and Huang, P. G. and Volker, S.},
  Journal                  = {Journal of Turbomachinery},
  Year                     = {2006},
  Number                   = {3},
  Pages                    = {423-434},
  Volume                   = {128},

  Keywords                 = {correlation theory
computational fluid dynamics
turbomachinery
aerodynamics
drag
wind turbines
compressors
blades
transonic flow
wakes},
  Type                     = {Journal Article},
  Url                      = {http://link.aip.org/link/?JTM/128/423/1
http://dx.doi.org/10.1115/1.2184353}
}

@Article{Lapenta2011,
  Title                    = {DEMOCRITUS: An adaptive particle in cell (PIC) code for object-plasma interactions},
  Author                   = {Lapenta, Giovanni},
  Journal                  = {Journal of Computational Physics},
  Year                     = {2011},
  Number                   = {12},
  Pages                    = {4679-4695},
  Volume                   = {230},

  Abstract                 = {A new method for the simulation of plasma materials interactions is presented. The method is based on the particle in cell technique for the description of the plasma and on the immersed boundary method for the description of the interactions between materials and plasma particles. A technique to adapt the local number of particles and grid adaptation are used to reduce the truncation error and the noise of the simulations, to increase the accuracy per unit cost. In the present work, the computational method is verified against known results. Finally, the simulation method is applied to a number of specific examples of practical scientific and engineering interest.},
  Doi                      = {10.1016/j.jcp.2011.02.041},
  ISSN                     = {0021-9991},
  Keywords                 = {Plasma-material interaction
Adaptive
Particle in cell},
  Type                     = {Journal Article},
  Url                      = {http://www.sciencedirect.com/science/article/pii/S0021999111001306}
}

@Article{Larocque,
  Title                    = {Parametric study of LES subgrid terms in a turbulent phase separation flow},
  Author                   = {Larocque, Jerome and Vincent, Stéphane and Lacanette, Delphine and Lubin, Pierre and Caltagirone, Jean-Paul},
  Journal                  = {International Journal of Heat and Fluid Flow},
  Volume                   = {In Press, Corrected Proof},

  Abstract                 = {The present work provides an a priori estimate of the specific LES subgrid terms occurring in the filtered 1-fluid Navier-Stokes equations that model the dynamics of turbulent two-phase flows involving separated phases. Three-dimensional simulations of a water/oil phase separation in a cubic cavity are carried out without explicit turbulent modeling. A parametric study is investigated for various surface tension coefficients, density and dynamic viscosity ratios. The numerical results obtained with these parametric simulations are then explicitly filtered to quantify the relative magnitude of each LES subgrid terms. A classification of these subgrid terms is finally provided for the phase inversion configuration.},
  ISSN                     = {0142-727X},
  Keywords                 = {Large Eddy Simulation
Turbulent flow
Two-phase flow},
  Type                     = {Journal Article},
  Url                      = {http://www.sciencedirect.com/science/article/B6V3G-4YR8B2J-1/2/2d14debf7eecd64c9d856851cc719d22}
}

@Article{Larsen1994,
  Title                    = {Evaluation of Conventional Press-and-Breathe Metered-Dose Inhaler Technique in 501 Patients},
  Author                   = {Larsen, Julie S. and Hahn, Mary and Ekholm, Bruce and Wick, Karen A.},
  Journal                  = {Journal of Asthma},
  Year                     = {1994},
  Number                   = {3},
  Pages                    = {193-199},
  Volume                   = {31},

  Doi                      = {doi:10.3109/02770909409044826},
  Type                     = {Journal Article},
  Url                      = {http://informahealthcare.com/doi/abs/10.3109/02770909409044826}
}

@Article{Lau2010,
  Title                    = {Mitral valve dynamics in structural and fluid-structure interaction models},
  Author                   = {Lau, K. D. and Diaz, V. and Scambler, P. and Burriesci, G.},
  Journal                  = {Medical Engineering \& Physics},
  Year                     = {2010},
  Number                   = {9},
  Pages                    = {1057-1064},
  Volume                   = {32},

  Abstract                 = {Modelling and simulation of heart valves is a challenging biomechanical problem due to anatomical variability, pulsatile physiological pressure loads and 3D anisotropic material behaviour. Current valvular models based on the finite element method can be divided into: those that do model the interaction between the blood and the valve (fluid-structure interaction or [`]wet' models) and those that do not (structural models or [`]dry' models). Here an anatomically sized model of the mitral valve has been used to compare the difference between structural and fluid-structure interaction techniques in two separately simulated scenarios: valve closure and a cardiac cycle. Using fluid-structure interaction, the valve has been modelled separately in a straight tubular volume and in a U-shaped ventricular volume, in order to analyse the difference in the coupled fluid and structural dynamics between the two geometries. The results of the structural and fluid-structure interaction models have shown that the stress distribution in the closure simulation is similar in all the models, but the magnitude and closed configuration differ. In the cardiac cycle simulation significant differences in the valvular dynamics were found between the structural and fluid-structure interaction models due to difference in applied pressure loads. Comparison of the fluid domains of the fluid-structure interaction models have shown that the ventricular geometry generates slower fluid velocity with increased vorticity compared to the tubular geometry. In conclusion, structural heart valve models are suitable for simulation of static configurations (opened or closed valves), but in order to simulate full dynamic behaviour fluid-structure interaction models are required.},
  Doi                      = {10.1016/j.medengphy.2010.07.008},
  ISSN                     = {1350-4533},
  Keywords                 = {Biomechanics
Finite element
Fluid-structure interaction
Heart valves
Mitral valve},
  Type                     = {Journal Article},
  Url                      = {http://www.sciencedirect.com/science/article/pii/S1350453310001505}
}

@Article{Launder1984,
  Title                    = {Numerical computation of convective heat transfer in complex turbulent flows: time to abandon wall functions?},
  Author                   = {Launder, B.E.},
  Journal                  = {International Journal Heat Mass Transfer},
  Year                     = {1984},
  Pages                    = {1485-1491},
  Volume                   = {27},

  Type                     = {Journal Article}
}

@Book{Launder1972,
  Title                    = {Lectures in Mathematical Models of Turbulence.},
  Author                   = {Launder, B.E. and Spalding, D.B.},
  Publisher                = { Academic Press},
  Year                     = {1972},

  Address                  = {London, England},

  Type                     = {Book}
}

@Article{Launder1975,
  Title                    = {Progress in the dvelopment of a Reynolds-Stress turbulence closure},
  Author                   = {Launder, B. E. and Reece, G. J. and Rodi, W.},
  Journal                  = {Journal of Fluid Mechanics},
  Year                     = {1975},
  Note                     = {Cited By (since 1996): 123
Export Date: 5 June 2011
Source: Scopus},
  Number                   = {3},
  Volume                   = {68},

  Type                     = {Journal Article},
  Url                      = {http://www.scopus.com/inward/record.url?eid=2-s2.0-0347154273&partnerID=40&md5=73d2652818e4b9c81a142bd45bf9f4ee}
}

@Article{Launder1974,
  Title                    = {The numerical computation of turbulent flows},
  Author                   = {Launder, B. E. and Spalding, D. B.},
  Journal                  = {Computer Methods in Applied Mechanics and Engineering},
  Year                     = {1974},
  Number                   = {2},
  Pages                    = {269-289},
  Volume                   = {3},

  ISSN                     = {0045-7825},
  Type                     = {Journal Article},
  Url                      = {http://www.sciencedirect.com/science/article/B6V29-481DWF5-10/2/4cdf7cb1ac680e1955dbc37a53bdb6bf}
}

@Article{Laurence2005,
  Title                    = {A robust formulation of the v2−f model},
  Author                   = {Laurence, D. and Uribe, J. and Utyuzhnikov, S.},
  Journal                  = {Flow, Turbulence and Combustion},
  Year                     = {2005},
  Note                     = {10.1007/s10494-005-1974-8},
  Number                   = {3},
  Pages                    = {169-185},
  Volume                   = {73},

  Abstract                 = {Abstract&nbsp;&nbsp;The elliptic relaxation approach of Durbin (Durbin, P.A., J. Theor. Comput. Fluid. Dyn. 3 (1991) 1–13), which accounts for wall blocking effects on the Reynolds stresses, is analysed herein from the numerical stability point of view, in the form of the $$\bar v^2 - f$$ . This model has been shown to perform very well on many challenging test cases such as separated, impinging and bluff-body flows, and including heat transfer. However, numerical convergence of the original model suggested by Durbin is quite difficult due to the boundary conditions requiring a coupling of variables at walls. A ‘code-friendly’ version of the model was suggested by Lien and Durbin (Lien, F.S. and Durbin, P.A., Non linear κ − ε − υ 2 modelling with application to high-lift. In: Proceedings of the Summer Program 1996, Stanford University (1996), pp. 5–22) which removes the need of this coupling to allow a segregated numerical procedure, but with somewhat less accurate predictions. A robust modification of the model is developed to obtain homogeneous boundary conditions at a wall for both $$\bar v^2 $$ and f. The modification is based on both a change of variables and alteration of the governing equations. The new version is tested on a channel, a diffuser flow and flow over periodic hills and shown to reproduce the better results of the original model, while retaining the easier convergence properties of the ‘code-friendly’ version.},
  Type                     = {Journal Article},
  Url                      = {http://dx.doi.org/10.1007/s10494-005-1974-8}
}

@Article{Laurent2012,
  Title                    = {Crucial Ignored Parameters on Nanotoxicology: The Importance of Toxicity Assay Modifications and “Cell Vision�},
  Author                   = {Laurent, Sophie and Burtea, Carmen and Thirifays, Coralie and Häfeli, Urs O. and Mahmoudi, Morteza},
  Journal                  = {PLoS ONE},
  Year                     = {2012},
  Number                   = {1},
  Pages                    = {e29997},
  Volume                   = {7},

  Abstract                 = {<p>Until now, the results of nanotoxicology research have shown that the interactions between nanoparticles (NPs) and cells are remarkably complex. In order to get a deep understanding of the NP-cell interactions, scientists have focused on the physicochemical effects. However, there are still considerable debates about the regulation of nanomaterials and the reported results are usually in contradictions. Here, we are going to introduce the potential key reasons for these conflicts. In this case, modification of conventional <italic>in vitro</italic> toxicity assays, is one of the crucial ignored matter in nanotoxicological sciences. More specifically, the conventional methods neglect important factors such as the sedimentation of NPs and absorption of proteins and other essential biomolecules onto the surface of NPs. Another ignored matter in nanotoxicological sciences is the effect of cell “vision� (i.e., cell type). In order to show the effects of these ignored subjects, we probed the effect of superparamagnetic iron oxide NPs (SPIONs), with various surface chemistries, on various cell lines. We found thatthe modification of conventional toxicity assays and the consideration of the “cell vision� concept are crucial matters to obtain reliable, and reproducible nanotoxicology data. These new concepts offer a suitable way to obtain a deep understanding on the cell-NP interactions. In addition, by consideration of these ignored factors, the conflict of future toxicological reports would be significantly decreased.</p>},
  Doi                      = {10.1371/journal.pone.0029997},
  Type                     = {Journal Article},
  Url                      = {http://dx.doi.org/10.1371%2Fjournal.pone.0029997}
}

@Article{Lavorini2008,
  Title                    = {Effect of incorrect use of dry powder inhalers on management of patients with asthma and COPD},
  Author                   = {Lavorini, Federico and Magnan, Antoine and Christophe Dubus, Jean and Voshaar, Thomas and Corbetta, Lorenzo and Broeders, Marielle and Dekhuijzen, Richard and Sanchis, Joaquin and Viejo, Jose L. and Barnes, Peter and Corrigan, Chris and Levy, Mark and Crompton, Graham K.},
  Journal                  = {Respiratory Medicine},
  Year                     = {2008},
  Number                   = {4},
  Pages                    = {593-604},
  Volume                   = {102},

  ISSN                     = {0954-6111},
  Keywords                 = {Dry powder inhaler
Inhalation technique
Asthma
COPD},
  Type                     = {Journal Article},
  Url                      = {http://www.sciencedirect.com/science/article/B6WWS-4RD3WCN-1/2/f073f8504e2dba786fc446a46f1cce9f}
}

@Article{Lawless1996,
  Title                    = {Particle charging bounds, symmetry relations, and an analytic charging rate model for the continuum regime},
  Author                   = {Lawless, Phil A.},
  Journal                  = {Journal of Aerosol Science},
  Year                     = {1996},
  Number                   = {2},
  Pages                    = {191-215},
  Volume                   = {27},

  ISSN                     = {0021-8502},
  Type                     = {Journal Article},
  Url                      = {http://www.sciencedirect.com/science/article/pii/0021850295005412}
}

@Article{Le2000,
  Title                    = {Near-wall turbulence structures in three-dimensional boundary layers},
  Author                   = {Le, Anh-Tuan and Coleman, Gary N. and Kim, John},
  Journal                  = {International Journal of Heat and Fluid Flow},
  Year                     = {2000},
  Note                     = {doi: DOI: 10.1016/S0142-727X(00)00035-7},
  Number                   = {5},
  Pages                    = {480-488},
  Volume                   = {21},

  ISSN                     = {0142-727X},
  Type                     = {Journal Article},
  Url                      = {http://www.sciencedirect.com/science/article/B6V3G-41362JX-3/2/b0b1cffbcb0bf2d6224cc74af0c3ea98}
}

@Article{Leader2004,
  Title                    = {Size and Morphology of the Trachea Before and After Lung Volume Reduction Surgery},
  Author                   = {Leader, Joseph K. and Rogers, Robert M. and Fuhrman, Carl R. and Sciurba, Frank C. and Zheng, Bin and Thompson, Paul F. and Weissfeld, Joel L. and Golla, Sara K. and Gur, David},
  Journal                  = {Am. Journal Roentgenol.},
  Year                     = {2004},
  Number                   = {2},
  Pages                    = {315-321},
  Volume                   = {183},

  Abstract                 = {OBJECTIVE. The purpose of this investigation was to determine the effect of lung volume reduction surgery on measured tracheal features. MATERIALS AND METHODS. Twenty-four male and 19 female patients with emphysema underwent lung volume reduction surgery, pulmonary function testing, and repeated CT. The tracheal air column was segmented from axial images. The sagittal and coronal dimensions of the intrathoracic trachea were determined. Tracheal morphology was quantified using the tracheal (coronal and sagittal dimensions) and circularity indexes. The results were compared with pulmonary function test results. RESULTS. Morphologic appearance of the intrathoracic trachea was consistent before and 3 months after surgery. The group means of the tracheal length, mean area, and volume were 78.60 mm ({+/-} 16.88 mm), 283.84 mm2 ({+/-} 61.47 mm2), and 22.59 cm3 ({+/-} 7.69 cm3), respectively, before surgery and 67.53 mm ({+/-} 15.78 mm), 309.12 mm2 ({+/-} 79.83 mm2), and 20.99 cm3 ({+/-} 7.27 cm3), respectively, after surgery (p < 0.05). Mean tracheal indexes were 0.85 ({+/-} 0.11) before surgery and 0.82 ({+/-} 0.04) after surgery (p < 0.01). Mean circularity indexes were 0.91 ({+/-} 0.03) before surgery and 0.90 ({+/-} 0.04) after surgery (p < 0.05). The size of the trachea was significantly correlated with lung volume before and after surgery (p < 0.05). The changes in tracheal features and changes in pulmonary function were not correlated (p > 0.05), except for tracheal area (p < 0.05). CONCLUSION. Our data suggest that tracheal dimensions reflect the severity of emphysema as reflected by increased lung volumes. Tracheal features were poor predictors of changes in postsurgical pulmonary function parameters evaluated in this preliminary study.},
  Type                     = {Journal Article},
  Url                      = {http://www.ajronline.org/cgi/content/abstract/183/2/315}
}

@Article{Leal1980,
  Title                    = {Particle Motions in a Viscous Fluid},
  Author                   = {Leal, L.G. },
  Journal                  = {Annual Review of Fluid Mechanics},
  Year                     = {1980},
  Pages                    = {435-476},
  Volume                   = {12},

  Type                     = {Journal Article}
}

@Article{Leclerc2014,
  Title                    = {Assessing sinus aerosol deposition: Benefits of SPECT-CT imaging},
  Author                   = {Leclerc, Lara and Pourchez, Jeremie and Prevot, Nathalie and Vecellio, Laurent and Le Guellec, Sandrine and Cottier, Michele and Durand, Marc},
  Journal                  = {International Journal of Pharmaceutics},
  Year                     = {2014},
  Note                     = {C:\Users\sean\AppData\Roaming\Zotero\Zotero\Profiles\16a4oype.default\zotero\storage\44Z5IBFM\Leclerc et al. - 2014 - Assessing sinus aerosol deposition Benefits of SP.pdf},
  Pages                    = {135-141},
  Volume                   = {462},

  Abstract                 = {Purpose: Aerosol inhalation therapy is one of the methods to treat rhinosinusitis. However the topical drug delivery to the posterior nose and paranasal sinuses shows only limited efficiency. A precise sinusal targeting remains a main challenge for aerosol treatment of sinus disorders. This paper proposes a comparative study of the nasal deposition patterns of micron and submicron particles using planar gamma-scintigraphy imaging vs. a new 3-dimensional (3D) imaging approach based on SPECT-CT measurements. Methods: Radiolabelled nebulizations have been performed on a plastinated model of human nasal cast coupled with a respiratory pump. First, the benefits provided by SPECT-CT imaging were compared with 2D gamma-scintigraphy and radioactive quantification of maxillary sinus lavage as reference for the sonic 2.8 mu m aerosol sinusal deposition. Then, the impact on nasal deposition of various airborne particle sizes was assessed. Results: The 2D methodology overestimates aerosol deposition in the maxillary sinuses by a factor 9 whereas the 3D methodology is in agreement with the maxillary sinus lavage reference methodology. Then with the SPECT-CT approach we highlighted that the higher particle size was mainly deposited in the central nasal cavity contrary to the submicron aerosol particles (33.8 +/- 0.6% of total deposition for the 2.8 mu m particles vs. 1 +/- 0.3% for the 230 nm particles). Conclusion: Benefits of SPECT/CT for the assessment of radiolabelled aerosol deposition in rhinology are clearly demonstrated. This 3D methodology should be preferentially used for scintigraphic imaging of sinusal deposition in Human. Copyright (C) 2013 Elsevier B.V. All rights reserved.},
  Doi                      = {10.1016/j.ijpharm.2013.12.032},
  ISSN                     = {0378-5173},
  Keywords                 = {aerosol deposition
Aerosol therapy
clearance
drug-delivery
lung
model
nasal cast
paranasal sinuses
pulsating air-flow
SPECT/CT imaging},
  Type                     = {Journal Article},
  Url                      = {http://ac.els-cdn.com/S0378517313011009/1-s2.0-S0378517313011009-main.pdf?_tid=2373ac6a-4220-11e4-969c-00000aab0f27&acdnat=1411366832_ce4add051045fe733d3aa865a2ea7abd}
}

@Article{Lee2013,
  Title                    = {Standardization of Malaysian adult female nasal cavity},
  Author                   = {Lee, C.F. and Abdullah, M.Z. and Ahmad, K.A. and Lutfi Shuaib, I.},
  Journal                  = {Computational and Mathematical Methods in Medicine},
  Year                     = {2013},
  Note                     = {cited By (since 1996)1},
  Volume                   = {2013},

  Art_number               = {519071},
  Document_type            = {Article},
  Source                   = {Scopus},
  Url                      = {http://www.scopus.com/inward/record.url?eid=2-s2.0-84880179187&partnerID=40&md5=4b6622097623ebf4e00fec2a2a7ee9a3}
}

@Article{Lee,
  Title                    = {Anatomical analysis of nasal obstruction: nasal cavity of patients complaining of stuffy nose},
  Author                   = {Lee, Dong Chang and Shin, Ji-Hyeon and Kim, Sung Won and Kim, Soo Whan and Kim, Byung Guk and Kang, Jun Myung and Cho, Jin Hee and Park, Yong Jin},
  Journal                  = {The Laryngoscope},
  Number                   = {6},
  Pages                    = {1381},
  Volume                   = {123},

  Abstract                 = {To evaluate the relationship between subjective symptoms of nasal obstruction and the corresponding nasal anatomical parameters using paranasal computed tomography (PNS CT).},
  Doi                      = {10.1002/lary.23841},
  ISSN                     = {0023-852X},
  Keywords                 = {Nasal Cavity -- Radiography
Nasal Obstruction -- Radiography
Nasal Septum -- Radiography
Tomography, X-ray Computed -- Methods},
  Type                     = {Journal Article}
}

@Article{Lee2002,
  Title                    = {Dispersion of aerosol bolus during one respiration cycle in a model of lung airways},
  Author                   = {Lee, Dong Y. and Lee, Jin W.},
  Journal                  = {Journal of Aerosol Science},
  Year                     = {2002},
  Number                   = {9},
  Pages                    = {1219-1234},
  Volume                   = {33},

  Abstract                 = {The dispersion of an aerosol bolus for one complete respiration cycle is analyzed numerically, using a lung airway model with four successive bifurcations. Unsteady Navier-Stokes equation and diffusion equation are solved by CFX-F3D, an FVM commercial code. Reynolds number based on the inlet conditions is varied between 0.533 and 24, and Womersley number between 0.0915 and 0.495, which correspond to the flow conditions in the 10th to the 18th generations of Weibel model at typical conditions of bolus experiments (tracheal breathing rate of 0.25 l/s, respiration period of 8 s and 0.8 [mu]m particle size). The dispersion characteristics during the inhalation period are seen the same as in steady inhalation, the total dispersion increasing monotonically with inhalation. When flow direction is reversed to exhalation, bolus dispersion decreases with exhalation at first, but comes to increase again at a rate almost equal to that for steady exhalation after a certain amount of exhalation. The maximum reduction of bolus dispersion after flow reversal is comparable to the increase of bolus dispersion in the last generation just before flow reversal.},
  ISSN                     = {0021-8502},
  Type                     = {Journal Article},
  Url                      = {http://www.sciencedirect.com/science/article/B6V6B-459J2VR-1/2/3899de4f37880b976ba9572252334f74}
}

@Article{Lee2011,
  Title                    = {Rolling/sliding of a particle on a flat wall in a linear shear flow at finite Re},
  Author                   = {Lee, Hyungoo and Ha, Man Yeong and Balachandar, S.},
  Journal                  = {International Journal of Multiphase Flow},
  Year                     = {2011},
  Number                   = {2},
  Pages                    = {108-124},
  Volume                   = {37},

  Abstract                 = {Recently Lee and Balachandar proposed analytically-based expressions for drag and lift coefficients for a spherical particle moving on a flat wall in a linear shear flow at finite Reynolds number. In order to evaluate the accuracy of these expressions, we have conducted direct numerical simulations of a rolling particle for shear Reynolds number up to 100. We assume that the particle rolls on a horizontal flat wall with a small gap separating the particle from the wall (L = 0.505) and thus avoiding the logarithmic singularity. The influence of the shear Reynolds number and the translational velocity of the particle on the hydrodynamic forces of the particle was investigated under both transient and the final drag-free and torque-free steady state. It is observed that the quasi-steady drag and lift expressions of Lee and Balachandar provide good approximation for the terminal state of the particle motion ranging from perfect sliding to perfect rolling. With regards to transient particle motion in a wall-bounded shear flow it is observed that the above validated quasi-steady drag and lift forces must be supplemented with appropriate wall-corrected added-mass and history forces in order to accurately predict the time-dependent approach to the terminal steady state. Quantitative comparison with the actual particle motion computed in the numerical simulations shows that the theoretical models quite effective in predicting rolling/sliding motion of a particle in a wall-bounded shear flow at moderate Re.},
  Doi                      = {10.1016/j.ijmultiphaseflow.2010.10.005},
  ISSN                     = {0301-9322},
  Keywords                 = {Drag and lift forces
Resuspension of particles
Rolling/sliding motion of a particle
Direct numerical simulations
Immersed boundary method
Added-mass and history forces},
  Type                     = {Journal Article},
  Url                      = {http://www.sciencedirect.com/science/article/pii/S0301932210001783}
}

@Article{Lee1996,
  Title                    = {Characteristics of inertial deposition in a double bifurcation},
  Author                   = {Lee, J. W. and Goo, J. H. and Chung, M. K.},
  Journal                  = {Journal of Aerosol Science},
  Year                     = {1996},
  Note                     = {doi: DOI: 10.1016/0021-8502(95)00538-2},
  Number                   = {1},
  Pages                    = {119-138},
  Volume                   = {27},

  ISSN                     = {0021-8502},
  Type                     = {Journal Article},
  Url                      = {http://www.sciencedirect.com/science/article/B6V6B-3WBXSPR-9/2/a02c8ac89c17b80ad1f4758910610cfb}
}

@Article{Lee2010,
  Title                    = {Unsteady flow characteristics through a human nasal airway},
  Author                   = {Lee, Jong-Hoon and Na, Yang and Kim, Sung-Kyun and Chung, Seung-Kyu},
  Journal                  = {Respiratory Physiology \& Neurobiology},
  Year                     = {2010},
  Number                   = {3},
  Pages                    = {136-146},
  Volume                   = {172},

  ISSN                     = {1569-9048},
  Keywords                 = {Respiratory airway
Human nose
Computational fluid dynamics simulation
Unsteady simulation
Heat transfer},
  Type                     = {Journal Article},
  Url                      = {http://www.sciencedirect.com/science/article/B6X16-502V6XT-2/2/74427e1cf44126371063625b0a18926f}
}

@Article{Lee2010a,
  Title                    = {Unsteady flow characteristics through a human nasal airway},
  Author                   = {Lee, J.-H., Na, Y., Kim, S.-K., Chung, S.-K.},
  Journal                  = {Respiratory Physiology \& Neurobiology},
  Year                     = {2010},
  Number                   = {136-146},
  Volume                   = {172},

  Type                     = {Journal Article}
}

@Article{Lee1982,
  Title                    = {On the motion of particles in turbulent duct flows},
  Author                   = {Lee, S. L. and Durst, F.},
  Journal                  = {International Journal of Multiphase Flow},
  Year                     = {1982},
  Number                   = {2},
  Pages                    = {125-146},
  Volume                   = {8},

  ISSN                     = {0301-9322},
  Type                     = {Journal Article},
  Url                      = {http://www.sciencedirect.com/science/article/pii/0301932282900131}
}

@Book{Lefebvre1989,
  Title                    = {Atomization and Sprays},
  Author                   = {Lefebvre, A.H.},
  Publisher                = {Hemisphere Publishing Corporation},
  Year                     = {1989},

  Type                     = {Book}
}

@Book{Lefebvre1983,
  Title                    = {Gas Turbine Combustion},
  Author                   = {Lefebvre, A.H.},
  Publisher                = {Hemisphere},
  Year                     = {1983},

  Address                  = {Washington D.C., USA},

  Type                     = {Book}
}

@Article{Leighton1985,
  Title                    = {The lift on a small sphere touching a plane in the presence of a simple shear flow},
  Author                   = {Leighton, David and Acrivos, Andreas},
  Journal                  = {Zeitschrift für Angewandte Mathematik und Physik (ZAMP)},
  Year                     = {1985},
  Number                   = {1},
  Pages                    = {174-178-178},
  Volume                   = {36},

  Doi                      = {10.1007/bf00949042},
  ISSN                     = {0044-2275},
  Keywords                 = {Physics and Astronomy},
  Type                     = {Journal Article},
  Url                      = {http://dx.doi.org/10.1007/BF00949042}
}

@Article{Leith1987,
  Title                    = {Drag on nonspherical objects},
  Author                   = {Leith, D.},
  Journal                  = {Aerosol Science and Technology},
  Year                     = {1987},
  Pages                    = {153-161},
  Volume                   = {6},

  Type                     = {Journal Article}
}

@Article{Lenney2000,
  Title                    = {Inappropriate inhaler use: assessment of use and patient preference of seven inhalation devices},
  Author                   = {Lenney, J. and Innes, J. A. and Crompton, G. K.},
  Journal                  = {Respiratory Medicine},
  Year                     = {2000},
  Number                   = {5},
  Pages                    = {496-500},
  Volume                   = {94},

  ISSN                     = {0954-6111},
  Keywords                 = {inhaler devices
inhaler technique
inhaler preference.},
  Type                     = {Journal Article},
  Url                      = {http://www.sciencedirect.com/science/article/B6WWS-45BCP9V-5M/2/3a9291a581d38952244961d6d49f09aa}
}

@TechReport{Leonard1990,
  Title                    = {ULTRA-SHARP Nonoscillatory convection schemes for high-speed steady multidimensional flow},
  Author                   = {Leonard, B.P. and Mokhtari, S.},
  Institution              = {NASA Lewis Research Center},
  Year                     = {1990},
  Type                     = {Report}
}

@Article{Leong2009,
  Title                    = {A systematic review of the nasal index and the significance of the shape and size of the nose in rhinology},
  Author                   = {Leong, S.C. and Eccles, R.},
  Journal                  = {Clinical Otolaryngology},
  Year                     = {2009},
  Number                   = {3},
  Pages                    = {191--198},
  Volume                   = {34},

  Doi                      = {10.1111/j.1749-4486.2009.01905.x},
  ISSN                     = {1749-4486},
  Publisher                = {Blackwell Publishing Ltd},
  Url                      = {http://dx.doi.org/10.1111/j.1749-4486.2009.01905.x}
}

@Book{Lerman1979,
  Title                    = {Geochemical processes},
  Author                   = {Lerman, A. },
  Publisher                = {Wiley},
  Year                     = {1979},

  Address                  = {New York},

  Type                     = {Book}
}

@Article{Lesieur2001,
  Title                    = {Favre filtering and macro-temperature in large-eddy simulations of compressible turbulence},
  Author                   = {Lesieur, Marcel and Comte, Pierre},
  Journal                  = {Comptes Rendus de l'Académie des Sciences - Series IIB - Mechanics},
  Year                     = {2001},
  Number                   = {5},
  Pages                    = {363-368},
  Volume                   = {329},

  Abstract                 = {We show how Favre density-weighted filtering, used with a macro-temperature, simplifies considerably the formalism of large-eddy simulations in compressible turbulence. The method gives good results for a plane channel at Mach 0.3, and a transonic flow above a rectangular cavity. In both cases, the shedding of [Lambda]-shaped coherent vortices is very well characterized thanks to positive iso-surfaces of Q, the velocity-gradient tensor second invariant.},
  ISSN                     = {1620-7742},
  Keywords                 = {instability
turbulence
Favre filtering
instabilité
filtrage de Favre},
  Type                     = {Journal Article},
  Url                      = {http://www.sciencedirect.com/science/article/B6W82-4378T85-9/2/21dc6349e4acad66a24db8cc72fac09d}
}

@Article{Letzel2008,
  Title                    = {High resolution urban large-eddy simulation studies from street canyon to neighbourhood scale},
  Author                   = {Letzel, Marcus Oliver and Krane, Martina and Raasch, Siegfried},
  Journal                  = {Atmospheric Environment},
  Year                     = {2008},
  Number                   = {38},
  Pages                    = {8770-8784},
  Volume                   = {42},

  Abstract                 = {Urban turbulence characteristics are investigated at street canyon and neighbourhood scale. Three high resolution urban large-eddy simulation (LES) studies are performed using the urban version of the parallelized LES model PALM. Validation shows that the urban PALM version is in line with experimental and previous LES results, i.e. superior to the faster/cheaper conventional Reynolds-averaged (RANS) models. Two studies focus on quasi-2D urban street canyons driven by perpendicular flow. First is a parametric study of turbulence characteristics and flow dynamics within the canyon. The main results are: (1) Integral vertical turbulence profiles in deep canyons scale with canyon width. This is relevant for urban canopy parameterizations in larger-scale meteorological models. (2) A new concept of a "cavity shear layer" complements classical free shear layer concepts. (3) For the first time in urban LES Kelvin-Helmholtz instabilities are identified at the top of the urban street canyon. This is relevant for modelling urban dispersion, because the street canyon circulation is more intermittent than suggested by previous RANS results. Second, an Eulerian dispersion case study shows that differences in canyon flow dynamics are reflected in canyon dispersion characteristics compared to a previous RANS study. Third is a neighbourhood scale urban LES feasibility study: a passive tracer Lagrangian dispersion animation of Shinjuku, downtown Tokyo reveals turbulent flow features, upstream flow and intermittency. The main implications are always to use 3D models for turbulence simulations even in quasi-2D geometries, and not to underestimate the intermittency of turbulent flow. Standard deviations of velocity components within the canyon should not be treated as constant for perpendicular ambient wind but may be parameterized conveniently based on a vertical scaling with canyon width in deep canyons.},
  ISSN                     = {1352-2310},
  Keywords                 = {Urban large-eddy simulation
Complex topography
Street canyon
Atmospheric dispersion
Turbulent flow visualization},
  Type                     = {Journal Article},
  Url                      = {http://www.sciencedirect.com/science/article/B6VH3-4T72X31-3/2/de3e539c8bbc4d236fdaa2f084944524}
}

@Article{Levasseur2008,
  Title                    = {Unstructured Large Eddy Simulation of the passive control of the flow in a weapon bay},
  Author                   = {Levasseur, V. and Sagaut, P. and Mallet, M. and Chalot, F.},
  Journal                  = {Journal of Fluids and Structures},
  Year                     = {2008},
  Number                   = {8},
  Pages                    = {1204-1215},
  Volume                   = {24},

  Abstract                 = {The control of cavity flows has been investigated by the means of Large Eddy Simulations. The computations have been carried out on unstructured meshes to assess the efficiency of two passive acoustic oscillation suppression devices: the rod-in-crossflow and the flat-top spoiler. Despite a sustained interest and many experiments, a clear explanation for observed reduction in the flow-induced structure load is still missing. This work explores different hypotheses: the modification of the mean field and its linear stability properties, a pure deflection effect of the separated shear layer, or scale coupling between the rod wake and the turbulent mixing layer over the cavity. The aim here is to enhance the experimental database and provide leads towards a better understanding of the phenomena. The selected test-case is a cavity of length/depth ratio equal to 5, at Mach and Reynolds number of M[infinity]=0.85 and ReL=7.106, respectively.},
  ISSN                     = {0889-9746},
  Keywords                 = {Large Eddy simulation
Rossiter aero-acoustic mode
Flow control
Cavity flow
Turbulence},
  Type                     = {Journal Article},
  Url                      = {http://www.sciencedirect.com/science/article/B6WJG-4TTM32R-2/2/29498e5e395edaadabb2d718e2e9381d}
}

@Book{Levich1962,
  Title                    = {Physicochemical Hydrodynamics},
  Author                   = {Levich, V. },
  Publisher                = {Prentice-Hall},
  Year                     = {1962},

  Address                  = {Englewood Cliffs, NJ},

  Type                     = {Book}
}

@Article{Li1995,
  Title                    = {Computer simulation of particle deposition in the upper tracheobronchial tree},
  Author                   = {Li, A. and Ahmadi, G.},
  Journal                  = {Aerosol Science and Technology},
  Year                     = {1995},
  Note                     = {Cited By (since 1996): 26
Export Date: 5 June 2011
Source: Scopus},
  Number                   = {2},
  Pages                    = {201-223},
  Volume                   = {23},

  Type                     = {Journal Article},
  Url                      = {http://www.scopus.com/inward/record.url?eid=2-s2.0-0029347552&partnerID=40&md5=d8485ec4b873ece9c9f2b99092a4ad25}
}

@Article{Li1993,
  Title                    = {Aerosol Particle Deposition with Electrostatic Attraction in a Turbulent Channel Flow},
  Author                   = {Li, A. and Ahmadi, G.},
  Journal                  = {Journal of Colloid And Interface Science},
  Year                     = {1993},
  Note                     = {Cited By (since 1996): 13
Export Date: 5 June 2011
Source: Scopus},
  Number                   = {2},
  Pages                    = {476-482},
  Volume                   = {158},

  Type                     = {Journal Article},
  Url                      = {http://www.scopus.com/inward/record.url?eid=2-s2.0-0001111182&partnerID=40&md5=998ac3a8e1df407852d50e5aa66fe109}
}

@Article{Li1993a,
  Title                    = {Computer Simulation of Deposition of Aerosols in a Turbulent Channel Flow with Rough Walls},
  Author                   = {Li, Amy and Ahmadi, Goodarz},
  Journal                  = {Aerosol Science and Technology},
  Year                     = {1993},
  Number                   = {1},
  Pages                    = {11-24},
  Volume                   = {18},

  ISSN                     = {0278-6826},
  Type                     = {Journal Article},
  Url                      = {http://www.informaworld.com/10.1080/02786829308959581}
}

@Article{Li1993b,
  Title                    = {Deposition of aerosols on surfaces in a turbulent channel flow},
  Author                   = {Li, Amy and Ahmadi, Goodarz},
  Journal                  = {International Journal of Engineering Science},
  Year                     = {1993},
  Number                   = {3},
  Pages                    = {435-451},
  Volume                   = {31},

  ISSN                     = {0020-7225},
  Type                     = {Journal Article},
  Url                      = {http://www.sciencedirect.com/science/article/pii/002072259390017O}
}

@Article{Li1992,
  Title                    = {Dispersion and deposition of spherical particles from point sources in a turbulent channel flow},
  Author                   = {Li, A. and Ahmadi, G.},
  Journal                  = {Aerosol Science and Technology},
  Year                     = {1992},
  Number                   = {4},
  Pages                    = {209 - 226},
  Volume                   = {16},

  ISSN                     = {0278-6826},
  Type                     = {Journal Article},
  Url                      = {http://www.informaworld.com/10.1080/02786829208959550}
}

@Article{Li1994,
  Title                    = {Aerosol particle deposition in an obstructed turbulent duct flow},
  Author                   = {Li, Amy and Ahmadi, Goodarz and Bayer, Raymond G. and Gaynes, Micheal A.},
  Journal                  = {Journal of Aerosol Science},
  Year                     = {1994},
  Number                   = {1},
  Pages                    = {91-112},
  Volume                   = {25},

  ISSN                     = {0021-8502},
  Type                     = {Journal Article},
  Url                      = {http://www.sciencedirect.com/science/article/pii/0021850294901848}
}

@Article{Li2012,
  Title                    = {Particle inhalation and deposition in a human nasal cavity from the external surrounding environment},
  Author                   = {Li, Xiangdong and Inthavong, Kiao and Tu, Jiyuan},
  Journal                  = {Building and Environment},
  Year                     = {2012},
  Number                   = {0},
  Pages                    = {32-39},
  Volume                   = {47},

  Doi                      = {http://dx.doi.org/10.1016/j.buildenv.2011.04.032},
  ISSN                     = {0360-1323},
  Keywords                 = {Facial features
Inlet velocity profiles
Particle deposition
Nasal cavity
CFD},
  Type                     = {Journal Article},
  Url                      = {http://www.sciencedirect.com/science/article/pii/S036013231100134X}
}

@Article{Li2012a,
  Title                    = {Particle inhalation and deposition in a human nasal cavity from the external surrounding environment},
  Author                   = {Li, Xiangdong and Inthavong, Kiao and Tu, Jiyuan},
  Journal                  = {Building and Environment},
  Year                     = {2012},
  Number                   = {0},
  Pages                    = {32-39},
  Volume                   = {47},

  Doi                      = {10.1016/j.buildenv.2011.04.032},
  ISSN                     = {0360-1323},
  Keywords                 = {Facial features
Inlet velocity profiles
Particle deposition
Nasal cavity
CFD},
  Type                     = {Journal Article},
  Url                      = {http://www.sciencedirect.com/science/article/pii/S036013231100134X}
}

@Misc{Li2012b,
  Title                    = {Particle inhalation and deposition in a human nasal cavity from the external surrounding environment},

  Author                   = {Li, X. and Inthavong, K. and Tu, J.},
  Year                     = {2012},

  Keywords                 = {CFD
Facial features
Inlet velocity profiles
Nasal cavity
Particle deposition},
  Publisher                = {Pergamon},
  Type                     = {Generic}
}

@Misc{Li2012c,
  Title                    = {Particle inhalation and deposition in a human nasal cavity from the external surrounding environment},

  Author                   = {Li, X. and Inthavong, K. and Tu, J.},
  Year                     = {2012},

  Keywords                 = {CFD
Facial features
Inlet velocity profiles
Nasal cavity
Particle deposition},
  Publisher                = {Pergamon},
  Type                     = {Generic}
}

@Article{Li2012d,
  Title                    = {Particle inhalation and deposition in a human nasal cavity from the external surrounding environment},
  Author                   = {Li, Xiangdong and Inthavong, Kiao and Tu, Jiyuan},
  Journal                  = {Building and Environment},
  Year                     = {2012},
  Pages                    = {32-39},
  Volume                   = {47},

  Doi                      = {http://dx.doi.org/10.1016/j.buildenv.2011.04.032},
  ISSN                     = {0360-1323},
  Keywords                 = {cavity
CFD
Deposition
Facial
features
Inlet
Nasal
Particle
profiles
velocity},
  Type                     = {Journal Article},
  Url                      = {http://ac.els-cdn.com/S036013231100134X/1-s2.0-S036013231100134X-main.pdf?_tid=d6166a7a-421f-11e4-be87-00000aab0f6b&acdnat=1411366703_4cf9cd01813280d3cfed36c51d94ad49}
}

@Article{Li2007,
  Title                    = {Role of ventilation in airborne transmission of infectious agents in the built environment 2013; a multidisciplinary systematic review},
  Author                   = {Li, Y. and Leung, G. M. and Tang, J. W. and Yang, X. and Chao, C. Y. H. and Lin, J. Z. and Lu, J. W. and Nielsen, P. V. and Niu, J. and Qian, H. and Sleigh, A. C. and Su, H. J. J. and Sundell, J. and Wong, T. W. and Yuen, P. L.},
  Journal                  = {Indoor Air},
  Year                     = {2007},
  Note                     = {10.1111/j.1600-0668.2006.00445.x},
  Number                   = {1},
  Pages                    = {2-18},
  Volume                   = {17},

  ISSN                     = {1600-0668},
  Type                     = {Journal Article},
  Url                      = {http://dx.doi.org/10.1111/j.1600-0668.2006.00445.x }
}

@Article{Li2008,
  Title                    = {Effect of sidewall boundary conditions on unsteady high speed cavity flow and acoustics},
  Author                   = {Li, Zhisong and Hamed, Awatef},
  Journal                  = {Computers \& Fluids},
  Year                     = {2008},
  Number                   = {5},
  Pages                    = {584-592},
  Volume                   = {37},

  Abstract                 = {The effect of sidewall boundary conditions on the computed unsteady flow and sound pressure level is investigated in a transonic open cavity. The hybrid approach used for modeling turbulence combines a Reynolds averaged mode in the boundary layer, and a large eddy simulation mode in the massively separated flow region within the cavity to resolve the wide dynamic range involved. Computational results are presented for the instantaneous vorticity and for the sound pressure level spectra. Comparison of the results obtained using inviscid and periodic sidewall boundary conditions show the sensitivity of the computed SPL spectra and autocorrelation to the conditions enforced at the sidewalls. The computed SPL spectra are also compared with available experimental results, with LES computational results, and with prior investigations based on the same hybrid turbulence model without the wall function used in the current investigation. The comparisons show that the current results obtained using inviscid sidewall boundary conditions are closest to the experimental sound pressure level spectra and that agreement is achieved at considerable saving in required computational resources.},
  ISSN                     = {0045-7930},
  Type                     = {Journal Article},
  Url                      = {http://www.sciencedirect.com/science/article/B6V26-4PC8RDT-2/2/d74c02937abed6244cf777dfccd06040}
}

@Article{Li,
  Title                    = {Simulation of airflow fields and microparticle deposition in realistic human lung airway models. Part II: Particle transport and deposition},
  Author                   = {Li, Zheng and Kleinstreuer, Clement and Zhang, Zhe},
  Journal                  = {European Journal of Mechanics - B/Fluids},
  Note                     = {doi: DOI: 10.1016/j.euromechflu.2007.02.004},
  Number                   = {5},
  Pages                    = {650-668},
  Volume                   = {26},

  ISSN                     = {0997-7546},
  Keywords                 = {Asymmetric upper airways
Matching inlet Reynolds number
3-D microparticle transport simulations
Particle depositions and segmental correlations},
  Type                     = {Journal Article},
  Url                      = {http://www.sciencedirect.com/science/article/B6VKX-4N7SBM6-1/2/7574533db19007cb1e807b1c37d7b672}
}

@Article{Li2007a,
  Title                    = {Particle deposition in the human tracheobronchial airways due to transient inspiratory flow patterns},
  Author                   = {Li, Z. and Kleinstreuer, C. and Zhang, Z.},
  Journal                  = {Aerosol Science},
  Year                     = {2007},
  Pages                    = {625-644},
  Volume                   = {38},

  Type                     = {Journal Article}
}

@Article{Li2007b,
  Title                    = {Simulation of airflow fields and microparticle deposition in realistic human lung airway models. Part I: Airflow patterns},
  Author                   = {Li, Z. and Kleinstreuer, C. and Zhang, Z.},
  Journal                  = {European Journal of Mechanics B/Fluids},
  Year                     = {2007},
  Pages                    = {632-649},
  Volume                   = {26},

  Type                     = {Journal Article}
}

@Article{Li2007c,
  Title                    = {Simulation of airflow fields and microparticle deposition in realistic human lung airway models. Part II: Particle transport and deposition},
  Author                   = {Li, Zheng and Kleinstreuer, Clement and Zhang, Zhe},
  Journal                  = {European Journal of Mechanics - B/Fluids},
  Year                     = {2007},
  Note                     = {doi: DOI: 10.1016/j.euromechflu.2007.02.004},
  Number                   = {5},
  Pages                    = {650-668},
  Volume                   = {26},

  ISSN                     = {0997-7546},
  Keywords                 = {Asymmetric upper airways
Matching inlet Reynolds number
3-D microparticle transport simulations
Particle depositions and segmental correlations},
  Type                     = {Journal Article},
  Url                      = {http://www.sciencedirect.com/science/article/B6VKX-4N7SBM6-1/2/7574533db19007cb1e807b1c37d7b672}
}

@Article{Lian-Ping1993,
  Title                    = {Dispersion of heavy particles by turbulent motion},
  Author                   = {Lian-Ping, Wang and Stock, D. E.},
  Journal                  = {Journal of the Atmospheric Sciences},
  Year                     = {1993},
  Note                     = {Cited By (since 1996): 101
Export Date: 6 June 2011
Source: Scopus},
  Number                   = {13},
  Pages                    = {1897-1913},
  Volume                   = {50},

  Type                     = {Journal Article},
  Url                      = {http://www.scopus.com/inward/record.url?eid=2-s2.0-0027789393&partnerID=40&md5=fb98c10436b024c1322330db6f3bd0da}
}

@Book{Lide1994,
  Title                    = {Characteristics of Particles and Particle Dispersoids. Handbook of Chemistry and Physics},
  Author                   = {Lide, David R.},
  Publisher                = {CRC Press, 1994},
  Year                     = {1994},

  Address                  = { Florida},
  Edition                  = {75th Edition},

  Type                     = {Book}
}

@Article{Lien2001,
  Title                    = {Computations of transonic flow with the v2-f turbulence model},
  Author                   = {Lien, F. and Kalitzin, G.},
  Journal                  = {International Journal of Heat and Fluid Flow},
  Year                     = {2001},
  Pages                    = {53-56},
  Volume                   = {22},

  Type                     = {Journal Article}
}

@Article{Lien2001a,
  Title                    = {Computations of transonic flow with the v2-f turbulence model},
  Author                   = {Lien, F. S. and Kalitzin, G.},
  Journal                  = {International Journal of Heat and Fluid Flow},
  Year                     = {2001},
  Note                     = {[1]},
  Pages                    = {53-61},
  Volume                   = {22},

  Type                     = {Journal Article},
  Url                      = {http://www.ingentaconnect.com/content/els/0142727x/2001/00000022/00000001/art00073
http://dx.doi.org/10.1016/S0142-727X(00)00073-4}
}

@Article{Lienar2003,
  Title                    = {Nasal mucosal temperature after exposure to cold, dry air and hot, humid air},
  Author                   = {Lienar, K. and Leiacker, R. and Lindemann, J. and Rettinger, G. and Keck, T.},
  Journal                  = {Acta Otolaryngol},
  Year                     = {2003},
  Pages                    = {851-856},
  Volume                   = {123},

  Type                     = {Journal Article}
}

@Article{Lin2006,
  Title                    = {Multiscale simulation of air flow in the CT-based lung model},
  Author                   = {Lin, C. L. and Tawhai, M. H. and McLennan, G. and Hoffman, E. A.},
  Journal                  = {Journal of Biomechanics},
  Year                     = {2006},
  Number                   = {Supplement 1},
  Pages                    = {S265-S265},
  Volume                   = {39},

  ISSN                     = {0021-9290},
  Type                     = {Journal Article},
  Url                      = {http://www.sciencedirect.com/science/article/B6T82-4KR88PB-1F0/2/ca85ce20c4fd803089fee44365fdc6cd}
}

@Article{Lin2007,
  Title                    = {Characteristics of the turbulent laryngeal jet and its effect on airflow in the human intra-thoracic airways},
  Author                   = {Lin, Ching-Long and Tawhai, Merryn H. and McLennan, Geoffrey and Hoffman, Eric A.},
  Journal                  = {Respiratory Physiology \& Neurobiology},
  Year                     = {2007},
  Note                     = {doi: DOI: 10.1016/j.resp.2007.02.006},
  Number                   = {2-3},
  Pages                    = {295-309},
  Volume                   = {157},

  ISSN                     = {1569-9048},
  Keywords                 = {MDCT-based airway model
Computational fluid dynamics
Turbulence
Laryngeal jet
Ventilation
Computed tomography},
  Type                     = {Journal Article},
  Url                      = {http://www.sciencedirect.com/science/article/B6X16-4N25VVB-1/2/63a42511e593845033bb3f9d6c65d333}
}

@Article{Lin2009,
  Title                    = {Transport and deposition of nanoparticles in bend tube with circular cross-section},
  Author                   = {Lin, Peifeng and Lin, Jianzhong},
  Journal                  = {Progress in Natural Science},
  Year                     = {2009},
  Note                     = {doi: DOI: 10.1016/j.pnsc.2008.05.028},
  Number                   = {1},
  Pages                    = {33-39},
  Volume                   = {19},

  Abstract                 = {Transport and deposition of nanoparticles in bend tube with circular cross-section were simulated numerically for different Reynolds numbers and Dean numbers. A finite-volume code and the SIMPLE scheme were used to solve the equations. The results show that the distribution of nanoparticle concentration is symmetrical with respect to the top and bottom sides of the tube. The diameter of the nanoparticles has a weak effect on the distribution of nanoparticle concentration. The maximum and minimum of the deposition enhancement factor occur near the outside and inside walls of the bend tube, respectively. The higher the Reynolds number is, the shorter is the time for nanoparticle deposition. The bend curvature radius has a slight effect on the deposition enhancement factor.},
  ISSN                     = {1002-0071},
  Keywords                 = {Nanoparticles
Transport
Deposition
Numerical simulation},
  Type                     = {Journal Article},
  Url                      = {http://www.sciencedirect.com/science/article/B8JH4-4TYPJKV-1/2/6845dc90e8ab6b4d9e6eae1b61c6f3fd}
}

@Article{Lin1998,
  Title                    = {DROP AND SPRAY FORMATION FROM A LIQUID JET},
  Author                   = {Lin, S. P. and Reitz, R. D.},
  Journal                  = {Annual Review of Fluid Mechanics},
  Year                     = {1998},
  Number                   = {1},
  Pages                    = {85-105},
  Volume                   = {30},

  Doi                      = {doi:10.1146/annurev.fluid.30.1.85},
  Type                     = {Journal Article},
  Url                      = {http://arjournals.annualreviews.org/doi/abs/10.1146/annurev.fluid.30.1.85}
}

@Article{Lin2005,
  Title                    = {CFD study on effect of the air supply location on the performance of the displacement ventilation system},
  Author                   = {Lin, Zhang and Chow, T. T. and Tsang, C. F. and Fong, K. F. and Chan, L. S.},
  Journal                  = {Building and Environment},
  Year                     = {2005},
  Note                     = {doi: DOI: 10.1016/j.buildenv.2004.09.003},
  Number                   = {8},
  Pages                    = {1051-1067},
  Volume                   = {40},

  ISSN                     = {0360-1323},
  Keywords                 = {Displacement ventilation
Thermal comfort
Indoor air quality
Air velocity
Temperature
Percentage of dissatisfied people
Predicted percentage of dissatisfied
Toluene
Benzene
Formaldehyde
Mean age of air
Carbon dioxide
Air supply and exhaust locations},
  Type                     = {Journal Article},
  Url                      = {http://www.sciencedirect.com/science/article/B6V23-4DTPBYM-1/2/cfbe3e7baa193dc00615f14272f34efe}
}

@Article{Lindemann,
  Title                    = {Numerical simulation of intranasal airflow after radical sinus surgery},
  Author                   = {Lindemann, Joerg and Brambs, Hans-Juergen and Keck, Tilman and Wiesmiller, Kerstin M. and Rettinger, Gerhard and Pless, Daniela},
  Journal                  = {American Journal of Otolaryngology},
  Number                   = {3},
  Pages                    = {175-180},
  Volume                   = {26},

  Abstract                 = {Purpose Radical sinus surgery disturbs intranasal humidification and heating of inspired air, resulting in reduced air conditioning mainly caused by a disturbed airflow. Therefore, the aim of this study was to simulate the intranasal airflow after radical sinus surgery during inspiration by means of numerical simulation.Material and methods A bilateral model of the human nose with maxillectomy, ethmoidectomy, and resection of the lateral nasal wall and the turbinates on one side based on a multislice computed tomographic scan was reconstructed. An unsteady numerical simulation displaying the intranasal airflow patterns applying the computational fluid dynamics solver Fluent 6.1.22 was performed.Results Spacious vortices throughout the entire nasal cavity and the paranasal sinuses caused by the radical resections occurred, causing a less-intense contact between air and the surrounding nasal wall. An enlargement of the nasal cavity volume and a reduction of the nasal surface area in ratio to the nasal cavity volume could be observed.Conclusions Aggressive sinus surgery leads to disturbed intranasal air conditioning caused by disturbed intranasal airflow patterns and a reduction of the surface area in relation to the nasal volume. The presented numerical simulation demonstrates the close relation between air conditioning and intranasal airflow. It can be helpful to understand and interpret in vivo measured data of intranasal temperature and humidity.},
  ISSN                     = {0196-0709},
  Type                     = {Journal Article},
  Url                      = {http://www.sciencedirect.com/science/article/B6W9S-4G1V46N-9/2/abf4ba23e350962a153f1d71c12b89a6}
}

@Article{Lindemann2005,
  Title                    = {Numerical simulation of intranasal airflow after radical sinus surgery},
  Author                   = {Lindemann, Joerg and Brambs, Hans-Juergen and Keck, Tilman and Wiesmiller, Kerstin M. and Rettinger, Gerhard and Pless, Daniela},
  Journal                  = {American Journal of Otolaryngology},
  Year                     = {2005},
  Note                     = {doi: DOI: 10.1016/j.amjoto.2005.02.010},
  Number                   = {3},
  Pages                    = {175-180},
  Volume                   = {26},

  Abstract                 = {Purpose Radical sinus surgery disturbs intranasal humidification and heating of inspired air, resulting in reduced air conditioning mainly caused by a disturbed airflow. Therefore, the aim of this study was to simulate the intranasal airflow after radical sinus surgery during inspiration by means of numerical simulation.Material and methods A bilateral model of the human nose with maxillectomy, ethmoidectomy, and resection of the lateral nasal wall and the turbinates on one side based on a multislice computed tomographic scan was reconstructed. An unsteady numerical simulation displaying the intranasal airflow patterns applying the computational fluid dynamics solver Fluent 6.1.22 was performed.Results Spacious vortices throughout the entire nasal cavity and the paranasal sinuses caused by the radical resections occurred, causing a less-intense contact between air and the surrounding nasal wall. An enlargement of the nasal cavity volume and a reduction of the nasal surface area in ratio to the nasal cavity volume could be observed.Conclusions Aggressive sinus surgery leads to disturbed intranasal air conditioning caused by disturbed intranasal airflow patterns and a reduction of the surface area in relation to the nasal volume. The presented numerical simulation demonstrates the close relation between air conditioning and intranasal airflow. It can be helpful to understand and interpret in vivo measured data of intranasal temperature and humidity.},
  ISSN                     = {0196-0709},
  Type                     = {Journal Article},
  Url                      = {http://www.sciencedirect.com/science/article/B6W9S-4G1V46N-9/2/abf4ba23e350962a153f1d71c12b89a6}
}

@Article{Lindemann2006,
  Title                    = {Nasal air temperature and airflow during respiration in numerical simulation based on multislice computed tomograpthy scan},
  Author                   = {Lindemann, J. and Keck, T. and Wiesmiller, K. and Sander, B. and Brambs, H.J. and Rettinger, G. and Pless, D.},
  Journal                  = {Am J Rhinol},
  Year                     = {2006},
  Pages                    = {219-223},
  Volume                   = {20},

  Type                     = {Journal Article}
}

@Article{Lindemann2004,
  Title                    = {A numerical simulation of intranasal air temperature during inspiration},
  Author                   = {Lindemann, J. and Keck, T. and Wiesmiller, K. and Sander, B. and Brambs, H.J. and Rettinger, G. and Pless, P.},
  Journal                  = {The Laryngoscope},
  Year                     = {2004},
  Pages                    = {1037-1041},
  Volume                   = {114},

  Type                     = {Journal Article}
}

@Article{Lindemann2002,
  Title                    = {Nasal mucosal temperature during respiration},
  Author                   = {Lindemann, J. and Leiacker, R. and Rettinger, G. and Keck, T.},
  Journal                  = {Clin. Otolaryngol},
  Year                     = {2002},
  Pages                    = {135-139},
  Volume                   = {27},

  Type                     = {Journal Article}
}

@Article{Lindemann2008,
  Title                    = {Age Related Changes in Intranasal Air Conditioning in the Elderly},
  Author                   = {Lindemann, Joerg and Sannwald, Diana and Wiesmiller, Kerstin},
  Journal                  = {Laryngoscope},
  Year                     = {2008},
  Pages                    = {1472-1475},
  Volume                   = {118},

  Abstract                 = {Objectives/Hypothesis: Elderly patients frequently complain about the feeling of a dry nose and recurrent crusting probably due to age-related degenerative effects of the nasal mucosa. Data on intranasal air conditioning in elderly patients are missing. The aim of the study was to compare intranasal heating and humidification of respiratory air in elderly subjects and a younger control group. Additionally, rhinomanometrical and rhinometrical data were examined. Study Design: The study was conceived as randomized, prospective study. Methods: Forty study subjects (median age, 70 years; range, 61-84) and 40 control subjects (median age, 27 years; range, 20-40) were included in the study. In vivo air temperature and humidity values were simultaneously measured at defined intranasal sites. Active anterior rhinomanometry and acoustic rhinometry also were performed in every subject. Results: In the study group, the median end-inspiratory air temperature (degrees C)/absolute humidity (g/m(3)) values were 24.0 degrees C/13.8 g/m(3) within the nasal valve region and 24.3 degrees C/14.7 g/m(3) anterior to the head of the middle turbinate. In the control group, the corresponding values were 27.0 degrees C/15.5 g/m(3) and 26.7 degrees C/17.0 g/m(3). Temperature and humidity values were significantly lower in the study group (P < .05). The minimal cross-sectional areas and volumes were significantly higher in the study group (P < .05). Conclusions: Nasal complaints in elderly patients are a consequence of lower intranasal air temperature and humidity values combined with relatively enlarged nasal cavities due to involution atrophy of the nasal mucosa.},
  Doi                      = {10.1097/MLG.0b013e3181758174},
  ISSN                     = {0023-852X},
  Keywords                 = {acoustic rhinometry
acoustic rhinometry
active anterior rhinomanometry
active anterior rhinomanometry
aging
aging
air conditioning
air conditioning
humidity profile
humidity profile
impact
impact
intranasal air humidity
intranasal air humidity
Intranasal air temperature
Intranasal air temperature
nasal mucociliary clearance
nasal mucociliary clearance
nasal patency
nasal patency
numerical-simulation
numerical-simulation
resection
resection
septal perforation
septal perforation
sinus surgery
sinus surgery
temperature
temperature
turbinoplasty
turbinoplasty},
  Type                     = {Journal Article},
  Url                      = {http://onlinelibrary.wiley.com/store/10.1097/MLG.0b013e3181758174/asset/5541180827_ftp.pdf?v=1&t=i0df577w&s=096301863c44265a6d51f362c9aee531175335b4}
}

@Article{Lindemann2009a,
  Title                    = {Nasal air conditioning in relation to acoustic rhinometry values},
  Author                   = {Lindemann,Joerg and Tsakiropoulou,Evangelia and Keck,Tilman and Leiacker,Richard and Wiesmiller,Kerstin M.},
  Journal                  = {American Journal of Rhinology \& Allergy},
  Year                     = {2009},

  Month                    = {11},
  Note                     = {Copyright - Copyright OceanSide Publications Nov 2009; Last updated - 2012-03-03},
  Number                   = {6},
  Pages                    = {575-7},
  Volume                   = {23},

  Abstract                 = {Changes of nasal dimensions can influence the air-conditioning capacity of the nose because of alterations of airflow patterns. The goal of this study was to evaluate the correlation between intranasal temperature and humidity values and nasal dimensions, assessed by means of acoustic rhinometry. Eighty healthy volunteers (40 men and 40 women; median age, 51 years; range, 20-84 years) were enrolled in the study. In total, 160 nasal cavities were examined. All volunteers underwent a standardized acoustic rhinometry. Additionally, intranasal air temperature and humidity measurements at defined intranasal detection sites within the anterior nasal segment were performed. There was no statistically significant difference between the right and left side of the nose regarding air temperature, absolute humidity, and acoustic rhinometric values. A negative correlation was established between the rhinometric nasal volumes/minimal cross-sectional areas and air temperature and absolute humidity values at the three intranasal detection sites. According to our results, nasal volumes and cross- sectional areas relevantly influence nasal air conditioning. A healthy nasal cavity with smaller volumes and cross-sectional areas seems to present a more effective air-conditioning function than a too "wide" open nose because of changes in airflow patterns. This observation should be considered as a limitation for overly extensive nasal surgery especially of the turbinates.},
  ISBN                     = {19458924},
  Keywords                 = {Medical Sciences--Otorhinolaryngology; Nasal Cavity -- pathology; Rhinometry, Acoustic; Nasal Cavity -- chemistry; Humans; Adult; Temperature; Humidity; Aged; Middle Aged; Aged, 80 & over; Organ Size; Male; Female; Pulmonary Ventilation; Nasal Cavity -- physiopathology; Air -- analysis},
  Language                 = {English},
  Url                      = {http://search.proquest.com/docview/230823813?accountid=13552}
}

@Article{Lindemann2010,
  Title                    = {Normal aging does not deteriorate nose-related quality of life: Assessment with NOSE and SNOT-20 questionnaires},
  Author                   = {Lindemann, Joerg and Tsakiropoulou, Evangelia and Konstantinidis, Iordanis and Lindemann, Kerstin},
  Journal                  = {Auris Nasus Larynx},
  Year                     = {2010},
  Pages                    = {303-307},
  Volume                   = {37},

  Abstract                 = {Objective
Aging causes changes in nasal morphology and function. This study assesses if the age-related nasal changes are detectable with objective measurements and reflected in two validated quality of life outcome tools: the Nasal Obstruction Symptom Evaluation Scale (NOSE) and the Sino-Nasal Outcome Test (SNOT-20) questionnaires.
Methods
Two study groups were included: the “young� group A (n = 40) with a mean age of 27 years and the “elder� group B (n = 40) with a mean age of 70 years. The subjective nasal complaints and quality of life status were recorded by means of the NOSE and SNOT-20 questionnaires. Objective assessment of intranasal dimensions and nasal airflow in all subjects were performed with the use of acoustic rhinometry and active anterior rhinomanometry respectively.
Results
The values obtained from acoustic rhinometry were significantly higher in the older group compared to the younger, presenting wider nasal airway passages for the elderly. However this was not the case with rhinomanometry values as no significant differences between groups were found. In addition no statistically significant difference was demonstrated in both questionnaires scoring between younger and older subjects.
Conclusion
The outcome of the NOSE and SNOT questionnaires show no deterioration of quality of life in elderly related with changes in nasal function. Acoustic rhinometry confirmed that nasal cavities are becoming larger across the lifespan.},
  Doi                      = {10.1016/j.anl.2009.09.012},
  ISSN                     = {0385-8146},
  Keywords                 = {acoustic rhinometry
Aging nose
Nasal obstruction
nose
Quality of life
Rhinomanometry
SNOT-20},
  Type                     = {Journal Article}
}

@Article{Lindemann2014,
  Title                    = {The intranasal Schirmer test: a preliminary study to quantify nasal secretion},
  Author                   = {Lindemann, Joerg and Tsakiropoulou, Evangelia and Rettinger, Gerhard and Gutter, Caroline and Scheithauer, Marc Oliver and Picavet, Valerie and Sommer, Fabian},
  Journal                  = {European Archives of Oto-Rhino-Laryngology},
  Year                     = {2014},
  Note                     = {C:\Users\sean\AppData\Roaming\Zotero\Zotero\Profiles\16a4oype.default\zotero\storage\UTC2CG4B\Lindemann et al. - 2014 - The intranasal Schirmer test a preliminary study .pdf},
  Pages                    = {1-5},

  Abstract                 = {Adequate secretion of the nasal mucosa is essential for normal nasal physiology. A “dry� nose is a frequent complaint of ENT patients. Measurement of secretion is currently impossible because of the absence of a recognized test. The aim of the present study was to investigate the feasibility of an intranasal Schirmer test in a large number of patients and to define standard values for nasal secretion. The test population comprised 159 healthy, non-smoking volunteers and 30 healthy smoking volunteers. All subjects were examined by nasal endoscopy for anatomic or mucosal disease. A Schirmer test strip was placed on both sides of the mucosa of the anterior nasal septum by anterior rhinoscopy. After 10 min in standardized conditions, the strip was removed and the wetted distance was measured. Active anterior rhinomanometry (ARR) and acoustic rhinometry (AR) were later performed. In the non-smoking group (n = 159), the median wetting distance of the test strip was 10.3 mm (range 3.6–35.0 mm). Age, gender, nasal geometry, and flow (according to ARR and AR) had no significant influence on nasal secretion. The test for normal distribution was negative. In the smoking group (n = 30), the median wetting distance was 8.4 mm (range 2.5–28.0 mm), significantly shorter than the wetting distance in the non-smoking group (p < 0.05). The Schirmer test offers a practical method to quantify mucosal humidification. The test is inexpensive and well tolerated by patients. In healthy people, wetting distances from 6 to 18 mm are considered normal.},
  Doi                      = {10.1007/s00405-014-2988-4},
  ISSN                     = {0937-4477, 1434-4726},
  Keywords                 = {Head and Neck Surgery
Humidification
Nasal air-conditioning
Nasal secretion
Neurosurgery
Otorhinolaryngology
Schirmer test
Warming},
  Type                     = {Journal Article},
  Url                      = {http://download.springer.com/static/pdf/391/art%253A10.1007%252Fs00405-014-2988-4.pdf?auth66=1411539415_7c4e9fb0fd05ae9c472224ecd34272a6&ext=.pdf}
}

@Article{Lindemann2009,
  Title                    = {Influence of the turbinate volumes as measured by magnetic resonance imaging on nasal air conditioning},
  Author                   = {Lindemann,Joerg and Tsakiropoulou,Evangelia and Vital,Victor and Keck,Tilman and Leiacker,Richard and Pauls,Sandra and Wacke,Florian and Wiesmiller,Kerstin M.},
  Journal                  = {American journal of rhinology \& allergy},
  Year                     = {2009},
  Note                     = {Date completed - 2009-07-30; Date created - 2009-06-03; Date revised - 2014-01-13; Last updated - 2014-01-15},
  Number                   = {3},
  Pages                    = {250-254},
  Volume                   = {23},

  Abstract                 = {Changes in nasal airflow caused by varying intranasal volumes and cross-sectional areas affect the contact between air and surrounding mucosa entailing alterations in nasal air conditioning. This study evaluates the correlation between nasal air conditioning and the volumes of the inferior and middle turbinates as measured by magnetic resonance imaging (MRI). Fourteen healthy volunteers were enrolled. Each volunteer had been examined by rhinomanometry, acoustic rhinometry, intranasal air temperature, and humidity measurements at defined intranasal sites as well as MRI of the nasal cavity and the paranasal sinuses. The volumetric data of the turbinates was based on the volumetric software Amira. Comparable results were obtained regarding absolute humidity values and temperature values within the nasal valve area and middle turbinate area for both the right and the left side of the nasal cavity. No statistically significant differences were found in the rhinomanometric values and the acoustic rhinometry results of both sides (p > 0.05). No statistical correlations were found between the volumes of the inferior (mean, 6.1 cm3) and middle turbinate (mean, 1.8 cm3) and the corresponding humidity and temperature values. Additionally, the air temperature and humidity values did not correlate with the rhinometrical endonasal volumes (0-20 mm and 20-50 mm from the nasal entrance). The normal range of volumes of the inferior and middle turbinate does not seem to have a significant impact on intranasal air conditioning in healthy subjects. The exact limits where alterations of the turbinate volume negatively affect nasal air conditioning are still unknown.},
  ISBN                     = {1945-8924, 1945-8924},
  Keywords                 = {Index Medicus; Humans; Adult; Temperature; Humidity; Aged; Middle Aged; Male; Female; Magnetic Resonance Imaging -- methods; Nasal Cavity -- physiology; Turbinates -- anatomy & histology},
  Language                 = {English},
  Url                      = {http://search.proquest.com/docview/67313231?accountid=13552}
}

@Article{Lindemann2009b,
  Title                    = {Influence of the turbinate volumes as measured by magnetic resonance imaging on nasal air conditioning},
  Author                   = {Lindemann, Joerg and Tsakiropoulou, Evangelia and Vital, Victor and Keck, Tilman and Leiacker, Richard and Pauls, Sandra and Wacke, Florian and Wiesmiller, Kerstin M},
  Journal                  = {American journal of rhinology \& allergy},
  Year                     = {2009},
  Number                   = {3},
  Pages                    = {250--254},
  Volume                   = {23},

  Publisher                = {OceanSide Publications, Inc}
}

@Article{Lintermann2013,
  Title                    = {Fluid mechanics based classification of the respiratory efficiency of several nasal cavities },
  Author                   = {Andreas Lintermann and Matthias Meinke and Wolfgang Schröder},
  Journal                  = {Computers in Biology and Medicine },
  Year                     = {2013},
  Number                   = {11},
  Pages                    = {1833 - 1852},
  Volume                   = {43},

  Doi                      = {http://dx.doi.org/10.1016/j.compbiomed.2013.09.003},
  ISSN                     = {0010-4825},
  Keywords                 = {Lattice Boltzmann},
  Url                      = {http://www.sciencedirect.com/science/article/pii/S0010482513002540}
}

@Article{Lippman1990,
  Title                    = {Effects of fibre charcteristics on lung deposition, retention and disease},
  Author                   = {Lippman, M.},
  Journal                  = {Environ. Health Perspect.},
  Year                     = {1990},
  Pages                    = {311-317},
  Volume                   = {88},

  Type                     = {Journal Article}
}

@Article{Lippman1993,
  Title                    = {Deposition, retention and clearance of inhaled particles},
  Author                   = {Lippman, M. and Yeates, D.B. and Albert, R.E.},
  Journal                  = {Journal Ind. Medicine},
  Year                     = {1993},
  Pages                    = {337-362},
  Volume                   = {37},

  Type                     = {Journal Article}
}

@Book{Liseikin1999,
  Title                    = { Grid Generation Methods},
  Author                   = {Liseikin, V.D.},
  Publisher                = {Springer-Verlag},
  Year                     = {1999},

  Address                  = {Berlin},

  Type                     = {Book}
}

@Article{Littman1995,
  Title                    = {Effect of particle diameter, particle density and loading ratio on the effective drag coefficient in steady turbulent gas-solids transport},
  Author                   = {Littman, H. and Morgan, M.H. and Jovanovic, S.D. and Paccione, J.D. and Grbavcic, Z.B. and Vukovic, D.V.},
  Journal                  = {Powder Technology},
  Year                     = {1995},
  Pages                    = {49-56},
  Volume                   = {84},

  Type                     = {Journal Article}
}

@Article{Liu1993,
  Title                    = {Effects of drop drag and break-up on fuel sprays},
  Author                   = {Liu, A. B. and Mather, D. and Reitz, R.D.},
  Journal                  = {SAE Paper No.930072},
  Year                     = {1993},

  Type                     = {Journal Article}
}

@Article{Liu1974,
  Title                    = {Experimental observation of aerosol deposition in turbulent flow},
  Author                   = {Liu, Benjamin Y. H. and Agarwal, Jugal K.},
  Journal                  = {Journal of Aerosol Science},
  Year                     = {1974},
  Note                     = {doi: DOI: 10.1016/0021-8502(74)90046-9},
  Number                   = {2},
  Pages                    = {145-148, IN1-IN2, 149-155},
  Volume                   = {5},

  ISSN                     = {0021-8502},
  Type                     = {Journal Article},
  Url                      = {http://www.sciencedirect.com/science/article/B6V6B-4893W3W-48/2/cf00f8d5c99e37d7fe39d55167cc256b}
}

@Article{Liu1978,
  Title                    = {Combined field and diffusion charging of aerosol particles in the continuum regime},
  Author                   = {Liu, B. Y. H. and Kapadia, A.},
  Journal                  = {Journal of Aerosol Science},
  Year                     = {1978},
  Number                   = {3},
  Pages                    = {227-242},
  Volume                   = {9},

  ISSN                     = {0021-8502},
  Type                     = {Journal Article},
  Url                      = {http://www.sciencedirect.com/science/article/pii/0021850278900459}
}

@Article{Liu2009,
  Title                    = {LES numerical simulation of cavitation Bubble shedding on ALE 25 and ALE 15 hydrofoils},
  Author                   = {Liu, De-min and Liu, Shu-hong and Wu, Yu-lin and Xu, Hong-yuan},
  Journal                  = {Journal of Hydrodynamics, Ser. B},
  Year                     = {2009},
  Number                   = {6},
  Pages                    = {807-813},
  Volume                   = {21},

  Abstract                 = {A cavitation calculation scheme is developed and applied to ALE 15 and ALE 25 hydrofoils, based on the Bubble Two-phase Flow (BTF) cavity model with a Large Eddy Simulation (LES) methodology. The Navier-Stokes equations including cavitation bubble clusters are solved through the finite-volume approach with a time-marching scheme. Simulations are carried out in a 3-D field with a hydrofoil ALE 15 or ALE 25 at an angle of attack of 8° and cavitation number [sigma] = 2.3 with a 2 × 106, meshing system. With the time-marching, the cavitation bubble gradually grows to a steady lump shape and then produces an irregular small bubble behind the main cavitation bubble, finally shedding from the leading edge of the cloud cavitation structure. The calculated results including velocity field and pressure field are consistent with experiment data at the same Reynolds number and cavitation number. The vortex and reverse flow are observed on the hydrofoil surface.},
  ISSN                     = {1001-6058},
  Keywords                 = {cavitation
ALE hydrofoil
Large Eddy Simulation (LES)},
  Type                     = {Journal Article},
  Url                      = {http://www.sciencedirect.com/science/article/B8CX5-4Y29MXM-B/2/f77bc00a99d5d3581e72abc6175e7026}
}

@Misc{Liu2000,
  Title                    = {Science and Engineering of Droplets - Fundamentals and Applications},

  Author                   = {Liu, Huimin},
  Year                     = {2000},

  ISBN                     = {978-0-8155-1436-7},
  Publisher                = {William Andrew Publishing/Noyes},
  Type                     = {Generic},
  Url                      = {http://www.knovel.com/web/portal/browse/display?_EXT_KNOVEL_DISPLAY_bookid=445}
}

@Article{Liu2006,
  Title                    = {Thermal lattice-BGK model based on large-eddy simulation of turbulent natural convection due to internal heat generation},
  Author                   = {Liu, Hongjuan and Zou, Chun and Shi, Baochang and Tian, Zhiwei and Zhang, Liqi and Zheng, Chuguang},
  Journal                  = {International Journal of Heat and Mass Transfer},
  Year                     = {2006},
  Number                   = {23-24},
  Pages                    = {4672-4680},
  Volume                   = {49},

  Abstract                 = {To simulate turbulent convection at high Rayleigh number (Ra), we propose a new thermal lattice-BGK (LBGK) model based on large eddy simulation (LES). Two-dimensional numerical simulations of natural convection with internal heat generation in a square cavity were performed at Ra from 106 to 1013 with Prandtl numbers (Pr) at 0.25 and 0.60. Simulation results indicate that our model is fit to simulate high Ra flow for its better numerical stability. At Ra = 1013, a global turbulent has occurred. With a further increase in Ra, the flow will arrive in a fully turbulence regime. The Nusselt-Rayleigh relationship is also discussed.},
  ISSN                     = {0017-9310},
  Keywords                 = {Thermal LBGK model
LES
Turbulent convection
High Ra},
  Type                     = {Journal Article},
  Url                      = {http://www.sciencedirect.com/science/article/B6V3H-4K606RJ-3/2/3dd5bb0e1b381e99a3aaff4b499d814a}
}

@Article{Liu1999,
  Title                    = {Dynamic simulation of the centripetal packing of mono-sized spheres},
  Author                   = {Liu, L. F. and Zhang, Z. P. and Yu, A. B.},
  Journal                  = {Physica A: Statistical and Theoretical Physics},
  Year                     = {1999},
  Note                     = {doi: DOI: 10.1016/S0378-4371(99)00106-5},
  Number                   = {3-4},
  Pages                    = {433-453},
  Volume                   = {268},

  ISSN                     = {0378-4371},
  Keywords                 = {Packing
Packing structure
Coordination number
Radial distribution function},
  Type                     = {Journal Article},
  Url                      = {http://www.sciencedirect.com/science/article/B6TVG-3WWMYX3-B/2/a8fbb9ae9115f4759bb5c4fa6b13e835}
}

@Article{Liu2010,
  Title                    = {Evaluation of droplet velocity and size from nasal spray devices using phase Doppler anemometry (PDA)},
  Author                   = {Liu, Xiaofei and Doub, William H. and Guo, C.},
  Journal                  = {International Journal of Pharmaceutics},
  Year                     = {2010},
  Pages                    = {82-87},
  Volume                   = {388},

  Type                     = {Journal Article}
}

@Article{Liu2014,
  Title                    = {A fast and simple numerical model for a deeply buried underground tunnel in heating and cooling applications},
  Author                   = {Liu, Xichen and Xiao, Yimin and Inthavong, Kiao and Tu, Jiyuan},
  Journal                  = {Applied Thermal Engineering},
  Year                     = {2014},
  Number                   = {2},
  Pages                    = {545-552},
  Volume                   = {62},

  Doi                      = {http://dx.doi.org/10.1016/j.applthermaleng.2013.10.017},
  ISSN                     = {1359-4311},
  Keywords                 = {Underground air tunnel
Transportation tunnel
Geothermal energy
Numerical
Heat transfer},
  Type                     = {Journal Article},
  Url                      = {http://www.sciencedirect.com/science/article/pii/S1359431113007266}
}

@Article{Liu2009a,
  Title                    = {Modelling of dense gas-particle flow in a circulating fluidized bed by Distinct Cluster Method (DCM)},
  Author                   = {Liu, Xiangjun and Xu, Xuchang},
  Journal                  = {Powder Technology},
  Year                     = {2009},
  Number                   = {3},
  Pages                    = {235-244},
  Volume                   = {195},

  Abstract                 = {Computational Fluid Dynamics (CFD) is a powerful tool to study the dense gas-solid flow in a circulating fluidized bed. Most of the existing methods focus on the microscopic properties of individual particle. Therefore, the simulation scale is significantly limited by the huge number of individual particles, and so far the numbers of particles in most of the reported simulations are less than 105. The hydrodynamics behaviour of particle clustering in a dense gas-solid two-phase flow has been verified by several experimental results. The Distinct Cluster Method (DCM) was proposed in this paper by studying the macroscopic particle clustering behaviour, and comprehensive models for cluster motion, collision, break-up, and coalescence have been well developed. We model the dense two-phase flow field as gas-rich lean phase and solid-rich cluster phase. The particle cluster is directly treated as one discrete phase. The gas turbulent flow is calculated by Eulerian approach, and the particle behaviour is studied by Lagrangian approach. Using the proposed method, a three-dimension dense gas-particle two-phase flow field in a circulating fluidized bed with square-cross-section, with particle number up to 7.162 × 107 are able to be numerically studied, on which few results have been reported. Details on instantaneous and time-averaged distributions are obtained. Developing process of non-uniform particle distribution is visualized. These results are in agreements with experimental observations, which justified the feasibility of using the DCM method to model and simulate dense gas-solid flow in a circulating fluidized bed with large number of particle numbers.},
  Doi                      = {10.1016/j.powtec.2009.06.007},
  ISSN                     = {0032-5910},
  Keywords                 = {Dense gas-particle two-phase flow
Particle cluster
Distinct Cluster Method},
  Type                     = {Journal Article},
  Url                      = {http://www.sciencedirect.com/science/article/pii/S0032591009003805}
}

@Article{Liu2007,
  Title                    = {Identification of Appropriate CFD Models for Simulating Aerosol Particle and Droplet Indoor Transport},
  Author                   = {Liu, Xiang and Zhai, Zhiqiang},
  Journal                  = {Indoor and Built Environment},
  Year                     = {2007},
  Number                   = {4},
  Pages                    = {322-330},
  Volume                   = {16},

  Abstract                 = {Computational fluid dynamics (CFD) has been widely used to predict indoor particle and droplet transport and dispersion. CFD solves simplified conservative equations that describe the major characteristics of particle and droplet indoor movement, along with the flow governing equations. This paper reviews the principles of three prevalent CFD models for indoor particle and droplet simulation: the lazy particle model, isothermal particle model and vaporizing droplet model, with a focus on the disparities between these models. The study verifies that different particle and droplet models provide distinct simulation results in which size of particle and droplet is a critical factor. To justify proper application of these models for particles and droplets with different sizes, the paper theoretically analyzes the Lagrangian transport equations for particle and droplet and identifies two crucial time numbers -- particle momentum response time and evaporation lifetime. Upon these numbers, two new indices have been introduced -- Stokes number and evaporation effectiveness number, which can be used as simple criteria to guide the model selection. The case studies confirm the value of the indices and provide the rules of thumb for determining appropriate CFD models for particle and droplet indoor transport under typical room conditions.},
  Doi                      = {10.1177/1420326x06079890},
  Type                     = {Journal Article},
  Url                      = {http://ibe.sagepub.com/cgi/content/abstract/16/4/322}
}

@Article{Liu2009b,
  Title                    = {Creation of a standardized geometry of the human nasal cavity},
  Author                   = {Liu, Y. and Johnson, M. R. and Matida, E. A. and Kherani, S. and Marsan, J.},
  Journal                  = {Journal of Applied Physiology},
  Year                     = {2009},
  Number                   = {3},
  Pages                    = {784-795},
  Volume                   = {106},

  Abstract                 = {A novel, standardized geometry of the human nasal cavity was created by aligning and processing 30 sets of computed tomography (CT) scans of nasal airways of healthy subjects. Digital three-dimensional (3-D) geometries of the 60 single human nasal cavities (30 right and 30 mirrored left cavities) were generated from the CT scans and measurements of physical parameters of each single nasal cavity were performed. A methodology was developed to scale, orient, and align the nasal geometries, after which 2-D digital coronal cross-sectional slices were generated. With the use of an innovative image processing algorithm, median cross-sectional geometries were created to match median physical parameters while retaining the unique geometric features of the human nasal cavity. From these idealized 2-D images, an original 3-D standardized median human nasal cavity was created. This new standardized geometry was compared against the original geometries of all subjects as well as limited existing data from the literature. The new model has potential for use as a geometric standard in future experimental and numerical studies of deposition of inhaled aerosols, as well as for use as a reference during diagnosis of unhealthy patients. The specific procedure developed could also be applied to build standard nasal geometries for different identifiable groups within the larger population.},
  Doi                      = {10.1152/japplphysiol.90376.2008},
  Type                     = {Journal Article},
  Url                      = {http://jap.physiology.org/cgi/content/abstract/106/3/784}
}

@Article{Liu2010a,
  Title                    = {Experimental measurements and computational modeling of aerosol deposition in the Carleton-Civic standardized human nasal cavity},
  Author                   = {Liu, Yuan and Matida, Edgar A. and Johnson, Matthew R.},
  Journal                  = {Journal of Aerosol Science},
  Year                     = {2010},
  Number                   = {6},
  Pages                    = {569-586},
  Volume                   = {41},

  ISSN                     = {0021-8502},
  Keywords                 = {Aerosol deposition
Standardized human nasal cavity
Nasal airway
Deposition measurements
Nasal deposition},
  Type                     = {Journal Article},
  Url                      = {http://www.sciencedirect.com/science/article/B6V6B-4YJCKX7-1/2/f614c87d2f429ac212cb41cbd52d6521}
}

@Article{Liu2010c,
  Title                    = {Experimental measurements and computational modeling of aerosol deposition in the Carleton-Civic standardized human nasal cavity },
  Author                   = {Yuan Liu and Edgar A. Matida and Matthew R. Johnson},
  Journal                  = {Journal of Aerosol Science },
  Year                     = {2010},
  Number                   = {6},
  Pages                    = {569 - 586},
  Volume                   = {41},

  Doi                      = {http://dx.doi.org/10.1016/j.jaerosci.2010.02.014},
  ISSN                     = {0021-8502},
  Keywords                 = {Aerosol deposition},
  Url                      = {http://www.sciencedirect.com/science/article/pii/S0021850210000479}
}

@Article{Liu2007a,
  Title                    = {Numerical simulation of aerosol deposition in a 3-D human nasal cavity using RANS, RANS/EIM, and LES},
  Author                   = {Liu, Y. and Matida, E.A. and Junjie, G.U. and Johnson, M.R.},
  Journal                  = {Journal of Aerosol Science},
  Year                     = {2007},
  Number                   = {7},
  Pages                    = {683-700},
  Volume                   = {38},

  Type                     = {Journal Article}
}

@Article{Liu2002,
  Title                    = {Modeling the bifurcating flow in a human lung airway},
  Author                   = {Liu, Y. and So, R.M.C. and Zhang, C.H.},
  Journal                  = {Journal of Biomechanics},
  Year                     = {2002},
  Pages                    = {465-473},
  Volume                   = {35},

  Type                     = {Journal Article}
}

@Article{Liu2003,
  Title                    = {Modeling the bifurcating flow in an asymmetric human lung airway},
  Author                   = {Liu, Y. and So, R. M. C. and Zhang, C. H.},
  Journal                  = {Journal of Biomechanics},
  Year                     = {2003},
  Note                     = {doi: DOI: 10.1016/S0021-9290(03)00064-2},
  Number                   = {7},
  Pages                    = {951-959},
  Volume                   = {36},

  ISSN                     = {0021-9290},
  Keywords                 = {Asymmetric bifurcation flow
Human lung
Three-dimensional modeling},
  Type                     = {Journal Article},
  Url                      = {http://www.sciencedirect.com/science/article/B6T82-4870VTC-1/2/2c97ae20110e1affb79654e4cb2b455f}
}

@Article{Liu2008,
  Title                    = {Linear and nonlinear model large-eddy simulations of a plane jet},
  Author                   = {Liu, Y. and Tucker, P. G. and Kerr, R. M.},
  Journal                  = {Computers \&amp; Fluids},
  Year                     = {2008},
  Number                   = {4},
  Pages                    = {439-449},
  Volume                   = {37},

  Doi                      = {10.1016/j.compfluid.2007.02.005},
  ISSN                     = {0045-7930},
  Type                     = {Journal Article},
  Url                      = {http://www.sciencedirect.com/science/article/pii/S0045793007000278}
}

@InBook{Liu2010b,
  Title                    = {Modeling and Simulation of Human Upper Airway},
  Author                   = {Liu, Zishun and Xu, Xiangguo and Lim, Franco Fang Jeng and Luo, Xiaoyu and Hirtum, Annemie and Hill, N. A.},
  Editor                   = {Lim, C. T. and Goh, J. C. H.},
  Pages                    = {686-689},
  Publisher                = {Springer Berlin Heidelberg},
  Year                     = {2010},
  Series                   = {IFMBE Proceedings},
  Type                     = {Book Section},
  Volume                   = {31},

  Booktitle                = {6th World Congress of Biomechanics (WCB 2010). August 1-6, 2010 Singapore},
  Doi                      = {10.1007/978-3-642-14515-5_175},
  ISBN                     = {978-3-642-14515-5},
  Keywords                 = {Engineering},
  Url                      = {http://dx.doi.org/10.1007/978-3-642-14515-5_175}
}

@Article{Liu1991,
  Title                    = {High resolution measurement of turbulent structure in a channel with particle image velocimetry},
  Author                   = {Liu, Z. C. and Landreth, C. C. and Adrian, R. J. and Hanratty, T. J.},
  Journal                  = {Experiments in Fluids},
  Year                     = {1991},
  Note                     = {10.1007/BF00190246},
  Number                   = {6},
  Pages                    = {301-312},
  Volume                   = {10},

  Abstract                 = {High resolution particle image velocimetry is used to measure the turbulent velocity field for fully developed flow (Re = 2,872) in an enclosed channel. Photographs of particle displacement are obtained in a plane that is parallel to the flow and perpendicular to the walls. These are analyzed to give simultaneous measurements of two components of the velocity at more than 10,000 points. Maps of velocity vectors, spanwise vorticity and Reynolds stress reveal structural aspects of the turbulence. In particular, internal shear layers are observed, in agreement with predictions of direct numerical simulation. Ensemble-averaging of a number of photographs yields statistical properties of the velocity in good agreement with laser-Doppler velocimeter measurements, and with direct numerical simulations.},
  Type                     = {Journal Article},
  Url                      = {http://dx.doi.org/10.1007/BF00190246}
}

@Article{Liu2007b,
  Title                    = {Snoring source identification and snoring noise prediction},
  Author                   = {Liu, Z. S. and Luo, X. Y. and Lee, H. P. and Lu, C.},
  Journal                  = {Journal of Biomechanics},
  Year                     = {2007},
  Number                   = {4},
  Pages                    = {861-870},
  Volume                   = {40},

  Abstract                 = {This paper investigates the snoring mechanism of humans by applying the concept of structural intensity to a three-dimensional (3D) finite element model of a human head, which includes: the upper part of the head, neck, soft palate, hard palate, tongue, nasal cavity and the surrounding walls of the pharynx. Results show that for 20, 40 and 60 Hz pressure loads, tissue vibration is mainly in the areas of the soft palate, the tongue and the nasal cavity. For predicting the snoring noise level, a 3D boundary element cavity model of the upper airway in the nasal cavity is generated. The snoring noise level is predicted for a prescribed airflow loading, and its range agrees with published measurements. These models may be further developed to study the various snoring mechanisms for different groups of patients.},
  ISSN                     = {0021-9290},
  Keywords                 = {Finite element method
Boundary element method
Vibration
Snoring noise
Soft palate
Tongue
Nasal cavity
Human head model
Structural intensity},
  Type                     = {Journal Article},
  Url                      = {http://www.sciencedirect.com/science/article/B6T82-4K427XC-1/2/296a7abd5b6b6d0290be2e59df65b79a}
}

@Article{Lo1985,
  Title                    = {A new mesh generation scheme for arbitrary planar domains},
  Author                   = {Lo, S.H.},
  Journal                  = {International Journal Numerical Methods Engineering},
  Year                     = {1985},
  Pages                    = {1403=1426},
  Volume                   = {21},

  Type                     = {Journal Article}
}

@Article{Longest,
  Title                    = {CFD simulations of enhanced condensational growth (ECG) applied to respiratory drug delivery with comparisons to in vitro data},
  Author                   = {Longest, P.W. and Hindle, Michael},
  Journal                  = {Journal of Aerosol Science},
  Volume                   = {In Press, Corrected Proof},

  Abstract                 = {Enhanced condensational growth (ECG) is a newly proposed concept for respiratory drug delivery in which a submicrometer aerosol is inhaled in combination with saturated or supersaturated water vapor. The initially small aerosol size provides for very low extrathoracic deposition, whereas condensation onto droplets in vivo results in size increase and improved lung retention. The objective of this study was to develop and evaluate a CFD model of ECG in a simple tubular geometry with direct comparisons to in vitro results. The length (29 cm) and diameter (2 cm) of the tubular geometry were representative of respiratory airways of an adult from the mouth to the first tracheobronchial bifurcation. At the model inlet, separate streams of humidified air (25, 30, and 39 °C) and submicrometer aerosol droplets with mass median aerodynamic diameters (MMADs) of 150, 560, and 900 nm were combined. The effects of condensation and droplet growth on water vapor concentrations and temperatures in the continuous phase (i.e., two-way coupling) were also considered. For an inlet saturated air temperature of 39 °C, the two-way coupled numerical (and in vitro) final aerosol MMADs for initial sizes of 150, 560, and 900 nm were 1.75 [mu]m (vs. 1.23 [mu]m), 2.58 [mu]m (vs. 2.66 [mu]m), and 2.65 [mu]m (vs. 2.63 [mu]m), respectively. By including the effects of two-way coupling in the model, agreements with the in vitro results were significantly improved compared with a one-way coupled assumption. Results indicated that both mass and thermal two-way coupling effects were important in the ECG process. Considering the initial aerosol sizes of 560 and 900 nm, the final sizes were most influenced by inlet saturated air temperature and aerosol number concentration and were not largely influenced by initial size. Considering the growth of submicrometer aerosols to above 2 [mu]m at realistic number concentrations, ECG may be an effective respiratory drug delivery approach for minimizing mouth-throat deposition and maximizing aerosol retention in a safe and simple manner. However, future studies are needed to explore effects of in vivo boundary conditions, more realistic respiratory geometries, and transient breathing.},
  ISSN                     = {0021-8502},
  Keywords                 = {Hygroscopic aerosol growth
Targeted drug delivery
Reduced extrathoracic deposition
Respiratory drug delivery
Aerosol modeling},
  Type                     = {Journal Article},
  Url                      = {http://www.sciencedirect.com/science/article/B6V6B-4YWC178-3/2/07bad52d9847476e6179d058ee00c285}
}

@Article{Longest2010,
  Title                    = {CFD simulations of enhanced condensational growth (ECG) applied to respiratory drug delivery with comparisons to in vitro data},
  Author                   = {Longest, P.W. and Hindle, Michael},
  Journal                  = {Journal of Aerosol Science},
  Year                     = {2010},
  Number                   = {8},
  Pages                    = {805-820},
  Volume                   = {41},

  Abstract                 = {Enhanced condensational growth (ECG) is a newly proposed concept for respiratory drug delivery in which a submicrometer aerosol is inhaled in combination with saturated or supersaturated water vapor. The initially small aerosol size provides for very low extrathoracic deposition, whereas condensation onto droplets in vivo results in size increase and improved lung retention. The objective of this study was to develop and evaluate a CFD model of ECG in a simple tubular geometry with direct comparisons to in vitro results. The length (29 cm) and diameter (2 cm) of the tubular geometry were representative of respiratory airways of an adult from the mouth to the first tracheobronchial bifurcation. At the model inlet, separate streams of humidified air (25, 30, and 39 °C) and submicrometer aerosol droplets with mass median aerodynamic diameters (MMADs) of 150, 560, and 900 nm were combined. The effects of condensation and droplet growth on water vapor concentrations and temperatures in the continuous phase (i.e., two-way coupling) were also considered. For an inlet saturated air temperature of 39 °C, the two-way coupled numerical (and in vitro) final aerosol MMADs for initial sizes of 150, 560, and 900 nm were 1.75 [mu]m (vs. 1.23 [mu]m), 2.58 [mu]m (vs. 2.66 [mu]m), and 2.65 [mu]m (vs. 2.63 [mu]m), respectively. By including the effects of two-way coupling in the model, agreements with the in vitro results were significantly improved compared with a one-way coupled assumption. Results indicated that both mass and thermal two-way coupling effects were important in the ECG process. Considering the initial aerosol sizes of 560 and 900 nm, the final sizes were most influenced by inlet saturated air temperature and aerosol number concentration and were not largely influenced by initial size. Considering the growth of submicrometer aerosols to above 2 [mu]m at realistic number concentrations, ECG may be an effective respiratory drug delivery approach for minimizing mouth-throat deposition and maximizing aerosol retention in a safe and simple manner. However, future studies are needed to explore effects of in vivo boundary conditions, more realistic respiratory geometries, and transient breathing.},
  Doi                      = {10.1016/j.jaerosci.2010.04.006},
  ISSN                     = {0021-8502},
  Keywords                 = {Hygroscopic aerosol growth
Targeted drug delivery
Reduced extrathoracic deposition
Respiratory drug delivery
Aerosol modeling},
  Type                     = {Journal Article},
  Url                      = {http://www.sciencedirect.com/science/article/pii/S0021850210001011}
}

@Article{Longest2007,
  Title                    = {Validating CFD predictions of respiratory aerosol deposition: Effects of upstream transition and turbulence},
  Author                   = {Longest, P.W. and Vinchurkar, Samir},
  Journal                  = {Journal of Biomechanics},
  Year                     = {2007},
  Note                     = {doi: DOI: 10.1016/j.jbiomech.2006.01.006},
  Number                   = {2},
  Pages                    = {305-316},
  Volume                   = {40},

  ISSN                     = {0021-9290},
  Keywords                 = {Respiratory particle dynamics
Particle deposition
Respiratory dosimetry
Bifurcation models
Lagrangian particle tracking},
  Type                     = {Journal Article},
  Url                      = {http://www.sciencedirect.com/science/article/B6T82-4JGBF6B-1/2/12aa66e82440af986ff174243f42fe0a}
}

@Article{Longest2006,
  Title                    = {Transport and deposition of respiratory aerosols in models of childhood asthma},
  Author                   = {Longest, P.W. and Vinchurkara, S. and Martonen, T.},
  Journal                  = {Aerosol Science},
  Year                     = {2006},
  Pages                    = {1234-1257},
  Volume                   = {37},

  Type                     = {Journal Article}
}

@Article{Longest2007a,
  Title                    = {Effectiveness of direct Lagrangian tracking models for simulating nanoparticle deposition in the upper airways},
  Author                   = {Longest, P.W. and Xi, J.},
  Journal                  = {Aerosol Science and Technology},
  Year                     = {2007},
  Number                   = {4},
  Pages                    = {380-397},
  Volume                   = {41},

  Type                     = {Journal Article}
}

@Article{Longest2009,
  Title                    = {Evaluation of the Respimat soft mist inhaler using a concurrent CFD and in vitro approach},
  Author                   = {Longest, P. W. and Hindle, M.},
  Journal                  = {Journal of Aerosol Medicine},
  Year                     = {2009},
  Pages                    = {99-112},
  Volume                   = {22},

  Type                     = {Journal Article}
}

@Article{Longest2007b,
  Title                    = {Numerical Simulations of Capillary Aerosol Generation: CFD Model Development and Comparisons with Experimental Data},
  Author                   = {Longest, P. Worth and Hindle, Michael and Choudhuri, Suparna Das and Byron, Peter R.},
  Journal                  = {Aerosol Science and Technology},
  Year                     = {2007},
  Number                   = {10},
  Pages                    = {952-973},
  Volume                   = {41},

  ISSN                     = {0278-6826},
  Type                     = {Journal Article},
  Url                      = {http://www.informaworld.com/10.1080/02786820701607027}
}

@Article{Longest2008,
  Title                    = {Comparison of ambient and spray aerosol deposition in a standard induction port and more realistic mouth-throat geometry},
  Author                   = {Longest, P. Worth and Hindle, Michael and Das Choudhuri, Suparna and Xi, Jinxiang},
  Journal                  = {Journal of Aerosol Science},
  Year                     = {2008},
  Number                   = {7},
  Pages                    = {572-591},
  Volume                   = {39},

  Abstract                 = {For inhalation aerosols, it is well known that spray momentum and geometry characteristics can significantly influence deposition in the mouth-throat (MT) region. However, little is know about the quantitative influence of spray momentum on aerosol transport and deposition. The objective of this study was to evaluate the effect of spray momentum on deposition in a standard induction port (IP) and a representative MT geometry using capillary-generated aerosols. Capillary aerosol generation (CAG) was selected as a model spray aerosol system that has not been previously tested in a realistic throat geometry. To evaluate the effects of spray momentum, the transport and deposition characteristics of transient capillary-generated aerosols were compared with ambient particles of the same size inhaled at a steady flow rate of 30 L/min. To evaluate the influence of geometry, aerosols were considered in a standard IP and a more realistic MT model. A previously tested CFD model was employed to simulate aerosol transport and deposition for ambient and CAG spray aerosols in both the IP and MT geometries. Considering the capillary-generated spray, good agreement was observed for the deposition of drug mass between the in vitro experiments (IP--15.3%, MT--19.4%) and CFD model predictions without droplet evaporation (IP--14.7%, MT--20.8%). In all cases considered, deposition was increased for spray vs. ambient aerosols and in the MT geometry vs. the IP model. Based on CFD results for a representative polydisperse aerosol distribution, the deposition of ambient particles was highly sensitive to the geometry considered, with 2.9 times more deposition in the MT compared to the IP model. In contrast, deposition was less influenced by the geometry for a CAG spray aerosol, with only a 25-40% deposition increase in the MT. As a result, use of the simple IP model may provide a reasonable approximation of total MT deposition for systems with high spray momentum. However, the IP model may be less useful for evaluating the total MT deposition in systems with reduced spray momentum effects.},
  ISSN                     = {0021-8502},
  Keywords                 = {Capillary aerosol generation
Aerosol transport and deposition
USP induction port
Mouth-throat geometry
Respiratory aerosol dynamics
Respiratory drug delivery
Spray momentum effects},
  Type                     = {Journal Article},
  Url                      = {http://www.sciencedirect.com/science/article/B6V6B-4S7JG21-1/2/824ae5f6310ca865901cd08954405dfc}
}

@Article{Longest2011,
  Title                    = {In silico models of aerosol delivery to the respiratory tract -- Development and applications},
  Author                   = {Longest, P. Worth and Holbrook, Landon T.},
  Journal                  = {Advanced Drug Delivery Reviews},
  Year                     = {2011},
  Volume                   = {In Press, Corrected Proof},

  Abstract                 = {This review discusses the application of computational models to simulate the transport and deposition of inhaled pharmaceutical aerosols from the site of particle or droplet formation to deposition within the respiratory tract. Traditional one-dimensional (1-D) whole-lung models are discussed briefly followed by a more in-depth review of three-dimensional (3-D) computational fluid dynamics (CFD) simulations. The review of CFD models is organized into sections covering transport and deposition within the inhaler device, the extrathoracic (oral and nasal) region, conducting airways, and alveolar space. For each section, a general review of significant contributions and advancements in the area of simulating pharmaceutical aerosols is provided followed by a more in-depth application or case study that highlights the challenges, utility, and benefits of in silico models. Specific applications presented include the optimization of an existing spray inhaler, development of charge-targeted delivery, specification of conditions for optimal nasal delivery, analysis of a new condensational delivery approach, and an evaluation of targeted delivery using magnetic aerosols. The review concludes with recommendations on the need for more refined model validations, use of a concurrent experimental and CFD approach for developing aerosol delivery systems, and development of a stochastic individual path (SIP) model of aerosol transport and deposition throughout the respiratory tract.},
  Doi                      = {10.1016/j.addr.2011.05.009},
  ISSN                     = {0169-409X},
  Keywords                 = {Respiratory drug delivery
Pharmaceutical aerosols
Aerosol deposition
Lung models
1-D models
CFD models
Spray momentum
Targeted aerosol delivery
Validation of deposition simulations
Stochastic individual path (SIP) model},
  Type                     = {Journal Article},
  Url                      = {http://www.sciencedirect.com/science/article/pii/S0169409X11001244}
}

@Article{Longest2004,
  Title                    = {Interacting effects of uniform flow, plane shear, and near-wall proximity on the heat and mass transfer of respiratory aerosols},
  Author                   = {Longest, P. Worth and Kleinstreuer, Clement},
  Journal                  = {International Journal of Heat and Mass Transfer},
  Year                     = {2004},
  Number                   = {22},
  Pages                    = {4745-4759},
  Volume                   = {47},

  Abstract                 = {Individual and interacting effects of uniform flow, plane shear, and near-wall proximity on spherical droplet heat and mass transfer have been assessed for low Reynolds number conditions beyond the creeping flow regime. Validated resolved volume simulations were used to compute heat and mass transfer surface gradients of two-dimensional axisymmetric droplets and three-dimensional spherical droplets near planar wall boundaries for conditions consistent with inhalable aerosols (5 [less-than-or-equals, slant] d [less-than-or-equals, slant] 300 [mu]m) in the upper respiratory tract. Results indicate that planar shear significantly impacts droplet heat and mass transfer for shear-based Reynolds numbers greater than 1, which occur for near-wall respiratory aerosols with diameters in excess of 50 [mu]m. Wall proximity is shown to significantly enhance heat and mass transfer due to conduction and diffusion at separation distances less than five particle diameters and for small Reynolds numbers. For the Reynolds number conditions of interest, significant non-linear effects arise due to the concurrent interaction of uniform flow and shear such that linear superposition of Sherwood or Nusselt number terms is not allowable. Based on the validated numeric simulations, multivariable Sherwood and Nusselt number correlations are provided to account for individual flow characteristics and concurrent non-linear interactions of uniform flow, planar shear, and near-wall proximity. These heat and mass transfer correlations can be applied to effectively compute condensation and evaporation rates of potentially toxic or therapeutic aerosols in the upper respiratory tract, where non-uniform flow and wall proximity are expected to significantly affect droplet transport, deposition, and vapor formation.},
  ISSN                     = {0017-9310},
  Keywords                 = {Droplet heat and mass transfer
Respiratory aerosols
Near-wall droplet evaporation
Droplet Nusselt and Sherwood number correlations},
  Type                     = {Journal Article},
  Url                      = {http://www.sciencedirect.com/science/article/B6V3H-4D10JNX-3/2/b6093e10cf92ef2457e2acfaa7ac19f9}
}

@Article{Longest2004a,
  Title                    = {Efficient computation of micro-particle dynamics including wall effects},
  Author                   = {Longest, P. Worth and Kleinstreuer, Clement and Buchanan, John R.},
  Journal                  = {Computers \& Fluids},
  Year                     = {2004},
  Number                   = {4},
  Pages                    = {577-601},
  Volume                   = {33},

  Abstract                 = {This study describes an effective method for one-way coupled Eulerian-Lagrangian simulations of spherical micro-size particles, including particle-wall interactions and the quantification of near-wall stasis at possibly elevated concentrations. The focus is on particle-hemodynamics simulations where particle suspensions are composed of critical blood cells, such as monocytes, and the carrier fluid is non-Newtonian. Issues regarding adaptive time-step integration of the particle motion equation, relevant point-force model terms, and adaptation of surface-induced particle forces to arbitrary three-dimensional geometries are outlined. By comparison to available experimental trajectories, it is shown that fluid-element pathlines may be used to simulate non-interacting blood particles removed from wall boundaries under dilute transient conditions. However, when particle-wall interactions are significant, an extended form of the particle trajectory equation is required which includes terms for Stokes drag, near-wall drag modifications, or lubrication forces, pressure gradients, and near-wall particle lift. Still, additional physical and/or biochemical wall forces in the nano-meter range cannot be readily calculated; hence the near-wall residence time (NWRT) model indicating the probability of blood particle deposition is presented. The theory is applied to a virtual model of a femoral bypass end-to-side anastomosis, where profiles of the Lagrangian-based NWRT parameter are illustrated and convergence is verified. In order to effectively compute the large number of particle trajectories required to resolve regions of particle stasis, the proposed particle tracking algorithm stores all transient velocity field solution data on a shared memory architecture (SGI Origin 2400) and computes particle trajectories using an adaptive parallel approach. Compared to commercially available particle tracking packages, the algorithm presented is capable of reducing computational time by an order of magnitude for typical transient one-way coupled blood particle simulations in complex cyclical flow domains.},
  ISSN                     = {0045-7930},
  Type                     = {Journal Article},
  Url                      = {http://www.sciencedirect.com/science/article/B6V26-49N96GW-1/2/2d3ab11649bab94f47d02f31b4c570be}
}

@Article{Longest2008a,
  Title                    = {Numerical and experimental deposition of fine respiratory aerosols: Development of a two-phase drift flux model with near-wall velocity corrections},
  Author                   = {Longest, P. Worth and Oldham, Michael J.},
  Journal                  = {Journal of Aerosol Science},
  Year                     = {2008},
  Number                   = {1},
  Pages                    = {48-70},
  Volume                   = {39},

  Abstract                 = {Simulations of fine aerosols (100 nm ) in the respiratory tract are challenging due to low particle deposition rates and the combined effects of diffusional and inertial deposition mechanisms. Furthermore, validations of fine respiratory aerosol deposition are difficult due to a lack of localized experimental data. The objective of this study is to develop and test a continuous two-phase model for simulating the regional and local deposition of dilute fine aerosols in an idealized double bifurcation segment of the respiratory tract. To evaluate the developed transport model, novel in vitro deposition results for 400 nm particles have been determined in a double bifurcation geometry of respiratory generations G3-G5. In addition, previously reported local deposition characteristics for aerosols have also been considered. Computational two-phase models that have been evaluated include a standard chemical species (CS) mass fraction approximation, the drift flux (DF) approach to account for finite particle inertia, and two novel extensions of the DF model to correct for near-wall particle velocity. The first velocity correction model (DF-VC1) applies a continuous field solution for particle slip at the wall surface. As an alternative, a sub-grid near-wall Lagrangian solution has been proposed as the DF-VC2 model. Localized experimental results for the deposition of 400 nm particles indicated elevated deposition contours ranging from 1% to 5% of total deposition at the first bifurcation and 0.1-1% at the second. Of the computational models tested, the DF-VC2 method provided the best match to experimental deposition values on a regional and highly localized basis. Specifically, the DF-VC2 model matched total experimental deposition results to within 10% for both 400 nm and particles. Considering the local deposition of fine aerosols, the DF-VC2 model matched the experimentally determined elevated contours at the first and second bifurcations for both 400 nm and particles. In conclusion, a DF particle transport model with near-wall velocity corrections appears to provide a highly effective solution for the deposition of fine respiratory aerosols.},
  ISSN                     = {0021-8502},
  Keywords                 = {Submicrometer particle deposition
Eulerian mixture model
Respiratory dosimetry
Microdosimetry
Respiratory drug delivery
Deposition enhancement factor
Respiratory particle dynamics},
  Type                     = {Journal Article},
  Url                      = {http://www.sciencedirect.com/science/article/B6V6B-4PW05FC-1/2/c927591111db765030760eae4dbada50}
}

@Article{Longest2006a,
  Title                    = {Mutual Enhancements of CFD Modeling and Experimental Data: A Case Study of 1-μm Particle Deposition in a Branching Airway Model},
  Author                   = {Longest, P. Worth and Oldham, Michael J.},
  Journal                  = {Inhalation Toxicology},
  Year                     = {2006},
  Number                   = {10},
  Pages                    = {761 - 771},
  Volume                   = {18},

  ISSN                     = {0895-8378},
  Type                     = {Journal Article},
  Url                      = {http://www.informaworld.com/10.1080/08958370600748653}
}

@Article{Longest2012,
  Title                    = {Comparing MDI and DPI aerosol deposition using in vitro experiments and a new stochastic individual path (SIP) model of the conducting airways},
  Author                   = {Longest, P. W. and Tian, G. and Walenga, R. L. and Hindle, M.},
  Journal                  = {Pharmaceutical Research},
  Year                     = {2012},
  Number                   = {6},
  Pages                    = {1670},
  Volume                   = {29},

  Doi                      = {10.1007/s11095-012-0691-y},
  Type                     = {Journal Article}
}

@Article{Longest2009a,
  Title                    = {Inertial deposition of aerosols in bifurcating models during steady expiratory flow},
  Author                   = {Longest, P. Worth and Vinchurkar, Samir},
  Journal                  = {Journal of Aerosol Science},
  Year                     = {2009},
  Note                     = {doi: DOI: 10.1016/j.jaerosci.2008.11.007},
  Number                   = {4},
  Pages                    = {370-378},
  Volume                   = {40},

  ISSN                     = {0021-8502},
  Keywords                 = {Respiratory particle dynamics
Particle deposition
Respiratory dosimetry
Exhalation
Computational fluid dynamics},
  Type                     = {Journal Article},
  Url                      = {http://www.sciencedirect.com/science/article/B6V6B-4V47CJ7-1/2/1903d49ced18a4422a933eb68b135600}
}

@Article{Longest2007c,
  Title                    = {Effects of mesh style and grid convergence on particle deposition in bifurcating airway models with comparisons to experimental data},
  Author                   = {Longest, P. Worth and Vinchurkar, Samir},
  Journal                  = {Medical Engineering \& Physics},
  Year                     = {2007},
  Number                   = {3},
  Pages                    = {350-366},
  Volume                   = {29},

  Abstract                 = {A number of research studies have employed a wide variety of mesh styles and levels of grid convergence to assess velocity fields and particle deposition patterns in models of branching biological systems. Generating structured meshes based on hexahedral elements requires significant time and effort; however, these meshes are often associated with high quality solutions. Unstructured meshes that employ tetrahedral elements can be constructed much faster but may increase levels of numerical diffusion, especially in tubular flow systems with a primary flow direction. The objective of this study is to better establish the effects of mesh generation techniques and grid convergence on velocity fields and particle deposition patterns in bifurcating respiratory models. In order to achieve this objective, four widely used mesh styles including structured hexahedral, unstructured tetrahedral, flow adaptive tetrahedral, and hybrid grids have been considered for two respiratory airway configurations. Initial particle conditions tested are based on the inlet velocity profile or the local inlet mass flow rate. Accuracy of the simulations has been assessed by comparisons to experimental in vitro data available in the literature for the steady-state velocity field in a single bifurcation model as well as the local particle deposition fraction in a double bifurcation model. Quantitative grid convergence was assessed based on a grid convergence index (GCI), which accounts for the degree of grid refinement. The hexahedral mesh was observed to have GCI values that were an order of magnitude below the unstructured tetrahedral mesh values for all resolutions considered. Moreover, the hexahedral mesh style provided GCI values of approximately 1% and reduced run times by a factor of 3. Based on comparisons to empirical data, it was shown that inlet particle seedings should be consistent with the local inlet mass flow rate. Furthermore, the mesh style was found to have an observable effect on cumulative particle depositions with the hexahedral solution most closely matching empirical results. Future studies are needed to assess other mesh generation options including various forms of the hybrid configuration and unstructured hexahedral meshes.},
  ISSN                     = {1350-4533},
  Keywords                 = {Respiratory particle dynamics
Grid convergence index
Respiratory dosimetry
Bifurcation models
Lagrangian particle tracking},
  Type                     = {Journal Article},
  Url                      = {http://www.sciencedirect.com/science/article/B6T9K-4K9C5PY-1/2/a076786525210900b2a5ab8064919cb5}
}

@Article{Longest2006b,
  Title                    = {Transport and deposition of respiratory aerosols in models of childhood asthma},
  Author                   = {Longest, P. Worth and Vinchurkar, Samir and Martonen, Ted},
  Journal                  = {Journal of Aerosol Science},
  Year                     = {2006},
  Number                   = {10},
  Pages                    = {1234-1257},
  Volume                   = {37},

  Abstract                 = {The effects of asthma induced airway constriction on aerosol dynamics and particle deposition have been investigated in bifurcating models of pediatric airways. Geometries considered include double bifurcation models of upper (G3-G5) and central (G7-G9) airways for a four-year-old child under healthy and 30% constricted conditions. Steady inhalation flow rates consistent with sedentary, light and high activity conditions have been employed. For this study, the deposition patterns of particles ranging from 1 to have been evaluated in terms of branch-averaged filtering efficiencies and highly localized cellular-level deposition rates. These cellular-level deposition values were calculated within diameter regions based on a novel microdosimetry factor. For both upper and central airway models, airway constriction increased branch-averaged filtering efficiencies by factors ranging from 1.5 to 10. More significantly, airway constriction was found to increase local cellular-level deposition rates by one to two orders of magnitude, as encapsulated by the microdosimetry factor. Based on these results, asthma constriction is observed to induce a significant increase in branch-averaged particle deposition as well as an even larger increase in local cellular-level deposition and cellular burden. These increased cellular-level deposition values may translate to an increased health risk associated with the inhalation of particulate matter in cases of asthma related bronchoconstriction.},
  ISSN                     = {0021-8502},
  Keywords                 = {Pediatric medicine
Children's lungs
Respiratory particle dynamics
Particle deposition
Respiratory dosimetry
Asthma
Airway constriction
Microdosimetry},
  Type                     = {Journal Article},
  Url                      = {http://www.sciencedirect.com/science/article/B6V6B-4JF8HM1-2/2/e1e36bf615b312b40a9893613f0d8868}
}

@Article{Longest2008b,
  Title                    = {Condensational Growth May Contribute to the Enhanced Deposition of Cigarette Smoke Particles in the Upper Respiratory Tract},
  Author                   = {Longest, P. Worth and Xi, J.},
  Journal                  = {Aerosol Science and Technology},
  Year                     = {2008},
  Pages                    = {579-602},
  Volume                   = {42},

  Doi                      = {DOI: 10.1080/02786820802232964
},
  ISSN                     = {0278-6826},
  Type                     = {Journal Article}
}

@Article{Lorenson1987,
  Title                    = {Marching cubes: a high resolution 3D surface construction algorithm},
  Author                   = {Lorenson, W.E. and Cline, H.E.},
  Journal                  = { Comput Graphics},
  Year                     = {1987},
  Pages                    = {163-169},
  Volume                   = {21},

  Type                     = {Journal Article}
}

@Article{Lorig2000,
  Title                    = {The application of electroencephalographic techniques to the study of human olfaction: a review and tutorial},
  Author                   = {Lorig, Tyler S.},
  Journal                  = {International Journal of Psychophysiology},
  Year                     = {2000},
  Number                   = {2},
  Pages                    = {91-104},
  Volume                   = {36},

  Abstract                 = {The use of a variety of electrophysiological techniques to determine the effects of odor on the nervous system is reviewed. Methods and problems associated with the collection of on-going EEG, chemosensory event-related potentials, and contingent negative variation data are discussed in depth as is the use of odors as modulators of brain potentials produced by other senses. In addition, the advantages of several seldom used EEG analysis techniques are discussed with respect to the unique problems of understanding olfaction.},
  ISSN                     = {0167-8760},
  Keywords                 = {Olfaction
Odor
EEG
CSERP
Evoked potential
Human},
  Type                     = {Journal Article},
  Url                      = {http://www.sciencedirect.com/science/article/B6T3M-3YWXJSR-2/2/9a7202ddf8e0ff229ac0c77dadab3352}
}

@Article{Loth2000,
  Title                    = {Numerical approaches for motion of dispersed particles, droplets and bubbles},
  Author                   = {Loth, E.},
  Journal                  = {Progress in Energy and Combustion Science},
  Year                     = {2000},
  Note                     = {doi: DOI: 10.1016/S0360-1285(99)00013-1},
  Number                   = {3},
  Pages                    = {161-223},
  Volume                   = {26},

  ISSN                     = {0360-1285},
  Keywords                 = {Numerical
Particle
Droplet
Bubble
Turbulence
Forces},
  Type                     = {Journal Article},
  Url                      = {http://www.sciencedirect.com/science/article/B6V3W-4090JFN-1/2/35a4055a954741f6422324a202e93e38}
}

@Article{Louis2006,
  Title                    = {Numerical and experimental study on nasal airflow},
  Author                   = {Louis, B. and Croce, C. and Papon, J. F. and Blondeau, J. R. and Caillibotte, G. and Coste, A. and Sbirlea-Apiou, G. and Isabey, D. and Fodil, R.},
  Journal                  = {Journal of Biomechanics},
  Year                     = {2006},
  Number                   = {Supplement 1},
  Pages                    = {S271-S272},
  Volume                   = {39},

  ISSN                     = {0021-9290},
  Type                     = {Journal Article},
  Url                      = {http://www.sciencedirect.com/science/article/B6T82-4KR88PB-1G3/2/7cd44fa69192c98911083db5ffc356b7}
}

@Article{Lu2014,
  Title                    = {Accuracy evaluation of a numerical simulation model of nasal airflow},
  Author                   = {Lu, Jiuxing and Han, Demin and Zhang, Luo},
  Journal                  = {Acta Oto-laryngologica},
  Year                     = {2014},
  Note                     = {PMID: 24702230},
  Number                   = {5},
  Pages                    = {513-519},
  Volume                   = {134},

  Doi                      = {10.3109/00016489.2013.863430},
  Eprint                   = { 
 http://dx.doi.org/10.3109/00016489.2013.863430
 
},
  Url                      = { 
 http://dx.doi.org/10.3109/00016489.2013.863430
 
}
}

@Article{Lu1995,
  Title                    = {An approach to modeling particle motion in turbulent flows--I. Homogeneous, isotropic turbulence},
  Author                   = {Lu, Q. Q.},
  Journal                  = {Atmospheric Environment},
  Year                     = {1995},
  Note                     = {doi: DOI: 10.1016/1352-2310(94)00269-Q},
  Number                   = {3},
  Pages                    = {423-436},
  Volume                   = {29},

  ISSN                     = {1352-2310},
  Type                     = {Journal Article},
  Url                      = {http://www.sciencedirect.com/science/article/B6VH3-3YKKGM5-1C/2/cbfb54adfda64e9ca70136f0e2ca4ceb}
}

@Article{Lu1996,
  Title                    = {Numerical analysis of indoor aerosol particle deposition and distribution in two-zone ventilation system},
  Author                   = {Lu, W. and Howarth, A.T.},
  Journal                  = {Building and Environment},
  Year                     = {1996},
  Number                   = {1},
  Pages                    = {41-50},
  Volume                   = {31},

  Type                     = {Journal Article}
}

@Article{Lucey2010,
  Title                    = {Measurement, Reconstruction, and Flow-Field Computation of the Human Pharynx With Application to Sleep Apnea},
  Author                   = {Lucey, A. D. and King, A. J. C. and Tetlow, G. A. and Wang, J. and Armstrong, J.J. and Leigh, M.S. and Paduch, A. and Walsh, J.H. and Sampson, D.D. and Eastwood, P.R. and Hillman, D.R.},
  Journal                  = {IEEE Transactions on biomedical engineering},
  Year                     = {2010},
  Pages                    = {14},
  Volume                   = {57},

  Type                     = {Journal Article}
}

@Article{Lucic2004,
  Title                    = {Interferometric and numerical study of the temperature field in the boundary layer and heat transfer in subcooled flow boiling},
  Author                   = {Lucic, Anita and Emans, Maximilian and Mayinger, Franz and Zenger, Christoph},
  Journal                  = {International Journal of Heat and Fluid Flow},
  Year                     = {2004},
  Number                   = {2},
  Pages                    = {180-195},
  Volume                   = {25},

  Abstract                 = {An interferometric study and a numerical simulation are presented of the combined process of the bulk turbulent convection and the dynamic of a vapor bubble which is formed in the superheated boundary layer of a subcooled flowing liquid, in order to determine the heat transfer to the flowing subcooled liquid. In this investigation focus has been given on a single vapor bubble at a defined cavity site to provide reproducible conditions. In the experimental study single bubbles were generated at a single artificial cavity by means of a CO2-laser as a spot heater at a uniformly heated wall of a vertical rectangular channel with water as the test fluid. The experiments were performed at various degrees of subcooling and mass flow rates. The bubble growth and the temporal decrease of the bubble volume were captured by means of the high-speed cinematography. The thermal boundary layer and the temperature field at the phase-interface between fluid and bubble were visualized by means of the optical measurement method holographic interferometry with a high temporal and spatial resolution, and thus the local and temporal heat transfer could be quantified. The experimental results form a significant data basis for the description of the mean as well as the local heat transfer as a function of the flow conditions. According to the experimental configuration and the obtained data the numerical simulations were performed. A numerical method has been developed to simulate the influence of single bubbles on the surrounding fluid which is based on a Lagrangian approach to describe the motion of the bubbles. The method is coupled to a large-eddy simulations by the body force term which is locally evaluated based on the density field. The obtained experimental data correspond well with the numerical predictions, both of which demonstrate the thermo- and fluiddynamic characteristics of the interaction between the vapor bubble and the subcooled liquid.},
  ISSN                     = {0142-727X},
  Keywords                 = {Holographic interferometry
Thermal boundary layer
Bubble dynamics
Local heat transfer
Euler-Lagrange model
Force-coupling method},
  Type                     = {Journal Article},
  Url                      = {http://www.sciencedirect.com/science/article/B6V3G-4BMTPTK-1/2/6033ede7609b75fd524e7ad7c9d2b00e}
}

@Article{Luft1966,
  Title                    = {Fine Structures of Capillary and Endocapillary Layer as Revealed by Ruthenium Red.},
  Author                   = {Luft, J.H. },
  Journal                  = {Federation Proceedings},
  Year                     = {1966},
  Pages                    = {1773-1783},
  Volume                   = {25},

  Type                     = {Journal Article}
}

@Article{Lund2014,
  Title                    = {European position paper on the anatomical terminology of the internal nose and paranasal sinuses.},
  Author                   = {Lund, Valerie J. and Stammberger, Heinz and Fokkens, Wytske J. and Beale, Tim and Bernal-Sprekelsen, Manuel and Eloy, Philippe and Georgalas, Christos and Gerstenberger, Claus and Hellings, Peter and Herman, Philippe and Hosemann, Werner G. and Jankowski, Roger and Jones, Nick and Jorissen, Mark and Leunig, Andreas and Onerci, Metin and Rimmer, Joanne and Rombaux, Philippe and Simmen, Daniel and Tomazic, Peter Valentin and Tschabitscherr, Manfred and Welge-Luessen, Antje},
  Journal                  = {Rhinology. Supplement},
  Year                     = {2014},
  Pages                    = {1-34},

  Abstract                 = {The advent of endoscopic sinus surgery led to a resurgence of interest in the detailed anatomy of the internal nose and paranasal sinuses. However, the official Terminologica Anatomica used by basic anatomists omits many of the structures of surgical importance. This led to numerous clinical anatomy papers and much discussion about the exact names and definitions for the structures of surgical relevance. This European Position Paper on the Anatomical Terminology of the Internal Nose and Paranasal Sinuses was conceived to re-evaluate the anatomical terms in common usage by endoscopic sinus surgeons and to compare this with the official Terminologica Anatomica. The text is a concise summary of all the structures encountered during routine endoscopic surgery in the nasal cavity, paranasal sinuses and at the interface with the orbit and skull base but does not provide a comprehensive text for advanced skull base surgery. It draws on a detailed review of the literature and provides a consensus where several options are available, defining the anatomical structure in simple terms and in English. It is recognised that this is an area of great variation and some indication of the frequency with which these variants are encountered is given in the text and table. All major anatomical points are illustrated, drawing on the expertise of the multi-national and multi-disciplinary contributors to this project.},
  ISSN                     = {1013-0047},
  Type                     = {Journal Article}
}

@Article{Lundblad2007,
  Title                    = {Airways hyperresponsiveness in allergically inflamed mice: the role of airway closure},
  Author                   = {Lundblad, L.K. and Thompson-Figueroa, J. and Allen, G.B. and Rinaldi, L. and Norton, R.J. and Irvin, C.G. and Bates, J.H. },
  Journal                  = {Am J Resp Crit Care Med},
  Year                     = {2007},
  Pages                    = {768-774},
  Volume                   = {175},

  Type                     = {Journal Article}
}

@Article{Luo2008,
  Title                    = {An immersed-boundary method for flow-structure interaction in biological systems with application to phonation},
  Author                   = {Luo, Haoxiang and Mittal, Rajat and Zheng, Xudong and Bielamowicz, Steven A. and Walsh, Raymond J. and Hahn, James K.},
  Journal                  = {Journal of Computational Physics},
  Year                     = {2008},
  Number                   = {22},
  Pages                    = {9303-9332},
  Volume                   = {227},

  Abstract                 = {A new numerical approach for modeling a class of flow-structure interaction problems typically encountered in biological systems is presented. In this approach, a previously developed, sharp-interface, immersed-boundary method for incompressible flows is used to model the fluid flow and a new, sharp-interface Cartesian grid, immersed-boundary method is devised to solve the equations of linear viscoelasticity that governs the solid. The two solvers are coupled to model flow-structure interaction. This coupled solver has the advantage of simple grid generation and efficient computation on simple, single-block structured grids. The accuracy of the solid-mechanics solver is examined by applying it to a canonical problem. The solution methodology is then applied to the problem of laryngeal aerodynamics and vocal fold vibration during human phonation. This includes a three-dimensional eigen analysis for a multi-layered vocal fold prototype as well as two-dimensional, flow-induced vocal fold vibration in a modeled larynx. Several salient features of the aerodynamics as well as vocal fold dynamics are presented.},
  ISSN                     = {0021-9991},
  Keywords                 = {Immersed-boundary method
Elasticity
Flow-structure interaction
Bio-flow mechanics
Phonation
Laryngeal flow
Flow-induced vibration},
  Type                     = {Journal Article},
  Url                      = {http://www.sciencedirect.com/science/article/B6WHY-4SH0Y66-4/2/716e392669c14ac22eec45bb15203d3f}
}

@Article{Luo2009,
  Title                    = {Particle deposition in a CT-scanned human lung airway},
  Author                   = {Luo, H. Y. and Liu, Y.},
  Journal                  = {Journal of Biomechanics},
  Year                     = {2009},
  Note                     = {doi: DOI: 10.1016/j.jbiomech.2009.05.004},
  Number                   = {12},
  Pages                    = {1869-1876},
  Volume                   = {42},

  ISSN                     = {0021-9290},
  Keywords                 = {Human lung model
Particle deposition
Turbulent flow
Numerical simulation},
  Type                     = {Journal Article},
  Url                      = {http://www.sciencedirect.com/science/article/B6T82-4WF2NM8-1/2/95fb61abdd2a7952b4330952c340486c}
}

@Article{Luo2008a,
  Title                    = {Modeling the bifurcating flow in a CT-scanned human lung airway},
  Author                   = {Luo, H. Y. and Liu, Y.},
  Journal                  = {Journal of Biomechanics},
  Year                     = {2008},
  Number                   = {12},
  Pages                    = {2681-2688},
  Volume                   = {41},

  Abstract                 = {The inspiratory flow characteristics in a CT-scanned human lung model were numerically investigated using low Reynolds number (LRN) [kappa]-[omega] turbulent model. The five-generation airway is extracted from the trachea to segmental bronchi of a 60-year-old Chinese male patient. Computations were carried out in the Reynolds number range of 900-2100, corresponding to mouth-air breathing rates of 190-440 ml/s. Flow patterns on the Re=2100 and flow rate distribution were presented. In this model, the flow pattern is very complex. To count the effect of laryngeal jet on trachea inlet, the trachea was extended and modified to simulate the larynx, consequently the inlet velocity profile is biased towards the rear wall. In the inferior lobar bronchi, there are two stems in which the axial velocity is stronger but secondary velocity is weaker. Secondary flow in the lateral bronchi is stronger than the medial ones. With increasing Re, the air flow increases in the middle, inferior lobes and left main bronchus, i.e., flow biases to left and downward.},
  ISSN                     = {0021-9290},
  Keywords                 = {CT scan
Human lung
Inspiratory flow
Numerical simulation},
  Type                     = {Journal Article},
  Url                      = {http://www.sciencedirect.com/science/article/B6T82-4T3CR3J-2/2/3b5401afe2253a9e8ec2db8f6513d290}
}

@Article{Luo2007,
  Title                    = {Particle deposition in obstructed airways},
  Author                   = {Luo, H. Y. and Liu, Y. and Yang, X. L.},
  Journal                  = {Journal of Biomechanics},
  Year                     = {2007},
  Note                     = {doi: DOI: 10.1016/j.jbiomech.2007.03.027},
  Number                   = {14},
  Pages                    = {3096-3104},
  Volume                   = {40},

  ISSN                     = {0021-9290},
  Keywords                 = {Particle deposition
COPD
Four-generation bifurcation airways},
  Type                     = {Journal Article},
  Url                      = {http://www.sciencedirect.com/science/article/B6T82-4NRVV08-2/2/beefea5c173a0f68abd151ea7be51ccf}
}

@Article{Luo2004,
  Title                    = {Simulation of air flow in the IEA Annex 20 test room—validation of a simplified model for the nozzle diffuser in isothermal test cases},
  Author                   = {Luo, S. and Heikkinen, J. and Roux, Bernard},
  Journal                  = {Building and Environment},
  Year                     = {2004},
  Number                   = {12},
  Pages                    = {1403-1415},
  Volume                   = {39},

  Doi                      = {10.1016/j.buildenv.2004.04.006},
  ISSN                     = {0360-1323},
  Keywords                 = {CFD
Air supply diffuser
Ventilation
Wall jet
Flow asymmetry
Velocity correction
Local mesh refinement},
  Type                     = {Journal Article},
  Url                      = {http://www.sciencedirect.com/science/article/pii/S0360132304001283}
}

@Article{Luo2004a,
  Title                    = {LES modelling of flow in a simple airway model},
  Author                   = {Luo, X. Y. and Hinton, J. S. and Liew, T. T. and Tan, K. K.},
  Journal                  = {Medical Engineering \& Physics},
  Year                     = {2004},
  Number                   = {5},
  Pages                    = {403-413},
  Volume                   = {26},

  Abstract                 = {Detailed information about the flow field pattern is highly important in accurately predicting particle deposition sites in the human airway. Flow in the upper airway during heavy breathing can have a Reynolds number as high as 9300, and therefore presents turbulent features. Although turbulence is believed to have an important effect on the airflow and other transport processes in the bronchial tree, to date both theoretical and numerical studies have predominantly assumed the flow to be laminar. In this paper, transitional/turbulent flow during inspiration is studied using a large eddy simulation (LES) in a single asymmetric bifurcation model of human upper airway. The influence of the non-laminar flow on the patterns and the particle paths is investigated in both 2D and 3D models. Throughout the investigation, comparisons with the laminar and conventional k-[var epsilon] models for the same configuration and flow conditions are made. The LES model is also carefully validated against published experimental data in a stenotic tube model. The results demonstrate that the LES model is capable of capturing instantaneous eddy formation and flow separation in (almost) laminar, transitional and turbulent flow regimes, and hence may be used as a powerful and practical tool to provide much of the detailed flow information required for tracing the particle trajectories and particle deposition in human airways.},
  ISSN                     = {1350-4533},
  Keywords                 = {Airway
Transitional/turbulent flow
LES
CFD
Particle deposition},
  Type                     = {Journal Article},
  Url                      = {http://www.sciencedirect.com/science/article/B6T9K-4C4BKT5-4/2/8431646e679bbb7e3b5b3a12ef9525cb}
}

@Article{Lynch1993,
  Title                    = {Uncomplicated asthma in adults: comparison of CT appearance of the lungs in asthmatic and healthy subjects. },
  Author                   = {Lynch, D.A. and Newell, J.D. and Tschomper, B.A. and Cink, T.M. and Newman, L.S. and Bethel, R.},
  Journal                  = {Radiology},
  Year                     = {1993},
  Pages                    = {829-833},
  Volume                   = {188},

  Type                     = {Journal Article}
}

@Article{Ma2009,
  Title                    = {CFD simulation and experimental validation of fluid flow and particle transport in a model of alveolated airways},
  Author                   = {Ma, Baoshun and Ruwet, Vincent and Corieri, Patricia and Theunissen, Raf and Riethmuller, Michel and Darquenne, Chantal},
  Journal                  = {Journal of Aerosol Science},
  Year                     = {2009},
  Note                     = {doi: DOI: 10.1016/j.jaerosci.2009.01.002},
  Number                   = {5},
  Pages                    = {403-414},
  Volume                   = {40},

  ISSN                     = {0021-8502},
  Keywords                 = {Pulmonary fluid mechanics
Aerosol transport
PIV
PTV},
  Type                     = {Journal Article},
  Url                      = {http://www.sciencedirect.com/science/article/B6V6B-4VG5J04-2/2/3bdfe756014221798edd2d682f429e58}
}

@Article{Ma1996,
  Title                    = {Detection of two clusters of mechanical properties of smooth muscle along the airway tree},
  Author                   = {Ma, X. and Li, W. and Stephens, N.L.},
  Journal                  = {Journal of Applied Physiology},
  Year                     = {1996},
  Pages                    = {857-861},
  Volume                   = {80},

  Type                     = {Journal Article}
}

@Article{MA¶ller2010,
  Title                    = {Pulsating aerosols for drug delivery to the sinuses in healthy volunteers},
  Author                   = {Möller, Winfried and Schuschnig, Uwe and Khadem Saba, Gülnaz and Meyer, Gabriele and Junge-Hülsing, Bernhard and Keller, Manfred and Häussinger, Karl},
  Journal                  = {Otolaryngology - Head and Neck Surgery},
  Year                     = {2010},
  Number                   = {3},
  Pages                    = {382-388},
  Volume                   = {142},

  ISSN                     = {0194-5998},
  Type                     = {Journal Article},
  Url                      = {http://www.sciencedirect.com/science/article/B6WP4-4YDSNXK-J/2/eccb2ae4280a396dd1678eb86fce76fa}
}

@Article{MA¼hlfeld2008,
  Title                    = {Interactions of nanoparticles with pulmonary structures and cellular responses},
  Author                   = {Mühlfeld, Christian and Rothen-Rutishauser, Barbara and Blank, Fabian and Vanhecke, Dimitri and Ochs, Matthias and Gehr, Peter},
  Journal                  = {American Journal of Physiology - Lung Cellular and Molecular Physiology},
  Year                     = {2008},
  Number                   = {5},
  Pages                    = {L817-L829},
  Volume                   = {294},

  Abstract                 = {Combustion-derived and synthetic nano-sized particles (NSP) have gained considerable interest among pulmonary researchers and clinicians for two main reasons. 1) Inhalation exposure to combustion-derived NSP was associated with increased pulmonary and cardiovascular morbidity and mortality as suggested by epidemiological studies. Experimental evidence has provided a mechanistic picture of the adverse health effects associated with inhalation of combustion-derived and synthetic NSP. 2) The toxicological potential of NSP contrasts with the potential application of synthetic NSP in technological as well as medicinal settings, with the latter including the use of NSP as diagnostics or therapeutics. To shed light on this paradox, this article aims to highlight recent findings about the interaction of inhaled NSP with the structures of the respiratory tract including surfactant, alveolar macrophages, and epithelial cells. Cellular responses to NSP exposure include the generation of reactive oxygen species and the induction of an inflammatory response. Furthermore, this review places special emphasis on methodological differences between experimental studies and the caveats associated with the dose metrics and points out ways to overcome inherent methodological problems.},
  Doi                      = {10.1152/ajplung.00442.2007},
  Type                     = {Journal Article},
  Url                      = {http://ajplung.physiology.org/content/294/5/L817.abstract}
}

@Article{Macinnes1992,
  Title                    = {Stochastic particle dispersion modeling and the tracer particle limit},
  Author                   = {Macinnes, J.M. and Bracco, F.V.},
  Journal                  = {Physics of Fluids A},
  Year                     = {1992},
  Pages                    = {2809-23},
  Volume                   = {4},

  Type                     = {Journal Article}
}

@Article{Madhav1995,
  Title                    = {Drag on non-spherical particles in viscous fluids},
  Author                   = {Madhav, G. Venu and Chhabra, R. P.},
  Journal                  = {International Journal of Mineral Processing},
  Year                     = {1995},
  Note                     = {doi: DOI: 10.1016/0301-7516(94)00038-2},
  Number                   = {1-2},
  Pages                    = {15-29},
  Volume                   = {43},

  ISSN                     = {0301-7516},
  Type                     = {Journal Article},
  Url                      = {http://www.sciencedirect.com/science/article/B6VBN-3YS8YDX-C/2/5e04b6dbba070de90f5d416f97360134}
}

@Article{Madsen1993,
  Title                    = {A numerical and experimental study of local exhaust capture efficiency.},
  Author                   = {Madsen, U. and Breum, N.O. and Nielsen, P.V.},
  Journal                  = {Ann. Occup. Hyg.},
  Year                     = {1993},
  Number                   = {6},
  Pages                    = {593-605},
  Volume                   = {37},

  Type                     = {Journal Article}
}

@Article{Maina2009,
  Title                    = {Inspiratory aerodynamic valving occurs in the ostrich, Struthio camelus lung: A computational fluid dynamics study under resting unsteady state inhalation},
  Author                   = {Maina, J. N. and Singh, P. and Moss, E. A.},
  Journal                  = {Respiratory Physiology \& Neurobiology},
  Year                     = {2009},
  Number                   = {3},
  Pages                    = {262-270},
  Volume                   = {169},

  Abstract                 = {In the avian lung, inhaled air is shunted past the openings of the medioventral secondary bronchi (MVSB) by a mechanism termed [`]inspiratory aerodynamic valving' (IAV). Sizes and orientations of the trachea (Tr), syrinx (Sx), extrapulmonary primary bronchus (EPPB), intrapulmonary primary bronchus (IPPB), MVSB, mediodorsal secondary bronchi (MDSB), lateroventral secondary bronchi (LVSB) and the ostium (Ot) were determined in the ostrich, Struthio camelus. Air flow was simulated through computationally generated models and its dynamics analysed. The [`]truncated normal model' (TNM) consisted of the Tr, Sx, EPPB, IPPB, MVSB and the Ot. For the [`]inclusive normal model' (INM), the MDSB and the LDSB were added. Variations of these models included the [`]truncated MVSB1 rotated model' (TMVSB1RM), the [`]truncated constriction fitted model' (TCFM) and the [`]inclusive MVSB1 rotated model' (IMVSB1RM). In the TNM, the TMVSB1RM and the TCFM, the air flow exited through the MVSB while for the INM and the IMVSB1RM, very little of it did: IAV did not occur in the partial models. In the IMVSB1RM, rotating the MVSB1 clockwise did not affect IAV. The incomplete models may be faulty because the velocity/pressure profiles in different parts of the interconnected airways form an integrated functional continuum in which different parts of the system considerably impact on each other.},
  ISSN                     = {1569-9048},
  Keywords                 = {Avian lung
Inspiration
Air flow
Inspiratory aerodynamic valving
Computational fluid dynamics
Unsteady state condition
Modeling},
  Type                     = {Journal Article},
  Url                      = {http://www.sciencedirect.com/science/article/B6X16-4X9TTWK-2/2/7189fb81cbafed885876d2ed3be41f47}
}

@Article{Malin1989,
  Title                    = {Modeling the effects of lateral divergence on radially spreading turbulent jets},
  Author                   = {Malin, M. R.},
  Journal                  = {Computers \&amp; Fluids},
  Year                     = {1989},
  Number                   = {3},
  Pages                    = {453-465},
  Volume                   = {17},

  Doi                      = {10.1016/0045-7930(89)90037-6},
  ISSN                     = {0045-7930},
  Type                     = {Journal Article},
  Url                      = {http://www.sciencedirect.com/science/article/pii/0045793089900376}
}

@Article{Malve2011,
  Title                    = {FSI Analysis of a Human Trachea Before and After Prosthesis Implantation},
  Author                   = {Malve, M. and del Palomar, A. Perez and Chandra, S. and Lopez-Villalobos, J. L. and Finol, E. A. and Ginel, A. and Doblare, M.},
  Journal                  = {Journal of Biomechanical Engineering},
  Year                     = {2011},
  Number                   = {7},
  Pages                    = {071003-12},
  Volume                   = {133},

  Keywords                 = {biological tissues
biomechanics
deformation
elasticity
finite element analysis
fractals
medical computing
physiological models
pneumodynamics
stents
stress effects
surgery},
  Type                     = {Journal Article},
  Url                      = {http://dx.doi.org/10.1115/1.4004315}
}

@Article{Mankovich1990,
  Title                    = {Three-dimensional image display in medicine},
  Author                   = {Mankovich, N.J. and Robertson, D.R. and Cheeseman, A.M.},
  Journal                  = {J Digit Imaging},
  Year                     = {1990},
  Number                   = {2},
  Pages                    = {69-80},
  Volume                   = {3},

  Type                     = {Journal Article}
}

@Article{Marchioli2006,
  Title                    = {Particle dispersion and wall-dependent turbulent flow scales: implications for local equilibrium models},
  Author                   = {Marchioli, Cristian and Picciotto, Maurizio and Soldati, Alfredo},
  Journal                  = {Journal of Turbulence},
  Year                     = {2006},
  Pages                    = {N60},
  Volume                   = {7},

  ISSN                     = {1468-5248},
  Type                     = {Journal Article},
  Url                      = {http://www.informaworld.com/10.1080/14685240600925171}
}

@Article{Marcum1995,
  Title                    = {Unstructured grid generation using iterative point insertion and local reconnection},
  Author                   = {Marcum, D. L. and Weatherill, N.P.},
  Journal                  = {AIAA},
  Year                     = {1995},
  Number                   = {9},
  Pages                    = {1619-1625},
  Volume                   = {33},

  Type                     = {Journal Article}
}

@Article{MarkS.Dykewicz2010,
  Title                    = {Rhinitis and sinusitis},
  Author                   = {Mark S. Dykewicz, MD, a and Daniel L. Hamilos, MDb},
  Journal                  = {J Allergy Clin Immunol},
  Year                     = {2010},

  Type                     = {Journal Article}
}

@Article{Marko1997,
  Title                    = {3D PTV and its application on Lagrangian motion},
  Author                   = {Marko, Virant and Themistocles, Dracos},
  Journal                  = {Measurement Science and Technology},
  Year                     = {1997},
  Number                   = {12},
  Pages                    = {1539},
  Volume                   = {8},

  Abstract                 = {Three-dimensional particle tracking velocimetry (3D PTV) is a flow measurement technique for the determination of velocity vectors and trajectories within a three-dimensional observation volume. This makes this technique suitable not only for Eulerian but also for Lagrangian investigation of flow phenomena, especially in the field of turbulence and turbulent diffusion. The principle and the application on open channel flow of 3D PTV are briefly described in the first part of this paper. By decomposition of the velocity vectors velocity profiles and turbulence intensities are derived and presented in the second part. Turbulent energy dissipation, Kolmogorov length and time scales and Taylor micro scales are computed from the measured data sets. In the Lagrangian analysis pdf's of the Lagrangian acceleration, correlation coefficients and Taylor micro scales are determined. The individual determination of the particle positions and displacements allows finally the application on the one and two particle approaches for turbulent diffusion from Taylor (1921), Richardson (1926) and Batchelor (1949, 1952).},
  ISSN                     = {0957-0233},
  Type                     = {Journal Article},
  Url                      = {http://stacks.iop.org/0957-0233/8/i=12/a=017}
}

@Article{Marr1980,
  Title                    = {Theory of edge detection},
  Author                   = {Marr, D. and Hildreth, E.},
  Journal                  = {Proceedings of Royal Society of London},
  Year                     = {1980},
  Pages                    = {187-217},
  Volume                   = {207},

  Type                     = {Journal Article}
}

@Article{Marsden2003,
  Title                    = {Direct noise computation of adaptive control applied to a cavity flow},
  Author                   = {Marsden, Olivier and Gloerfelt, Xavier and Bailly, Christophe},
  Journal                  = {Comptes Rendus Mecanique},
  Year                     = {2003},
  Number                   = {6},
  Pages                    = {423-429},
  Volume                   = {331},

  Abstract                 = {The Large Eddy Simulation of closed-loop active flow control applied to a 3D cavity excited by a compressible airflow with a Mach number of 0.6 is presented. The control actuator is an idealized synthetic jet located at the upstream cavity edge, and the control function is supplied by a feedback LMS-type algorithm whose input is a pressure signal measured inside the cavity. The radiated sound, provided directly by the LES simulation, was shown to decrease substantially when active control was applied. A simultaneous reduction of the vertical velocity fluctuations in the shear layer was observed. The intensity of vortical structures inside the cavity was also reduced, although the general aspect of the recirculation zone was not modified. The direct noise computation technique, which supplies the pressure field by solving the fluid mechanics equations, is shown to constitute a powerful tool for studying active aeroacoustic noise control. To cite this article: O. Marsden et al., C. R. Mecanique 331 (2003).},
  ISSN                     = {1631-0721},
  Keywords                 = {Acoustique
Ondes
Vibrations
Contrôle actif adaptatif
Aeroacoustique numérique
Bruit de cavité},
  Type                     = {Journal Article},
  Url                      = {http://www.sciencedirect.com/science/article/B6X1C-48WB32R-8/2/0eddf3d7c7f8b577651ace56501919e5}
}

@Article{Marshall1998,
  Title                    = {Electrostructural Phase Changes in Charged Particulate Clouds: Planetary and Astrophysical Implications},
  Author                   = {Marshall, J.R. },
  Journal                  = {LPSC},
  Year                     = {1998},
  Pages                    = {1132},
  Volume                   = {29},

  Type                     = {Journal Article}
}

@Article{Marshall2009,
  Title                    = {Discrete-element modeling of particulate aerosol flows},
  Author                   = {Marshall, J. S.},
  Journal                  = {Journal of Computational Physics},
  Year                     = {2009},
  Number                   = {5},
  Pages                    = {1541-1561},
  Volume                   = {228},

  Abstract                 = {A multiple-time step computational approach is presented for efficient discrete-element modeling of aerosol flows containing adhesive solid particles. Adhesive aerosol particulates are found in numerous dust and smoke contamination problems, including smoke particle transport in the lungs, particle clogging of heat exchangers in construction vehicles, industrial nanoparticle transport and filtration systems, and dust fouling of electronic systems and MEMS components. Dust fouling of equipment is of particular concern for potential human occupation on dusty planets, such as Mars. The discrete-element method presented in this paper can be used for prediction of aggregate structure and breakup, for prediction of the effect of aggregate formation on the bulk fluid flow, and for prediction of the effects of small-scale flow features (e.g., due to surface roughness or MEMS patterning) on the aggregate formation. After presentation of the overall computational structure, the forces and torques acting on the particles resulting from fluid motion, particle-particle collision, and adhesion under van der Waals forces are reviewed. The effect of various parameters of normal collision and adhesion of two particles are examined in detail. The method is then used to examine aggregate formation and particle clogging in pipe and channel flow.},
  Doi                      = {10.1016/j.jcp.2008.10.035},
  ISSN                     = {0021-9991},
  Keywords                 = {Particulate flow
Aerosols
Particle adhesion
Aggregation},
  Type                     = {Journal Article},
  Url                      = {http://www.sciencedirect.com/science/article/pii/S002199910800572X}
}

@Article{Martin2008,
  Title                    = {Enhanced deposition of high aspect ratio aerosols in small airway bifurcations using magnetic field alignment},
  Author                   = {Martin, Andrew R. and Finlay, Warren H.},
  Journal                  = {Journal of Aerosol Science},
  Year                     = {2008},
  Note                     = {doi: DOI: 10.1016/j.jaerosci.2008.04.003},
  Number                   = {8},
  Pages                    = {679-690},
  Volume                   = {39},

  ISSN                     = {0021-8502},
  Keywords                 = {Aerosol drug delivery
Magnetic drug targeting
Aerosol deposition
Magnetic nanoparticles
Inhalation},
  Type                     = {Journal Article},
  Url                      = {http://www.sciencedirect.com/science/article/B6V6B-4S9P5V0-1/2/07d70aeaf3dbfdc7945dcda30c0dec88}
}

@Article{Martin1989,
  Title                    = {An evaluation of the toxicity of graphite fiber composites for lung cells in vitro and in vivo},
  Author                   = {Martin, T.R. and Meyer, S.W. and Luchtel, D.R.},
  Journal                  = {Environmental Research},
  Year                     = {1989},
  Pages                    = {246-261},
  Volume                   = {49},

  Type                     = {Journal Article}
}

@Article{Martonen2000,
  Title                    = {Human lung morphology models for particle deposition studies.},
  Author                   = {Martonen, T.},
  Journal                  = {Inhalation Toxicology},
  Year                     = {2000},
  Pages                    = {829–849},
  Volume                   = {55},

  Type                     = {Journal Article}
}

@Article{Martonen2003,
  Title                    = {In silico modeling of asthma},
  Author                   = {Martonen, T. and Fleming, J. and Schroeter, J. and Conway, J. and Hwang, D.},
  Journal                  = {Advanced Drug Delivery Reviews},
  Year                     = {2003},
  Pages                    = {829–849},
  Volume                   = {55},

  Type                     = {Journal Article}
}

@Article{Martonen1994,
  Title                    = {Effects of Carinal Ridge Shapes on Lung Airstreams},
  Author                   = {Martonen, T.B. and Yang, Y. and Xue, Z.Q.},
  Journal                  = {Aerosol Science and Technology},
  Year                     = {1994},
  Number                   = {2},
  Pages                    = {119-136},
  Volume                   = {21},

  Type                     = {Journal Article}
}

@Article{Martonen1992,
  Title                    = {Comments on recent data for particle deposition in human nasal passages},
  Author                   = {Martonen, T. and Zhang, Z.},
  Journal                  = {Journal of Aerosol Science},
  Year                     = {1992},
  Number                   = {6},
  Pages                    = {667-674},
  Volume                   = {23},

  Type                     = {Journal Article}
}

@Article{Martonen1996,
  Title                    = {Particle diffusion with entrance effects in a smooth-walled cylinder},
  Author                   = {Martonen, Ted and Zhang, Zongqin and Yang, Yadong},
  Journal                  = {Journal of Aerosol Science},
  Year                     = {1996},
  Note                     = {doi: DOI: 10.1016/0021-8502(95)00530-7},
  Number                   = {1},
  Pages                    = {139-150},
  Volume                   = {27},

  ISSN                     = {0021-8502},
  Type                     = {Journal Article},
  Url                      = {http://www.sciencedirect.com/science/article/B6V6B-3WBXSPR-B/2/a8fda1f637a6f136c44b90735d38748f}
}

@Article{Martonen1993,
  Title                    = {Fluid dynamics of the human larynx and upper tracheobronchial airways},
  Author                   = {Martonen, T. B. and Zhang, Z. and Lessmann, R. C.},
  Journal                  = {Aerosol science and technology},
  Year                     = {1993},
  Number                   = {2},
  Pages                    = {23},
  Volume                   = {19},

  ISSN                     = {0278-6826 },
  Type                     = {Journal Article},
  Url                      = {http://cat.inist.fr/?aModele=exportN&cpsidt=4006372}
}

@Article{Martonen2003a,
  Title                    = {Fine particle deposition within human nasal airways},
  Author                   = {Martonen, T. B. and Zhang, Z. and Yue, G. and Musante, C. J.},
  Journal                  = {Inhalation Toxicology},
  Year                     = {2003},
  Note                     = {Cited By (since 1996): 13
Export Date: 5 June 2011
Source: Scopus},
  Number                   = {4},
  Pages                    = {283-303},
  Volume                   = {15},

  Type                     = {Journal Article},
  Url                      = {http://www.scopus.com/inward/record.url?eid=2-s2.0-0037395340&partnerID=40&md5=4bb9af8a24fcb8325ea6eaaff4c6c352}
}

@Article{Martonen2002,
  Title                    = {3-D Particle transport within the human upper respiratory tract},
  Author                   = {Martonen, T. B. and Zhang, Z. and Yue, G. and Musante, C. J.},
  Journal                  = {Journal of Aerosol Science},
  Year                     = {2002},
  Number                   = {8},
  Pages                    = {1095-1110},
  Volume                   = {33},

  Abstract                 = {In this study, trajectories of inhaled particulate matter (PM) were simulated within a three-dimensional (3-D) computer model of the human upper respiratory tract (URT). The airways were described by computer-reconstructed images of a silicone rubber cast of the human head, throat, and trachea and main bronchi. Computational fluid dynamics simulations of airflow patterns were performed using commercially available software (CFX-F3D) on a Sun Sparc-5 workstation. For each simulation, particles were introduced within a nostril grid and their trajectories calculated. A typical computer run of 400 iterations took 25 h. Particle deposition was designated in nasal (N), laryngeal (L), and tracheobronchial (T) regions, or penetration (P) of the URT. Deposition was calculated as a function of particle size (0.5-5 [mu]m), density (1 and 2 g/cm3), and flow rate (9, 17, and 34 l/min). The computations should be addressed on a case-by-case basis as detailed herein; however, it can be stated that in a given N, L, or T region, deposition fractions as determined by the calculation of trajectories increased more with flow rate than particle size. This knowledge of factors affecting particle trajectories and deposition patterns may have important implications for PM risk assessment programs.},
  ISSN                     = {0021-8502},
  Keywords                 = {Inhaled particulate matter (PM)
Computational fluid dynamics (CFD)
Risk assessment},
  Type                     = {Journal Article},
  Url                      = {http://www.sciencedirect.com/science/article/B6V6B-45KSRRM-1/2/e1066392daf14517ef2d5e86aecb8d6b}
}

@InBook{Mary2005,
  Title                    = {Large-Eddy Simulation of a Controlled Flow Cavity},
  Author                   = {Mary, I. and Lê, T. H.},
  Pages                    = {565-573},
  Publisher                = {Elsevier Science B.V.},
  Year                     = {2005},

  Address                  = {Amsterdam},
  Type                     = {Book Section},

  Abstract                 = {Summary Numerical results from a computational and experimental research program directed towards understanding cavity noise attenuation using a vortex generator placed within the boundary layer, just upstream the cavity, are presented. A LES computation, based on the compressible filtered Navier-Stokes equations, is conducted to show the effects of such suppression resonance device. Wall functions and a local mesh refinement technique are used in a multi-block structured solver to reduce the computational cost of the simulation. The M219 cavity has been retained for this study and the inflow Mach number is equal to 0.85. Inflow boundary conditions are prescribed using profiles obtained from a RANS boundary layer calculation. The results obtained show that the vortex generator device reduces significantly and simultaneously the four Rossiter modes which are driven by the strong interaction between shear layer dynamics and acoustic propagation, leading to a self-sustained resonance phenomenon. No similar behavior is observed with the controlled cavity, in which a flat plate of zero thickness across the cavity generates a von Karman street and the resulting vortices, impinging in the rear part, do not produced strong acoustic tones.},
  Booktitle                = {Engineering Turbulence Modelling and Experiments 6},
  ISBN                     = {978-0-08-044544-1},
  Url                      = {http://www.sciencedirect.com/science/article/B8612-4P9MYBV-37/2/f76395971eba87a4c90c6708830ec32d}
}

@Article{Masing1967,
  Title                    = {Investigations about the course of flow in the nose model},
  Author                   = {Masing, H.},
  Journal                  = {Arch Klin Exp Ohren Nasen Kehlkopfheilkd},
  Year                     = {1967},
  Pages                    = {371-381},
  Volume                   = {189},

  Type                     = {Journal Article}
}

@Article{Materialise2008,
  Title                    = {Mimics Documentation},
  Author                   = {Materialise},
  Journal                  = {Materialise NV},
  Year                     = {2008},

  Type                     = {Journal Article}
}

@Article{MathWorks2007,
  Title                    = {Matlab 7.1},
  Author                   = {MathWorks},
  Journal                  = {MathWorks Inc. USA},
  Year                     = {2007},

  Type                     = {Journal Article}
}

@Article{Matida2004,
  Title                    = {A new add-on spacer design concept for dry-powder inhalers},
  Author                   = {Matida, E. A. and Finlay, W. H. and Rimkus, M. and Grgic, B. and Lange, C. F.},
  Journal                  = {Journal of Aerosol Science},
  Year                     = {2004},
  Number                   = {7},
  Pages                    = {823-833},
  Volume                   = {35},

  ISSN                     = {0021-8502},
  Type                     = {Journal Article},
  Url                      = {http://www.sciencedirect.com/science/article/B6V6B-4BRTC4V-1/2/b93ae5f4fd607e3834e27535e1d0e230}
}

@Article{Matida2000,
  Title                    = {Statistical simulation of particle deposition on the wall from turbulent dispersed pipe flow},
  Author                   = {Matida, Edgar Akio and Nishino, Koichi and Torii, Kahoru},
  Journal                  = {International Journal of Heat and Fluid Flow},
  Year                     = {2000},
  Note                     = {doi: DOI: 10.1016/S0142-727X(00)00004-7},
  Number                   = {4},
  Pages                    = {389-402},
  Volume                   = {21},

  ISSN                     = {0142-727X},
  Keywords                 = {Two-phase flow
Turbulent pipe flow
Particle deposition
Numerical simulation
Lagrangian particle tracking
One-way coupling},
  Type                     = {Journal Article},
  Url                      = {http://www.sciencedirect.com/science/article/B6V3G-40NMTV9-1/2/c035b49a415b1bd7bb309ad3d4ab9be7}
}

@Article{Matida2004a,
  Title                    = {Improved numerical simulation for aerosol deposition in an idealized mouth-throat},
  Author                   = {Matida, E.A. and Finlay, W.H. and Lange, C.F. and Grgic, B.},
  Journal                  = {Journal of Aerosol Science},
  Year                     = {2004},
  Pages                    = {1-19},
  Volume                   = {35},

  Type                     = {Journal Article}
}

@Article{Matos1999,
  Title                    = {Large-eddy simulation of turbulent flow over a two-dimensional cavity with temperature fluctuations},
  Author                   = {Matos, Arlindo de and Pinho, Francisco A. A. and Silveira-Neto, Aristeu},
  Journal                  = {International Journal of Heat and Mass Transfer},
  Year                     = {1999},
  Number                   = {1},
  Pages                    = {49-59},
  Volume                   = {42},

  Abstract                 = {Large-eddy simulation of turbulent flow over a plane free cavity and over a plane symmetric cavity was performed. The Smagorinsky sub-grid scale turbulence model was used. The compressible formulation and the MacComarck finite volume method were employed. The dynamic and thermal behavior of a thermal capacitor were analyzed. This behavior presents strong dependence on the geometry and the temperature fluctuation frequency which is imposed in the entrance of capacitor and a little dependence on the Reynolds Number.},
  ISSN                     = {0017-9310},
  Type                     = {Journal Article},
  Url                      = {http://www.sciencedirect.com/science/article/B6V3H-45D32W4-5/2/5e18d12dfd21960969211c8c2d904e7b}
}

@Article{Matthys1990,
  Title                    = {Inhalation delivery of asthma drugs},
  Author                   = {Matthys, H.},
  Journal                  = {Lung},
  Year                     = {1990},
  Number                   = {Supp},
  Pages                    = {645-652},
  Volume                   = {1990},

  Type                     = {Journal Article}
}

@Article{Maugis1984,
  Title                    = {Surface forces, deformation and adherence at metal microcontacts},
  Author                   = {Maugis, D. and Pollock, H. M.},
  Journal                  = {Acta Metallurgica},
  Year                     = {1984},
  Number                   = {9},
  Pages                    = {1323-1334},
  Volume                   = {32},

  ISSN                     = {0001-6160},
  Type                     = {Journal Article},
  Url                      = {http://www.sciencedirect.com/science/article/pii/0001616084900786}
}

@Article{Mavriplis1997,
  Title                    = {Unstructured grid techniques},
  Author                   = {Mavriplis, D. J.},
  Journal                  = {Annual Review of Fluid Mechanics},
  Year                     = {1997},
  Number                   = {1},
  Pages                    = {473-514},
  Volume                   = {29},

  Doi                      = {doi:10.1146/annurev.fluid.29.1.473},
  Type                     = {Journal Article},
  Url                      = {http://www.annualreviews.org/doi/abs/10.1146/annurev.fluid.29.1.473}
}

@Article{Maxey1987,
  Title                    = {GRAVITATIONAL SETTLING OF AEROSOL PARTICLES IN HOMOGENEOUS TURBULENCE AND RANDOM FLOW FIELDS},
  Author                   = {Maxey, M. R.},
  Journal                  = {Journal of Fluid Mechanics},
  Year                     = {1987},
  Note                     = {Cited By (since 1996): 291
Export Date: 5 June 2011
Source: Scopus},
  Pages                    = {441-465},
  Volume                   = {174},

  Type                     = {Journal Article},
  Url                      = {http://www.scopus.com/inward/record.url?eid=2-s2.0-0023142873&partnerID=40&md5=a830488dff76f1866650ada9ca4ac797}
}

@Article{Maxey1983,
  Title                    = {Equation of motion for a small rigid sphere in a nonuniform flow},
  Author                   = {Maxey, M. R. and Riley, J. J.},
  Journal                  = {Physics of Fluids},
  Year                     = {1983},
  Note                     = {Cited By (since 1996): 868
Export Date: 5 June 2011
Source: Scopus},
  Number                   = {4},
  Pages                    = {883-889},
  Volume                   = {26},

  Type                     = {Journal Article},
  Url                      = {http://www.scopus.com/inward/record.url?eid=2-s2.0-0020572570&partnerID=40&md5=76ac7838354405e2fccaf7fe4ac23ab7}
}

@Article{May2007,
  Title                    = {Maya User Documentation},
  Author                   = {Maya},
  Journal                  = {Alias Systems Corporation},
  Year                     = {2007},

  Type                     = {Journal Article}
}

@Misc{Mazaheri2004,
  Title                    = {Modeling Inspiratory Particle Deposition},

  Author                   = {Mazaheri, A.R. and Ahmadi, G. },
  Month                    = {July 11-15, 2004},
  Year                     = {2004},

  Type                     = {Conference Paper}
}

@Misc{Mazaheri2003,
  Title                    = {Inspiratory Particle Deposition in the Upper Three Airway bifurcation},

  Author                   = {Mazaheri, A.R. and Ahmadi, G. },
  Month                    = {October 7 11, 2002},
  Year                     = {2003},

  Type                     = {Conference Paper}
}

@InProceedings{Mazumdar,
  Title                    = {Impact of moving bodies on airflow and contaminant transport inside aircraft cabins. },
  Author                   = {Mazumdar, S. and Chen, Q.},
  Booktitle                = {Proceedings of the 10th International Conference on Air Distribution in Rooms, Roomvent 2007},
  Pages                    = {165},

  Type                     = {Conference Proceedings}
}

@Article{McCarthy2007,
  Title                    = {Targeted delivery of multifunctional magnetic nanoparticles},
  Author                   = {McCarthy, J.R. and Kelly, K.A. and Sun, E.Y. and Weissleder, R.},
  Journal                  = {Nanomedicine},
  Year                     = {2007},
  Number                   = {2},
  Pages                    = {153-167},
  Volume                   = {2},

  Type                     = {Journal Article}
}

@Article{McCoy1977,
  Title                    = {Rate of deposition of droplets in annular two-phase flow},
  Author                   = {McCoy, D. D. and Hanratty, T. J.},
  Journal                  = {International Journal of Multiphase Flow},
  Year                     = {1977},
  Number                   = {4},
  Pages                    = {319-331},
  Volume                   = {3},

  ISSN                     = {0301-9322},
  Type                     = {Journal Article},
  Url                      = {http://www.sciencedirect.com/science/article/pii/030193227790012X}
}

@Article{McGurk2013,
  Title                    = {Reviews. Junqueira's Basic Histology Text and Atlas -- 13th edition},
  Author                   = {McGurk, Simon},
  Journal                  = {Nursing Standard},
  Year                     = {2013},
  Note                     = {book review; pictorial. Journal Subset: Double Blind Peer Reviewed; Europe; Expert Peer Reviewed; Nursing; Peer Reviewed; UK \& Ireland. Special Interest: Nursing Education. NLM UID: 8508427.},
  Number                   = {16},
  Pages                    = {34-34},
  Volume                   = {28},

  ISSN                     = {0029-6570},
  Keywords                 = {Histology
CD ROM},
  Type                     = {Journal Article},
  Url                      = {http://search.ebscohost.com/login.aspx?direct=true&db=c8h&AN=2012408768&site=ehost-live&scope=site}
}

@Article{McLaughlin1991,
  Title                    = {Inertial migration of a small sphere in linear shear flows},
  Author                   = {McLaughlin, J.B.},
  Journal                  = {Journal Fluid Mechanics},
  Year                     = {1991},
  Pages                    = {261-274},
  Volume                   = {224},

  Type                     = {Journal Article}
}

@Article{McLaughlin1989,
  Title                    = {Aerosol Particle Deposition in Numerically Simulated Channel Flow},
  Author                   = {McLaughlin, J.B. },
  Journal                  = {Physics of Fluids A},
  Year                     = {1989},
  Pages                    = {1211-1224},
  Volume                   = {7},

  Type                     = {Journal Article}
}

@Article{McLaughlin1994,
  Title                    = {Numerical computation of particles-turbulence interaction},
  Author                   = {McLaughlin, J. B.},
  Journal                  = {International Journal of Multiphase Flow},
  Year                     = {1994},
  Note                     = {Cited By (since 1996): 54
Export Date: 5 June 2011
Source: Scopus},
  Number                   = {SUPPL. 1},
  Pages                    = {211-232},
  Volume                   = {20},

  Type                     = {Journal Article},
  Url                      = {http://www.scopus.com/inward/record.url?eid=2-s2.0-0028002254&partnerID=40&md5=ec950289091c6703a2c3e73cc8dce03e}
}

@Article{McLaughlin1993,
  Title                    = {The lift on a small sphere in wall-bounded linear shear flows},
  Author                   = {McLaughlin, J. B.},
  Journal                  = {Journal of Fluid Mechanics},
  Year                     = {1993},
  Note                     = {Cited By (since 1996): 85
Export Date: 5 June 2011
Source: Scopus},
  Pages                    = {249-265},
  Volume                   = {246},

  Type                     = {Journal Article},
  Url                      = {http://www.scopus.com/inward/record.url?eid=2-s2.0-0027386883&partnerID=40&md5=490f69c49fd92da269aacdc36ffb9734}
}

@Article{McLennan2007,
  Title                    = {Virtual Bronchoscopy},
  Author                   = {McLennan, G. and Namati, E. and Ganatra, J. and Suter, M. and O'Brien, E.E. and Lecamwasam, K. and van Beek, E.J.R. and Hoffman, E.A.},
  Journal                  = {Imaging Decisions MRI},
  Year                     = {2007},
  Note                     = {10.1111/j.1617-0830.2007.00087.x},
  Number                   = {1},
  Pages                    = {10-20},
  Volume                   = {11},

  ISSN                     = {1617-0830},
  Type                     = {Journal Article},
  Url                      = {http://dx.doi.org/10.1111/j.1617-0830.2007.00087.x}
}

@Article{McNitt-Gray2002,
  Title                    = {AAPM/RSNA Physics Tutorial for Residents: Topics in CT},
  Author                   = {McNitt-Gray, M. F.},
  Journal                  = {Radiographics},
  Year                     = {2002},
  Pages                    = {1541-1553},
  Volume                   = {22},

  Type                     = {Journal Article}
}

@Article{Mead1970,
  Title                    = {Stress distribution in lungs: a model of pulmonary elasticity},
  Author                   = {Mead, J. and Takishima, T. and Leith, D.},
  Journal                  = {J Appl Physiol},
  Year                     = {1970},
  Number                   = {5},
  Pages                    = {596-608},
  Volume                   = {28},

  Type                     = {Journal Article},
  Url                      = {http://jap.physiology.org}
}

@Article{Mehel2010,
  Title                    = {The influence of an anisotropic Langevin dispersion model on the prediction of micro- and nanoparticle deposition in wall-bounded turbulent flows},
  Author                   = {Mehel, Amine and Tanière, Anne and Oesterlé, Benoît and Fontaine, Jean-Raymond},
  Journal                  = {Journal of Aerosol Science},
  Year                     = {2010},
  Number                   = {8},
  Pages                    = {729-744},
  Volume                   = {41},

  ISSN                     = {0021-8502},
  Keywords                 = {Aerosol deposition
Nanoparticles
Langevin model
CFD
Lagrangian method},
  Type                     = {Journal Article},
  Url                      = {http://www.sciencedirect.com/science/article/B6V6B-5010709-2/2/d6ea7eb0b74ccce5ef97a834b9fc1e42}
}

@Article{Mei1992,
  Title                    = {An approximate expression for the shear lift force on a spherical particle at finite reynolds number},
  Author                   = {Mei, R.},
  Journal                  = {International Journal of Multiphase Flow},
  Year                     = {1992},
  Note                     = {Cited By (since 1996): 156
Export Date: 5 June 2011
Source: Scopus},
  Number                   = {1},
  Pages                    = {145-147},
  Volume                   = {18},

  Type                     = {Journal Article},
  Url                      = {http://www.scopus.com/inward/record.url?eid=2-s2.0-0026742675&partnerID=40&md5=32cde9fc3b4efc3d9c3a7995043dd9f7}
}

@Article{Meiselman1967,
  Title                    = {Effect of Dextran on Rheology of Human Blood – Low Shear Viscometry},
  Author                   = {Meiselman, H.J. and Merrill, E.W. and Salzman, E. and Gilliland, E.R. and Pelletier, G.A. },
  Journal                  = {Journal of Applied Physiology},
  Year                     = {1967},
  Pages                    = {480-486},
  Volume                   = {22},

  Type                     = {Journal Article}
}

@Article{Melchionna2011,
  Title                    = {Incorporation of smooth spherical bodies in the Lattice Boltzmann method},
  Author                   = {Melchionna, Simone},
  Journal                  = {Journal of Computational Physics},
  Year                     = {2011},
  Number                   = {10},
  Pages                    = {3966-3976},
  Volume                   = {230},

  Abstract                 = {A method to simulate bodies suspended in a Lattice Boltzmann solvent is proposed. It is based on a generalized reaction force that enforces no-slip boundary conditions at the fluid-body interface as the limiting case of an iterative procedure. A smooth version of the Heaviside function allows to treat spherical particles of arbitrary size and produces smooth hydrodynamic forces as particles move in the continuum. Numerical tests demonstrate the accuracy of the method in reproducing the hydrodynamic field around a single particle and the fluid-mediated forces between pairs of particles. The drag force experienced by a particle moving in a straight channel and at various Reynolds numbers is studied as a non-trivial testcase.},
  Doi                      = {10.1016/j.jcp.2011.02.021},
  ISSN                     = {0021-9991},
  Keywords                 = {Lattice Boltzmann
Particle suspensions
Immersed Boundary
Thermal fluctuations},
  Type                     = {Journal Article},
  Url                      = {http://www.sciencedirect.com/science/article/pii/S0021999111001100}
}

@Article{Mendelson2012,
  Title                    = {{Changes in the Facial Skeleton With Aging: Implications and Clinical
 Applications in Facial Rejuvenation}},
  Author                   = {Mendelson, Bryan and Wong, Chin-Ho},
  Journal                  = {{AESTHETIC PLASTIC SURGERY}},
  Year                     = {{2012}},

  Month                    = {{AUG}},
  Number                   = {{4}},
  Pages                    = {{753-760}},
  Volume                   = {{36}},

  Abstract                 = {{In principle, to achieve the most natural and harmonious rejuvenation of
 the face, all changes that result from the aging process should be
 corrected. Traditionally, soft tissue lifting and redraping have
 constituted the cornerstone of most facial rejuvenation procedures.
 Changes in the facial skeleton that occur with aging and their impact on
 facial appearance have not been well appreciated. Accordingly, failure
 to address changes in the skeletal foundation of the face may limit the
 potential benefit of any rejuvenation procedure. Correction of the
 skeletal framework is increasingly viewed as the new frontier in facial
 rejuvenation. It currently is clear that certain areas of the facial
 skeleton undergo resorption with aging. Areas with a strong
 predisposition to resorption include the midface skeleton, particularly
 the maxilla including the pyriform region of the nose, the superomedial
 and inferolateral aspects of the orbital rim, and the prejowl area of
 the mandible. These areas resorb in a specific and predictable manner
 with aging. The resultant deficiencies of the skeletal foundation
 contribute to the stigmata of the aging face. In patients with a
 congenitally weak skeletal structure, the skeleton may be the primary
 cause for the manifestations of premature aging. These areas should be
 specifically examined in patients undergoing facial rejuvenation and
 addressed to obtain superior aesthetic results.
 Level of Evidence IV This journal requires that authors assign a level
 of evidence to each article. For a full description of these
 Evidence-Based Medicine ratings, please refer to the Table of Contents
 or the online Instructions to Authors http://www.springer.com/00266.}},
  Address                  = {{233 SPRING ST, NEW YORK, NY 10013 USA}},
  Affiliation              = {{Mendelson, B (Reprint Author), Ctr Facial Plast Surg, 109 Mathoura Rd, Toorak, Vic 3142, Australia.
 Mendelson, Bryan, Ctr Facial Plast Surg, Toorak, Vic 3142, Australia.
 Wong, Chin-Ho, Mt Elizabeth Novena Hosp, Singapore 329563, Singapore.}},
  Author-email             = {{bmendelson@bigpond.com
 wchinho@hotmail.com}},
  Doc-delivery-number      = {{979BA}},
  Doi                      = {{10.1007/s00266-012-9904-3}},
  ISSN                     = {{0364-216X}},
  Journal-iso              = {{Aesthet. Plast. Surg.}},
  Keywords                 = {{Aging; Changes; Correction; Facial; Rejuvenation; Skeleton}},
  Keywords-plus            = {{POROUS HYDROXYAPATITE GRANULES; AGE-RELATED-CHANGES; CRANIOFACIAL
 SKELETON; AESTHETIC SURGERY; AUGMENTATION; GROWTH; SHAPE;
 HYDROXYLAPATITE; ATTRACTIVENESS; RHINOPLASTY}},
  Language                 = {{English}},
  Number-of-cited-references = {{50}},
  Publisher                = {{SPRINGER}},
  Research-areas           = {{Surgery}},
  Times-cited              = {{8}},
  Type                     = {{Review}},
  Unique-id                = {{ISI:000306788600001}},
  Web-of-science-categories = {{Surgery}}
}

@Article{Meng2005,
  Title                    = {Computer Simulation of the Pharyngeal Bolus Transport of Newtonian and Non-Newtonian Fluids},
  Author                   = {Meng, Y. and Rao, M. A. and Datta, A. K.},
  Journal                  = {Food and Bioproducts Processing},
  Year                     = {2005},
  Number                   = {4},
  Pages                    = {297-305},
  Volume                   = {83},

  Abstract                 = {A computational fluid dynamics (CFD) programme was used to study dysphagia, a swallowing disorder, and demonstrated that the rheological properties of a liquid affect the pharyngeal transport of a food bolus. A fully coupled, Newton-Raphson solution algorithm was used in conjunction with the Backward-Euler scheme. Concomitant axial and radial movement of the fluid bolus was assumed, and the force exerted by the base of the tongue was assumed to be linear. Boundary conditions were based on data published in the clinical literature. The properties of three fluid types were modelled: water ([rho] = 1000 kg m-3, [eta] = 0.001 Pa s), 250% w/v barium sulfate mixture ([rho] = 1800 kg m-3, [eta] = 0.150 Pa s), and and starch-thickened beverage (power law parameters K = 2.0 Pa sn, n = 0.7). Results show that when the base of the tongue pushes against the throat with the same amount of force, water is transported through the pharynx at a much higher flow rate than the barium sulphate mixture, causing parts of the water bolus to flow backwards. A typical starch-thickened beverage, which is a shear-thinning non-Newtonian fluid, undergoes much lower flow rates. Furthermore, under the same conditions, a smaller volume of the non-Newtonian bolus (2 mL compared to 20 mL of the Newtonian fluids) is passed through by the end of the swallow. Values for the time to swallow a critical bolus volume, tcv, show that non-Newtonian fluids increase swallowing time more effectively than Newtonian fluids and are thus safer to swallow for patients with dysphagia. These findings suggest that non-Newtonian foods may either slow down the swallowing process or trigger the subject to swallow a smaller amount, allowing the neuromuscular system more time to shut off air passages and reduce the risk of aspiration. Based on this simple CFD modelling of the swallowing process, the effects of fluid properties on bolus transit can be predicted.},
  ISSN                     = {0960-3085},
  Keywords                 = {CFD
pharyngeal transport
Newtonian
non-Newtonian
shear-thinning
starch},
  Type                     = {Journal Article},
  Url                      = {http://www.sciencedirect.com/science/article/B8JGD-4RTVVNF-7/2/d096e007b876e7dc3abced543efe39ef}
}

@Article{Menon1984,
  Title                    = {Model study of flow dynamics in human central airways. Part III: Oscillatory velocity profiles},
  Author                   = {Menon, A.S. and Weber, M.E. and Chang, H.K.},
  Journal                  = {Respiration Physiology},
  Year                     = {1984},
  Pages                    = {255-275},
  Volume                   = {55},

  Type                     = {Journal Article}
}

@InProceedings{Menter,
  Title                    = {Elements of Industrial Heat Transfer Predictions},
  Author                   = {Menter, F.R. and Esch, T.},
  Booktitle                = {16th Brazilian Congress of Mechanical Engineering (COBEM)},

  Type                     = {Conference Proceedings}
}

@InBook{Menter2003,
  Title                    = {Ten Years of Industrial Experience with the SST Turbulence Model},
  Author                   = {Menter, F.R. and Kuntz, M. and Langtry, R. B.},
  Editor                   = {Hanjalic, K. and Nagano, Y. and Tummers, M.},
  Pages                    = {625-632},
  Publisher                = {Begell House Inc.},
  Year                     = {2003},
  Type                     = {Book Section},

  Booktitle                = {Turbulence, Heat and Mass Transfer 4}
}

@Article{Menter2009,
  Title                    = {Review of the shear-stress transport turbulence model experience from an industrial perspective},
  Author                   = {Menter, Florian R.},
  Journal                  = {International Journal of Computational Fluid Dynamics},
  Year                     = {2009},
  Number                   = {4},
  Pages                    = {305 - 316},
  Volume                   = {23},

  ISSN                     = {1061-8562},
  Type                     = {Journal Article},
  Url                      = {http://www.informaworld.com/10.1080/10618560902773387}
}

@Article{Menter1994,
  Title                    = {Two-equation eddy-viscosity turbulence models for engineering applications},
  Author                   = {Menter, F. R.},
  Journal                  = {American Institute of Aeronautics and Astronautics Journal},
  Year                     = {1994},
  Number                   = {8},
  Pages                    = {1598-1605},
  Volume                   = {32},

  Type                     = {Journal Article}
}

@Article{Menter2010,
  Title                    = {The Scale-Adaptive Simulation Method for Unsteady Turbulent Flow Predictions. Part 1: Theory and Model Description},
  Author                   = {Menter, F. R. and Egorov, Y.},
  Journal                  = {Flow, Turbulence and Combustion},
  Year                     = {2010},
  Number                   = {1},
  Pages                    = {113-138},
  Volume                   = {85},

  Doi                      = {10.1007/s10494-010-9264-5},
  ISSN                     = {1386-6184},
  Keywords                 = {Turbulence model
Scale-adaptive simulation
SAS
Hybrid RANS–LES},
  Type                     = {Journal Article},
  Url                      = {http://dx.doi.org/10.1007/s10494-010-9264-5}
}

@Article{Menter2006,
  Title                    = {A Correlation-Based Transition Model Using Local Variables---Part I: Model Formulation},
  Author                   = {Menter, F. R. and Langtry, R. B. and Likki, S. R. and Suzen, Y. B. and Huang, P. G. and Volker, S.},
  Journal                  = {Journal of Turbomachinery},
  Year                     = {2006},
  Number                   = {3},
  Pages                    = {413-422},
  Volume                   = {128},

  Doi                      = {10.1115/1.2184352},
  Keywords                 = {computational fluid dynamics
correlation theory
turbines
blades
aerodynamics},
  Type                     = {Journal Article},
  Url                      = {http://link.aip.org/link/?JTM/128/413/1}
}

@Book{Mercer1973,
  Title                    = {Aerosol Technology in Hazard Evaluation of Airborne Particles},
  Author                   = {Mercer, T.T. },
  Publisher                = {Academic Press},
  Year                     = {1973},

  Address                  = {New York},

  Type                     = {Book}
}

@Article{Mickelson1998,
  Title                    = {Upper airway bypass surgery for obstructive sleep apnea syndrome.},
  Author                   = {Mickelson, S.A. },
  Journal                  = {Otolaryngol Clin North Am},
  Year                     = {1998},
  Number                   = {6},
  Pages                    = {1013-1023},
  Volume                   = {31},

  Type                     = {Journal Article}
}

@Article{Mihaescu2008,
  Title                    = {Large Eddy Simulation and Reynolds-Averaged Navier-Stokes modeling of flow in a realistic pharyngeal airway model: An investigation of obstructive sleep apnea},
  Author                   = {Mihaescu, Mihai and Murugappan, Shanmugam and Kalra, Maninder and Khosla, Sid and Gutmark, Ephraim},
  Journal                  = {Journal of Biomechanics},
  Year                     = {2008},
  Number                   = {10},
  Pages                    = {2279-2288},
  Volume                   = {41},

  Abstract                 = {Computational fluid dynamics techniques employing primarily steady Reynolds-Averaged Navier-Stokes (RANS) methodology have been recently used to characterize the transitional/turbulent flow field in human airways. The use of RANS implies that flow phenomena are averaged over time, the flow dynamics not being captured. Further, RANS uses two-equation turbulence models that are not adequate for predicting anisotropic flows, flows with high streamline curvature, or flows where separation occurs. A more accurate approach for such flow situations that occur in the human airway is Large Eddy Simulation (LES). The paper considers flow modeling in a pharyngeal airway model reconstructed from cross-sectional magnetic resonance scans of a patient with obstructive sleep apnea. The airway model is characterized by a maximum narrowing at the site of retropalatal pharynx. Two flow-modeling strategies are employed: steady RANS and the LES approach. In the RANS modeling framework both k-[epsilon] and k-[omega] turbulence models are used. The paper discusses the differences between the airflow characteristics obtained from the RANS and LES calculations. The largest discrepancies were found in the axial velocity distributions downstream of the minimum cross-sectional area. This region is characterized by flow separation and large radial velocity gradients across the developed shear layers. The largest difference in static pressure distributions on the airway walls was found between the LES and the k-[epsilon] data at the site of maximum narrowing in the retropalatal pharynx.},
  ISSN                     = {0021-9290},
  Keywords                 = {Pharyngeal airway
Obstructive sleep apnea
LES
RANS},
  Type                     = {Journal Article},
  Url                      = {http://www.sciencedirect.com/science/article/B6T82-4SN8VB1-5/2/897fedab28e75e106f95a6e5c9ddcf1c}
}

@Article{Mills2006,
  Title                    = {Do Inhaled Carbon Nanoparticles Translocate Directly into the Circulation in Humans?},
  Author                   = {Mills, Nicholas L. and Amin, Nadia and Robinson, Simon D. and Anand, Atul and Davies, John and Patel, Dilip and de la Fuente, Jesus M. and Cassee, Flemming R. and Boon, Nicholas A. and MacNee, William and Millar, Alistair M. and Donaldson, Ken and Newby, David E.},
  Journal                  = {Am. Journal Respir. Crit. Care Medicine},
  Year                     = {2006},
  Number                   = {4},
  Pages                    = {426-431},
  Volume                   = {173},

  Abstract                 = {Rationale: Increased exposure to particulate air pollution (PM10) is a risk factor for death and hospitalization with cardiovascular disease. It has been suggested that the nanoparticulate component of PM10 is capable of translocating into the circulation with the potential for direct effects on the vasculature. Objective: The study's aim was to determine the extent to which inhaled technetium-99m (99mTc)-labeled carbon nanoparticles (Technegas) were able to access the systemic circulation. Methods and Main Results: Ten healthy volunteers inhaled Technegas and blood samples were taken sequentially over the following 6 h. Technegas particles were 4-20 nm in diameter and aggregated to a median particle diameter of approximately 100 nm. Radioactivity was immediately detected in blood, with levels increasing over 60 min. Thin-layer chromatography of whole blood identified a species that moved with the solvent front, corresponding to unbound 99mTc-pertechnetate, which was excreted in urine. There was no evidence of particle-bound 99mTc at the origin. {gamma} Camera images demonstrated high levels of Technegas retention (95.6 {+/-} 1.7% at 6 h) in the lungs, with no accumulation of radioactivity detected over the liver or spleen. Conclusions: The majority of 99mTc-labeled carbon nanoparticles remain within the lung up to 6 h after inhalation. In contrast to previous published studies, thin-layer chromatography did not support the hypothesis that inhaled Technegas carbon nanoparticles pass directly from the lungs into the systemic circulation.},
  Doi                      = {10.1164/rccm.200506-865OC},
  Type                     = {Journal Article},
  Url                      = {http://ajrccm.atsjournals.org/cgi/content/abstract/173/4/426}
}

@Article{Mitakakis2000,
  Title                    = {Personal exposure to allergenic pollen and mould spores in inland New South Wales, Australia},
  Author                   = {Mitakakis, T.Z. and Tovey, E.R. and Xuan, W. and Marks, G.B.},
  Journal                  = {Clin. And Experimental Allergy},
  Year                     = {2000},
  Pages                    = {1733-1739},
  Volume                   = {30},

  Type                     = {Journal Article}
}

@Article{Mitchell1999,
  Title                    = {Assessment of the dynamic relationship between external diameter and lumen flow in isolated bronchi},
  Author                   = {Mitchell, H.W. and Gray, P.R.},
  Journal                  = {Respir Physiol},
  Year                     = {1999},
  Pages                    = {67-76},
  Volume                   = {116},

  Type                     = {Journal Article}
}

@Article{Mitsakou2005,
  Title                    = {Eulerian modelling of lung deposition with sectional representation of aerosol dynamics},
  Author                   = {Mitsakou, C. and Helmis, C. and Housiadas, C.},
  Journal                  = {Journal of Aerosol Science},
  Year                     = {2005},
  Note                     = {doi: DOI: 10.1016/j.jaerosci.2004.08.008},
  Number                   = {1},
  Pages                    = {75-94},
  Volume                   = {36},

  ISSN                     = {0021-8502},
  Keywords                 = {Aerosol dynamics
Lung deposition
Inhalation dosimetry
Sectional method},
  Type                     = {Journal Article},
  Url                      = {http://www.sciencedirect.com/science/article/B6V6B-4DTKCHT-2/2/fbdb244117661cef4bbe996d8a922566}
}

@Article{Mitsakou2007,
  Title                    = {A Simple Mechanistic Model of Deposition of Water-Soluble Aerosol Particles in the Mouth and Throat},
  Author                   = {Mitsakou, C. and Mitrakos, D. and Neofytou, P. and Housiadas, C. },
  Journal                  = {Journal of Aerosol Medicine},
  Year                     = {2007},
  Number                   = {4 (December 1)},
  Pages                    = {519-29},
  Volume                   = {20},

  Type                     = {Journal Article}
}

@Article{Mittal2014,
  Title                    = {Insights into direct nose to brain delivery: current status and future perspective},
  Author                   = {Mittal, Deepti and Ali, Asgar and Md, Shadab and Baboota, Sanjula and Sahni, Jasjeet K. and Ali, Javed},
  Journal                  = {Drug Delivery},
  Year                     = {2014},
  Note                     = {C:\Users\sean\AppData\Roaming\Zotero\Zotero\Profiles\16a4oype.default\zotero\storage\3CCSXUID\Mittal et al. - 2013 - Insights into direct nose to brain delivery curre.pdf},
  Pages                    = {75-86},
  Volume                   = {21},

  Abstract                 = {Now a day's intranasal (i.n) drug delivery is emerging as a reliable method to bypass the blood-brain barrier (BBB) and deliver a wide range of therapeutic agents including both small and large molecules, growth factors, viral vectors and even stem cells to the brain and has shown therapeutic effects in both animals and humans. This route involves the olfactory or trigeminal nerve systems which initiate in the brain and terminate in the nasal cavity at the olfactory neuroepithelium or respiratory epithelium. They are the only externally exposed portions of the central nervous system (CNS) and therefore represent the most direct method of noninvasive entry into the brain. This approach has been primarily used to explore therapeutic avenues for neurological diseases. The potential for treatment possibilities with olfactory transfer of drugs will increase as more effective formulations and delivery devices are developed. Recently, the apomorphine hydrochloride dry powders have been developed for i.n. delivery (Apomorphine nasal, Lyonase technology, Britannia Pharmaceuticals, Surrey, UK). The results of clinical trial Phase III suggested that the prepared formulation had clinical effect equivalent to subcutaneously administered apomorphine. In coming years, intranasal delivery of drugs will demand more complex and automated delivery devices to ensure accurate and repeatable dosing. Thus, new efforts are needed to make this noninvasive route of delivery more efficient and popular, and it is also predicted that in future a range of intranasal products will be used in diagnosis as well as treatment of CNS diseases. This review will embark the existing evidence of nose-to-brain transport. It also provides insights into the most relevant pre-clinical studies of direct nose-brain delivery and delivery devices which will provide relative success of intranasal delivery system. We have, herein, outlined the relevant aspects of CNS drugs given intranasally to direct the brain in treating CNS disorders like Alzheimer's disease, depression, migraine, schizophrenia, etc.},
  Doi                      = {10.3109/10717544.2013.838713},
  ISSN                     = {1071-7544},
  Keywords                 = {alzheimers-disease
Blood-brain barrier
brain targeting
central-nervous-system
chitosan nanoparticles
clinical-efficacy
drug-delivery
focal cerebral-ischemia
intranasal delivery
intranasal mucoadhesive microemulsions
lipid nanoparticles
multiple-sclerosis
nanoparticles
olfactory pathway
targeting efficiency},
  Type                     = {Journal Article},
  Url                      = {http://informahealthcare.com/doi/pdfplus/10.3109/10717544.2013.838713}
}

@Article{Mlynski2004,
  Title                    = {Numerical simulation of airflow in the human nose},
  Author                   = {Mlynski, Ivo Weinhold · Gunter},
  Journal                  = {Eur Arch Otorhinolaryngol},
  Year                     = {2004},

  Type                     = {Journal Article}
}

@Article{Mobley2014,
  Title                    = {Aging in the olfactory system },
  Author                   = {Arie S. Mobley and Diego J. Rodriguez-Gil and Fumiaki Imamura and Charles A. Greer},
  Journal                  = {Trends in Neurosciences },
  Year                     = {2014},
  Number                   = {2},
  Pages                    = {77 - 84},
  Volume                   = {37},

  Abstract                 = {With advancing age, the ability of humans to detect and discriminate odors declines. In light of the rapid progress in analyzing molecular and structural correlates of developing and adult olfactory systems, the paucity of information available on the aged olfactory system is startling. A rich literature documents the decline of olfactory acuity in aged humans, but the underlying cellular and molecular mechanisms are largely unknown. Using animal models, preliminary work is beginning to uncover differences between young and aged rodents that may help address the deficits seen in humans, but many questions remain unanswered. Recent studies of odorant receptor (OR) expression, synaptic organization, adult neurogenesis, and the contribution of cortical representation during aging suggest possible underlying mechanisms and new research directions. },
  Doi                      = {http://dx.doi.org/10.1016/j.tins.2013.11.004},
  ISSN                     = {0166-2236},
  Keywords                 = {adult neurogenesis},
  Url                      = {http://www.sciencedirect.com/science/article/pii/S0166223613002245}
}

@Article{Modarress1984,
  Author                   = {Modarress, D. and Wuerer, J. and Elghobashi, S. },
  Journal                  = {Chemical Engineering Communications},
  Year                     = {1984},
  Pages                    = {341-354},
  Volume                   = {28},

  Type                     = {Journal Article}
}

@Article{Moeller2014,
  Title                    = {Drug Delivery to Paranasal Sinuses Using Pulsating Aerosols},
  Author                   = {Moeller, Winfried and Schuschnig, Uwe and Bartenstein, Peter and Meyer, Gabriele and Haeussinger, Karl and Schmid, Otmar and Becker, Sven},
  Journal                  = {Journal of Aerosol Medicine and Pulmonary Drug Delivery},
  Year                     = {2014},
  Pages                    = {255-263},
  Volume                   = {27},

  Abstract                 = {Chronic rhinosinusitis (CRS) is the major disorder of the upper airways, affecting about 10-15% of the total population. Topical treatment regimens show only modest efficacy, because drug delivery to the posterior nose and paranasal sinuses is still a challenge. Therefore, there is a high rate of functional endoscopic sinus surgery in CRS patients. Most nasally administered aerosolized drugs, like nasal pump sprays, are efficiently filtered by the nasal valve and do not reach the posterior nasal cavity and the sinuses, which are poorly ventilated. However, as highlighted in this review, sinus ventilation and paranasal aerosol delivery can be achieved by using pulsating airflow, offering new topical treatment options for nasal disorders. Radioaerosol inhalation and imaging studies in nasal casts and in healthy volunteers have shown 4-6% of the nasally administered dose within the sinuses. In CRS patients, significant aerosol deposition in the sinus cavities was reported before sinus surgery. After surgery, deposition increased to the amount observed in healthy volunteers. In addition, compared with nasal pump sprays, retention kinetics of the radiolabel deposited in the nasal cavity was prolonged, both in healthy volunteers and in CRS patients. These efficiencies may be sufficient for topical aerosol therapies of sinus disorders and, due to the prolonged retention kinetics, may reduce application modes, but have to be proven in future clinical trials. Pulsating aerosols may offer additional new topical treatment options of nasal and sinus disorders before as well as after surgery.},
  ISSN                     = {1941-2711},
  Keywords                 = {chronic rhinosinusitis, cystic-fibrosis, air-flow, nasal irrigation, controlled-trial, disease, sinusitis, asthma, ventilation, inhalation
chronic rhinosinusitis, topical therapy, pulsating aerosol, paranasal sinus ventilation, clearance},
  Type                     = {Journal Article}
}

@Article{Moghadas2011,
  Title                    = {Numerical investigation of septal deviation effect on deposition of nano/microparticles in human nasal passage},
  Author                   = {Moghadas, H. and Abouali, O. and Faramarzi, A. and Ahmadi, G.},
  Journal                  = {Respiratory Physiology \& Neurobiology},
  Year                     = {2011},
  Number                   = {1},
  Pages                    = {9-18},
  Volume                   = {177},

  Abstract                 = {Three dimensional computational models of both sides of human nasal passages were developed to investigate the effect of septal deviation on the flow patterns and deposition of micro/nano-particles in the realistic human nasal airways before and after septoplasty. A series of coronal CT scan images from a live 25-year old nonsmoking male with septal deviation in his right nasal passage was used to construct the model. For low to moderate activities, the steady airflows through the nasal passages were simulated. Eulerian and Lagrangian approaches were used, respectively, for nano- and micro-particles. The results show that the flow field and particle deposition strongly depend on the passage geometry especially for micro particles. In particular, the deposition rate in the passage with septal deviation was much higher compared with those in the normal (left) passage and the postoperative passage. Despite the similarity of total micro-particle deposition in the postoperative and the normal cavities, the regional deposition patterns were quite different in these passages. The deposition of nano-particles, however, showed similar trends in the postoperative right nasal passage and the normal left passage. The simulation results showed that in addition to the major alteration of the airflow pattern after the septoplasty operation, there are significant changes in the deposition pattern of nano- and micro-particles. Despite the anatomical differences between the available experimental configuration and the present computer model, the simulation results for the deposition efficiency of particles of different sizes are in qualitative agreement with the available data.},
  Doi                      = {10.1016/j.resp.2011.02.011},
  ISSN                     = {1569-9048},
  Keywords                 = {Nasal airway
Septal deviation
CFD
Nano-particle
Micro-particle},
  Type                     = {Journal Article},
  Url                      = {http://www.sciencedirect.com/science/article/pii/S1569904811000619}
}

@Article{Mohanarangam2009,
  Title                    = {Numerical study of particle turbulence interaction in liquid-particle flows},
  Author                   = {Mohanarangam, K. and Tu, J.Y.},
  Journal                  = {AIChE Journal},
  Year                     = {2009},
  Note                     = {10.1002/aic.11729},
  Number                   = {5},
  Pages                    = {1298-1302},
  Volume                   = {55},

  ISSN                     = {1547-5905},
  Type                     = {Journal Article},
  Url                      = {http://dx.doi.org/10.1002/aic.11729}
}

@Article{Moller2006,
  Title                    = {Mucociliary and long-term particle clearance in airways of patients with immotile cilia},
  Author                   = {Moller, Winfried and HauSZinger, Karl and Ziegler-Heitbrock, Loms and Heyder, Joachim},
  Journal                  = {Respiratory Research},
  Year                     = {2006},
  Number                   = {1},
  Pages                    = {10},
  Volume                   = {7},

  Abstract                 = {Spherical monodisperse ferromagnetic iron oxide particles of 1.9 mum geometric and 4.2 mum aerodynamic diameter were inhaled by seven patients with primary ciliary dyskinesia (PCD) using the shallow bolus technique, and compared to 13 healthy non-smokers (NS) from a previous study. The bolus penetration front depth was limiting to the phase1 dead space volume. In PCD patients deposition was 58+/-8 % after 8 s breath holding time. Particle retention was measured by the magnetopneumographic method over a period of nine months. Particle clearance from the airways showed a fast and a slow phase. In PCD patients airway clearance was retarded and prolonged, 42+/-12 % followed the fast phase with a mean half time of 16.8+/-8.6 hours. The remaining fraction was cleared slowly with a half time of 121+/-25 days. In healthy NS 49+/-9 % of particles were cleared in the fast phase with a mean half time of 3.0+/-1.6 hours, characteristic of an intact mucociliary clearance. There was no difference in the slow clearance phase between PCD patients and healthy NS. Despite non-functioning cilia the effectiveness of airway clearance in PCD patients is comparable to healthy NS, with a prolonged kinetics of one week, which may primarily reflect the effectiveness of cough clearance. This prolonged airway clearance allows longer residence times of bacteria and viruses in the airways and may be one reason for increased frequency of infections in PCD patients.},
  ISSN                     = {1465-9921},
  Type                     = {Journal Article},
  Url                      = {http://respiratory-research.com/content/7/1/10}
}

@Article{Mollinger1996,
  Title                    = {Measurement of the lift force on a particle fixed to the wall in the viscous sublayer of a fully developed turbulent boundary layer},
  Author                   = {Mollinger, A. M. and Nieuwstadt, F. T. M.},
  Journal                  = {Journal of Fluid Mechanics},
  Year                     = {1996},
  Note                     = {Cited By (since 1996): 34
Export Date: 5 June 2011
Source: Scopus},
  Pages                    = {285-306},
  Volume                   = {316},

  Type                     = {Journal Article},
  Url                      = {http://www.scopus.com/inward/record.url?eid=2-s2.0-0030153057&partnerID=40&md5=873cdbf6aa45b631be547de16041d6c2}
}

@Article{Mols1998,
  Title                    = {A turbulent diffusion model for particle dispersion and deposition in horizontal tube flow},
  Author                   = {Mols, B. and Oliemans, R. V. A.},
  Journal                  = {International Journal of Multiphase Flow},
  Year                     = {1998},
  Note                     = {doi: DOI: 10.1016/S0301-9322(97)00043-8},
  Number                   = {1},
  Pages                    = {55-75},
  Volume                   = {24},

  ISSN                     = {0301-9322},
  Keywords                 = {particle deposition
particle dispersion
deposition rate
turbulent diffusion
annular flow
horizontal flow},
  Type                     = {Journal Article},
  Url                      = {http://www.sciencedirect.com/science/article/B6V45-3SYPPV0-4/2/c375e5d687cbe2dfb1e7df3317851e7f}
}

@Article{Moon2009,
  Title                    = {Air flow and pressure inside a pressure-swirl spray and their effects on spray development},
  Author                   = {Moon, Seoksu and Abo-Serie, Essam and Bae, Choongsik},
  Journal                  = {Experimental Thermal and Fluid Science},
  Year                     = {2009},
  Note                     = {doi: DOI: 10.1016/j.expthermflusci.2008.08.005},
  Number                   = {2},
  Pages                    = {222-231},
  Volume                   = {33},

  ISSN                     = {0894-1777},
  Keywords                 = {Pressure-swirl spray
Air flow
Static air pressure
Swirl number
Recirculation vortex},
  Type                     = {Journal Article},
  Url                      = {http://www.sciencedirect.com/science/article/B6V34-4TCR1P2-1/2/9da613d7c6a1f42f2d869ed39a496e4a}
}

@Book{Moore2006,
  Title                    = {Clinically Oriented Anatomy},
  Author                   = {Moore, K. L. and Dalley, A. F.},
  Publisher                = {Lippincott Williams and Wilkins},
  Year                     = {2006},

  Address                  = {Baltimore},

  Type                     = {Book}
}

@Article{Mora1982,
  Title                    = {Aerosol and gas deposition to fully rough surfaces: Filtration model for blade-shaped elements},
  Author                   = {de la Mora, J. Fernandez and Friedlander, S. K.},
  Journal                  = {International Journal of Heat and Mass Transfer},
  Year                     = {1982},
  Number                   = {11},
  Pages                    = {1725-1735},
  Volume                   = {25},

  ISSN                     = {0017-9310},
  Type                     = {Journal Article},
  Url                      = {http://www.sciencedirect.com/science/article/pii/0017931082901521}
}

@Article{Morais-Almeida2013,
  Title                    = {Prevalence and classification of rhinitis in the elderly: a nationwide survey in Portugal},
  Author                   = {Morais-Almeida, M. and Pite, H. and Pereira, A. M. and Todo-Bom, A. and Nunes, C. and Bousquet, J. and Fonseca, J.},
  Journal                  = {Allergy},
  Year                     = {2013},
  Number                   = {9},
  Pages                    = {1150--1157},
  Volume                   = {68},

  Doi                      = {10.1111/all.12207},
  ISSN                     = {1398-9995},
  Keywords                 = {ARIA, classification, elderly, prevalence, rhinitis},
  Url                      = {http://dx.doi.org/10.1111/all.12207}
}

@Article{Morgan1997,
  Title                    = {Health effects of diesel emissions},
  Author                   = {Morgan, W. K. C. and Reger, R. B. and Tucker, D. M.},
  Journal                  = {Ann Occup Hyg},
  Year                     = {1997},
  Number                   = {6},
  Pages                    = {643-658},
  Volume                   = {41},

  Abstract                 = {We have reviewed the literature relating to the health effects of diesel emissions with particular reference to acute and chronic morbidity and to carcinogenicity. It is apparent that exposure to diesel fumes in sufficient concentrations may lead to eye and nasal irritation but there is no evidence of any permanent effect. A transient decline of ventilatory capacity has been noted following such exposures. There is also some evidence that the chronic inhalation of diesel fumes leads to the development of cough and sputum, that is chronic bronchitis, however, it is usually impossible to show a cause and effect relationship because of the concomitant and confounding exposures to mine dust and cigarette smoke. Although there have been a number of papers suggesting that diesel fumes may act as an carcinogen, the weight of the evidence is against this hypothesis. Finally, the role of small particles, less than 10 {micro}m, which are frequently present in diesel emissions requires further study since there is some limited evidence that they may be partly responsible for some of the exacerbations of asthma. (C) 1997 British Occupational Hygiene Society. Published by Elsevier Science Ltd.},
  Doi                      = {10.1093/annhyg/41.6.643},
  Type                     = {Journal Article},
  Url                      = {http://annhyg.oxfordjournals.org/cgi/content/abstract/41/6/643}
}

@Article{Morsi1972,
  Title                    = {An investigation of particle trajectories in two-phase flow systems.},
  Author                   = {Morsi, S.A. and Alexander, A.J.},
  Journal                  = {Journal Fluid Mechanics},
  Year                     = {1972},
  Number                   = {2},
  Pages                    = {193-208},
  Volume                   = {55},

  Type                     = {Journal Article}
}

@Article{Mortensen1983,
  Title                    = {A numerical identification system for airways in the lung},
  Author                   = {Mortensen, J.D. and Young, J.D. and Stout, L. and Stout, A. and Bagley, B. and Schaap, R.N.},
  Journal                  = {The Anatomical Record},
  Year                     = {1983},
  Pages                    = {103-114},
  Volume                   = {206},

  Type                     = {Journal Article}
}

@Article{Mortensen2009,
  Title                    = {Assessment of the finite volume method applied to the <I>v</I><SUP><FONT SIZE='-1'>2</FONT></SUP> - <I>f</I> model},
  Author                   = {Mortensen, Mikael and Reif, Bjørn Anders Pettersson and Wasberg, Carl Erik},
  Journal                  = {International Journal for Numerical Methods in Fluids},
  Year                     = {2009},
  Note                     = {10.1002/fld.2091},
  Number                   = {9999},
  Pages                    = {n/a},
  Volume                   = {9999},

  ISSN                     = {1097-0363},
  Type                     = {Journal Article},
  Url                      = {http://dx.doi.org/10.1002/fld.2091}
}

@Article{Mortensen2009a,
  Title                    = {Assessment of the finite volume method applied to the v2f model},
  Author                   = {Mortensen, Mikael and Reif, Bjørn Anders Pettersson and Wasberg, Carl Erik},
  Journal                  = {International Journal for Numerical Methods in Fluids},
  Year                     = {2009},
  Note                     = {10.1002/fld.2091},
  Number                   = {9999},
  Pages                    = {n/a},
  Volume                   = {9999},

  ISSN                     = {1097-0363},
  Type                     = {Journal Article},
  Url                      = {http://dx.doi.org/10.1002/fld.2091}
}

@Article{Moser1999,
  Title                    = {Direct numerical simulation of turbulent channel flow up to Re[sub tau] = 590},
  Author                   = {Moser, Robert D. and Kim, John and Mansour, Nagi N.},
  Journal                  = {Physics of Fluids},
  Year                     = {1999},
  Number                   = {4},
  Pages                    = {943-945},
  Volume                   = {11},

  Doi                      = {10.1063/1.869966},
  Keywords                 = {turbulence
channel flow
flow simulation},
  Type                     = {Journal Article},
  Url                      = {http://link.aip.org/link/?PHF/11/943/1}
}

@Article{Moshfegh2010,
  Title                    = {A new expression for spherical aerosol drag in slip flow regime},
  Author                   = {Moshfegh, Abouzar and Shams, Mehrzad and Ahmadi, Goodarz and Ebrahimi, Reza},
  Journal                  = {Journal of Aerosol Science},
  Year                     = {2010},
  Number                   = {4},
  Pages                    = {384-400},
  Volume                   = {41},

  Abstract                 = {A 3D simulation study for an incompressible slip flow around a spherical aerosol particle was performed. The full Navier-Stokes equations were solved and the velocity jump at the gas-particle interface was treated numerically by imposition of the slip boundary condition. Analytical solution to the Stokesian slip flow past a spherical particle was used as a benchmark for code verification, and excellent agreement was achieved. The simulation results showed that in addition to the Knudsen number, the Reynolds number affects the slip correction factor. Thus, the Cunningham-based slip corrections must be augmented by the inclusion of the effect of Reynolds number for application to Lagrangian tracking of fine particles. A new expression for the slip correction factor as a function of both Knudsen number and Reynolds number was developed. The particle total drag coefficient was also correlated against Re and Kn over the range of gas-particle relative speeds yielding the incompressible slip flow from the Stokesian regime up to the threshold of compressibility. Inclusion of gas slip on the particle surface enhances the accuracy of particle drag force prediction up to 40.9% in the range of 0.01<Kn<0.1 and 0.125<Re<20 compared to the no-slip continuum drag values.},
  ISSN                     = {0021-8502},
  Keywords                 = {Spherical aerosol
CFD
Cunningham correction
Slip correction factor
Drag coefficient},
  Type                     = {Journal Article},
  Url                      = {http://www.sciencedirect.com/science/article/B6V6B-4YDKK0B-1/2/77cf596d90b45dd27e9b58e31ad7b3ee}
}

@Article{Moshfegh2009,
  Title                    = {A novel surface-slip correction for microparticles motion},
  Author                   = {Moshfegh, Abouzar and Shams, Mehrzad and Ahmadi, Goodarz and Ebrahimi, Reza},
  Journal                  = {Colloids and Surfaces A: Physicochemical and Engineering Aspects},
  Year                     = {2009},
  Number                   = {1-3},
  Pages                    = {112-120},
  Volume                   = {345},

  Abstract                 = {As the fine particles expose to the rarefied regimes, the solid-surface effects become prominent and change the interfacial interactions between the gas and particle. The particle drag force, which is mainly concerned in aerosol researches, is affected through the changes observed in the way that fluid entities reflect from the particle surface. A 3D simulation study for an incompressible rarefied slip flow past a spherical aerosol particle was performed. The full Navier-Stokes equations were solved and the velocity jump at the gas-particle interface was treated numerically by imposition of the slip boundary condition. Analytical solution to the Stokesian slip flow past a spherical particle was used as a benchmark for code verification, and reasonable agreement was achieved. The simulation results showed that in addition to the Knudsen number, the Reynolds number affects the slip correction factor. Thus, the Cunningham-based slip corrections must be augmented by the inclusion of the effect of Reynolds number for application to Lagrangian tracking of fine particles. A novel expression for the surface-slip correction factor as a function of both Knudsen number (0.01 < Kn < 0.1) and Reynolds number (0 < Re < 1) was developed.},
  ISSN                     = {0927-7757},
  Keywords                 = {Gas-particle interface
CFD
Cunningham correction
Surface-slip correction},
  Type                     = {Journal Article},
  Url                      = {http://www.sciencedirect.com/science/article/B6TFR-4W6Y7YK-4/2/a3b0ebd9842b2289b064282de074304a}
}

@Article{Moshkin2014,
  Title                    = {Nasal aerodynamics protects brain and lung from inhaled dust in subterranean diggers, Ellobius talpinus.},
  Author                   = {Moshkin, M. P. and Petrovski, D. V. and Akulov, A. E. and Romashchenko, A. V. and Gerlinskaya, L. A. and Ganimedov, V. L. and Muchnaya, M. I. and Sadovsky, A. S. and Koptyug, I. V. and Savelov, A. A. and Troitsky, S. Yu and Moshkn, Y. M. and Bukhtiyarov, V. I. and Kolchanov, N. A. and Sagdeev, R. Z. and Fomin, V. M.},
  Journal                  = {Proceedings. Biological sciences / The Royal Society},
  Year                     = {2014},
  Volume                   = {281},

  Abstract                 = {Inhalation of air-dispersed sub-micrometre and nano-sized particles presents a risk factor for animal and human health. Here, we show that nasal aerodynamics plays a pivotal role in the protection of the subterranean mole vole Ellobius talpinus from an increased exposure to nano-aerosols. Quantitative simulation of particle flow has shown that their deposition on the total surface of the nasal cavity is higher in the mole vole than in a terrestrial rodent Mus musculus (mouse), but lower on the olfactory epithelium. In agreement with simulation results, we found a reduced accumulation of manganese in olfactory bulbs of mole voles in comparison with mice after the inhalation of nano-sized MnCl2 aerosols. We ruled out the possibility that this reduction is owing to a lower transportation from epithelium to brain in the mole vole as intranasal instillations of MnCl2 solution and hydrated nanoparticles of manganese oxide MnO · (H2O)x revealed similar uptake rates for both species. Together, we conclude that nasal geometry contributes to the protection of brain and lung from accumulation of air-dispersed particles in mole voles. 2014 The Author(s) Published by the Royal Society. All rights reserved.},
  Doi                      = {10.1098/rspb.2014.0919},
  ISSN                     = {1471-2954},
  Keywords                 = {adaptation to dust
Ellobius
Mus
nanoparticles
nasal aerodynamics
subterranean rodents},
  Type                     = {Journal Article},
  Url                      = {http://rspb.royalsocietypublishing.org/content/281/1792/20140919}
}

@Article{Moskal2002,
  Title                    = {Temporary and spatial deposition of aerosol particles in the upper human airways during breathing cycle},
  Author                   = {Moskal, Arkadiusz and Gradon, Leon},
  Journal                  = {Journal of Aerosol Science},
  Year                     = {2002},
  Note                     = {doi: DOI: 10.1016/S0021-8502(02)00108-8},
  Number                   = {11},
  Pages                    = {1525-1539},
  Volume                   = {33},

  ISSN                     = {0021-8502},
  Keywords                 = {Cyclic flow
Turbulence
Diffusion
Inertia
Bronchial tree},
  Type                     = {Journal Article},
  Url                      = {http://www.sciencedirect.com/science/article/B6V6B-475NSVM-2/2/b5d14dc8b4eeedb7c687eb837064c689}
}

@Article{Moskal2006,
  Title                    = {Estimation of the diffusion coefficient of aerosol particle aggregates using Brownian simulation in the continuum regime},
  Author                   = {Moskal, Arkadiusz and Payatakes, A. C.},
  Journal                  = {Journal of Aerosol Science},
  Year                     = {2006},
  Note                     = {doi: DOI: 10.1016/j.jaerosci.2005.10.005},
  Number                   = {9},
  Pages                    = {1081-1101},
  Volume                   = {37},

  ISSN                     = {0021-8502},
  Keywords                 = {Aggregates
Diffusion coefficient
Aerosol dynamics
Brownian dynamics},
  Type                     = {Journal Article},
  Url                      = {http://www.sciencedirect.com/science/article/B6V6B-4J2TSSG-1/2/6418f2d4fabd8286696ef843bb5dc68e}
}

@Article{Mossman2011,
  Title                    = {Pulmonary Endpoints (Lung Carcinomas and Asbestosis) Following Inhalation Exposure to Asbestos},
  Author                   = {Mossman, Brooke T. and Lippmann, Morton and Hesterberg, Thomas W. and Kelsey, Karl T. and Barchowsky, Aaron and Bonner, James C.},
  Journal                  = {Journal of Toxicology and Environmental Health, Part B},
  Year                     = {2011},
  Number                   = {1-4},
  Pages                    = {76-121},
  Volume                   = {14},

  Doi                      = {10.1080/10937404.2011.556047},
  ISSN                     = {1093-7404},
  Type                     = {Journal Article},
  Url                      = {http://dx.doi.org/10.1080/10937404.2011.556047}
}

@Article{Mouritz2009,
  Title                    = {Review of Smoke Toxicity of Fiber-Polymer Composites Used in Aircraft},
  Author                   = {Mouritz, A.},
  Journal                  = {Journal of Aircraft},
  Year                     = {2009},
  Number                   = {3},
  Pages                    = {737-745},
  Volume                   = {46},

  Type                     = {Journal Article}
}

@Article{Muhlfeld2008,
  Title                    = {Interactions of nanoparticles with pulmonary structures and cellular responses},
  Author                   = {Muhlfeld, Christian and Rothen-Rutishauser, Barbara and Blank, Fabian and Vanhecke, Dimitri and Ochs, Matthias and Gehr, Peter},
  Journal                  = {Am J Physiol Lung Cell Mol Physiol},
  Year                     = {2008},
  Number                   = {5},
  Pages                    = {L817-829},
  Volume                   = {294},

  Abstract                 = {Combustion-derived and synthetic nano-sized particles (NSP) have gained considerable interest among pulmonary researchers and clinicians for two main reasons. 1) Inhalation exposure to combustion-derived NSP was associated with increased pulmonary and cardiovascular morbidity and mortality as suggested by epidemiological studies. Experimental evidence has provided a mechanistic picture of the adverse health effects associated with inhalation of combustion-derived and synthetic NSP. 2) The toxicological potential of NSP contrasts with the potential application of synthetic NSP in technological as well as medicinal settings, with the latter including the use of NSP as diagnostics or therapeutics. To shed light on this paradox, this article aims to highlight recent findings about the interaction of inhaled NSP with the structures of the respiratory tract including surfactant, alveolar macrophages, and epithelial cells. Cellular responses to NSP exposure include the generation of reactive oxygen species and the induction of an inflammatory response. Furthermore, this review places special emphasis on methodological differences between experimental studies and the caveats associated with the dose metrics and points out ways to overcome inherent methodological problems.},
  Doi                      = {10.1152/ajplung.00442.2007},
  Type                     = {Journal Article},
  Url                      = {http://ajplung.physiology.org/cgi/content/abstract/294/5/L817}
}

@Article{Mui2009,
  Title                    = {Numerical modeling of exhaled droplet nuclei dispersion and mixing in indoor environments},
  Author                   = {Mui, K. W. and Wong, L. T. and Wu, C. L. and Lai, Alvin C. K.},
  Journal                  = {Journal of Hazardous Materials},
  Year                     = {2009},
  Number                   = {1-3},
  Pages                    = {736-744},
  Volume                   = {167},

  Abstract                 = {The increasing incidence of indoor airborne infections has prompted attention upon the investigation of expiratory droplet dispersion and transport in built environments. In this study, a source (i.e. a patient who generates droplets) and a receiver (i.e. a susceptible object other than the source) are modeled in a mechanically ventilated room. The receiver's exposure to the droplet nuclei is analyzed under two orientations relative to the source. Two droplet nuclei, 0.1 and 10 [mu]m, with different emission velocities, are selected to represent large expiratory droplets which can still be inhaled into the human respiratory tracts. The droplet dispersion and mixing characteristics under well-mixed and displacement ventilation schemes are evaluated and compared numerically. Results show that the droplet dispersion and mixing under displacement ventilation is consistently poorer. Very low concentration regions are also observed in the displacement scheme. For both ventilation schemes, the intake dose will be reduced substantially if the droplets are emitted under the face-to-wall orientation rather than the face-to-face orientation. Implications of using engineering strategies for reducing exposure are briefly discussed.},
  ISSN                     = {0304-3894},
  Keywords                 = {Particle dispersion
Drift-flux model
Mixing
Ventilation},
  Type                     = {Journal Article},
  Url                      = {http://www.sciencedirect.com/science/article/B6TGF-4VDS8KK-9/2/f69d2d8c6ede528a7a630dd4ad57b246}
}

@Article{Mulivor2004,
  Title                    = {Inflammation- and Ischemia-Induced Shedding of Venular Glycocalyx},
  Author                   = {Mulivor, A.W. and Lipowsky, H.H. },
  Journal                  = {Amweican Journal of Physiology - Heart and Circulatory Physiology},
  Year                     = {2004},
  Pages                    = {1672-1680},
  Volume                   = {286},

  Type                     = {Journal Article}
}

@Article{Mumford1989,
  Title                    = {Optimal approximations by piecewise smooth functions and associated variational problems},
  Author                   = {Mumford, D. and Shah, J.},
  Journal                  = {Commincations Pure and Applied Mathematics},
  Year                     = {1989},
  Pages                    = {577-685},
  Volume                   = {XLII},

  Type                     = {Journal Article}
}

@Article{Murakami2004,
  Title                    = {Analysis and design of the micro-climate around the human body with respiration by CFD},
  Author                   = {Murakami, S.},
  Journal                  = {Indoor Air},
  Year                     = {2004},
  Pages                    = {144-156},
  Volume                   = {14},

  Type                     = {Journal Article}
}

@Article{Murakami1989,
  Title                    = {Numerical and experimental study on room airflow—3-D predictions using the k-ϵ turbulence model},
  Author                   = {Murakami, Shuzo and Kato, Shinsuke},
  Journal                  = {Building and Environment},
  Year                     = {1989},
  Number                   = {1},
  Pages                    = {85-97},
  Volume                   = {24},

  Doi                      = {10.1016/0360-1323(89)90019-x},
  ISSN                     = {0360-1323},
  Type                     = {Journal Article},
  Url                      = {http://www.sciencedirect.com/science/article/pii/036013238990019X}
}

@Article{Murakami2000,
  Title                    = {Combined simulation of airflow, radiation and moisture transport for heat release from a human body},
  Author                   = {Murakami, S. and Kato, S. and Zeng, J.},
  Journal                  = {Building and Environment},
  Year                     = {2000},
  Pages                    = {489-500},
  Volume                   = {35},

  Type                     = {Journal Article}
}

@InBook{Musante2001,
  Title                    = {Computational fluid dynamics in human lungs II. Effects of airway disease.},
  Author                   = {Musante, C.J. and Martonen, T.B.},
  Editor                   = {Martonen, T.B.},
  Pages                    = {147-164},
  Publisher                = {Wit Press},
  Year                     = {2001},

  Address                  = {Boston, MA},
  Type                     = {Book Section},

  Booktitle                = {Medical applications of computer modelling: The respiratory system }
}

@Book{Myers1993,
  Title                    = {Biological basis of facial plastic surgery},
  Author                   = {Myers, A.D.},
  Publisher                = {Thieme Medical Publisher},
  Year                     = {1993},

  Address                  = {New York, NY},

  Type                     = {Book}
}

@InBook{Mygind1982,
  Title                    = {Morphology of the upper airway epithelium},
  Author                   = {Mygind, N. and Pedersen, M. and Nielsen, H.},
  Editor                   = {Proctor, D.F. and Anderson, I.},
  Pages                    = {45-71},
  Publisher                = {Elsevier Biomedical Press},
  Year                     = {1982},

  Address                  = {New York},
  Type                     = {Book Section},

  Booktitle                = {The Nose }
}

@Article{Mylavarapu2009,
  Title                    = {Validation of computational fluid dynamics methodology used for human upper airway flow simulations},
  Author                   = {Mylavarapu, Goutham and Murugappan, Shanmugam and Mihaescu, Mihai and Kalra, Maninder and Khosla, Sid and Gutmark, Ephraim},
  Journal                  = {Journal of Biomechanics},
  Year                     = {2009},
  Note                     = {doi: DOI: 10.1016/j.jbiomech.2009.03.035},
  Number                   = {10},
  Pages                    = {1553-1559},
  Volume                   = {42},

  ISSN                     = {0021-9290},
  Keywords                 = {Pharyngeal airway
Obstructive sleep apnea
CFD
Validation},
  Type                     = {Journal Article},
  Url                      = {http://www.sciencedirect.com/science/article/B6T82-4WFPPKF-1/2/ca586dd9bec9d34862372921a1e2db79}
}

@Article{Na2010,
  Title                    = {Life Cycle Cost Analysis of Air Conditioning Systems in a Perimeter Zone for a Variable Air Volume System in Office Buildings},
  Author                   = {Na, Yeon-Jeong and Nam, Eun-Ji and Yang, In-Ho},
  Journal                  = {Journal of Asian Architecture and Building Engineering},
  Year                     = {2010},
  Number                   = {1},
  Pages                    = {243-250},
  Volume                   = {9},

  Type                     = {Journal Article}
}

@Article{Naftali2005,
  Title                    = {The air-conditioning capacity of the human nose},
  Author                   = {Naftali, S. and Rosenfeld, M. and Wolf, M. and Elad, D.},
  Journal                  = {Annals of Biomedical Engineering},
  Year                     = {2005},
  Pages                    = {545-553},
  Volume                   = {33},

  Type                     = {Journal Article}
}

@Article{Naftali1998,
  Title                    = {Transport phenomena in the human nasal cavity: A computational model},
  Author                   = {Naftali, S. and Schroter, R.C. and Shiner, J. and Elad, D.},
  Journal                  = {Annals of Biomedical Engineering},
  Year                     = {1998},
  Pages                    = {831-839},
  Volume                   = {26},

  Type                     = {Journal Article}
}

@Article{Nagels2009,
  Title                    = {Large eddy simulation of high frequency oscillating flow in an asymmetric branching airway model},
  Author                   = {Nagels, Martin A. and Cater, John E.},
  Journal                  = {Medical Engineering \& Physics},
  Year                     = {2009},
  Number                   = {9},
  Pages                    = {1148-1153},
  Volume                   = {31},

  Abstract                 = {The implementation of artificial ventilation schemes is necessary when respiration fails. One approach involves the application of high frequency oscillatory ventilation (HFOV) to the respiratory system. Oscillatory airflow in the upper bronchial tree can be characterized by Reynolds numbers as high as 104, hence, the flow presents turbulent features. In this study, transitional and turbulent flow within an asymmetric bifurcating model of the upper airway during HFOV are studied using large eddy simulation (LES) methods. The flow, characterized by a peak Reynolds number of 8132, is analysed using a validated LES model of a three-dimensional branching geometry. The pressures, velocities, and vorticity within the flow are presented and compared with prior models for branching flow systems. The results demonstrate how pendelluft occurs at asymmetric branches within the respiratory system. These results may be useful in optimising treatments using HFOV methods.},
  ISSN                     = {1350-4533},
  Keywords                 = {Bifurcating flow
Human lung
Three-dimensional modelling
HFOV
Artificial ventilation
Large eddy simulation},
  Type                     = {Journal Article},
  Url                      = {http://www.sciencedirect.com/science/article/B6T9K-4X3DS5G-1/2/1e6ef4a0b82376bb6acc2f7ceb3cfda0}
}

@Article{Nair2013,
  Title                    = {A Review of the Clinicopathological and Radiological Features of Unilateral Nasal Mass},
  Author                   = {Nair, Satish and James, E. and Awasthi, S. and Nambiar, Sapna and Goyal, Sunil},
  Journal                  = {Indian Journal of Otolaryngology and Head \& Neck Surgery},
  Year                     = {2013},
  Note                     = {C:\Users\sean\AppData\Roaming\Zotero\Zotero\Profiles\16a4oype.default\zotero\storage\AI34RRBG\Nair et al. - 2013 - A Review of the Clinicopathological and Radiologic.pdf},
  Pages                    = {199-204},
  Volume                   = {65},

  Doi                      = {10.1007/s12070-011-0288-5},
  ISSN                     = {2231-3796, 0973-7707},
  Keywords                 = {Head and Neck Surgery
Inverted papilloma
Mucocele
Otorhinolaryngology
Unilateral nasal polyp},
  Type                     = {Journal Article},
  Url                      = {http://download.springer.com/static/pdf/537/art%253A10.1007%252Fs12070-011-0288-5.pdf?auth66=1411539357_d4f8ebaf0f8753c04b73bfac6e3b2b69&ext=.pdf}
}

@Article{Nallasamy1987,
  Title                    = {Turbulence models and their applications to the prediction of internal flows: A review},
  Author                   = {Nallasamy, M.},
  Journal                  = {Computers \& Fluids},
  Year                     = {1987},
  Number                   = {2},
  Pages                    = {151-194},
  Volume                   = {15},

  Doi                      = {10.1016/s0045-7930(87)80003-8},
  ISSN                     = {0045-7930},
  Type                     = {Journal Article},
  Url                      = {http://www.sciencedirect.com/science/article/pii/S0045793087800038}
}

@Article{Nanduri2009,
  Title                    = {CFD mesh generation for biological flows: Geometry reconstruction using diagnostic images},
  Author                   = {Nanduri, Jagannath R. and Pino-Romainville, Francisco A. and Celik, Ismail},
  Journal                  = {Computers \& Fluids},
  Year                     = {2009},
  Number                   = {5},
  Pages                    = {1026-1032},
  Volume                   = {38},

  Abstract                 = {A new thrust in the use of CFD techniques for simulation of biological flows has necessitated the demand for robust grid generation techniques to characterize the complex geometries. While the techniques of image manipulation required are simple, most researchers in this field use proprietary 3rd party software for image manipulation and grid generation. In the current study, we propose a simple MATLAB based grid generation techniques suitable for CFD studies of external and internal biological flows such as blood flow and respiration and flows around the human body. As an example, the flow inside two specific intracranial aneurysms is modeled by generating CFD grids from 3D rotational angiography images. Specific issues of modeling, such as boundary conditions and location of flow inlets and outlets, in relation to the reconstructed geometry are discussed. The reconstructed arterial geometry including the aneurysm matches the visual representation generated by the angiogram software (Leonardo software). The calculated CFD flow patterns also show a good correlation to the flow visualization presented by the Leonardo software. Areas of high pressure and wall shear stress are identified. The same technique is also used to generate the CFD grid of a human trachea to study the particle dispersion patterns during a human cough cycle. The fluid is modeled using an actual human cough signal with the particles simulating the influenza virus. The flow pattern out of the mouth along with the dispersion pattern of the particles is validated against similar human experimental studies to track the spread of the disease through cough. Work is also currently underway to use the present grid generation program to construct a superficial mesh of the human body from MRI/CAT scan images of cadavers. The goal is to build an accurate and scalable model of the human body surface with articulate joints which can be posed in any environment to model the air flow patterns around the body.},
  ISSN                     = {0045-7930},
  Type                     = {Journal Article},
  Url                      = {http://www.sciencedirect.com/science/article/B6V26-4S8TB64-2/2/a56bdb1a3cfc09507d3ac480ede75c42}
}

@Article{Nasr2009,
  Title                    = {A DNS study of effects of particle-particle collisions and two-way coupling on particle deposition and phasic fluctuations},
  Author                   = {Nasr, Hojjat and Ahmadi, Goodarz and Mclaughlin, J.B.},
  Journal                  = {Journal of Fluid Mechanics},
  Year                     = {2009},
  Pages                    = {507-536},
  Volume                   = {640},

  Doi                      = {doi:10.1017/S0022112009992011},
  ISSN                     = {0022-1120},
  Type                     = {Journal Article},
  Url                      = {http://dx.doi.org/10.1017/S0022112009992011}
}

@Article{Nazridoust2008,
  Title                    = {Unsteady-State Airflow and Particle Deposition in a Three-Generation Human Lung Geometry},
  Author                   = {Nazridoust, Kambiz and Asgharian, Bahman},
  Journal                  = {Inhalation Toxicology},
  Year                     = {2008},
  Number                   = {6},
  Pages                    = {595 - 610},
  Volume                   = {20},

  ISSN                     = {0895-8378},
  Type                     = {Journal Article},
  Url                      = {http://www.informaworld.com/10.1080/08958370801939374}
}

@Article{Nemmar2004,
  Title                    = {Possible mechanisms of the cardiovascular effects of inhaled particles: systemic translocation and prothrombotic effects},
  Author                   = {Nemmar, Abderrahim and Hoylaerts, Marc F. and Hoet, Peter H. M. and Nemery, Benoit},
  Journal                  = {Toxicology Letters},
  Year                     = {2004},
  Number                   = {1-3},
  Pages                    = {243-253},
  Volume                   = {149},

  ISSN                     = {0378-4274},
  Keywords                 = {Air pollution
Particles
Thrombosis
Lung inflammation
Platelet activation},
  Type                     = {Journal Article},
  Url                      = {http://www.sciencedirect.com/science/article/B6TCR-4BNVYHG-4/2/ea7b112a69135ec4a77282c731c47681}
}

@Article{Nerem1993,
  Title                    = {Hemodynamics and Vascular Endothelial Biology},
  Author                   = {Nerem, R.M. },
  Journal                  = {Journal of Cardiovascular Pharmacology},
  Year                     = {1993},
  Pages                    = {572-582},
  Volume                   = {21},

  Type                     = {Journal Article}
}

@Article{Nerem1985,
  Title                    = {Atherosclerosis: Hemodynamics},
  Author                   = {Nerem, R.M. },
  Journal                  = {Vascular Geometry, and the Endothelium. Biorheology},
  Year                     = {1985},
  Pages                    = {565-569},
  Volume                   = {21},

  Type                     = {Journal Article}
}

@Article{Nerem1990,
  Title                    = {Hemodynamic Influences on Vascular Endothelial Biology},
  Author                   = {Nerem, R.M. and Girard, P.R. },
  Journal                  = {Toxicologic Pathology},
  Year                     = {1990},
  Pages                    = {572-582},
  Volume                   = {18},

  Type                     = {Journal Article}
}

@Article{Newbery1997,
  Title                    = {Privatisation and liberalisation of network utilities},
  Author                   = {Newbery, David M.},
  Journal                  = {European Economic Review},
  Year                     = {1997},
  Number                   = {3-5},
  Pages                    = {357-383},
  Volume                   = {41},

  Abstract                 = {Privatisation of utilities is about ownership rather than control. Liberalisation can induce greater improvements in performance than privatisation alone. Regulation is inevitably inefficient, and adequately competitive network services may improve efficiency. History indicates that regulated vertical integration is durable so that liberalisation may be hard to sustain. Theory and evidence suggest that pricing network access and use is difficult, risking foreclosure without regulation. Progress in modelling competition over, for and between networks is reported. The English electricity industry demonstrates the importance of entry conditions and contracts, and the gains from restructuring are estimated.},
  ISSN                     = {0014-2921},
  Keywords                 = {Competition
Regulation
Network utilities
Privatisation},
  Type                     = {Journal Article},
  Url                      = {http://www.sciencedirect.com/science/article/B6V64-3SWYB3P-1/2/da5960370d21a816b0bc50083774bfc5}
}

@Article{Newling2008,
  Title                    = {Gas flow measurements by NMR},
  Author                   = {Newling, Benedict},
  Journal                  = {Progress in Nuclear Magnetic Resonance Spectroscopy},
  Year                     = {2008},
  Number                   = {1},
  Pages                    = {31-48},
  Volume                   = {52},

  ISSN                     = {0079-6565},
  Keywords                 = {NMR
MRI
Flow
Mass transport
Gas phase},
  Type                     = {Journal Article},
  Url                      = {http://www.sciencedirect.com/science/article/B6THC-4PP7727-1/2/d608499366e2777d75617c1079b5d645}
}

@Article{Newman1998,
  Title                    = {Deposition pattern of nasal sprays in man},
  Author                   = {Newman, S.P. and Moren, F. and Clarke, S.W.},
  Journal                  = {Rhinology},
  Year                     = {1998},
  Pages                    = {111-120},
  Volume                   = {26},

  Type                     = {Journal Article}
}

@TechReport{Nielsen1990,
  Title                    = {Specification of a Two-Dimensional Test Case},
  Author                   = {Nielsen, P.V.},
  Institution              = {Aalborg University, IEA Annex 20: Air Flow Patterns within Buildings},
  Year                     = {1990},
  Type                     = {Report}
}

@Article{Nieuwstadt1994,
  Title                    = {Direct and large-eddy simulations of turbulence in fluids},
  Author                   = {Nieuwstadt, F. T. M. and Eggels, J. G. M. and Janssen, R. J. A. and Pourquié, M. B. J. M.},
  Journal                  = {Future Generation Computer Systems},
  Year                     = {1994},
  Number                   = {2-3},
  Pages                    = {189-205},
  Volume                   = {10},

  Abstract                 = {As a result of the increasing power of supercomputers numerical simulation of turbulent flows has become feasible. In the present paper we give a short review of the requirements for such simulations. First we discuss so-called direct numerical simulation (DNS), where the equations of motion for a turbulent flow are solved in all detail. This application is illustrated with two examples, viz. the transition from laminar to turbulence in a differential heated cavity and a fully developed turbulent pipe flow. The second application of turbulence simulation is large-eddy modelling where only the large scales of turbulence are numerically resolved and the small scales are parameterized by a turbulence model. As example for this case we discuss the atmospheric boundary layer and in particular the dispersion of pollutants by atmospheric turbulence. We close our contribution with a discussion of the computer requirements necessary for turbulence simulation together with an outlook in the future in which we consider the pro's and contra's of massively parallel systems.},
  ISSN                     = {0167-739X},
  Keywords                 = {Direct numerical simulation
Large-eddy simulation
Turbulence
Computation requirements
Massively parallel systems},
  Type                     = {Journal Article},
  Url                      = {http://www.sciencedirect.com/science/article/B6V06-4998VY7-72/2/daa6e0b60fa6582835e8d05e47ee223c}
}

@Article{Nishimura2013,
  Title                    = {A New Methodology for Studying Dynamics of Aerosol Particles in Sneeze and Cough Using a Digital High-Vision, High-Speed Video System and Vector Analyses},
  Author                   = {Nishimura, Hidekazu and Sakata, Soichiro and Kaga, Akikazu},
  Journal                  = {PLoS ONE},
  Year                     = {2013},
  Note                     = {C:\Users\sean\AppData\Roaming\Zotero\Zotero\Profiles\16a4oype.default\zotero\storage\D28PFJR7\Nishimura et al. - 2013 - A New Methodology for Studying Dynamics of Aerosol.pdf},
  Pages                    = {e80244},
  Volume                   = {8},

  Abstract                 = {Microbial pathogens of respiratory infectious diseases are often transmitted through particles in sneeze and cough. Therefore, understanding the particle movement is important for infection control. Images of a sneeze induced by nasal cavity stimulation by healthy adult volunteers, were taken by a digital high-vision, high-speed video system equipped with a computer system and treated as a research model. The obtained images were enhanced electronically, converted to digital images every 1/300 s, and subjected to vector analysis of the bioparticles contained in the whole sneeze cloud using automatic image processing software. The initial velocity of the particles or their clusters in the sneeze was greater than 6 m/s, but decreased as the particles moved forward; the momentums of the particles seemed to be lost by 0.15–0.20 s and started a diffusion movement. An approximate equation of a function of elapsed time for their velocity was obtained from the vector analysis to represent the dynamics of the front-line particles. This methodology was also applied for a cough. Microclouds contained in a smoke exhaled with a voluntary cough by a volunteer after smoking one breath of cigarette, were traced as the visible, aerodynamic surrogates for invisible bioparticles of cough. The smoke cough microclouds had an initial velocity greater than 5 m/s. The fastest microclouds were located at the forefront of cloud mass that moving forward; however, their velocity clearly decreased after 0.05 s and they began to diffuse in the environmental airflow. The maximum direct reaches of the particles and microclouds driven by sneezing and coughing unaffected by environmental airflows were estimated by calculations using the obtained equations to be about 84 cm and 30 cm from the mouth, respectively, both achieved in about 0.2 s, suggesting that data relating to the dynamics of sneeze and cough became available by calculation.},
  Doi                      = {10.1371/journal.pone.0080244},
  Type                     = {Journal Article},
  Url                      = {http://www.ncbi.nlm.nih.gov/pmc/articles/PMC3842286/pdf/pone.0080244.pdf}
}

@Article{Nithiarasu2006,
  Title                    = {An artificial compressibility based characteristic based split (CBS) scheme for steady and unsteady turbulent incompressible flows},
  Author                   = {Nithiarasu, P. and Liu, C. B.},
  Journal                  = {Computer Methods in Applied Mechanics and Engineering},
  Year                     = {2006},
  Number                   = {23-24},
  Pages                    = {2961-2982},
  Volume                   = {195},

  Abstract                 = {An artificial compressibility method is presented for the solution of incompressible turbulent flow problems at moderate Reynolds numbers. The presented approach has been used to solve both steady and unsteady state turbulent flow problems. Three different RANS models have been employed to investigate the application of present matrix solution free approach. They are one-equation model of Wolfstein, Spalart-Allmaras model and standard [kappa]-[epsilon] model. In addition to the results of standard steady two-dimensional flow problems, solutions to transient two-dimensional flow past a circular cylinder and steady state three-dimensional turbulent flow in a model "upper human airway" are also presented. The results presented demonstrate that the artificial compressibility based CBS scheme is suitable for both steady and unsteady state incompressible turbulent flow calculations at moderate Reynolds numbers.},
  ISSN                     = {0045-7825},
  Keywords                 = {Explicit CBS scheme
Artificial compressibility
Turbulence
RANS
URANS
Incompressible
Finite element method
Unstructured mesh
Dual time stepping},
  Type                     = {Journal Article},
  Url                      = {http://www.sciencedirect.com/science/article/B6V29-4GYH7J5-3/2/a3729f6fcf9cb15b53c93d56aa69c720}
}

@Article{Nithiarasu2007,
  Title                    = {Laminar and turbulent flow calculations through a model human upper airway using unstructured meshes},
  Author                   = {Nithiarasu, P. and Liu, C.-B. and Massarotti, N.},
  Journal                  = {Communications in Numerical Methods in Engineering},
  Year                     = {2007},
  Note                     = {10.1002/cnm.939},
  Number                   = {12},
  Pages                    = {1057-1069},
  Volume                   = {23},

  ISSN                     = {1099-0887},
  Type                     = {Journal Article},
  Url                      = {http://dx.doi.org/10.1002/cnm.939}
}

@Article{Noakes2006,
  Title                    = {Use of CFD Modelling to Optimise the Design of Upper-room UVGI Disinfection Systems for Ventilated Rooms},
  Author                   = {Noakes, C. J. and Sleigh, P. A. and Fletcher, L. A. and Beggs, C. B.},
  Journal                  = {Indoor and Built Environment},
  Year                     = {2006},
  Number                   = {4},
  Pages                    = {347-356},
  Volume                   = {15},

  Abstract                 = {The installation of upper-room ultraviolet germicidal irradiation (UVGI) devices in ventilated rooms has the potential to reduce transmission of infections by an airborne route. However, the performance of such devices is dependant on several factors including the location of the lamp and the ventilation airflow in the room. This study uses a CFD model to evaluate the performance of UVGI devices by considering the cumulative UV-C dose received by the bulk room air in a ventilated room. By evaluating the UV dose rather than the resulting micro-organism inactivation the methodology can be used to optimise UVGI systems at the design stage, particularly when the source location of bioaerosol contaminants is not known. The study investigates the relationships between the lamp location, lamp power, ventilation system and room heating in a small, ventilated room. The results show that with ventilation air supplied at low level and extracted at high level the UVGI system performs better than with the air supplied at high level and extracted close to the floor. In addition the results show the presence of a heater in the room is unlikely to have a detrimental effect on performance and may promote mixing to increase the extent of disinfection.},
  Doi                      = {10.1177/1420326x06067353},
  Type                     = {Journal Article},
  Url                      = {http://ibe.sagepub.com/content/15/4/347.abstract}
}

@Article{Noback2011,
  Title                    = {Climate-related variation of the human nasal cavity},
  Author                   = {Noback, Marlijn L. and Harvati, Katerina and Spoor, Fred},
  Journal                  = {American Journal of Physical Anthropology},
  Year                     = {2011},
  Number                   = {4},
  Pages                    = {599--614},
  Volume                   = {145},

  Doi                      = {10.1002/ajpa.21523},
  ISSN                     = {1096-8644},
  Keywords                 = {nose, adaptation, geometric morphometrics, PLS},
  Publisher                = {Wiley Subscription Services, Inc., A Wiley Company},
  Url                      = {http://dx.doi.org/10.1002/ajpa.21523}
}

@Article{Nocon2009,
  Title                    = {Clinical presentation and management of geriatric rhinitis},
  Author                   = {Nocon,Cheryl C. and Pinto,Jayant M.},
  Journal                  = {Aging Health},
  Year                     = {2009},

  Month                    = {08},
  Number                   = {4},
  Pages                    = {569-583},
  Volume                   = {5},

  ISBN                     = {1745-509X},
  Keywords                 = {Gerontology And Geriatrics; Aging; Older people; Geriatrics; Nose},
  Language                 = {English},
  Url                      = {http://search.proquest.com/docview/221270864?accountid=13552}
}

@Article{Norris1869,
  Title                    = {On the Laws and Principles Concerned in the Aggregation of Blood-Corpuscles both within and without the Vessels},
  Author                   = {Norris, R. },
  Journal                  = {Proceedings of Royal Society of London},
  Year                     = {1869},
  Pages                    = {429-436},
  Volume                   = {17},

  Type                     = {Journal Article}
}

@Article{Novoselac2003,
  Title                    = {Comparison of Air Exchange Efficiency and Contaminant Removal Effectiveness as IAQ Indices},
  Author                   = {Novoselac, Atila and Srebric, Jelena},
  Journal                  = {ASHRAE Transactions},
  Year                     = {2003},
  Number                   = {2},
  Volume                   = {109},

  Type                     = {Journal Article}
}

@Article{Nowak2003,
  Title                    = {Computational fluid dynamics simulation of airflow and aerosol deposition in human lungs},
  Author                   = {Nowak, N. and Kakade, P.P and Annapragada, A.V.},
  Journal                  = {Annals of Biomedical Engineering},
  Year                     = {2003},
  Pages                    = {374-390},
  Volume                   = {31},

  Type                     = {Journal Article}
}

@Article{Nucci2003,
  Title                    = {Modeling airflow-related shear stress during heterogeneous constriction and mechanical ventilation},
  Author                   = {Nucci, Gianluca and Suki, Bela and Lutchen, Kenneth},
  Journal                  = {Journal of Applied Physiology},
  Year                     = {2003},
  Number                   = {1},
  Pages                    = {348-356},
  Volume                   = {95},

  Abstract                 = {Ventilator-induced lung injury has been proposed as being caused by overdistention and closure and reopening of small airways and alveoli. Here we investigate the possibility that heterogeneous constriction increases airflow-related shear stress to a dangerously high level that may be sufficient to cause injury to the epithelial cells during mechanical ventilation. We employed an anatomically consistent model of the respiratory system, based on Horsfield morphometric data, and solved for the time evolution of pressure and flow along the airway tree during mechanical ventilation. We simulated constant-flow ventilation with passive expiration in two different conditions: baseline and highly heterogeneous constriction. The constriction was applied with two strategies: establishing a simple diameter reduction or adding also a length shortening. The shear stress distribution on airway walls was analyzed for airways ranging from the trachea to the acini. Our results indicate that 1) heterogeneous constriction can amplify the maximal values of shear stress up to 50-fold, with peak values higher than 0.6 cmH2O; 2) the highest shear stress is found in pathways constricted by 60-80%; 3) simultaneous diameter reduction and shortening amplifies the shear stresses by three- to fourfold, with shear stresses reaching 2 cmH2O; and 4) there is a range of airways (diameters from 0.6 to 0.3 mm at baseline) that appear to be at risk of very high stresses. We conclude that elevated airflow-related shear stress on the epithelial cell layer can occur during heterogeneous constriction and conjecture that this may constitute a mechanism contributing to ventilator-induced lung injury.},
  Doi                      = {10.1152/japplphysiol.01179.2001},
  Type                     = {Journal Article},
  Url                      = {http://jap.physiology.org/cgi/content/abstract/95/1/348}
}

@Article{Oflynn1993,
  Title                    = {Posture and Nasal Geometry},
  Author                   = {O'flynn, Paul},
  Journal                  = {Acta Oto-laryngologica},
  Year                     = {1993},
  Number                   = {4},
  Pages                    = {530-532},
  Volume                   = {113},

  Doi                      = {doi:10.3109/00016489309135858},
  Type                     = {Journal Article},
  Url                      = {http://informahealthcare.com/doi/abs/10.3109/00016489309135858}
}

@PhdThesis{ORourke1981,
  Title                    = {Collective Drop Effects on Vaporizing Liquid Sprays},
  Author                   = {O'Rourke, P.J.},
  Year                     = {1981},
  Type                     = {Thesis},

  University               = {Princeton University}
}

@Book{ORourke1987,
  Title                    = {The tab method for numerical calculation of spray droplet breakup},
  Author                   = {O'Rourke, P.J. and Amsden, A.A.},
  Year                     = {1987},

  Abstract                 = {The authors present a method for calculating drop aerodynamic breakup in engine sprays. A short history is first give of the major milestones in the development of the stochastic particle method for calculating liquid fuel sprays. The most recent advance has been the discovery of the importance of drop breakup in engine sprays. They present a new method, called the TAB method, for calculating drop breakup. Some theoretical properties of the method are derived; its numerical implementation in the computer program KIVA is described; and comparisons are presented between TAB-method calculations and experiments and calculations using another breakup model.},
  ISBN                     = {CONF-871142- United StatesWed Feb 06 15:38:39 EST 2008Society of Automotive Engineers, 400 Commonwealth Dr., Warrendale, PA 15096.NOV; NOV-88-006210; ERA-13-034253; EDB-88-103155English},
  Keywords                 = {33 ADVANCED PROPULSION SYSTEMS
99 GENERAL AND MISCELLANEOUS//MATHEMATICS, COMPUTING, AND INFORMATION SCIENCE
FUEL SYSTEMS
FLOW MODELS
K CODES
AERODYNAMICS
COMPUTERIZED SIMULATION
DROPLETS
HYDRAULICS
INTERNAL COMBUSTION ENGINES
LIQUID FLOW
SPRAYS
COMPUTER CODES
ENGINES
FLUID FLOW
FLUID MECHANICS
HEAT ENGINES
MATHEMATICAL MODELS
MECHANICS
PARTICLES
SIMULATION},
  Pages                    = {Medium: X; Size: Pages: 10},
  Type                     = {Book}
}

@Article{OShaughnessy2003,
  Title                    = {A Small Whole-Body Exposure Chamber for Laboratory Use},
  Author                   = {O'Shaughnessy, Patrick T. and Achutan, Chandran and O'Neill, Marsha E. and Thorne, Peter S.},
  Journal                  = {Inhalation Toxicology},
  Year                     = {2003},
  Number                   = {3},
  Pages                    = {251-263},
  Volume                   = {15},

  Doi                      = {doi:10.1080/08958370304504},
  Type                     = {Journal Article},
  Url                      = {http://informahealthcare.com/doi/abs/10.1080/08958370304504}
}

@Article{OberdA¶rster2000,
  Title                    = {Pulmonary effects of inhaled ultrafine particles},
  Author                   = {Oberdörster, Günter},
  Journal                  = {International Archives of Occupational and Environmental Health},
  Year                     = {2000},
  Number                   = {1},
  Pages                    = {1-8},
  Volume                   = {74},

  Abstract                 = {Abstract&nbsp;&nbsp; Introduction and Objectives: Recent epidemiological studies have shown an association between increased particulate urban air pollution and adverse health effects on susceptible parts of the population, in particular the elderly with pre-existing respiratory and cardiovascular diseases. Urban particles consist of three modes: ultrafine particles, accumulation mode particles (which together form the fine particle mode) and coarse mode particles. Ultrafine particles (those of &lt;0.1 μm diameter) contribute very little to the overall mass, but are very high in number, which in episodic events can reach several hundred thousand/cm3 in the urban air. The hypothesis that ultrafine particles are causally involved in adverse responses seen in sensitive humans is based on several studies summarized in this brief review. Methods and Results: Studies on rodents demonstrate that ultrafine particles administered to the lung cause a greater inflammatory response than do larger particles, per given mass. Surface properties (surface chemistry) appear to play an important role in ultrafine particle toxicity. Contributing to the effects of ultrafine particles is their very high size-specific deposition when inhaled as singlet ultrafine particles rather than as aggregated particles. It appears also that ultrafine particles, after deposition in the lung, largely escape alveolar macrophage surveillance and gain access to the pulmonary interstitium. Inhaled low doses of carbonaceous ultrafine particles can cause mild pulmonary inflammation in rodents after exposure for 6 h. Old age and a compromised/sensitized respiratory tract in rodents can increase their susceptibility to the inflammatory effects of ultrafine particles significantly, and it appears that the aged organism is at a higher risk of oxidative stress induced lung injury from these particles, compared with the young organism. Results also show that ultrafine particle effects can be significantly enhanced by a gaseous co-pollutant such as ozone. Conclusions: The studies performed so far support the ultrafine particle hypothesis. Additional studies are necessary to evaluate mechanistic pathways of responses.},
  Type                     = {Journal Article},
  Url                      = {http://dx.doi.org/10.1007/s004200000185}
}

@Article{OberdA¶rster2004,
  Title                    = {Translocation of Inhaled Ultrafine Particles to the Brain},
  Author                   = {Oberdörster, G. and Sharp, Z. and Atudorei, V. and Elder, A. and Gelein, R. and Kreyling, W. and Cox, C.},
  Journal                  = {Inhalation Toxicology},
  Year                     = {2004},
  Number                   = {6},
  Pages                    = {437 - 445},
  Volume                   = {16},

  ISSN                     = {0895-8378},
  Type                     = {Journal Article},
  Url                      = {http://www.informaworld.com/10.1080/08958370490439597}
}

@Article{Oberdorster2012,
  Title                    = {Nanotoxicology: in vitro-in vivo dosimetry},
  Author                   = {Oberdorster, G.},
  Journal                  = {Environ Health Perspect},
  Year                     = {2012},
  Note                     = {Oberdorster, Gunter
Comment
Letter
United States
Environ Health Perspect. 2012 Jan;120(1):A13; author reply A13. doi: 10.1289/ehp.1104320.},
  Number                   = {1},
  Pages                    = {A13; author reply A13},
  Volume                   = {120},

  Doi                      = {10.1289/ehp.1104320},
  ISSN                     = {1552-9924 (Electronic)
0091-6765 (Linking)},
  Keywords                 = {Air Pollutants, Occupational/ toxicity
Animals
Computer Simulation
Humans
Inhalation Exposure
Occupational Exposure
Pulmonary Alveoli/ drug effects
Toxicity Tests/ methods},
  Type                     = {Journal Article}
}

@Article{Oberdorster2010,
  Title                    = {Safety assessment for nanotechnology and nanomedicine: concepts of nanotoxicology},
  Author                   = {Oberdorster, G.},
  Journal                  = {J Intern Med},
  Year                     = {2010},
  Note                     = {Oberdorster, G
ES01247/ES/NIEHS NIH HHS/United States
Research Support, N.I.H., Extramural
Research Support, U.S. Gov't, Non-P.H.S.
England
J Intern Med. 2010 Jan;267(1):89-105. doi: 10.1111/j.1365-2796.2009.02187.x.},
  Number                   = {1},
  Pages                    = {89-105},
  Volume                   = {267},

  Abstract                 = {Nanotechnology, nanomedicine and nanotoxicology are complementary disciplines aimed at the betterment of human life. However, concerns have been expressed about risks posed by engineered nanomaterials (ENMs), their potential to cause undesirable effects, contaminate the environment and adversely affect susceptible parts of the population. Information about toxicity and biokinetics of nano-enabled products combined with the knowledge of unintentional human and environmental exposure or intentional delivery for medicinal purposes will be necessary to determine real or perceived risks of nanomaterials. Yet, results of toxicological studies using only extraordinarily high experimental doses have to be interpreted with caution. Key concepts of nanotoxicology are addressed, including significance of dose, dose rate, and biokinetics, which are exemplified by specific findings of ENM toxicity, and by discussing the importance of detailed physico-chemical characterization of nanoparticles, specifically surface properties. Thorough evaluation of desirable versus adverse effects is required for safe applications of ENMs, and major challenges lie ahead to answer key questions of nanotoxicology. Foremost are assessment of human and environmental exposure, and biokinetics or pharmacokinetics, identification of potential hazards, and biopersistence in cells and subcellular structures to perform meaningful risk assessments. A specific example of multiwalled carbon nanotubes (MWCNT) illustrates the difficulty of extrapolating toxicological results. MWCNT were found to cause asbestos-like effects of the mesothelium following intracavitary injection of high doses in rodents. The important question of whether inhaled MWCNT will translocate to sensitive mesothelial sites has not been answered yet. Even without being able to perform a quantitative risk assessment for ENMs, due to the lack of sufficient data on exposure, biokinetics and organ toxicity, until we know better it should be made mandatory to prevent exposure by appropriate precautionary measures/regulations and practicing best industrial hygiene to avoid future horror scenarios from environmental or occupational exposures. Similarly, safety assessment for medical applications as key contribution of nanotoxicology to nanomedicine relies heavily on nano-specific toxicological concepts and findings and on a multidisciplinary collaborative approach involving material scientists, physicians and toxicologists.},
  Doi                      = {10.1111/j.1365-2796.2009.02187.x},
  ISSN                     = {1365-2796 (Electronic)
0954-6820 (Linking)},
  Keywords                 = {Humans
Inhalation Exposure
Nanoparticles/ adverse effects
Nanostructures/adverse effects
Nanotechnology/standards
Occupational Exposure/adverse effects
Particle Size
Risk Assessment},
  Type                     = {Journal Article}
}

@Article{Oberdorster2005,
  Title                    = {Principles for characterizing the potential human health effects fromexposure to nanomaterials: elements of a screening strategy, },
  Author                   = {Oberdorster, G. and A.Maynard and K. Donaldson, V. Castranova, J. Fitzpatrick, K. Ausman, J. Carter, B. Karn, W. Kreyling, D. Lai, S. Olin, N. Monteiro-Riviere, D. Warheit, H. Yang, },
  Journal                  = {Part. Fibre Toxicology},
  Year                     = {2005},
  Volume                   = {2},

  Type                     = {Journal Article}
}

@Article{Odegard2005,
  Title                    = {Modeling of the mechanical properties of nanoparticle/polymer composites},
  Author                   = {Odegard, G. M. and Clancy, T. C. and Gates, T. S.},
  Journal                  = {Polymer},
  Year                     = {2005},
  Number                   = {2},
  Pages                    = {553-562},
  Volume                   = {46},

  Abstract                 = {A continuum-based elastic micromechanics model is developed for silica nanoparticle/polyimide composites with various nanoparticle/polyimide interfacial treatments. The model incorporates the molecular structures of the nanoparticle, polyimide, and interfacial regions, which are determined using a molecular modeling method that involves coarse-grained and reverse-mapping techniques. The micromechanics model includes an effective interface between the polyimide and nanoparticle with properties and dimensions that are determined using the results of molecular dynamics simulations. It is shown that the model can be used to predict the elastic properties of silica nanoparticle/polyimide composites for a large range of nanoparticle radii, 10-10,000 A. For silica nanoparticle radii above 1000 A, the predicted properties are equal to those predicted using the standard Mori-Tanaka micromechanical approach, which does not incorporate the molecular structure. It is also shown that the specific silica nanoparticle/polyimide interface conditions have a significant effect on the composite mechanical properties for nanoparticle radii below 1000 A.},
  ISSN                     = {0032-3861},
  Keywords                 = {Molecular dynamics
Nanoparticles
Nanocomposites},
  Type                     = {Journal Article},
  Url                      = {http://www.sciencedirect.com/science/article/B6TXW-4DXT7X0-4/2/133c3074885a56828325456fb50042ed}
}

@Article{Oden1991,
  Title                    = {h-p adaptive finite element methods in computational fluid dynamics},
  Author                   = {Oden, J. T. and Demkowicz, L.},
  Journal                  = {Computer Methods in Applied Mechanics and Engineering},
  Year                     = {1991},
  Number                   = {1-3},
  Pages                    = {11-40},
  Volume                   = {89},

  Abstract                 = {The principal ideas of h-p adaptive finite element methods for fluid dynamics problems are discussed. Applications include acoustics, compressible Euler and both compressible and incompressible Navier-Stokes equations. Several numerical examples illustrate the presented concepts.},
  ISSN                     = {0045-7825},
  Type                     = {Journal Article},
  Url                      = {http://www.sciencedirect.com/science/article/B6V29-47X7C1B-4V/2/4d4e69c500b05f815580f9455f281bb6}
}

@Article{Ogawa1980,
  Title                    = {Surface roughness and thermal stratification effects on the flow behind a two-dimensional fence--II. A wind tunnel study and similarity considerations},
  Author                   = {Ogawa, Yasushi and Diosey, P. G.},
  Journal                  = {Atmospheric Environment (1967)},
  Year                     = {1980},
  Number                   = {11},
  Pages                    = {1309-1320},
  Volume                   = {14},

  Abstract                 = {A wind tunnel study of the flow behind a two-dimensional model fence placed perpendicular to the wind was conducted in the NIES Atmospheric Diffusion Simulation Wind Tunnel. Two inflow conditions were studied for their effects on the nondimensionalized cavity wake length, : surface roughness (Returb [similar, equals] 15 to 62) and thermal stratification (Ri [similar, equals] -0.20 to +0.23). It was found that as Returb increases, increases. In the unstable region, decreases only slightly as Ri decreases (from 13.0 in neutral conditions to ~12.5 for strong unstable); however, in the stable region, decreases drastically as the stability increases (to ~7.0 for strongly stable). Similar tendencies were found in a related field study (Ogawa and Diosey, 1980) and a comparison of both sets of data in order to determine proper similarity criteria showed that: 1. (1) as a practical choice, the equality of turbulent intensities may be the appropriate similarity criterion for this type of flow; 2. (2) when using over-sized roughness (roughness large relative to model height) to achieve similar flow behind the model, better agreement for the oncoming mean velocity profile was found with the inclusion of a zero-plane displacement factor assumed equal to the roughness element height; and 3. (3) using over-sized roughness for the purpose of simulating the large u-component eddy size found in the atmosphere can lead to an over-sized w-component scale of eddies in the wind tunnel.},
  ISSN                     = {0004-6981},
  Type                     = {Journal Article},
  Url                      = {http://www.sciencedirect.com/science/article/B757C-48CFW6K-1W0/2/289186b8440546d6dac88cfbc9afcad5}
}

@Article{Oldham2006,
  Title                    = {Challenges in Validating CFD-Derived Inhaled Aerosol Deposition Predictions},
  Author                   = {Oldham, Michael J.},
  Journal                  = {Inhalation Toxicology},
  Year                     = {2006},
  Number                   = {10},
  Pages                    = {781-786},
  Volume                   = {18},

  ISSN                     = {0895-8378},
  Type                     = {Journal Article},
  Url                      = {http://www.informaworld.com/10.1080/08958370600748752}
}

@Article{Oldham2004,
  Title                    = {Performance of a portable whole-body mouse exposure system},
  Author                   = {Oldham, M. J. and Phalen, R. F. and Robinson, R. J. and Kleinman, M. T.},
  Journal                  = {Inhal Toxicol},
  Year                     = {2004},
  Note                     = {Using Smart Source Parsing
Aug},
  Number                   = {9},
  Pages                    = {657-62},
  Volume                   = {16},

  Abstract                 = {A mobile whole-body exposure system was developed for exposing mice to concentrated ambient particulate matter smaller than 2.5 microm in mass median aerodynamic diameter (MMAD). Each 20-L exposure cage was designed to hold 9 mice within individual compartments. This allowed for transport and subsequent exposure. Airflow mixing and the potential for stagnant areas within the compartments were modeled using computational fluid dynamic modeling (CFD). CFD analysis showed no stagnant areas and good mixing throughout the exposure cage. The actual performance of the exposure system was determined for 0.5 to 2.0 microm diameter aerosols by measuring (1) uniformity of aerosol distribution and (2) particle deposition in the tracheobronchial and pulmonary regions of mice exposed in the system. A 0.6-microm MMAD (GSD=2.0) cigarette smoke aerosol was used to experimentally measure the uniformity of aerosol distribution to the nine individual compartments. The average data from three runs showed no statistically significant difference among individual compartments. Particle deposition efficiency in adult male BALB/c mice was measured after exposure (30 min) in the system using monodisperse fluorescent polystyrene latex particles (0.5, 1, and 2 microm aerodynamic diameter). The measured deposition efficiency in this mobile exposure system for the combined tracheobronchial and pulmonary regions of the adult male BALBc mice was 21% for 0.5 microm, 11% for 1.0 microm, and 6.5% for 2.0 microm particles. These deposition efficiencies are similar to those reported for mice exposed in a nose-only exposure system, which indicates that particle losses to animal fur and exposure system surfaces were acceptable.},
  ISSN                     = {0895-8378 (Print)
0895-8378 (Linking)},
  Type                     = {Journal Article}
}

@Article{Omri2007,
  Title                    = {Numerical analysis of turbulent buoyant flows in enclosures: Influence of grid and boundary conditions},
  Author                   = {Omri, Mohamed and Galanis, Nicolas},
  Journal                  = {International Journal of Thermal Sciences},
  Year                     = {2007},
  Number                   = {8},
  Pages                    = {727-738},
  Volume                   = {46},

  Abstract                 = {Turbulent natural convection in a differentially heated 2D cavity has been simulated numerically using the Shear Stress Transitional (SST) k-[omega] turbulence model. Comparisons with experimental benchmark values show that, in this case, conduction in the horizontal walls has a significant effect on the calculated results. They also show that the agreement between calculated and measured values at mid-height of the cavity does not suffice to establish the validity of the model and numerical procedure. Finally, these comparisons prove that grid independent results in good agreement with the experimental benchmark values can be obtained with a number of nodes smaller than that used for a Large Eddy Simulation (LES), as long as their distribution in the boundary layers is well chosen.},
  ISSN                     = {1290-0729},
  Keywords                 = {Natural convection
Turbulent flow
Modeling
Experimental validation},
  Type                     = {Journal Article},
  Url                      = {http://www.sciencedirect.com/science/article/B6VT1-4MFKKB5-2/2/1492e26b5d9c12958031a4e5f6d2f3a6}
}

@Article{Onuh2001,
  Title                    = {Improving Stereolithography Part Accuracy for Industrial Applications},
  Author                   = {Onuh, S. O. and Hon, K. K. B.},
  Journal                  = {The International Journal of Advanced Manufacturing Technology},
  Year                     = {2001},
  Number                   = {1},
  Pages                    = {61-68},
  Volume                   = {17},

  Abstract                 = {High dimensional accuracy and part stability are significant elements of the end product in rapid prototyping technology (RPT) processes whether it is a component or a tool. However, in most cases, models built in acrylic-based resin in the stereolithography (SL) process have not been of the desired quality and this has led to the use of more expensive resins that have a longer build time. An experimental investigation has been carried out to determine statistically the optimum build parameters to be used in the Taguchi method, to improve the SLA product quality. The two new hatch styles developed in this study have resulted in an overall improvement of the part accuracy, and a new layer thickness for part building in stereolithography has been proposed.&nbsp;&nbsp;},
  Doi                      = {10.1007/s001700170210},
  Type                     = {Journal Article},
  Url                      = {http://dx.doi.org/10.1007/s001700170210}
}

@Article{Oostveen2006,
  Title                    = {Respiratory input impedance in normal subjects and COPD patients},
  Author                   = {Oostveen, E.},
  Journal                  = {Journal of Biomechanics},
  Year                     = {2006},
  Number                   = {Supplement 1},
  Pages                    = {S269-S270},
  Volume                   = {39},

  ISSN                     = {0021-9290},
  Type                     = {Journal Article},
  Url                      = {http://www.sciencedirect.com/science/article/B6T82-4KR88PB-1FS/2/bb8947a69c8c156caa661940ee3659f6}
}

@Article{Orive2003,
  Title                    = {Drug delivery in biotechnology: present and future},
  Author                   = {Orive, G. and Hernandez, R.M. and Gascon, A.R. and Dominguez-Gil, A. and Pedraz, J.L.},
  Journal                  = {Current Opinion in Biotechnology},
  Year                     = {2003},
  Pages                    = {659-664},
  Volume                   = {14},

  Type                     = {Journal Article}
}

@Article{Oseen1910,
  Title                    = {Uber die Stokessche Formel und uber die verwandte Aufgabe in der Hydrodynamik},
  Author                   = {Oseen, C.W.},
  Journal                  = {Arkiv Math Aston Fys},
  Year                     = {1910},
  Number                   = {29},
  Pages                    = {237-262},
  Volume                   = {6},

  Type                     = {Journal Article}
}

@Article{Otsu1979,
  Title                    = {A threshold selection method from grey-level histograms},
  Author                   = {Otsu, N.},
  Journal                  = {IEEE Trans. Systems, Man, and Cybernetics},
  Year                     = {1979},
  Pages                    = {62-66},
  Volume                   = {9},

  Type                     = {Journal Article}
}

@Article{Ounis1991,
  Title                    = {Motions of small particles in a turbulent simple shear flow field under microgravity condition},
  Author                   = {Ounis, H. and Ahmadi, G.},
  Journal                  = {Physics of Fluids A},
  Year                     = {1991},
  Note                     = {Cited By (since 1996): 5
Export Date: 5 June 2011
Source: Scopus},
  Number                   = {11},
  Pages                    = {2559-2570},
  Volume                   = {3},

  Type                     = {Journal Article},
  Url                      = {http://www.scopus.com/inward/record.url?eid=2-s2.0-0025966213&partnerID=40&md5=e6f30ce68eeec4bcc6a670bb3b7ca13a}
}

@Article{Ounis1990,
  Title                    = {Analysis of dispersion of small spherical particles in a random velocity field},
  Author                   = {Ounis, H. and Ahmadi, G.},
  Journal                  = {ASME Journal of Fluids Engineering},
  Year                     = {1990},
  Note                     = {Cited By (since 1996): 16
Export Date: 5 June 2011
Source: Scopus},
  Number                   = {1 , Mar., 1990},
  Pages                    = {114-120},
  Volume                   = {112},

  Type                     = {Journal Article},
  Url                      = {http://www.scopus.com/inward/record.url?eid=2-s2.0-0025154168&partnerID=40&md5=78564d738fd8439bff508e268e5ee2a8}
}

@Article{Ounis1990a,
  Title                    = {A Comparison of Brownian and Turbulent Diffusion},
  Author                   = {Ounis, Hadj and Ahmadi, Goodarz},
  Journal                  = {Aerosol Science and Technology},
  Year                     = {1990},
  Number                   = {1},
  Pages                    = {47 - 53},
  Volume                   = {13},

  ISSN                     = {0278-6826},
  Type                     = {Journal Article},
  Url                      = {http://www.informaworld.com/10.1080/02786829008959423}
}

@Article{Ounis1989,
  Title                    = {Motions of Small Rigid Spheres in a Simulated Random Velocity Field},
  Author                   = {Ounis, H. and Ahmadi, G. },
  Journal                  = {ASCE Journal of Engineering Mechanics},
  Year                     = {1989},
  Pages                    = {2107-2121},
  Volume                   = {115},

  Type                     = {Journal Article}
}

@Article{Ounis1989a,
  Title                    = {Motions of Small Rigid Spheres in Simulated Random Velocity Field},
  Author                   = {Ounis, Hadj and Ahmadi, Goodarz},
  Journal                  = {Journal of Engineering Mechanics},
  Year                     = {1989},
  Number                   = {10},
  Pages                    = {2107-2121},
  Volume                   = {115},

  Keywords                 = {Turbulent diffusion
Equations of motion
Dispersion
Three-dimensional flow
Random processes
Particles
Velocity
Response time},
  Type                     = {Journal Article},
  Url                      = {http://link.aip.org/link/?QEM/115/2107/1}
}

@Article{Ounis1991a,
  Title                    = {Brownian diffusion of submicrometer particles in the viscous sublayer.},
  Author                   = {Ounis, H. and Ahmadi, G. and J.B., McLaughlin.},
  Journal                  = {Journal Colloid and Interface Science},
  Year                     = {1991},
  Number                   = {1},
  Pages                    = {266-277},
  Volume                   = {143},

  Type                     = {Journal Article}
}

@Article{Ounis1992,
  Title                    = {Brownian particle deposition in a directly simulated turbulent channel flow},
  Author                   = {Ounis, H. and Ahmadi, G. and McLaughlin, J. B.},
  Journal                  = {Physics of Fluids A},
  Year                     = {1992},
  Note                     = {Cited By (since 1996): 37
Export Date: 5 June 2011
Source: Scopus},
  Number                   = {6},
  Pages                    = {1427-1432},
  Volume                   = {5},

  Type                     = {Journal Article},
  Url                      = {http://www.scopus.com/inward/record.url?eid=2-s2.0-0027471067&partnerID=40&md5=130c39c5fc5fec439082c2c0eecfbb49}
}

@Article{Ounis1991b,
  Title                    = {Dispersion and deposition of Brownian particles from point sources in a simulated turbulent channel flow},
  Author                   = {Ounis, Hadj and Ahmadi, Goodarz and McLaughlin, John B.},
  Journal                  = {Journal of Colloid And Interface Science},
  Year                     = {1991},
  Number                   = {1},
  Pages                    = {233-250},
  Volume                   = {147},

  ISSN                     = {0021-9797},
  Type                     = {Journal Article},
  Url                      = {http://www.sciencedirect.com/science/article/pii/002197979190151W}
}

@Book{Owen1960,
  Title                    = {Aerodynamic Capture of Particles},
  Author                   = {Owen, P.R.},
  Publisher                = {Pergamon Press},
  Year                     = {1960},

  Address                  = {Oxford},

  Type                     = {Book}
}

@Article{Owen1986,
  Title                    = {Subchronic inhalation toxicology of graphite fibres},
  Author                   = {Owen, P.E. and Glazier, J.R. and Ballantyne, B. and Clary, J.J.},
  Journal                  = {Journal of Occupational Medicine},
  Year                     = {1986},
  Pages                    = {373-376},
  Volume                   = {28},

  Type                     = {Journal Article}
}

@Article{Ozlugedik2008,
  Title                    = {Numerical Study of the Aerodynamic Effects of Septoplasty and Partial Lateral Turbinectomy},
  Author                   = {Ozlugedik, Samet and Nakiboglu, Gunes and Sert, Cuneyt and Elhan, Alaittin and Tonuk, Ergin and Akyar, Serdar and Tekdemir, Ibrahim},
  Journal                  = {The Laryngoscope},
  Year                     = {2008},
  Number                   = {2},
  Pages                    = {330-334},
  Volume                   = {118},

  ISSN                     = {1531-4995},
  Type                     = {Journal Article},
  Url                      = {http://dx.doi.org/10.1097/MLG.0b013e318159aa26}
}

@Article{Padaki2009,
  Title                    = {Ozone uptake during inspiratory flow in a model of the larynx, trachea and primary bronchial bifurcation},
  Author                   = {Padaki, Amit and Ultman, James S. and Borhan, Ali},
  Journal                  = {Chemical Engineering Science},
  Year                     = {2009},
  Number                   = {22},
  Pages                    = {4640-4648},
  Volume                   = {64},

  Abstract                 = {Three-dimensional simulations of the transport and uptake of a reactive gas such as O3 were compared between an idealized model of the larynx, trachea, and first bifurcation and a second "control" model in which the larynx was replaced by an equivalent, cylindrical, tube segment. The Navier-Stokes equations, Spalart-Allmaras turbulence equation, and convection-diffusion equation were implemented at conditions reflecting inhalation into an adult human lung. Simulation results were used to analyze axial velocity, turbulent viscosity, local fractional uptake, and regional uptake. Axial velocity data revealed a strong laryngeal jet with a reattachment point in the proximal trachea. Turbulent viscosity data indicated that jet turbulence occurred only at high Reynolds numbers and was attenuated by the first bifurcation. Local fractional uptake data affirmed hotspots previously reported at the first carina, and suggested additional hotspots at the glottal constriction and jet reattachment point in the proximal trachea. These laryngeal effects strongly depended on inlet Reynolds number, with maximal effects (approaching 15%) occurring at maximal inlet flow rates. While the increase in the regional uptake caused by the larynx subsided by the end of the model, the effect of the larynx on cumulative uptake persisted further downstream. These results suggest that with prolonged exposure to a reactive gas, entire regions of the larynx and proximal trachea could show signs of tissue injury.},
  ISSN                     = {0009-2509},
  Keywords                 = {Computational fluid dynamics
Larynx
Simulation
Transport processes
Turbulence},
  Type                     = {Journal Article},
  Url                      = {http://www.sciencedirect.com/science/article/B6TFK-4WBC1V5-1/2/a7f9135db712d4d5297284fdfcc25b89}
}

@Article{Padilla2008,
  Title                    = {Large-eddy simulation of transition to turbulence in natural convection in a horizontal annular cavity},
  Author                   = {Padilla, E. L. M. and Silveira-Neto, A.},
  Journal                  = {International Journal of Heat and Mass Transfer},
  Year                     = {2008},
  Number                   = {13-14},
  Pages                    = {3656-3668},
  Volume                   = {51},

  Abstract                 = {Large-eddy simulations (LES) of transition to turbulence in a horizontal annular cavity are performed, using a dynamic sub-grid scale model and second order schemes for time and space discretizations. Solutions for Prandtl number of 0.707 and Rayleigh number up to 7.5 × 105 are obtained. The onset of transition to turbulence and turbulence regimes are pointed out, as well as the dynamic characteristics of the thermal plume transition. The instantaneous and time average behavior of the flows, related to the velocity and temperature fields, are analyzed and compared with numerical and experimental results from other authors. The influence of transitional and turbulent flows on local and mean Nusselt number are also investigated.},
  ISSN                     = {0017-9310},
  Keywords                 = {Natural convection
Transition to turbulence
Dynamic model},
  Type                     = {Journal Article},
  Url                      = {http://www.sciencedirect.com/science/article/B6V3H-4S8B33P-1/2/e503e219ead00a75fec981d349b9f27f}
}

@Article{Pal1993,
  Title                    = {A review on image segmentation techniques},
  Author                   = {Pal, Nikhil R. and Pal, Sankar K.},
  Journal                  = {Pattern Recognition},
  Year                     = {1993},
  Number                   = {9},
  Pages                    = {1277-1294},
  Volume                   = {26},

  Keywords                 = {Image segmentation
Fuzzy sets
Thresholding
Edge detection
Clustering
Relaxation
Markov Random Field},
  Type                     = {Journal Article},
  Url                      = {http://www.sciencedirect.com/science/article/B6V14-48MPPFT-1T0/2/e14ee69e6208152164c7c7791b53b6da }
}

@Article{Pallares2002,
  Title                    = {Laminar and turbulent Rayleigh-Bénard convection in a perfectly conducting cubical cavity},
  Author                   = {Pallares, J. and Cuesta, I. and Grau, F. X.},
  Journal                  = {International Journal of Heat and Fluid Flow},
  Year                     = {2002},
  Number                   = {3},
  Pages                    = {346-358},
  Volume                   = {23},

  ISSN                     = {0142-727X},
  Keywords                 = {Rayleigh-Bénard convection
Cubical cavity
Large-eddy simulation
Turbulent natural convection},
  Type                     = {Journal Article},
  Url                      = {http://www.sciencedirect.com/science/article/B6V3G-45F936N-F/2/5929933d26ace762bea5016a90fb69a8}
}

@Article{Papavergos1984,
  Title                    = {Particle deposition behaviour from turbulent flows},
  Author                   = {Papavergos, P. G. and Hedley, A. B.},
  Journal                  = {Chemical Engineering Research and Design},
  Year                     = {1984},
  Note                     = {Cited By (since 1996): 50
Export Date: 5 June 2011
Source: Scopus},
  Number                   = {5},
  Pages                    = {275-295},
  Volume                   = {62},

  Type                     = {Journal Article},
  Url                      = {http://www.scopus.com/inward/record.url?eid=2-s2.0-0021494745&partnerID=40&md5=785a8e54c9d6ef562e5f1697452cbf7e}
}

@Article{Park1997,
  Title                    = {Airway obstruction in asthmatic and healthy individuals: inspiratory and expiratory thin-section CT findings.},
  Author                   = {Park, C.S. and Müller, N.L. and Worthy, S.A. and Kim, J.S. and Awadh, N. and Fitzgerald, M.},
  Journal                  = {Radiology},
  Year                     = {1997},
  Pages                    = {361-367},
  Volume                   = {203},

  Type                     = {Journal Article}
}

@InProceedings{Park,
  Title                    = {Experimental study of velocity fields in a model of human nasal cavity by DPIV},
  Author                   = {Park, K.I. and Brucker, C. and Limberg, W.},
  Booktitle                = {Laser Anemometry Advances and Applications: Proceedings of
the 7th International Conference},
  Editor                   = {Ruck, B. and Leder, A. and Dopheide, D. },

  Type                     = {Conference Proceedings},
  Url                      = {http://www.researchgate.net/publication/234138993_Experimental_study_of_velocity_fields_in_a_model_of_the_human_nasal_cavity_by_DPIV/file/79e4150f84b964df96.pdf}
}

@Article{Park2007,
  Title                    = {Particle deposition in the pulmonary region of the human lung: A semi-empirical model of single breath transport and deposition},
  Author                   = {Park, S. S. and Wexler, A. S.},
  Journal                  = {Journal of Aerosol Science},
  Year                     = {2007},
  Note                     = {doi: DOI: 10.1016/j.jaerosci.2006.11.009},
  Number                   = {2},
  Pages                    = {228-245},
  Volume                   = {38},

  ISSN                     = {0021-8502},
  Keywords                 = {Aerosol bolus dispersion
Particle transport profile
Mixing intensity},
  Type                     = {Journal Article},
  Url                      = {http://www.sciencedirect.com/science/article/B6V6B-4MSR8VH-1/2/a6406160c0e58323fcd96e4f9c7d01a7}
}

@Article{Parker2008,
  Title                    = {Towards quantitative prediction of aerosol deposition from turbulent flows},
  Author                   = {Parker, Simon and Foat, Timothy and Preston, Steve},
  Journal                  = {Journal of Aerosol Science},
  Year                     = {2008},
  Number                   = {2},
  Pages                    = {99-112},
  Volume                   = {39},

  Abstract                 = {Predictions of aerosol deposition rates from turbulent flow using computational fluid dynamics simulations have been compared with experimental data. The influence of turbulence model choice has been assessed. The use of isotropic turbulence models resulted in over-prediction of V+ by more than 3 orders of magnitude for [tau]+~0.2, whilst the anisotropic RSM gave results in good agreement with experiment. For [tau]+>10 ( for Re=9894) there was little difference between the turbulence models. Simulations for both Re=9894 and 50,000 were carried out and good performance was seen for both. The effect of drag model was assessed and resulted in little difference in predicted deposition velocity. The influence of grid resolution was also investigated and it was seen that the cell centre of the wall-adjacent cell should be at a distance of y+=2 or less for quantitative prediction of deposition. Coarser grids resulted in over-prediction for low [tau]+.},
  ISSN                     = {0021-8502},
  Keywords                 = {Aerosol deposition
Computational fluid dynamics
Modelling
Pipe flow
Turbulence model},
  Type                     = {Journal Article},
  Url                      = {http://www.sciencedirect.com/science/article/B6V6B-4PWF0W0-1/2/41f7e77f431cb0165a5b51a4547e5e42}
}

@Article{Pascual2005,
  Title                    = {Airway remodeling contributes to the progressive loss of lung function in asthma: an overview.},
  Author                   = {Pascual, R.M. and Peters, S.P.},
  Journal                  = {J Allergy Clin Immunol.},
  Year                     = {2005},
  Pages                    = {477-86},
  Volume                   = {116},

  Type                     = {Journal Article}
}

@Article{Passali1999,
  Title                    = {Monitoring methods of nasal pathology},
  Author                   = {Passali, D. and Mezzedimi, C. and Passali, C. G. and Bellussi, L.},
  Journal                  = {International Journal of Pediatric Otorhinolaryngology},
  Year                     = {1999},
  Number                   = {0},
  Pages                    = {S199-S202},
  Volume                   = {49, Supplement 1},

  Abstract                 = {The importance of correct clinical and therapeutic monitoring of allergic rhinitis is understandable in the light of the social and economic impact of this pathology: its prevalence is over 10% of the total population all over the world. For the evaluation of the local nasal pathology we include: (1) anterior rhinoscopy, (2) active anterior rhinomanometry, (3) positioned acoustic rhinometry, (4) determination of mucociliary transport time, (5) specific nasal provocation test. Active anterior rhinomanometry allows reliable assessment of the nasal respiratory function, acoustic rhinometry shows the geometry of nasal cavity, mucociliary transport time is an indicator of the mucosa eutrophism. In our experience, the specific nasal provocation test is one of the most important tests in this field. It is more sensitive than the skin test and the radioallergosorbent test (RAST) in the asymptomatic phase and it is able to show organ allergies. In this study we review the importance of this test and the methodology we commonly use.},
  Doi                      = {http://dx.doi.org/10.1016/S0165-5876(99)00160-3},
  ISSN                     = {0165-5876},
  Keywords                 = {Active anterior rhinomanometry
Acoustic rhinometry
Mucociliary transport time
Specific nasal provocation test},
  Type                     = {Journal Article},
  Url                      = {http://www.sciencedirect.com/science/article/pii/S0165587699001603}
}

@Book{Patankar1970,
  Title                    = {Heat and Mass Transfer in Boundary Layers},
  Author                   = {Patankar, S.V. and Spalding, D.B.},
  Publisher                = {Intertext Books},
  Year                     = {1970},

  Address                  = {London},

  Type                     = {Book}
}

@Book{Patankar1980,
  Title                    = {Numerical heat transfer and fluid flow},
  Author                   = {Patankar, S. V.},
  Publisher                = {Taylor \& Francis},
  Year                     = {1980},

  Type                     = {Book}
}

@Article{Patel2014,
  Title                    = {Charged particle therapy versus photon therapy for paranasal sinus and nasal cavity malignant diseases: a systematic review and meta-analysis.},
  Author                   = {Patel, Samir H. and Wang, Zhen and Wong, William W. and Murad, Mohammad Hassan and Buckey, Courtney R. and Mohammed, Khaled and Alahdab, Fares and Altayar, Osama and Nabhan, Mohammed and Schild, Steven E. and Foote, Robert L.},
  Journal                  = {The Lancet. Oncology},
  Year                     = {2014},
  Pages                    = {1027-38},
  Volume                   = {15},

  Abstract                 = {BACKGROUND: Malignant tumours arising within the nasal cavity and paranasal sinuses are rare and composed of several histological types, rendering controlled clinical trials to establish the best treatment impractical. We undertook a systematic review and meta-analysis to compare the clinical outcomes of patients treated with charged particle therapy with those of individuals receiving photon therapy.; METHODS: We identified studies of nasal cavity and paranasal sinus tumours through searches of databases including Embase, Medline, Scopus, and the Cochrane Collaboration. We included treatment-naive cohorts (both primary and adjuvant radiation therapy) and those with recurrent disease. Primary outcomes of interest were overall survival, disease-free survival, and locoregional control, at 5 years and at longest follow-up. We used random-effect models to pool outcomes across studies and compared event rates of combined outcomes for charged particle therapy and photon therapy using an interaction test.; FINDINGS: 43 cohorts from 41 non-comparative observational studies were included. Median follow-up for the charged particle therapy group was 38 months (range 5-73) and for the photon therapy group was 40 months (14-97). Pooled overall survival was significantly higher at 5 years for charged particle therapy than for photon therapy (relative risk 1·51, 95% CI 1·14-1·99; p=0·0038) and at longest follow-up (1·27, 1·01-1·59; p=0·037). At 5 years, disease-free survival was significantly higher for charged particle therapy than for photon therapy (1·93, 1·36-2·75, p=0·0003) but, at longest follow-up, this event rate did not differ between groups (1·51, 1·00-2·30; p=0·052). Locoregional control did not differ between treatment groups at 5 years (1·06, 0·68-1·67; p=0·79) but it was higher for charged particle therapy than for photon therapy at longest follow-up (1·18, 1·01-1·37; p=0·031). A subgroup analysis comparing proton beam therapy with intensity-modulated radiation therapy showed significantly higher disease-free survival at 5 years (relative risk 1·44, 95% CI 1·01-2·05; p=0·045) and locoregional control at longest follow-up (1·26, 1·05-1·51; p=0·011).; INTERPRETATION: Compared with photon therapy, charged particle therapy could be associated with better outcomes for patients with malignant diseases of the nasal cavity and paranasal sinuses. Prospective studies emphasising collection of patient-reported and functional outcomes are strongly encouraged.; FUNDING: Mayo Foundation for Medical Education and Research. Copyright 2014 Elsevier Ltd. All rights reserved.},
  Doi                      = {10.1016/S1470-2045(14)70268-2},
  ISSN                     = {1474-5488},
  Type                     = {Journal Article},
  Url                      = {http://ac.els-cdn.com/S1470204514702682/1-s2.0-S1470204514702682-main.pdf?_tid=1abbb57c-4220-11e4-b69b-00000aacb360&acdnat=1411366818_bf352eb66680323b10984ad72be52b3e}
}

@Article{Patterson1980,
  Title                    = {The aging nose: characteristics and correction},
  Author                   = {Patterson, C. N.},
  Journal                  = {Otolaryngologic clinics of North America},
  Year                     = {1980},
  Note                     = {Date completed - 1980-09-23
Date created - 1980-09-23
Date revised - 2014-01-13},
  Number                   = {2},
  Pages                    = {275-288},
  Volume                   = {13},

  Abstract                 = {The effects of aging are particularly manifested in the face. The vertical dimension of the lower third of the face decreases but the same dimension increases in the nose. The chin moves up; the nose droops down. In order to restore normal nasal physiologic function in persons who present with this type of nose, correction of the cartilaginous and occasionally the bony external nose is necessary. This produces an improved physiologic and cosmetic result.},
  ISSN                     = {0030-6665, 0030-6665},
  Keywords                 = {Index Medicus
Airway Obstruction -- etiology
Face -- anatomy & histology
Cartilage -- surgery
Skin Physiological Phenomena
Humans
Adult
Nasal Septum -- surgery
Middle Aged
Nasal Bone -- surgery
Male
Female
Rhinoplasty -- methods
Aging
Nose -- physiology
Nose -- anatomy & histology},
  Type                     = {Journal Article},
  Url                      = {http://search.proquest.com/docview/75177394?accountid=13552
http://primoapac01.hosted.exlibrisgroup.com/openurl/RMITU/RMIT_SERVICES_PAGE??url_ver=Z39.88-2004&rft_val_fmt=info:ofi/fmt:kev:mtx:journal&genre=article&sid=ProQ:ProQ%3Amedlineshell&atitle=The+aging+nose%3A+characteristics+and+correction.&title=Otolaryngologic+clinics+of+North+America&issn=00306665&date=1980-05-01&volume=13&issue=2&spage=275&au=Patterson%2C+C+N&isbn=&jtitle=Otolaryngologic+clinics+of+North+America&btitle=&rft_id=info:eric/&rft_id=info:doi/}
}

@Article{Patterson2014,
  Title                    = {Particle Deposition in Respiratory Tracts of School-Aged Children},
  Author                   = {Patterson, Regan F. and Zhang, Qunfang and Zheng, Mei and Zhu, Yifang},
  Journal                  = {Aerosol and Air Quality Research},
  Year                     = {2014},
  Pages                    = {64-73},
  Volume                   = {14},

  Abstract                 = {Exposure to ultrafine particles poses a potential health risk to school children. While many studies have focused on measuring ultrafine particle (UFP) concentrations in environments where children are at risk of high exposure, few studies have examined the particle deposition in the respiratory tract of children. This study aims to examine the particle deposition in the respiratory tract of school children in different microenvironments. UFP size distribution data were collected in residential homes, school buses, school classrooms, and from school outdoor air in both rural and urban areas of South Texas. The size distribution data were input to the Multiple Path Particle Dosimetry model to calculate regional and total particle deposition fraction. A 24-hour-school-day exposure was simulated by adding the time children spend in each microenvironment. The maximum pulmonary deposition fraction occurs at a diameter ranging from 18 to 40 nm, depending on condition. Age mostly affected the pulmonary region and the total lung deposition with the highest deposition fraction observed for younger children. In addition, comparison of size-dependent regional deposition and particle concentration establishes an accurate depiction of children's exposure and dose profiles. While children only spend 4% of the day in the home source environment, that environment may account for up to 77% of total daily dose intake. Higher deposition fraction occurs at smaller particle size. Younger children show increased levels of particle deposition than older children. Exposure period does not correlate to daily percentage of dose intake. This research can be used to assess children's accumulative exposure to UFPs.},
  Doi                      = {10.4209/aaqr.2013.04.0113},
  ISSN                     = {1680-8584},
  Keywords                 = {diffusion
Dose
Exposure
flow
growth
human airway bifurcation
human-lung
Lung deposition
major highway
Microenvironments
model
nano-particles
Nasal cavity
ultrafine particles},
  Type                     = {Journal Article}
}

@InBook{Pattle1961,
  Title                    = {The retention of gases and particles in the human nose},
  Author                   = {Pattle, R.E.},
  Editor                   = {Davies, C.N.},
  Pages                    = {302-309},
  Publisher                = {Pergamon Press},
  Year                     = {1961},

  Address                  = {Oxford, UK},
  Type                     = {Book Section},

  Booktitle                = {Inhaled particles and vapors}
}

@Article{Payri2004,
  Title                    = {The influence of cavitation on the internal flow and the spray characteristics in diesel injection nozzles},
  Author                   = {Payri, F. and Bermúdez, V. and Payri, R. and Salvador, F. J.},
  Journal                  = {Fuel},
  Year                     = {2004},
  Note                     = {doi: DOI: 10.1016/j.fuel.2003.09.010},
  Number                   = {4-5},
  Pages                    = {419-431},
  Volume                   = {83},

  ISSN                     = {0016-2361},
  Keywords                 = {Cavitation
Diesel
Injection
Nozzle Geometry
Spray},
  Type                     = {Journal Article},
  Url                      = {http://www.sciencedirect.com/science/article/B6V3B-49S6P9B-1/2/9f0f82ab914af7a5db32ee0c133d7e7d}
}

@InBook{Pedley1977,
  Title                    = {Gas flow and mixing in the airways},
  Author                   = {Pedley, T.J. and Schroter, R.C. and Sudlow, M.F.},
  Editor                   = {West, J.},
  Publisher                = {Dekker},
  Year                     = {1977},

  Address                  = {New York},
  Type                     = {Book Section},

  Booktitle                = {Bioengineering Aspects of the Lung,}
}

@Article{Peng2008,
  Title                    = {Coarse Particulate Matter Air Pollution and Hospital Admissions for Cardiovascular and Respiratory Diseases Among Medicare Patients},
  Author                   = {Peng, Roger D. and Chang, Howard H. and Bell, Michelle L. and McDermott, Aidan and Zeger, Scott L. and Samet, Jonathan M. and Dominici, Francesca},
  Journal                  = {JAMA},
  Year                     = {2008},
  Number                   = {18},
  Pages                    = {2172-2179},
  Volume                   = {299},

  Abstract                 = {Context Health risks of fine particulate matter of 2.5 {micro}m or less in aerodynamic diameter (PM2.5) have been studied extensively over the last decade. Evidence concerning the health risks of the coarse fraction of greater than 2.5 {micro}m and 10 {micro}m or less in aerodynamic diameter (PM10-2.5) is limited. Objective To estimate risk of hospital admissions for cardiovascular and respiratory diseases associated with PM10-2.5 exposure, controlling for PM2.5. Design, Setting, and Participants Using a database assembled for 108 US counties with daily cardiovascular and respiratory disease admission rates, temperature and dew-point temperature, and PM10-2.5 and PM2.5 concentrations were calculated with monitoring data as an exposure surrogate from January 1, 1999, through December 31, 2005. Admission rates were constructed from the Medicare National Claims History Files, for a study population of approximately 12 million Medicare enrollees living on average 9 miles (14.4 km) from collocated pairs of PM10 and PM2.5 monitors. Main Outcome Measures Daily counts of county-wide emergency hospital admissions for primary diagnoses of cardiovascular or respiratory disease. Results There were 3.7 million cardiovascular disease and 1.4 million respiratory disease admissions. A 10-{micro}g/m3 increase in PM10-2.5 was associated with a 0.36% (95% posterior interval [PI], 0.05% to 0.68%) increase in cardiovascular disease admissions on the same day. However, when adjusted for PM2.5, the association was no longer statistically significant (0.25%; 95% PI, -0.11% to 0.60%). A 10-{micro}g/m3 increase in PM10-2.5 was associated with a nonstatistically significant unadjusted 0.33% (95% PI, -0.21% to 0.86%) increase in respiratory disease admissions and with a 0.26% (95% PI, -0.32% to 0.84%) increase in respiratory disease admissions when adjusted for PM2.5. The unadjusted associations of PM2.5 with cardiovascular and respiratory disease admissions were 0.71% (95% PI, 0.45%-0.96%) for same-day exposure and 0.44% (95% PI, 0.06% to 0.82%) for exposure 2 days before hospital admission. Conclusion After adjustment for PM2.5, there were no statistically significant associations between coarse particulates and hospital admissions for cardiovascular and respiratory diseases.},
  Doi                      = {10.1001/jama.299.18.2172},
  Type                     = {Journal Article},
  Url                      = {http://jama.ama-assn.org/cgi/content/abstract/299/18/2172}
}

@PhdThesis{Peng1998,
  Title                    = {Modelling of Turbulent Flow and Heat Transfer for Building Ventilation},
  Author                   = {Peng, S.},
  Year                     = {1998},
  Type                     = {Thesis},

  University               = {Chalmers University of Technology}
}

@InProceedings{Peng,
  Title                    = {Performance of two equation turbulence models for numerical simulation of ventilated rooms},
  Author                   = {Peng, S.H. and Davidson, L. and Holmberg, S.},
  Booktitle                = {Proceedings Room ventilation 2 pp. 153–160.},
  Pages                    = {153-160},

  Type                     = {Conference Proceedings}
}

@Article{Peng2001,
  Title                    = {Large eddy simulation for turbulent buoyant flow in a confined cavity},
  Author                   = {Peng, Shia-Hui and Davidson, Lars},
  Journal                  = {International Journal of Heat and Fluid Flow},
  Year                     = {2001},
  Number                   = {3},
  Pages                    = {323-331},
  Volume                   = {22},

  ISSN                     = {0142-727X},
  Keywords                 = {Large eddy simulation
Subgrid-scale model
Buoyant flow
Cavity},
  Type                     = {Journal Article},
  Url                      = {http://www.sciencedirect.com/science/article/B6V3G-430G5B2-J/2/04f3ba3dbd05e5853a3987d03e867373}
}

@Article{Pennati2001,
  Title                    = {In vitro steady-flow analysis of systemic-to-pulmonary shunt haemodynamics},
  Author                   = {Pennati, G. and Fiore, G. B. and Migliavacca, F. and Laganà, K. and Fumero, R. and Dubini, G.},
  Journal                  = {Journal of Biomechanics},
  Year                     = {2001},
  Number                   = {1},
  Pages                    = {23-30},
  Volume                   = {34},

  Abstract                 = {A modified Blalock-Taussig shunt is a connection created between the systemic and pulmonary arterial circulations to improve pulmonary perfusion in children with congenital heart diseases. Survival of these patients is critically dependent on blood flow distribution between the pulmonary and systemic circulations which in turn depends upon the flow resistance of the shunt. Previously, we investigated the pressure-flow relationship in rigid shunts with a computational approach, to estimate the pulmonary blood flow rate on the basis of the in vivo measured pressure drop. The present study aims at evaluating in vitro how the anastomotic distensibility and restrictions due to suture presence affect the shunt pressure-flow relationship. Two actual Gore-Tex® shunts (3 and 4 mm diameters) were sutured to compliant conduits by a surgeon and tested at different steady flow rates (0.25-1 l min-1) and pulmonary pressures (3-34 mmHg). Corresponding computational models were also created to investigate the role of the anastomotic restrictions due to sutures. In vitro experiments showed that pulmonary artery pressure affects the pressure-flow relationship of the anastomoses, particularly at the distal site. However, this occurrence scarcely influences the total shunt pressure drop. Comparisons between in vitro and computational models without anastomotic restrictions show that the latter underestimates the in vitro pressure drops at any flow rate. The addition of the anastomotic restrictions (31 and 47% of the original area of 3 and 4 mm shunts, respectively) to the computational models reduces the gap, especially at high shunt flow rate and high pulmonary pressure.},
  ISSN                     = {0021-9290},
  Keywords                 = {Modified Blalock-Taussig shunt
Pressure-flow relationship
Anastomosis
Wall distensibility},
  Type                     = {Journal Article},
  Url                      = {http://www.sciencedirect.com/science/article/B6T82-41V2N4V-3/2/4022ffd0070806d659664a048343d75b}
}

@Book{Pepper1992,
  Title                    = {The Finite Element Method: Basic Concepts and Applications},
  Author                   = {Pepper, D.W. and Heinrich, J.C. },
  Publisher                = { Taylor and Francis},
  Year                     = {1992},

  Address                  = {Bristol},

  Type                     = {Book}
}

@Article{Permutt2007,
  Title                    = {The role of the large airways on smooth muscle contraction in asthma},
  Author                   = {Permutt, S.},
  Journal                  = {Journal of Applied Physiology; doi:10.1152/japplphysiol.00568},
  Year                     = {2007},

  Type                     = {Journal Article}
}

@Article{Perot2006,
  Title                    = {Modeling turbulent dissipation at low and moderate Reynolds numbers},
  Author                   = {Perot, J. B. and Kops, S. M. De Bruyn},
  Journal                  = {Journal of Turbulence},
  Year                     = {2006},
  Pages                    = {N69},
  Volume                   = {7},

  Abstract                 = {The dissipation of kinetic energy is one of the key features of turbulent flows that must be modeled accurately in order to obtain useful engineering predictions. At high Reynolds numbers the assumption of scale separation can be invoked in the modeling of the dissipation process. This paper focuses on the more difficult issue of modeling the dissipation process at moderate and low Reynolds numbers. The low and moderate Reynolds number range is very important for tuning turbulence models and for many practical engineering problems. To approach this problem, an alternative formulation to the classic dissipation scale equation is proposed. The interesting feature of this formulation, an inverse lengthscale equation, is that it captures both the high Reynolds number and low Reynolds number decay limits. A careful assessment of existing data then allows us to clearly identify the region of transition between high and low Re and propose a very simple equation system which can accurately model dissipation at any Reynolds number. The equivalent <i>K</i>/ε model is derived and the proposed model is compared with a number of other low Re dissipation modifications for the <i>K</i>/ε equation system. To complete the discussion, the issue of near-wall dissipation modeling is carefully examined and shown to be fundamentally different from the low Reynolds number limit. This is shown to be an important distinction of practical modeling importance.},
  ISSN                     = {1468-5248},
  Type                     = {Journal Article},
  Url                      = {http://www.informaworld.com/10.1080/14685240600907310}
}

@Article{Persak2011,
  Title                    = {Noninvasive estimation of pharyngeal airway resistance and compliance in children based on volume-gated dynamic MRI and computational fluid dynamics},
  Author                   = {Persak, Steven C and Sin, Sanghun and McDonough, Joseph M and Arens, Raanan and Wootton, David M},
  Journal                  = {Journal of Applied Physiology},
  Year                     = {2011},
  Note                     = {C:\Users\sean\AppData\Roaming\Zotero\Zotero\Profiles\16a4oype.default\zotero\storage\VEADDCR6\noninvasive estimation of pharyngeal airway resistance and compliance in children based on volume-gated dynamic mri and computational fluid dynamics.pdf},
  Pages                    = {1819–1827},
  Volume                   = {111},

  Type                     = {Journal Article},
  Url                      = {http://jap.physiology.org/content/jap/111/6/1819.full.pdf}
}

@Article{Perzl1996,
  Title                    = {Reconstruction of the lung geometry for the simulation of aerosol transport},
  Author                   = {Perzl, M.A. and Schultz, H. and Parezke, H.G. and Englmeier, K.H. and Heyder, J.},
  Journal                  = {Journal of Aerosol Medicine},
  Year                     = {1996},
  Pages                    = {409-418},
  Volume                   = {9},

  Type                     = {Journal Article}
}

@Article{Peters1989,
  Title                    = {The effects of electrostatic and inertial forces on the diffusive deposition of small particles onto large disks: Viscous axisymmetric stagnation point flow approximations},
  Author                   = {Peters, Michael H. and Cooper, Douglas W. and Miller, Robert J.},
  Journal                  = {Journal of Aerosol Science},
  Year                     = {1989},
  Number                   = {1},
  Pages                    = {123-136},
  Volume                   = {20},

  ISSN                     = {0021-8502},
  Type                     = {Journal Article},
  Url                      = {http://www.sciencedirect.com/science/article/pii/0021850289900360}
}

@Article{Peters2004,
  Title                    = {Particle Deposition in Industrial Duct Bends},
  Author                   = {Peters, Thomas M. and Leith, David},
  Journal                  = {Ann Occup Hyg},
  Year                     = {2004},
  Pages                    = {1-8},

  Abstract                 = {A study of particle deposition in industrial duct bends is presented. Particle deposition by size was measured by comparing particle size distributions upstream and downstream of bends that had geometries and flow conditions similar to those used in industrial ventilation. As the interior surface of the duct bend was greased to prevent particle bounce, the results are applicable to liquid drops and solid particles where duct walls are sticky. Factors investigated were: (i) flow Reynolds number (Re = 203 000, 36 000); (ii) particle Reynolds number (10 < Rep{infty} < 200); (iii) particle Stokes number (0.08 < Stk < 16); (iv) bend angle ({theta} = 45{degrees}, 90{degrees}, 180{degrees}); (v) bend curvature ratio (1.7 < R0 < 12); (vi) orientation (horizontal-to-horizontal and horizontal-tovertical); and (vii) construction technique (smooth, gored, segmented). Measured deposition was compared with models developed for bends in small diameter sampling lines (Re < 20 000; Rep{infty} < 13). Whereas deposition measured in this work generally agreed with that estimated with models for particles <30 {micro}m (Stk < 0.7), it was significantly lower than that estimated for larger particles. As the flow around larger particles became increasingly turbulent, the models progressively under-represented drag forces and over-estimated deposition. For particles >20 {micro}m, deposition was slightly greater in the horizontal-to-horizontal orientation than in the horizontal-to-vertical orientation due to gravitational settling. Penetration was not a multiplicative function of bend angle as theory predicts, due to the developing nature of turbulent flow in bends. Deposition in a smooth bend was similar to that in a gored bend; however, a tight radius segmented bend (R0 = 1.7) exhibited much lower deposition. For more gradual bends (3 < R0 < 12), curvature ratio had negligible effect on deposition.},
  Doi                      = {10.1093/annhyg/meh031},
  Type                     = {Journal Article},
  Url                      = {http://annhyg.oxfordjournals.org/cgi/content/abstract/meh031v1}
}

@Article{Peyronnet2005,
  Title                    = {Toward a generation of patient-independent nasal spray devices},
  Author                   = {Peyronnet, L.},
  Journal                  = {Drug Delivery Technology},
  Year                     = {2005},
  Number                   = {1},
  Pages                    = {1-5},
  Volume                   = {5},

  Type                     = {Journal Article}
}

@Article{Phalen2010,
  Title                    = {New developments in aerosol dosimetry},
  Author                   = {Phalen, Robert F. and Mendez, Loyda B. and Oldham, Michael J.},
  Journal                  = {Inhalation Toxicology},
  Year                     = {2010},
  Number                   = {S2},
  Pages                    = {6-14},
  Volume                   = {22},

  Doi                      = {doi:10.3109/08958378.2010.516031},
  Type                     = {Journal Article},
  Url                      = {http://informahealthcare.com/doi/abs/10.3109/08958378.2010.516031}
}

@Book{Piegl1995,
  Title                    = {The NURBS book},
  Author                   = {Piegl, L.A. and Tiller, W},
  Publisher                = {Springer-Verlag},
  Year                     = {1995},

  Address                  = {New York, NY},

  Type                     = {Book}
}

@Article{Pilch1987,
  Title                    = {Use of breakup time data and velocity history data to predict the maximum size of stable fragments for acceleration-induced breakup of a liquid drop},
  Author                   = {Pilch, M. and Erdman, C. A.},
  Journal                  = {International Journal of Multiphase Flow},
  Year                     = {1987},
  Number                   = {6},
  Pages                    = {741-757},
  Volume                   = {13},

  Doi                      = {10.1016/0301-9322(87)90063-2},
  ISSN                     = {0301-9322},
  Type                     = {Journal Article},
  Url                      = {http://www.sciencedirect.com/science/article/pii/0301932287900632}
}

@Article{20465792,
  Title                    = {Rhinitis in the geriatric population},
  Author                   = {Pinto, Jayant and Jeswani, Seema},
  Journal                  = {Allergy, Asthma \& Clinical Immunology},
  Year                     = {2010},
  Number                   = {1},
  Pages                    = {10},
  Volume                   = {6},

  Abstract                 = {The current geriatric population in the United States accounts for approximately 12% of the total population and is projected to reach nearly 20% (71.5 million people) by 2030[1]. With this expansion of the number of older adults, physicians will face the common complaint of rhinitis with increasing frequency. Nasal symptoms pose a significant burden on the health of older people and require attention to improve quality of life. Several mechanisms likely underlie the pathogenesis of rhinitis in these patients, including inflammatory conditions and the influence of aging on nasal physiology, with the potential for interaction between the two. Various treatments have been proposed to manage this condition; however, more work is needed to enhance our understanding of the pathophysiology of the various forms of geriatric rhinitis and to develop more effective therapies for this important patient population.},
  Doi                      = {10.1186/1710-1492-6-10},
  ISSN                     = {1710-1492},
  Pubmedid                 = {20465792},
  Url                      = {http://www.aacijournal.com/content/6/1/10}
}

@Article{Pinto2010,
  Title                    = {Rhinitis in the geriatric population},
  Author                   = {Pinto, Jayant and Jeswani, Seema},
  Journal                  = {Allergy, Asthma \& Clinical Immunology},
  Year                     = {2010},
  Number                   = {1},
  Pages                    = {10},
  Volume                   = {6},

  Abstract                 = {The current geriatric population in the United States accounts for approximately 12% of the total population and is projected to reach nearly 20% (71.5 million people) by 2030[1]. With this expansion of the number of older adults, physicians will face the common complaint of rhinitis with increasing frequency. Nasal symptoms pose a significant burden on the health of older people and require attention to improve quality of life. Several mechanisms likely underlie the pathogenesis of rhinitis in these patients, including inflammatory conditions and the influence of aging on nasal physiology, with the potential for interaction between the two. Various treatments have been proposed to manage this condition; however, more work is needed to enhance our understanding of the pathophysiology of the various forms of geriatric rhinitis and to develop more effective therapies for this important patient population.},
  Doi                      = {10.1186/1710-1492-6-10},
  ISSN                     = {1710-1492},
  Pubmedid                 = {20465792},
  Url                      = {http://www.aacijournal.com/content/6/1/10}
}

@Article{Pless2004,
  Title                    = {Numerical simulation of air temperature and airflow patterns in the human nose during expiration},
  Author                   = {Pless, D. and Keck, T. and Wiesmiller, K. and Rettinger, G. and Aschoff, A.J. and Fleiter, T.R. and Lindemann, J.},
  Journal                  = {Clinical Otolaryngology},
  Year                     = {2004},
  Note                     = {10.1111/j.1365-2273.2004.00862.x},
  Number                   = {6},
  Pages                    = {642-647},
  Volume                   = {29},

  Abstract                 = {pless d., keck t., wiesmiller k., rettinger g., aschoff a.j., fleiter t.r. & lindemann j.(2004) Clin. Otolaryngol.29, 6422013647 Numerical simulation of air temperature and airflow patterns in the human nose during expiration Recovery of heat and water during expiration is an important but not yet fully understood function of the nose. The presented study investigated cooling of the expiratory air for heat recovery within the human nose applying numerical simulation. A numerical simulation in a bilateral three-dimensional model of the human nose based on computed tomography was employed. Temperature distribution and airflow patterns during expiration were displayed. Cooling of the expiratory air primarily takes place in the areas of inferior and middle turbinate. Areas of the highest decrease in temperature are characterized by turbulent airflow with vortices of low velocity. Numerical results showed good concordance with experimental in vivo temperature measurements. Heating of inspired air not only depends on inspiration but also on expiration. Cooling the warm expiratory air may be regarded as an important factor for heat recovery. Furthermore, the results demonstrate the close relation between heat exchange and airflow patterns.},
  ISSN                     = {1365-2273},
  Type                     = {Journal Article},
  Url                      = {http://dx.doi.org/10.1111/j.1365-2273.2004.00862.x}
}

@Article{Poncet2009,
  Title                    = {High-order LES of turbulent heat transfer in a rotor-stator cavity},
  Author                   = {Poncet, Sébastien and Serre, Éric},
  Journal                  = {International Journal of Heat and Fluid Flow},
  Year                     = {2009},
  Number                   = {4},
  Pages                    = {590-601},
  Volume                   = {30},

  Abstract                 = {The present work examines the turbulent flow in an enclosed rotor-stator system subjected to heat transfer effects. Besides their fundamental importance as three-dimensional prototype flows, such flows arise in many industrial applications but also in many geophysical and astrophysical settings. Large eddy simulations (LES) are here performed using a spectral vanishing viscosity technique. The LES results have already been favorably compared to velocity measurements in the isothermal case (Séverac, E., Poncet, S., Serre, E., Chauve, M.P., 2007. Large eddy simulation and measurements of turbulent enclosed rotor-stator flows. Phys. Fluids, 19, 085113) for a large range of Reynolds numbers 105[less-than-or-equals, slant]Re=[Omega]b2/[nu][less-than-or-equals, slant]106, in an annular cavity of large aspect ratio G=(b-a)/H=5 and weak curvature parameter Rm=(b-a)/(b+a)=1.8 (a,b the inner and outer radii of the rotor and H the interdisk spacing). The purpose of this paper is to extend these previous results in the non-isothermal case using the Boussinesq approximation to take into account the buoyancy effects. Thus, the effects of thermal convection have been examined for a turbulent flow Re=106 of air in the same rotor-stator system for Rayleigh numbers up to Ra=108. These LES results provide accurate, instantaneous quantities which are of interest in understanding the physics of turbulent flows and heat transfers in an interdisk cavity. Even at high Rayleigh numbers, the structure of the iso-values of the instantaneous normal temperature gradient at the disk surfaces resembles the one of the iso-values of the tangential velocity with large spiral arms along the rotor and more thin structures along the stator. The averaged results show small effects of density variation on the mean and turbulent fields. The turbulent Prandtl number is a decreasing function of the distance to the wall with 1.4 close to the disks and about 0.3 in the outer layers. The local Nusselt number is found to be proportional to the local Reynolds number to the power 0.7. The evolution of the averaged Bolgiano length scale <LB> with the Rayleigh number indicates that temperature fluctuations may have a large influence on the dynamics only at the largest scales of the system for Ra[greater-or-equal, slanted]107, since <LB> remains lower than the thermal boundary layer thicknesses.},
  ISSN                     = {0142-727X},
  Keywords                 = {Large eddy simulation
Rotor-stator
Heat transfer
Boussinesq approximation},
  Type                     = {Journal Article},
  Url                      = {http://www.sciencedirect.com/science/article/B6V3G-4VPV8S4-1/2/4b90cbae99425c39fe0d8a2e7e8688e5}
}

@Article{Pope1991,
  Title                    = {Stochastic lagrangian models for turbulence,},
  Author                   = {Pope, S.B.},
  Journal                  = {Annual Review of Fluid Mechanics},
  Year                     = {1991},
  Pages                    = {23-63},
  Volume                   = {26},

  Type                     = {Journal Article}
}

@Article{Popel2005,
  Title                    = {Microcirculation and Hemorheology},
  Author                   = {Popel, A. S. and Johnson, P. C.},
  Journal                  = {Annu Rev Fluid Mech},
  Year                     = {2005},
  Note                     = {Popel, Aleksander S
Johnson, Paul C
R01 HL052684-04/HL/NHLBI NIH HHS/United States
Annu Rev Fluid Mech. 2005 Jan 1;37:43-69.},
  Pages                    = {43-69},
  Volume                   = {37},

  Abstract                 = {Major experimental and theoretical studies on microcirculation and hemorheology are reviewed with the focus on mechanics of blood flow and the vascular wall. Flow of the blood formed elements (red blood cells (RBCs), white blood cells or leukocytes (WBCs) and platelets) in individual arterioles, capillaries and venules, and in microvascular networks is discussed. Mechanical and rheological properties of the formed elements and their interactions with the vascular wall are reviewed. Short-term and long-term regulation of the microvasculature is discussed; the modes of regulation include metabolic, myogenic and shear-stress-dependent mechanisms as well as vascular adaptation such as angiogenesis and vascular remodeling.},
  Doi                      = {10.1146/annurev.fluid.37.042604.133933},
  ISSN                     = {0066-4189 (Print)
0066-4189 (Linking)},
  Type                     = {Journal Article}
}

@Article{Popiolek2006,
  Title                    = {Finite element analysis of laminar and turbulent flows using LES and subgrid-scale models},
  Author                   = {Popiolek, T. L. and Awruch, A. M. and Teixeira, P. R. F.},
  Journal                  = {Applied Mathematical Modelling},
  Year                     = {2006},
  Number                   = {2},
  Pages                    = {177-199},
  Volume                   = {30},

  Abstract                 = {Numerical simulations of laminar and turbulent flows in a lid driven cavity and over a backward-facing step are presented in this work. The main objectives of this research are to know more about the structure of turbulent flows, to identify their three-dimensional characteristic and to study physical effects due to heat transfer. The filtered Navier-Stokes equations are used to simulate large scales, however they are supplemented by subgrid-scale (SGS) models to simulate the energy transfer from large scales toward subgrid-scales, where this energy will be dissipated by molecular viscosity. Two SGS models are applied: the classical Smagorinsky's model and the Dynamic model for large eddy simulation (LES). Both models are implemented in a three-dimensional finite element code using linear tetrahedral elements. Qualitative and quantitative aspects of two and three-dimensional flows in a lid-driven cavity and over a backward-facing step, using LES, are analyzed comparing numerical and experimental results obtained by other authors.},
  ISSN                     = {0307-904X},
  Keywords                 = {Laminar and turbulent flows
Large eddy simulation
Finite elements
Subgrid-scale model},
  Type                     = {Journal Article},
  Url                      = {http://www.sciencedirect.com/science/article/B6TYC-4G65CG5-5/2/af5286cb39f97dfc2b329fbea98eb5d5}
}

@Article{Posner2003,
  Title                    = {Measurement and prediction of indoor air flow in a model room},
  Author                   = {Posner, J.D. and Buchanan, C.R. and Dunn-Rankin, D.},
  Journal                  = {Energy and Buildings},
  Year                     = {2003},
  Number                   = {5},
  Pages                    = {515-526},
  Volume                   = {35},

  Type                     = {Journal Article}
}

@Article{Poussou2010,
  Title                    = {Flow and contaminant transport in an airliner cabin induced by a moving body: model experiments and CFD prediction},
  Author                   = {Poussou, S.B. and Mazumdar, S. and Plesniak, M.W. and Sojka, P.E. and Chen, Q.},
  Journal                  = {Atmospheric Environment},
  Year                     = {2010},
  Number                   = {24},
  Pages                    = {2830-2839},
  Volume                   = {44},

  Type                     = {Journal Article}
}

@Article{Poussou2010a,
  Title                    = {Flow and contaminant transport in an airliner cabin induced by a moving body: Model experiments and CFD predictions},
  Author                   = {Poussou, Stephane B. and Mazumdar, Sagnik and Plesniak, Michael W. and Sojka, Paul E. and Chen, Qingyan},
  Journal                  = {Atmospheric Environment},
  Year                     = {2010},
  Number                   = {24},
  Pages                    = {2830-2839},
  Volume                   = {44},

  Abstract                 = {The effects of a moving human body on flow and contaminant transport inside an aircraft cabin were investigated. Experiments were performed in a one-tenth scale, water-based model. The flow field and contaminant transport were measured using the Particle Image Velocimetry (PIV) and Planar Laser-Induced Fluorescence (PLIF) techniques, respectively. Measurements were obtained with (ventilation case) and without (baseline case) the cabin environmental control system (ECS). The PIV measurements show strong intermittency in the instantaneous near-wake flow. A symmetric downwash flow was observed along the vertical centerline of the moving body in the baseline case. The evolution of this flow pattern is profoundly perturbed by the flow from the ECS. Furthermore, a contaminant originating from the moving body is observed to convect to higher vertical locations in the presence of ventilation. These experimental data were used to validate a Computational Fluid Dynamic (CFD) model. The CFD model can effectively capture the characteristic flow features and contaminant transport observed in the small-scale model.},
  Doi                      = {http://dx.doi.org/10.1016/j.atmosenv.2010.04.053},
  ISSN                     = {1352-2310},
  Keywords                 = {PIV
PLIF
CFD
Contaminant transport
Aircraft cabin
Ventilation
Human wake},
  Type                     = {Journal Article},
  Url                      = {http://www.sciencedirect.com/science/article/pii/S1352231010003602}
}

@Article{Poussou2010b,
  Title                    = {Flow and contaminant transport in an airliner cabin induced by a moving body: Model experiments and CFD predictions},
  Author                   = {Poussou, Stephane B. and Mazumdar, Sagnik and Plesniak, Michael W. and Sojka, Paul E. and Chen, Qingyan},
  Journal                  = {Atmospheric Environment},
  Year                     = {2010},
  Pages                    = {2830-2839},
  Volume                   = {44},

  Abstract                 = {The effects of a moving human body on flow and contaminant transport inside an aircraft cabin were investigated. Experiments were performed in a one-tenth scale, water-based model. The flow field and contaminant transport were measured using the Particle Image Velocimetry (PIV) and Planar Laser-Induced Fluorescence (PLIF) techniques, respectively. Measurements were obtained with (ventilation case) and without (baseline case) the cabin environmental control system (ECS). The PIV measurements show strong intermittency in the instantaneous near-wake flow. A symmetric downwash flow was observed along the vertical centerline of the moving body in the baseline case. The evolution of this flow pattern is profoundly perturbed by the flow from the ECS. Furthermore, a contaminant originating from the moving body is observed to convect to higher vertical locations in the presence of ventilation. These experimental data were used to validate a Computational Fluid Dynamic (CFD) model. The CFD model can effectively capture the characteristic flow features and contaminant transport observed in the small-scale model.},
  Doi                      = {http://dx.doi.org/10.1016/j.atmosenv.2010.04.053},
  ISSN                     = {1352-2310},
  Keywords                 = {Aircraft
cabin
CFD
Contaminant
Human
PIV
PLIF
transport
Ventilation
wake},
  Type                     = {Journal Article},
  Url                      = {http://ac.els-cdn.com/S1352231010003602/1-s2.0-S1352231010003602-main.pdf?_tid=db0292d4-421f-11e4-9998-00000aab0f01&acdnat=1411366711_397fa482c6776d7aa629d9abe4c4cf12}
}

@InBook{Pozorski2008,
  Title                    = {Analysis of SGS Particle Dispersion Model in LES of Channel Flow},
  Author                   = {Pozorski, J. and Luniewski, M.L.},
  Editor                   = {Meyers, J. and Geurts, B.J. and Sagaut, P.},
  Pages                    = {331-342},
  Publisher                = {Springer},
  Year                     = {2008},

  Address                  = {Netherlands},
  Type                     = {Book Section},
  Volume                   = {12},

  Booktitle                = {Quality and Reliability of Large-Eddy Simulations}
}

@Book{Pozrikidis2003,
  Title                    = {Modeling and Simulation of Capsules and Biological Cells},
  Author                   = {Pozrikidis, C.},
  Publisher                = {Chapman and Hall},
  Year                     = {2003},

  Address                  = {Boca Raton},
  Series                   = {CRC Mathematical Biology and Medicine Series},

  Type                     = {Book}
}

@Book{Pozrikidis2002,
  Title                    = {A Practical Guide to Boundary Element Methods with the Software Library BEMLIB},
  Author                   = {Pozrikidis, C. },
  Publisher                = {Chapman and Hall},
  Year                     = {2002},

  Address                  = {Boca Raton},
  Series                   = {CRC Mathematical Biology and Medicine Series},

  Type                     = {Book}
}

@Article{Pozrikidis1990,
  Title                    = {Axisymmetric deformation of a red blood cell in uniaxial straining Stokes flow},
  Author                   = {Pozrikidis, C.},
  Journal                  = {Journal of Fluid Mechanics},
  Year                     = {1990},
  Note                     = {Cited By (since 1996): 37
Export Date: 5 June 2011
Source: Scopus},
  Pages                    = {231-254},
  Volume                   = {216},

  Type                     = {Journal Article},
  Url                      = {http://www.scopus.com/inward/record.url?eid=2-s2.0-0025460020&partnerID=40&md5=fb9f7c0ba8774a39e4907ea790a7be35}
}

@Article{Primiano1988,
  Title                    = {Water vapour and temperature dynamics in the upper airways of normal and CF subjects},
  Author                   = {Primiano, F.J. and Saidel, G.M. and Montague, F.W. and Kruse, K.L. and Green, C.G. and Horowitz, J.G.},
  Journal                  = {Eur Respir J},
  Year                     = {1988},
  Pages                    = {407-414},
  Volume                   = {1},

  Type                     = {Journal Article}
}

@InBook{Proctor1982,
  Title                    = {The upper airway},
  Author                   = {Proctor, D.F.},
  Editor                   = {Proctor, D.F. and Anderson, I.},
  Pages                    = {22-43},
  Publisher                = {Elsevier Biomedical Press},
  Year                     = {1982},

  Address                  = {New York},
  Type                     = {Book Section},

  Booktitle                = { The Nose}
}

@Article{Proetz1951,
  Title                    = {Air currents in the upper respiratory tract and their clinical importance},
  Author                   = {Proetz, A.W.},
  Journal                  = {Ann Otol Rhinol Laryngol },
  Year                     = {1951},
  Pages                    = {439-467},
  Volume                   = {60},

  Type                     = {Journal Article}
}

@TechReport{Program2005,
  Title                    = {Wood dust, Report on Carcinogens, Eleventh Edition},
  Author                   = {National Toxicology Program},
  Institution              = {U.S. Department of Health and Human Services, Public Health Service},
  Year                     = {2005},
  Type                     = {Report}
}

@Article{Prota2011,
  Title                    = {Leucine enhances aerosol performance of Naringin dry powder and its activity on cystic fibrosis airway epithelial cells},
  Author                   = {Prota, Lucia and Santoro, Antonietta and Bifulco, Maurizio and Aquino, Rita P. and Mencherini, Teresa and Russo, Paola},
  Journal                  = {International Journal of Pharmaceutics},
  Year                     = {2011},
  Number                   = {1-2},
  Pages                    = {8-19},
  Volume                   = {412},

  Abstract                 = {The effect of different amino acids (AAs) on the aerosol performance of N spray-dried powders was studied. Morphology, size distribution, density, dissolution rate were evaluated and correlated to process parameters. The aerosol performance was analyzed by both Single Stage Glass Impinger and Andersen Cascade Impactor. Results indicated that powders containing 5% (w/w) of leucine, proline or histidine and dried from 3:7 ethanol/water feeds showed very satisfying aerodynamic properties with fine particle fraction > 60%. Both neat N (raw and spray-dried) and N-leu1 dry-powder showing good aerodynamic properties were tested in cystic fibrosis (CF) and normal bronchial epithelial cells. Cell proliferation and expression levels of the key enzymes of the NF-[kappa]B and MAPK/ERK pathways, overactivated in CF cell lines, were evaluated. N-leu1 was able to significantly inhibit the expression levels of IKK[alpha], IKK[beta], as well as of the direct NF-[kappa]B inhibitor, I[kappa]B[alpha]. In addition N-Leu1 inhibited phosphorylation of ERK1/2 kinase and did not reduce cell proliferation as observed for the neat raw drug. Leucine co-spray-dried with the drug improved both aerodynamic properties and in vitro pharmacological activity of Naringin. The optimized N-Leu formulation as dry powder is potentially able to reduce hyperinflammatory status associated to CF.},
  Doi                      = {10.1016/j.ijpharm.2011.03.055},
  ISSN                     = {0378-5173},
  Keywords                 = {Naringin
Leucine
Turbospin®
Cystic Fibrosis
airway epithelial cells
NF-[kappa]B
ERK 1/2 inhibition},
  Type                     = {Journal Article},
  Url                      = {http://www.sciencedirect.com/science/article/pii/S0378517311002912}
}

@Article{Pui1987,
  Title                    = {Experimental Study of Particle Deposition in Bends of Circular Cross Section},
  Author                   = {Pui, David Y. H. and Romay-Novas, Francisco and Liu, Benjamin Y. H.},
  Journal                  = {Aerosol Science and Technology},
  Year                     = {1987},
  Number                   = {3},
  Pages                    = {301 - 315},
  Volume                   = {7},

  Abstract                 = {The deposition efficiency of liquid particles in tube bends of circular cross section has been measured for flow Reynolds numbers of 100, 1000, 6000, and 10,000. The particle Reynolds number, <i>Re</i><sub>p</sub>, was in the range 0.6–3.9 for the laminar flow cases (i.e., <i>Re</i> = 100 and 1000), whereas for the turbulent flow cases (i.e., <i>Re</i> = 6000 and 10,000) <i>Re</i><sub>p</sub> was in the range 1.3–12.7. Bends constructed of stainless steel and glass tubes of different diameters were used. The experiments were performed using monodisperse aerosols generated by the vibrating orifice aerosol generator. The results were in good agreement with the theory of Cheng and Wang for <i>Re</i> = 1000, but differed from theory for <i>Re</i> = 100. For the turbulent cases, no dependence was found on the flow Reynolds number and an exponential curve of deposition efficiency versus Stokes number was fitted to the experimental results. A theoretical justification of the form of the curve is given in the paper.},
  ISSN                     = {0278-6826},
  Type                     = {Journal Article},
  Url                      = {http://www.informaworld.com/10.1080/02786828708959166}
}

@Article{Pulazzini2001,
  Title                    = {Sector comes under spotlight},
  Author                   = {Pulazzini, A. and Segantini, L.},
  Journal                  = {Scrip Magazine},
  Year                     = {2001},
  Number                   = {May (101)},
  Pages                    = {7},

  Type                     = {Magazine Article}
}

@Article{PuncochA¡rovA¡-PorA­zkovA¡,
  Title                    = {Numerical solutions of unsteady flows with low inlet Mach numbers},
  Author                   = {Puncochárová-Porízková, Petra and Fürst, Jirí and Horácek, Jaromír and Kozel, Karel},
  Journal                  = {Mathematics and Computers in Simulation},
  Volume                   = {In Press, Corrected Proof},

  Abstract                 = {This study deals with a numerical solution of a 2D unsteady flow of a compressible viscous fluid in a channel for low inlet airflow velocity. The unsteadiness of the flow is caused by a prescribed periodic motion of a part of the channel wall with large amplitudes, nearly closing the channel during oscillations. The channel is a simplified model of the glottal space in the human vocal tract and the flow can represent a model of airflow coming from the trachea, through the glottal region with periodically vibrating vocal folds, and to the human vocal tract. The flow is described by a system of Navier-Stokes equations for laminar flows. The numerical solution is implemented using the finite volume method (FVM) and the predictor-corrector MacCormack scheme with Jameson artificial viscosity using a grid of quadrilateral cells. Due to the unsteadiness of the grid (motion of the grid), the basic system of conservation laws is considered in a changed form using the Arbitrary Lagrangian-Eulerian (ALE) method. The numerical solution has been carried out for a symmetric and a non-symmetric flow field in two computational domains. The numerical results for unsteady flows in the channel are presented for inlet Mach number M[infinity]=0.012, Reynolds number Re=5×103 and wall motion frequency 100 Hz. The authors present the numerical solution and simulations of flow fields in the channel, acquired from a program that has been developed exclusively for this purpose. The reason for using a numerical simulation was the lack of data from experimental results of flows through the glottal region, and the capability of such simulation to model the problem.},
  ISSN                     = {0378-4754},
  Keywords                 = {CFD
Finite volume method
Unsteady flow
Low Mach number
Viscous compressible fluid},
  Type                     = {Journal Article},
  Url                      = {http://www.sciencedirect.com/science/article/B6V0T-4Y0D67D-1/2/3e73f2583277924fc368bffcab0704fa}
}

@Article{raey,
  Title                    = {Review of computational fluid dynamics in the assessment of nasal air flow and analysis of its limitations},
  Author                   = {Quadrio, Maurizio and Pipolo, Carlotta and Corti, Stefano and Lenzi, Riccardo and Messina, Francesco and Pesci, Chiara and Felisati, Giovanni},
  Journal                  = {European Archives of Oto-Rhino-Laryngology},
  Year                     = {2014},
  Number                   = {9},
  Pages                    = {2349-2354},
  Volume                   = {271},

  Doi                      = {10.1007/s00405-013-2742-3},
  ISSN                     = {0937-4477},
  Keywords                 = {Computational fluid dynamics; Nasal cavity; 3D model; Airflow; Surgical planning},
  Language                 = {English},
  Publisher                = {Springer Berlin Heidelberg},
  Url                      = {http://dx.doi.org/10.1007/s00405-013-2742-3}
}

@Book{Raabe1976,
  Title                    = {Tracheobronchial Geometry: Human, Dog, Rat and Hamster. LF53. },
  Author                   = {Raabe, O.G. and Yeh, H.C. and G.M., Schum and R.F., Phalen},
  Year                     = {1976},

  Address                  = {New Mexico},
  Series                   = {Lovelace Foundation Report},

  Type                     = {Book}
}

@TechReport{Raabe1976a,
  Title                    = {Tracheobronchial geometry: Human, dog, rat, hamster},
  Author                   = {Raabe, O.G. and Yeh, H.C. and Schum, G.M. and Phalen, R.F. },
  Institution              = {Lovelace Foundation},
  Year                     = {1976},
  Type                     = {Report}
}

@Book{Radcliffe1960,
  Title                    = {Fuel Injection},
  Author                   = {Radcliffe, A.},
  Publisher                = {Princeton University Press},
  Year                     = {1960},

  Address                  = {Princeton, NJ, USA},
  Series                   = {High Speed Aerodynamics and Jet Propulsion},
  Volume                   = {XI},

  Type                     = {Edited Book}
}

@Book{RadjaA¯2011,
  Title                    = {Discrete-element Modeling of Granular Materials},
  Author                   = {Radjaï , F. and Dubois, F.},
  Publisher                = {Wiley},
  Year                     = {2011},

  Address                  = {University of Montpellier, France},

  Type                     = {Edited Book}
}

@Article{Rahman2007,
  Title                    = {A simplified v2-f model for near-wall turbulence},
  Author                   = {Rahman, M. M. and Siikonen, T.},
  Journal                  = {International Journal for Numerical Methods in Fluids},
  Year                     = {2007},
  Note                     = {10.1002/fld.1411},
  Number                   = {12},
  Pages                    = {1387-1406},
  Volume                   = {54},

  ISSN                     = {1097-0363},
  Type                     = {Journal Article},
  Url                      = {http://dx.doi.org/10.1002/fld.1411}
}

@Article{Rasani2011,
  Title                    = {Simulation of Pharyngeal Airway Interaction with Air Flow Using Low-Re Turbulence Model},
  Author                   = {Rasani, M.R. and Inthavong, K. and Tu, J.Y.},
  Journal                  = {Modelling and Simulation in Engineering},
  Year                     = {2011},
  Number                   = {510472, },
  Pages                    = {1-9},
  Volume                   = {2011},

  Doi                      = {10.1155/2011/510472},
  Type                     = {Journal Article}
}

@Article{Ratnasingam2010,
  Title                    = {Particle Size Distribution of Wood Dust in Rubberwood (Hevea Brasiliensis) Furniture Manufacturing},
  Author                   = {Ratnasingam, J. and Scholz, F. and Natthondan, V.},
  Journal                  = {European Journal of Wood and Wood Products},
  Year                     = {2010},
  Number                   = {2},
  Pages                    = {241-242},
  Volume                   = {68},

  Abstract                 = {Subject&nbsp;&nbsp;The study evaluated the concentration and particle size distribution of air-borne wood dust in the Rubberwood furniture manufacturing industry. Air quality samples were measured at routing and hand-sanding work stations in furniture factories using the micro-orifice uniform deposit impactor (MOUDI) air-quality measuring instrument. It was found that less than 25% of the air-borne wood dust particles at the two work stations were less than 10 μm in size, which in turn did not pose major respiratory health hazards. However, the high wood dust concentrations at the two work stations is a&nbsp;matter of concern, and efforts must be taken to minimize the air-borne wood dust exposure levels workers are subjected to in the Rubberwood furniture manufacturing industry.},
  Type                     = {Journal Article},
  Url                      = {http://dx.doi.org/10.1007/s00107-009-0369-2}
}

@Article{Rautio2007,
  Title                    = {Modelling of airborne dust emissions in CNC MDF milling},
  Author                   = {Rautio, Sari and Hynynen, Pasi and Welling, Irma and Hemmilä, Pasi and Usenius, Arto and Närhi, Pertti},
  Journal                  = {European Journal of Wood and Wood Products},
  Year                     = {2007},
  Number                   = {5},
  Pages                    = {335-341},
  Volume                   = {65},

  Abstract                 = {Abstract&nbsp;&nbsp;All dust control measures are necessary to reduce dust exposure in MDF (Medium Density Fibreboard)-milling, because of the high amount and fineness of the dust produced and a&nbsp;potential risk of exposure to formaldehyde or other glue chemicals during the machining of MDF. The most effective way of reducing dust exposure is to reduce the emission of dust at the source. Airborne dust emission was studied and modelled in the milling. In the milling of MDF, airborne dust emission was much higher than in the milling of solid materials. Milling of MDF produced airborne particles with a&nbsp;mass median diameter of 6–7&nbsp;µm. The most significant factor affecting the amount of dust created from milling was average chip thickness. In order to reduce the amount of dust, milling parameters should be chosen so that the average chip thickness is greater than 0.05&nbsp;mm. The average chip thickness could be obtained with different milling parameters, for example with different combinations of feed and traverse rates. The same chip thicknesses resulted in around the same percentage fraction of fine dust mass regardless of how the average chip thickness was obtained. The relationship between the percentage fraction of fine dust mass from the removed mass (c%) and the chip thickness (hm) was modelled and presented in the form of c% = 0.194&nbsp;h m -1. The model developed can be used to estimate the percentage fraction of fine dust mass as a&nbsp;function of chip thickness. The model can be used in optimisation programs for CNC milling machines to minimize the airborne dust generated and to reduce dust exposure.},
  Type                     = {Journal Article},
  Url                      = {http://dx.doi.org/10.1007/s00107-007-0179-3}
}

@Article{Reed1974,
  Title                    = {Particle interactions in viscous flow at small values of knudsen number},
  Author                   = {Reed, L. D. and Morrison Jr, F. A.},
  Journal                  = {Journal of Aerosol Science},
  Year                     = {1974},
  Note                     = {doi: DOI: 10.1016/0021-8502(74)90048-2},
  Number                   = {2},
  Pages                    = {175-189},
  Volume                   = {5},

  ISSN                     = {0021-8502},
  Type                     = {Journal Article},
  Url                      = {http://www.sciencedirect.com/science/article/B6V6B-4893W3W-4B/2/cc9f0620737f6d1f9f288f04a4dd510c}
}

@Article{Reeks1983,
  Title                    = {The transport of discrete particles in inhomogeneous turbulence},
  Author                   = {Reeks, M. W.},
  Journal                  = {Journal of Aerosol Science},
  Year                     = {1983},
  Note                     = {doi: DOI: 10.1016/0021-8502(83)90055-1},
  Number                   = {6},
  Pages                    = {729-739},
  Volume                   = {14},

  ISSN                     = {0021-8502},
  Type                     = {Journal Article},
  Url                      = {http://www.sciencedirect.com/science/article/B6V6B-48BCVCF-5B/2/318ba6947cb21144e44b9abc04e52245}
}

@Article{Reeks1977,
  Title                    = {On the dispersion of small particles suspended in an isotropic turbulent fluid},
  Author                   = {Reeks, M. W.},
  Journal                  = {Journal of Fluid Mechanics},
  Year                     = {1977},
  Note                     = {Cited By (since 1996): 122
Export Date: 5 June 2011
Source: Scopus},
  Number                   = {pt 3},
  Pages                    = {529-546},
  Volume                   = {83},

  Type                     = {Journal Article},
  Url                      = {http://www.scopus.com/inward/record.url?eid=2-s2.0-0017768366&partnerID=40&md5=32bd75eccf9aeff73e11a73706b2142b}
}

@Article{Reeks1984,
  Title                    = {The dispersive effects of Basset history forces on particle motion in a turbulent flow},
  Author                   = {Reeks, M. W. and McKee, S.},
  Journal                  = {Physics of Fluids},
  Year                     = {1984},
  Note                     = {Cited By (since 1996): 21
Export Date: 5 June 2011
Source: Scopus},
  Number                   = {7},
  Pages                    = {1573-1582},
  Volume                   = {27},

  Type                     = {Journal Article},
  Url                      = {http://www.scopus.com/inward/record.url?eid=2-s2.0-0021469293&partnerID=40&md5=437e923c66443aa100583935ba66538e}
}

@Article{Reeves-Hoche1994,
  Title                    = {Nasal CPAP: an objective evaluation of patient compliance},
  Author                   = {Reeves-Hoche, MK and Meck, R and Zwillich, CW},
  Journal                  = {Am. J. Respir. Crit. Care Medicine},
  Year                     = {1994},
  Number                   = {1},
  Pages                    = {149-154},
  Volume                   = {149},

  Type                     = {Journal Article},
  Url                      = {http://ajrccm.atsjournals.org/cgi/content/abstract/149/1/149}
}

@Article{Reitz1987,
  Title                    = {Structure of High-Pressure Fuel Sprays},
  Author                   = {Reitz, R.D. and Diwakar, R.},
  Journal                  = {SAE Technical Paper, 870598},
  Year                     = {1987},

  Type                     = {Journal Article}
}

@Article{ReneeAnthony2005,
  Title                    = {Computational fluid dynamics analysis of particle inhalability},
  Author                   = {Renee Anthony, T and Flynn, Michael R},
  Journal                  = {Journal of aerosol science},
  Year                     = {2005},
  Pages                    = {16},
  Volume                   = {37},

  Type                     = {Journal Article}
}

@Article{ReneeAnthony2005a,
  Title                    = {Computational fluid dynamics analysis of particle inhalability},
  Author                   = {Renee Anthony, T and Flynn, Michael R},
  Journal                  = {Journal of aerosol science},
  Year                     = {2005},
  Pages                    = {16},
  Volume                   = {37},

  Type                     = {Journal Article}
}

@Article{Rennen2004,
  Title                    = {Oral-to-inhalation route extrapolation in occupational health risk assessment: a critical assessment},
  Author                   = {Rennen, Monique A. J. and Bouwman, Tialda and Wilschut, Annette and Bessems, Jos G. M. and Heer, Cees De},
  Journal                  = {Regulatory Toxicology and Pharmacology},
  Year                     = {2004},
  Number                   = {1},
  Pages                    = {5-11},
  Volume                   = {39},

  Doi                      = {10.1016/j.yrtph.2003.09.003},
  ISSN                     = {0273-2300},
  Keywords                 = {Route extrapolation
Inhalation
Respiratory
Oral
Risk assessment
Local effects},
  Type                     = {Journal Article},
  Url                      = {http://www.sciencedirect.com/science/article/pii/S0273230003001259}
}

@Article{Rhodin2007,
  Title                    = {Numerical predictions of indoor climate in large industrial premises. A comparison between different k-e models supported by field measurements.},
  Author                   = {Rhodin, P. and Moshfegh, B.},
  Journal                  = {Building and Environment},
  Year                     = {2007},
  Pages                    = {3872-3882},
  Volume                   = {42},

  Type                     = {Journal Article}
}

@Article{Richmond2004,
  Title                    = {Sir Godfrey Hounsfield},
  Author                   = {Richmond, C.},
  Journal                  = {British Medical Journal},
  Year                     = {2004},
  Number                   = {7467},
  Pages                    = {687},
  Volume                   = {329},

  Type                     = {Journal Article}
}

@Article{Ridler1978,
  Title                    = {Picture thresholding using an iterative selection method},
  Author                   = {Ridler, T. W. and Calvard, S.},
  Journal                  = { IEEE Trans. Systems, Man and Cybernetics},
  Year                     = {1978},
  Pages                    = {630-632},
  Volume                   = {8},

  Type                     = {Journal Article}
}

@Article{Rigopoulos2010,
  Title                    = {Population balance modelling of polydispersed particles in reactive flows},
  Author                   = {Rigopoulos, S.},
  Journal                  = {Progress in Energy and Combustion Science},
  Year                     = {2010},
  Number                   = {4},
  Pages                    = {412-443},
  Volume                   = {36},

  Abstract                 = {Polydispersed particles in reactive flows is a wide subject area encompassing a range of dispersed flows with particles, droplets or bubbles that are created, transported and possibly interact within a reactive flow environment - typical examples include soot formation, aerosols, precipitation and spray combustion. One way to treat such problems is to employ as a starting point the Newtonian equations of motion written in a Lagrangian framework for each individual particle and either solve them directly or derive probabilistic equations for the particle positions (in the case of turbulent flow). Another way is inherently statistical and begins by postulating a distribution of particles over the distributed properties, as well as space and time, the transport equation for this distribution being the core of this approach. This transport equation, usually referred to as population balance equation (PBE) or general dynamic equation (GDE), was initially developed and investigated mainly in the context of spatially homogeneous systems. In the recent years, a growth of research activity has seen this approach being applied to a variety of flow problems such as sooting flames and turbulent precipitation, but significant issues regarding its appropriate coupling with CFD pertain, especially in the case of turbulent flow. The objective of this review is to examine this body of research from a unified perspective, the potential and limits of the PBE approach to flow problems, its links with Lagrangian and multi-fluid approaches and the numerical methods employed for its solution. Particular emphasis is given to turbulent flows, where the extension of the PBE approach is met with challenging issues. Finally, applications including reactive precipitation, soot formation, nanoparticle synthesis, sprays, bubbles and coal burning are being reviewed from the PBE perspective. It is shown that population balance methods have been applied to these fields in varying degrees of detail, and future prospects are discussed.},
  ISSN                     = {0360-1285},
  Keywords                 = {Polydispersed particles
Reactive flows
Population balance
Soot
Spray},
  Type                     = {Journal Article},
  Url                      = {http://www.sciencedirect.com/science/article/B6V3W-4YCNJPS-1/2/844b427e3067720f0117eee0cc01d50e}
}

@Article{Riley1974,
  Title                    = {The Relation of Turbulent Diffusivities to Lagrangian Velocity Statistics for the Simplest Shear Flow},
  Author                   = {Riley, J.J. and Corrsin, S. },
  Journal                  = {Journal of Geophysical. Research},
  Year                     = {1974},
  Pages                    = {1768-1771},
  Volume                   = {27},

  Type                     = {Journal Article}
}

@Article{Riley1974a,
  Title                    = {Diffusion experiments with numerically integrated isotropic turbulence},
  Author                   = {Riley, J. J. and Patterson Jr, G. S.},
  Journal                  = {Physics of Fluids},
  Year                     = {1974},
  Note                     = {Cited By (since 1996): 13
Export Date: 5 June 2011
Source: Scopus},
  Number                   = {2},
  Pages                    = {292-297},
  Volume                   = {17},

  Type                     = {Journal Article},
  Url                      = {http://www.scopus.com/inward/record.url?eid=2-s2.0-0000261608&partnerID=40&md5=a030af410ac67e0397e07ac3f326ec9c}
}

@Article{Rim2010,
  Title                    = {Ventilation effectiveness as an indicator of occupant exposure to particles from indoor sources},
  Author                   = {Rim, Donghyun and Novoselac, Atila},
  Journal                  = {Building and Environment},
  Year                     = {2010},
  Number                   = {5},
  Pages                    = {1214-1224},
  Volume                   = {45},

  Doi                      = {10.1016/j.buildenv.2009.11.004},
  ISSN                     = {0360-1323},
  Keywords                 = {Ventilation effectiveness
Particles
Occupant exposure
Indoor airflow
Source location},
  Type                     = {Journal Article},
  Url                      = {http://www.sciencedirect.com/science/article/pii/S0360132309003291}
}

@Article{Rim2009,
  Title                    = {Transport of particulate and gaseous pollutants in the vicinity of a human body},
  Author                   = {Rim, Donghyun and Novoselac, Atila},
  Journal                  = {Building and Environment},
  Year                     = {2009},
  Number                   = {9},
  Pages                    = {1840-1849},
  Volume                   = {44},

  Doi                      = {http://dx.doi.org/10.1016/j.buildenv.2008.12.009},
  ISSN                     = {0360-1323},
  Keywords                 = {Indoor environment
Particulate matter
Thermal plume
Human exposure},
  Type                     = {Journal Article},
  Url                      = {http://www.sciencedirect.com/science/article/pii/S0360132308002989}
}

@Article{Rim2009a,
  Title                    = {Transport of particulate and gaseous pollutants in the vicinity of a human body},
  Author                   = {Rim, Donghyun and Novoselac, Atila},
  Journal                  = {Building and Environment},
  Year                     = {2009},
  Pages                    = {1840-1849},
  Volume                   = {44},

  Doi                      = {http://dx.doi.org/10.1016/j.buildenv.2008.12.009},
  ISSN                     = {0360-1323},
  Keywords                 = {environment
exposure
Human
Indoor
matter
Particulate
plume
Thermal},
  Type                     = {Journal Article},
  Url                      = {http://ac.els-cdn.com/S0360132308002989/1-s2.0-S0360132308002989-main.pdf?_tid=ddbd61ca-421f-11e4-a65c-00000aacb35d&acdnat=1411366716_ea1f74bf9930cb91957544d7a64a9056}
}

@Article{Rizk1985,
  Title                    = {The motion of a spherical particle suspended in a turbulent flow near a plane wall},
  Author                   = {Rizk, M. A. and Elghobashi, S. E.},
  Journal                  = {Physics of Fluids},
  Year                     = {1985},
  Note                     = {Cited By (since 1996): 40
Export Date: 5 June 2011
Source: Scopus},
  Number                   = {3},
  Pages                    = {806-817},
  Volume                   = {28},

  Type                     = {Journal Article},
  Url                      = {http://www.scopus.com/inward/record.url?eid=2-s2.0-0022269122&partnerID=40&md5=0781d680fdc490a35db74d6627896fbd}
}

@Article{Rizk1985a,
  Title                    = {Internal flow characteristics of simplex swirl atomisers},
  Author                   = {Rizk, N.K. and Lefebvre, A.H.},
  Journal                  = {AIAA Journal Propul. Power},
  Year                     = {1985},
  Number                   = {3},
  Pages                    = {193-199},
  Volume                   = {1},

  Type                     = {Journal Article}
}

@Article{Rizzetta2008,
  Title                    = {A high-order compact finite-difference scheme for large-eddy simulation of active flow control},
  Author                   = {Rizzetta, Donald P. and Visbal, Miguel R. and Morgan, Philip E.},
  Journal                  = {Progress in Aerospace Sciences},
  Year                     = {2008},
  Number                   = {6},
  Pages                    = {397-426},
  Volume                   = {44},

  Abstract                 = {The purpose of this article is to summarize a computational approach, which developed and matured over an extended period of time, and has been shown to be useful for performing large-eddy simulation (LES) of flows with active control. Because of the nature of active flow control, simulation of this class of problems typically cannot be carried out accurately by methods less sophisticated than LES. Active control flowfields are highly unsteady, and can be characterized by small-scale fluid structures which are produced by the control process, but may also be inherent in the original uncontrolled situation. The numerical scheme is predicated upon an implicit time-marching algorithm, and utilizes a high-order compact finite-difference approximation to represent spatial derivatives. Robustness of the scheme is maintained by employing a low-pass Pade-type nondispersive spatial filter, which also accounts for the fine-scale turbulent dissipation that otherwise is traditionally provided by an explicitly added subgrid-scale (SGS) stress model. Geometrically complex applications are accommodated by an overset grid technique, where spatial accuracy is preserved through use of high-order interpolation. Utility of the method is illustrated by specific computational examples, including suppression of acoustic resonance in supersonic cavity flow, leading-edge vortex control of a delta wing, efficiency enhancement of a transitional highly loaded low-pressure turbine blade, and separation control of a wall-mounted hump model. Control techniques represented in these examples are comprised of both steady and pulsed mass injection or removal, as well as plasma-based actuation. For each case, features of the flowfield are elucidated and the solutions are compared to the baseline situation where no control was enforced. Where available, comparisons are also made with experimental data.},
  ISSN                     = {0376-0421},
  Type                     = {Journal Article},
  Url                      = {http://www.sciencedirect.com/science/article/B6V3V-4TD5VVF-3/2/75ff775d8ade023ac9412ad5d7ee4a18}
}

@Article{Robert2001,
  Title                    = {Forced inspiratory nasal flow–volume curves: A simple test of nasal airflow},
  Author                   = {Robert, G.H.},
  Journal                  = {Mayo Clin. Proc.},
  Year                     = {2001},
  Pages                    = {990-994},
  Volume                   = {76},

  Type                     = {Journal Article}
}

@Article{Robinson2006,
  Title                    = {Experimental and Numerical Smoke Carcinogen Deposition in a Multi-Generation Human Replica Tracheobronchial Model},
  Author                   = {Robinson, Risa and Oldham, Michael and Clinkenbeard, Rodney and Rai, Pravir},
  Journal                  = {Annals of Biomedical Engineering},
  Year                     = {2006},
  Note                     = {10.1007/s10439-005-9049-5},
  Number                   = {3},
  Pages                    = {373-383},
  Volume                   = {34},

  Abstract                 = {A better understanding of submicron particle deposition in the respiratory tract is needed to study the health effects caused by carcinogenic particles. Recent studies indicate that random diffusion is not sufficient to describe the motion of these particles in complex geometries, rendering conventional models inaccurate. A solid replica of excised human lung segments was used to create digital and hollow models of the tracheobronchial region to investigate deposition of mainstream (MS) and sidestream (SS) cigarette smoke particles. Particle sizes for the carcinogen Benzo(a)pyrene (BaP) in SS smoke, and total particulate matter (UVPM) in SS and MS smoke were measured and used to compare the simulation to experimental data. Excellent agreement was found between predicted and measured results. Random diffusion was not found to be significant for submicron particles indicating that particles were instead transported to the airway wall by convective diffusion. BaP in SS smoke was an average 0.3&nbsp;μm compared to 0.36&nbsp;μm for UVPM in SS smoke. The trends in both experimental and numerical results indicated that the BaP in SS smoke deposits at a slightly higher efficiency than the UVPM, indicating that carcinogen-specific deposition, rather than total particulate matter should be considered when investigating health effects.},
  Type                     = {Journal Article},
  Url                      = {http://dx.doi.org/10.1007/s10439-005-9049-5}
}

@Article{Robinson2009,
  Title                    = {3D Airway Reconstruction Using Visible Human Data Set and Human Casts with Comparison to Morphometric Data},
  Author                   = {Robinson, Risa J. and Russo, Jackie and Doolittle, Richard L.},
  Journal                  = {The Anatomical Record: Advances in Integrative Anatomy and Evolutionary Biology},
  Year                     = {2009},
  Note                     = {10.1002/ar.20898},
  Number                   = {7},
  Pages                    = {1028-1044},
  Volume                   = {292},

  ISSN                     = {1932-8494},
  Type                     = {Journal Article},
  Url                      = {http://dx.doi.org/10.1002/ar.20898}
}

@Article{Robinson2007,
  Title                    = {Comparison of Particle Tracking Algorithms in Commercial CFD Packages: Sedimentation and Diffusion},
  Author                   = {Robinson, Risa J. and Snyder, Pam and Oldham, Michael J.},
  Journal                  = {Inhalation Toxicology},
  Year                     = {2007},
  Number                   = {6},
  Pages                    = {517 - 531},
  Volume                   = {19},

  ISSN                     = {0895-8378},
  Type                     = {Journal Article},
  Url                      = {http://www.informaworld.com/10.1080/08958370701260889}
}

@Article{Rochefort2007,
  Title                    = {In vitro validation of computational fluid dynamic simulation in human proximal airways with hyperpolarized 3He magnetic resonance phase-contrast velocimetry},
  Author                   = {de Rochefort, Ludovic and Vial, Laurence and Fodil, Redouane and Maitre, Xavier and Louis, Bruno and Isabey, Daniel and Caillibotte, Georges and Thiriet, Marc and Bittoun, Jacques and Durand, Emmanuel and Sbirlea-Apiou, Gabriela},
  Journal                  = {Journal of Applied Physiology},
  Year                     = {2007},
  Number                   = {5},
  Pages                    = {2012-2023},
  Volume                   = {102},

  Abstract                 = {Computational fluid dynamics (CFD) and magnetic resonance (MR) gas velocimetry were concurrently performed to study airflow in the same model of human proximal airways. Realistic in vivo-based human airway geometry was segmented from thoracic computed tomography. The three-dimensional numerical description of the airways was used for both generation of a physical airway model using rapid prototyping and mesh generation for CFD simulations. Steady laminar inspiratory experiments (Reynolds number Re = 770) were performed and velocity maps down to the fourth airway generation were extracted from a new velocity mapping technique based on MR velocimetry using hyperpolarized 3He gas. Full two-dimensional maps of the velocity vector were measured within a few seconds. Numerical simulations were carried out with the experimental flow conditions, and the two sets of data were compared between the two modalities. Flow distributions agreed within 3%. Main and secondary flow velocity intensities were similar, as were velocity convective patterns. This work demonstrates that experimental and numerical gas velocity data can be obtained and compared in the same complex airway geometry. Experiments validated the simulation platform that integrates patient-specific airway reconstruction process from in vivo thoracic scans and velocity field calculation with CFD, hence allowing the results of this numerical tool to be used with confidence in potential clinical applications for lung characterization. Finally, this combined numerical and experimental approach of flow assessment in realistic in vivo-based human airway geometries confirmed the strong dependence of airway flow patterns on local and global geometrical factors, which could contribute to gas mixing.},
  Doi                      = {10.1152/japplphysiol.01610.2005},
  Type                     = {Journal Article},
  Url                      = {http://jap.physiology.org/cgi/content/abstract/102/5/2012}
}

@Article{Roco2005,
  Title                    = {International perspective on government nanotechnology funding in 2005. },
  Author                   = {Roco, M.C.},
  Journal                  = {Journal Nanoparticle Research 7(6). },
  Year                     = {2005},
  Number                   = {6},
  Pages                    = {1-8},
  Volume                   = {7},

  Type                     = {Journal Article}
}

@Article{RodrA­guez2004,
  Title                    = {Immunohistochemical Labelling of Cytokines in Lung Lesions of Pigs Naturally Infected with Mycoplasma hyopneumoniae},
  Author                   = {Rodríguez, F. and Ramírez, G. A. and Sarradell, J. and Andrada, M. and Lorenzo, H.},
  Journal                  = {Journal of Comparative Pathology},
  Year                     = {2004},
  Number                   = {4},
  Pages                    = {306-312},
  Volume                   = {130},

  ISSN                     = {0021-9975},
  Keywords                 = {bacterial infection
cytokines
Mycoplasma hyopneumoniae
pig
porcine enzootic pneumonia},
  Type                     = {Journal Article},
  Url                      = {http://www.sciencedirect.com/science/article/B6WHW-4BXN639-2/2/ad18b222feaeb0dab866048a99995bf8}
}

@Article{RodrA­guez2006,
  Title                    = {Mechanical stress in abdominal aneurysm: influence of GEOMETRY and material anisotropy},
  Author                   = {Rodríguez, J. F. and Ruiz, C. and Holzapfel, G. and Doblaré, M.},
  Journal                  = {Journal of Biomechanics},
  Year                     = {2006},
  Number                   = {Supplement 1},
  Pages                    = {S272-S273},
  Volume                   = {39},

  ISSN                     = {0021-9290},
  Type                     = {Journal Article},
  Url                      = {http://www.sciencedirect.com/science/article/B6T82-4KR88PB-1G8/2/ca686f08d29bd8a8280dc6a1bb7ce52d}
}

@Article{Rogers1994,
  Title                    = {Airway goblet cells: responsive and adaptable front-line defenders},
  Author                   = {Rogers, D.F.},
  Journal                  = {Eur Respir J},
  Year                     = {1994},
  Pages                    = {1690-1708},
  Volume                   = {7},

  Type                     = {Journal Article}
}

@Article{Roisman2007,
  Title                    = {Effect of ambient pressure on penetration of a diesel spray},
  Author                   = {Roisman, I. V. and Araneo, Lucio and Tropea, C.},
  Journal                  = {International Journal of Multiphase Flow},
  Year                     = {2007},
  Note                     = {doi: DOI: 10.1016/j.ijmultiphaseflow.2007.01.004},
  Number                   = {8},
  Pages                    = {904-920},
  Volume                   = {33},

  ISSN                     = {0301-9322},
  Keywords                 = {Spray
Spray penetration
Fuel injection
Ambient pressure},
  Type                     = {Journal Article},
  Url                      = {http://www.sciencedirect.com/science/article/B6V45-4N3P09B-1/2/cb0cab5623bfe3af3c1543d900d4ead6}
}

@Article{Roithmann1995,
  Title                    = {Acoustic rhinometry in the evaluation of nasal obstruction. },
  Author                   = {Roithmann, R. and Cole, P. and Chapnik, J. and Shpirer, I. and Hoffstein, V. and Zamel, N.},
  Journal                  = {Laryngoscope},
  Year                     = {1995},
  Pages                    = {275-281},
  Volume                   = {105},

  Type                     = {Journal Article}
}

@Article{Roland2004,
  Title                    = {The local side effects of inhaled corticosteroids: Current understanding and review of the literature},
  Author                   = {Roland, N.J. and Bhalla, R.K. and Earis, J.},
  Journal                  = {Chest},
  Year                     = {2004},
  Pages                    = {213-219},
  Volume                   = {126},

  Type                     = {Journal Article}
}

@Article{Rombaux,
  Title                    = {Assessment of olfactory and trigeminal function using chemosensory event-related potentials},
  Author                   = {Rombaux, P. and Mouraux, A. and Bertrand, B. and Guerit, J. M. and Hummel, T.},
  Journal                  = {Neurophysiologie Clinique/Clinical Neurophysiology},
  Number                   = {2},
  Pages                    = {53-62},
  Volume                   = {36},

  Abstract                 = {Goals To give an overview on the theoretical and practical applications of chemosensory event-related potentials.Methods Chemosensory event-related potentials (ERPs) may be elicited by brief and precisely defined odorous stimuli. Based on the principles of air-dilution olfactometry, a stimulator was developed in the late 1970s, which allows stimulation of the olfactory neuroepithelium and the nasal mucosa with no concomitant mechanical stimulation. Chemosensory ERPs were obtained after stimulation of the olfactory nerve (olfactory ERPs) or the trigeminal nerve (somatosensory or trigeminal ERPs). The characteristics of the stimulator for chemosensory research as well as the variables influencing the responses are discussed in this paper.Results Implementation and normative data from our department are reported with different clinical examples from otorhinolaryngologic clinic. The bulk of the evoked response consists of a large negative component (often referred to as N1), which occurs between 320 and 450 ms after stimulus onset. This component is followed by a large positive component, often referred to as P2, occurring between 530 and 800 ms after stimulus onset. Absence of olfactory ERPs and presence (even with subtle changes) of somatosensory ERPs is a strong indicator of the presence of an olfactory dysfunction.Conclusions This review examines and discusses the methods of chemosensory stimulation as well as the electrophysiological correlates elicited by such stimuli. The clinical applications of chemosensory ERPs in neurology and otorhinolaryngology are outlined.},
  ISSN                     = {0987-7053},
  Keywords                 = {Smell
Olfaction
Chemosensory event-related potential
Stimulator
Olfactory and trigeminal information
Odorat
Potentiel évoqué chémosensitif
Olfactomètre
Informations olfactive et trigéminale},
  Type                     = {Journal Article},
  Url                      = {http://www.sciencedirect.com/science/article/B6VMP-4JS1XPD-1/2/6482d95b9f5a263b386b21dbb11a31b4}
}

@Article{Rotter2001,
  Title                    = {Age dependence of cellular properties of human septal cartilage: Implications for tissue engineering},
  Author                   = {N Rotter and LJ Bonassar and G Tobias and M Lebl and AK Roy and CA Vacanti},
  Journal                  = {Archives of Otolaryngology–Head \& Neck Surgery},
  Year                     = {2001},
  Number                   = {10},
  Pages                    = {1248-1252},
  Volume                   = {127},

  Doi                      = {10.1001/archotol.127.10.1248},
  Eprint                   = {/data/Journals/OTOL/11854/OOA90277.pdf},
  Url                      = { + http://dx.doi.org/10.1001/archotol.127.10.1248}
}

@Article{Rouaud2002,
  Title                    = {Computation of the airflow in a pilot scale clean room using K-ε turbulence models},
  Author                   = {Rouaud, Olivier and Havet, Michel},
  Journal                  = {International Journal of Refrigeration},
  Year                     = {2002},
  Number                   = {3},
  Pages                    = {351-361},
  Volume                   = {25},

  Doi                      = {10.1016/s0140-7007(01)00014-7},
  ISSN                     = {0140-7007},
  Keywords                 = {Clean room
Food
Design
Air distribution
Modelling
Computational fluid dynamics (CFD)
Salle blanche
Produit alimentaire
Distribution d'air
Modélisation
CFD (dynamique numérisée des fluides)},
  Type                     = {Journal Article},
  Url                      = {http://www.sciencedirect.com/science/article/pii/S0140700701000147}
}

@Article{Rouhiainen1970,
  Title                    = {ON THE DEPOSITION OF SMALL PARTICLES FROM TURBULENT STREAMS},
  Author                   = {Rouhiainen, P. O. and Stachiewicz, J. W.},
  Journal                  = {Journal of Heat Transfer, Transactions ASME},
  Year                     = {1970},
  Note                     = {Cited By (since 1996): 6
Export Date: 5 June 2011
Source: Scopus},
  Number                   = {1},
  Pages                    = {169-177},
  Volume                   = {92 Ser C},

  Type                     = {Journal Article},
  Url                      = {http://www.scopus.com/inward/record.url?eid=2-s2.0-0014735892&partnerID=40&md5=2a7b88d04900ae9e8e757f49c7c147fa}
}

@Article{Roy2003,
  Title                    = {Grid Convergence Error Analysis for Mixed-Order Numerical Schemes},
  Author                   = {Roy, C.},
  Journal                  = {AIAA},
  Year                     = {2003},
  Number                   = {4},
  Volume                   = {41},

  Type                     = {Journal Article}
}

@Article{Russo2008,
  Title                    = {Effects of cartilage rings on airflow and particle deposition in the trachea and main bronchi},
  Author                   = {Russo, J. and Robinson, R. and Oldham, Michael J.},
  Journal                  = {Medical Engineering \& Physics},
  Year                     = {2008},
  Note                     = {doi: DOI: 10.1016/j.medengphy.2007.06.010},
  Number                   = {5},
  Pages                    = {581-589},
  Volume                   = {30},

  ISSN                     = {1350-4533},
  Keywords                 = {Trachea
Main bronchi
Morphometry
Particle deposition
Laryngeal jet
Tracheobronchial airways},
  Type                     = {Journal Article},
  Url                      = {http://www.sciencedirect.com/science/article/B6T9K-4PGPM3C-1/2/f8dda391f21e6c99b63aaab3899a4a6b}
}

@Article{SA¸rensen2002,
  Title                    = {Modeling-gas phase reactions in indoor environments using computational fluid dynamics},
  Author                   = {Sørensen, Dan Nørtoft and Weschler, Charles J.},
  Journal                  = {Atmospheric Environment},
  Year                     = {2002},
  Number                   = {1},
  Pages                    = {9-18},
  Volume                   = {36},

  Abstract                 = {This CFD modeling study examines the concentrations of two gaseous compounds that react in an indoor setting to produce a hypothetical product. The reactants are ozone and either d-limonene or [alpha]-terpinene (which reacts with ozone about 40 times faster than d-limonene). In addition to two different terpenes, the scenarios include two air exchange rates (0.5 and 2.0 h-1). The terpene is introduced as a floor source with an emission pattern similar to a floor-care product. These four scenarios have been set in a fairly large two-dimensional room (13.6×40.6 m) with a supply at the top of the left wall and an exhaust at the bottom of the right wall. The room has been deliberately scaled so that the Reynolds numbers for key flow regimes match those of a room in which the calculated flow field has been validated against measured data. It has been further assumed that ozone interacts with room surfaces while the terpenes do not. The results show that for all four scenarios, under steady-state conditions, there are large concentration gradients within the room for both reactants and product. To some extent this is due to imperfect mixing. However, it also reflects that reactions occur at different rates across the room (because of varying reactant concentrations) and that the time available for reactions to occur varies with the room location (because the "age of the air" varies from point to point). Locally, within the room, the concentrations calculated by the CFD method differ significantly from those calculated by a one-compartment mass-balance model assuming perfect mixing.},
  ISSN                     = {1352-2310},
  Keywords                 = {Indoor chemistry
CFD
Incomplete mixing
Well-mixed assumption
Concentration gradients},
  Type                     = {Journal Article},
  Url                      = {http://www.sciencedirect.com/science/article/B6VH3-44MX50S-1/2/511d6cb4e877be37227e259b7ac47eed}
}

@Article{Sada2002,
  Title                    = {Numerical calculation of flow and stack-gas concentration fluctuation around a cubical building},
  Author                   = {Sada, K. and Sato, A.},
  Journal                  = {Atmospheric Environment},
  Year                     = {2002},
  Number                   = {35},
  Pages                    = {5527-5534},
  Volume                   = {36},

  Abstract                 = {A numerical simulation model was developed to predict the instantaneous concentration fluctuation of a plume and applied to stack-gas diffusion around a cubical building. The flow field, including an instantaneous velocity component, was predicted using the large eddy simulation (LES) method in the developed numerical model. Then, the instantaneous concentration fluctuation was predicted using the obtained unsteady flow field. Concentration was calculated using the finite difference method, in which the LES is expanded for concentration, and the puff method, in which small volumes of the tracer gas are divided and combined according to the calculation mesh sizes. In order to avoid numerical viscous effects, a puff method and finite difference method were applied separately in the regions near and far from the stack-gas release point, respectively. Then, the flow field around a cubical building and the diffusion of stack-gas, emitted from an elevated point source at an upstream position of the building, were calculated using the model mentioned above. Numerical calculation results were compared with those obtained in wind tunnel experiments in which concentration fluctuation was measured using high-response flame ionization detectors. Although there were some discrepancies in the flow field between the calculated results and those of wind tunnel experiments, e.g., the calculated windward length of a cavity region behind the building, the calculated mean velocity and turbulent intensity showed good agreement with those of the wind tunnel experiments. Furthermore, the calculated concentration fluctuation showed good agreement with that in the wind tunnel, not only regarding the features of fluctuating concentration signals, but also statistic quantities, viz., mean concentration, fluctuation intensity and high-concentration values.},
  ISSN                     = {1352-2310},
  Keywords                 = {Diffusion
Concentration fluctuation
Large eddy simulation
Puff method
Building
Wind tunnel},
  Type                     = {Journal Article},
  Url                      = {http://www.sciencedirect.com/science/article/B6VH3-472HWWF-G/2/475281a74ca03dbda637c7470fa9def6}
}

@Article{Saejiw2009,
  Title                    = {Exposure to Wood Dust and Its Particle Size Distribution in a Rubberwood Sawmill in Thailand},
  Author                   = {Saejiw, Nutjaree and Chaiear, Naesinee and Sadhra, Steven},
  Journal                  = {Journal of Occupational and Environmental Hygiene},
  Year                     = {2009},
  Number                   = {8},
  Pages                    = {483 - 490},
  Volume                   = {6},

  ISSN                     = {1545-9624},
  Type                     = {Journal Article},
  Url                      = {http://www.informaworld.com/10.1080/15459620902967065}
}

@Article{Safarik2002,
  Title                    = {Magnetic nanoparticles and biosciences},
  Author                   = {Safarik, I and Safarikova, M},
  Journal                  = {Mon Chem},
  Year                     = {2002},
  Pages                    = {737 - 759},
  Volume                   = {133},

  Type                     = {Journal Article}
}

@Article{Saffman1968,
  Title                    = {Corrigendum to the Lift on a Small Sphere in a Slow Shear Flow},
  Author                   = {Saffman, P.G. },
  Journal                  = {Journal of Fluid Mechanics},
  Year                     = {1968},
  Volume                   = {31},

  Type                     = {Journal Article}
}

@Article{Saffman1965,
  Title                    = {The lift on a small sphere in a slow shear flow},
  Author                   = {Saffman, P.G.},
  Journal                  = {Journal Fluid Mechanics},
  Year                     = {1965},
  Pages                    = {385-400},
  Volume                   = {22},

  Type                     = {Journal Article}
}

@Article{Saiyed2003,
  Title                    = {Application of magnetic techniques in the field of drug discovery and biomedicine},
  Author                   = {Saiyed, ZM and Telang, SD and Ramchand, CN},
  Journal                  = {BioMagnetic Research and Technology},
  Year                     = {2003},
  Number                   = {1},
  Pages                    = {2},
  Volume                   = {1},

  Abstract                 = {Magnetic separation technology, using magnetic particles, is quick and easy method for sensitive and reliable capture of specific proteins, genetic material and other biomolecules. The technique offers an advantage in terms of subjecting the analyte to very little mechanical stress compared to other methods. Secondly, these methods are non-laborious, cheap and often highly scalable. Moreover, techniques employing magnetism are more amenable to automation and miniaturization. Now that the human genome is sequenced and about 30,000 genes are annotated, the next step is to identify the function of these individual genes, carrying out genotyping studies for allelic variation and SNP analysis, ultimately leading to identification of novel drug targets. In this post-genomic era, technologies based on magnetic separation are becoming an integral part of todays biology laboratory. This article briefly reviews the selected applications of magnetic separation techniques in the field of biotechnology, biomedicine and drug discovery.},
  ISSN                     = {1477-044X},
  Type                     = {Journal Article},
  Url                      = {http://www.biomagres.com/content/1/1/2}
}

@Article{Saksena2007,
  Title                    = {Daily exposure to air pollutants in indoor outdoor and in-vehicle micro-environments: Pilot study in Delhi},
  Author                   = {Saksena, S. and Prasad, R.K. and Shankar, V.R.},
  Journal                  = {Indoor and Built Environment},
  Year                     = {2007},
  Number                   = {1},
  Pages                    = {39-46},
  Volume                   = {16},

  Type                     = {Journal Article}
}

@Article{Salem2004,
  Title                    = {Nasorespiratory function and craniofacial morphology--a review of the surgical management of the upper airway},
  Author                   = {Salem, Omar H. and Briss, Barry S. and Annino, Donald J.},
  Journal                  = {Seminars in Orthodontics},
  Year                     = {2004},
  Number                   = {1},
  Pages                    = {54-62},
  Volume                   = {10},

  ISSN                     = {1073-8746},
  Type                     = {Journal Article},
  Url                      = {http://www.sciencedirect.com/science/article/B75KK-4BYK123-6/2/fb35a30753062c898965f2b40b417558}
}

@Article{Salzano2010,
  Title                    = {Nasal tactile sensitivity in elderly.},
  Author                   = {Salzano, FA and Guastini, L and Mora, R and Dellepiane, M and Salzano, G and Santomauro, V and Salami, A},
  Journal                  = {Acta Oto-Laryngologica},
  Year                     = {2010},
  Number                   = {12},
  Pages                    = {1389 - 1393},
  Volume                   = {130},

  Abstract                 = {Conclusion: Although older people varied widely in tactile sensitivity, our results show that tactile thresholds increased with age. Objectives: The aim of this study was to evaluate the effects of aging on nasal tactile sensitivity. Methods: A total of 160 healthy patients aged between 50 and 90 years were included. According to their age, patients were divided into groups (A, B, C, D, E, F, G, and H). From the age of 50, each group included subjects with an age range of 5 years (i.e. group A, 50-55 years; group B, 56-60 years, etc.). Each patient's outcome was assessed through the nasal monofilament test: a set of 20 Semmes-Weinstein monofilaments was used to detect nasal sensitivity for both nasal cavities. The sensitivity threshold was recorded as the minimum monofilament size from which patients could detect at least two of three stimuli. Results: In groups D (66-70 years), E (71-75 years), F (76-80 years), G (81-85 years), and H (86-90 years) a significantly ( p < 0.05) higher s},
  ISSN                     = {0001-6489},
  Keywords                 = {Aging, Sensation Disorders, Nose -- Physiopathology, Human, Aged, Middle Age, Aged, 80 and Over, Male, Female, Outcome Assessment, Unpaired T-Tests, Chi Square Test, Descriptive Statistics},
  Url                      = {http://search.ebscohost.com.ezproxy.lib.rmit.edu.au/login.aspx?direct=true&db=c8h&AN=2010857445&site=ehost-live&scope=site}
}

@Article{Samolinski2007,
  Title                    = {Changes in Nasal Cavity Dimensions in Children and Adults by Gender and Age},
  Author                   = {Samoliński, Bolesław K. and Grzanka, Antoni and Gotlib, Tomasz},
  Journal                  = {The Laryngoscope},
  Year                     = {2007},
  Number                   = {8},
  Pages                    = {1429--1433},
  Volume                   = {117},

  Doi                      = {10.1097/MLG.0b013e318064e837},
  ISSN                     = {1531-4995},
  Keywords                 = {nose growth, intranasal spaces, nasal parameters, acoustic rhinometry},
  Publisher                = {John Wiley \& Sons, Inc.},
  Url                      = {http://dx.doi.org/10.1097/MLG.0b013e318064e837}
}

@Article{Sankar2008,
  Title                    = {Molecular dynamics modeling of thermal conductivity enhancement in metal nanoparticle suspensions},
  Author                   = {Sankar, N. and Mathew, Nithin and Sobhan, C. B.},
  Journal                  = {International Communications in Heat and Mass Transfer},
  Year                     = {2008},
  Number                   = {7},
  Pages                    = {867-872},
  Volume                   = {35},

  Abstract                 = {A theoretical approach based on molecular dynamics modeling, for the estimation of the enhancement of the thermal conductivity of liquids by the introduction of suspended metallic nanoparticles is proposed. Algorithms are developed for simulating the nanofluid abiding the procedural steps of the molecular dynamics method. The method is presented as a solution to the generic problem of thermal conductivity enhancement of liquids in the presence of nanoparticles, and illustrated using a specific simulation procedure with properties representing water and platinum nanoparticles. The thermal conductivity enhancement estimated using the simulations are compared with existing experimental results and those predicted by conventional effective medium theories. Parametric studies are conducted to obtain the variation of thermal conductivity enhancement with the temperature and the volume fraction of the nanoparticles in the suspension.},
  ISSN                     = {0735-1933},
  Keywords                 = {Molecular dynamics
Nanofluids
Thermal conductivity},
  Type                     = {Journal Article},
  Url                      = {http://www.sciencedirect.com/science/article/B6V3J-4SBYH76-3/2/a57bb40d249280630b2840e4de4af905}
}

@Article{Sarangapani2000,
  Title                    = {Modeling Particle Deposition in Extrathoracic Airways},
  Author                   = {Sarangapani, Ramesh},
  Journal                  = {Aerosol Science and Technology},
  Year                     = {2000},
  Number                   = {1},
  Pages                    = {72-89},
  Volume                   = {32},

  ISSN                     = {0278-6826},
  Type                     = {Journal Article},
  Url                      = {http://www.informaworld.com/10.1080/027868200303948}
}

@Article{Sauret2002,
  Title                    = {Study of the three-dimensional geometry of the central conducting airways in man using computed tomographic (CT) images},
  Author                   = {Sauret, V. and Halson, P.M. and Brown, I.W. and Fleming, J.S. and Bailey, A.G.},
  Journal                  = {Journal of Anatomy},
  Year                     = {2002},
  Pages                    = {123-134},
  Volume                   = {200},

  Type                     = {Journal Article}
}

@Article{Sbirlea-Apiou2004,
  Title                    = {Simulation of the regional manifestation of asthma},
  Author                   = {Sbirlea-Apiou, G. and Lemaire, M. and Katz, I. and Conway, J. and Fleming, J.S. and Martonen, T.B.},
  Journal                  = {Journal of Pharmaceutical Sciences},
  Year                     = {2004},
  Number                   = {5},
  Pages                    = {1205-1216},
  Volume                   = {93},

  Type                     = {Journal Article}
}

@Article{SchA¤lin2004,
  Title                    = {Impact of turbulence anisotropy near walls in room airflow},
  Author                   = {Schälin, A. and Nielsen, P. V.},
  Journal                  = {Indoor Air},
  Year                     = {2004},
  Number                   = {3},
  Pages                    = {159-168},
  Volume                   = {14},

  Doi                      = {10.1111/j.1600-0668.2004.00201e.x},
  ISSN                     = {1600-0668},
  Keywords                 = {Computational fluid dynamics
Turbulence model
k–ɛ model
Reynolds stress model
Room air distribution
Three-dimensional wall jet
Growth rate in jet},
  Type                     = {Journal Article},
  Url                      = {http://dx.doi.org/10.1111/j.1600-0668.2004.00201e.x}
}

@Article{Schamberger1990,
  Title                    = {Collection of prolate spheroidal aerosol particles by charged spherical collectors},
  Author                   = {Schamberger, M. R. and Peters, J. E. and Leong, K. H.},
  Journal                  = {Journal of Aerosol Science},
  Year                     = {1990},
  Number                   = {4},
  Pages                    = {539-554},
  Volume                   = {21},

  ISSN                     = {0021-8502},
  Type                     = {Journal Article},
  Url                      = {http://www.sciencedirect.com/science/article/pii/002185029090130P}
}

@Article{Scherer1989,
  Title                    = {The biophysics of nasal airflow},
  Author                   = {Scherer, P.W. and Hahn, I.I. and Mozell, M.M.},
  Journal                  = {Otolaryngol. Clin. North Am},
  Year                     = {1989},
  Pages                    = {265-278},
  Volume                   = {22},

  Type                     = {Journal Article}
}

@Article{Scherer1994,
  Title                    = {Nasal Dosimetry Modeling for Humans, Inhalation Toxicology},
  Author                   = {Scherer, P.W. and Keyhani, K. and M.M., Mozell},
  Journal                  = {Inhalation Toxicology},
  Year                     = {1994},
  Pages                    = {85-97},
  Volume                   = {6},

  Type                     = {Journal Article}
}

@Article{Scherer1994a,
  Title                    = {Nasal Dosimetry Modeling for Humans},
  Author                   = {Scherer, P.W. and Keyhani, K. and Mozell, M.M.},
  Journal                  = { Inhalation Toxicology},
  Year                     = {1994},
  Pages                    = {85-97},
  Volume                   = {6},

  Type                     = {Journal Article}
}

@Article{Schiller1935,
  Title                    = {A Drag Coefficient Correlation},
  Author                   = {Schiller, L. and Naumann, Z.},
  Journal                  = {Z. Ver. Deutsch. Ing},
  Year                     = {1935},
  Number                   = {318},
  Volume                   = {77},

  Type                     = {Journal Article}
}

@Article{SchlA¼nssen2001,
  Title                    = {Wood dust exposure in the Danish furniture industry using conventional and passive monitors},
  Author                   = {Schlünssen, Vivi and Vinzents, Peter S. and Mikkelsen, Anders B. and Schaumburg, Inger},
  Journal                  = {Annals of Occupational Hygiene},
  Year                     = {2001},
  Number                   = {2},
  Pages                    = {157-164},
  Volume                   = {45},

  Abstract                 = {A study of wood dust exposure at furniture factories in one county in Denmark was performed as a cross sectional study. Dust exposure was measured with personal passive dust monitors and calibrated against active sampling on filters. Measurements of 1685 workers were included in the exposure assessment. The passive dust monitor conversion models for equivalent concentrations of inhalable dust and total dust based on data from the present study were not significantly different from the original models. Therefore models based on all available data were used. The parameters of the distribution of equivalent concentration of inhalable dust were 0.94 mg/m3 (geometric mean) and 2.10 (geometric standard deviation). Compared with a national cross sectional study from 1988 the exposure level (geometric mean) was reduced by a factor 2.0. Inhalable dust exposure was about 50% higher than exposure measured by the Danish ‘total’ dust method.},
  Doi                      = {10.1093/annhyg/45.2.157},
  Type                     = {Journal Article},
  Url                      = {http://annhyg.oxfordjournals.org/content/45/2/157.abstract}
}

@Book{Schleimer2001,
  Title                    = {Inhaled Steroids in Asthma},
  Author                   = {Schleimer, R.P. and O’Byrne, P.M. and Szefler, S.},
  Publisher                = {Marcel Dekker},
  Year                     = {2001},

  Address                  = {New York},
  Series                   = {Lung Biology in Health and Disease},
  Volume                   = {163},

  Type                     = {Edited Book}
}

@Article{Schlesinger1982,
  Title                    = {Particle deposition within bronchial airways: comparisons using constant and cyclic inspiratory flows},
  Author                   = {Schlesinger, RICHARD B. and Gurman, JOSHUA L. and Lippmann, MORTON},
  Journal                  = {Ann Occup Hyg},
  Year                     = {1982},
  Number                   = {1},
  Pages                    = {47-64},
  Volume                   = {26},

  Abstract                 = {Particle deposition patterns were measured in the first five branching generations of replicate hollow casts made from a solid cast of a human tracheobronchial tree. The casts were exposed to radioactively tagged aerosols, having MMADs of approx. 3 and 8 {micro}m, under a series of constant and cyclic inspiratory flow patterns having equivalent mean flow rates. Deposition efficiencies at bifurcation and length subregions of each generation were greater under cyclic flow and these differences were greater for the smaller sized aerosol.},
  Doi                      = {10.1093/annhyg/26.1.47},
  Type                     = {Journal Article},
  Url                      = {http://annhyg.oxfordjournals.org/cgi/content/abstract/26/1/47}
}

@Book{Schlichting1979,
  Title                    = {Boundary-Layer Theory},
  Author                   = {Schlichting, H.},
  Publisher                = { McGraw-Hill},
  Year                     = {1979},

  Address                  = {New York},

  Type                     = {Book}
}

@Misc{Schmehl2000,
  Title                    = {CFD Analysis of Fuel Atomization, Secondary Droplet Breakup and Spray Dispersion in the Premix Duct of a LPP Combustor},

  Author                   = {Schmehl, R. and Maier, G. and Wittig, S.},
  Year                     = {2000},

  Type                     = {Conference Paper}
}

@Article{Schmehl1999,
  Title                    = {CFD analysis of spray propagation and evaporation including wall film formation and spray/film interactions},
  Author                   = {Schmehl, R. and Rosskamp, H. and Willmann, M. and Wittig, S.},
  Journal                  = {International Journal of Heat and Fluid Flow},
  Year                     = {1999},
  Note                     = {doi: DOI: 10.1016/S0142-727X(99)00041-7},
  Number                   = {5},
  Pages                    = {520-529},
  Volume                   = {20},

  ISSN                     = {0142-727X},
  Keywords                 = {Sprays
Evaporation
CFD
Films
Droplet-trajectories},
  Type                     = {Journal Article},
  Url                      = {http://www.sciencedirect.com/science/article/B6V3G-3X9JK1K-C/2/8955a3e1589f178480dd75083a13a0a3}
}

@Article{Schmid2003,
  Title                    = {On the modelling of the particle dynamics in electro-hydrodynamic flow-fields: I. Comparison of Eulerian and Lagrangian modelling approach},
  Author                   = {Schmid, Hans-Joachim and Vogel, Lutz},
  Journal                  = {Powder Technology},
  Year                     = {2003},
  Note                     = {doi: DOI: 10.1016/j.powtec.2003.08.009},
  Pages                    = {118-135},
  Volume                   = {135-136},

  ISSN                     = {0032-5910},
  Keywords                 = {Corona discharge
Electro-hydrodynamics (EHD)
Euler
Lagrange
Modelling
Particle charging},
  Type                     = {Journal Article},
  Url                      = {http://www.sciencedirect.com/science/article/B6TH9-49WMBS6-6/2/9f33b62c726f8b94dae46d586ab66062}
}

@Article{Schmidt1999,
  Title                    = {Pressure-Swirl Atomization in the Near Field},
  Author                   = {Schmidt, D.P. and Nouar, I. and Senecal, P.K. and Rutland, C.J. and Martin, J.K. and Reitz, R.D. and Hoffman, J.A.},
  Journal                  = {Transactions of SAE},
  Year                     = {1999},
  Pages                    = {1999-01-0496},
  Volume                   = {1},

  Type                     = {Journal Article}
}

@Article{Schreck1993,
  Title                    = {Correlations between flow resistance and geometry in a model of the human nose},
  Author                   = {Schreck, S. and Sullivan, K.J. and Ho, C.M. and H.K., Chang},
  Journal                  = {Journal Appl.Physiol},
  Year                     = {1993},
  Number                   = {4},
  Pages                    = {1767-1775},
  Volume                   = {75},

  Type                     = {Journal Article}
}

@Article{Schreck1993a,
  Title                    = {Correlations between flow resistance and geometry in a model of the human nose},
  Author                   = {Schreck, S. and Sullivan, K. J. and Ho, C. M. and Chang, H. K.},
  Journal                  = {Journal of Applied Physiology},
  Year                     = {1993},
  Note                     = {Cited By (since 1996): 50
Export Date: 6 June 2011
Source: Scopus},
  Number                   = {4},
  Pages                    = {1767-1775},
  Volume                   = {75},

  Type                     = {Journal Article},
  Url                      = {http://www.scopus.com/inward/record.url?eid=2-s2.0-0027421603&partnerID=40&md5=0cf5b9739b316a34d5079c8834fba92f}
}

@Article{Schroeter2006,
  Title                    = {Analysis of particle deposition in the turbinate and olfactory regions using a human nasal computational fluid dynamics model},
  Author                   = {Schroeter, J.D. and Kimbell, J.S. and Asgharian, B.},
  Journal                  = {Journal of Aerosol Medicine},
  Year                     = {2006},
  Number                   = {3},
  Pages                    = {301-313},
  Volume                   = {19},

  Type                     = {Journal Article}
}

@Article{Schroeter2011,
  Title                    = {Effects of surface smoothness on inertial particle deposition in human nasal models},
  Author                   = {Schroeter, Jeffry D. and Garcia, Guilherme J. M. and Kimbell, Julia S.},
  Journal                  = {Journal of Aerosol Science},
  Year                     = {2011},
  Number                   = {1},
  Pages                    = {52-63},
  Volume                   = {42},

  Doi                      = {DOI: 10.1016/j.jaerosci.2010.11.002},
  ISSN                     = {0021-8502},
  Keywords                 = {Nasal airway
Particle deposition
Wall roughness
Nasal deposition
Computational fluid dynamics},
  Type                     = {Journal Article},
  Url                      = {http://www.sciencedirect.com/science/article/B6V6B-51M0NFD-1/2/4fe969fff47505f8d54bcd8dae1ee0ff}
}

@Article{Schroeter2010,
  Title                    = {A computational fluid dynamics approach to assess interhuman variability in hydrogen sulfide nasal dosimetry},
  Author                   = {Schroeter, Jeffry D. and Garcia, Guilherme J. M. and Kimbell, Julia S.},
  Journal                  = {Inhalation Toxicology},
  Year                     = {2010},
  Number                   = {4},
  Pages                    = {277-286},
  Volume                   = {22},

  Doi                      = {doi:10.3109/08958370903278077},
  Type                     = {Journal Article},
  Url                      = {http://informahealthcare.com/doi/abs/10.3109/08958370903278077}
}

@Article{Schroeter2008,
  Title                    = {Application of Physiological Computational Fluid Dynamics Models to Predict Interspecies Nasal Dosimetry of Inhaled Acrolein},
  Author                   = {Schroeter, Jeffry D. and Kimbell, Julia S. and Gross, Elizabeth A. and Willson, Gabrielle A. and Dorman, David C. and Tan, Yu-Mei and Clewell, Harvey J.},
  Journal                  = {Inhalation Toxicology},
  Year                     = {2008},
  Number                   = {3},
  Pages                    = {227-243},
  Volume                   = {20},

  ISSN                     = {0895-8378},
  Type                     = {Journal Article},
  Url                      = {http://www.informaworld.com/10.1080/08958370701864235}
}

@Article{Schroter1969,
  Title                    = {Flow patterns in models of the human bronchial airways},
  Author                   = {Schroter, R.C. and Sudlow, M.F.},
  Journal                  = {Respiration Physiology},
  Year                     = {1969},
  Number                   = {341-355},
  Volume                   = {7},

  Type                     = {Journal Article}
}

@Article{Segal2008,
  Title                    = {Effects of Differences in Nasal Anatomy on Airflow Distribution: A Comparison of Four Individuals at Rest},
  Author                   = {Segal, R.A. and Kepler, G.M. and Kimbell, J.},
  Journal                  = {Annals of Biomedical Engineering},
  Year                     = {2008},
  Number                   = {11},
  Pages                    = {1870-1882},
  Volume                   = {36},

  Type                     = {Journal Article}
}

@Article{Sehmel1973,
  Title                    = {Particle eddy diffusivities and deposition velocities for isothermal flow and smooth surfaces},
  Author                   = {Sehmel, G. A.},
  Journal                  = {Journal of Aerosol Science},
  Year                     = {1973},
  Number                   = {2},
  Pages                    = {125-138},
  Volume                   = {4},

  ISSN                     = {0021-8502},
  Type                     = {Journal Article},
  Url                      = {http://www.sciencedirect.com/science/article/pii/0021850273900645}
}

@Book{Seinfeld1986,
  Title                    = {Atmospheric Chemistry and Physics of Air Pollution},
  Author                   = {Seinfeld, J.H. },
  Publisher                = {John Wiley and Sons},
  Year                     = {1986},

  Address                  = {New York},

  Type                     = {Book}
}

@Article{Sekhar2004,
  Title                    = {Impact of airflow profile on indoor air quality-A tropical study},
  Author                   = {Sekhar, S.C. and Willem, H.C.},
  Journal                  = {Building and Environment},
  Year                     = {2004},
  Pages                    = {255-266},
  Volume                   = {39},

  Type                     = {Journal Article}
}

@Article{Sellier,
  Title                    = {Migration of a solid particle in the vicinity of a plane fluid-fluid interface},
  Author                   = {Sellier, A. and Pasol, L.},
  Journal                  = {European Journal of Mechanics - B/Fluids},
  Number                   = {1},
  Pages                    = {76-88},
  Volume                   = {30},

  Abstract                 = {The slow migration of a small and solid particle in the vicinity of a gas-liquid, fluid-fluid or solid-fluid plane boundary when subject to a gravity or an external flow field is addressed. By contrast with previous works, the advocated approach holds for arbitrarily shaped particles and arbitrary external Stokes flow fields complying with the conditions on the boundary. It appeals to a few theoretically established and numerically solved boundary-integral equations on the particle's surface. This integral formulation of the problem allows us to provide asymptotic approximations for a distant boundary and also, implementing a boundary element technique, accurate numerical results for arbitrary locations of the boundary. The results obtained for spheroids, both settling or immersed in external pure shear and straining flows, reveal that the rigid-body motion experienced by a particle deeply depends upon its shape and also upon the boundary location and properties.},
  Doi                      = {10.1016/j.euromechflu.2010.09.006},
  ISSN                     = {0997-7546},
  Keywords                 = {Solid particle
Stokes flow
Fluid-fluid interface
Boundary-integral method},
  Type                     = {Journal Article},
  Url                      = {http://www.sciencedirect.com/science/article/pii/S099775461000097X}
}

@Article{Senecal1999,
  Title                    = {Modeling high-speed viscous liquid sheet atomization},
  Author                   = {Senecal, P. K. and Schmidt, D. P. and Nouar, I. and Rutland, C. J. and Reitz, R. D. and Corradini, M. L.},
  Journal                  = {International Journal of Multiphase Flow},
  Year                     = {1999},
  Number                   = {6-7},
  Pages                    = {1073-1097},
  Volume                   = {25},

  Doi                      = {10.1016/s0301-9322(99)00057-9},
  ISSN                     = {0301-9322},
  Keywords                 = {Sheet atomization
Hydrodynamic instability
Drop distortion
Breakup
Multi-dimensional modeling},
  Type                     = {Journal Article},
  Url                      = {http://www.sciencedirect.com/science/article/pii/S0301932299000579}
}

@Article{Sera2003,
  Title                    = {Three-dimensional visualization and morphometry of small airways from microfocal X-ray computed tomography},
  Author                   = {Sera, Toshihiro and Fujioka, Hideki and Yokota, Hideo and Makinouchi, Akitake and Himeno, Ryutaro and Schroter, Robert C. and Tanishita, Kazuo},
  Journal                  = {Journal of Biomechanics},
  Year                     = {2003},
  Note                     = {doi: DOI: 10.1016/S0021-9290(03)00179-9},
  Number                   = {11},
  Pages                    = {1587-1594},
  Volume                   = {36},

  ISSN                     = {0021-9290},
  Keywords                 = {Cone-beam microfocal X-ray CT
Three-dimensional thinning algorithm
Sodium diatrizoate
Soft tissue},
  Type                     = {Journal Article},
  Url                      = {http://www.sciencedirect.com/science/article/B6T82-48TKD4H-1/2/532412d9f9be14f999a4996a84181678}
}

@Article{Seren2009,
  Title                    = {Morphological adaptation of the nasal valve area to climate},
  Author                   = {Seren, E. and Seren, S.},
  Journal                  = {Medical Hypotheses},
  Year                     = {2009},
  Pages                    = {471-472},
  Volume                   = {72},

  Type                     = {Journal Article}
}

@Article{Seren2009a,
  Title                    = {Morphological adaptation of the nasal valve area to climate},
  Author                   = {Seren, Erdal and Seren, Sule},
  Journal                  = {MEDICAL HYPOTHESES},
  Year                     = {2009},
  Pages                    = {471-472},
  Volume                   = {72},

  Abstract                 = {Ecogeographic variation in nasal valve angles stands as one of the best examples of human morphological adaptation to climate. A major physiological function of the nasal cavity is to condition the inhaled air to body core temperature and saturated with vapour to prevent damage to the alveolar epithelium in the lungs. The air conditioning capability of the nose is dependent on the nasal mucosal temperature and the airflow dynamics caused by the airway geometry. Morphological variation of the human nose has been attributed to the ecogeographic adaptation to climate where nasal cavities have been broadly categorised as tall and narrow (leptorrhines) or short and broad (platyrrhines) according to their morphology. We believe that there is a relationship between nasal valve angles and climate adaptation. Crown Copyright (C) 2008 Published by Elsevier Ltd. All rights reserved.},
  Doi                      = {10.1016/j.mehy.2008.11.028},
  ISSN                     = {0306-9877},
  Type                     = {Journal Article},
  Url                      = {http://ac.els-cdn.com/S0306987708006154/1-s2.0-S0306987708006154-main.pdf?_tid=07fb956a-4220-11e4-9cc4-00000aab0f26&acdnat=1411366786_2d4a909e122e515400bf7be0e4c8529c}
}

@Article{Sergent2000,
  Title                    = {Extension du modèle d'échelles mixtes à la diffusivité de sous-maille},
  Author                   = {Sergent, Anne and Joubert, Patrice and Le Quéré, Patrick and Tenaud, Christian},
  Journal                  = {Comptes Rendus de l'Académie des Sciences - Series IIB - Mechanics},
  Year                     = {2000},
  Number                   = {12},
  Pages                    = {891-897},
  Volume                   = {328},

  Abstract                 = {Résumé Dans le cadre de la simulation des grandes échelles en écoulements turbulents anisothermes, le modèle d'échelles mixtes est étendu à la diffusivité de sous-maille, afin d'évaluer indépendamment viscosité et diffusivité. L'identification du terme de dissipation thermique dans l'équation de conservation du flux de sous-maille permet d'obtenir une expression algébrique de la diffusivité, produit d'un modèle de type Smagorinsky et d'un modèle basé sur le flux de sous-maille. Appliqué à un cas de convection naturelle turbulente, ce modèle conduit à une amélioration sensible de la représentation du champ thermique, relativement à une analogie de Reynolds à nombre de Prandtl de sous-maille constant.},
  ISSN                     = {1620-7742},
  Keywords                 = {turbulence
simulation des grandes échelles
modèle de sous-maille
large eddy simulation
subgrid model},
  Type                     = {Journal Article},
  Url                      = {http://www.sciencedirect.com/science/article/B6W82-42KD3MG-9/2/359fd5d00957ffb990709cc488d6188b}
}

@Article{Settipane2011,
  Title                    = {{Other Causes of Rhinitis: Mixed Rhinitis, Rhinitis Medicamentosa,
 Hormonal Rhinitis, Rhinitis of the Elderly, and Gustatory Rhinitis}},
  Author                   = {Settipane, Russell A.},
  Journal                  = {{IMMUNOLOGY AND ALLERGY CLINICS OF NORTH AMERICA}},
  Year                     = {{2011}},

  Month                    = {{AUG}},
  Number                   = {{3}},
  Pages                    = {{457+}},
  Volume                   = {{31}},

  Abstract                 = {{It is important to consider a comprehensive differential of possible
 rhinitis types when considering the diagnosis of chronic rhinitis,
 including at least 9 subtypes of nonallergic rhinitis: drug-induced
 rhinitis, gustatory rhinitis, hormonal-induced rhinitis, infectious
 rhinitis, nonallergic rhinitis with eosinophilia syndrome, occupational
 rhinitis, senile rhinitis, atrophic rhinitis, and nonallergic
 rhinopathy. This article focuses on some of the most common types of
 chronic rhinitis, including mixed rhinitis (allergic and nonallergic
 overlap), rhinitis medicamentosa, hormonal rhinitis, rhinitis of the
 elderly, and gustatory rhinitis.}},
  Address                  = {{1600 JOHN F KENNEDY BOULEVARD, STE 1800, PHILADELPHIA, PA 19103-2899 USA}},
  Affiliation              = {{Settipane, RA (Reprint Author), Brown Univ, Alpert Med Sch, Providence, RI 02912 USA.
 Brown Univ, Alpert Med Sch, Providence, RI 02912 USA.}},
  Author-email             = {{setti5@aol.com}},
  Doc-delivery-number      = {{799SP}},
  Doi                      = {{10.1016/j.iac.2011.05.011}},
  Funding-acknowledgement  = {{Merck; Alcon; Meda; Teva; Sunovion; Astra-Zeneca}},
  Funding-text             = {{Dr Settipane is a speaker, and/or research grant recipient for Merck,
 Alcon, Meda, Teva, Sunovion, and Astra-Zeneca.}},
  ISSN                     = {{0889-8561}},
  Journal-iso              = {{Immunol. Allerg. Clin. North Am.}},
  Keywords                 = {{Rhinitis; Rhinitis medicamentosa; Hormonal rhinitis; Gustatory rhinitis;
 Mixed rhinitis; Geriatric}},
  Keywords-plus            = {{PERENNIAL ALLERGIC RHINITIS; SKIN-TEST REACTIVITY; FUROATE NASAL SPRAY;
 SERUM-SPECIFIC IGE; NONALLERGIC RHINITIS; FLUTICASONE PROPIONATE;
 DIFFERENT ROUTES; AIR PROVOCATION; PRICK TEST; ANTIHISTAMINES}},
  Language                 = {{English}},
  Number-of-cited-references = {{84}},
  Publisher                = {{W B SAUNDERS CO-ELSEVIER INC}},
  Research-areas           = {{Allergy; Immunology}},
  Times-cited              = {{7}},
  Type                     = {{Article}},
  Unique-id                = {{ISI:000293306700005}},
  Web-of-science-categories = {{Allergy; Immunology}}
}

@Article{Severac2007,
  Title                    = {A spectral vanishing viscosity for the LES of turbulent flows within rotating cavities},
  Author                   = {Severac, E. and Serre, E.},
  Journal                  = {Journal of Computational Physics},
  Year                     = {2007},
  Number                   = {2},
  Pages                    = {1234-1255},
  Volume                   = {226},

  Abstract                 = {A spectral vanishing viscosity technique (SVV) is presented for the simulation of 3D turbulent incompressible flows within a rotor-stator cavity. One characteristic of this technique is that the SVV is active only for the short length scales, a feature which is reminiscent of Large Eddy Simulation models. The Spectral Vanishing Viscosity, first introduced by E. Tadmor for the inviscid Burgers equation [E. Tadmor, Convergence of spectral methods for nonlinear conservation laws, SIAM J. Numer. Anal. 26 (1) (1989) 30], is incorporated into the cylindrical Navier-Stokes equations written in velocity pressure formulation. The second-order operator involved in the SVV-method is implemented in a Chebyshev-collocation Fourier-Galerkin pseudo-spectral code. The SVV is shown to lead to stable discretizations without sacrificing the formal accuracy, i.e., exponential convergence, in the proposed discretization. LES results are presented here for rotational Reynolds numbers ranging from Re=7×104 to Re=7×105. Turbulent quantities are shown to compare very favorably with results of direct numerical simulation (DNS) and experimental measurements.},
  ISSN                     = {0021-9991},
  Keywords                 = {Spectral vanishing viscosity
Large eddy simulation
Spectral method
Rotating flows},
  Type                     = {Journal Article},
  Url                      = {http://www.sciencedirect.com/science/article/B6WHY-4NXHCJ1-6/2/38b913b9ead6ecf38fde118e6ae8bb92}
}

@Article{Shaida2000,
  Title                    = {The nasal valves: changes in anatomy and physiology in normal subjects},
  Author                   = {Shaida, A.M. and Kenyon, G.S.},
  Journal                  = {Rhinology},
  Year                     = {2000},
  Pages                    = {7-12},
  Volume                   = {38},

  Type                     = {Journal Article}
}

@Article{Shalaby2008,
  Title                    = {Numerical calculation of particle-laden cyclone separator flow using LES},
  Author                   = {Shalaby, H. and Wozniak, K. and Wozniak, G.},
  Journal                  = {Engineering Applications of Computational Fluid Mechanics},
  Year                     = {2008},
  Number                   = {4},
  Pages                    = {382-392},
  Volume                   = {2},

  Type                     = {Journal Article}
}

@Article{Shanley2011,
  Title                    = {A Numerical Model for Simulating the Motions of Ellipsoidal Fibers Suspended in Low Reynolds Number Shear Flows},
  Author                   = {Shanley, Kevin T. and Ahmadi, Goodarz},
  Journal                  = {Aerosol Science and Technology},
  Year                     = {2011},
  Number                   = {7},
  Pages                    = {838-848},
  Volume                   = {45},

  Doi                      = {10.1080/02786826.2011.566293},
  ISSN                     = {0278-6826},
  Type                     = {Journal Article},
  Url                      = {http://dx.doi.org/10.1080/02786826.2011.566293}
}

@Article{Shanley2008,
  Title                    = {Numerical Simulations Investigating the Regional and Overall Deposition Efficiency of the Human Nasal Cavity},
  Author                   = {Shanley, Kevin T. and Zamankhan, Parsa and Ahmadi, Goodarz and Hopke, Philip K. and Cheng, Yung-Sung},
  Journal                  = {Inhalation Toxicology},
  Year                     = {2008},
  Number                   = {12},
  Pages                    = {1093 - 1100},
  Volume                   = {20},

  ISSN                     = {0895-8378},
  Type                     = {Journal Article},
  Url                      = {http://www.informaworld.com/10.1080/08958370802130379}
}

@Article{Shapiro1993,
  Title                    = {Deposition of glass fiber particles from turbulent air flow in a pipe},
  Author                   = {Shapiro, Michael and Goldenberg, Moshe},
  Journal                  = {Journal of Aerosol Science},
  Year                     = {1993},
  Number                   = {1},
  Pages                    = {65-87},
  Volume                   = {24},

  ISSN                     = {0021-8502},
  Type                     = {Journal Article},
  Url                      = {http://www.sciencedirect.com/science/article/pii/002185029390085N}
}

@Article{Shepard1992,
  Title                    = {Hypertension, cardiac arrhythmias, myocardial infarction, and stroke in relation to obstructive sleep apnea.},
  Author                   = {Shepard, J.W.},
  Journal                  = {Clin Chest Med},
  Year                     = {1992},
  Number                   = {3},
  Pages                    = {437-458},
  Volume                   = {13},

  Type                     = {Journal Article}
}

@Article{Shephard1991,
  Title                    = {Automatic three-dimensional mesh generation by the finite octree technique},
  Author                   = {Shephard, Mark S. and Georges, Marcel K.},
  Journal                  = {International Journal for Numerical Methods in Engineering},
  Year                     = {1991},
  Number                   = {4},
  Pages                    = {709-749},
  Volume                   = {32},

  Doi                      = {10.1002/nme.1620320406},
  ISSN                     = {1097-0207},
  Type                     = {Journal Article},
  Url                      = {http://dx.doi.org/10.1002/nme.1620320406}
}

@Article{Shi2007,
  Title                    = {Simulation and Analysis of High-Speed Droplet Spray Dynamics},
  Author                   = {Shi, H. and Kleinstreuer, C.},
  Journal                  = {Journal of Fluids Engineering},
  Year                     = {2007},
  Number                   = {5},
  Pages                    = {621-633},
  Volume                   = {129},

  Keywords                 = {drops
sprays
two-phase flow
flow simulation
jets
stratified flow
turbulent diffusion
drag
nozzles
confined flow
slip flow},
  Type                     = {Journal Article},
  Url                      = {http://link.aip.org/link/?JFG/129/621/1}
}

@Article{Shi2008,
  Title                    = {Dilute suspension flow with nanoparticle deposition in a representative nasal airway model},
  Author                   = {Shi, H. and Kleinstreuer, C. and Zhang, Z.},
  Journal                  = {Physics of Fluids},
  Year                     = {2008},
  Number                   = {1},
  Pages                    = {1-23},
  Volume                   = {20},

  Type                     = {Journal Article}
}

@Article{Shi2007a,
  Title                    = {Modeling of inertial particle transport and deposition in human nasal cavities with wall roughness},
  Author                   = {Shi, Huawei and Kleinstreuer, Clement and Zhang, Zhe},
  Journal                  = {Journal of Aerosol Science},
  Year                     = {2007},
  Number                   = {4},
  Pages                    = {398-419},
  Volume                   = {38},

  Abstract                 = {Nasal inhalation helps to protect the lungs from detrimental effects of toxic particles which, however, may also place the nasal and adjacent tissues at risk. Alternatively, drug-aerosol deposition on pre-determined nasal airway surfaces can be a modern pathway for rapid medical treatment. The present study focuses on inertial particles in the range of , subject to steady laminar flow rates of 7.5 and 20 L/min. In contrast to ultrafine particles, for certain fine particle sizes deposition is strongly affected by wall roughness, which was incorporated with a selective micro-size airway-surface layer. The validated computer simulation results show that the inertial particle deposition in human nasal cavities increases with increasing impaction parameter, . Most of the deposition occurs in the anterior part of the human nasal cavities, especially in the nasal valve region. Considering drug-aerosol targeting, an optimal impaction parameter value exists which generates for normal inlet conditions the largest deposition in desired areas, e.g., the middle meatus, inferior meatus and olfactory regions. However, the absolute deposition efficiencies, especially in the inferior meatus and olfactory region, are very small because particles hardly reach those regions due to the complex nasal geometric structures. The influence of gravity was also analyzed and an experimentally validated correlation for inertial particle deposition in human nasal cavities has been provided.},
  ISSN                     = {0021-8502},
  Keywords                 = {Human nasal cavity
Inertial particle decomposition
Computer analysis
Wall roughness},
  Type                     = {Journal Article},
  Url                      = {http://www.sciencedirect.com/science/article/B6V6B-4N3WYPX-2/2/9709080f5ffc9deee53f3dc1b850c529}
}

@Article{Shi2007b,
  Title                    = {Modeling of inertial particle transport and deposition in human nasal cavities with wall roughness},
  Author                   = {Shi, H.W. and Kleinstreuer, C. and Zhang, Z.},
  Journal                  = {Journal of Aerosol Science},
  Year                     = {2007},
  Number                   = {4},
  Pages                    = {398-419},
  Volume                   = {38},

  Type                     = {Journal Article}
}

@Article{Shi2006,
  Title                    = {Laminar airflow and nanoparticle or vapor deposition in a human nasal cavity model.},
  Author                   = {Shi, H. and Kleinstreuer, C. and Zhang, Z.},
  Journal                  = {Journal Biomech. Eng.},
  Year                     = {2006},
  Number                   = {5},
  Pages                    = {697-706},
  Volume                   = {128},

  Type                     = {Journal Article}
}

@Article{Shi1984,
  Title                    = {Nucleoside transport. Photoaffinity labelling of high-affinity nitrobenzylthioinosine binding sites in rat and guinea pig lung},
  Author                   = {Shi, Maggie M. and Wu, Jin-Shyun R. and Lee, Chi-Ming and Young, James D.},
  Journal                  = {Biochemical and Biophysical Research Communications},
  Year                     = {1984},
  Number                   = {2},
  Pages                    = {594-600},
  Volume                   = {118},

  ISSN                     = {0006-291X},
  Type                     = {Journal Article},
  Url                      = {http://www.sciencedirect.com/science/article/B6WBK-4F031HH-SV/2/8046d860bc7efb99cd2844537cf91ff9}
}

@Article{Shih1995,
  Title                    = {A new k-ε eddy viscosity model for high Reynolds number turbulent flows},
  Author                   = {Shih, T.H. and Liou, W.W. and Shabbir, A. and Yang, Z. and Zhu, J.},
  Journal                  = {Comput. Fluids},
  Year                     = {1995},
  Pages                    = {227–238},
  Volume                   = {24},

  Type                     = {Journal Article}
}

@Article{Shipper1991,
  Title                    = {The nasal mucociliary clearance: relevance to nasal drug delivery.},
  Author                   = {Shipper, G. M. and Verhoef, J. and Merkus, W.H.M.},
  Journal                  = {Pharma. Res.},
  Year                     = {1991},
  Pages                    = {807-814},
  Volume                   = {7},

  Type                     = {Journal Article}
}

@Article{Shome1998,
  Title                    = {Modeling of Airflow in the Pharynx With Application to Sleep Apnea},
  Author                   = {Shome, B and Wang, L.P and Santare, M.H. and Prasad, A.K. and Szeri, A.Z. and Roberts, D.},
  Journal                  = {Journal of Biomechanical Engineering },
  Year                     = {1998},
  Number                   = {3},
  Pages                    = {7 pages},
  Volume                   = {Volume 120},

  Type                     = {Journal Article},
  Url                      = {http://scitation.aip.org/getabs/servlet/GetabsServlet?prog=normal&id=JBENDY000120000003000416000001&idtype=cvips&gifs=yes&ref=no}
}

@Article{Siebes2010,
  Title                    = {The Role of Biofluid Mechanics in the Assessment of Clinical and Pathological Observations},
  Author                   = {Siebes, Maria and Ventikos, Yiannis},
  Journal                  = {ANNALS OF BIOMEDICAL ENGINEERING},
  Year                     = {2010},
  Pages                    = {1216-1224},
  Volume                   = {38},

  Abstract                 = {Biofluid mechanics is increasingly applied in support of diagnosis and decision-making for treatment of clinical pathologies. Exploring the relationship between blood flow phenomena and pathophysiological observations is enhanced by continuing advances in the imaging modalities, measurement techniques, and capabilities of computational models. When combined with underlying physiological models, a powerful set of tools becomes available to address unmet clinical needs, predominantly in the direction of enhanced diagnosis, as well as assessment and prediction of treatment outcomes. This position paper presents an overview of current approaches and future developments along this theme that were discussed at the 5th International Biofluid Symposium and Workshop held at the California Institute of Technology in 2008. The introduction of novel mechanical biomarkers in device design and optimization, and applications in the characterization of more specific and focal conditions such as aneurysms, are at the center of attention. Further advances in integrative modeling, incorporating multiscale and multiphysics techniques are also discussed.},
  Doi                      = {10.1007/s10439-010-9903-y},
  ISSN                     = {0090-6964},
  Keywords                 = {Cardiovascular fluid mechanics
Computational modeling
Hemodynamics
Imaging
Physiological modeling},
  Type                     = {Journal Article},
  Url                      = {http://www.ncbi.nlm.nih.gov/pmc/articles/PMC2841261/pdf/10439_2010_Article_9903.pdf}
}

@Article{Sittitavornwong2009,
  Title                    = {Evaluation of Obstructive Sleep Apnea Syndrome by Computational Fluid Dynamics},
  Author                   = {Sittitavornwong, Somsak and Waite, Peter D. and Shih, Alan M. and Koomullil, Roy and Ito, Yasushi and Cheng, Gary C. and Wang, Deli},
  Journal                  = {Seminars in Orthodontics},
  Year                     = {2009},
  Number                   = {2},
  Pages                    = {105-131},
  Volume                   = {15},

  Abstract                 = {The amelioration of obstructive sleep apnea syndrome (OSAS) by maxillomandibular advancement (MMA) surgery can be predicted by analyzing anatomical airway changes with 3-dimensional (3D) geometrical reconstruction and computational fluid dynamics. Computer Enabling Technology Lab (ETLab) and Computational Simulation Lab (CSLab) can be used to analyze anatomic airway change for previously operated patients with a clinical cure of OSAS. MMA surgery reduces airway resistance and pressure effort (gradient) of OSAS by increasing the dimension of the airway. ETLab has been used to reconstruct the upper airway as a 3D computer model (bone and soft tissue surrounding the pharyngeal airway) from existing helical computed tomography scan format of OSAS patients. ETLab can compare and construct the geometry with numerical meshes of the airway between pre- and postoperative MMA by the use of bioengineering software. This technology uses high-fidelity computation fluid dynamic simulations, developed at the CSLab, for prediction and analysis of the flow field in the airway for pre- and postoperative MMA. It is possible to use the simulation to predict the likely success of future treatment and develop a prognostic factor. The soft- and hard-tissue mesh is used to determine the pre- and postoperative differences in the facial and pharyngeal tongue base for soft-tissue change associated with hard-tissue movement. This correlation predicts the amount of surgical movement necessary to create an adequate airflow. These results help define the surgical techniques in OSAS for more precise identification of upper airway anatomical features. This process correlates the area and pressure change at the velopharynx, oropharynx, and retroglossal space of the upper airway by the ETLab. Results can compare with polysomnogram and cure rates. 3D computer analysis can be used to test flow dynamics in the human airway for surgical treatment of OSAS.},
  ISSN                     = {1073-8746},
  Type                     = {Journal Article},
  Url                      = {http://www.sciencedirect.com/science/article/B75KK-4WD9RX0-8/2/fd34d876da3d62acee1160f8cfa42745}
}

@Article{Skalak1989,
  Title                    = {Biofluid Mechanics},
  Author                   = {Skalak, R. and Ozkaya, N. and Skalak, T.C. },
  Journal                  = {Annual Review of Fluid Mechanics},
  Year                     = {1989},
  Pages                    = {167-204},
  Volume                   = {21},

  Type                     = {Journal Article}
}

@Article{Skalak1973,
  Title                    = {Strain energy function of red blood cell membranes},
  Author                   = {Skalak, R. and Tozeren, A. and Zarda, R. P. and Chien, S.},
  Journal                  = {Biophysical Journal},
  Year                     = {1973},
  Note                     = {Cited By (since 1996): 131
Export Date: 5 June 2011
Source: Scopus},
  Number                   = {3},
  Pages                    = {245-264},
  Volume                   = {13},

  Type                     = {Journal Article},
  Url                      = {http://www.scopus.com/inward/record.url?eid=2-s2.0-0015595017&partnerID=40&md5=387881d76b9ee4bdd25cd50a1582e9d4}
}

@Article{Sleeth2009,
  Title                    = {Inhalability for aerosols at ultra-low windspeeds},
  Author                   = {Sleeth, D.K. and Vincent, J.H.},
  Journal                  = {Journal of Physics: Conference Series},
  Year                     = {2009},
  Number                   = {1},
  Pages                    = {012062},
  Volume                   = {151},

  Abstract                 = {Most previous experimental studies of aerosol inhalability were conducted in wind tunnels for windspeeds greater than 0.5 ms- 1 . While that body of work was used to establish a convention for the inhalable fraction, results from studies in calm air chambers (for essentially zero windspeed) are being discussed as the basis of a modified criterion. However, information is lacking for windspeeds in the intermediate range, which - it so happens - pertain to most actual workplaces. With this in mind, we have developed a new experimental system to assess inhalability - and, ultimately, personal sampler performance - for aerosols with particle aerodynamic diameter within the range from about 9 to 90 μm for ultra-low windspeed environments from about 0.1 to 0.5 ms 1 . This new system contains an aerosol test facility, fully described elsewhere, that combines the physical attributes and performance characteristics of moving air wind tunnels and calm air chambers, both of which have featured individually in previous research. It also contains a specially-designed breathing, heated, life-sized mannequin that allows for accurate recovery of test particulate material that has been inhaled. Procedures have been developed that employ test aerosols of well-defined particle size distribution generated mechanically from narrowly-graded powders of fused alumina. Using this new system, we have conducted an extensive set of new experiments to measure the inhalability of a human subject (as represented by the mannequin), aimed at filling the current knowledge gap for conditions that are more realistic than those embodied in most previous research. These data reveal that inhalability throughout the range of interest is significantly different based on windspeed, indicating a rise in aspiration efficiency as windspeed decreases. Breathing flowrate and mode of breathing (i.e. nose versus mouth breathing) did not show significant differences for the inhalability of aerosols. On the whole however, the data obtained here are within the range of inhalability data that exist from the large body of the previous experimental work performed at the higher windspeeds. These latest findings are an important contribution to the ongoing discussion in international standards-setting bodies about the possible adjustment of the quantitative definition of what constitutes the inhalable fraction.},
  ISSN                     = {1742-6596},
  Type                     = {Journal Article},
  Url                      = {http://stacks.iop.org/1742-6596/151/i=1/a=012062}
}

@Article{Slutsky1981,
  Title                    = {Oscillatory flow and quasi-steady behavior in a model of human central airways},
  Author                   = {Slutsky, A.S. and Berdine, G.G. and Drazen, J.M.},
  Journal                  = {Journal Appl. Physiol.},
  Year                     = {1981},
  Number                   = {6},
  Pages                    = {1293-1299},
  Volume                   = {50},

  Type                     = {Journal Article}
}

@Article{Smith2001,
  Title                    = {Deposition of ultrafine particles in human tracheobronchial airways of adults and children},
  Author                   = {Smith, S. and Cheng, Y.S. and Yeh, H.C.},
  Journal                  = {Aerosol Science and Technology},
  Year                     = {2001},
  Pages                    = {697-709 },
  Volume                   = {35},

  Type                     = {Journal Article}
}

@Article{Smith2001a,
  Title                    = {Deposition of Ultrafine Particles in Human Tracheobronchial Airways of Adults and Children},
  Author                   = {Smith, Shawna and Cheng, Yung-Sung and Yeh, Hsu Chi},
  Journal                  = {Aerosol Science and Technology},
  Year                     = {2001},
  Number                   = {3},
  Pages                    = {697-709},
  Volume                   = {35},

  ISSN                     = {0278-6826},
  Type                     = {Journal Article},
  Url                      = {http://www.informaworld.com/10.1080/02786820152546743}
}

@Article{Smith2001b,
  Title                    = {Simulations of flow around a cubical building: comparison with towing-tank data and assessment of radiatively induced thermal effects},
  Author                   = {Smith, W. S. and Reisner, J. M. and Kao, C. Y. J.},
  Journal                  = {Atmospheric Environment},
  Year                     = {2001},
  Number                   = {22},
  Pages                    = {3811-3821},
  Volume                   = {35},

  Abstract                 = {A three-dimensional (3-D) computational fluid dynamics (CFD) model, coupled with a meteorological radiation and surface physics package, is used to model the mean flow field and tracer dispersion in the vicinity of an idealized cubical building. We first compare the simulations with earlier numerical studies as well as towing-tank laboratory experiments, where radiation effects were not included. Our simulations capture most of the features revealed by the towing-tank data, including the variation of the flow reattachment point as a function of Froude number and the induction of a prominent lee wave in the low Froude number regime. The simulated tracer concentration also compares very favorably with the data. We then assess the thermal effects due to radiative heating on the ground and building including shading by the building, on the mean flow and tracer dispersion. Our simulations show that convergence within and beyond the cavity zone causes a substantial lofting of the air mass downstream from the building. This lofting results from the combination of thermal heating of the ground and building roof, and vortex circulation associated with the horseshoe eddy along the lateral sides of the building. The specific effect of shading on the flow field is isolated by comparing simulations for which the radiative heating and shading patterns are kept constant, but the environmental wind direction is altered. It is found that the shading exerts local cooling, which can be combined into the overall thermodynamic interaction, described above, to effectively alter the circulation downstream from the building.},
  ISSN                     = {1352-2310},
  Keywords                 = {Large eddy simulation
Dispersion
Transport
Urban
Building
Thermal effects},
  Type                     = {Journal Article},
  Url                      = {http://www.sciencedirect.com/science/article/B6VH3-43B28TY-7/2/1a15035792b37bf6561c1c93789926bc}
}

@Article{Smoluchowski1916,
  Title                    = {Drei Vorträge über Diffusion, Brownsche Molekularbewegung und Koagulation von Kolloidteilchen},
  Author                   = {Smoluchowski, Marian},
  Journal                  = {Physik. Zeit},
  Year                     = {1916},
  Pages                    = {557–571},
  Volume                   = {17},

  Type                     = {Journal Article}
}

@Article{Snyder1971,
  Title                    = {Some Measurements of Particle Velocity Autocorrelation Functions in a Turbulent Flow},
  Author                   = {Snyder, W.H. and Lumley, J.L. },
  Journal                  = {Journal of Fluid Mech},
  Year                     = {1971},
  Pages                    = {41-71},
  Volume                   = {48},

  Type                     = {Journal Article}
}

@Article{Snyder1987,
  Title                    = {A wind tunnel study of the flow structure and dispersion from sources upwind of three-dimensional hills},
  Author                   = {Snyder, W. H. and Britter, R. E.},
  Journal                  = {Atmospheric Environment - Part A General Topics},
  Year                     = {1987},
  Note                     = {Cited By (since 1996): 7
Export Date: 6 June 2011
Source: Scopus},
  Number                   = {4},
  Pages                    = {735-751},
  Volume                   = {21},

  Type                     = {Journal Article},
  Url                      = {http://www.scopus.com/inward/record.url?eid=2-s2.0-0023138186&partnerID=40&md5=a20787d0ce9170397697cc6fb3caf7fc}
}

@Article{Soltani2000,
  Title                    = {Direct numerical simulation of curly fibers in turbulent channel flow},
  Author                   = {Soltani, M. and Ahmadi, G.},
  Journal                  = {Aerosol Science and Technology},
  Year                     = {2000},
  Note                     = {Cited By (since 1996): 7
Export Date: 6 June 2011
Source: Scopus},
  Number                   = {5},
  Pages                    = {392-418},
  Volume                   = {33},

  Type                     = {Journal Article},
  Url                      = {http://www.scopus.com/inward/record.url?eid=2-s2.0-0034545145&partnerID=40&md5=f4321f0efca9a46ff3c95e1d860ec828}
}

@Article{Soltani1995,
  Title                    = {Direct numerical simulation of particle entrainment in turbulent channel flow},
  Author                   = {Soltani, M. and Ahmadi, G.},
  Journal                  = {Physics of Fluids},
  Year                     = {1995},
  Note                     = {Cited By (since 1996): 27
Export Date: 6 June 2011
Source: Scopus},
  Number                   = {3},
  Pages                    = {647-657},
  Volume                   = {7},

  Type                     = {Journal Article},
  Url                      = {http://www.scopus.com/inward/record.url?eid=2-s2.0-0028976493&partnerID=40&md5=ed78417cf0ba1e5dcd6ac52a8ee21fca}
}

@Article{Soltani1998,
  Title                    = {Direct simulation of charged particle deposition in a turbulent flow},
  Author                   = {Soltani, M. and Ahmadi, G. and Ounis, H. and McLaughlin, J. B.},
  Journal                  = {International Journal of Multiphase Flow},
  Year                     = {1998},
  Note                     = {doi: DOI: 10.1016/S0301-9322(97)00042-6},
  Number                   = {1},
  Pages                    = {77-92},
  Volume                   = {24},

  ISSN                     = {0301-9322},
  Keywords                 = {charged particle deposition
turbulent flow
direct simulation},
  Type                     = {Journal Article},
  Url                      = {http://www.sciencedirect.com/science/article/B6V45-3SYPPV0-5/2/cffd1453c63d1405a209f6ba085f8a42}
}

@Article{Sommerfeld1990,
  Title                    = {Particle Dispersion in Turbulent Flow: The effect of particle size distribution},
  Author                   = {Sommerfeld, Martin},
  Journal                  = {Particle and Particle Systems Characterization},
  Year                     = {1990},
  Note                     = {10.1002/ppsc.19900070135},
  Number                   = {1-4},
  Pages                    = {209-220},
  Volume                   = {7},

  ISSN                     = {1521-4117},
  Type                     = {Journal Article},
  Url                      = {http://dx.doi.org/10.1002/ppsc.19900070135}
}

@Article{Sommerfeld2003,
  Title                    = {Numerical calculation of particle transport in turbulent wall bounded flows},
  Author                   = {Sommerfeld, Martin and Ho, Chi Anh},
  Journal                  = {Powder Technology},
  Year                     = {2003},
  Note                     = {doi: DOI: 10.1016/S0032-5910(02)00293-0},
  Number                   = {1},
  Pages                    = {1-6},
  Volume                   = {131},

  ISSN                     = {0032-5910},
  Type                     = {Journal Article},
  Url                      = {http://www.sciencedirect.com/science/article/B6TH9-48165G5-1/2/1b2328c570eb9bcac48760b35b4c0773}
}

@Article{Song2013,
  Title                    = {{Rhinitis in a community elderly population: relationships with age,
 atopy, and asthma}},
  Author                   = {Song, Woo-Jung and Kim, Mi-Yeong and Jo, Eun-Jung and Kim, Min-Hye and
 Kim, Tae-Hui and Kim, Sae-Hoon and Kim, Ki-Woong and Cho, Sang-Heon and
 Min, Kyung-Up and Chang, Yoon-Seok},
  Journal                  = {{ANNALS OF ALLERGY ASTHMA \& IMMUNOLOGY}},
  Year                     = {{2013}},

  Month                    = {{NOV}},
  Number                   = {{5}},
  Pages                    = {{347-351}},
  Volume                   = {{111}},

  Abstract                 = {{Background: Rhinitis is one of the most frequent medical conditions.
 However, there is sparse epidemiologic evidence for rhinitis in the
 elderly population.
 Objective: To investigate the prevalence of rhinitis in elderly adults
 and its relations to asthma and other comorbidities.
 Methods: A cross-sectional analysis was performed using the baseline
 dataset of the Korean Longitudinal Study on Health and Aging, a
 community-based elderly population cohort in Korea (>= 65 years old).
 Structured questionnaires were used to define rhinitis, asthma, and
 comorbidity, and allergen skin prick tests were used to define atopy.
 Health-related quality of life was assessed by short-form 36
 questionnaires.
 Results: In total, 982 elderly adults (98.2\%) were included in the
 present study. The prevalence of rhinitis was 25.6\% and did not
 decrease until 90 years of age. The prevalence of atopy was 17.2\%
 (18.8\% in participants with rhinitis), and atopy did not show a
 significant association with rhinitis. In multivariate logistic
 regression analyses, relations between asthma and rhinitis were
 significant. Among comorbid conditions, none were significantly
 associated with rhinitis. In the short-form 36 questionnaire analyses,
 rhinitis was independently related to a decrease in the physical aspects
 of quality of life.
 Conclusion: The present study found a high prevalence of nonallergic
 rhinitis in elderly participants, which was significantly related to
 asthma and quality of life. (c) 2013 American College of Allergy, Asthma
 \& Immunology. Published by Elsevier Inc. All rights reserved.}},
  Address                  = {{360 PARK AVE SOUTH, NEW YORK, NY 10010-1710 USA}},
  Affiliation              = {{Chang, YS (Reprint Author), Seoul Natl Univ, Bundang Hosp, Dept Internal Med, 300 Gumi Dong, Songnam 463802, South Korea.
 Song, Woo-Jung; Kim, Mi-Yeong; Jo, Eun-Jung; Kim, Min-Hye; Kim, Sae-Hoon; Cho, Sang-Heon; Min, Kyung-Up; Chang, Yoon-Seok, Seoul Natl Univ, Coll Med, Dept Internal Med, Seoul 151, South Korea.
 Song, Woo-Jung; Kim, Mi-Yeong; Jo, Eun-Jung; Kim, Min-Hye; Kim, Sae-Hoon; Cho, Sang-Heon; Min, Kyung-Up; Chang, Yoon-Seok, Seoul Natl Univ, Med Res Ctr, Inst Allergy \& Clin Immunol, Seoul, South Korea.
 Jo, Eun-Jung; Kim, Min-Hye; Kim, Sae-Hoon; Chang, Yoon-Seok, Seoul Natl Univ, Bundang Hosp, Dept Internal Med, Songnam 463802, South Korea.
 Kim, Tae-Hui; Kim, Ki-Woong, Seoul Natl Univ, Bundang Hosp, Dept Neuropsychiat, Songnam, South Korea.
 Kim, Ki-Woong, Seoul Natl Univ, Coll Med, Dept Psychiat, Seoul, South Korea.}},
  Author-email             = {{addchang@snu.ac.kr}},
  Doc-delivery-number      = {{235XT}},
  Doi                      = {{10.1016/j.anai.2013.08.015}},
  Eissn                    = {{1534-4436}},
  Funding-acknowledgement  = {{Developing Seongnam Health Promotion Program for the Elderly from the
 Seongnam City Government in the Republic of Korea {[}800-20050211];
 Korean Health 21 R\&D Project, Ministry of Health, Welfare and Family
 Affairs in the Republic of Korea {[}A102065]}},
  Funding-text             = {{This study was supported by the Developing Seongnam Health Promotion
 Program for the Elderly from the Seongnam City Government in the
 Republic of Korea (grant 800-20050211) and the Korean Health 21 R\&D
 Project, Ministry of Health, Welfare and Family Affairs in the Republic
 of Korea (grantA102065).}},
  ISSN                     = {{1081-1206}},
  Journal-iso              = {{Ann. Allergy Asthma Immunol.}},
  Keywords-plus            = {{INDEPENDENT RISK-FACTOR; ALLERGIC RHINITIS; NONALLERGIC RHINITIS;
 GENERAL-POPULATION; PERENNIAL RHINITIS; ECONOMIC BURDEN; BLOOD-PRESSURE;
 ADULTS; ASSOCIATION; PREVALENCE}},
  Language                 = {{English}},
  Number-of-cited-references = {{44}},
  Publisher                = {{ELSEVIER SCIENCE INC}},
  Research-areas           = {{Allergy; Immunology}},
  Times-cited              = {{0}},
  Type                     = {{Article}},
  Unique-id                = {{ISI:000325756900010}},
  Web-of-science-categories = {{Allergy; Immunology}}
}

@Article{Sorkness2007,
  Title                    = {Commentary on "The role of the large airways on smooth muscle contraction in asthma"},
  Author                   = {Sorkness, R.},
  Journal                  = {Journal of Applied Physiology},
  Year                     = {2007},
  Pages                    = {1464 - doi:10.1152/japplphysiol.00702.2007},
  Volume                   = {103},

  Type                     = {Journal Article}
}

@Article{Spalart2000,
  Title                    = {Strategies for turbulence modelling and simulations},
  Author                   = {Spalart, P.R.},
  Journal                  = {International Journal of Heat and Fluid Flow},
  Year                     = {2000},
  Pages                    = {25-263},
  Volume                   = {21},

  Type                     = {Journal Article}
}

@Article{Spence2011,
  Title                    = {Unsteady flow in the nasal cavity with high flow therapy measured by stereoscopic PIV},
  Author                   = {Spence, C. and Buchmann, N. and Jermy, M.},
  Journal                  = {Experiments in Fluids},
  Year                     = {2011},
  Pages                    = {1-11},

  Doi                      = {10.1007/s00348-011-1044-z},
  ISSN                     = {0723-4864},
  Keywords                 = {Physics and Astronomy},
  Type                     = {Journal Article},
  Url                      = {http://dx.doi.org/10.1007/s00348-011-1044-z}
}

@Misc{Spence2010,
  Title                    = {Pulsatile flow in the nasal cavity with high flow therapy measured by stereoscopic PIV},

  Author                   = {Spence, C.J.T. and Buchmann, N.A. and Jermy, M.C.},
  Month                    = { 05-08 July, 2010},
  Year                     = {2010},

  Type                     = {Conference Paper}
}

@Article{Spencer2001,
  Title                    = {Computer simulations of lung airway structures using data-driven surface modeling techniques},
  Author                   = {Spencer, Richard M. and Schroeter, Jeffry D. and Martonen, Ted B.},
  Journal                  = {Computers in Biology and Medicine},
  Year                     = {2001},
  Number                   = {6},
  Pages                    = {499-511},
  Volume                   = {31},

  Abstract                 = {Knowledge of human lung morphology is a subject critical to many areas of medicine. The visualization of lung structures naturally lends itself to computer graphics modeling due to the large number of airways involved and the complexities of the branching systems. In this study, a method of generating three-dimensional computer simulations of human lung airway networks using data-driven, surface modeling techniques is presented. By simulating the tubular airway structures and realistic bifurcation shapes, anatomically accurate representations of human lungs are obtained. These computer models are designed for use in computational fluid dynamic applications and particle trajectory analyses, and to be complimentary to medical imaging (gamma scintigraphy) protocols.},
  ISSN                     = {0010-4825},
  Keywords                 = {Computer simulations
Lung morphology
Airway structure
Surface modeling
NURBS surfaces
Computational fluid dynamics
Aerosol therapy},
  Type                     = {Journal Article},
  Url                      = {http://www.sciencedirect.com/science/article/B6T5N-445RJG0-7/2/287183b309c1c610a45cf1cb9597cf57}
}

@Article{Spengler2000,
  Title                    = { Indoor air quality factors in designing a healthy building},
  Author                   = {Spengler, J.D. and Chen, Q.},
  Journal                  = {Annual Review of Energy and the Environment},
  Year                     = {2000},
  Pages                    = {567-600},
  Volume                   = {25},

  Type                     = {Journal Article}
}

@Book{Spurny1986,
  Title                    = {Physical and Chemical Characterization of Individual Airborne Particles},
  Author                   = {Spurny, K.R. },
  Publisher                = {John Wiley and Sons},
  Year                     = {1986},

  Address                  = {New York},

  Type                     = {Book}
}

@Article{Squire2001,
  Title                    = {Quasi-Periodic Substructure in the Microvessel Endothelial Glycocalyx: A Possible Explanation for Molecular Filtering?},
  Author                   = {Squire, John M. and Chew, Michael and Nneji, Gwen and Neal, Chris and Barry, John and Michel, Charles},
  Journal                  = {Journal of Structural Biology},
  Year                     = {2001},
  Number                   = {3},
  Pages                    = {239-255},
  Volume                   = {136},

  ISSN                     = {1047-8477},
  Keywords                 = {small blood vessels
glycocalyx
endothelial cells
ultrafiltration
fibre matrix model},
  Type                     = {Journal Article},
  Url                      = {http://www.sciencedirect.com/science/article/pii/S1047847702944412}
}

@Article{Squires1991,
  Title                    = {On the preferential concentration of solid particles in turbulent},
  Author                   = {Squires, K.D. and Eaton, J.K. },
  Journal                  = {Physics of Fluids A},
  Year                     = {1991},
  Pages                    = {1169-1178},
  Volume                   = {3},

  Type                     = {Journal Article}
}

@Article{Srebric2002,
  Title                    = {Simplified numerical models for complex air supply diffusers},
  Author                   = {Srebric, J. and Chen, Q.},
  Journal                  = {HVAC\&R Research},
  Year                     = {2002},
  Number                   = {3},
  Pages                    = {277-294},
  Volume                   = {8},

  Type                     = {Journal Article}
}

@Article{Srebric2008,
  Title                    = {CFD boundary conditions for contaminant dispersion, heat transfer and airflow simulations around human occupants in indoor environments.},
  Author                   = {Srebric, J. and Vukovic, V. and He, G. and Yang, X.},
  Journal                  = {Building and Environment },
  Year                     = {2008},
  Pages                    = {294-303},
  Volume                   = {43},

  Type                     = {Journal Article}
}

@Article{St.Martin2007,
  Title                    = {Deposition of aerosolized particles in the maxillary sinuses before and after endoscopic sinus surgery},
  Author                   = {St. Martin, Michele B. and Hitzman, Cory J. and Wiedmann, Timothy S. and Rimell, Frank L.},
  Journal                  = {American Journal of Rhinology},
  Year                     = {2007},
  Note                     = {[1]
[2]
[3]},
  Pages                    = {196-197},
  Volume                   = {21},

  Type                     = {Journal Article},
  Url                      = {http://www.ingentaconnect.com/content/ocean/ajr/2007/00000021/00000002/art00013
http://dx.doi.org/10.2500/ajr.2007.21.2963}
}

@InBook{StA¶ber1972,
  Title                    = {Dynamic shape factors of nonspherical aerosol particles},
  Author                   = {Stöber, W.},
  Editor                   = {al., T. Mercer et},
  Pages                    = {249-289},
  Publisher                = {Charles C. Thomas},
  Year                     = {1972},

  Address                  = {Springfield, IL},
  Type                     = {Book Section},

  Booktitle                = {Assessment of airborne particles}
}

@Article{Stabl1992,
  Title                    = {Experimental investigation of the vortex flow on delta wings at high incidence},
  Author                   = {Stabl, W.H.},
  Journal                  = {AIAA Journal},
  Year                     = {1992},
  Number                   = {4},
  Pages                    = {1027-1032},
  Volume                   = {30},

  Type                     = {Journal Article}
}

@Article{Stamou2006,
  Title                    = {Verification of a CFD model for indoor airflow and heat transfer},
  Author                   = {Stamou, A. and Katsiris, I.},
  Journal                  = {Building and Environment},
  Year                     = {2006},
  Number                   = {9},
  Pages                    = {1171-1181},
  Volume                   = {41},

  Doi                      = {10.1016/j.buildenv.2005.06.029},
  ISSN                     = {0360-1323},
  Keywords                 = {Computational fluid dynamics (CFD)
Mathematical models
Indoor environment
Office spaces
Thermal comfort
CFX},
  Type                     = {Journal Article},
  Url                      = {http://www.sciencedirect.com/science/article/pii/S0360132306000394}
}

@Article{Stanek2007,
  Title                    = {On a mechanism of stabilizing turbulent free shear layers in cavity flows},
  Author                   = {Stanek, Michael J. and Visbal, Miguel R. and Rizzetta, Donald P. and Rubin, Stanley G. and Khosla, Prem K.},
  Journal                  = {Computers \& Fluids},
  Year                     = {2007},
  Number                   = {10},
  Pages                    = {1621-1637},
  Volume                   = {36},

  Abstract                 = {Turbulent free shear flows are subject to the well-known Kelvin-Helmholtz type [Panton RL. Incompressible flow. John Wiley and Sons; 1984. p. 675] instability, and it is well-known that any free shear flow which approximates a thin vorticity layer will be unstable to a wide range of amplitudes and frequencies of disturbance. In fact, much of what constitutes flow control in turbulent free shear layers consists of feeding a prescribed destabilizing disturbance to these layers. The question in the control of free shear flows is not whether the shear layer will be stable, but whether you can influence how the layer becomes unstable. In most cases, since these flows are so receptive to forcing input, and naturally tend toward instability, large changes in flow conditions can be achieved with very small amplitude periodic inputs. Recently, it has been discovered that turbulent free shear flows can also be stabilized using periodic forcing. This is, at first glance, counter-intuitive, considering our long history of considering these flows to be very unstable to forcing input. It is a phenomenon not described in modern fluid dynamic text books. The forcing required to achieve this effect (which we will call turbulent shear layer stabilization) is of a much higher amplitude and frequency than the more traditional type of shear layer flow control effect seen in the literature (which we will call turbulent shear layer destabilization). A numerical study is undertaken to investigate the effect of frequency of pulsed mass injection on the nature of stabilization, destabilization and acoustic suppression in high speed cavity flows. An implicit, 2nd-order in space and time flow solver, coupled with a recently developed hybrid RANS-LES (Reynolds Averaged Navier Stokes-Large Eddy Simulation) turbulence model by Nichols and Nelson [Nichols RH, Nelson CC. Weapons bay acoustic predictions using a multi-scale turbulence model. In: Proceedings of the ITEA 2001 aircraft-stores compatibility symposium, March 2001], is utilized in a Chimera-based parallel format. This tool is used to numerically simulate both an unsuppressed cavity in resonance, as well as the effect of mass-addition pulsed jet flow control on cavity flow physics and ultimately, cavity acoustic levels. Frequency (and in a limited number of cases, amplitude) of pulse is varied, from 0 Hz (steady) up to 5000 Hz. The change in the character of the flow control effect as pulsing frequency is changed is described, and linked to changes in acoustic levels. Limited comparison to 1/10th scale experiments is presented. The observed local stabilization of the cavity turbulent shear layer, when subjected to high frequency pulsed blowing, is shown in simulation to be the result of a violent instability and breakdown of a pair of opposite sign vortical structures created with each high frequency "pulse". This unique shear layer stabilization behavior is only observed in simulation above a certain critical pulsing frequency. Below this critical frequency, pulsing is shown in simulation to provide little benefit with respect to suppression of high cavity acoustic levels.},
  ISSN                     = {0045-7930},
  Type                     = {Journal Article},
  Url                      = {http://www.sciencedirect.com/science/article/B6V26-4NCJCJD-2/2/e4762a2c96e6695e033b603435316f16}
}

@Article{Stapleton2000,
  Title                    = {On the suitability of k-[var epsilon] turbulence modeling for aerosol deposition in the mouth and throat: a comparison with experiment},
  Author                   = {Stapleton, K. W. and Guentsch, E. and Hoskinson, M. K. and Finlay, W. H.},
  Journal                  = {Journal of Aerosol Science},
  Year                     = {2000},
  Number                   = {6},
  Pages                    = {739-749},
  Volume                   = {31},

  ISSN                     = {0021-8502},
  Type                     = {Journal Article},
  Url                      = {http://www.sciencedirect.com/science/article/pii/S0021850299005479}
}

@Article{Stapper1992,
  Title                    = {An Experimental Study of the Effects of Liquid Properties on the Breakup of a Two-Dimensional Liquid Sheet},
  Author                   = {Stapper, B. E. and Sowa, W. A. and Samuelsen, G. S.},
  Journal                  = {Journal of Engineering for Gas Turbines and Power},
  Year                     = {1992},
  Note                     = {10.1115/1.2906305},
  Number                   = {1},
  Pages                    = {39-45},
  Volume                   = {114},

  Abstract                 = {The breakup of a liquid sheet is of fundamental interest in the atomization of liquid fuels. The present study explores the breakup of a two-dimensional liquid sheet in the presence of co-flow air with emphasis on the extent to which liquid properties affect breakup. Three liquids, selected with varying values of viscosity and surface tension, are introduced through a twin-fluid, two-dimensional nozzle. A pulsed laser imaging system is used to determine the sheet structure at breakup, the distance and time to breakup, and the character of the ligaments and droplets formed. Experiments are conducted at two liquid flow rates with five flow rates of co-flowing air. Liquid properties affect the residence time required to initiate sheet breakup, and alter the time and length scales in the breakup mechanism.},
  Doi                      = {10.1115/1.2906305},
  ISSN                     = {0742-4795},
  Type                     = {Journal Article},
  Url                      = {http://dx.doi.org/10.1115/1.2906305}
}

@Article{Stephens2003,
  Title                    = {The biophysics of asthmatic airway smooth muscle},
  Author                   = {Stephens, Newman L. and Li, Weilong and Jiang, He and Unruh, H. and Ma, Xuefei},
  Journal                  = {Respiratory Physiology \& Neurobiology},
  Year                     = {2003},
  Note                     = {doi: DOI: 10.1016/S1569-9048(03)00142-3},
  Number                   = {2-3},
  Pages                    = {125-140},
  Volume                   = {137},

  ISSN                     = {1569-9048},
  Keywords                 = {Airway, smooth muscle
Disease, asthma
Mammals, dog, humans
Muscle, smooth, airway, plasticity},
  Type                     = {Journal Article},
  Url                      = {http://www.sciencedirect.com/science/article/B6X16-49H1M17-1/2/1589812454b547ad33e82ae726afb9ba}
}

@Article{Stepnowsky2002,
  Title                    = {Determinants of nasal CPAP compliance},
  Author                   = {Stepnowsky, Carl J. and Marler, Matthew R. and Ancoli-Israel, Sonia},
  Journal                  = {Sleep Medicine},
  Year                     = {2002},
  Number                   = {3},
  Pages                    = {239-247},
  Volume                   = {3},

  ISSN                     = {1389-9457},
  Keywords                 = {Sleep apnea
Continuous positive airway pressure
Patient compliance
Behavior change
Self-efficacy
Social cognitive theory
Transtheoretical model},
  Type                     = {Journal Article},
  Url                      = {http://www.sciencedirect.com/science/article/B6W6N-44NM5C8-6/2/d652e074ab6902e6e3985fdbcdc8b86c}
}

@Article{Straatsma1999,
  Title                    = {Spray drying of food products:1.Simulation model.},
  Author                   = {Straatsma, J. and Van Houwelingen, G. and Steenbergen, A.E. and De Jong, P.},
  Journal                  = {Journal Food Eng.},
  Year                     = {1999},
  Number                   = {2},
  Pages                    = {67-72},
  Volume                   = {42},

  Type                     = {Journal Article}
}

@Misc{Stringer2010,
  Title                    = {Numerical comparison of airflow patterns in the upper airways of adults and neonates},

  Author                   = {Stringer, N.M. and Cater, J.E. and Eaton-Evans, J. and White, C.},
  Month                    = {5-9 December 2010},
  Year                     = {2010},

  Type                     = {Conference Paper}
}

@InProceedings{Strong,
  Title                    = {Deposition of Ultrafine Particles in a Human Nasal Cast},
  Author                   = {Strong, J.C. and Swift, D.L. },
  Booktitle                = {In proceedings of the First Conference of Aerosol Society},
  Pages                    = {109-112},

  Type                     = {Conference Proceedings}
}

@Article{Sturm2005,
  Title                    = {3D-Visualization of particle deposition patterns in the human lung generated by Monte Carlo modeling: methodology and applications},
  Author                   = {Sturm, R. and Hofmann, W.},
  Journal                  = {Computers in Biology and Medicine},
  Year                     = {2005},
  Number                   = {1},
  Pages                    = {41-56},
  Volume                   = {35},

  Abstract                 = {An advanced stochastic model is described which enables the generation of three-dimensional particle deposition patterns in the human lung. While particle trajectories are represented as a combination of randomly oriented vectors in a coordinate system with the trachea defining the z direction, deposition sites of single particles are determined by using a grid of specific volume elements (voxels). After storage in an array, the spatial coordinates are visualized with an appropriate graphic editor, enabling the combination of respective deposition images with lung outlines and the creation of two-dimensional distributions by sectioning the three-dimensional structures at pre-defined positions.},
  ISSN                     = {0010-4825},
  Keywords                 = {Lung
Stochastic model
Deposition
Voxel
Visualization
Monte Carlo algorithm},
  Type                     = {Journal Article},
  Url                      = {http://www.sciencedirect.com/science/article/B6T5N-4BJ1XSF-1/2/71ae05674ab2944222ed070c39137cb9}
}

@Article{Su2005,
  Title                    = {Deposition of fiber in the human nasal airway},
  Author                   = {Su, W.C. and Cheng, Y.S.},
  Journal                  = {Aerosol Science andTechnology},
  Year                     = {2005},
  Pages                    = {888–901},
  Volume                   = {39},

  Type                     = {Journal Article}
}

@Article{Subramaniam1998,
  Title                    = {Computational fluid dynamics simulations of inspiratory airflow in the human nose and nasopharynx},
  Author                   = {Subramaniam, R.P. and Richardson, R.B. and Morgan, K.T. and Kimbell, J. S. and Guilmette, R.A.},
  Journal                  = {Inhalation Toxicology},
  Year                     = {1998},
  Number                   = {91-120},
  Volume                   = {10},

  Type                     = {Journal Article}
}

@Article{Sullivan1991,
  Title                    = {Steady and oscillatory trans-nasal pressure-flow relationships in healthy adults.},
  Author                   = {Sullivan, K.J. and Chang, H.K.},
  Journal                  = {Journal Appl. Physiol.},
  Year                     = {1991},
  Pages                    = {983-992},
  Volume                   = {71},

  Type                     = {Journal Article}
}

@Article{Suman2002,
  Title                    = {Validity of in vitro tests on aqueous spray pumps as surrogates for nasal deposition.},
  Author                   = {Suman, J.D. and Laube, B.L. and Lin, T.C. and Brouet, G. and Dalby, R.},
  Journal                  = {Pharma. Res.},
  Year                     = {2002},
  Number                   = {1},
  Pages                    = {1-6},
  Volume                   = {19},

  Type                     = {Journal Article}
}

@Article{Sung2009,
  Title                    = {Formulation and Pharmacokinetics of Self-Assembled Rifampicin Nanoparticle Systems for Pulmonary Delivery},
  Author                   = {Sung, Jean and Padilla, Danielle and Garcia-Contreras, Lucila and VerBerkmoes, Jarod and Durbin, David and Peloquin, Charles and Elbert, Katharina and Hickey, Anthony and Edwards, David},
  Journal                  = {Pharmaceutical Research},
  Year                     = {2009},
  Number                   = {8},
  Pages                    = {1847-1855},
  Volume                   = {26},

  Abstract                 = {Abstract Purpose&nbsp;&nbsp;To formulate rifampicin, an anti-tuberculosis antibiotic, for aerosol delivery in a dry powder ‘porous nanoparticle-aggregate particle’ (PNAP) form suited for shelf stability, effective dispersibility and extended release with local lung and systemic drug delivery. Methods&nbsp;&nbsp;Rifampicin was encapsulated in PLGA nanoparticles by a solvent evaporation process, spray dried into PNAPs containing varying amounts of nanoparticles, and characterized for physical and aerosol properties. Pharmacokinetic studies were performed with formulations delivered to guinea pigs by intratracheal insufflation and compared to oral and intravenous delivery of rifampicin. Results&nbsp;&nbsp;The PNAP formulations possessed properties suitable for efficient deposition in the lungs. In vitro release showed an initial burst of rifampicin, with the remainder available for release beyond eight hours. PNAPs delivered to guinea pigs by insufflation achieved systemic levels of rifampicin detected for six to eight hours. Moreover, rifampicin concentrations remained detectable in lung tissue and cells up to and beyond eight hours. Conversely, after pulmonary delivery of an aerosol without nanoparticles, rifampicin could not be detected in the lungs at eight hours. Conclusions&nbsp;&nbsp;Our results indicate that rifampicin can be formulated into an aggregated nanoparticle form that, once delivered to animals, achieves systemic exposure and extends levels of drug in the lungs.},
  Type                     = {Journal Article},
  Url                      = {http://dx.doi.org/10.1007/s11095-009-9894-2}
}

@Book{Suri2005,
  Title                    = {Handbook of Biomedical Image Analysis: Volume II: Segmentation Models},
  Author                   = {Suri, J., Wilson, D., Laxminarayan, S.},
  Publisher                = {Springer},
  Year                     = {2005},

  Type                     = {Book}
}

@Article{Sussman1991,
  Title                    = {Asbestos fiber deposition in human tracheobronchial cast. I.- Experimental.},
  Author                   = {Sussman, R. G. and Cohen, B. S. and Lippmann, M. },
  Journal                  = {Inhalation Toxicology},
  Year                     = {1991},
  Pages                    = {145-160},
  Volume                   = {3},

  Type                     = {Journal Article}
}

@Article{Swift1991,
  Title                    = {Inspiratory inertial deposition of aerosols in human nasal airway replicate casts: Implication for the proposed NCRP lung model},
  Author                   = {Swift, D.L.},
  Journal                  = {Radiat. Prot. Dosim.},
  Year                     = {1991},
  Number                   = {29-34},
  Volume                   = {38},

  Type                     = {Journal Article}
}

@Article{Swift1981,
  Title                    = {Aerosol deposition and clearance in the human upper airways},
  Author                   = {Swift, David},
  Journal                  = {Annals of Biomedical Engineering},
  Year                     = {1981},
  Number                   = {5},
  Pages                    = {593-604},
  Volume                   = {9},

  Abstract                 = {Abstract&nbsp;&nbsp;The human upper respiratory tract, defined here as the airways above the trachea, is the portal of entry for airborne particles. Whether inhaled particles reach the bronchial or alveolated airways or are deposited in the upper airways (nasal or oral passage) depends upon a number of factors which are discussed in this paper. These factors can be divided into two groups: those which relate to the air flow properties and those which relate to the particles. If these factors are known to a sufficient degree, the regional deposition efficiency of the inhaled particles can be estimated. Particles once deposited in the upper airway are cleared by one of several mechanisms depending on the type of particle, site of deposition and functional state of the airway surface. The clearance processes and their importance for various particles are discussed.},
  Type                     = {Journal Article},
  Url                      = {http://dx.doi.org/10.1007/BF02364773}
}

@Article{Swift1996,
  Title                    = {The anterior human nasal passage as a fibrous filter for particles},
  Author                   = {Swift, D.L. and Kesavanathan, J.},
  Journal                  = {Chemical Engineering Communications},
  Year                     = {1996},
  Number                   = {1},
  Pages                    = {65-78},
  Volume                   = {151},

  Type                     = {Journal Article}
}

@Article{Swift1992,
  Title                    = {Inspiratory deposition of ultrafine particles in human nasal replicate cast},
  Author                   = {Swift, D.L. and Montassier, N. and Hopke, P.H. and Karpen-Hayes, K. and Cheng, Y.S. and Su, Y.F. and Yeh, H.C. and Strong, J.C.},
  Journal                  = {Journal of Aerosol Science},
  Year                     = {1992},
  Number                   = {1},
  Pages                    = {65-72},
  Volume                   = {23},

  Type                     = {Journal Article}
}

@InBook{Swift1977,
  Title                    = {Access of air to the respiratory tract.},
  Author                   = {Swift, D.L. and Proctor, D.F.},
  Editor                   = {Brain, J.D. and Proctor, D.F. and Reid, L.M.},
  Pages                    = {63-93},
  Publisher                = {Marcel Dekker},
  Year                     = {1977},

  Address                  = {New York, NY},
  Type                     = {Book Section},

  Booktitle                = {Respiratory Defence Mechanisms. }
}

@Article{Swift1996a,
  Title                    = {Nasal deposition of ultrafine218 Po Aerosols in human subjects.},
  Author                   = {Swift, D.L. and Strong, J.C.},
  Journal                  = {Journal of Aerosol Science},
  Year                     = {1996},
  Number                   = {7},
  Pages                    = {1125-1132},
  Volume                   = {27},

  Type                     = {Journal Article}
}

@Article{Swift1992a,
  Title                    = {Inspiratory deposition of ultrafine particles in human nasal replicate cast},
  Author                   = {Swift, David L. and Montassier, Nathalie and Hopke, Philip K. and Karpen-Hayes, Kim and Cheng, Yung-Sung and Su, Yin Fong and Yeh, Hsu Chi and Strong, John C.},
  Journal                  = {Journal of Aerosol Science},
  Year                     = {1992},
  Pages                    = {65 - 72},
  Volume                   = {23},

  Abstract                 = {The deposition of particles in replicate cast models of the human nasal cavity has been measured in three different laboratories for a range of particle sizes from 0.6 to 200 nm. The results of these measurements on four different casts can be fit by a single equation of the form η=1 −exp [−bQ−18 D12], where η is the fraction of particles deposited in the nasal cavity, Q is the volumetric flow rate (1 min−1), D is the particle diffusion coefficient (cm2s−1), and b is found to be 12.65 ± 0.17. The measurements were conducted over a range of flow rates from 1.4 to 28.71 min−1 (501 min−1 for sizes from 4.6 to 200 nm) using radon and thoron decay product aerosols as well as larger ultrafine particles. These results thus represent a current best estimate of the diffusional deposition of ultrafine particles in the human nasal cavity.},
  Doi                      = {http://dx.doi.org/10.1016/0021-8502(92)90318-P},
  ISSN                     = {0021-8502},
  Type                     = {Journal Article},
  Url                      = {http://ac.els-cdn.com/002185029290318P/1-s2.0-002185029290318P-main.pdf?_tid=a7a2a208-421f-11e4-bb57-00000aab0f02&acdnat=1411366625_b9c8fde919c953d4ab1ea98a81d30f08}
}

@Article{Swift1977a,
  Title                    = {Access of air to the respiratory tract},
  Author                   = {Swift, DAVID L and Proctor, DONALD F},
  Journal                  = {Respiratory defense mechanisms},
  Year                     = {1977},
  Number                   = {part 1},
  Pages                    = {63--93},
  Volume                   = {5},

  Publisher                = {Marcel Dekker New York}
}

@Article{Swiftc1996,
  Title                    = {Nasal Deposition of Ultrafine Particles in Human Volunteers and Its Relationship to Airway Geometry},
  Author                   = {Swiftc, Y. S. Chenga; H. C. Yeha; R. A. Guilmettea; S. Q. Simpsonb; K. H. Chengc; D. L.},
  Journal                  = {Aerosol Science and Technology},
  Year                     = {1996},

  Type                     = {Journal Article}
}

@PhdThesis{Sznitman2008,
  Title                    = {Respiratory Flows in the Pulmonary Acinus and Insights on the Control of Alveolar Flows},
  Author                   = {Sznitman, Josue},
  Year                     = {2008},
  Type                     = {Thesis},

  University               = {Swiss Federal Institute of Technology}
}

@InBook{Sznitman2007,
  Title                    = {CFD investigation of respiratory flows in a space-filling pumonary acinus model},
  Author                   = {Sznitman, J. and Schmuki, S. and Sutter, R. and Tsuda, A. and Gehr, P.},
  Editor                   = {Brebbia, C.A.},
  Pages                    = {147-156},
  Year                     = {2007},
  Type                     = {Book Section},
  Volume                   = {12},

  Booktitle                = {Modelling in Medicine and Biology VII, WIT Transactions on Biomedicine and Health}
}

@Article{TA¼rker2004,
  Title                    = {Nasal route and drug delivery systems},
  Author                   = {Türker, Selcan and Onur, Erten and Ózer, Yekta},
  Journal                  = {Pharmacy World \& Science},
  Year                     = {2004},
  Number                   = {3},
  Pages                    = {137-142},
  Volume                   = {26},

  Abstract                 = {Nasal drug administration has been used as an alternative route for the systemic availability of drugs restricted to intravenous administration. This is due to the large surface area, porous endothelial membrane, high total blood flow, the avoidance of first-pass metabolism, and ready accessibility. The nasal administration of drugs, including numerous compound, peptide and protein drugs, for systemic medication has been widely investigated in recent years. Drugs are cleared rapidly from the nasal cavity after intranasal administration, resulting in rapid systemic drug absorption. Several approaches are here discussed for increasing the residence time of drug formulations in the nasal cavity, resulting in improved nasal drug absorption. The article highlights the importance and advantages of the drug delivery systems applied via the nasal route, which have bioadhesive properties. Bioadhesive, or more appropriately, mucoadhesive systems have been prepared for both oral and peroral administration in the past. The nasal mucosa presents an ideal site for bioadhesive drug delivery systems. In this review we discuss the effects of microspheres and other bioadhesive drug delivery systems on nasal drug absorption. Drug delivery systems, such as microspheres, liposomes and gels have been demonstrated to have good bioadhesive characteristics and that swell easily when in contact with the nasal mucosa. These drug delivery systems have the ability to control the rate of drug clearance from the nasal cavity as well as protect the drug from enzymatic degradation in nasal secretions. The mechanisms and effectiveness of these drug delivery systems are described in order to guide the development of specific and effective therapies for the future development of peptide preparations and other drugs that otherwise should be administered parenterally. As a consequence, bioavailability and residence time of the drugs that are administered via the nasal route can be increased by bioadhesive drug delivery systems. Although the majority of this work involving the use of microspheres, liposomes and gels is limited to the delivery of macromolecules (e.g., insulin and growth hormone), the general principles involved could be applied to other drug candidates. It must be emphasized that many drugs can be absorbed well if the contact time between formulation and the nasal mucosa is optimized.},
  Type                     = {Journal Article},
  Url                      = {http://dx.doi.org/10.1023/B:PHAR.0000026823.82950.ff}
}

@Article{Takano2006,
  Title                    = {Inhaled Particle Deposition in Unsteady-State Respiratory Flow at a Numerically Constructed Model of the Human Larynx},
  Author                   = {Takano, H. and Nishida, N. and Itoh, M. and Hyo, N. and Majima, Y.},
  Journal                  = {Journal of Aerosol Medicine},
  Year                     = {2006},
  Number                   = {3},
  Pages                    = {314-328},
  Volume                   = {19},

  Type                     = {Journal Article}
}

@Article{Takeshita2000,
  Title                    = {Shear stress enhances glutathione peroxidase expression in endothelial cells},
  Author                   = {Takeshita, S. and Inoue, N. and Ueyama, T. and Kawashima, S. and Yokoyama, M.},
  Journal                  = {Biochemical and Biophysical Research Communications},
  Year                     = {2000},
  Note                     = {Cited By (since 1996): 42
Export Date: 6 June 2011
Source: Scopus},
  Number                   = {1},
  Pages                    = {66-71},
  Volume                   = {273},

  Type                     = {Journal Article},
  Url                      = {http://www.scopus.com/inward/record.url?eid=2-s2.0-0034709598&partnerID=40&md5=c64f2a7e2cb0c3d4b21bfd7777ab4264}
}

@Article{Talbot1980,
  Title                    = {Thermophoresis of particles in a heated boundary layer},
  Author                   = {Talbot, L. and Cheng, R.K. and Schefer, R.W. and Willis, D.R.},
  Journal                  = {Journal Fluid Mech},
  Year                     = {1980},
  Number                   = {4},
  Pages                    = {737-758},
  Volume                   = {101},

  Type                     = {Journal Article}
}

@Article{Tambasco2001,
  Title                    = {Calculating particle-to-wall distances in unstructured computational fluid dynamic models},
  Author                   = {Tambasco, Mauro and Steinman, David A.},
  Journal                  = {Applied Mathematical Modelling},
  Year                     = {2001},
  Number                   = {10},
  Pages                    = {803-814},
  Volume                   = {25},

  ISSN                     = {0307-904X},
  Keywords                 = {Particle deposition
Lagrangian particle tracking
Computation fluid dynamics
Finite element methods
Unstructured mesh generation
Boundary layer},
  Type                     = {Journal Article},
  Url                      = {http://www.sciencedirect.com/science/article/B6TYC-43VRXTM-1/2/7ccffd81a445d473335d0ea1dbf7e5a9}
}

@Article{Tan2009,
  Title                    = {Endoscopic Sinus Surgery in the Management of Nasal Obstruction},
  Author                   = {Tan, Bruce K. and Lane, Andrew P.},
  Journal                  = {Otolaryngologic Clinics of North America},
  Year                     = {2009},
  Note                     = {doi: DOI: 10.1016/j.otc.2009.01.012},
  Number                   = {2},
  Pages                    = {227-240},
  Volume                   = {42},

  Abstract                 = {Nasal obstruction is the leading symptom observed among patients who have chronic rhinosinusitis (CRS) with or without nasal polyposis. After failure of medical therapy, functional endoscopic sinus surgery (FESS) has emerged as the preferred treatment of CRS. Interestingly, although patient-reported outcomes show unequivocal relief of nasal obstruction after FESS, studies measuring nasal airflow and resistance demonstrate more modest improvements. This article provides an overview of the physiology of nasal airflow sensation, the burden of nasal obstruction in patients who have CRS, and the efficacy of FESS in addressing nasal obstruction in this population. Additionally, advances in airflow modeling that may enable improved preoperative planning for the relief of nasal obstruction after FESS are discussed.},
  ISSN                     = {0030-6665},
  Keywords                 = {Endoscopic sinus surgery
Nasal obstruction
Nasal congestion
Outcomes
Chronic rhinosinusitis},
  Type                     = {Journal Article},
  Url                      = {http://www.sciencedirect.com/science/article/B75JM-4VXW9CY-6/2/82c5dacc272af47d8656e84a9017e03f}
}

@Article{Tan2012,
  Title                    = {Numerical simulation of normal nasal cavity airflow in Chinese adult: a computational flow dynamics model},
  Author                   = {Tan, Jie and Han, Demin and Wang, Jie and Liu, Ting and Wang, Tong and Zang, Hongrui and Li, Yunchuan and Wang, Xiangdong},
  Journal                  = {European Archives of Oto-Rhino-Laryngology},
  Year                     = {2012},
  Number                   = {3},
  Pages                    = {881-889},
  Volume                   = {269},

  Doi                      = {10.1007/s00405-011-1771-z},
  ISSN                     = {0937-4477},
  Keywords                 = {Airflow; Nasal cavity; Computational fluid dynamics; Numerical simulation},
  Language                 = {English},
  Publisher                = {Springer-Verlag},
  Url                      = {http://dx.doi.org/10.1007/s00405-011-1771-z}
}

@Article{Tandon,
  Title                    = {Particle deposition from turbulent flow in a pipe},
  Author                   = {Tandon, Pushkar and Adewumi, Michael A.},
  Journal                  = {Journal of Aerosol Science},
  Note                     = {doi: DOI: 10.1016/S0021-8502(97)00039-6},
  Number                   = {1-2},
  Pages                    = {141-156},
  Volume                   = {29},

  ISSN                     = {0021-8502},
  Type                     = {Journal Article},
  Url                      = {http://www.sciencedirect.com/science/article/B6V6B-3SX6XPK-B/2/e71da37b0b78987761d5e936e8c0e423}
}

@Article{Tandon1998,
  Title                    = {Particle deposition from turbulent flow in a pipe},
  Author                   = {Tandon, Pushkar and Adewumi, Michael A.},
  Journal                  = {Journal of Aerosol Science},
  Year                     = {1998},
  Note                     = {doi: DOI: 10.1016/S0021-8502(97)00039-6},
  Number                   = {1-2},
  Pages                    = {141-156},
  Volume                   = {29},

  ISSN                     = {0021-8502},
  Type                     = {Journal Article},
  Url                      = {http://www.sciencedirect.com/science/article/B6V6B-3SX6XPK-B/2/e71da37b0b78987761d5e936e8c0e423}
}

@Article{Tang2011,
  Title                    = {Multi-physics MRI-based two-layer fluid-structure interaction anisotropic models of human right and left ventricles with different patch materials: Cardiac function assessment and mechanical stress analysis},
  Author                   = {Tang, Dalin and Yang, Chun and Geva, Tal and Gaudette, Glenn and del Nido, Pedro J.},
  Journal                  = {Computers \& Structures},
  Year                     = {2011},
  Number                   = {11-12},
  Pages                    = {1059-1068},
  Volume                   = {89},

  Abstract                 = {Multi-physics right and left ventricle (RV/LV) fluid-structure interaction (FSI) models were introduced to perform mechanical stress analysis and evaluate the effect of patch materials on RV function. The FSI models included three different patch materials (Dacron scaffold, treated pericardium, and contracting myocardium), two-layer construction, fiber orientation, and active anisotropic material properties. The models were constructed based on cardiac magnetic resonance (CMR) images acquired from a patient with severe RV dilatation and solved by ADINA. Our results indicate that the patch model with contracting myocardium leads to decreased stress level in the patch area, improved RV function and patch area contractility.},
  Doi                      = {10.1016/j.compstruc.2010.12.012},
  ISSN                     = {0045-7949},
  Keywords                 = {Right ventricle
Congenital heart disease
Heart model
Dacron scaffold patch
Fluid-structural interaction},
  Type                     = {Journal Article},
  Url                      = {http://www.sciencedirect.com/science/article/pii/S0045794910002956}
}

@InProceedings{Tanganelli,
  Title                    = {Distributions of Lipid and Raised Lesions in Aortas of Young People of Different Geographic Origins (WHO-ISFC PBDAY Study)},
  Author                   = {Tanganelli, P. },
  Pages                    = {1700-1710},
  Publisher                = {World Health Organization-International Society and Federation of Cardiology},
  Volume                   = {13},

  Type                     = {Conference Proceedings}
}

@Book{Tannehill1997,
  Title                    = {Computational fluid mechanics and heat transfer},
  Author                   = {Tannehill, J.C. and Anderson, D.A. and Pletcher, R.H.},
  Publisher                = {Taylor \& Francis},
  Year                     = {1997},

  ISBN                     = {9781560320463},
  Type                     = {Book},
  Url                      = {http://books.google.com.au/books?id=ZJPbtHeilCgC}
}

@Article{Tannen1993,
  Title                    = {Advanced Composite Materials},
  Author                   = {Tannen, K.},
  Journal                  = {Fire and Arson Investigator },
  Year                     = {1993},
  Number                   = {50-51},
  Volume                   = {1},

  Type                     = {Journal Article}
}

@Article{Tarabichi1993,
  Title                    = {Finite element analysis of airflow in the nasal valve},
  Author                   = {Tarabichi, M. and Fanous, N.},
  Journal                  = {Arch Otolaryngol Head Neck Surg},
  Year                     = {1993},
  Pages                    = {169-172},
  Volume                   = {119},

  Type                     = {Journal Article}
}

@Article{Tarbell2003,
  Title                    = {Mass Transport in Arteries and the Localization of Atherosclerosis},
  Author                   = {Tarbell, J. M.},
  Journal                  = {Annual Review of Biomedical Engineering},
  Year                     = {2003},
  Note                     = {Cited By (since 1996): 69
Export Date: 6 June 2011
Source: Scopus},
  Pages                    = {79-118},
  Volume                   = {5},

  Type                     = {Journal Article},
  Url                      = {http://www.scopus.com/inward/record.url?eid=2-s2.0-0642305420&partnerID=40&md5=a0ea6038fcba9ed92f567d7eed5d50d4}
}

@Article{Tarish1991,
  Title                    = {Ultrastructural localization of Dirofilaria immitis antigen in canine lung tissue using the protein A-gold labelling technique},
  Author                   = {Tarish, J. H. and Atwell, R. B.},
  Journal                  = {Veterinary Parasitology},
  Year                     = {1991},
  Number                   = {1},
  Pages                    = {23-31},
  Volume                   = {38},

  ISSN                     = {0304-4017},
  Type                     = {Journal Article},
  Url                      = {http://www.sciencedirect.com/science/article/B6TD7-476MNN5-4F/2/1e7bef7e50e84db76f5d4825b80b18a7}
}

@Article{Tavoularis1985,
  Title                    = {Effects of shear on the turbulent diffusivity tensor},
  Author                   = {Tavoularis, S. and Corrsin, S.},
  Journal                  = {International Journal of Heat and Mass Transfer},
  Year                     = {1985},
  Note                     = {Cited By (since 1996): 12
Export Date: 6 June 2011
Source: Scopus},
  Number                   = {1},
  Pages                    = {265-276},
  Volume                   = {28},

  Type                     = {Journal Article},
  Url                      = {http://www.scopus.com/inward/record.url?eid=2-s2.0-0021892363&partnerID=40&md5=01644a5a9c5ab99c45f00ac874cb1764}
}

@Article{Tavoularis1981,
  Title                    = {Experiments in nearly homogeneous turbulent shear flow with a uniform mean temperature gradient. Part 2. The fine structure},
  Author                   = {Tavoularis, Stavros and Corrsin, Stanley},
  Journal                  = {Journal of Fluid Mechanics},
  Year                     = {1981},
  Pages                    = {349-367},
  Volume                   = {104},

  Doi                      = {doi:10.1017/S0022112081002942},
  ISSN                     = {0022-1120},
  Type                     = {Journal Article},
  Url                      = {http://dx.doi.org/10.1017/S0022112081002942}
}

@Article{Tavoularis1981a,
  Title                    = {Experiments in nearly homogenous turbulent shear flow with a uniform mean temperature gradient. Part 1},
  Author                   = {Tavoularis, Stavros and Corrsin, Stanley},
  Journal                  = {Journal of Fluid Mechanics},
  Year                     = {1981},
  Pages                    = {311-347},
  Volume                   = {104},

  Doi                      = {doi:10.1017/S0022112081002930},
  ISSN                     = {0022-1120},
  Type                     = {Journal Article},
  Url                      = {http://dx.doi.org/10.1017/S0022112081002930}
}

@Article{Tavoularis1989,
  Title                    = {Further experiments on the evolution of turbulent stresses and scales in uniformly sheared turbulence},
  Author                   = {Tavoularis, S. and Karnik, U.},
  Journal                  = {Journal of Fluid Mechanics},
  Year                     = {1989},
  Pages                    = {457-478},
  Volume                   = {204},

  Doi                      = {doi:10.1017/S0022112089001837},
  ISSN                     = {0022-1120},
  Type                     = {Journal Article},
  Url                      = {http://dx.doi.org/10.1017/S0022112089001837}
}

@Article{Tawahai2004,
  Title                    = {Modeling Water Vapor and Heat Transfer in the Normal and the Intubated Airways},
  Author                   = {Tawahai, M.H. and Hunter, P.J.},
  Journal                  = {Annals of Biomedical Engineering},
  Year                     = {2004},
  Number                   = {4},
  Pages                    = {609–622},
  Volume                   = {32},

  Type                     = {Journal Article}
}

@Article{Tawhai2000,
  Title                    = {Generation of an Anatomically Based Three-Dimensional Model of the Conducting Airways},
  Author                   = {Tawhai, M.H. and Pullan, A.J. and Hunter, P.J.},
  Journal                  = {Annals Biomedical Engineering},
  Year                     = {2000},
  Pages                    = {793-802},
  Volume                   = {28},

  Type                     = {Journal Article}
}

@Article{Taylor2010,
  Title                    = {Inflow boundary profile prescription for numerical simulation of nasal airflow},
  Author                   = {Taylor, D. J. and Doorly, D. J. and Schroter, R. C.},
  Journal                  = {JOURNAL OF THE ROYAL SOCIETY INTERFACE},
  Year                     = {2010},
  Pages                    = {515-527},
  Volume                   = {7},

  Abstract                 = {Knowledge of how airflows through the nasal passages relies heavily on model studies, as the complexity and relative inaccessibility of the anatomy prevents detailed in vivo measurement. Almost all models to date fail to incorporate the geometry of the external nose, instead employing a truncated inflow. Typically, flow is specified to enter the model domain either directly at the nares (nostrils), or via an artificial pipe inflow tract attached to the nares. This study investigates the effect of the inflow geometry on flow predictions during steady nasal inspiration. Models that fully replicate the internal and external nasal airways of two anatomically distinct subjects are used as a reference to compare the effects of common in flow treatments on physiologically relevant quantities including regional wall shear stress and particle residence time distributions. In flow geometry truncation is found to affect flow predictions significantly, though slightly less so for the subject displaying more pronounced passage area contraction up to the internal nasal valve. For both subject geometries, a tapered pipe in flow provides a better approximation to the natural in flow than a blunt velocity pro. le applied to the nares. Computational modelling issues are also briefly outlined, by comparing quantities predicted using different surface tessellations, and by evaluation of domain-splitting techniques.},
  Doi                      = {10.1098/rsif.2009.0306},
  ISSN                     = {1742-5689},
  Keywords                 = {biomechanics
boundary conditions
computational fluid dynamics
inflow
nasal airflow
rhinology},
  Type                     = {Journal Article},
  Url                      = {http://www.ncbi.nlm.nih.gov/pmc/articles/PMC2842801/pdf/rsif20090306.pdf}
}

@Article{Taylor2006,
  Title                    = {Airflow in the human nasal cavity},
  Author                   = {Taylor, D. J. and Doorly, D. J. and Schroter, R. C.},
  Journal                  = {Journal of Biomechanics},
  Year                     = {2006},
  Number                   = {Supplement 1},
  Pages                    = {S272-S272},
  Volume                   = {39},

  ISSN                     = {0021-9290},
  Type                     = {Journal Article},
  Url                      = {http://www.sciencedirect.com/science/article/B6T82-4KR88PB-1G6/2/b1abd1b014ba864d96e73a191d4ff26d}
}

@Article{Taylor2010a,
  Title                    = { Inflow boundary profile prescription for numerical simulation of nasal airflow},
  Author                   = {Taylor, D. J., Doorly, D. J., Schroter, R. C.},
  Journal                  = {Journal of The Royal Society Interface},
  Year                     = {2010},
  Pages                    = {515-527},
  Volume                   = {7},

  Type                     = {Journal Article}
}

@PhdThesis{Tchen1947,
  Title                    = {Mean Value and Correlation Problems Connected with the Motion of Small Particles Suspended in a Turbulent Field},
  Author                   = {Tchen, C.M. },
  Year                     = {1947},
  Type                     = {Thesis},

  University               = {University of Delft}
}

@Article{Tenneti,
  Title                    = {Drag law for monodisperse gas-solid systems using particle-resolved direct numerical simulation of flow past fixed assemblies of spheres},
  Author                   = {Tenneti, S. and Garg, R. and Subramaniam, S.},
  Journal                  = {International Journal of Multiphase Flow},
  Volume                   = {In Press, Corrected Proof},

  Abstract                 = {Gas-solid momentum transfer is a fundamental problem that is characterized by the dependence of normalized average fluid-particle force F on solid volume fraction [phi] and the Reynolds number based on the mean slip velocity Rem. In this work we report particle-resolved direct numerical simulation (DNS) results of interphase momentum transfer in flow past fixed random assemblies of monodisperse spheres with finite fluid inertia using a continuum Navier-Stokes solver. This solver is based on a new formulation we refer to as the Particle-resolved Uncontaminated-fluid Reconcilable Immersed Boundary Method (PUReIBM). The principal advantage of this formulation is that the fluid stress at the particle surface is calculated directly from the flow solution (velocity and pressure fields), which when integrated over the surfaces of all particles yields the average fluid-particle force. We demonstrate that PUReIBM is a consistent numerical method to study gas-solid flow because it results in a force density on particle surfaces that is reconcilable with the averaged two-fluid theory. The numerical convergence and accuracy of PUReIBM are established through a comprehensive suite of validation tests. The normalized average fluid-particle force F is obtained as a function of solid volume fraction [phi] (0.1 [less-than-or-equals, slant] [phi] [less-than-or-equals, slant] 0.5) and mean flow Reynolds number Rem (0.01 [less-than-or-equals, slant] Rem [less-than-or-equals, slant] 300) for random assemblies of monodisperse spheres. These results extend previously reported results of (Hill et al., 2001a) and (Hill et al., 2001b) to a wider range of [phi], Rem, and are more accurate than those reported by Beetstra et al. (2007). Differences between the drag values obtained from PUReIBM and the drag correlation of Beetstra et al. (2007) are as high as 30% for Rem in the range 100-300. We take advantage of PUReIBM's ability to directly calculate the relative contributions of pressure and viscous stress to the total fluid-particle force, which is useful in developing drag correlations. Using a scaling argument, Hill et al. (2001b) proposed that the viscous contribution is independent of Rem but the pressure contribution is linear in Rem (for Rem > 50). However, from PUReIBM simulations we find that the viscous contribution is not independent of the mean flow Reynolds number, although the pressure contribution does indeed vary linearly with Rem in accord with the analysis of Hill et al. (2001b). An improved correlation for F in terms of [phi] and Rem is proposed that corrects the existing correlations in Rem range 100-300. Since this drag correlation has been inferred from simulations of fixed particle assemblies, it does not include the effect of mobility of the particles. However, the fixed-bed simulation approach is a good approximation for high Stokes number particles, which are encountered in most gas-solid flows. This improved drag correlation can be used in CFD simulations of fluidized beds that solve the average two-fluid equations where the accuracy of the drag law affects the prediction of overall flow behavior.},
  Doi                      = {10.1016/j.ijmultiphaseflow.2011.05.010},
  ISSN                     = {0301-9322},
  Keywords                 = {Drag law
Gas-solid flow
Particle-resolved direct numerical simulation
Immersed boundary method},
  Type                     = {Journal Article},
  Url                      = {http://www.sciencedirect.com/science/article/pii/S0301932211001170}
}

@Article{Teschke1999,
  Title                    = {Determinants of Exposure to Inhalable Particulate, Wood Dust, Resin Acids, and Monoterpenes in a Lumber Mill Environment},
  Author                   = {Teschke, Kay and Demers, Paul A. and Davies, Hugh W. and Kennedy, Susan M. and Marion, Stephen A. and Leung, Victor},
  Journal                  = {Annals of Occupational Hygiene},
  Year                     = {1999},
  Number                   = {4},
  Pages                    = {247-255},
  Volume                   = {43},

  Abstract                 = {In a lumber mill in the northern inland region of British Columbia, Canada, we measured inhalable particulate, resin acid, and monoterpene exposures, and estimated wood dust exposures. Potential determinants of exposure were documented concurrently, including weather conditions, tree species, wood conditions, jobs, tasks, equipment used, and certain control measures. Over 220 personal samples were taken for each contaminant. Geometric mean concentrations were 0.98mg/m3 for inhalable particulate, 0.49mg/m3 for estimated wood dust, 8.04μg/m3 for total resin acids, and 1.11mg/m3 for total monoterpenes. Multiple regression models for all contaminants indicated that spruce and pine produced higher exposures than alpine fir or mixed tree species, cleaning up sawdust increased exposures, and personnel enclosure was an effective means of reducing exposures. Sawing wood in the primary breakdown areas of the mill was the main contributor to monoterpene exposures, so exposures were highest for the barker operator, the head rig operator, the canter operator, the board edgers, and a roving utility worker in the sawmill, and lowest in the planer mills (after kiln drying of the lumber) and yard. Cleaning up sawdust, planing kiln-dried lumber, and driving mobile equipment in the yard substantially increased exposures to both inhalable particulate and estimated wood dust. Jobs at the front end of the sawmill where primary breakdown of the logs takes place had lower exposures. Resin acid exposures followed a similar pattern, except that yard driving jobs did not increase exposures.},
  Doi                      = {10.1093/annhyg/43.4.247},
  Type                     = {Journal Article},
  Url                      = {http://annhyg.oxfordjournals.org/content/43/4/247.abstract}
}

@Book{Thompson1985,
  Title                    = {Numerical Grid Generation – Foundations and Applications},
  Author                   = {Thompson, J.F. and Warsi, Z.U.A. and Mastin, C.W.},
  Publisher                = {Elsevier},
  Year                     = {1985},

  Address                  = {New York},

  Type                     = {Book}
}

@Article{Thompson2007,
  Title                    = {Commentary on "The role of the large airways on smooth muscle contraction in asthma"},
  Author                   = {Thompson, R. and King, G. and Harding, R.},
  Journal                  = {Journal of Applied Physiology},
  Year                     = {2007},
  Pages                    = {1465 - doi:10.1152/japplphysiol.00696.2007 },
  Volume                   = {103},

  Type                     = {Journal Article}
}

@Article{Thompson1991,
  Title                    = {A holistic approach to particle drag prediction},
  Author                   = {Thompson, T.L. and Clark, N.N.},
  Journal                  = {Powder Technology},
  Year                     = {1991},
  Pages                    = {57-66},
  Volume                   = {67},

  Type                     = {Journal Article}
}

@Article{Thomson1987,
  Title                    = {Criteria for the selection of stochastic models of particle trajectories in turbulent flows},
  Author                   = {Thomson, D. J.},
  Journal                  = {Journal of Fluid Mechanics},
  Year                     = {1987},
  Note                     = {Cited By (since 1996): 540
Export Date: 6 June 2011
Source: Scopus},
  Pages                    = {529-556},
  Volume                   = {180},

  Type                     = {Journal Article},
  Url                      = {http://www.scopus.com/inward/record.url?eid=2-s2.0-0023093311&partnerID=40&md5=7664786e7daf98342fb5d5ba65b520ab}
}

@Article{Thoroddsen2008,
  Title                    = {High-Speed Imaging of Drops and Bubbles},
  Author                   = {Thoroddsen, S.T. and Etoh, T.G. and Takehara, K.},
  Journal                  = {Annual Review of Fluid Mechanics},
  Year                     = {2008},
  Number                   = {1},
  Pages                    = {257-285},
  Volume                   = {40},

  Doi                      = {doi:10.1146/annurev.fluid.40.111406.102215},
  Type                     = {Journal Article},
  Url                      = {http://arjournals.annualreviews.org/doi/abs/10.1146/annurev.fluid.40.111406.102215}
}

@Article{Thrall2003,
  Title                    = {A Real-Time Method to Evaluate the Nasal Deposition and Clearance of Acetone in the Human Volunteer},
  Author                   = {Thrall, Karla D. and Schwartz, Ronald E. and Weitz, Karl K. and Soelberg, Jolen J. and Foureman, Gary L. and Prah, James D. and Timchalk, Charles},
  Journal                  = {Inhalation Toxicology},
  Year                     = {2003},
  Number                   = {6},
  Pages                    = {523-538},
  Volume                   = {15},

  ISSN                     = {0895-8378},
  Type                     = {Journal Article},
  Url                      = {http://www.informaworld.com/10.1080/08958370304470}
}

@Article{Tian2010,
  Title                    = {Development of a CFD Boundary Condition to Model Transient Vapor Absorption in the Respiratory Airways},
  Author                   = {Tian, Geng and Longest, P. Worth},
  Journal                  = {Journal of Biomechanical Engineering},
  Year                     = {2010},
  Number                   = {5},
  Pages                    = {051003-13},
  Volume                   = {132},

  Doi                      = {10.1115/1.4001045},
  Keywords                 = {biochemistry
biological fluid dynamics
biological tissues
blood
computational fluid dynamics
respiratory protection
transient analysis},
  Type                     = {Journal Article},
  Url                      = {http://link.aip.org/link/?JBY/132/051003/1}
}

@Article{Tian2007,
  Title                    = {Particle deposition in turbulent duct flow- Comparison of different model predictions},
  Author                   = {Tian, L. and Ahmadi, G.},
  Journal                  = {Journal Aerosol Science},
  Year                     = {2007},
  Pages                    = {377-397},
  Volume                   = {38},

  Type                     = {Journal Article}
}

@Misc{Tian2005,
  Title                    = {Particle Deposition in 3-D Asymmetric Human Lung Bifurcations},

  Author                   = {Tian, L and Ahmadi, G. and Mazaheri, A. and Hopke, P.K. and Cheng, S-Y.},
  Month                    = {June 12-15, 2005},
  Year                     = {2005},

  Type                     = {Conference Paper}
}

@Article{Tian2008,
  Title                    = {Numerical investigation into the effects of wall roughness on a gas-particle flow in a 90-degree bend},
  Author                   = {Tian, Z.F. and Inthavong, K. and Tu, J.Y. and Yeoh, G.H.},
  Journal                  = {International Journal of Heat and Mass Transfer},
  Year                     = {2008},
  Pages                    = {1238-1250},
  Volume                   = {51},

  Type                     = {Journal Article}
}

@Article{Tian2005a,
  Title                    = {Numerical simulation and validation of dilute gas-particle flow over a backward-facing step},
  Author                   = {Tian, Z.F. and Tu, J.Y. and Guan, Y.H.},
  Journal                  = {Aerosol Science and Technology},
  Year                     = {2005},
  Pages                    = {319-332},
  Volume                   = {39},

  Type                     = {Journal Article}
}

@Article{Tian2007a,
  Title                    = {Numerical studies of indoor airflow and particle dispersion by large eddy simulation},
  Author                   = {Tian, Z.F. and Tu, J.Y. and Yeoh, G.H. and Yuen, R.K.K},
  Journal                  = {Building and Environment},
  Year                     = {2007},
  Pages                    = {3483-3492},
  Volume                   = {42},

  Type                     = {Journal Article}
}

@Article{Tian2007b,
  Title                    = {Deposition of inhaled wood dust in the nasal cavity},
  Author                   = {Tian, Z. F. and Inthavong, K. and Tu, J. Y.},
  Journal                  = {Inhalation Toxicology},
  Year                     = {2007},
  Number                   = {14},
  Pages                    = {1155-1165},
  Volume                   = {19},

  Type                     = {Journal Article}
}

@Article{Tian2007c,
  Title                    = {Numerical simulation of gas-particle flows over an in-line tube bank},
  Author                   = {Tian, Z. F. and Inthavong, K. and Tu, J. Y. and Yeoh, G.H.},
  Journal                  = {ANZIAM, Journal},
  Year                     = {2007},
  Number                   = {CTAC2006},
  Pages                    = {C509-C526},
  Volume                   = {48},

  Type                     = {Journal Article}
}

@Article{Tian2006,
  Title                    = {On the numerical study of contaminant particle concentration in indoor airflow},
  Author                   = {Tian, Z. F. and Tu, J. Y. and Yeoh, G. H. and Yuen, R. K. K.},
  Journal                  = {Building and Environment},
  Year                     = {2006},
  Number                   = {11},
  Pages                    = {1504-1514},
  Volume                   = {41},

  Doi                      = {10.1016/j.buildenv.2005.06.006},
  ISSN                     = {0360-1323},
  Keywords                 = {CFD
Large eddy simulation
re-normalization group (RNG)
k–ε turbulence models
Indoor contaminant particles},
  Type                     = {Journal Article},
  Url                      = {http://www.sciencedirect.com/science/article/pii/S0360132305002350}
}

@Article{Timbrell1982,
  Title                    = {Deposition and retention of fibres in the human lung},
  Author                   = {Timbrell, V.},
  Journal                  = {Ann. occup. Hyg.},
  Year                     = {1982},
  Pages                    = {347-369},
  Volume                   = {26},

  Type                     = {Journal Article}
}

@Article{Tinley2014,
  Title                    = {Contaminants in human nail dust: an occupational hazard in podiatry?},
  Author                   = {Tinley, Paul D. and Eddy, Karen and Collier, Peter},
  Journal                  = {Journal of Foot and Ankle Research},
  Year                     = {2014},
  Note                     = {C:\Users\sean\AppData\Roaming\Zotero\Zotero\Profiles\16a4oype.default\zotero\storage\MSND6MAA\Tinley et al. - 2014 - Contaminants in human nail dust an occupational h.pdf},
  Pages                    = {15},
  Volume                   = {7},

  Abstract                 = {There has been limited literature indicating that podiatrists’ health may be at risk from exposure to human nail dust. Previous studies carried out in the UK have shown that large amounts of dust become airborne during the human nail drilling procedure and are present in the air up to 10 hours after a clinical session. This increases the risk of Respiratory Tract (RT) infection for the practitioner.
PMID: 24552311},
  Doi                      = {10.1186/1757-1146-7-15},
  ISSN                     = {1757-1146},
  Type                     = {Journal Article},
  Url                      = {http://www.ncbi.nlm.nih.gov/pmc/articles/PMC3937521/pdf/1757-1146-7-15.pdf}
}

@Article{Tippe1999,
  Title                    = {Experimental analysis of flow calculations based on HRCT imaging of individual bifurcations},
  Author                   = {Tippe, A. and Perzl, M. and Li, W. and Schulz, H.},
  Journal                  = {Respiration Physiology},
  Year                     = {1999},
  Number                   = {2-3},
  Pages                    = {181-191},
  Volume                   = {117},

  Abstract                 = {Flow simulations in airways and arteries allow the non-invasive study of transport and deposition processes in these vessel systems. Individual vessel geometries as input for such simulations are highly desirable. Computed tomography (CT) permits the acquisition of binary data to reconstruct such geometries. To prove the suitability of this reconstruction method, we compared measured with simulated velocities within model bifurcations. Particle image velocimetry was applied to measure flow velocities. Numerical simulations of these velocities were carried out by using the CT data of the same models as input to flow calculations (CFD). Within the resolution limits good agreement between measured and simulated velocities was found. For the smallest bifurcation (tube diameter: 2 mm) the agreement was less, indicating a methodical limitation by the actual resolution of the CT-scan technique. The study showed that a combination of CT and CFD can be considered as an appropriate step towards realistic simulations of particle transportation and deposition in individual geometries of the respiratory or cardiovascular systems.},
  ISSN                     = {0034-5687},
  Keywords                 = {Methods, particle image velocimetry
Upper airways, bifurcation, flow simulation},
  Type                     = {Journal Article},
  Url                      = {http://www.sciencedirect.com/science/article/B6T3J-3XJTT7J-C/2/c9c58478cfad0c7c6391e38170be2485}
}

@Article{Toffolo2005,
  Title                    = {On the theoretical link between design parameters and performance in cross-flow fans: a numerical and experimental study},
  Author                   = {Toffolo, Andrea},
  Journal                  = {Computers \& Fluids},
  Year                     = {2005},
  Number                   = {1},
  Pages                    = {49-66},
  Volume                   = {34},

  Abstract                 = {Cross-flow fan performance strictly depends on the complex configuration of the non-axisymmetrical flow field within the machine. The flow field, in turn, is deeply influenced by the design parameters of both casing and impeller geometry. In this paper, the relationship between the design parameters of the geometrical configuration and fan performance is discussed in a theoretical perspective, analyzing the features of the corresponding flow fields. These are reconstructed by a numerical study on cross-flow fan operation carried out for a representative set of configurations at different throttling conditions. Time-accurate solutions for a two-dimensional viscous and incompressible model of the fan using a sliding mesh technique are calculated with a commercial CFD code. The numerical results are validated with experimental data obtained from tests on performance and from local measurements of the flow field.},
  ISSN                     = {0045-7930},
  Type                     = {Journal Article},
  Url                      = {http://www.sciencedirect.com/science/article/B6V26-4CVX2GW-2/2/a8d5929579394eaa1476ce16dfebf616}
}

@Article{Tolman2009,
  Title                    = {Characterization and pharmacokinetic analysis of aerosolized aqueous voriconazole solution},
  Author                   = {Tolman, Justin A. and Nelson, Nicole A. and Son, Yoen Ju and Bosselmann, Stephanie and Wiederhold, Nathan P. and Peters, Jay I. and McConville, Jason T. and Williams Iii, Robert O.},
  Journal                  = {European Journal of Pharmaceutics and Biopharmaceutics},
  Year                     = {2009},
  Number                   = {1},
  Pages                    = {199-205},
  Volume                   = {72},

  Abstract                 = {Invasive fungal infections in immunocompromised patients have high mortality rates despite current treatment modalities. This study was designed to evaluate the suitability of an aqueous solution of voriconazole solubilized with sulfobutyl ether-[beta]-cyclodextrin for targeted drug delivery to the lungs via nebulization. A solution was prepared such that the inspired aerosol dose was isotonic with an acceptable mass median aerodynamic diameter of 2.98 [mu]m and a fine particle fraction of 71.7%. Following single and multiple inhaled doses, high voriconazole concentrations were observed within 30 min in the lung tissue and plasma. Drug solubilization with sulfobutyl ether-[beta]-cyclodextrin contributed to the rapid and high drug concentrations in plasma following inhalation. Maximal concentrations in the lung and plasma were 11.0 ± 1.6 [mu]g/g wet lung weight and 7.9 ± 0.68 [mu]g/mL, respectively, following a single inhaled dose with a corresponding tissue/plasma concentration ratio of 1.4 to 1. Following multiple inhaled doses, peak concentrations in lung tissue and plasma were 6.73 ± 3.64 [mu]g/g wet lung weight and 2.32 ± 1.52 [mu]g/mL, respectively. AUC values in lung tissue and plasma were also high. The clinically relevant observed pharmacokinetic parameters of inhaled aqueous solutions of voriconazole suggest that therapeutic outcomes could be benefitted through the use of inhaled voriconazole.},
  ISSN                     = {0939-6411},
  Keywords                 = {Voriconazole
Antifungal
Cyclodextrin
Pharmacokinetics
Single dose
Multiple dose
Inhaled
Pulmonary
Mouse
Animal
Isotonic},
  Type                     = {Journal Article},
  Url                      = {http://www.sciencedirect.com/science/article/B6T6C-4VCH6VD-2/2/017623b5ab9f1ac78f9c06ee1998e032}
}

@InBook{Tomenzoli2005,
  Title                    = {Physiology of the Nose and Paranasal Sinuses},
  Author                   = {Tomenzoli, Davide},
  Pages                    = {29-34},
  Year                     = {2005},
  Type                     = {Book Section},

  Url                      = {http://dx.doi.org/10.1007/3-540-26631-3_3}
}

@Article{Toschi2009,
  Title                    = {Lagrangian Properties of Particles in Turbulence},
  Author                   = {Toschi, Federico and Bodenschatz, Eberhard},
  Journal                  = {Annual Review of Fluid Mechanics},
  Year                     = {2009},
  Number                   = {1},
  Pages                    = {375-404},
  Volume                   = {41},

  Doi                      = {doi:10.1146/annurev.fluid.010908.165210},
  Type                     = {Journal Article},
  Url                      = {http://arjournals.annualreviews.org/doi/abs/10.1146/annurev.fluid.010908.165210}
}

@Article{Tran-Cong2004,
  Title                    = {Drag coefficients of irregularly shaped particles},
  Author                   = {Tran-Cong, S. and Gay, M. and Michaelides, E.E.},
  Journal                  = {Powder Technology},
  Year                     = {2004},
  Pages                    = {21-32},
  Volume                   = {139},

  Type                     = {Journal Article}
}

@Article{Travis2002,
  Title                    = {Non-neoplastic disorders of the lower respiratory tract.},
  Author                   = {Travis, W.E. and Colby, T.V. and Koss, M.N. and Rosado de Christenson, M.L. and Müller, N.L.},
  Journal                  = {Washington, DC: Armed Forces Institute of Pathology},
  Year                     = {2002},
  Pages                    = {381-471},

  Type                     = {Journal Article}
}

@Article{Treeck,
  Title                    = {Extension of a hybrid thermal LBE scheme for large-eddy simulations of turbulent convective flows},
  Author                   = {van Treeck, Christoph and Rank, Ernst and Krafczyk, Manfred and Tölke, Jonas and Nachtwey, Björn},
  Journal                  = {Computers \& Fluids},
  Number                   = {8-9},
  Pages                    = {863-871},
  Volume                   = {35},

  Abstract                 = {Following the work of Lallemand and Luo [Lallemand P, Luo L-S. Theory of the lattice Boltzmann method: acoustic and thermal properties in two and three dimensions. Phys Rev E 2003;68:036706] we validate, apply and extend the hybrid thermal lattice Boltzmann scheme (HTLBE) by a large-eddy approach to simulate turbulent convective flows. For the mass and momentum equations, a multiple-relaxation-time LBE scheme is used while the heat equation is solved numerically by a finite difference scheme. We extend the hybrid model by a Smagorinsky subgrid scale model for both the fluid flow and the heat flux. Validation studies are presented for laminar and turbulent natural convection in a cavity at various Rayleigh numbers up to 5 × 1010 for Pr = 0.71 using a serial code in 2D and a parallel code in 3D, respectively. Correlations of the Nusselt number are discussed and compared to benchmark data. As an application we simulated forced convection in a building with inner courtyard at Re = 50 000.},
  ISSN                     = {0045-7930},
  Type                     = {Journal Article},
  Url                      = {http://www.sciencedirect.com/science/article/B6V26-4K24NTW-2/2/c94eac0118aa47ec55a05abca5de52cf}
}

@Misc{Trinh1998,
  Title                    = {Acoustic Streaming in Microgravity: Flow Stability and Mass and Heat Transfer Enhancement},

  Author                   = {Trinh, E.H. },
  Month                    = {August 12-14},
  Year                     = {1998},

  Pages                    = {117-117},
  Type                     = {Conference Paper}
}

@Article{Tschirren2005,
  Title                    = {Segmentation and Quantitative Analysis of Intrathoracic Airway Trees from Computed Tomography Images},
  Author                   = {Tschirren, Juerg and Hoffman, Eric A. and McLennan, Geoffrey and Sonka, Milan},
  Journal                  = {Proc Am Thorac Soc},
  Year                     = {2005},
  Number                   = {6},
  Pages                    = {484-487},
  Volume                   = {2},

  Abstract                 = {The segmentation of the human airway tree from volumetric multidetector-row computed tomography images is an important prerequisite for many clinical applications and physiologic studies. We present a new airway segmentation method based on fuzzy connectivity. Small adaptive regions of interest are used that follow the airway branches as they are segmented. This method works on various types of scans (low dose and regular dose, normal subjects and diseased subjects) without the need for the user to manually adjust any parameters. Comparison with a commonly used region-growing segmentation algorithm shows that this method retrieves a significantly higher count of airway branches. In an additional processing step, this method provides accurate cross-sectional airway measurements that are conducted in the original gray-level volume. Validation on a phantom shows that subvoxel accuracy is achieved for all airway sizes and airway orientations. The utility of the reported method is demonstrated in a comparative analysis of normal and cystic fibrosis airway trees.},
  Doi                      = {10.1513/pats.200507-078DS},
  Type                     = {Journal Article},
  Url                      = {http://pats.atsjournals.org/cgi/content/abstract/2/6/484}
}

@Article{Tsuda2008,
  Title                    = {Finite element 3D reconstruction of the pulmonary acinus imaged by synchrotron X-ray tomography},
  Author                   = {Tsuda, A., Filipovic, N., Haberthur, D., Dickie, R., Matsui, Y., Stampanoni, M., Schittny, J. C.},
  Journal                  = {Journal Applied Physiology},
  Year                     = {2008},
  Pages                    = {964-976},
  Volume                   = {105},

  Type                     = {Journal Article}
}

@Article{Tsuda1995,
  Title                    = {Chaotic mixing of alveolated duct flow in rhythmically expanding pulmonary acinus},
  Author                   = {Tsuda, A., Henry, F. S., Butler, J. P.},
  Journal                  = {Journal Applied Physiology},
  Year                     = {1995},
  Pages                    = {1055-1063},
  Volume                   = {79},

  Type                     = {Journal Article}
}

@Misc{Tu2012,
  Title                    = {Particle inhalation and deposition in a human nasal cavity from the external surrounding environment},

  Author                   = {Tu, J.},
  Year                     = {2012},

  Keywords                 = {CFD
Facial features
Inlet velocity profiles
Nasal cavity
Particle deposition},
  Publisher                = {Pergamon},
  Type                     = {Generic}
}

@Article{Tu2000,
  Title                    = {Numerical investigation of particle flow behavior in particle-wall function},
  Author                   = {Tu, J.Y.},
  Journal                  = { Aerosol Science and Technology},
  Year                     = {2000},
  Pages                    = {509-526},
  Volume                   = {32},

  Type                     = {Journal Article}
}

@InProceedings{Tu,
  Title                    = {CFD simulation of air/particle flow in the human nasal cavity. },
  Author                   = {Tu, J.Y. and Abu-Hijleh, B. and Xue, C. and Li, C.G.},
  Booktitle                = {Proceedings of 5th International Conference on Multiphase Flow},

  Type                     = {Conference Proceedings}
}

@Article{Tu1992,
  Title                    = {Overlapping Grids and Multigrid Methods for Three-Dimensional Unsteady Flow Calculations in IC Engines},
  Author                   = {Tu, J.Y. and Fuchs, L.},
  Journal                  = {International Journal Numerical Methods Fluids},
  Year                     = {1992},
  Pages                    = {693-714},
  Volume                   = {15},

  Type                     = {Journal Article}
}

@Book{Tu2012a,
  Title                    = {Computational Fluid Particle Dynamics in the Human Respiratory System},
  Author                   = {Tu, J.Y. and Inthavong, K. and Ahmadi, G.},
  Publisher                = {Springer},
  Year                     = {2012},

  Address                  = {Berlin, Heidelberg},
  Series                   = {Springer Series: Biological and Medical Physics, Biomedical Engineering},

  Type                     = {Book}
}

@Book{Tu2008,
  Title                    = {Computational fluid dynamics: a practical approach},
  Author                   = {Tu, J. and Yeoh, G.H. and Liu, C.},
  Publisher                = {Butterworth-Heinemann},
  Year                     = {2008},

  ISBN                     = {9780750685634},
  Type                     = {Book},
  Url                      = {http://books.google.com.au/books?id=BJwuMQOhM0sC}
}

@InBook{Tu2008a,
  Title                    = {Some Applications of CFD with Examples},
  Author                   = {Tu, Jiyuan and Yeoh, Guan Heng and Liu, Chaoqun},
  Pages                    = {277-363},
  Publisher                = {Butterworth-Heinemann},
  Year                     = {2008},

  Address                  = {Burlington},
  Type                     = {Book Section},

  Abstract                 = {Summary The increasing demand of CFD in resolving numerous fluid-flow problems is growing within the scientific community. Needless to say, such an accomplishment could only have been made possible through the meticulous developments by persistent researchers and more recently, the extension to industrial usage by dedicated code developers that have resulted in the availability of a number of commercial CFD packages. The cornerstone of any CFD analysis lies in the heart of transport equations and the building blocks of efficient numerical techniques. Throughout this book, the authors have continually stressed the importance of grasping the essential conservation equations as described in Chapter 3 and the basic understanding of numerical approximations considered in Chapters 4 and 5. In Chapter 6, the authors further provided useful guidelines of handling practical flow problems such as the requirement of suitable turbulence models to resolve real fluid-flow processes, which incidentally exemplifies the range of computed results presented in Chapter 1.},
  Booktitle                = {Computational Fluid Dynamics},
  ISBN                     = {978-0-75-068563-4},
  Url                      = {http://www.sciencedirect.com/science/article/B8KV4-4S9261R-6/2/1c8f2e02e9a41ff9ecf1f14e1fb27b9e}
}

@Article{Tuliszka-Sznitko2009,
  Title                    = {LES of the non-isothermal transitional flow in rotating cavity},
  Author                   = {Tuliszka-Sznitko, E. and Zielinski, A. and Majchrowski, W.},
  Journal                  = {International Journal of Heat and Fluid Flow},
  Year                     = {2009},
  Number                   = {3},
  Pages                    = {534-548},
  Volume                   = {30},

  Abstract                 = {The paper presents the 3D LES study of the non-isothermal transitional and turbulent flow in rotor/stator sealed cavity with a rotating inner and a stationary outer cylinder. The stator and the outer cylinder are warmer than the rotor and the inner cylinder. Computations have been performed for the cavity of aspect ratio L = (R1 - R0)/2h = 5, curvature parameter Rm = (R1 + R0)/(R1 - R0) = 3, Reynolds numbers , thermal Rossby numbers B = [beta](T2 - T1) = 0.01-0.4, and for the cavity (L = 9, Rm = 1.5, Re = 100,000-150,000, B = 0.01-0.4). Computations we based on the efficient pseudo-spectral Chebyshev-Fourier method (Serre, E., Pulicani, J.P., 2001. A three-dimensional pseudospectral method for rotating flows in a cylinder. Computers and Fluids, 30, 491). In Large Eddy Simulations we used a version of the dynamic Smagorinsky eddy viscosity model proposed by Meneveau (Meneveau, C., Lund, T.S., Cabot, W.H., 1996, A Lagrangian dynamic subgrid-scale model of turbulence. Journal of Fluid Mechanics, 319, 353-385), in which the averaging is performed over the particle pathline. This approach allowed us to perform computations for higher Reynolds numbers for confined rotating flows, which are strongly inhomogeneous and anisotropic. Results showed that the turbulence is concentrated mostly in the stator boundary layer with a maximum at the junction between the stator and the outer cylinder. For the cavity of aspect ratio L = 5 the stator boundary layer was fully turbulent for Re [greater-or-equal, slanted] 100,000, whereas the rotor boundary layer was still laminar. The influence of the thermal Rossby number on the flow structure and the basic state was not significant. The correlation formulas, for predicting the distribution of local Nusselt numbers along disks as well as the correlation formulas for the averaged Nusselt numbers depending on the Reynolds numbers are given on the basis of our results.},
  ISSN                     = {0142-727X},
  Keywords                 = {Rotating cavity
LES
Heat transfer
Turbulent flow
Laminar-turbulent transition},
  Type                     = {Journal Article},
  Url                      = {http://www.sciencedirect.com/science/article/B6V3G-4VY16CB-3/2/bfd69e08fcb624d5e5f9b66863a3ed19}
}

@Book{Twomey1976,
  Title                    = {Atmospheric Aerosols},
  Author                   = {Twomey, S. },
  Publisher                = {Elsevier},
  Year                     = {1976},

  Address                  = {Amsterdam},

  Type                     = {Book}
}

@Article{Udaykumar2006,
  Title                    = {Computation of peristaltic transport and mixing in the small intestine},
  Author                   = {Udaykumar, H. S. and Krishman, S. and Dillard, S. and Marshall, J. S. and Schulze, K.},
  Journal                  = {Journal of Biomechanics},
  Year                     = {2006},
  Number                   = {Supplement 1},
  Pages                    = {S442-S443},
  Volume                   = {39},

  ISSN                     = {0021-9290},
  Type                     = {Journal Article},
  Url                      = {http://www.sciencedirect.com/science/article/B6T82-4KR88PB-2FB/2/74db682413d79e6008f8445c3800ba3a}
}

@Article{Ultman2006,
  Title                    = {Simulation of reactive gas uptake distributions into airway bifurcations},
  Author                   = {Ultman, J. S. and Taylor, A. B. and Borhan, A.},
  Journal                  = {Journal of Biomechanics},
  Year                     = {2006},
  Number                   = {Supplement 1},
  Pages                    = {S264-S265},
  Volume                   = {39},

  ISSN                     = {0021-9290},
  Type                     = {Journal Article},
  Url                      = {http://www.sciencedirect.com/science/article/B6T82-4KR88PB-1DX/2/a34367dc4b6e98ea19adc8210a697279}
}

@Book{Umbaugh2005,
  Title                    = {Computer imaging: digital image analysis and processing},
  Author                   = {Umbaugh, S.},
  Publisher                = {Taylor \& Francis.},
  Year                     = {2005},

  Type                     = {Book}
}

@Article{UpchurchJr2001,
  Title                    = {Nitric oxide inhibition increases matrix metalloproteinase-9 expression by rat aortic smooth muscle cells in vitro},
  Author                   = {Upchurch Jr, G. R. and Ford, J. W. and Weiss, S. J. and Knipp, B. S. and Peterson, D. A. and Thompson, R. W. and Eagleton, M. J. and Broady, A. J. and Proctor, M. C. and Stanley, J. C.},
  Journal                  = {Journal of Vascular Surgery},
  Year                     = {2001},
  Note                     = {Cited By (since 1996): 51
Export Date: 6 June 2011
Source: Scopus},
  Number                   = {1},
  Pages                    = {76-83},
  Volume                   = {34},

  Type                     = {Journal Article},
  Url                      = {http://www.scopus.com/inward/record.url?eid=2-s2.0-0035406154&partnerID=40&md5=e7aee9a06d3f5bbc7900a135e0265411}
}

@Article{VanDyke,
  Title                    = {An Album of Fluid Motion,� Parabolic Press, P.O. Box 3032, Stanford, CA 94305-0030, 1982.},
  Author                   = {Van Dyke, Milton, “},

  Type                     = {Journal Article}
}

@Article{Vandoormaal1984,
  Title                    = {Enhancements of the SIMPLE method for predicting incompressible fluid flows},
  Author                   = {Vandoormaal, J.P. and Raithby, G.D.},
  Journal                  = {Numer. Heat Transfer},
  Year                     = {1984},
  Number                   = {147-163},
  Volume                   = {7},

  Type                     = {Journal Article}
}

@Article{Vanpeperstraete1974,
  Title                    = {The cartilaginous skeleton of the bronchial tree},
  Author                   = {Vanpeperstraete, F.},
  Journal                  = {Advances in Anatomical Embryology Cell Biology},
  Year                     = {1974},
  Pages                    = {1-15},
  Volume                   = {48},

  Type                     = {Journal Article}
}

@Book{Vargaftik1975,
  Title                    = {Tables on Thermophysical Properties of Liquids and Gases},
  Author                   = {Vargaftik, N.B.},
  Publisher                = {Hemisphere},
  Year                     = {1975},

  Address                  = { Washington, DC},
  Edition                  = {2nd},

  Type                     = {Book}
}

@Article{Varghese2003,
  Title                    = {Numerical modeling of pulsatile turbulent flow in stenotic vessels},
  Author                   = {Varghese, S.S. and Frankel, S.H.},
  Journal                  = {ASME Journal Biomech. Eng.},
  Year                     = {2003},
  Pages                    = {445-460},
  Volume                   = {123},

  Type                     = {Journal Article}
}

@Article{Venegas2005,
  Title                    = {Self-organized patchiness in asthma as a prelude to catastrophic shifts},
  Author                   = {Venegas, J.G. and Winkler, T. and Musch, G. and F., Vidal Melo. and Layfield, D. and Tgavalekos, N and Fischman, A.J. and Callahan, R.J. and Bellani, G. and Harris, R.S.},
  Journal                  = {Nature},
  Year                     = {2005},
  Pages                    = {777-782},
  Volume                   = {434},

  Type                     = {Journal Article}
}

@Article{Verhulst2008,
  Title                    = {The prevalence, anatomical correlates and treatment of sleep-disordered breathing in obese children and adolescents},
  Author                   = {Verhulst, Stijn L. and Van Gaal, Luc and De Backer, Wilfried and Desager, Kristine},
  Journal                  = {Sleep Medicine Reviews},
  Year                     = {2008},
  Number                   = {5},
  Pages                    = {339-346},
  Volume                   = {12},

  Abstract                 = {Summary The prevalence of childhood obesity is increasing worldwide. One of the obesity-related complications that has received increasing attention in recent years is sleep-disordered breathing. Obese children are at a higher risk of developing sleep-disordered breathing, including habitual snoring, obstructive sleep apnea syndrome and desaturations preceded by central apneas. Both adiposity and upper airway factors, such as adenotonsillar hypertrophy, modulate the severity of sleep-disordered breathing in these children. Adenotonsillectomy seems to be effective against obstructive sleep apnea syndrome in obese children. On the other hand, there are limited data on the effects of weight loss and of treatment with continuous positive airway pressure on the severity of sleep apnea in obese children and adolescents.},
  ISSN                     = {1087-0792},
  Keywords                 = {Adenotonsillectomy
Central sleep apnea
Childhood
Continuous positive airway pressure
Habitual snoring
Obesity
Obstructive sleep apnea
Sleep-disordered breathing
Weight loss},
  Type                     = {Journal Article},
  Url                      = {http://www.sciencedirect.com/science/article/B6WX7-4S862BK-1/2/c26bc4e5b081c71763817fb0f1857537}
}

@Article{Verney2011,
  Title                    = {Behaviour of a floc population during a tidal cycle: Laboratory experiments and numerical modelling},
  Author                   = {Verney, Romaric and Lafite, Robert and Claude Brun-Cottan, Jean and Le Hir, Pierre},
  Journal                  = {Continental Shelf Research},
  Year                     = {2011},
  Number                   = {10, Supplement 1},
  Pages                    = {S64-S83},
  Volume                   = {31},

  Abstract                 = {An approach combining laboratory experiments and numerical modelling was used to investigate the behaviour of a floc population during an idealized tidal cycle. The experiment was conducted on suspended sediments at a concentration of 93 mg l-1 collected in the field. It was based on a jar test device to reproduce tidal-induced turbulence and coupled with a CCD camera system and image post-processing software to monitor floc size distribution. At the same time, a 0D size-class based aggregation/fragmentation model (FLOCMOD) was developed to simulate changes in the floc population over the tidal cycle. Experimental results revealed strong variability of the behaviour of microfloc and macrofloc populations with respect to the varying shear rates observed in situ. In particular, the major dependency of floc sizes on the Kolmogorov microscale was confirmed. Time-scale differences were also observed for aggregation and fragmentation processes which led to asymmetrical floc behaviour despite symmetrical tidal forcing. Model results, i.e. average diameter, maximum diameter and floc size distribution, were in good agreement with experimental data with an RMS error between observed and computed average diameters of below 25 [mu]m over the tidal cycle. FLOCMOD was optimized in terms of the time step, number of size classes and size range: only seven classes ranging from 50 to 643 [mu]m associated with a dynamically-adaptable time step were needed to correctly reproduce experimental results, characterized by an RMS error of less than 5 [mu]m with respect to the reference case (100 classes from 4 to 1500 [mu]m). Sensitivity analyses were performed on major parameters or processes: initial floc size distribution, primary particle size, fractal dimension and fragmentation function (binary, ternary, erosion or collision-induced fragmentation). Results showed that initial floc size distribution played a role only during the first aggregation stage. Low variability of the fractal dimension did not significantly modify model results, while larger differences were observed when the primary particle size was changed, especially towards the largest sizes (10 [mu]m). Nevertheless, these two structural parameters had a strong impact on the calculated mean settling velocity with differences of 0.2 mm s-1 compared with the reference case. Different fragmentation functions were shown to significantly modify model results, except for collision-induced shear stress. In particular, combining floc erosion with binary breakup in the shear fragmentation term enabled us to reproduce bimodal distributions, patterns that are typically observed in situ.},
  Doi                      = {10.1016/j.csr.2010.02.005},
  ISSN                     = {0278-4343},
  Keywords                 = {Flocculation
Tidal cycle
Laboratory experiment
FLOCMOD
0D model
Sensitivity analysis
Optimization},
  Type                     = {Journal Article},
  Url                      = {http://www.sciencedirect.com/science/article/pii/S0278434310000415}
}

@Book{Versteeg2007,
  Title                    = {An introduction to computational fluid dynamics: the finite volume method},
  Author                   = {Versteeg, Henk Kaarle and Malalasekera, Weeratunge},
  Publisher                = {Pearson Education Ltd},
  Year                     = {2007},
  Edition                  = {2nd},

  Type                     = {Book}
}

@Article{Vial2005,
  Title                    = {Airflow modeling of steady inspiration in two realistic proximal airway trees reconstructed from human thoracic tomodensitometric images},
  Author                   = {Vial, L. and Perchet, D. and Fodil, R. and Caillibotte, G. and Fetita, C. and Preteux, F. and Beigelman-Aubry, C. and Grenier, P. and Thiriet, M. and Isabey, D. and Sbirlea-Apiou, S.},
  Journal                  = {Comput. Method Biomech. Biomed Eng.},
  Year                     = {2005},
  Pages                    = {267-277},
  Volume                   = {8},

  Type                     = {Journal Article}
}

@Article{Viceconti1999,
  Title                    = {CT data sets surface extraction for biomechanical modeling of long bones},
  Author                   = {Viceconti, M. and Zannoni, C. and Testi, D. and Capello, A.},
  Journal                  = {Comput Methods Programs Biomed},
  Year                     = {1999},
  Pages                    = {159-166},
  Volume                   = {59},

  Type                     = {Journal Article}
}

@Article{Vig1998,
  Title                    = {Nasal obstruction and facial growth: The strength of evidence for clinical assumptions},
  Author                   = {Vig, Katherine W. L.},
  Journal                  = {American Journal of Orthodontics and Dentofacial Orthopedics},
  Year                     = {1998},
  Number                   = {6},
  Pages                    = {603-611},
  Volume                   = {113},

  ISSN                     = {0889-5406},
  Type                     = {Journal Article},
  Url                      = {http://www.sciencedirect.com/science/article/B6W9R-4HJYV42-2/2/4e38a1489341ae235279222eb1d53d27}
}

@Article{Vignola2000,
  Title                    = {Tissue remodeling as a feature of persistent asthma.},
  Author                   = {Vignola, A.M. and Kips, J. and Bousquet, J. },
  Journal                  = {J Allergy Clin Immunol},
  Year                     = {2000},
  Pages                    = {1041-53},
  Volume                   = {105},

  Type                     = {Journal Article}
}

@Article{Vigouroux2005,
  Title                    = {A stimulation method using odors suitable for PET and fMRI studies with recording of physiological and behavioral signals},
  Author                   = {Vigouroux, M. and Bertrand, B. and Farget, V. and Plailly, J. and Royet, J. P.},
  Journal                  = {Journal of Neuroscience Methods},
  Year                     = {2005},
  Number                   = {1},
  Pages                    = {35-44},
  Volume                   = {142},

  Abstract                 = {A design for a semi-automatic olfactometric system is described for PET and fMRI experiments. The olfactometer presents several advantages because it enables the use of an [`]infinite' number of odorants and the synchronization of stimuli with breathing. These advantages mean that the subject is recorded while breathing normally during olfactory judgment tasks. In addition, the design includes a system for recording the behavioral (rating scale) and physiological (breathing, electrodermal reaction (ED), plethysmography (PL)) signals given by the subject. Both systems present the advantage of being compatible with fMRI magnetic fields since no ferrous material is used in the Faraday cage and signals are transmitted via an optical transmission interface to an acquisition system.},
  ISSN                     = {0165-0270},
  Keywords                 = {Olfactometric system
PET
fMRI
Breathing
Electrodermal reaction
Plethysmography},
  Type                     = {Journal Article},
  Url                      = {http://www.sciencedirect.com/science/article/B6T04-4D9R9YS-1/2/6f963596cf62cbd03474a05d3d1ba244}
}

@Book{Vincent1995,
  Title                    = {Aerosol Science for Industrial Hygienists},
  Author                   = {Vincent, J.H. },
  Publisher                = {Pergamon Press},
  Year                     = {1995},

  Address                  = {Oxford, UK},

  Type                     = {Book}
}

@Article{Vincent1990,
  Title                    = {Aerosol inhalability at higher windspeeds},
  Author                   = {Vincent, J.H. and Mark, D. and Miller, B.G. and Armbruster, L. and Ogden, T.L.},
  Journal                  = {Journal of Aerosol Science},
  Year                     = {1990},
  Number                   = {4},
  Pages                    = {577-586},
  Volume                   = {21},

  Type                     = {Journal Article}
}

@Article{Vinchurkar2008,
  Title                    = {Evaluation of hexahedral, prismatic and hybrid mesh styles for simulating respiratory aerosol dynamics},
  Author                   = {Vinchurkar, Samir and Longest, P. Worth},
  Journal                  = {Computers \& Fluids},
  Year                     = {2008},
  Number                   = {3},
  Pages                    = {317-331},
  Volume                   = {37},

  Abstract                 = {In simulating biofluid flow domains, structured hexahedral meshes are often associated with high quality solutions. However, extensive time and effort are required to generate these meshes for complex branching geometries. This study evaluates potential mesh configurations that may maintain the advantages of the structured hexahedral style while providing significant savings in grid construction time and complexity. Specifically, the objective of this study is to evaluate the performance of unstructured hexahedral, prismatic and hybrid meshes based on grid convergence and local particle deposition fractions in a bifurcating model of the respiratory tract. A grid convergence index (GCI) has been implemented to assess the mesh-independence of solutions in cases where true grid halving is not feasible. Localized and total deposition values have been evaluated for particles ranging from 1 through 10 [mu]m in planar and out-of-plane geometries. Structured hexahedral, unstructured hexahedral and prismatic meshes were found to provide GCI values of approximately 5% and nearly identical velocity fields. In contrast, the hexahedral-tetrahedral hybrid model resulted in GCI values that were significantly higher in comparison to the other meshes. The resulting velocity field for the hybrid configuration differed from the hexahedral and prismatic solutions by up to an order of magnitude at some locations. Considering the deposition of 10 [mu]m particles in the planar configuration, all meshes considered provided relatively close agreement (2-20% difference) with an available experimental study. For all particle sizes considered, local and total deposition results for the structured and unstructured hexahedral meshes were similar. In contrast, the prismatic and hybrid geometries resulted in significantly higher deposition rates when compared to the hexahedral meshes for particles less than 10 [mu]m. As a result, only the unstructured hexahedral mesh was found to provide overall performance similar to the structured hexahedral configuration with the advantage of a significant savings in construction time. These results emphasize the importance of aligning control volume gridlines with the predominant flow direction in biofluid applications that involve long and thin internal flow domains. Future studies are needed to assess other forms of the hybrid configuration and the effects of other element styles.},
  ISSN                     = {0045-7930},
  Type                     = {Journal Article},
  Url                      = {http://www.sciencedirect.com/science/article/B6V26-4P8B0N8-1/2/1fc529a71bc0c12d5a66b2d8df2024d6}
}

@Article{Vinchurkar2009,
  Title                    = {CFD simulations of the Andersen cascade impactor: Model development and effects of aerosol charge},
  Author                   = {Vinchurkar, Samir and Longest, P. Worth and Peart, Joanne},
  Journal                  = {Journal of Aerosol Science},
  Year                     = {2009},
  Number                   = {9},
  Pages                    = {807-822},
  Volume                   = {40},

  Abstract                 = {Cascade impactors are commonly used to assess the size characteristics of aerosols in toxicology and pharmaceutical applications. These aerosol instruments have been developed and refined over decades. However, a number of questions remain related to impactor performance, including the influence of electrostatic charge on measured size distributions. The objective of this study was to develop a validated CFD model of the Mark II Andersen cascade impactor (ACI) and apply this model to evaluate the effects of particle charge on deposition. The flow field was simulated using a commercial CFD code for incompressible laminar and transitional flows. Particle trajectories and deposition were evaluated using a well tested Lagrangian tracking approach that accounts for impaction, sedimentation, diffusion, and electrostatic attraction. Particle charge levels typical of dry powder inhaler (DPI) and metered dose inhaler (MDI) aerosols were considered for a particle size range of 0.3-12 [mu]m. As a model validation, computational predictions of cutoff d50 diameters for each of the eight ACI stages were found to be within 10% difference of existing experimental and manufacturer data. Results indicated that charges consistent with DPI and MDI aerosols increased deposition fraction in Stages 0-3 by up to 30% and increased deposition fraction in Stages 4-7 by up to an order of magnitude. For Stages 0-3, both DPI and MDI charges reduced the d50 value by approximately 10% or less. In contrast, charged aerosols reduced the d50 values in Stages 4 and 5 by 200% and 60%, respectively. All charged submicrometer aerosols considered deposited in Stages 6 and 7. Increasing the particle charge by an order of magnitude from DPI to MDI values had a relatively small effect on further decreasing the cutoff size of each stage. In conclusion, these results can be used to approximate the actual aerodynamic diameter of a charged pharmaceutical aerosol based on measurements in a standard ACI. Future applications of the developed ACI model include evaluating the influence of space charge on deposition and quantifying the effects of aerosol condensation and evaporation on size assessment.},
  ISSN                     = {0021-8502},
  Keywords                 = {Andersen cascade impactor
Computational fluid dynamics
Image force
Electrostatic charge
Aerosol size assessment
Metered dose inhalers
Dry powder inhalers
Aerosol characterization},
  Type                     = {Journal Article},
  Url                      = {http://www.sciencedirect.com/science/article/B6V6B-4WDNKTT-1/2/084d98cfb8b34f66ebef667a5a40d7ea}
}

@Article{Vink2000,
  Title                    = {Capillary endothelial surface layer selectively reduces plasma solute distribution volume},
  Author                   = {Vink, H. and Duling, B. R.},
  Journal                  = {American Journal of Physiology - Heart and Circulatory Physiology},
  Year                     = {2000},
  Note                     = {Cited By (since 1996): 118
Export Date: 6 June 2011
Source: Scopus},
  Number                   = {1 47-1},
  Pages                    = {H285-H289},
  Volume                   = {278},

  Type                     = {Journal Article},
  Url                      = {http://www.scopus.com/inward/record.url?eid=2-s2.0-0033976142&partnerID=40&md5=d5f681469dcf08696a7e754dd85d4918}
}

@Article{Vink1996,
  Title                    = {Identification of distinct luminal domains for macromolecules, erythrocytes, and leukocytes within mammalian capillaries},
  Author                   = {Vink, H. and Duling, B. R.},
  Journal                  = {Circulation Research},
  Year                     = {1996},
  Note                     = {Cited By (since 1996): 222
Export Date: 6 June 2011
Source: Scopus},
  Number                   = {3},
  Pages                    = {581-589},
  Volume                   = {79},

  Type                     = {Journal Article},
  Url                      = {http://www.scopus.com/inward/record.url?eid=2-s2.0-0029797605&partnerID=40&md5=978f415be8c6a35889761cdf89c17879}
}

@Article{Virchow2008,
  Title                    = {Importance of inhaler devices in the management of airway disease},
  Author                   = {Virchow, J. C. and Crompton, G. K. and Dal Negro, R. and Pedersen, S. and Magnan, A. and Seidenberg, J. and Barnes, P. J.},
  Journal                  = {Respiratory Medicine},
  Year                     = {2008},
  Number                   = {1},
  Pages                    = {10-19},
  Volume                   = {102},

  ISSN                     = {0954-6111},
  Keywords                 = {Asthma
COPD
Management
Inhalation
Devices
Inhaler technique},
  Type                     = {Journal Article},
  Url                      = {http://www.sciencedirect.com/science/article/B6WWS-4PXG7H2-1/2/7826234c5c488282fb89a1393cb88571}
}

@Article{VisageImaging2008,
  Title                    = {Amira User Documentation},
  Author                   = {VisageImaging},
  Journal                  = {Visage Imaging GmbH},
  Year                     = {2008},

  Type                     = {Journal Article}
}

@Article{Volkov2007,
  Title                    = {Numerical modeling of short pulse laser interaction with Au nanoparticle surrounded by water},
  Author                   = {Volkov, Alexey N. and Sevilla, Carlos and Zhigilei, Leonid V.},
  Journal                  = {Applied Surface Science},
  Year                     = {2007},
  Number                   = {15},
  Pages                    = {6394-6399},
  Volume                   = {253},

  Abstract                 = {Short pulse laser interaction with a metal nanoparticle surrounded by water is investigated with a hydrodynamic computational model that includes a realistic equation of state for water and accounts for thermoelastic behavior and the kinetics of electron-phonon equilibration in the nanoparticle. Computational results suggest that, at laser fluences close to the threshold for vapor bubble formation, the region of biological damage due to the laser-induced thermal spike and the interaction of the pressure wave with internal cell structures can be localized within short distances from the absorbing particle comparable to the particle diameter. This irradiation regime is suitable for targeted generation of thermal and mechanical damage at the sub-cellular level.},
  ISSN                     = {0169-4332},
  Keywords                 = {Computer modeling
Nanoparticles
Cell targeting
Laser damage},
  Type                     = {Journal Article},
  Url                      = {http://www.sciencedirect.com/science/article/B6THY-4MYVG3D-8/2/19b24d6b462a62f194dec0aaf2a32ee0}
}

@Article{Vos2002,
  Title                    = {Navier-Stokes solvers in European aircraft design},
  Author                   = {Vos, J. B. and Rizzi, A. and Darracq, D. and Hirschel, E. H.},
  Journal                  = {Progress in Aerospace Sciences},
  Year                     = {2002},
  Number                   = {8},
  Pages                    = {601-697},
  Volume                   = {38},

  Abstract                 = {The paper gives a broad perspective of the progress made during the last 10 years in solving the Navier-Stokes equations and traces how this simulation technique went from being a specialized research topic to a practical engineering tool that design engineers use on a routine basis. The scope is limited to Navier-Stokes solvers applied to industrial design of airframes with attention focused particularly on developments in Europe. An overview of the different Navier-Stokes codes used in Europe is given, and on-going developments are outlined. The current state of progress is illustrated by computed steady and unsteady solutions to industrial problems, ranging from airfoil characteristics, flow around an isolated wing, to full aircraft configurations. A discussion on the future industrial design environment is given, and developments in Europe towards a more integrated design approach with underlying concepts like [`]concurrent engineering (CE)' and the [`]virtual product (VP)' are summarized. The paper concludes with a discussion on future challenging applications.},
  ISSN                     = {0376-0421},
  Type                     = {Journal Article},
  Url                      = {http://www.sciencedirect.com/science/article/B6V3V-47MJ8RH-1/2/96e98b785508af9d25d53298172e75e2}
}

@Article{Vos2007,
  Title                    = {Correlation between severity of sleep apnea and upper airway morphology based on advanced anatomical and functional imaging},
  Author                   = {Vos, W. and De Backer, J. and Devolder, A. and Vanderveken, O. and Verhulst, S. and Salgado, R. and Germonpre, P. and Partoens, B. and Wuyts, F. and Parizel, P. and De Backer, W.},
  Journal                  = {Journal of Biomechanics},
  Year                     = {2007},
  Number                   = {10},
  Pages                    = {2207-2213},
  Volume                   = {40},

  Abstract                 = {Determination of the apnea hypopnea index (AHI) as a measure of the severity of obstructive sleep apnea/hypopnea syndrome (OSAHS) is a widely accepted methodology. However, the outcome of such a determination depends on the method used, is time consuming and insufficient for prediction of the effect of all treatment modalities. For these reasons more methods for evaluating the severity of OSAHS, based on different imaging modalities, have been looked into and recent studies have shown that anatomical properties determined from three-dimensional (3D) computed tomography (CT) images are good markers for the severity of the OSAHS. Therefore, we correlated anatomical measurements of a 3D reconstruction of the upper airway together with flow simulation results with the severity of OSAHS in order to find a combination of variables to indicate the severity of OSAHS in patients. The AHI of 20 non-selected, consecutive patients has been determined during a polysomnography. All patients also underwent a CT scan from which a 3D model of the upper airway geometry was reconstructed. This 3D model was used to evaluate the anatomical properties of the upper airway in OSAHS patients as well as to perform computational fluid dynamics (CFD) computations to evaluate the airflow and resistance of this upper airway. It has been shown that a combination of the smallest cross-sectional area and the resistance together with the body mass index (BMI) form a set of markers that predict very well the severity of OSAHS in patients within this study. We believe that these markers can be used to evaluate the outcome of an OSAHS treatment.},
  ISSN                     = {0021-9290},
  Keywords                 = {OSAHS
Upper airway
3D reconstruction
CFD},
  Type                     = {Journal Article},
  Url                      = {http://www.sciencedirect.com/science/article/B6T82-4MKV2S5-1/2/503977e04ed1487441aaee5c8f56f456}
}

@Article{Wadell1933,
  Title                    = {Sphericity and roundness of rock particles},
  Author                   = {Wadell, H.},
  Journal                  = {Journal Geol.},
  Year                     = {1933},
  Pages                    = {310-331},
  Volume                   = {41},

  Type                     = {Journal Article}
}

@Article{Wagers2007,
  Title                    = {Intrinsic and antigen-induced airway hyperresponsiveness are the result of diverse physiological mechanisms},
  Author                   = {Wagers, S. and Haverkamp, H.C. and Bates, J.H.T. and Norton, R.J. and Thompson-Figueroa, J.A. and Sullivan, M.J. and Irvin, C.G.},
  Journal                  = {Journal of Applied Physiology},
  Year                     = {2007},
  Pages                    = {221-230},
  Volume                   = {102},

  Type                     = {Journal Article}
}

@Article{Wallace2008,
  Title                    = {The diagnosis and management of rhinitis: An updated practice parameter },
  Author                   = {Dana V. Wallace and Mark S. Dykewicz and David I. Bernstein and Joann Blessing-Moore and Linda Cox and David A. Khan and David M. Lang and Richard A. Nicklas and John Oppenheimer and Jay M. Portnoy and Christopher C. Randolph and Diane Schuller and Sheldon L. Spector and Stephen A. Tilles},
  Journal                  = {Journal of Allergy and Clinical Immunology },
  Year                     = {2008},
  Note                     = {The Diagnosis and Management of Rhinitis: An Updated Practice Parameter },
  Number                   = {2, Supplement},
  Pages                    = {S1 - S84},
  Volume                   = {122},

  Abstract                 = {These parameters were developed by the Joint Task Force on Practice Parameters, representing the American Academy of Allergy, Asthma &amp; Immunology; the American College of Allergy, Asthma and Immunology; and the Joint Council of Allergy, Asthma and Immunology. The American Academy of Allergy, Asthma &amp; Immunology (AAAAI) and the American College of Allergy, Asthma and Immunology (ACAAI) have jointly accepted responsibility for establishing “The diagnosis and Management of Rhinitis: An Updated Practice Parameter.” This is a complete and comprehensive document at the current time. The medical environment is a changing environment, and not all recommendations will be appropriate for all patients. Because this document incorporated the efforts of many participants, no single individual, including those who served on the Joint Task Force, is authorized to provide an official \{AAAAI\} or \{ACAAI\} interpretation of these practice parameters. Any request for information about or an interpretation of these practice parameters by the \{AAAAI\} or \{ACAAI\} should be directed to the Executive Offices of the AAAAI, the ACAAI, and the Joint Council of Allergy, Asthma and Immunology. These parameters are not designed for use by pharmaceutical companies in drug promotion. },
  Doi                      = {http://dx.doi.org/10.1016/j.jaci.2008.06.003},
  ISSN                     = {0091-6749},
  Url                      = {http://www.sciencedirect.com/science/article/pii/S0091674908011238}
}

@Article{Wallinga1999,
  Title                    = {Perspective: human contact patterns and the spread of airborne infectious diseases},
  Author                   = {Wallinga, Jacco and Edmunds, W. John and Kretzschmar, Mirjam},
  Journal                  = {Trends in Microbiology},
  Year                     = {1999},
  Number                   = {9},
  Pages                    = {372-377},
  Volume                   = {7},

  ISSN                     = {0966-842X},
  Keywords                 = {Social network
Contact structure
Transmission dynamics
Age-specific contacts
Global and local contacts
Evolution of virulence
Epidemic model
DNA fingerprints},
  Type                     = {Journal Article},
  Url                      = {http://www.sciencedirect.com/science/article/B6TD0-3XBTG8N-J/2/5953a034387140d6b380734591292d52}
}

@Article{Wang2008,
  Title                    = {Assessment of the lung microstructure in patients with asthma using hyperpolarized 3He diffusion MRI at two time scales: Comparison with healthy subjects and patients with COPD},
  Author                   = {Wang, Chengbo and Altes, Talissa A. and Mugler, John P. and Miller, G. Wilson and Ruppert, Kai and Mata, Jaime F. and Cates, Gordon D. and Borish, Larry and de Lange, Eduard E.},
  Journal                  = {Journal of Magnetic Resonance Imaging},
  Year                     = {2008},
  Number                   = {1},
  Pages                    = {80-88},
  Volume                   = {28},

  ISSN                     = {1522-2586},
  Keywords                 = {asthma
small airways
hyperpolarized gas
hyperpolarized helium-3
MRI
diffusion MRI},
  Type                     = {Journal Article},
  Url                      = {http://dx.doi.org/10.1002/jmri.21408}
}

@Article{Wang2007,
  Title                    = {Large eddy simulation of a sheet/cloud cavitation on a NACA0015 hydrofoil},
  Author                   = {Wang, G. and Ostoja-Starzewski, M.},
  Journal                  = {Applied Mathematical Modelling},
  Year                     = {2007},
  Number                   = {3},
  Pages                    = {417-447},
  Volume                   = {31},

  Abstract                 = {A single fluid model of sheet/cloud cavitation is developed and applied to a NACA0015 hydrofoil. First, a cavity formation model is set up, based on a three-dimensional (3D) non-cavitation model of Navier-Stokes equations with a large eddy simulation (LES) scheme for weakly compressible flows. A fifth-order polynomial curve is adopted to describe the relationship between density coefficient ratio and pressure coefficient when cavitation occurs. The Navier-Stokes equations including cavitation bubble clusters are solved using the finite-volume approach with time-marching scheme, and MacCormack's explicit-corrector scheme is adopted. Simulations are carried out in a 3D field acting on a hydrofoil NACA0015 at angles of attack 4°, 8° and 20°, with cavitation numbers [sigma] = 1.0, 1.5 and 2.0, Re = 106, and a 360 × 63 × 29 meshing system. We study time-dependent sheet/cloud cavitation structures, caused by the interaction of viscous objects, such as vortices, and cavitation bubbles. At small angles of attack (4°), the sheet cavity is relatively stable just by oscillating in size at the accumulation stage; at 8° it has a tendency to break away from the upper foil section, with the cloud cavitation structure becoming apparent; at 20°, the flow separates fully from the leading edge of the hydrofoil, and the vortex cavitation occurs. Comparisons with other studies, carried out mainly in the context of flow patterns on which prior experiments and simulations were done, demonstrate the power of our model. Overall, it can snapshot the collapse of cloud cavitation, and allow a study of flow patterns and their instabilities, such as "crescent-shaped regions."},
  ISSN                     = {0307-904X},
  Keywords                 = {Sheet/cloud cavitation
Finite-volume method
Shock waves
Hydrofoil
Vortex shedding
Weakly compressible flow
Large eddy simulation (LES)},
  Type                     = {Journal Article},
  Url                      = {http://www.sciencedirect.com/science/article/B6TYC-4J022XH-1/2/9cb8efc1ed3ce5f44f6fa972bd826610}
}

@Article{Wang2002,
  Title                    = {Diffusional losses in particle sampling systems containing bends and elbows},
  Author                   = {Wang, Jian and Flagan, Richard C. and Seinfeld, John H.},
  Journal                  = {Journal of Aerosol Science},
  Year                     = {2002},
  Note                     = {doi: DOI: 10.1016/S0021-8502(02)00042-3},
  Number                   = {6},
  Pages                    = {843-857},
  Volume                   = {33},

  Abstract                 = {Classical theoretical treatments for diffusional deposition of particles in tube flow describe the losses within straight tubes. The plumbing in many systems of practical interest, notably aerosol instruments, consists of short segments of tubing connected by elbows, bends, and other disturbances. To understand the particle losses in such systems, particle losses in tube flows containing bends and elbows has been studied for Reynolds numbers ranging from 80 to 950. Monodisperse aerosol of 5-15 nm diameter particles passed through individual bends or elbows, and through a number of bends or elbows in series. The results show that the effect of bends and elbows on particle diffusion loss is significant. For a flow configuration with four elbows in series, the penetration efficiency drops as much as 44% when compared to a straight tube with the same length. For Reynolds number smaller than 250, the enhancement of diffusion losses due to bends and elbows is sensitive to both the relative orientations of the bends and elbows and the lengths of straight tubing between them. Because of this sensitivity, direct calibration or simulation is required to assess nanoparticle penetration efficiencies for any flow system containing bends or elbows at low Reynolds number. When the Reynolds number exceeds 250, the enhancement is insensitive to the actual flow configurations. Experimental results are presented, which may be useful for design of aerosol flow systems at Reynolds number larger than 250.},
  ISSN                     = {0021-8502},
  Keywords                 = {Particle diffusion loss
Bend
Elbow},
  Type                     = {Journal Article},
  Url                      = {http://www.sciencedirect.com/science/article/B6V6B-4538GP7-1/2/8b0509dec0d6bacc084d0f0b1e009aac}
}

@Article{Wang2008a,
  Title                    = {Time-dependent translocation and potential impairment on central nervous system by intranasally instilled TiO2 nanoparticles},
  Author                   = {Wang, Jiangxue and Liu, Ying and Jiao, Fang and Lao, Fang and Li, Wei and Gu, Yiqun and Li, Yufeng and Ge, Cuicui and Zhou, Guoqiang and Li, Bai and Zhao, Yuliang and Chai, Zhifang and Chen, Chunying},
  Journal                  = {Toxicology},
  Year                     = {2008},
  Number                   = {1-2},
  Pages                    = {82-90},
  Volume                   = {254},

  Doi                      = {10.1016/j.tox.2008.09.014},
  ISSN                     = {0300-483X},
  Keywords                 = {TiO2 nanomaterials
Translocation
Neurotoxicology
Redox status
Proinflammatory cytokines
Immune response},
  Type                     = {Journal Article},
  Url                      = {http://www.sciencedirect.com/science/article/pii/S0300483X08004411}
}

@InProceedings{Wang,
  Title                    = {Numerical simulation of air flow in the human nasal cavity},
  Author                   = {Wang, K. and Denney, T.S. and E.E., Morrison and Vodyanoy, V.J.},
  Pages                    = {5607-5610},
  Publisher                = {Proceeding of the 2005 IEEE},

  Type                     = {Conference Proceedings}
}

@Article{Wang1992,
  Title                    = {Stochastic trajectory models for turbulent diffusion: Monte Carlo process versus Markov chains},
  Author                   = {Wang, L. P. and Stock, D. E.},
  Journal                  = {Atmospheric Environment - Part A General Topics},
  Year                     = {1992},
  Note                     = {Cited By (since 1996): 43
Export Date: 6 June 2011
Source: Scopus},
  Number                   = {9},
  Pages                    = {1599-1607},
  Volume                   = {26 A},

  Type                     = {Journal Article},
  Url                      = {http://www.scopus.com/inward/record.url?eid=2-s2.0-0026883966&partnerID=40&md5=28d2198119c507eac062d88f4f9ced88}
}

@Article{Wang2009,
  Title                    = {Assessment of Various Turbulence Models for Transitional Flows in an Enclosed Environment (RP-1271)},
  Author                   = {Wang, M. and Chen, Q.},
  Journal                  = {HVAC\&R Research},
  Year                     = {2009},
  Number                   = {6},
  Pages                    = {1099-1119},
  Volume                   = {15},

  Type                     = {Journal Article}
}

@Article{Wang1997,
  Title                    = {On the role of the lift force in turbulence simulations of particle deposition},
  Author                   = {Wang, Q. and Squires, K. D. and Chen, M. and McLaughlin, J. B.},
  Journal                  = {International Journal of Multiphase Flow},
  Year                     = {1997},
  Note                     = {doi: DOI: 10.1016/S0301-9322(97)00014-1},
  Number                   = {4},
  Pages                    = {749-763},
  Volume                   = {23},

  ISSN                     = {0301-9322},
  Keywords                 = {lift force
particle deposition},
  Type                     = {Journal Article},
  Url                      = {http://www.sciencedirect.com/science/article/B6V45-3SP77DR-9/2/1db3dc290443e7e37d052673d95c4ec3}
}

@Article{Wang2007a,
  Title                    = {Modeling of nanoparticle-induced Rayleigh-Gans scattering for nanofiber optical sensing},
  Author                   = {Wang, Shanshan and Pan, Xinyun and Tong, Limin},
  Journal                  = {Optics Communications},
  Year                     = {2007},
  Number                   = {2},
  Pages                    = {293-297},
  Volume                   = {276},

  Abstract                 = {Based on Rayleigh-Gans scattering theory of spherical particles and evanescent-wave guiding properties of nanofibers, we calculate the scattered power of nanoparticles in the vicinity of typical waveguiding nanofibers. It shows that, by optimizing the wavelength of the probing light and the diameter of the nanofiber, nanoparticle-induced scattering intensity can reach detectable level with possibilities for single-molecule detection. Results presented in this work suggest a simple approach to high-sensitivity nanofiber optical sensing of nanoparticles in aqueous solutions.},
  ISSN                     = {0030-4018},
  Keywords                 = {Rayleigh-Gans scattering theory
Fraction of the scattered power
Nanofiber optical sensing},
  Type                     = {Journal Article},
  Url                      = {http://www.sciencedirect.com/science/article/B6TVF-4NNYDD4-1/2/9b530652c9c0f82445fbaef0891d6dcb}
}

@Article{Wang2009a,
  Title                    = {Comparison of micron- and nanoparticle deposition patterns in a realistic human nasal cavity},
  Author                   = {Wang, S. M. and Inthavong, K. and Wen, J. and Tu, J. Y. and Xue, C. L.},
  Journal                  = {Respiratory Physiology and Neurobiology},
  Year                     = {2009},
  Number                   = {3},
  Pages                    = {142-151},
  Volume                   = {166},

  Doi                      = {10.1016/j.resp.2009.02.014},
  ISSN                     = {1569-9048},
  Keywords                 = {Nanoparticle
Micronparticle
Nasal cavity
Deposition},
  Type                     = {Journal Article},
  Url                      = {http://www.sciencedirect.com/science/article/pii/S1569904809000421}
}

@Article{Wang2007b,
  Title                    = {Spray characteristics of high-pressure swirl injector fueled with alcohol},
  Author                   = {Wang, Xibin and Chen, Wansheng and Gao, Jian and Jiang, Deming and Huang, Zuohua},
  Journal                  = {Frontiers of Energy and Power Engineering in China},
  Year                     = {2007},
  Number                   = {1},
  Pages                    = {105-112},
  Volume                   = {1},

  Abstract                 = {Abstract&nbsp;&nbsp;The spray characteristics of methanol and ethanol with high-pressure swirl injector were explored experimentally and numerically. experimental results show that the spray characteristics of methanol and ethanol had displayed the same trends as that of gasoline. Under the low back-pressure ambient conditions, the spray behavior exhibited a hollow cone with wide spray angle and initial spray slug at the tip, while the spray presented a solid cone in the case of high back-pressure. Vortexes in the opposite direction existed in the rear part of the spray under low back-pressure ambient conditions while the vortexes formed in the middle part under high back-pressure ambient conditions. Experiments also showed that methanol had the largest cone angle, while ethanol and gasoline presented almost the same cone angle. Simulation results indicated that methanol and ethanol had a slightly larger Sauter mean diameter (SMD) than that of gasoline with swirl injector. The SMD profile of methanol coincided well with that of ethanol under low back-pressure ambient conditions, but displayed a slightly larger value under high back-pressure due to fuel evaporation Numerical simulation could successfully demonstrate the spray characteristics of high-pressure swirl injector for methanol and ethanol fuels.},
  Type                     = {Journal Article},
  Url                      = {http://dx.doi.org/10.1007/s11708-007-0012-z}
}

@Article{Wang2005,
  Title                    = {Spray Characteristics of High-Pressure Swirl Injector Fueled with Methanol and Ethanol},
  Author                   = {Wang, Xibin and Gao, Jian and Jiang, Deming and Huang, Zuohua and Chen, Wansheng},
  Journal                  = {Energy \& Fuels},
  Year                     = {2005},
  Note                     = {doi: 10.1021/ef050135w},
  Number                   = {6},
  Pages                    = {2394-2401},
  Volume                   = {19},

  Doi                      = {10.1021/ef050135w},
  ISSN                     = {0887-0624},
  Type                     = {Journal Article},
  Url                      = {http://dx.doi.org/10.1021/ef050135w}
}

@Article{Wang1999,
  Title                    = {On the effect of anisotropy on the turbulent dispersion and deposition of small particles},
  Author                   = {Wang, Y. and James, P.W.},
  Journal                  = {Int. Journal of Multiphase Flows},
  Year                     = {1999},
  Pages                    = {551-558},
  Volume                   = {25},

  Type                     = {Journal Article}
}

@Article{Wang2008b,
  Title                    = {Fibrous particle deposition in human nasal passage: The influence of particle length, flow rate, and geometry of nasal airway},
  Author                   = {Wang, Zuocheng and Hopke, Philip K. and Ahmadi, Goodarz and Cheng, Yung-Sung and Baron, Paul A.},
  Journal                  = {Journal of Aerosol Science},
  Year                     = {2008},
  Number                   = {12},
  Pages                    = {1040-1054},
  Volume                   = {39},

  Doi                      = {http://dx.doi.org/10.1016/j.jaerosci.2008.07.008},
  ISSN                     = {0021-8502},
  Keywords                 = {Nasal deposition
Glass fiber
Realistic nasal model
Relaxation time
Particle deposition efficiency
Pressure drop
Friction coefficient},
  Type                     = {Journal Article},
  Url                      = {http://www.sciencedirect.com/science/article/pii/S0021850208001390}
}

@Article{Wang2005a,
  Title                    = {Fiber classification and the influence of average air humidity},
  Author                   = {Wang, Z. and Hopke, P. K. and Baron, P. A. and Ahmadi, G. and Cheng, Y. S. and Deye, G. and Su, W. C.},
  Journal                  = {Aerosol Science and Technology},
  Year                     = {2005},
  Note                     = {Cited By (since 1996): 1
Export Date: 6 June 2011
Source: Scopus},
  Number                   = {11},
  Pages                    = {1056-1063},
  Volume                   = {39},

  Type                     = {Journal Article},
  Url                      = {http://www.scopus.com/inward/record.url?eid=2-s2.0-28844469917&partnerID=40&md5=6a4a8ec907d8aac7e5b919204d09f202}
}

@Article{Waritz1990,
  Title                    = {Chronic inhalation of 3 um diameter graphite fibres},
  Author                   = {Waritz, R.S. and Collins, C.J. and Ballantyne, B. and Clary, J.J.},
  Journal                  = {The Toxicologist},
  Year                     = {1990},
  Pages                    = {70-71},
  Volume                   = {19},

  Type                     = {Journal Article}
}

@Article{Washino2011,
  Title                    = {Direct numerical simulation of solid-liquid-gas three-phase flow: Fluid-solid interaction},
  Author                   = {Washino, K. and Tan, H. S. and Salman, A. D. and Hounslow, M. J.},
  Journal                  = {Powder Technology},
  Year                     = {2011},
  Number                   = {1-2},
  Pages                    = {161-169},
  Volume                   = {206},

  Abstract                 = {A direct numerical simulation (DNS) model for three-phase flow (solid, liquid, and gas) with the main purpose of analysing wet granulation processes is presented in this paper. In the present model, liquid-gas two-phase flow is solved by the constrained interpolation profile (CIP) method developed by Yabe et al. (2001) [1], and the interaction between fluid phases and solid particle phase is taken into account by using the immersed boundary (IB) method developed by Kajishima et al. (2001) [2]. The surface tension as well as the wetting are modelled by using the continuous surface force (CSF) model suggested by Brackbill et al. (1992) [3], and the dynamic contact angle is represented by Fukai's (1995) [4] approach, which selectively uses advancing and receding contact angles depending on the movement of fluid interfaces on a solid surface. The accuracy of the model is examined in terms of (i) the drag force exerted on a single particle, (ii) the drag force exerted on a regular particle array, (iii) the surface tension force, and (iv) the wetting. A number of test simulations have been carried out with different numerical cell sizes, and the results are compared with the reported experimental work and theoretical values.},
  Doi                      = {10.1016/j.powtec.2010.07.015},
  ISSN                     = {0032-5910},
  Keywords                 = {Three-phase flow
Direct numerical simulation
Wet granulation
Constrained interpolation profile (CIP) method
Immersed boundary (IB) method},
  Type                     = {Journal Article},
  Url                      = {http://www.sciencedirect.com/science/article/pii/S0032591010003621}
}

@Misc{Wassgren2008,
  Title                    = {Discrete Element Method (DEM) Course Module},

  Author                   = {Wassgren, Carl and Sarkar, Avik},
  Month                    = {Jan},
  Year                     = {2008},

  Type                     = {Generic},
  Url                      = {http://pharmahub.org/resources/113}
}

@InBook{Weibel1997,
  Title                    = {Design of airways and blood vessels considered as branching trees},
  Author                   = {Weibel, E. R.},
  Editor                   = {Crystal, R. G. and West, J.B. and Weibel, E.R. and Barnes, P. J.},
  Pages                    = {1061-1071},
  Publisher                = {Lippincott-Raven Publishers},
  Year                     = {1997},

  Address                  = { Philadelphia},
  Type                     = {Book Section},

  Booktitle                = {The Lung: Scientific Foundations}
}

@Article{Weiner1954,
  Title                    = {Nose shape and climate},
  Author                   = {Weiner, J.S.},
  Journal                  = {Am J Phys Anthropol},
  Year                     = {1954},
  Pages                    = {1-4},
  Volume                   = {12},

  Type                     = {Journal Article}
}

@Article{Weinhold2004,
  Title                    = {Numerical simulation of airflow in the human nose},
  Author                   = {Weinhold, I. and Mlynski, G.},
  Journal                  = { European Archive Otorhinolaryngology},
  Year                     = {2004},
  Pages                    = {452-455},
  Volume                   = {261},

  Type                     = {Journal Article}
}

@Article{Welling2009,
  Title                    = {Wood Dust Particle and Mass Concentrations and Filtration Efficiency in Sanding of Wood Materials},
  Author                   = {Welling, Irma and Lehtimäki, Matti and Rautio, Sari and Lähde, Tero and Enbom, Seppo and Hynynen, Pasi and Hämeri, Kaarle},
  Journal                  = {Journal of Occupational and Environmental Hygiene},
  Year                     = {2009},
  Number                   = {2},
  Pages                    = {90 - 98},
  Volume                   = {6},

  ISSN                     = {1545-9624},
  Type                     = {Journal Article},
  Url                      = {http://www.informaworld.com/10.1080/15459620802623073}
}

@Misc{Wen2007,
  Title                    = {Airflow Patterns in Both Sides of a Realistic Human Nasal Cavity for Laminar and Turbulent Conditions},

  Author                   = {Wen, J. and Inthavong, K. and Tian, Z.F. and Tu, J. Y. and Xue, C.L. and Li, C.G.},
  Year                     = {2007},

  ISBN                     = {978-1-864998-94-8 },
  Publisher                = {School of Engineering, The University of Queensland },
  Type                     = {Conference Paper}
}

@Article{Wen2008,
  Title                    = {Numerical simulations for detailed airflow dynamics in a human nasal cavity},
  Author                   = {Wen, Jian and Inthavong, Kiao and Tu, Jiyuan and Wang, Simin},
  Journal                  = {Respiratory Physiology \& Neurobiology},
  Year                     = {2008},
  Number                   = {2},
  Pages                    = {125-135},
  Volume                   = {161},

  Abstract                 = {Nasal physiology is dependent on the physical structure of the nose. Individual aspects of the nasal cavity such as the geometry and flow rate collectively affect nasal function such as the filtration of foreign particles by bringing inspired air into contact with mucous-coated walls, humidifying and warming the air before it enters the lungs and the sense of smell. To better understand the physiology of the nose, this study makes use of CFD methods and post-processing techniques to present flow patterns between the left and right nasal cavities and compared the results with experimental and numerical data that are available in literature. The CFD simulation adopted a laminar steady flow for flow rates of 7.5 L/min and 15 L/min. General agreement of gross flow features were found that included high velocities in the constrictive nasal valve area region, high flow close to the septum walls, and vortex formations posterior to the nasal valve and olfactory regions. The differences in the left and right cavities were explored and the effects it had on the flow field were discussed especially in the nasal valve and middle turbinate regions. Geometrical differences were also compared with available models.},
  Doi                      = {http://dx.doi.org/10.1016/j.resp.2008.01.012},
  ISSN                     = {1569-9048},
  Keywords                 = {Nasal cavity
Modelling
Airflow
CFD
Nose
Simulation},
  Type                     = {Journal Article},
  Url                      = {http://www.sciencedirect.com/science/article/pii/S1569904808000189}
}

@Article{Wen2008a,
  Title                    = {Numerical simulations for detailed airflow dynamics in a human nasal cavity},
  Author                   = {Wen, Jian and Inthavong, Kiao and Tu, Jiyuan and Wang, Simin},
  Journal                  = {Respiratory Physiology \& Neurobiology},
  Year                     = {2008},
  Pages                    = {125-135},
  Volume                   = {161},

  Abstract                 = {Nasal physiology is dependent on the physical structure of the nose. Individual aspects of the nasal cavity such as the geometry and flow rate collectively affect nasal function such as the filtration of foreign particles by bringing inspired air into contact with mucous-coated walls, humidifying and warming the air before it enters the lungs and the sense of smell. To better understand the physiology of the nose, this study makes use of CFD methods and post-processing techniques to present flow patterns between the left and right nasal cavities and compared the results with experimental and numerical data that are available in literature. The CFD simulation adopted a laminar steady flow for flow rates of 7.5 L/min and 15 L/min. General agreement of gross flow features were found that included high velocities in the constrictive nasal valve area region, high flow close to the septum walls, and vortex formations posterior to the nasal valve and olfactory regions. The differences in the left and right cavities were explored and the effects it had on the flow field were discussed especially in the nasal valve and middle turbinate regions. Geometrical differences were also compared with available models.},
  Doi                      = {http://dx.doi.org/10.1016/j.resp.2008.01.012},
  ISSN                     = {1569-9048},
  Keywords                 = {Airflow
cavity
CFD
Modelling
Nasal
Nose
Simulation},
  Type                     = {Journal Article},
  Url                      = {http://ac.els-cdn.com/S1569904808000189/1-s2.0-S1569904808000189-main.pdf?_tid=e0d02686-421f-11e4-8be4-00000aab0f02&acdnat=1411366721_d442f28b30d6b1327313ef0904f656d2}
}

@Article{Wen2008b,
  Title                    = {Numerical simulations for detailed airflow dynamics in a human nasal cavity},
  Author                   = {Wen, J. and Inthavong, K. and Tu, J.Y. and Wang, S.},
  Journal                  = {Respiratory Physiology \& Neurobiology},
  Year                     = {2008},
  Number                   = {2},
  Pages                    = {125-135},
  Volume                   = {161},

  Type                     = {Journal Article}
}

@Book{Wendt2009,
  Title                    = {Computational fluid dynamics: an introduction},
  Author                   = {Wendt, J.F. and Anderson, J.D.},
  Publisher                = {Springer},
  Year                     = {2009},

  ISBN                     = {9783540850557},
  Type                     = {Book},
  Url                      = {http://books.google.com.au/books?id=IIUkqI-HNbQC}
}

@InBook{Weng2003,
  Title                    = {Large Eddy Simulation of Rotor-Stator Cavity Flow},
  Author                   = {Weng, P. S. and Lo, W. and Lin, C. A.},
  Editor                   = {Matsuno, K. and Ecer, A. and Satofuka, N. and Periaux, J. and Fox, P.},
  Pages                    = {547-554},
  Publisher                = {North-Holland},
  Year                     = {2003},

  Address                  = {Amsterdam},
  Type                     = {Book Section},

  Abstract                 = {Summary The present study is concerned with the parallel large eddy simulations of rotor-stator cavity flow using SPMD and MPI. The numerical procedure is based on the finite volume approach with staggered grid arrangement and possesses second order of accuracy for both in space and time. The parallel efficiency using different computing platforms and data structures are addressed. Application are applied to the rotor-stator cavity flow. Capability of the adopted scheme are examined by comparing the predicted flow quantities with available measured data.},
  Booktitle                = {Parallel Computational Fluid Dynamics 2002},
  ISBN                     = {978-0-44-450680-1},
  Url                      = {http://www.sciencedirect.com/science/article/B872D-4PKGKN4-2D/2/f2cf188f16a2cadd21f3ff1e1be55bdd}
}

@Book{West2008,
  Title                    = {Respiratory physiology: the essentials},
  Author                   = {West, J.B.},
  Publisher                = {Wolters Kluwer Health/Lippincott Williams \& Wilkins},
  Year                     = {2008},

  ISBN                     = {9780781772068},
  Type                     = {Book},
  Url                      = {http://books.google.com.au/books?id=rHaZVmFZRVcC}
}

@Article{Wexler2005,
  Title                    = {Aerodynamic Effects of Inferior Turbinate Reduction },
  Author                   = {Wexler, D. and Segal, R.A. and Kimbell, J.},
  Journal                  = {Arch. Otolarngol. Head Neck Surg.},
  Year                     = {2005},
  Pages                    = {1102-1107},
  Volume                   = {131},

  Type                     = {Journal Article}
}

@Article{WhanKim2007,
  Title                    = {Change of nasal function with aging in Korean},
  Author                   = {Whan Kim, Si and Mo, Ji-Hun and Kim, Jeong-Whun and Kim, Dong-Young and Rhee, Chae-Seo and Hee Lee, Chul and Min, Yang-Gi},
  Journal                  = {Acta Oto-Laryngologica},
  Year                     = {2007},
  Number                   = {S558},
  Pages                    = {90--94},
  Volume                   = {127},

  Publisher                = {Informa UK Ltd UK}
}

@Article{WhanKim2007a,
  Title                    = {Change of nasal function with aging in Korean},
  Author                   = {Whan Kim, Si and Mo, Ji-Hun and Kim, Jeong-Whun and Kim, Dong-Young and Rhee, Chae-Seo and Hee Lee, Chul and Min, Yang-Gi},
  Journal                  = {Acta Oto-Laryngologica},
  Year                     = {2007},
  Number                   = {S558},
  Pages                    = {90--94},
  Volume                   = {127},

  Publisher                = {Informa UK Ltd UK}
}

@Article{Whitehead2003,
  Title                    = {Lung function and airway inflammation in rats following exposure to combustion products of carbon-graphite/epoxy composite material: comparison to a rodent model of acute lung injury},
  Author                   = {Whitehead, G. S. and Grasman, K. A. and Kimmel, E. C.},
  Journal                  = {Toxicology},
  Year                     = {2003},
  Number                   = {1-3},
  Pages                    = {175-97},
  Volume                   = {183},

  ISSN                     = {0300-483X (Print)
0300-483X (Linking)},
  Keywords                 = {Animals
Bronchoalveolar Lavage Fluid/chemistry
Chemokine CXCL2
Disease Models, Animal
Epoxy Compounds/ toxicity
Graphite/ toxicity
Herbicides/toxicity
Histocytochemistry
Interferon-gamma/biosynthesis
Male
Monokines/biosynthesis
Paraquat/toxicity
Pneumonia/etiology/metabolism/ physiopathology
Random Allocation
Rats
Rats, Inbred F344
Respiratory Distress Syndrome, Adult/chemically
induced/metabolism/ physiopathology
Respiratory Function Tests
Smoke/adverse effects
Smoke Inhalation Injury/etiology/metabolism/ physiopathology
Tumor Necrosis Factor-alpha/biosynthesis},
  Type                     = {Journal Article}
}

@Article{WHO2000,
  Title                    = {Man-made vitreous fibers in Air Quality Guidelines Second Edition},
  Author                   = {WHO},
  Journal                  = {World Health Organization},
  Year                     = {2000},
  Volume                   = {Chapter 8.2 },

  Type                     = {Journal Article}
}

@InProceedings{Whybrew,
  Title                    = {Diode Lasers-A Cost Effective Tool for Simultaneous Visualisation, Sizing, and Velocity Measurements of Sprays},
  Author                   = {Whybrew, A. and Nicholls, T.R. and Boaler, J.J. and Booth, H.J.},
  Booktitle                = {Proc. 15th Annual Conference on Liquid Atomization and Spray Systems},

  Type                     = {Conference Proceedings}
}

@Article{Wickham1992,
  Title                    = {Air Filters},
  Author                   = {Wickham, F.},
  Journal                  = {Australia, AIRH},
  Year                     = {1992},
  Pages                    = {(No. DA 15)},

  Type                     = {Journal Article}
}

@Article{Wiesmiller2003,
  Title                    = {The impact of expiration on particle deposition within the nasal cavity},
  Author                   = {Wiesmiller, K. and Keck, T. and Leiacker, R. and Sikora, T. and Rettinger, G. and Lindemann, J.},
  Journal                  = {Clinical Otolaryngology \& Allied Sciences},
  Year                     = {2003},
  Number                   = {4},
  Pages                    = {304-307},
  Volume                   = {28},

  Type                     = {Journal Article}
}

@Book{Wilcox2006,
  Title                    = {Turbulence Modeling for Cfd},
  Author                   = {Wilcox, D.C.},
  Publisher                = {DCW Industries, Incorporated},
  Year                     = {2006},

  ISBN                     = {9781928729082},
  Type                     = {Book},
  Url                      = {http://books.google.com.au/books?id=tFNNPgAACAAJ}
}

@Book{Wilcox1993,
  Title                    = {Turbulence Modeling for CFD},
  Author                   = {Wilcox, D.},
  Publisher                = {DCW Industries, Inc.},
  Year                     = {1993},

  Address                  = {5354 Palm Drive, La Canada, California 91011},

  Type                     = {Book}
}

@Misc{Williams2009,
  Title                    = {Velocity profiling of sprays from pharmaceutical nasal spray pumps},

  Author                   = {Williams, Trevor J. and Gilles, J.C. and Murphy, S.},
  Year                     = {2009},

  Type                     = {Conference Paper}
}

@Article{Wilson2009,
  Title                    = {A computer model of the artificially ventilated human respiratory system in adult intensive care},
  Author                   = {Wilson, A. J. and Murphy, C. M. and Brook, B. S. and Breen, D. and Miles, A. W. and Tilley, D. G.},
  Journal                  = {Medical Engineering \& Physics},
  Year                     = {2009},
  Number                   = {9},
  Pages                    = {1118-1133},
  Volume                   = {31},

  Abstract                 = {A multi-technique approach to modelling artificially ventilated patients on the adult general intensive care unit (ICU) is proposed. Compartmental modelling techniques were used to describe the mechanical ventilator and the flexible hoses that connect it to the patient. 3D CFD techniques were used to model flow in the major airways and a Windkessel style balloon model was used to model the mechanical properties of the lungs. A multi-compartment model of the lung based on bifurcating tree structures representing the conducting airways and pulmonary circulation allowed lung disease to be modelled in terms of altered ratios within a lognormal distribution of values and it is from these that gas exchange was determined. A compartmental modelling tool, Bathfp, was used to integrate the different modelling techniques into a single model. The values of key parameters in the model could be obtained from measurements on patients in an ICU whilst a sensitivity analysis showed that the model was insensitive to the value of other parameters within it. Measured and modelled values for arterial blood gases and airflow parameters are compared for 46 ventilator settings obtained from 6 ventilator dependent patients. The results show correlation coefficients of 0.88 and 0.85 for the arterial partial pressures of the O2 and CO2, respectively (p < 0.01) and of 0.99 and 0.96 for upper airway pressure and tidal volume, respectively (p < 0.01). The difference between measured and modelled values was large in physiological terms, suggesting that some optimisation of the model is required.},
  ISSN                     = {1350-4533},
  Keywords                 = {Respiratory system
Intensive care
Mathematical modelling
Computer simulation
Lung damage},
  Type                     = {Journal Article},
  Url                      = {http://www.sciencedirect.com/science/article/B6T9K-4X1XYPK-1/2/0eacb40ccb08362d97a20143ed084923}
}

@Article{Winkler2009,
  Title                    = {Relative importance of the lift force on heavy particles due to turbulence driven secondary flows},
  Author                   = {Winkler, C. M. and Rani, Sarma L.},
  Journal                  = {Powder Technology},
  Year                     = {2009},
  Note                     = {doi: DOI: 10.1016/j.powtec.2008.08.015},
  Number                   = {3},
  Pages                    = {310-318},
  Volume                   = {190},

  ISSN                     = {0032-5910},
  Keywords                 = {Lift force
Large eddy simulations},
  Type                     = {Journal Article},
  Url                      = {http://www.sciencedirect.com/science/article/B6TH9-4T7F5MM-1/2/fdc5b934d4022813249e933c45fbd3e4}
}

@Article{Wolf2004,
  Title                    = {Air-conditioning characteristics of the human nose},
  Author                   = {Wolf, M. and Naftali, S. and Schrotter, C. and Elad, D.},
  Journal                  = {Journal Laryng. Otology},
  Year                     = {2004},
  Number                   = {2},
  Pages                    = {87-92},
  Volume                   = {118},

  Type                     = {Journal Article}
}

@Article{Wolkoff2010,
  Title                    = {Non-cancer effects of formaldehyde and relevance for setting an indoor air guideline},
  Author                   = {Wolkoff, Peder and Nielsen, Gunnar D.},
  Journal                  = {Environment International},
  Year                     = {2010},
  Number                   = {7},
  Pages                    = {788-799},
  Volume                   = {36},

  Abstract                 = {There is considerable recent focus and concern about formaldehyde (FA). We have reviewed the literature on FA with focus on chemosensory perception in the airways and lung effects in indoor environments. Concentrations of FA, both personal and stationary, are on average in the order of 0.05 mg/m3 or less in Europe and North America with the exception of new housing or buildings with extensive wooden surfaces, where the concentration may exceed 0.1 mg/m3. With the eye the most sensitive organ, subjective irritation is reported at 0.3-0.5 mg/m3, which is somewhat higher than reported odour thresholds. Objective effects in the eyes and airways occur around 0.6-1 mg/m3. Dose-response relationships between FA and lung function effects have not been found in controlled human exposure studies below 1 mg/m3, and epidemiological associations between FA concentrations and exacerbation of asthma in children and adults are encumbered by complex exposures. Neither experimental nor epidemiological studies point to major differences in susceptibility to FA among children, elderly, and asthmatics. People with personal trait of negative affectivity may report more symptoms. An air quality guideline of 0.1 mg/m3 (0.08 ppm) is considered protective against both acute and chronic sensory irritation in the airways in the general population assuming a log normal distribution of nasal sensory irritation.},
  Doi                      = {10.1016/j.envint.2010.05.012},
  ISSN                     = {0160-4120},
  Keywords                 = {Airways
Asthma
Children
Formaldehyde
Sensory irritation
Susceptible subgroups},
  Type                     = {Journal Article},
  Url                      = {http://www.sciencedirect.com/science/article/pii/S0160412010001133}
}

@Article{Wolpoff1968,
  Title                    = {Climatic influence on skeletal nasal aperture},
  Author                   = {Wolpoff, M.H.},
  Journal                  = {Am J Phys Anthropol},
  Year                     = {1968},
  Pages                    = {405-424},
  Volume                   = {29},

  Type                     = {Journal Article}
}

@InBook{Wolter2007,
  Title                    = {Markov prefetching in multi-block particle tracing},
  Author                   = {Wolter, Marc and Gerndt, Andreas and Kuhlen, Torsten and Bischof, Christian},
  Editor                   = {Kwon, J. H. and Ecer, A. and Satofuka, N. and Periaux, J. and Fox, P.},
  Pages                    = {27-34},
  Publisher                = {Elsevier Science B.V.},
  Year                     = {2007},

  Address                  = {Amsterdam},
  Type                     = {Book Section},

  Booktitle                = {Parallel Computational Fluid Dynamics 2006},
  ISBN                     = {978-0-44-453035-6},
  Url                      = {http://www.sciencedirect.com/science/article/B87J0-4PTGHCD-Y/2/81008866884997e122bd0a19242b7bd3}
}

@Book{Womersley1975,
  Title                    = {An Elastic Tube Theory of Pulse Transmission and Oscillatory Flow in Mammalian Arteries},
  Author                   = {Womersley, J.R. },
  Publisher                = {Wright Air Development Center, Wright-Patterson Air Force Base},
  Year                     = {1975},

  Address                  = {Ohio},

  Type                     = {Book}
}

@Article{Womersley1955,
  Title                    = {Method for the calculation of velocity, rate of flow and viscous drag in arteries when the pressure gradient is known},
  Author                   = {Womersley, J.R.},
  Journal                  = {Journal Physiol.},
  Year                     = {1955},
  Number                   = {553-563},
  Volume                   = {127},

  Type                     = {Journal Article}
}

@Article{Wood1981,
  Title                    = {MASS TRANSFER OF PARTICLES AND ACID VAPOUR TO COOLED SURFACES},
  Author                   = {Wood, N. B.},
  Journal                  = {Journal of the Institute of Energy},
  Year                     = {1981},
  Note                     = {Cited By (since 1996): 46
Export Date: 6 June 2011
Source: Scopus},
  Number                   = {419},
  Pages                    = {76-93},
  Volume                   = {54},

  Type                     = {Journal Article},
  Url                      = {http://www.scopus.com/inward/record.url?eid=2-s2.0-0019578879&partnerID=40&md5=990207e70cc8ad0142e83ff6a983d752}
}

@Article{Wood1981a,
  Title                    = {A simple method for the calculation of turbulent deposition to smooth and rough surfaces},
  Author                   = {Wood, N. B.},
  Journal                  = {Journal of Aerosol Science},
  Year                     = {1981},
  Number                   = {3},
  Pages                    = {275-290},
  Volume                   = {12},

  ISSN                     = {0021-8502},
  Type                     = {Journal Article},
  Url                      = {http://www.sciencedirect.com/science/article/pii/0021850281901270}
}

@Article{Wootton2014,
  Title                    = {Computational fluid dynamics endpoints to characterize obstructive sleep apnea syndrome in children},
  Author                   = {Wootton, David M and Luo, Haiyan and Persak, Steven C and Sin, Sanghun and McDonough, Joseph M and Isasi, Carmen R and Arens, Raanan},
  Journal                  = {Journal of Applied Physiology},
  Year                     = {2014},
  Note                     = {C:\Users\sean\AppData\Roaming\Zotero\Zotero\Profiles\16a4oype.default\zotero\storage\A4TKVPHA\Computational fluid dynamics endpoints to characterize obstructive sleep apnea syndrome in children.pdf},
  Pages                    = {104–112},
  Volume                   = {116},

  Type                     = {Journal Article},
  Url                      = {http://jap.physiology.org/content/jap/116/1/104.full.pdf}
}

@Article{Wright2004,
  Title                    = {Generation of resting membrane potential},
  Author                   = {Wright, Stephen H.},
  Journal                  = {Advances in Physiology Education},
  Year                     = {2004},
  Number                   = {4},
  Pages                    = {139-142},
  Volume                   = {28},

  Abstract                 = {This brief review is intended to serve as a refresher on the ideas associated with teaching students the physiological basis of the resting membrane potential. The presentation is targeted toward first-year medical students, first-year graduate students, or senior undergraduates. The emphasis is on general concepts associated with generation of the electrical potential difference that exists across the plasma membrane of every animal cell. The intention is to provide students a general view of the quantitative relationship that exists between 1) transmembrane gradients for K+ and Na+ and 2) the relative channel-mediated permeability of the membrane to these ions.},
  Doi                      = {10.1152/advan.00029.2004},
  Type                     = {Journal Article},
  Url                      = {http://advan.physiology.org/content/28/4/139.abstract}
}

@Article{rohtua,
  Title                    = {Breathing Resistance and Ultrafine Particle Deposition in Nasal-Laryngeal Airways of a Newborn, an Infant, a Child, and an Adult},
  Author                   = {Xi,Jinxiang and Berlinski,Ariel and Zhou,Yue and Greenberg,Bruce and Ou,Xiawei},
  Journal                  = {Annals of Biomedical Engineering},
  Year                     = {2012},

  Month                    = {12},
  Number                   = {12},
  Pages                    = {2579-2595},
  Volume                   = {40},

  ISBN                     = {0090-6964},
  Language                 = {English},
  Url                      = {http://search.proquest.com/docview/1433613334?accountid=13552}
}

@Article{Xi2012,
  Title                    = {Breathing Resistance and Ultrafine Particle Deposition in Nasal-Laryngeal Airways of a Newborn, an Infant, a Child, and an Adult},
  Author                   = {Xi, Jinxiang and Berlinski, Ariel and Zhou, Yue and Greenberg, Bruce and Ou, Xiawei},
  Journal                  = {Annals of Biomedical Engineering},
  Year                     = {2012},

  Doi                      = {http://dx.doi.org/10.1007/s10439-012-0603-7},
  ISSN                     = {0090-6964},
  Keywords                 = {Medical Sciences
Solid State and Superconductivity Abstracts
Mechanical & Transportation Engineering Abstracts
ANTE: Abstracts in New Technologies and Engineering
Biotechnology and Bioengineering Abstracts
Birth
Aerosols
Age
Respiration
Diffusion
Neonates
Children
Respiratory tract
Models
Infants
Yes:(AN)
W 30965:Miscellaneous, Reviews},
  Type                     = {Journal Article}
}

@Article{Xi2012a,
  Title                    = {Breathing Resistance and Ultrafine Particle Deposition in Nasal-Laryngeal Airways of a Newborn, an Infant, a Child, and an Adult},
  Author                   = {Xi, Jinxiang and Berlinski, Ariel and Zhou, Yue and Greenberg, Bruce and Ou, Xiawei},
  Journal                  = {Annals of Biomedical Engineering},
  Year                     = {2012},
  Note                     = {C:\Users\sean\AppData\Roaming\Zotero\Zotero\Profiles\16a4oype.default\zotero\storage\S5SID67Q\breathing resistance and ultrafine particle deposition in nasallaryngeal airways of a newborn, an infant and a 5 yearold .pdf},
  Pages                    = {2579-2595},
  Volume                   = {40},

  Abstract                 = {As a human grows from birth to adulthood, both airway anatomy and breathing conditions vary, altering the deposition rate and pattern of inhaled aerosols. However, deposition studies have typically focused on adult subjects, results of which may not be readily extrapolated to children. This study numerically evaluated the age-related effects on the airflow and aerosol dynamics in image-based nose-throat models of a 10-day-old newborn, a 7-month-old infant, a 5-year-old child, and a 53-year-old adult. Differences in airway physiology, breathing resistance, and aerosol filtering efficiency among the four models were quantified and compared. A high-fidelity fluid-particle transport model was employed to simulate the multi-regime airflows and particle transport within the nasal-laryngeal airways. Ultrafine particles were evaluated under breathing conditions ranging from sedentary to heavy activities. Results of this study indicate that the nasal-laryngeal airways at different ages, albeit differ significantly in morphology and dimension, do not significantly affect the total deposition fractions or maximum local deposition enhancement for ultrafine aerosols. Further, the deposition partitioning in the sub-regions of interest is different among the four models. Results of this study corroborate the use of the in vivo -based diffusion parameter (D super(0.5) Q super(-0.28) ) over the replica-based parameter in correlating nasal-laryngeal depositions of ultrafine aerosols. Improved correlations have been developed for the four age groups by implementing this in vivo -based diffusion parameter as well as the Cunningham correction factor.},
  Doi                      = {http://dx.doi.org/10.1007/s10439-012-0603-7},
  ISSN                     = {0090-6964},
  Keywords                 = {Medical Sciences Solid State and Superconductivity Abstracts Mechanical & Transportation Engineering Abstracts ANTE: Abstracts in New Technologies and Engineering Biotechnology and Bioengineering Abstracts Birth Aerosols Age Respiration Diffusion Neonates Children Respiratory tract Models Infants Yes:(AN) W 30965:Miscellaneous
Reviews},
  Type                     = {Journal Article},
  Url                      = {http://download.springer.com/static/pdf/27/art%253A10.1007%252Fs10439-012-0603-7.pdf?auth66=1411539348_aca015a231b9bce1a25fe3c12d0ecc1d&ext=.pdf}
}

@Article{Xi2014,
  Title                    = {Effects of the facial interface on inhalation and deposition of micrometer particles in calm air in a child airway model},
  Author                   = {Xi, Jinxiang and Kim, JongWon and Si, Xiuhua A. and Su, Wei Chung and Zhou, Yue},
  Journal                  = {Inhalation Toxicology},
  Year                     = {2014},
  Note                     = {C:\Users\sean\AppData\Roaming\Zotero\Zotero\Profiles\16a4oype.default\zotero\storage\AFPBFDJE\Xi et al. - 2014 - Effects of the facial interface on inhalation and .pdf},
  Pages                    = {492-505},
  Volume                   = {26},

  Abstract                 = {Context: How the facial interface affects particle inhalability and depositions within the airway is not well understood. Previous studies of inhalation dosimetry are limited to either inhalability or deposition, rather than the two studied in a systematic way. Objective: To systematically evaluate the effects of the facial interface on aerosol inhalability, nasal deposition and thoracic dose in a 5-year-old child airway model using a coupled imaging-computational fluid dynamics approach. Methods: A face-nose-throat model was developed from magnetic resonance imaging scans of a 5-year-old boy. Respiration airflows and particle transport were simulated with the low Reynolds number k-omega turbulence model and the Lagrangian tracking approach. Particles ranging from 1 to 70 mm were considered in a calm air. Results: Retaining the facial interface in the computational model induced substantial variations in flow dynamics, aerosol inhalability and thoracic doses. The nasal and thoracic deposition fractions were much lower with the facial interface due to the low inhalability into downward-facing nostrils and facial deposition losses. For a given inhalation rate of 10 L/min, including the facial interface reduced the thoracic dose by 5% for 2.5-mm particles and by 50% for 10 mm particles in the child model. Considering localized conditions, facial interface substantially increased depositions at the turbinate region and dorsal pharynx. Conclusion: This study highlighted the need to include facial interface in future numerical and in vitro studies. Findings of this study have practical implications in the design of aerosol samplers and interpretation of deposition data from studies without facial interfaces.},
  Doi                      = {10.3109/08958378.2014.925992},
  ISSN                     = {0895-8378},
  Keywords                 = {aerosol deposition
aspiration efficiencies
child respiratory deposition
cigarette-smoke
facial interface
flow
Human nasal cavity
image-CFD modeling
inhalability
nanoparticle deposition
particle inhalability
respiratory-tract
samplers
ultrafine particles},
  Type                     = {Journal Article},
  Url                      = {http://informahealthcare.com/doi/pdfplus/10.3109/08958378.2014.925992}
}

@Article{Xi2013,
  Title                    = {Hygroscopic aerosol deposition in the human upper respiratory tract under various thermo-humidity conditions},
  Author                   = {Xi, Jinxiang and Kim, Jongwon and Si, Xiuhua A. and Zhou, Yue},
  Journal                  = {Journal of Environmental Science and Health, Part A},
  Year                     = {2013},
  Note                     = {C:\Users\sean\AppData\Roaming\Zotero\Zotero\Profiles\16a4oype.default\zotero\storage\3ZW8C2VA\Xi et al. - 2013 - Hygroscopic aerosol deposition in the human upper .pdf},
  Pages                    = {1790-1805},
  Volume                   = {48},

  Abstract                 = {The deposition of hygroscopic aerosols is highly complex in nature, which results from a cumulative effect of dynamic particle growth and the real-time size-specific deposition mechanisms. The objective of this study is to evaluate hygroscopic effects on the particle growth, transport, and deposition of nasally inhaled aerosols across a range of 0.2–2.5 μm in an adult image-based nose-throat model. Temperature and relative humidity fields were simulated using the LRN k-ω turbulence model and species transport model under a spectrum of thermo-humidity conditions. Particle growth and transport were simulated using a well validated Lagrangian tracking model coupled with a user-defined hygroscopic growth module. Results of this study indicate that the saturation level and initial particle size are the two major factors that determine the particle growth rate (d/d0), while the effect of inhalation flow rate is found to be not significant. An empirical correlation of condensation growth of nasally inhaled hygroscopic aerosols in adults has been developed based on a variety of thermo-humidity inhalation conditions. Significant elevated nasal depositions of hygroscopic aerosols could be induced by condensation growth for both sub-micrometer and small micrometer particulates. In particular, the deposition of initially 2.5 μm hygroscopic aerosols was observed to be 5–8 times that of inert particles under warm to hot saturated conditions. Results of this study have important implications in exposure assessment in hot humid environments, where much higher risks may be expected compared to normal conditions.},
  Doi                      = {10.1080/10934529.2013.823333},
  ISSN                     = {1093-4529},
  Type                     = {Journal Article},
  Url                      = {http://www.tandfonline.com/doi/pdf/10.1080/10934529.2013.823333}
}

@Article{Xi2008,
  Title                    = {Evaluation of a Drift Flux Model for Simulating Submicrometer Aerosol Dynamics in Human Upper Tracheobronchial Airways},
  Author                   = {Xi, Jinxiang and Longest, P.},
  Journal                  = {Annals of Biomedical Engineering},
  Year                     = {2008},
  Note                     = {10.1007/s10439-008-9552-6},
  Number                   = {10},
  Pages                    = {1714-1734},
  Volume                   = {36},

  Type                     = {Journal Article},
  Url                      = {http://dx.doi.org/10.1007/s10439-008-9552-6}
}

@Article{Xi2008a,
  Title                    = {Effects of Oral Airway Geometry Characteristics on the Diffusional Deposition of Inhaled Nanoparticles},
  Author                   = {Xi, J. and Longest, P.W.},
  Journal                  = {Journal of Biomechanical Engineering},
  Year                     = {2008},
  Pages                    = {011008-1-011008-16},
  Volume                   = {130},

  Type                     = {Journal Article}
}

@Article{Xi2009,
  Title                    = {Characterization of Submicrometer Aerosol Deposition in Extrathoracic Airways during Nasal Exhalation},
  Author                   = {Xi, Jinxiang and Longest, P. Worth},
  Journal                  = {Aerosol Science and Technology},
  Year                     = {2009},
  Number                   = {8},
  Pages                    = {808 - 827},
  Volume                   = {43},

  ISSN                     = {0278-6826},
  Type                     = {Journal Article},
  Url                      = {http://www.informaworld.com/10.1080/02786820902950887}
}

@Article{Xi2008b,
  Title                    = {Numerical predictions of submicrometer aerosol deposition in the nasal cavity using a novel drift flux approach},
  Author                   = {Xi, Jinxiang and Longest, P. Worth},
  Journal                  = {International Journal of Heat and Mass Transfer},
  Year                     = {2008},
  Number                   = {23-24},
  Pages                    = {5562-5577},
  Volume                   = {51},

  Doi                      = {10.1016/j.ijheatmasstransfer.2008.04.037},
  ISSN                     = {0017-9310},
  Keywords                 = {Respiratory particle dynamics
Respiratory dosimetry
Microdosimetry
Health effects of submicrometer aerosols
Respiratory drug delivery
Drift flux model
Nasal deposition of fine and ultrafine aerosols},
  Type                     = {Journal Article},
  Url                      = {http://www.sciencedirect.com/science/article/pii/S0017931008002676}
}

@Article{Xi2011,
  Title                    = {Simulation of airflow and aerosol deposition in the nasal cavity of a 5-year-old child},
  Author                   = {Xi, Jinxiang and Si, Xiuhua and Kim, Jong Won and Berlinski, Ariel},
  Journal                  = {Journal of Aerosol Science},
  Year                     = {2011},
  Number                   = {3},
  Pages                    = {156-173},
  Volume                   = {42},

  Abstract                 = {As a human grows from birth to adulthood, both airway anatomy and breathing conditions vary that alter the deposition rate and pattern of inhaled aerosols. However, deposition studies have typically focused on adult subjects, results of which may not be readily extrapolated to children. Furthermore, because of greater ventilation rate per body weight, children receive a greater dose than adults and therefore are more susceptible to respiratory risks. This study is to evaluate the transport and deposition of respiratory aerosols in a nasal-laryngeal airway model based on MRI head images of a 5-year-old boy. Differences between this child and adults in nasal physiology and aerosol filtering efficiency will be emphasized. A validated low Reynolds number (LRN) k−ω turbulence model was employed to simulate laminar, transitional, and fully turbulent flow regimes within the nasal airways. Particle trajectories and deposition in the spectrum of 0.5–32 μm were evaluated using a well-tested Lagrangian tracking approach for inhalation flow rates ranging from sedentary (3 L/min) to heavily active (30 L/min) conditions. Simulation results of the inhalation pressure drop and particle deposition rate provided a reasonable match with existing experimental results in nasal airway casts of children. Much higher breathing resistance was observed in the 5-year-old child compared to adults. Furthermore, deposition patterns were sensitive to inhalation flow rate under low activity conditions. An empirical correlation of child nasal filtering efficiency was proposed for micrometer particles based on a wide range of test conditions. Results of this study demonstrate that significant child–adult difference exists in inhaled aerosol depositions, which should be taken into account for risk assessment of airborne toxicants on infants and children.},
  Doi                      = {http://dx.doi.org/10.1016/j.jaerosci.2010.12.004},
  ISSN                     = {0021-8502},
  Keywords                 = {Nasal physiology
Child–adult difference
Child airway model
Breathing resistance
Microdosimetry
Nasal deposition},
  Type                     = {Journal Article},
  Url                      = {http://www.sciencedirect.com/science/article/pii/S0021850210002405}
}

@Article{Xi2011a,
  Title                    = {Simulation of airflow and aerosol deposition in the nasal cavity of a 5-year-old child},
  Author                   = {Xi, Jinxiang and Si, Xiuhua and Kim, Jong Won and Berlinski, Ariel},
  Journal                  = {Journal of Aerosol Science},
  Year                     = {2011},
  Number                   = {3},
  Pages                    = {156-173},
  Volume                   = {42},

  Abstract                 = {As a human grows from birth to adulthood, both airway anatomy and breathing conditions vary that alter the deposition rate and pattern of inhaled aerosols. However, deposition studies have typically focused on adult subjects, results of which may not be readily extrapolated to children. Furthermore, because of greater ventilation rate per body weight, children receive a greater dose than adults and therefore are more susceptible to respiratory risks. This study is to evaluate the transport and deposition of respiratory aerosols in a nasal-laryngeal airway model based on MRI head images of a 5-year-old boy. Differences between this child and adults in nasal physiology and aerosol filtering efficiency will be emphasized. A validated low Reynolds number (LRN) k-[omega] turbulence model was employed to simulate laminar, transitional, and fully turbulent flow regimes within the nasal airways. Particle trajectories and deposition in the spectrum of 0.5-32 [mu]m were evaluated using a well-tested Lagrangian tracking approach for inhalation flow rates ranging from sedentary (3 L/min) to heavily active (30 L/min) conditions. Simulation results of the inhalation pressure drop and particle deposition rate provided a reasonable match with existing experimental results in nasal airway casts of children. Much higher breathing resistance was observed in the 5-year-old child compared to adults. Furthermore, deposition patterns were sensitive to inhalation flow rate under low activity conditions. An empirical correlation of child nasal filtering efficiency was proposed for micrometer particles based on a wide range of test conditions. Results of this study demonstrate that significant child-adult difference exists in inhaled aerosol depositions, which should be taken into account for risk assessment of airborne toxicants on infants and children.},
  Doi                      = {10.1016/j.jaerosci.2010.12.004},
  ISSN                     = {0021-8502},
  Keywords                 = {Nasal physiology
Child-adult difference
Child airway model
Breathing resistance
Microdosimetry
Nasal deposition},
  Type                     = {Journal Article},
  Url                      = {http://www.sciencedirect.com/science/article/pii/S0021850210002405}
}

@Article{Xi2011b,
  Title                    = {Simulation of airflow and aerosol deposition in the nasal cavity of a 5-year-old child},
  Author                   = {Xi, Jinxiang and Si, Xiuhua and Kim, Jong Won and Berlinski, Ariel},
  Journal                  = {Journal of Aerosol Science},
  Year                     = {2011},
  Note                     = {C:\Users\sean\AppData\Roaming\Zotero\Zotero\Profiles\16a4oype.default\zotero\storage\CS8NC47W\Xi et al. - 2011 - Simulation of airflow and aerosol deposition in th.pdf},
  Pages                    = {156-173},
  Volume                   = {42},

  Abstract                 = {As a human grows from birth to adulthood, both airway anatomy and breathing conditions vary that alter the deposition rate and pattern of inhaled aerosols. However, deposition studies have typically focused on adult subjects, results of which may not be readily extrapolated to children. Furthermore, because of greater ventilation rate per body weight, children receive a greater dose than adults and therefore are more susceptible to respiratory risks. This study is to evaluate the transport and deposition of respiratory aerosols in a nasal-laryngeal airway model based on MRI head images of a 5-year-old boy. Differences between this child and adults in nasal physiology and aerosol filtering efficiency will be emphasized. A validated low Reynolds number (LRN) k−ω turbulence model was employed to simulate laminar, transitional, and fully turbulent flow regimes within the nasal airways. Particle trajectories and deposition in the spectrum of 0.5–32 μm were evaluated using a well-tested Lagrangian tracking approach for inhalation flow rates ranging from sedentary (3 L/min) to heavily active (30 L/min) conditions. Simulation results of the inhalation pressure drop and particle deposition rate provided a reasonable match with existing experimental results in nasal airway casts of children. Much higher breathing resistance was observed in the 5-year-old child compared to adults. Furthermore, deposition patterns were sensitive to inhalation flow rate under low activity conditions. An empirical correlation of child nasal filtering efficiency was proposed for micrometer particles based on a wide range of test conditions. Results of this study demonstrate that significant child–adult difference exists in inhaled aerosol depositions, which should be taken into account for risk assessment of airborne toxicants on infants and children.},
  Doi                      = {10.1016/j.jaerosci.2010.12.004},
  ISSN                     = {0021-8502},
  Keywords                 = {Breathing resistance
Child–adult difference
Child airway model
Microdosimetry
Nasal deposition
Nasal physiology},
  Type                     = {Journal Article},
  Url                      = {http://ac.els-cdn.com/S0021850210002405/1-s2.0-S0021850210002405-main.pdf?_tid=105a7bea-4220-11e4-9a30-00000aab0f26&acdnat=1411366800_d03bd10cde8575812783ffd6d905b36a}
}

@Article{Xi2014a,
  Title                    = {Growth of nasal and laryngeal airways in children: implications in breathing and inhaled aerosol dynamics},
  Author                   = {Xi, Jinxiang and Si, Xiuhua and Zhou, Yue and Kim, JongWon and Berlinski, Ariel},
  Journal                  = {Respiratory Care},
  Year                     = {2014},

  Month                    = {2014/02//},
  Number                   = {2},
  Pages                    = {263+},
  Volume                   = {59},

  ISSN                     = {00201324},
  Keywords                 = {Aerosols
Airway
Child development
Larynx
Nose
Physiological research
Respiration},
  Type                     = {Magazine Article},
  Url                      = {http://go.galegroup.com/ps/i.do?id=GALE%7CA361242095&v=2.1&u=rmit&it=r&p=AONE&sw=w&asid=5ff715029f5408bb40c0e8bc1b4df3e6}
}

@Article{Xi2014b,
  Title                    = {Growth of nasal and laryngeal airways in children: implications in breathing and inhaled aerosol dynamics},
  Author                   = {Xi, Jinxiang and Si, Xiuhua and Zhou, Yue and Kim, JongWon and Berlinski, Ariel},
  Journal                  = {Respiratory Care},
  Year                     = {2014},

  Month                    = {2014/02//},
  Number                   = {2},
  Pages                    = {263+},
  Volume                   = {59},

  ISSN                     = {00201324},
  Keywords                 = {Aerosols
Airway
Child development
Larynx
Nose
Physiological research
Respiration},
  Type                     = {Magazine Article},
  Url                      = {http://go.galegroup.com/ps/i.do?id=GALE%7CA361242095&v=2.1&u=rmit&it=r&p=AONE&sw=w&asid=5ff715029f5408bb40c0e8bc1b4df3e6}
}

@Article{Xi2014c,
  Title                    = {Growth of nasal and laryngeal airways in children: implications in breathing and inhaled aerosol dynamics},
  Author                   = {Xi, Jinxiang and Si, Xiuhua and Zhou, Yue and Kim, JongWon and Berlinski, Ariel},
  Journal                  = {Respiratory Care},
  Year                     = {2014},

  Month                    = {February 2014},
  Pages                    = {263+},
  Volume                   = {59},

  ISSN                     = {00201324},
  Keywords                 = {Aerosols
Airway
Child development
Larynx
nose
Physiological research
Respiration},
  Type                     = {Magazine Article},
  Url                      = {http://rc.rcjournal.com/content/59/2/263.short}
}

@Article{Xia2008,
  Title                    = {An unstructured finite volume approach for structural dynamics in response to fluid motions},
  Author                   = {Xia, Guohua and Lin, Ching-Long},
  Journal                  = {Computers \& Structures},
  Year                     = {2008},
  Number                   = {7-8},
  Pages                    = {684-701},
  Volume                   = {86},

  Abstract                 = {A new cell-vortex unstructured finite volume method for structural dynamics is assessed for simulations of structural dynamics in response to fluid motions. A robust implicit dual-time stepping method is employed to obtain time accurate solutions. The resulting system of algebraic equations is matrix-free and allows solid elements to include structure thickness, inertia, and structural stresses for accurate predictions of structural responses and stress distributions. The method is coupled with a fluid dynamics solver for fluid-structure interaction, providing a viable alternative to the finite element method for structural dynamics calculations. A mesh sensitivity test indicates that the finite volume method is at least of second-order accuracy. The method is validated by the problem of vortex-induced vibration of an elastic plate with different initial conditions and material properties. The results are in good agreement with existing numerical data and analytical solutions. The method is then applied to simulate a channel flow with an elastic wall. The effects of wall inertia and structural stresses on the fluid flow are investigated.},
  ISSN                     = {0045-7949},
  Keywords                 = {Finite volume method
Structural dynamics
Vortex-induced plate vibration
Channel flow with elastic walls},
  Type                     = {Journal Article},
  Url                      = {http://www.sciencedirect.com/science/article/B6V28-4PK89Y9-2/2/d6df229b4d5e7f793596a95cf2d53418}
}

@Article{Xiong2008,
  Title                    = {Computational fluid dynamics simulation of airflow in the normal nasal cavity and paranasal sinuses},
  Author                   = {Xiong, G.X. and Zhan, J.M. and Jiang, H.Y. and Li, J.F. and Rong, L.W. and Xu, G.},
  Journal                  = {American Journal of Rhinology},
  Year                     = {2008},
  Pages                    = {477-482},
  Volume                   = {22},

  Doi                      = {doi: 10.2500/ajr.2008.22.3211},
  Type                     = {Journal Article},
  Url                      = {http://proquest.umi.com/pqdlink?index=5&did=1575078391&SrchMode=3&sid=1&Fmt=6&VInst=PROD&VType=PQD&RQT=309&VName=PQD&TS=1242000702&clientId=16532&aid=1}
}

@Article{Xiong2008a,
  Title                    = {Numerical flow simulation in the post-endoscopic sinus surgery nasal cavity},
  Author                   = {Xiong, G. and Zhan, J. and Zuo, K. and Li, J. and Rong, L. and Xu, G.},
  Journal                  = {Medical and Biological Engineering and Computing},
  Year                     = {2008},
  Number                   = {11},
  Pages                    = {1161-1167},
  Volume                   = {46},

  Type                     = {Journal Article}
}

@Article{Xu2006,
  Title                    = {Computational fluid dynamics modeling of the upper airway of children with obstructive sleep apnea syndrome in steady flow},
  Author                   = {Xu, Chun and Sin, SangHun and McDonough, Joseph M. and Udupa, Jayaram K. and Guez, Allon and Arens, Raanan and Wootton, David M.},
  Journal                  = {Journal of Biomechanics},
  Year                     = {2006},
  Note                     = {doi: DOI: 10.1016/j.jbiomech.2005.06.021},
  Number                   = {11},
  Pages                    = {2043-2054},
  Volume                   = {39},

  ISSN                     = {0021-9290},
  Keywords                 = {Magnetic resonance imaging (MRI)
Pharynx
Human
Pressure
Flow resistance},
  Type                     = {Journal Article},
  Url                      = {http://www.sciencedirect.com/science/article/B6T82-4GV9SMK-1/2/e637df0892db095959f7b703f4c1ac45}
}

@Article{Xue2006,
  Title                    = {Modeling of enhanced penetrant diffusion in nanoparticle-polymer composite membranes},
  Author                   = {Xue, Liping and Borodin, Oleg and Smith, Grant D.},
  Journal                  = {Journal of Membrane Science},
  Year                     = {2006},
  Number                   = {1-2},
  Pages                    = {293-300},
  Volume                   = {286},

  Abstract                 = {We have utilized 2D material point method (MPM) to study the influence of nanoparticles on the diffusivity of penentrants in model polymer membranes comprised of impenetrable spherical nanoparticles dispersed in a matrix with uniform penetrant solubility and diffusivity. Diffusion in the nanoparticle-polymer composite membrane was enhanced by the presence of a thin "skin" of matrix material next to the surface of the nanoparticles with a penetrant diffusion coefficient 100 times that of the bulk matrix. The influence of the skin thickness, nanoparticle area fraction and the manner in which the nanoparticles were distributed in the membrane on penetrant diffusion was studied. For a given skin thickness the penetrant diffusion in the composite membrane was found to increase exponentially with increasing area fraction of nanoparticles both above and below the percolation threshold. Membranes with clustered nanoparticles were found to have higher penetrants diffusivity than the membranes with fully dispersed particles due to the formation of anisotropic nanoparticle clusters. We found the total area fraction of particles + enhanced matrix (skin) to be a valid scaling variable for the effective diffusion coefficient of the nanoparticle-polymer composite membranes for the entire range of skin thickness and nanoparticle loadings investigated. Implications of the observed dependencies of effective penetrant diffusivity in the composite membrane on nanoparticle loading, skin thickness and nanoparticle morphology are discussed.},
  ISSN                     = {0376-7388},
  Keywords                 = {Nanoparticle-polymer composite membranes
Nanoparticle enhanced diffusion
Gas separation
Computer simulation
Material point method},
  Type                     = {Journal Article},
  Url                      = {http://www.sciencedirect.com/science/article/B6TGK-4M34S4K-2/2/7c9eeda160f2fe50ad5f3a9e4f99385d}
}

@Article{Yadav2001,
  Title                    = {Nasal mucociliary clearance in healthy children in a tropical country},
  Author                   = {Yadav, Jyoti and Ranga, Rupender K. and Singh, Jagat and Gathwala, Geeta},
  Journal                  = {International Journal of Pediatric Otorhinolaryngology},
  Year                     = {2001},
  Number                   = {1},
  Pages                    = {21-24},
  Volume                   = {57},

  Abstract                 = {Introduction: Nasal mucociliary clearance is an important physiological function of nasal cavity that helps in protecting the lower respiratory tract from undesirable organic and inorganic matter including the micro organisms. The study was designed to establish normal mucociliary clearance time in healthy children in a tropical environment. Material and methods: The study was carried out in 100 randomly selected normal school children aged 4–15 years using saccharin method. The diseases that are known to affect the nasal mucociliary clearance were excluded. The study variables were age and sex. Results: Mean nasal mucociliary clearance time was 5.7±2.59 min with no significant difference between males and females. The clearance time was found to be impaired in groups A and B including children of 4–7 and 8–11 years of age respectively. Conclusion: Nasal mucociliary clearance is impaired in children of either sex between 4 and 11 years probably due to subclinical adenoiditis. However clearance returns to normal level at the time of puberty, which coincides with adenoids involution.},
  Doi                      = {http://dx.doi.org/10.1016/S0165-5876(00)00429-8},
  ISSN                     = {0165-5876},
  Keywords                 = {Mucociliary clearance
Saccharin
Children
Adenoidal hypertrophy},
  Type                     = {Journal Article},
  Url                      = {http://www.sciencedirect.com/science/article/pii/S0165587600004298}
}

@Article{Yaglom1974,
  Title                    = {HEAT AND MASS TRANSFER BETWEEN A ROUGH WALL AND TURBULENT FLUID FLOW AT HIGH REYNOLDS AND PECLET NUMBERS},
  Author                   = {Yaglom, A. M. and Kader, B. A.},
  Journal                  = {Journal of Fluid Mechanics},
  Year                     = {1974},
  Note                     = {Cited By (since 1996): 21
Export Date: 6 June 2011
Source: Scopus},
  Number                   = {Part 3},
  Pages                    = {601-623},
  Volume                   = {62},

  Type                     = {Journal Article},
  Url                      = {http://www.scopus.com/inward/record.url?eid=2-s2.0-0016026086&partnerID=40&md5=4909529ba44b2bd01bccda9eb5ed61e4}
}

@Article{Yakhot1986,
  Title                    = {Renormalization group analysis of turbulence. I. Basic theory},
  Author                   = {Yakhot, Victor and Orszag, Steven A.},
  Journal                  = {Journal of Scientific Computing},
  Year                     = {1986},
  Number                   = {1},
  Pages                    = {3-51},
  Volume                   = {1},

  Doi                      = {10.1007/bf01061452},
  ISSN                     = {0885-7474},
  Keywords                 = {Mathematics and Statistics},
  Type                     = {Journal Article},
  Url                      = {http://dx.doi.org/10.1007/BF01061452}
}

@Article{Yamada1994,
  Title                    = {Deposition of Ultrafine Monodisperse Particles in A Human Tracheobronchial Cast},
  Author                   = {Yamada, Y. and Koizumi, A. and Fukuda, S. and Inaba, J. and Cheng, Y. S. and Yeh, H. C.},
  Journal                  = {Ann Occup Hyg},
  Year                     = {1994},
  Number                   = {inhaled_particles_VII},
  Pages                    = {91-100},
  Volume                   = {38},

  Abstract                 = {Total deposition in a human tracheobronchial cast was measured for ultrafine monodisperse particles below 0.1 {micro}m in diameter at flow rates of 4,10 and 201. min-1 of constant and cyclic flows. The cast model consisted of the trachea and bronchial airways with branching generations from 4 to 10. At 201. min-1 constant flow, inspiratory deposition efficiencies were 5 and 44% for 0.1 and 0.00475 {micro}m size particles, respectively. Deposition efficiency increased with decreasing particle size and flow rate. Compared with inspiratory deposition, expiratory deposition efficiencies were slightly higher for particles below 0.01 {micro}m. Deposition efficiencies under cyclic flow were found to be almost the same as those under constant flow having equivalent mean flow rate. It was also observed that varying the tidal volume, but keeping the minute volume constant (i.e. having the same mean flow rate), has little effect on deposition of ultrafine particles.},
  Doi                      = {10.1093/annhyg/38.inhaled_particles_VII.91},
  Type                     = {Journal Article},
  Url                      = {http://annhyg.oxfordjournals.org/cgi/content/abstract/38/inhaled_particles_VII/91}
}

@Article{Yan2004,
  Title                    = {Dynamic modelling and simulation for nasal drug delivery},
  Author                   = {Yan, J. and Karanjkar, A. and Branagan, M.},
  Journal                  = {PMPS-Drug Delivery},
  Year                     = {2004},
  Pages                    = {68-71},
  Volume                   = {August},

  Type                     = {Journal Article}
}

@Article{Yang2006,
  Title                    = {The effect of inlet velocity profile on the bifurcation COPD airway flow},
  Author                   = {Yang, X.L. and Liu, L. and So, R.M.C. and Yang, J.M.},
  Journal                  = {Computers in Biology and Medicine},
  Year                     = {2006},
  Number                   = {2},
  Pages                    = {181-194},
  Volume                   = {36},

  Type                     = {Journal Article}
}

@Article{Yang2006a,
  Title                    = {Respiratory flow in obstructed airways},
  Author                   = {Yang, X. L. and Liu, Yang and Luo, H. Y.},
  Journal                  = {Journal of Biomechanics},
  Year                     = {2006},
  Note                     = {doi: DOI: 10.1016/j.jbiomech.2005.10.009},
  Number                   = {15},
  Pages                    = {2743-2751},
  Volume                   = {39},

  ISSN                     = {0021-9290},
  Keywords                 = {Respiratory flow
COPD
Four-generation
CFD},
  Type                     = {Journal Article},
  Url                      = {http://www.sciencedirect.com/science/article/B6T82-4HM7S5G-1/2/afabf972c5f37c81a2ad7f9c148c1746}
}

@Article{Yang2006b,
  Title                    = {The effect of inlet velocity profile on the bifurcation COPD airway flow},
  Author                   = {Yang, X. L. and Liu, Y. and So, R. M. C. and Yang, J. M.},
  Journal                  = {Computers in Biology and Medicine},
  Year                     = {2006},
  Number                   = {2},
  Pages                    = {181-194},
  Volume                   = {36},

  Abstract                 = {The effect of inlet velocity profile on the flow features in obstructed airways is investigated in this study. In reality, the inlet velocity distributions on such models, which are extracted from medial branches of natural human lung, should be neither uniform, nor symmetric parabolic, but skewed-parabolic due to having been skewed by the upper carina ridges. Four different three-dimensional three-generation models based on the 23 generations model of Weibel have been considered, respectively. The fully three-dimensional incompressible laminar Navier-Stokes equations and continuity equation have been solved using CFD solver on unstructured tetrahedral meshes. To reduce the complexity of the simulations, only one Reynolds number of 900 was used in this calculation. Four types of inlet boundary conditions, namely uniform, parabolic, positive-skewed parabolic (skewed to the positive x-direction), and negative-skewed parabolic, were imposed on the obstructed airway models, which were considered to be obstructed at either the second generation or the third generation airways, respectively. The results show that the inlet velocity profile has significant influence on the flow patterns, mass distributions, and pressure drops in either the symmetric model, or the three obstructed models. The three generation airways may not be enough to study the bifurcation flow in chronic obstructive pulmonary disease (COPD) airways, and a four-generation or more airway model is necessary to get better predictive results.},
  ISSN                     = {0010-4825},
  Keywords                 = {Respiratory flow
COPD
Boundary condition},
  Type                     = {Journal Article},
  Url                      = {http://www.sciencedirect.com/science/article/B6T5N-4F4H9NK-1/2/9567606da54778a1bdef811f29a97ab6}
}

@Article{Yang2007,
  Title                    = {Fluid-structure interaction in a pulmonary arterial bifurcation},
  Author                   = {Yang, X. L. and Liu, Y. and Yang, J. M.},
  Journal                  = {Journal of Biomechanics},
  Year                     = {2007},
  Number                   = {12},
  Pages                    = {2694-2699},
  Volume                   = {40},

  Abstract                 = {A numerical method is developed to solve the fluid-structure interaction in three-dimensional pulmonary arterial bifurcation with collapsible tubes. A self-developed FEM code is used to calculate the nonlinear deformation of the thin-walled structure and a commercial CFD solver, FLUENT, is used to resolve the fluid flow. The large deformation of the structure alters the flow field significantly while the fluid pressure affects the deformation of the structure strongly. In the bifurcation branches, the relatively short tube collapses into wave number N=3 mode. The strong collapse of the branch tube leads to a large contraction of the cross-sectional area and increases the resistance on the fluid flow. The recirculation occurs at both the up- and down-stream of the collapsed tube.},
  ISSN                     = {0021-9290},
  Keywords                 = {Pulmonary arterial bifurcation
Fluid-structure interaction
Collapsible tubes},
  Type                     = {Journal Article},
  Url                      = {http://www.sciencedirect.com/science/article/B6T82-4N6FVGJ-1/2/5f1131f1bf2797f1606939af9793cafa}
}

@Article{Yanosky2002,
  Title                    = {A comparison of two direct-reading aerosol monitors with the federal reference method for PM2.5 in indoor air},
  Author                   = {Yanosky, Jeff D. and Williams, Phillip L. and MacIntosh, David L.},
  Journal                  = {Atmospheric Environment},
  Year                     = {2002},
  Number                   = {1},
  Pages                    = {107-113},
  Volume                   = {36},

  Abstract                 = {Two types of direct-reading aerosol monitoring devices, the TSI, Inc. Model 3320 Aerodynamic Particle Sizer (APS), and the TSI, Inc. Model 8520 DustTrak Aerosol Monitor (DustTrak), were collocated indoors with a US EPA designated Federal Reference Method (FRM) PM2.5 sampler, the BGI, Inc. PQ200, to assess the comparability of the sampling methods. Simultaneous 24-h samples were collected from two APS instruments, one DustTrak and one FRM sampler for 20 sample periods. The 30-min average concentrations during the 24-hour sample periods were also logged and compared for the APS and DustTrak. Statistical analysis on the mass concentrations obtained from each sampler type included paired t-tests and linear regression. The 24-h average PM2.5 levels from the FRM samplers were approximately normally distributed and ranged from 5.0 to 20.4 μg m−3 with mean and standard deviation 11.4 and 4.0 μg m−3, respectively. The 24-h average DustTrak levels are well correlated with FRM levels (R2=0.859) but show significant proportional bias (β1=2.57, p&lt;0.0001). The 24-h average mean collocated APS levels are less highly correlated with the FRM (R2=0.592) and do not show statistically significant proportional bias. The 30-min average levels between the two APS instruments show a high correlation (R2=0.979) but significant proportional bias (β1=1.31, p&lt;0.0001). The results suggest that though the DustTrak provides precise measurements of PM2.5, the accuracy of the measurements compared to the FRM can be improved through statistical adjustment. In contrast, APS PM2.5 measurements are less precise and less accurate compared to the FRM and therefore results from the APS should be interpreted with caution.},
  Doi                      = {http://dx.doi.org/10.1016/S1352-2310(01)00422-8},
  ISSN                     = {1352-2310},
  Keywords                 = {Aerosol sampling
Particulate matter
Aerodynamic particle sizer
DustTrak Aerosol Monitor
Federal reference method},
  Type                     = {Journal Article},
  Url                      = {http://www.sciencedirect.com/science/article/pii/S1352231001004228}
}

@Article{Yeh1980,
  Title                    = {Models of human lung airways and their application to inhaled particle deposition},
  Author                   = {Yeh, H.-C. and Schum, G. M.},
  Journal                  = {Bulletin of Mathematical Biology},
  Year                     = {1980},
  Pages                    = {461-480},
  Volume                   = {42},

  Type                     = {Journal Article}
}

@Book{Yeoh2009,
  Title                    = {Computational Techniques for Multiphase Flows},
  Author                   = {Yeoh, Guan Heng and Tu, Jiyuan},
  Publisher                = {Elsevier Science \& Technology},
  Year                     = {2009},

  Address                  = {Oxford, GB},

  Type                     = {Book}
}

@Article{Yerry1984,
  Title                    = { A modified-quadtree approach to finite element mesh generation,},
  Author                   = {Yerry, M. and Shephard, M.},
  Journal                  = {IEEE Computer Graphics Applications},
  Year                     = {1984},
  Number                   = {1},
  Pages                    = {39-46},
  Volume                   = {3},

  Type                     = {Journal Article}
}

@Article{Yeung2002,
  Title                    = {LAGRANGIAN INVESTIGATIONS OF TURBULENCE},
  Author                   = {Yeung, P. K.},
  Journal                  = {Annual Review of Fluid Mechanics},
  Year                     = {2002},
  Number                   = {1},
  Pages                    = {115-142},
  Volume                   = {34},

  Doi                      = {doi:10.1146/annurev.fluid.34.082101.170725},
  Type                     = {Journal Article},
  Url                      = {http://arjournals.annualreviews.org/doi/abs/10.1146/annurev.fluid.34.082101.170725}
}

@Article{Yilmaz2006,
  Title                    = {Rhinitis in the elderly},
  Author                   = {Yilmaz,A. A. and Corey, Jacquelynne P, MD,F.A.C.S., F.A.A.O.A.},
  Journal                  = {Current Allergy and Asthma Reports},
  Year                     = {2006},
  Note                     = {Copyright - Current Science Inc. 2006; Last updated - 2014-08-30},
  Number                   = {2},
  Pages                    = {125-31},
  Volume                   = {6},

  Abstract                 = {The effects of aging on the nose include structural, hormonal, mucosal, olfactory, and neural changes. As the US population ages and remains in overall better health, we will have more patients with rhinologic problems related to aging. In this manuscript, we review the available evidence on the structural and physiologic changes of the nose caused by aging, and we briefly describe management of common causes of rhinitis in the elderly.PUBLICATION ABSTRACT]},
  ISBN                     = {15297322},
  Keywords                 = {Medical Sciences--Respiratory Diseases; Aging; Rhinitis, Allergic, Perennial -- physiopathology; Nose -- pathology; Humans; Aged; Smell -- physiology; Rhinitis, Allergic, Perennial -- therapy; Rhinitis -- physiopathology; Rhinitis -- etiology; Rhinitis -- pathology; Rhinitis -- therapy},
  Language                 = {English},
  Url                      = {http://search.proquest.com/docview/879472979?accountid=13552}
}

@Article{Yin2010,
  Title                    = {Simulation of pulmonary air flow with a subject-specific boundary condition},
  Author                   = {Yin, Youbing and Choi, Jiwoong and Hoffman, Eric A. and Tawhai, Merryn H. and Lin, Ching-Long},
  Journal                  = {Journal of Biomechanics},
  Year                     = {2010},
  Number                   = {11},
  Pages                    = {2159-2163},
  Volume                   = {43},

  Abstract                 = {We present a novel image-based technique to estimate a subject-specific boundary condition (BC) for computational fluid dynamics (CFD) simulation of pulmonary air flow. The information of regional ventilation for an individual is derived by registering two computed tomography (CT) lung datasets and then passed to the CT-resolved airways as the flow BC. The CFD simulations show that the proposed method predicts lobar volume changes consistent with direct image-measured metrics, whereas the other two traditional BCs (uniform velocity or uniform pressure) yield lobar volume changes and regional pressure differences inconsistent with observed physiology.},
  Doi                      = {10.1016/j.jbiomech.2010.03.048},
  ISSN                     = {0021-9290},
  Keywords                 = {Computational fluid dynamics
Pulmonary air flow
Subject-specific boundary condition
Image registration
Finite element},
  Type                     = {Journal Article},
  Url                      = {http://www.sciencedirect.com/science/article/pii/S0021929010002137}
}

@Article{You2004,
  Title                    = {Motion of micro-particles in channel flow},
  Author                   = {You, C. F. and Li, G. H. and Qi, H. Y. and Xu, X. C.},
  Journal                  = {Atmospheric Environment},
  Year                     = {2004},
  Number                   = {11},
  Pages                    = {1559-1565},
  Volume                   = {38},

  Abstract                 = {The motion of micro-particles in channel flow was investigated using the DNS method for Re=3300 (based on center velocity and half channel width). The calculations used the fractional projection method to directly solve the Navier-Stokes equations, and the explicit third-order Runge-Kutta method for the time integration. A higher order finite difference scheme was used to discretize the fluid dynamics equations. The numerical results provided the two-phase flow field in the near-wall region with micro-particles. The simulation results showed that particles with Stokes number on the order of 0.1 tend to congregate near the wall while the micro-particles 1 [mu]m in diameter distribute homogeneously in the whole flow field. Simulations of micro-particulate matter movement using the [kappa]-[var epsilon] turbulence model applied to the same problem as the DNS analysis showed that the turbulence model could not accurately predict the micro-particle motion in the turbulent flow near the wall, with higher particle concentrations close to the centerline and lower concentrations near the wall.},
  ISSN                     = {1352-2310},
  Keywords                 = {Micro-particle
Particulate matter
DNS
Numerical simulation
Two-phase flow
Air pollution},
  Type                     = {Journal Article},
  Url                      = {http://www.sciencedirect.com/science/article/B6VH3-4BN0DT4-2/2/e9d4ae44cd58de640beee1255c26136f}
}

@Article{You2008,
  Title                    = {Active control of flow separation over an airfoil using synthetic jets},
  Author                   = {You, D. and Moin, P.},
  Journal                  = {Journal of Fluids and Structures},
  Year                     = {2008},
  Number                   = {8},
  Pages                    = {1349-1357},
  Volume                   = {24},

  Abstract                 = {We perform large-eddy simulation of turbulent flow separation over an airfoil and evaluate the effectiveness of synthetic jets as a separation control technique. The flow configuration consists of flow over an NACA 0015 airfoil at Reynolds number of 896,000 based on the airfoil chord length and freestream velocity. A small slot across the entire span connected to a cavity inside the airfoil is employed to produce oscillatory synthetic jets. Detailed flow structures inside the synthetic-jet actuator and the synthetic-jet/cross-flow interaction are simulated using an unstructured-grid finite-volume large-eddy simulation solver. Simulation results are compared with the 2005 experimental data of Gilarranz et al., and qualitative and quantitative agreements are obtained for both uncontrolled and controlled cases. As in the experiment, the present large-eddy simulation confirms that synthetic-jet actuation effectively delays the onset of flow separation and causes a significant increase in the lift coefficient. Modification of the blade boundary layer due to oscillatory blowing and suction and its role in separation control is discussed.},
  ISSN                     = {0889-9746},
  Keywords                 = {Separation control
Synthetic jets
Large-eddy simulation},
  Type                     = {Journal Article},
  Url                      = {http://www.sciencedirect.com/science/article/B6WJG-4TTM32R-3/2/55d6271bb0e03eaaabe979f0e612ef5a}
}

@Article{Youngentob1986,
  Title                    = {Effect of airway resistance on perceived odor intensity},
  Author                   = {Youngentob, Steven L. and Stern, Neil M. and Mozell, Maxwell M. and Leopold, Donald A. and Hornung, David E.},
  Journal                  = {American Journal of Otolaryngology},
  Year                     = {1986},
  Number                   = {3},
  Pages                    = {187-193},
  Volume                   = {7},

  Abstract                 = {Because of the wide range of human nasal anatomic configurations, some people sniff odorants against comparatively high resistances. To assess the relationship between sniff resistance and olfaction, ten subjects without nasal pathology or complaint were asked to estimate the perceived magnitude of the odorant, ethyl butyrate, at each of four concentrations and against each of four different resistances. In addition, the airflow profile of the subjects' sniffs was monitored during the performance of the odor task. As expected, perceived intensity increased with concentration, but more noteworthy was the finding that perceived intensity decreased with increasing resistance. Initially, this latter finding, together with the lack of interaction between concentration and resistance, suggested an olfactory analogue to conductive hearing losses. However, the sniffing data suggested that under the conditions of the experiment, the subjects attempted to maintain consistent sniffing behavior across the 16 different treatment combinations of concentration and resistance. These observations, taken together with the finding that subjects could estimate the perceived effort of sniffing, give support to the concept of a perceptual constancy model in olfaction. That is, olfactory magnitude may depend not only on the odorant itself, but also on the perceived effort associated with the sniff.},
  Doi                      = {http://dx.doi.org/10.1016/S0196-0709(86)80005-9},
  ISSN                     = {0196-0709},
  Type                     = {Journal Article},
  Url                      = {http://www.sciencedirect.com/science/article/pii/S0196070986800059}
}

@Article{Youngentob1986a,
  Title                    = {Effect of airway resistance on perceived odor intensity},
  Author                   = {Youngentob, Steven L. and Stern, Neil M. and Mozell, Maxwell M. and Leopold, Donald A. and Hornung, David E.},
  Journal                  = {American Journal of Otolaryngology},
  Year                     = {1986},
  Pages                    = {187-193},
  Volume                   = {7},

  Abstract                 = {Because of the wide range of human nasal anatomic configurations, some people sniff odorants against comparatively high resistances. To assess the relationship between sniff resistance and olfaction, ten subjects without nasal pathology or complaint were asked to estimate the perceived magnitude of the odorant, ethyl butyrate, at each of four concentrations and against each of four different resistances. In addition, the airflow profile of the subjects' sniffs was monitored during the performance of the odor task. As expected, perceived intensity increased with concentration, but more noteworthy was the finding that perceived intensity decreased with increasing resistance. Initially, this latter finding, together with the lack of interaction between concentration and resistance, suggested an olfactory analogue to conductive hearing losses. However, the sniffing data suggested that under the conditions of the experiment, the subjects attempted to maintain consistent sniffing behavior across the 16 different treatment combinations of concentration and resistance. These observations, taken together with the finding that subjects could estimate the perceived effort of sniffing, give support to the concept of a perceptual constancy model in olfaction. That is, olfactory magnitude may depend not only on the odorant itself, but also on the perceived effort associated with the sniff.},
  Doi                      = {http://dx.doi.org/10.1016/S0196-0709(86)80005-9},
  ISSN                     = {0196-0709},
  Type                     = {Journal Article}
}

@Article{Yu2009,
  Title                    = {Review of research on air-conditioning systems and indoor air quality control for human health},
  Author                   = {Yu, B. F. and Hu, Z. B. and Liu, M. and Yang, H. L. and Kong, Q. X. and Liu, Y. H.},
  Journal                  = {International Journal of Refrigeration},
  Year                     = {2009},
  Number                   = {1},
  Pages                    = {3-20},
  Volume                   = {32},

  Abstract                 = {With the improvement of standard of living, air-conditioning has widely been applied. However, health problems associated with air-conditioning systems and indoor air quality appear more frequently. In this paper, recent research is reviewed on air-conditioning systems and indoor air quality control for human health. The problems in the existing research are summarized. A further study is suggested on air-conditioning systems and indoor air quality control for healthy indoor air environment.},
  ISSN                     = {0140-7007},
  Keywords                 = {Air conditioning
Survey
Technology
Cooling system
Ventilation
Air quality
Health
Comfort
Human
Conditionnement d'air
Enquête
Technologie
Système frigorifique
Qualité de l'air
Santé
Confort
Homme},
  Type                     = {Journal Article},
  Url                      = {http://www.sciencedirect.com/science/article/B6V4R-4SHF4J4-2/2/a606568ef29dfabbafb9e5d3aa18a850}
}

@Article{Yu1998,
  Title                    = {Fluid flow and particle deposition in the human upper respiratory system},
  Author                   = {Yu, G. and Zhang, Z. and Lessman, R.},
  Journal                  = {Aerosol Science and Technology},
  Year                     = {1998},
  Pages                    = {146-158},
  Volume                   = {28},

  Type                     = {Journal Article}
}

@Article{Yu1998a,
  Title                    = {Fluid flow and particle diffusion in the human upper respiratory system},
  Author                   = {Yu, G. and Zhang, Z. and Lessmann, R.},
  Journal                  = {Aerosol Science and Technology},
  Year                     = {1998},
  Note                     = {Cited By (since 1996): 33
Export Date: 6 June 2011
Source: Scopus},
  Number                   = {2},
  Pages                    = {146-158},
  Volume                   = {28},

  Type                     = {Journal Article},
  Url                      = {http://www.scopus.com/inward/record.url?eid=2-s2.0-0032005964&partnerID=40&md5=0c51bc8210afdbc80a55b953ce0dfd7d}
}

@InBook{Yule1978,
  Title                    = {Measurement of Particle Size in Sprays by the Automated Analysis of Spark Photographs},
  Author                   = {Yule, A.J. and Cox, N.W. and Chigier, N.A.},
  Editor                   = {Groves, M.J.},
  Pages                    = {61-73},
  Year                     = {1978},

  Address                  = {Heyden, London},
  Type                     = {Book Section},

  Booktitle                = {Particle Size Analysis}
}

@Article{Yule2000,
  Title                    = {The performance characteristics of solid-cone-spray pressure-swirl atomizers.},
  Author                   = {Yule, A.J. and Sharief, R.A. and Jeong, J.R. and Nasr, G.G. and James, D.D.},
  Journal                  = {Atomization and Sprays},
  Year                     = {2000},
  Number                   = {6},
  Volume                   = {10},

  Type                     = {Journal Article}
}

@InBook{Yule1978a,
  Title                    = {Measurement of Particle Sizes in Sprays by the Automated Analysis of Spark Photographs},
  Author                   = {Yule, A. J. and Cox, N. W. and Chigier, N. A.},
  Editor                   = {Groves, M. J.},
  Pages                    = {61-73},
  Publisher                = {Heyden Press},
  Year                     = {1978},

  Address                  = {London},
  Type                     = {Book Section},

  Booktitle                = {Particle Size Analysis}
}

@Article{Yushkevich2006,
  Title                    = {User-guided 3D active contour segmentation of anatomical structures: Significantly improved efficiency and reliability},
  Author                   = {Yushkevich, P. A., Piven, J., Hazlett, H. C., Smith, R. G., Ho, S., Gee, J. C., Gerig, G.},
  Journal                  = {NeuroImage},
  Year                     = {2006},
  Pages                    = {1116-1128},
  Volume                   = {31},

  Type                     = {Journal Article}
}

@Article{ZA©licourt2009,
  Title                    = {Flow simulations in arbitrarily complex cardiovascular anatomies - An unstructured Cartesian grid approach},
  Author                   = {Zélicourt, Diane de and Ge, Liang and Wang, Chang and Sotiropoulos, Fotis and Gilmanov, Anvar and Yoganathan, Ajit},
  Journal                  = {Computers \& Fluids},
  Year                     = {2009},
  Number                   = {9},
  Pages                    = {1749-1762},
  Volume                   = {38},

  Abstract                 = {Image guided computational fluid dynamics is attracting increasing attention as a tool for refining in vivo flow measurements or predicting the outcome of different surgical scenarios. Sharp interface Cartesian/Immersed-Boundary methods constitute an attractive option for handling complex in vivo geometries but their capability to carry out fine-mesh simulations in the branching, multi-vessel configurations typically encountered in cardiovascular anatomies or pulmonary airways has yet to be demonstrated. A major computational challenge stems from the fact that when such a complex geometry is immersed in a rectangular Cartesian box the excessively large number of grid nodes in the exterior of the flow domain imposes an unnecessary burden on both memory and computational overhead of the Cartesian solver without enhancing the numerical resolution in the region of interest. For many anatomies, this added burden could be large enough to render comprehensive mesh refinement studies impossible. To remedy this situation, we recast the original structured Cartesian formulation of Gilmanov and Sotiropoulos [Gilmanov A, Sotiropoulos F. A hybrid Cartesian/immersed boundary method for simulating flows with 3D, geometrically complex, moving bodies. J Comput Phys 2005;207(2):457-92] into an unstructured Cartesian grid layout. This simple yet powerful approach retains the simplicity and computational efficiency of a Cartesian grid solver, while drastically reducing its memory footprint. The method is applied to carry out systematic mesh refinement studies for several internal flow problems ranging in complexity from flow in a 90° pipe bend to flow in an actual, patient-specific anatomy reconstructed from magnetic resonance images. Finally, we tackle the challenging clinical scenario of a single-ventricle patient with severe arterio-venous malformations, seeking to provide a fluid dynamics prospective on a clinical problem and suggestions for procedure improvements. Results from these simulations demonstrate very complex cardiovascular flow dynamics and underscore the need for high-resolution simulations prior to drawing any clinical recommendations.},
  ISSN                     = {0045-7930},
  Type                     = {Journal Article},
  Url                      = {http://www.sciencedirect.com/science/article/B6V26-4VYXMH3-1/2/fc9388baafd98f00eca573e2fbab668a}
}

@Article{ZaA¯di2010,
  Title                    = {Turbulence model choice for the calculation of drag forces when using the CFD method},
  Author                   = {Zaïdi, H. and Fohanno, S. and Taïar, R. and Polidori, G.},
  Journal                  = {Journal of Biomechanics},
  Year                     = {2010},
  Number                   = {3},
  Pages                    = {405-411},
  Volume                   = {43},

  Abstract                 = {The aim of this work is to specify which model of turbulence is the most adapted in order to predict the drag forces that a swimmer encounters during his movement in the fluid environment. For this, a Computational Fluid Dynamics (CFD) analysis has been undertaken with a commercial CFD code (Fluent®). The problem was modelled as 3D and in steady hydrodynamic state. The 3D geometry of the swimmer was created by means of a complete laser scanning of the swimmer's body contour. Two turbulence models were tested, namely the standard k-[epsilon] model with a specific treatment of the fluid flow area near the swimmer's body contour, and the standard k-[omega] model. The comparison of numerical results with experimental measurements of drag forces shows that the standard k-[omega] model accurately predicts the drag forces while the standard k-[epsilon] model underestimates their values. The standard k-[omega] model also enabled to capture the vortex structures developing at the swimmer's back and buttocks in underwater swimming; the same vortices had been visualized by flow visualization experiments carried out at the INSEP (National Institute for Sport and Physical Education in Paris) with the French national swimming team.},
  ISSN                     = {0021-9290},
  Keywords                 = {Underwater swimming
Turbulence model
Computational fluid dynamics},
  Type                     = {Journal Article},
  Url                      = {http://www.sciencedirect.com/science/article/B6T82-4XKXRS7-4/2/55c4b42915a415f0b5a587174598e67e}
}

@Article{Zachow2009,
  Title                    = {Visual Exploration of Nasal Airflow},
  Author                   = {Zachow, Stefan and Muigg, Philipp and Hildebrandt, Thomas and Doleisch, Helmut and Hege, Hans-Christian},
  Journal                  = {IEEE TRANSACTIONS ON VISUALIZATION AND COMPUTER GRAPHICS},
  Year                     = {2009},
  Pages                    = {1407-1414},
  Volume                   = {15},

  Abstract                 = {Rhinologists are often faced with the challenge of assessing nasal breathing from a functional point of view to derive effective therapeutic interventions. While the complex nasal anatomy can be revealed by visual inspection and medical imaging, only vague information is available regarding the nasal airflow itself: Rhinomanometry delivers rather unspecific integral information on the pressure gradient as well as on total flow and nasal flow resistance. In this article we demonstrate how the understanding of physiological nasal breathing can be improved by simulating and visually analyzing nasal airflow, based on an anatomically correct model of the upper human respiratory tract. In particular we demonstrate how various Information Visualization (InfoVis) techniques, such as a highly scalable implementation of parallel coordinates, time series visualizations, as well as unstructured grid multi-volume rendering, all integrated within a multiple linked views framework, can be utilized to gain a deeper understanding of nasal breathing. Evaluation is accomplished by visual exploration of spatio-temporal airflow characteristics that include not only information on flow features but also on accompanying quantities such as temperature and humidity. To our knowledge, this is the first in-depth visual exploration of the physiological function of the nose over several simulated breathing cycles under consideration of a complete model of the nasal airways, realistic boundary conditions, and all physically relevant time-varying quantities.},
  ISSN                     = {1077-2626},
  Keywords                 = {Analytical models
Anatomy
biomedical imaging
complex nasal anatomy
Computer Graphics
Computer Simulation
data analysis
data visualisation
exploratory data analysis
exploratory data analysis
flow visualisation
flow visualization
Flow visualization
Humans
Humidity
Image Processing, Computer-Assisted
Immune system
information visualization technique
Inspection
interactive systems
interactive visual analysis of scientific data
interactive visual analysis of scientific data
interactive visualFlow visualization
medical computing
medical imaging
Medical simulation
Models, Anatomic
nasal airflow visual exploration
nasal breathing
nose
physiological nasal breathing
Pulmonary Ventilation
rendering (computer graphics)
Respiration
rhinologists
Rhinomanometry
spatio-temporal airflow characteristics
temperature
therapeutic interventions
time-dependent data
time-dependent data.
time series
time series visualization
Tomography, X-Ray Computed
unstructured grid multivolume rendering
visual inspection
Visualization},
  Type                     = {Journal Article},
  Url                      = {http://www.computer.org/csdl/trans/tg/2009/06/ttg2009061407-abs.html}
}

@Article{Zachow2006,
  Title                    = {CFD simulation of nasal airflow: Towards treatment planning for functional rhinosurgery},
  Author                   = {Zachow, S. and Steinmann, A. and Hildebrandt, Th. and Weber, R. and Heppt, W.},
  Journal                  = {Int J Computer Assisted Radiology and Surgery},
  Year                     = {2006},
  Pages                    = {165-167},
  Volume                   = {Springer},

  Type                     = {Journal Article}
}

@Article{Zachow2006a,
  Title                    = {CFD simulation of nasal airflow: Towards treatment planning for functional rhinosurgery},
  Author                   = {Zachow, S. and Steinmann, A. and Hildebrandt, Th. and Weber, R. and Heppt, W.},
  Journal                  = {Int J Computer Assisted Radiology and Surgery},
  Year                     = {2006},
  Pages                    = {165-167},
  Volume                   = {Springer},

  Type                     = {Journal Article}
}

@Misc{Zahmatkesh2006,
  Title                    = {Numerical Simulation of Turbulent Airflow and Particle Deposition in Human Upper Oral Airway. FEDSM2006-98309},

  Author                   = {Zahmatkesh, I., Abouali, O. and Ahmadi, G. },
  Month                    = {July 17-20, 2006},
  Year                     = {2006},

  Type                     = {Conference Paper}
}

@Book{Zaichik2008,
  Title                    = {Particles in turbulent flows},
  Author                   = {Zaichik, L. and Alipchenkov, V.M. and Sinaiski, E.G.},
  Publisher                = {Wiley-VCH},
  Year                     = {2008},

  ISBN                     = {9783527407392},
  Type                     = {Book},
  Url                      = {http://books.google.com.au/books?id=IfuzWAd_aAAC}
}

@Article{Zakaria2006,
  Title                    = {Analysis of the importance of the ratio of aneurysm size to parent artery diameter on hemodynamic conditions},
  Author                   = {Zakaria, H. and Yonas, H. and Robertson, A. M.},
  Journal                  = {Journal of Biomechanics},
  Year                     = {2006},
  Number                   = {Supplement 1},
  Pages                    = {S272-S272},
  Volume                   = {39},

  ISSN                     = {0021-9290},
  Type                     = {Journal Article},
  Url                      = {http://www.sciencedirect.com/science/article/B6T82-4KR88PB-1G7/2/2dc571f826c05befa4eb74de9a88c90f}
}

@Article{Zamankhan2006,
  Title                    = {Airflow and Deposition of Nano-Particles in a Human Nasal Cavity},
  Author                   = {Zamankhan, P. and Ahmadi, G. and Wang, Z. and Hopke, P.K. and Cheng, Yung-Sung and Chung Su, W and Leonard, D.},
  Journal                  = {Aerosol Science and Technology},
  Year                     = {2006},

  Type                     = {Journal Article}
}

@Article{Zeka2006,
  Title                    = {Inflammatory markers and particulate air pollution: characterizing the pathway to disease},
  Author                   = {Zeka, Ariana and Sullivan, James R and Vokonas, Pantel S and Sparrow, David and Schwartz, Joel},
  Journal                  = {Int. Journal Epidemiol.},
  Year                     = {2006},
  Number                   = {5},
  Pages                    = {1347-1354},
  Volume                   = {35},

  Abstract                 = {Background Increased concentrations of particles in air have been related to changes in inflammatory markers that in turn are hypothesized in mediating the particle effects on cardiovascular disease. The present work examined this association in an elderly cohort in the Greater Boston area and addresses the relative role of particles from different sources. Methods The study included 710 subjects, active members of the VA Normative Aging Study cohort with measurements of blood markers. Concentrations of particle number (PN), black carbon (BC), fine particulate matter (PM2.5), and sulphates were measured at a central site near the examination site. Results Positive associations were found between traffic-related particles (PN and BC) and inflammatory markers, but only suggestive associations were found with exposures to PM2.5 and sulphates. The particle effect on the inflammatory markers was greater among subjects older than 78 years and among obese. A suggestion for a greater effect of particles on inflammatory markers among GSTM1-null subjects and non-users of statin drugs was also seen. Conclusions The findings of the study support the hypothesis that particles can induce cardiovascular disease through inflammatory pathways, suggestive of a greater toxicity of traffic-related particles.},
  Doi                      = {10.1093/ije/dyl132},
  Type                     = {Journal Article},
  Url                      = {http://ije.oxfordjournals.org/cgi/content/abstract/35/5/1347}
}

@Article{Zhai2006,
  Title                    = {Application of Computational Fluid Dynamics in Building Design: Aspects and Trends},
  Author                   = {Zhai, Zhiqiang},
  Journal                  = {Indoor and Built Environment},
  Year                     = {2006},
  Number                   = {4},
  Pages                    = {305-313},
  Volume                   = {15},

  Abstract                 = {Computational fluid dynamics (CFD), as the most sophisticated airflow modelling method, can simultaneously predict airflow, heat transfer and contaminant transportation in and around buildings. This paper introduces the roles of CFD in building design, demonstrating its typical application in designing a thermally conformable, healthy and energy-efficient building. The paper discusses the primary challenges of using CFD in the building modelling and design practice. Furthermore, it analyses the developing trends in applying CFD to building design, by thoroughly reviewing the literatures in all the proceedings of the International Conference on Building Simulation, one of the most influential symposiums in the building simulation field.},
  Doi                      = {10.1177/1420326x06067336},
  Type                     = {Journal Article},
  Url                      = {http://ibe.sagepub.com/content/15/4/305.abstract}
}

@Article{Zhai2007,
  Title                    = {Evaluation of various turbulence models in predicting airflow and turbulence in enclosed environments by CFD: part-1: summary of prevalent turbulence models},
  Author                   = {Zhai, Z. and Zhang, Z. and Zhang, W. and Chen, Q.},
  Journal                  = { HVAC\&R Research},
  Year                     = {2007},
  Number                   = {6},
  Pages                    = {853-870},
  Volume                   = {13},

  Type                     = {Journal Article}
}

@Article{ZhangZ,
  Author                   = {Zhang Z, Zhang W, Zhai Z, and Chen Q. 2007. Evaluation of Various Turbulence Models in Predicting Airflow and Turbulence in Enclosed Environments by CFD: Part-2: Comparison with Experimental Data from Literature. HVAC\&R Research, 13(6): 871-886.},

  Type                     = {Journal Article}
}

@Article{Zhang2000,
  Title                    = {Aerosol particle transport and deposition in vertical and horizontal turbulent duct flows},
  Author                   = {Zhang, H. and Ahmadi, G.},
  Journal                  = {Journal of Fluid Mechanics},
  Year                     = {2000},
  Note                     = {Cited By (since 1996): 52
Export Date: 6 June 2011
Source: Scopus},
  Pages                    = {55-80},
  Volume                   = {406},

  Type                     = {Journal Article},
  Url                      = {http://www.scopus.com/inward/record.url?eid=2-s2.0-0033936918&partnerID=40&md5=bd4fd02986f6f6da6ed3e71d00974b2e}
}

@Article{Zhang2001,
  Title                    = {Ellipsoidal particles transport and deposition in turbulent channel flows},
  Author                   = {Zhang, H. and Ahmadi, G. and Fan, F.G. and McLaughlin, J.B.},
  Journal                  = {International Journal Multiphase Flows},
  Year                     = {2001},
  Pages                    = {971-1009},
  Volume                   = {27},

  Type                     = {Journal Article}
}

@Article{Zhang2008,
  Title                    = {Computational fluid dynamics simulations of respiratory airflow in human nasal cavity and its characteristic dimension study},
  Author                   = {Zhang, J. and Liu, Y. and Sun, X. and Yu, S. and Yu, C.},
  Journal                  = {Acta Mech Sin},
  Year                     = {2008},
  Pages                    = {223-228},
  Volume                   = {24},

  Type                     = {Journal Article}
}

@Article{Zhang2007,
  Title                    = {Identification of contaminant sources in enclosed spaces by a single sensor},
  Author                   = {Zhang, T. and Chen, Q.},
  Journal                  = {Indoor Air},
  Year                     = {2007},
  Number                   = {6},
  Pages                    = {439-449},
  Volume                   = {17},

  ISSN                     = {1600-0668},
  Type                     = {Journal Article},
  Url                      = {http://dx.doi.org/10.1111/j.1600-0668.2007.00489.x}
}

@Article{Zhang2005,
  Title                    = {Measurement of the effect of cartilaginous rings on particle deposition in a proximal lung bifurcation model},
  Author                   = {Zhang, Y. and Finlay, W.H.},
  Journal                  = {Aerosol Science and Technology},
  Year                     = {2005},
  Pages                    = {394–399},
  Volume                   = {39},

  Type                     = {Journal Article}
}

@Article{Zhang2004,
  Title                    = {Particle deposition measurements and numerical simulation in a highly idealized mouth-throat},
  Author                   = {Zhang, Y. and Finlay, W.H. and Matida, E.A.},
  Journal                  = {Journal of Aerosol Science},
  Year                     = {2004},
  Number                   = {7},
  Pages                    = {789-803},
  Volume                   = {35},

  Type                     = {Journal Article}
}

@Article{Zhang2009,
  Title                    = {Improvements of Particle Near-Wall Velocity and Erosion Predicitions Using a Commercial CFD Code},
  Author                   = {Zhang, Yongli and Mcaury, Brenton S. and Shirazi, Siamack A.},
  Journal                  = {Journal of Fluids Engineering},
  Year                     = {2009},
  Number                   = {3},
  Pages                    = {9},
  Volume                   = {131},

  Doi                      = {10.1115/1.3077139},
  Keywords                 = {CFD,particle tracking, particle-wall interaction, erosion modeling},
  Type                     = {Journal Article},
  Url                      = {http://scitation.aip.org/getabs/servlet/GetabsServlet?prog=normal&id=JFEGA4000131000003031303000001&idtype=cvips&gifs=yes}
}

@Article{Zhang2009a,
  Title                    = {Prediction of Particle Deposition onto Indoor Surfaces by CFD with a Modified Lagrangian Method},
  Author                   = {Zhang, Z. and Chen, Q.},
  Journal                  = {Atmospheric Environment},
  Year                     = {2009},
  Number                   = {2},
  Pages                    = {319-328},
  Volume                   = {43},

  Type                     = {Journal Article}
}

@Article{Zhang2006,
  Title                    = {Experimental measurements and numerical simulations of particle transport and distribution in ventilated rooms},
  Author                   = {Zhang, Z. and Chen, Q.},
  Journal                  = {Atmospheric Environment},
  Year                     = {2006},
  Note                     = {ISI Document Delivery No.: 047EW
Times Cited: 50
Cited Reference Count: 30
Zhang, Z Chen, Q
Pergamon-elsevier science ltd
Oxford},
  Number                   = {18},
  Pages                    = {3396-3408},
  Volume                   = {40},

  Abstract                 = {Prediction of particle dispersion and distribution in a room is very important for creating and maintaining a healthy indoor environment. The present study used a CFD program with a Lagrangian particle tracking method to predict particle dispersion and concentration distribution in ventilated rooms. Since the Lagrangian method could generate great uncertainty in particle concentration calculations, this study first investigated such uncertainty using a statistical approach. It was found that the stability of the concentration solution became well controlled as a sufficient number of particles were analyzed. Although the overall computational cost was considerable, the numerical results agreed well with associated experimental data. In all cases studied, particle size distribution was monodisperse, and particle diameter ranged from 0.31 to 4.5 mu m. Particle deposition rate was neglected, and particles were hence removed only by the ventilation system. Thus the particle removal performance of different ventilation systems can be evaluated. Three ventilation systems have been studied, including ceiling and side wall supply systems and an underfloor air distribution (UFAD) system. It was found that the UFAD system had a better particle removal performance than the ceiling and side wall supply systems in the study. However, resuspended particles at the floor level can still cause problems in an UFAD system. (c) 2006 Elsevier Ltd. All rights reserved.},
  Doi                      = {10.1016/j.atmosenv.2006.01.014},
  ISSN                     = {1352-2310},
  Keywords                 = {fine particles
indoor air
underfloor air distribution
ventilation
systems
Lagrangian particle tracking
CFD
channel flow
turbulent-flow
air-flow
deposition
dispersion
dynamics
motion
wall},
  Type                     = {Journal Article},
  Url                      = {<Go to ISI>://WOS:000237863600018}
}

@Article{Zhang2009b,
  Title                    = {Experimental and numerical investigation of airflow and contaminant transport in an airliner cabin mockup},
  Author                   = {Zhang, Zhao and Chen, Xi and Mazumdar, Sagnik and Zhang, Tengfei and Chen, Qingyan},
  Journal                  = {Building and Environment},
  Year                     = {2009},
  Number                   = {1},
  Pages                    = {85-94},
  Volume                   = {44},

  Abstract                 = {The study of airflow and contaminant transport in airliner cabins is very important for creating a comfortable and healthy environment. This paper shows the results of such a study by conducting experimental measurements and numerical simulations of airflow and contaminant transport in a section of half occupied, twin-aisle cabin mockup. The air velocity and air temperature were measured by ultrasonic and omni-directional anemometers. A gaseous contaminant was simulated by a tracer gas, sulfur hexafluoride or SF6, and measured by a photo-acoustic multi-gas analyzer. A particulate contaminant was simulated by 0.7 μm di-ethyl-hexyl-sebacat (DEHS) particles and measured by an optical particle sizer. The numerical simulations used the Reynolds averaged Navier–Stokes equations based on the RNG k–ε model to solve the air velocity, air temperature, and gas contaminant concentration; and employed a Lagrangian method to model the particle transport. The numerical results quantitatively agreed with the experimental data while some remarkable differences exist in airflow distributions. Both the experimental measurements and computer simulations were not free from errors. A complete and accurate validation for a complicated cabin environment is challenging and difficult.},
  Doi                      = {http://dx.doi.org/10.1016/j.buildenv.2008.01.012},
  ISSN                     = {0360-1323},
  Keywords                 = {Airflow
Transport and dispersion of aerosol contaminants
Measurements
CFD
Aircraft cabin},
  Type                     = {Journal Article},
  Url                      = {http://www.sciencedirect.com/science/article/pii/S036013230800022X}
}

@Article{Zhang2009c,
  Title                    = {Experimental and numerical investigation of airflow and contaminant transport in an airliner cabin mockup},
  Author                   = {Zhang, Zhao and Chen, Xi and Mazumdar, Sagnik and Zhang, Tengfei and Chen, Qingyan},
  Journal                  = {Building and Environment},
  Year                     = {2009},
  Pages                    = {85-94},
  Volume                   = {44},

  Abstract                 = {The study of airflow and contaminant transport in airliner cabins is very important for creating a comfortable and healthy environment. This paper shows the results of such a study by conducting experimental measurements and numerical simulations of airflow and contaminant transport in a section of half occupied, twin-aisle cabin mockup. The air velocity and air temperature were measured by ultrasonic and omni-directional anemometers. A gaseous contaminant was simulated by a tracer gas, sulfur hexafluoride or SF6, and measured by a photo-acoustic multi-gas analyzer. A particulate contaminant was simulated by 0.7 μm di-ethyl-hexyl-sebacat (DEHS) particles and measured by an optical particle sizer. The numerical simulations used the Reynolds averaged Navier–Stokes equations based on the RNG k–ε model to solve the air velocity, air temperature, and gas contaminant concentration; and employed a Lagrangian method to model the particle transport. The numerical results quantitatively agreed with the experimental data while some remarkable differences exist in airflow distributions. Both the experimental measurements and computer simulations were not free from errors. A complete and accurate validation for a complicated cabin environment is challenging and difficult.},
  Doi                      = {http://dx.doi.org/10.1016/j.buildenv.2008.01.012},
  ISSN                     = {0360-1323},
  Keywords                 = {aerosol
Aircraft
Airflow
and
cabin
CFD
contaminants
dispersion
Measurements
of
transport},
  Type                     = {Journal Article},
  Url                      = {http://ac.els-cdn.com/S036013230800022X/1-s2.0-S036013230800022X-main.pdf?_tid=e673d358-421f-11e4-a4d8-00000aab0f26&acdnat=1411366730_6a9ae9832e5cee69a527e18d36ceffc0}
}

@Article{Zhang2011,
  Title                    = {Computational analysis of airflow and nanoparticle deposition in a combined nasal-oral-tracheobronchial airway model},
  Author                   = {Zhang, Zhe and Kleinstreuer, Clement},
  Journal                  = {Journal of Aerosol Science},
  Year                     = {2011},
  Number                   = {3},
  Pages                    = {174-194},
  Volume                   = {42},

  Abstract                 = {In light of the exponentially increasing industrial production and consumer use of ultrafine particles, deposition in the human lung is of great environmental and biomedical concern, especially for children, asthmatics and the elderly. Considering spherical nanoparticles in the 1-100 nm mean-diameter range and different breathing routes with Qtotal=30 and 60 L/min, local deposition fractions and global surface concentrations were predicted employing an experimentally validated computer simulation model. It was found that the change in breathing route (from nasal to oral breathing) not only significantly influences nanoparticle deposition in the regions of nasal and oral cavities, nasopharynx and oropharynx, but also measurably affects depositions from pharynx to bronchial airways for tiny nanoparticles (<=5 nm). The effect of breathing routes on deposition of larger nanoparticles (>5 nm) after the pharynx tends to be minor. The impact of different outlet flow-rate ratios generated by downstream resistances, e.g., caused by airway inflammation or tumors, is discussed in this study as well. Specifically, different outlet pressures primarily influence the velocity profiles and nanoparticle deposition fractions at that particular branch and adjacent bifurcations. In addition, the impact of change in outlet flow rate ratio on total deposition is confined to all same-level bifurcations and direct upstream-level bifurcations. The mass transfer coefficients of depositing nanoparticles (in terms of Sherwood number) can be well correlated as a function of Reynolds number and Schmidt number. The influence of downstream resistance on the Sherwood number in bronchial airways is smaller than intra-subject effects, i.e., variations of bifurcation levels and geometric parameters.},
  Doi                      = {10.1016/j.jaerosci.2011.01.001},
  ISSN                     = {0021-8502},
  Keywords                 = {Breathing routes
Computational analysis
Laminar-to-turbulent airflow
Nanoparticle deposition efficiencies
Downstream resistances
Mass transfer correlation},
  Type                     = {Journal Article},
  Url                      = {http://www.sciencedirect.com/science/article/pii/S0021850211000097}
}

@Article{Zhang2011a,
  Title                    = {Computational analysis of airflow and nanoparticle deposition in a combined nasal-oral-tracheobronchial airway model},
  Author                   = {Zhang, Zhe and Kleinstreuer, Clement},
  Journal                  = {Journal of Aerosol Science},
  Year                     = {2011},
  Pages                    = {174-194},
  Volume                   = {42},

  Abstract                 = {In light of the exponentially increasing industrial production and consumer use of ultrafine particles, deposition in the human lung is of great environmental and biomedical concern, especially for children, asthmatics and the elderly. Considering spherical nanoparticles in the 1-100 nm mean-diameter range and different breathing routes with Q(total) = 30 and 60 L/min, local deposition fractions and global surface concentrations were predicted employing an experimentally validated computer simulation model. It was found that the change in breathing route (from nasal to oral breathing) not only significantly influences nanoparticle deposition in the regions of nasal and oral cavities, nasopharynx and oropharynx, but also measurably affects depositions from pharynx to bronchial airways for tiny nanoparticles (<= 5 nm). The effect of breathing routes on deposition of larger nanoparticles (> 5 nm) after the pharynx tends to be minor. The impact of different outlet flow-rate ratios generated by downstream resistances, e.g., caused by airway inflammation or tumors, is discussed in this study as well. Specifically, different outlet pressures primarily influence the velocity profiles and nanoparticle deposition fractions at that particular branch and adjacent bifurcations. In addition, the impact of change in outlet flow rate ratio on total deposition is confined to all same-level bifurcations and direct upstream-level bifurcations. The mass transfer coefficients of depositing nanoparticles (in terms of Sherwood number) can be well correlated as a function of Reynolds number and Schmidt number. The influence of downstream resistance on the Sherwood number in bronchial airways is smaller than intra-subject effects, i.e., variations of bifurcation levels and geometric parameters. (C) 2011 Elsevier Ltd. All rights reserved.},
  Doi                      = {10.1016/j.jaerosci.2011.01.001},
  ISSN                     = {0021-8502},
  Keywords                 = {aerosol deposition
bifurcation
Breathing routes
cavity
Computational analysis
Downstream resistances
extra-thoracic airway
human-lung
Laminar-to-turbulent airflow
Mass transfer correlation
Nanoparticle deposition efficiencies
nano-particles
patterns
respiratory system
transport
ultrafine particle deposition},
  Type                     = {Journal Article},
  Url                      = {http://ac.els-cdn.com/S0021850211000097/1-s2.0-S0021850211000097-main.pdf?_tid=ee68b182-421f-11e4-9cc4-00000aab0f26&acdnat=1411366744_53ab7f3acd8d24b363f10598be34184a}
}

@Article{Zhang2011b,
  Title                    = {Laminar-to-turbulent fluid–nanoparticle dynamics simulations: Model comparisons and nanoparticle-deposition applications},
  Author                   = {Zhang, Zhe and Kleinstreuer, Clement},
  Journal                  = {International Journal for Numerical Methods in Biomedical Engineering},
  Year                     = {2011},
  Number                   = {12},
  Pages                    = {1930-1950},
  Volume                   = {27},

  Doi                      = {10.1002/cnm.1447},
  ISSN                     = {2040-7947},
  Keywords                 = {laminar-to-turbulent flow
nanoparticle deposition
computer simulation
human upper airways},
  Type                     = {Journal Article},
  Url                      = {http://dx.doi.org/10.1002/cnm.1447}
}

@Article{Zhang2004a,
  Title                    = {Airflow structures and nano-particle deposition in a human upper airway model},
  Author                   = {Zhang, Z. and Kleinstreuer, C.},
  Journal                  = {Journal of Computational Physics},
  Year                     = {2004},
  Number                   = {1},
  Pages                    = {178-210},
  Volume                   = {198},

  Abstract                 = {Considering a human upper airway model, or equivalently complex internal flow conduits, the transport and deposition of nano-particles in the 1-150 nm diameter range are simulated and analyzed for cyclic and steady flow conditions. Specifically, using a commercial finite-volume software with user-supplied programs as a solver, the Euler-Euler approach for the fluid-particle dynamics is employed with a low-Reynolds-number k-[omega] model for laminar-to-turbulent airflow and the mass transfer equation for dispersion of nano-particles or vapors. Presently, the upper respiratory system consists of two connected segments of a simplified human cast replica, i.e., the oral airways from the mouth to the trachea (Generation G0) and an upper tracheobronchial tree model of G0-G3. Experimentally validated computational fluid-particle dynamics results show the following: (i) transient effects in the oral airways appear most prominently during the decelerating phase of the inspiratory cycle; (ii) selecting matching flow rates, total deposition fractions of nano-size particles for cyclic inspiratory flow are not significantly different from those for steady flow; (iii) turbulent fluctuations which occur after the throat can persist downstream to at least Generation G3 at medium and high inspiratory flow rates (i.e., Qin[greater-or-equal, slanted]30 l/min) due to the enhancement of flow instabilities just upstream of the flow dividers; however, the effects of turbulent fluctuations on nano-particle deposition are quite minor in the human upper airways; (iv) deposition of nano-particles occurs to a relatively greater extent around the carinal ridges when compared to the straight tubular segments in the bronchial airways; (v) deposition distributions of nano-particles vary with airway segment, particle size, and inhalation flow rate, where the local deposition is more uniformly distributed for large-size particles (say, dp=100 nm) than for small-size particles (say, dp=1 nm); (vi) dilute 1 nm particle suspensions behave like certain (fuel) vapors which have the same diffusivities; and (vii) new correlations for particle deposition as a function of a diffusion parameter are most useful for global lung modeling.},
  ISSN                     = {0021-9991},
  Keywords                 = {Inspiratory flow
Nano-size particle deposition
Human airways
Computational fluid-particle dynamics simulation},
  Type                     = {Journal Article},
  Url                      = {http://www.sciencedirect.com/science/article/B6WHY-4BRPJVJ-2/2/82efdff72f464b977b44bd9adb1342ae}
}

@Article{Zhang2003,
  Title                    = {Modeling of low Reynolds number turbulent flows in locally constricted conduits: A comparison study},
  Author                   = {Zhang, Z. and Kleinstreuer, C.},
  Journal                  = { AIAA Journal},
  Year                     = {2003},
  Pages                    = {831-840},
  Volume                   = {41},

  Type                     = {Journal Article}
}

@Article{Zhang2003a,
  Title                    = {Species heat and mass transfer in a human upper airway model},
  Author                   = {Zhang, Z. and Kleinstreuer, C.},
  Journal                  = {International Journal of Heat and Mass Transfer},
  Year                     = {2003},
  Number                   = {25},
  Pages                    = {4755-4768},
  Volume                   = {46},

  Abstract                 = {Steady 3-D airflow and scalar transport of ultrafine particles, dp<0.1 [mu]m, and fuel vapors within the human upper airways are simulated and analyzed for laminar as well as locally turbulent flow conditions. Presently, our respiratory system consists of two major segments of a simplified human cast replica, i.e., a representative oral airway from mouth to trachea (Generation 0) and a symmetric four-generation upper bronchial tree model (G0-G3). The simulation has been validated with experimental data in terms of ultrafine particle deposition efficiencies. The present computational results show the following: (1) At low breathing rates (Qin[approximate]15 l/min), ambient temperature variations ([Delta]Tmax=47 °C) influence the local velocity fields and vapor concentrations; however, the total and segmental deposition fractions of fuel vapor in the upper airway are essentially unaffected. (2) The inlet flow rate has a significant effect on vapor deposition, i.e., the higher the flow rate the lower the deposition fraction. (3) The convective heat transfer coefficient averaged over an individual bifurcation unit can be correlated as Nu=0.568(RePr)0.495 (600 < Re < 6000). (4) Two new Sherwood number correlations capture the convective mass transfer for the oral airway and individual bifurcations. The methodology outlined and physical insight provided can be also applied to other intake configurations, such as engine ports and inlets to air-breathing propulsion systems.},
  ISSN                     = {0017-9310},
  Type                     = {Journal Article},
  Url                      = {http://www.sciencedirect.com/science/article/B6V3H-498TRFF-7/2/e3850e5f5b59adc4fe58f0520fa2b672}
}

@Article{Zhang2002,
  Title                    = {Modeling of Low Reynolds Number Turbulent Flows in Locally Constricted Conduits: A Comparison Study},
  Author                   = {Zhang, Z. and Kleinstreuer, C. },
  Journal                  = {AIAA Journal},
  Year                     = {2002},
  Pages                    = {831-840},
  Volume                   = {41},

  Type                     = {Journal Article}
}

@Article{Zhang2001a,
  Title                    = {Effect of particle inlet distributions on deposition in a triple bifurcation lung airway model},
  Author                   = {Zhang, Z. and Kleinstreuer, C.},
  Journal                  = {Journal of Aerosol Medicine—Deposition Clearance and Effects in the Lung},
  Year                     = {2001},
  Number                   = {1},
  Pages                    = {13-29},
  Volume                   = {14},

  Type                     = {Journal Article}
}

@Article{Zhang2009d,
  Title                    = {Comparison of analytical and CFD models with regard to micron particle deposition in a human16-generation tracheobronchial airway model},
  Author                   = {Zhang, Z. and Kleinstreuer, C. and Kim, C.S.},
  Journal                  = {Journal of Aerosol Science},
  Year                     = {2009},
  Number                   = {1},
  Volume                   = {40},

  Type                     = {Journal Article}
}

@Article{Zhang2008a,
  Title                    = {Comparison of analytical and CFD models with regard to micron particle deposition in a human16-generation tracheobronchial airway model},
  Author                   = {Zhang, Z. and Kleinstreuer, C. and Kim, C.S.},
  Journal                  = {Journal of Aerosol Science},
  Year                     = {2008},
  Volume                   = {doi: 10.1016/j.jaerosci.2008.08.003},

  Type                     = {Journal Article}
}

@Article{Zhang2002a,
  Title                    = {Aerosol deposition efficiencies and upstream release positions for different inhalation modes in an upper bronchial airway model},
  Author                   = {Zhang, Z. and Kleinstreuer, C. and Kim, C.S.},
  Journal                  = {Aerosol Science and Technology},
  Year                     = {2002},
  Pages                    = {828-844},
  Volume                   = {36},

  Type                     = {Journal Article}
}

@Article{Zhang2002b,
  Title                    = {Cyclic micron-size particle inhalation and deposition in a triple bifurcation lung airway model.},
  Author                   = {Zhang, Z. and Kleinstreuer, C. and Kim, C.S.},
  Journal                  = {Journal Aerosol Science},
  Year                     = {2002},
  Pages                    = {257–281},
  Volume                   = {33},

  Type                     = {Journal Article}
}

@Article{Zhang2002c,
  Title                    = {Gas-solid two-phase flow in a triple bifurcating lung airway model},
  Author                   = {Zhang, Z. and Kleinstreuer, C. and Kim, C.S.},
  Journal                  = {Int. Journal Multiphase Flow},
  Year                     = {2002},
  Pages                    = {1021-1046},
  Volume                   = {28},

  Type                     = {Journal Article}
}

@Article{Zhang2002d,
  Title                    = {Micro-particle transport and deposition in a human oral airway model},
  Author                   = {Zhang, Z. and Kleinstreuer, C. and Kim, C.S.},
  Journal                  = {Aerosol Science},
  Year                     = {2002},
  Pages                    = {1635-1652},
  Volume                   = {33},

  Type                     = {Journal Article}
}

@Article{Zhang2001b,
  Title                    = {Flow structure and particle transport in a triple bifurcation airway model. },
  Author                   = {Zhang, Z. and Kleinstreuer, C. and Kim, C.S.},
  Journal                  = {Journal of Fluids Engineering—Transactions of the ASME},
  Year                     = {2001},
  Number                   = {2},
  Pages                    = {320-330},
  Volume                   = {123},

  Type                     = {Journal Article}
}

@Article{Zhang2002e,
  Title                    = {Aerosol transport and deposition in a triple bifurcation bronchial airway model with local tumours},
  Author                   = {Zhang, Z. and Kleinstreuer, C. and Kim, C.S. and Hickey, A.J.},
  Journal                  = {Inhalation Toxicology},
  Year                     = {2002},
  Pages                    = {1111-1133},
  Volume                   = {14},

  Type                     = {Journal Article}
}

@Article{Zhang2001c,
  Title                    = {Effects of curved inlet tubes on air flow and particle deposition in bifurcating lung models},
  Author                   = {Zhang, Z. and Kleinstreuer, C. and Kim, C. S.},
  Journal                  = {Journal of Biomechanics},
  Year                     = {2001},
  Number                   = {5},
  Pages                    = {659-669},
  Volume                   = {34},

  Abstract                 = {In vivo bifurcating airways are complex and the airway segments leading to the bifurcations are not always straight, but curved to various degrees. How do such curved inlet tubes influence the motion as well as local deposition and hence the biological responses of inhaled particulate matter in lung airways? In this paper steady laminar dilute suspension flows of micron-particles are simulated in realistic double bifurcations with curved inlet tubes, i.e., 0°[less-than-or-equals, slant][theta][less-than-or-equals, slant]90°, using a commercial finite-volume code with user-enhanced programs. The resulting air-flow patterns as well as particle transport and wall depositions were analyzed for different flow inlet conditions, i.e., uniform and parabolic velocity profiles, and geometric configurations. The curved inlet segments have quite pronounced effects on air-flow, particle motion and wall deposition in the downstream bifurcating airways. In contrast to straight double bifurcations, those with bent parent tubes also exhibit irregular variations in particle deposition efficiencies as a function of Stokes number and Reynolds number. There are fewer particles deposited at mildly curved inlet segments, but the particle deposition efficiencies at the downstream sequential bifurcations vary much when compared to those with straight inlets. Under certain flow conditions in sharply curved lung airways, relatively high, localized particle depositions may take place. The findings provide necessary information for toxicologic or therapeutic impact assessments and for global lung dosimetry models of inhaled particulate matter.},
  ISSN                     = {0021-9290},
  Keywords                 = {Double airway bifurcations
Bent parent tubes
Uniform and parabolic inlet velocity profiles
Aerosol transport and deposition
Computational parametric sensitivity analyses},
  Type                     = {Journal Article},
  Url                      = {http://www.sciencedirect.com/science/article/B6T82-42SXF5K-C/2/e774b56138905249ec94ce311dd97633}
}

@Article{Zhang2007a,
  Title                    = {Evaluation of various turbulence models in predicting airflow and turbulence in enclosed environments by CFD: part-2: Comparison with experimental data from literature},
  Author                   = {Zhang, Z. and Zhang, W. and Zhai, Z. and Chen, Q.},
  Journal                  = { HVAC\&R Research},
  Year                     = {2007},
  Number                   = {6},
  Pages                    = {871-886},
  Volume                   = {13},

  Type                     = {Journal Article}
}

@Article{Zhao2009,
  Title                    = {Modeling of ultrafine particle dispersion in indoor environments with an improved drift flux model},
  Author                   = {Zhao, Bin and Chen, Chun and Tan, Zhongchao},
  Journal                  = {Journal of Aerosol Science},
  Year                     = {2009},
  Note                     = {doi: DOI: 10.1016/j.jaerosci.2008.09.001},
  Number                   = {1},
  Pages                    = {29-43},
  Volume                   = {40},

  ISSN                     = {0021-8502},
  Keywords                 = {Indoor air quality (IAQ)
Ultrafine particle
Dispersion/distribution
Drift flux model
Computational fluid dynamics},
  Type                     = {Journal Article},
  Url                      = {http://www.sciencedirect.com/science/article/B6V6B-4TG9HYJ-1/2/1b6ded8765c13f31522a039c0b6c57c0}
}

@Article{Zhao2004,
  Title                    = {Comparison of indoor aerosol particle concentration and deposition in different ventilated rooms by numerical method.},
  Author                   = {Zhao, B. and Zhang, Y. and Li, X. and Yang, X. and Huang, D.T. },
  Journal                  = {Building and Environment},
  Year                     = {2004},
  Pages                    = {1-8},
  Volume                   = {39},

  Type                     = {Journal Article}
}

@Article{Zhao2007,
  Title                    = {The way the wind blows Implications of modeling nasal airflow},
  Author                   = {Zhao, K. and Dalton, P.},
  Journal                  = {Current Allergy and Asthma Reports},
  Year                     = {2007},
  Number                   = {2},
  Pages                    = {117-124},
  Volume                   = {7},

  Type                     = {Journal Article}
}

@Article{Zhao2006,
  Title                    = {Numerical Modeling of Turbulent and Laminar Airflow and Odorant Transport during Sniffing in the Human and Rat Nose},
  Author                   = {Zhao, Kai and Dalton, Pamela and Yang, Geoffery C. and Scherer, Peter W.},
  Journal                  = {Chem. Senses},
  Year                     = {2006},
  Number                   = {2},
  Pages                    = {107-118},
  Volume                   = {31},

  Abstract                 = {Human sniffing behavior usually involves bouts of short, high flow rate inhalation (>300 ml/s through each nostril) with mostly turbulent airflow. This has often been characterized as a factor enabling higher amounts of odorant to deposit onto olfactory mucosa than for laminar airflow and thereby aid in olfactory detection. Using computational fluid dynamics human nasal cavity models, however, we found essentially no difference in predicted olfactory odorant flux (g/cm2 s) for turbulent versus laminar flow for total nasal flow rates between 300 and 1000 ml/s and for odorants of quite different mucosal solubility. This lack of difference was shown to be due to the much higher resistance to lateral odorant mass transport in the mucosal nasal airway wall than in the air phase. The simulation also revealed that the increase in airflow rate during sniffing can increase odorant uptake flux to the nasal/olfactory mucosa but lower the cumulative total uptake in the olfactory region when the inspired air/odorant volume was held fixed, which is consistent with the observation that sniff duration may be more important than sniff strength for optimizing olfactory detection. In contrast, in rats, sniffing involves high-frequency bouts of both inhalation and exhalation with laminar airflow. In rat nose odorant uptake simulations, it was observed that odorant deposition was highly dependent on solubility and correlated with the locations of different types of receptors.},
  Doi                      = {10.1093/chemse/bjj008},
  Type                     = {Journal Article},
  Url                      = {http://chemse.oxfordjournals.org/cgi/content/abstract/31/2/107}
}

@Article{Zhao2014,
  Title                    = {Regional peak mucosal cooling predicts the perception of nasal patency},
  Author                   = {Zhao, K. and Jiang, J. and Blacker, K. and Lyman, B. and Dalton, P. and Cowart, B.J. and Pribitkin, E.A.},
  Journal                  = {Laryngoscope},
  Year                     = {2014},
  Note                     = {cited By (since 1996)6},
  Number                   = {3},
  Pages                    = {589-595},
  Volume                   = {124},

  Document_type            = {Article},
  Source                   = {Scopus},
  Url                      = {http://www.scopus.com/inward/record.url?eid=2-s2.0-84894488287&partnerID=40&md5=344b1e924dab55a74241522691a13b70}
}

@Article{Zhao2014589,
  Title                    = {Regional peak mucosal cooling predicts the perception of nasal patency},
  Author                   = {Zhao, K. and Jiang, J. and Blacker, K. and Lyman, B. and Dalton, P. and Cowart, B.J. and Pribitkin, E.A.},
  Journal                  = {Laryngoscope},
  Year                     = {2014},
  Note                     = {cited By (since 1996)6},
  Number                   = {3},
  Pages                    = {589-595},
  Volume                   = {124},

  Document_type            = {Article},
  Source                   = {Scopus},
  Url                      = {http://www.scopus.com/inward/record.url?eid=2-s2.0-84894488287&partnerID=40&md5=344b1e924dab55a74241522691a13b70}
}

@Article{LARY:LARY24265,
  Title                    = {Regional peak mucosal cooling predicts the perception of nasal patency},
  Author                   = {Zhao, Kai and Jiang, Jianbo and Blacker, Kara and Lyman, Brian and Dalton, Pamela and Cowart, Beverly J. and Pribitkin, Edmund A.},
  Journal                  = {The Laryngoscope},
  Year                     = {2014},
  Number                   = {3},
  Pages                    = {589--595},
  Volume                   = {124},

  Doi                      = {10.1002/lary.24265},
  ISSN                     = {1531-4995},
  Keywords                 = {Nasal congestion, nasal obstruction, TRPM8, nasal cooling, cool perception, nasal trigeminal sensitivity},
  Url                      = {http://dx.doi.org/10.1002/lary.24265}
}

@Article{Zhao2004a,
  Title                    = {Effect of Anatomy on Human Nasal Air Flow and Odorant Transport Patterns: Implications for Olfaction},
  Author                   = {Zhao, Kai and Scherer, Peter W. and Hajiloo, Shoreh A. and Dalton, Pamela},
  Journal                  = {Chem. Senses},
  Year                     = {2004},
  Number                   = {5},
  Pages                    = {365-379},
  Volume                   = {29},

  Abstract                 = {Recent studies that have compared CT or MRI images of an individual's nasal anatomy and measures of their olfactory sensitivity have found a correlation between specific anatomical areas and performance on olfactory assessments. Using computational fluid dynamics (CFD) techniques, we have developed a method to quickly (<few days) convert nasal CT scans from an individual patient into an anatomically accurate 3-D numerical nasal model that can be used to predict airflow and odorant transport, which may ultimately determine olfactory sensitivity. The 3-D model can be also be rapidly modified to depict various anatomical deviations, such as polyps and their removal, that may alter nasal airflow and impair olfactory ability. To evaluate the degree to which variations in critical nasal areas such as the olfactory slit and nasal valve can alter airflow and odorant transport, inspiratory and expiratory airflow with odorants were simulated using numerical finite volume methods. Results suggest that anatomical changes in the olfactory region (upper meatus below the cribriform plate) and the nasal valve region will strongly affect airflow patterns and odorant transport through the olfactory region, with subsequent effects on olfactory function. The ability to model odorant transport through individualized models of the nasal passages holds promise for relating anatomical deviations to generalized or selective disturbances in olfactory perception and may provide important guidance for treatments for nasal-sinus disease, occupational rhinitis and surgical interventions that seek to optimize airflow and improve deficient olfactory function.},
  Doi                      = {10.1093/chemse/bjh033},
  Type                     = {Journal Article},
  Url                      = {http://chemse.oxfordjournals.org/cgi/content/abstract/29/5/365}
}

@Article{Zhao1994,
  Title                    = {Steady inspiratory flow in a model symmetric bifurcation.},
  Author                   = {Zhao, Y. and Lieber, B.B.},
  Journal                  = {Journal Biomech. Eng.},
  Year                     = {1994},
  Pages                    = {488-496},
  Volume                   = {116},

  Type                     = {Journal Article}
}

@Article{Zheng2009,
  Title                    = {The influence of Saffman lift force on nanoparticle concentration distribution near a wall},
  Author                   = {Zheng, Xu and Silber-Li, Zhanhua},
  Journal                  = {Applied Physics Letters},
  Year                     = {2009},
  Number                   = {12},
  Pages                    = {124105-3},
  Volume                   = {95},

  Keywords                 = {microchannel flow
nanofluidics
nanoparticles},
  Type                     = {Journal Article},
  Url                      = {http://link.aip.org/link/?APL/95/124105/1}
}

@Article{Zhou2005,
  Title                    = {Particle deposition in a cast of human tracheobronchial airways},
  Author                   = {Zhou, Y. and Cheng, Y.S.},
  Journal                  = {Aerosol Science and Technology},
  Year                     = {2005},
  Pages                    = {492-500},
  Volume                   = {39},

  Type                     = {Journal Article}
}

@Article{Zhou2014,
  Title                    = {Nasal Deposition in Infants and Children},
  Author                   = {Zhou, Yue and Guo, Mindy and Xi, Jinxiang and Irshad, Hammad and Cheng, Yung-Sung},
  Journal                  = {Journal of Aerosol Medicine and Pulmonary Drug Delivery},
  Year                     = {2014},
  Pages                    = {110-116},
  Volume                   = {27},

  Abstract                 = {Background: The variability of particle deposition in infant and child nasal airways is significant due to the airway geometry and breathing rate. Estimation of particle deposition in the nasal airway of this age group is necessary, especially for inhalation drug delivery application. Previous studies on nasal aerosol deposition were focused mostly on adult. A few empirical equations were also developed to calculate nasal deposition in different age groups of children. However, those studies have their limitations. The aim of this study is to find a simple way to calculate the nasal aerosol deposition in all age groups. Methods: An in vitro test of micrometer particle deposition in nasal airways for three different ages of infants and children is conducted. An adult nasal replica is also studied as a comparison. Monodisperse oleic acid aerosols ranging in size between 2 and 28 mu m are delivered into the replica at the rest condition. This size range covers the deposition efficiency up to around 100%. This study also compares results from our previous deposition tests with a 5-year-old replica. Results: Nasal deposition of micrometer aerosols in small children and infants is higher than that in adults under equivalent breathing conditions, e.g., sitting awake in this study. Combining the data set of infants, children, and adults, we found the deposition in the nasal airway strongly depends on the particle size and pressure drop. The particle deposition can be calculated based on a single empirical equation in all age groups. The intersubject variability within the same age group was not addressed in this study. Conclusions: An empirical equation for all age groups is developed. From this equation, particle deposition efficiency in the nasal airway can best be estimated with input data of particle size and pressure drop of the airway.},
  Doi                      = {10.1089/jamp.2013.1039},
  ISSN                     = {1941-2711},
  Keywords                 = {adults
aerosol deposition
aerosol distribution
airways
cavity
central-nervous-system
geometry
lung function
modeling
particle deposition
replicas
resistance
Rhinomanometry},
  Type                     = {Journal Article},
  Url                      = {http://online.liebertpub.com/doi/abs/10.1089/jamp.2013.1039?url_ver=Z39.88-2003&rfr_id=ori:rid:crossref.org&rfr_dat=cr_pub%3dpubmed&}
}

@Article{Zhu2007,
  Title                    = {Discrete particle simulation of particulate systems: Theoretical developments. },
  Author                   = {Zhu, H.P. and Zhou, Z.Y. and Yang, R.Y. and Yu, A.B.},
  Journal                  = {Chemical Engineering Science},
  Year                     = {2007},
  Pages                    = {3378-3396},
  Volume                   = {62},

  Type                     = {Journal Article}
}

@Article{Zhu2011,
  Title                    = {Evaluation and comparison of nasal airway flow patterns among three subjects from Caucasian, Chinese and Indian ethnic groups using computational fluid dynamics simulation},
  Author                   = {Zhu, Jian Hua and Lee, Heow Pueh and Lim, Kian Meng and Lee, Shu Jin and Wang, De Yun},
  Journal                  = {Respiratory Physiology \& Neurobiology},
  Year                     = {2011},
  Number                   = {1},
  Pages                    = {62-69},
  Volume                   = {175},

  Abstract                 = {Nasal airflow is one of the most important determinants for nasal physiology. During the long evolution of human beings, different races have developed their own attributes of nasal morphologies which result in variations of nasal airflow patterns and nasal functions. This study evaluated and compared the effects of differences of nasal morphology among three healthy male subjects from Caucasian, Chinese and Indian ethnic groups on nasal airflow patterns using computational fluid dynamics simulation. By examining the anterior nasal airway, the nasal indices and the nostril shapes of the three subjects were found to be similar to nasal cavities of respective ethnic groups. Computed tomography images of these three subjects were obtained to reconstruct 3-dimensional models of nasal cavities. To retain the flow characteristics around the nasal vestibules, a 40 mm-radius semi sphere was assembled around the human face for the prescription of zero ambient gauge pressure. The results show that more airflow tends to pass through the middle passage of the nasal airway in the Caucasian model, and through the inferior portion in the Indian model. The Indian model was found with extremely low flow flux flowing through the olfactory region. The sizes of vortexes near the anterior cavity were found to be correlated with the angles between the upper nasal valve wall and the anterior head of the nasal cavity.},
  Doi                      = {10.1016/j.resp.2010.09.008},
  ISSN                     = {1569-9048},
  Keywords                 = {Ethnic group
Nasal cavity
Nasal index
Airflow
CFD},
  Type                     = {Journal Article},
  Url                      = {http://www.sciencedirect.com/science/article/pii/S1569904810003642}
}

@Article{Zhu2011a,
  Title                    = {Comparing Gravimetric and Real-Time Sampling of PM2.5 Concentrations Inside Truck Cabins},
  Author                   = {Zhu, Ying and Smith, Thomas J. and Davis, Mary E. and Levy, Jonathan I. and Herrick, Robert and Jiang, Hongyu},
  Journal                  = {Journal of Occupational and Environmental Hygiene},
  Year                     = {2011},
  Number                   = {11},
  Pages                    = {662-672},
  Volume                   = {8},

  Doi                      = {10.1080/15459624.2011.617234},
  ISSN                     = {1545-9624},
  Type                     = {Journal Article},
  Url                      = {http://dx.doi.org/10.1080/15459624.2011.617234}
}

@Article{Zhuang2010,
  Title                    = {Facial Anthropometric Differences among Gender, Ethnicity, and Age Groups},
  Author                   = {Zhuang, Ziqing and Landsittel, Douglas and Benson, Stacey and Roberge, Raymond and Shaffer, Ronald},
  Journal                  = {Annals of Occupational Hygiene},
  Year                     = {2010},
  Number                   = {4},
  Pages                    = {391-402},
  Volume                   = {54},

  Abstract                 = {Objectives: The impact of race/ethnicity upon facial anthropometric data in the US workforce, on the development of personal protective equipment, has not been investigated to any significant degree. The proliferation of minority populations in the US workforce has increased the need to investigate differences in facial dimensions among these workers. The objective of this study was to determine the face shape and size differences among race and age groups from the National Institute for Occupational Safety and Health survey of 3997 US civilian workers.Methods: Survey participants were divided into two gender groups, four racial/ethnic groups, and three age groups. Measurements of height, weight, neck circumference, and 18 facial dimensions were collected using traditional anthropometric techniques. A multivariate analysis of the data was performed using Principal Component Analysis. An exploratory analysis to determine the effect of different demographic factors had on anthropometric features was assessed via a linear model. The 21 anthropometric measurements, body mass index, and the first and second principal component scores were dependent variables, while gender, ethnicity, age, occupation, weight, and height served as independent variables.Results: Gender significantly contributes to size for 19 of 24 dependent variables. African-Americans have statistically shorter, wider, and shallower noses than Caucasians. Hispanic workers have 14 facial features that are significantly larger than Caucasians, while their nose protrusion, height, and head length are significantly shorter. The other ethnic group was composed primarily of Asian subjects and has statistically different dimensions from Caucasians for 16 anthropometric values. Nineteen anthropometric values for subjects at least 45 years of age are statistically different from those measured for subjects between 18 and 29 years of age. Workers employed in manufacturing, fire fighting, healthcare, law enforcement, and other occupational groups have facial features that differ significantly than those in construction.Conclusions: Statistically significant differences in facial anthropometric dimensions (P < 0.05) were noted between males and females, all racial/ethnic groups, and the subjects who were at least 45 years old when compared to workers between 18 and 29 years of age. These findings could be important to the design and manufacture of respirators, as well as employers responsible for supplying respiratory protective equipment to their employees.},
  Doi                      = {10.1093/annhyg/meq007},
  Type                     = {Journal Article},
  Url                      = {http://annhyg.oxfordjournals.org/content/54/4/391.abstract}
}

@Article{Zinreich1988,
  Title                    = {Concha bullosa: CT evaluation},
  Author                   = {Zinreich, S.J. and Mattox, D.E. and Kennedy, D.W. and Chisholm, H.L. and Diffley, D.M. and Rosenbaum, A.E.},
  Journal                  = {J Comput Assist Tomogr},
  Year                     = {1988},
  Number                   = {5},
  Pages                    = {778-784},
  Volume                   = {12},

  Type                     = {Journal Article}
}

@Article{Zwartz2001,
  Title                    = {Effect of flow rate on particle deposition in a replica of a human nasal airway.},
  Author                   = {Zwartz, G.J. and Guilmette, R.A.},
  Journal                  = {Inhalation Toxicology},
  Year                     = {2001},
  Number                   = {2},
  Pages                    = {109-127},
  Volume                   = {13},

  Type                     = {Journal Article}
}

@Article{Zwartz2000,
  Title                    = {Effect of the nasal vestibule on particle deposition in a model of a human nasal airway},
  Author                   = {Zwartz, G. J. and Smuin, S. R. and Guilmette, R. A.},
  Journal                  = {Journal of Aerosol Science},
  Year                     = {2000},
  Note                     = {doi: DOI: 10.1016/S0021-8502(00)90139-3},
  Number                   = {Supplement 1},
  Pages                    = {132-133},
  Volume                   = {31},

  ISSN                     = {0021-8502},
  Keywords                 = {particle deposition efficiency
nasal airway
inter-subject variation},
  Type                     = {Journal Article},
  Url                      = {http://www.sciencedirect.com/science/article/B6V6B-487T30N-2G/2/a7bff0e369ae6c9eb2ff18080d5166dc}
}

@Article{1996,
  Title                    = {Annual literature survey 1995: Multiphase flow},
  Journal                  = {International Journal of Multiphase Flow},
  Year                     = {1996},
  Number                   = {Supplement 1},
  Pages                    = {89-153},
  Volume                   = {22},

  ISSN                     = {0301-9322},
  Type                     = {Journal Article},
  Url                      = {http://www.sciencedirect.com/science/article/B6V45-4CDK2TC-5/2/37e5a544ca27c4d9c76c40e560efa8af}
}

@Article{1996a,
  Title                    = {In situ end-labelling, light microscopic assessment and ultrastructure of apoptosis in lung carcinoma : Graffney EF, O'Neill AJ, Staunton MJ. Department of Histopathology, St James's Hospital, Dublin 8. J Clin Pathol 1995; 48: 1017-1021},
  Journal                  = {Lung Cancer},
  Year                     = {1996},
  Number                   = {2-3},
  Pages                    = {384-384},
  Volume                   = {14},

  ISSN                     = {0169-5002},
  Type                     = {Journal Article},
  Url                      = {http://www.sciencedirect.com/science/article/B6T9C-3Y3YTH0-3H/2/0f63a1fcb7c89b4207482e7396f6f721}
}

