\section{Fluid dynamics}
The aim of Computational fluid dynamics is to approximate numerically the physical conservation laws of newtonian physics:

\begin{itemize}

  \item conservation of mass

  \item The conservation of momentum (Newton’s second law, the rate of change of momentum equals the sum of forces acting on the fluid);

  \item The conservation of energy (first law of thermodynamics, the rate of change of energy equals the sum of rate of heat addition to and the rate of work done on the fluid).

\end{itemize}

    \subsection{Mass Conservation}

    The principle of conservation of mass is that, in a closed system, the mass remains constant. This means that fluid will move through a set region in such a way the mass is conserved. For an incompressible flow, this means that the outflows and the inflows will be equal. This can be written as
    \begin{equation} \label{eq:1}
      0 = \sum_{in} \dot{m} - \sum_{out} \dot{m}
    \end{equation}

    The mass flow rate can be written as $ \rho u A $, where, for flow in the x direction, $ A = \Delta z \Delta x $. For a two dimensional flow, $ \Delta z = 1 $, giving:

    \begin{equation} \label{eq:2}
      \dot{m}_{in} = \rho u \Delta y
    \end{equation}

    For a compressible flow:


    \begin{equation} \label{eq:3}
      \dot{m}_{out} = [\rho u + \Delta x] \Delta y
    \end{equation}




\section{Governing Equations}

\section{Time Dependant}

\section{Heat Transfer}

\section{Species Transport}
