\section{Fluid dynamics}
The aim of Computational fluid dynamics is to approximate numerically the physical conservation laws of newtonian physics:

\begin{itemize}

  \item conservation of mass

  \item The conservation of momentum (Newton’s second law, the rate of change of momentum equals the sum of forces acting on the fluid);

  \item The conservation of energy (first law of thermodynamics, the rate of change of energy equals the sum of rate of heat addition to and the rate of work done on the fluid).

\end{itemize}

    \subsection{Mass Conservation}

    The principle of conservation of mass is that, in a closed system, the mass remains constant. This means that fluid will move through a set region in such a way the mass is conserved. For an incompressible flow, this means that the outflows and the inflows will be equal. This can be written as
    \begin{equation} \label{eq:1}
      0 = \sum_{in} \dot{m} - \sum_{out} \dot{m}
    \end{equation} 

    Where $\dot{m}$ = mass flow rate.
    
    The mass flow rate can be written as $ \rho u A $,  for $\rho$ = density, $u$ = velocity and A is the scross sectional area of the flow. For flow in the x direction, $ A = \Delta z \Delta x $. For a two dimensional flow, $ \Delta z = 1 $, giving:

    \begin{equation} \label{eq:2}
      \dot{m}_{in} = \rho u \Delta y
    \end{equation}

    Extending this to equation \ref{eq:1}, in the x direction, for an incompressible flow we get

    \begin{equation} \label{eq:3}
      0 = \rho u_{in} \Delta y_{in} - \rho u_{out} \Delta y_{out}
    \end{equation}

    This can easily be extrapolated to three dimensions.

    In the case of the nasal cavity, this can be conceptualised in relation to the nasal cavity geometry

    \subsection{Momentum Conservation}

    Momentum conservation is based on the Newton's second law,

    $ \sum F = ma $

    Here $m$ is the mass of the system, $a$ is its rate of acceleration, and  $\sum F$ is the sum of forces acting on the system. F can generally be divided in to body and surface forces. Rewriting the mass as the product of volume and the density, and acceleration as the first derivative of velocity:


    \begin{equation} \label{eq:4}
      \sum F_{body} + \sum F_{surface} = (\rho \Delta x \Delta y \Delta Z) \frac{DU}{Dt}
    \end{equation}

    Body forces generally include gravity, centrifugal, Coriolis and electromagnetic forces; these all act on the volume from a distance.

    Surface forces are those that act directly on the surface of a fluid element. These fluid forces include normal stress, in the x direction $\sigma_{xx}$, which is made up of pressure forces $p$ exerted on the body and normal viscous stress components $tau_{xx}$; and tangential stresses, $\tau_{xy}$ and $\tau_{xz}$.

    Summing these forces in the x direction:
    

    \begin{equation} \label{eq:4}
      \sum F_{surface, x} = [\sigma_{xx} \Delta y \Delta z - (\frac{\delta \sigma_{xx}}{\delta x} \Delta x) \Delta y \Delta z] + [(\tau_{xy} + \frac{\delta \tau_{yx}}{\delta y} \Delta y) \Delta x \Delta z -
    \end{equation}

\section{Governing Equations}

\section{Time Dependant}

\section{Heat Transfer}

\section{Species Transport}
