Aberrations in nasal cavity form and functionality associated with the ageing process are important issues. 
Their importance is growing as the global population ages.
The underlying mechanisms behind many pathologies associated with the ageing process are still largely unknown.

This thesis presents findings from a set of preliminary investigations into in to older nasal cavity anatomy and airflow characteristics.
Although the sample size here is limited, it provides some preliminary insights in to questions raised by previous experimental results looking at the effects of old age on nasal cavity geometry and airflow.

In particular it provides a clear mechanistic alternative take on the relationship between geometry and nasal resistance to that provided by previous investigations using rhinomanometry and acoustic rhinometry. These (rhinomanometry and acoustic rhinometry) results show no clear relationship between cavity volume and resistance, whereas our computational investigation shows a clear relationship. Although a definitive answer is not offered as to the nature of the observed discrepency in findings, it does serve to identify the lack of understanding of this area in the current literature.

It also provides a first look at mechanistic variations in older nasal cavity airflow, as well as previously unseen details regarding geometric formations.
Significant variations in flow concentration and developement are seen between the more and less voluminous cavities, particularly in regions such as the nasal valve, cited as being of particular significance to flow developement. The impact of this on flow developement is clearly seen in Figure \ref{fig:peakvel}, where a clear relationship is seen between valve cross sectional area and overall peak velocity.

Preliminary examinations are made into heat and vapour transfer within the cavities, suggestion some relationship between geometry and these functions in the nasal cavity.

\section{Future works}
The first gap left answered by this thesis is as to the validity of the results for the more voluminous cavities.
Although the results make sense when compared with previous results from rhinomanometry, as well as pipe flow theory, for completeness the results should be validated experimentally.

Another significant area which should be investigated is the impact of the aberrations observed in this thesis on particle deposition zones in the nasal cavity.
This area is significant for both olfaction and particle filtration.
It is the hope and intention of the author of this thesis that the insights provided through this thesis will lay the groundwork for future developments in the field of geriatric rhinology
