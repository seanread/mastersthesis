Evaluation of nasal respiratory function for diagnostic or research purposes relies on categorisation of an individual's demographics, including age, gender, and ethnicity. It is known that normal changes in inter- or intra-individual nasal cavities manifest through age. This is especially true of intranasal volume, which is thought to increase due to hormonal changes leading to mucosal atrophy \cite{Kalmovich2005, Lindemann2010}. Edelstein \cite{Edelstein1996}  and Lindemann et al. \cite{Lindemann2008}  found correlations between aging and the occurrence of a range of rhinological conditions (e.g. dry cold itchy noses; crusting, postnasal dripping and obstructed nasal breathing) although their association with nasal morphology was not established. Thus, a detailed analysis of the nasal morphology and its effect on respiratory function in older adult patients may contribute towards this understanding.

Demographic factors such as age, gender, and ethnicity play an important role in
determining one’s nasal anatomy and physiology. For example, it was found that males exhibited larger nasal cavities e.g. larger normal pharyngeal areas in males over females \cite{Huang1998, BROOKS1992}. However, very few studies have quantified these differences. In terms of geographic variations with age, Kalmovich, et al. \cite{Kalmovich2005} found a significant increase in nasal cavity volume and minimal cross sectional area with age using acoustic rhinometry on 165 patients. Later, Lindemann, et al. \cite{Lindemann2010} used acoustic rhinometry in conjunction with rhinomanometry and validated questionnaires to survey the nasal cavities of eighty patients. While the questionnaire and rhinomanometry did not show any difference between the older and younger subjects, a marked increase in the volume of the nasal cavity with age was found from the acoustic rhinometry data. More recently, Loftus, et al. \cite{Loftus2016}  found higher intranasal volumes with age on CT volumetric analysis.

In the past, quantifying differences in nasal airflow has been limited by a lack of sensitive tools that can quantify the complexities of the sinonasal cavity. More recently, CFD has been proven to be a useful tool for detailed analysis of nasal airflow. With this new tool, normative data based on demographics is required so that these can be compared with disease states. We aimed at investigating the nasal airflow anatomy and physiology of older Chinese adult males to understand geometry features and dynamics of inhaled flow field in this specific demographic group.

Our geometry analyses revealed the significant influence of the area to perimeter ratio on fluid flow development in the nasal cavity. The thinner cavities exhibited higher wall shear stresses due to sharper velocity gradients that were produced. The narrow geometry would also increase heat and vapour transfer as there is less distance for the heat energy and vapour to transport from the wall to condition the inhaled air.

The cross-sectional area of slices taken along the nasal cavity, provided insight into the patient’s airway patency based on the degree of openness of the nasal chamber. These observations (from Figure \ref{fig:area}) can be compared with cross sectional outlines in Figure \ref{fig:sil} which allow predictions of the likely relationships between fluid flow properties caused by the individual nasal geometry features.

A shape factor in the form of the circularity,   described the overall shape of each cross-sectional slice along the nasal cavity geometry. A circularity of 1 implies a circular shape, although this can be masked by irregular shapes that coincidently have the same values as a smooth circle. A value close to zero, however is more definite providing a strong suggestion of a jagged and non-circular shape. The sharp increase in the circularity plot indicated where merging of the two chambers occurs. The circularity is a well-established measurement commonly used in image analysis of particles, and this is the first time it has been applied for shape detection in nasal cavity geometries. The results in this study showed that circularity is useful as an indicator and quantifiable metric for identifying the anterior, middle, and posterior nasal cavity regions.

Fluid flow analysis showed the relationship between inhaled fluid dynamics and its impact on nasal sensation in the form of fluid-surface shear. Regions of high acceleration were often caused by the airway converging into a smaller cross-section thereby accelerating the flow through.  This corresponded with a high wall shear stress region in the vicinity of the minimised cross-section. The results showed that the middle region partition in this study incorporated the minimum cross-sectional area into its region, which is known to occur around the nasal valve, although it could be considered as part of the anterior nasal cavity.

The pressure drop across the cavities as a function of volume corresponds well with pipe flow theory analytically, and the level of variation observed is not excessive when compared with the level of variance seen in previous studies using rhinomanometry \cite{Edelstein1996}. The relationship between volume and pressure drop is interesting, however, as it is not seen in previous studies examining relationships between aging, volume and pressure drop \cite{Lindemann2008}. This is likely related to the limitations of acoustic rhinometry in measuring cavity volume past the turbinal region of the nasal cavity. Experimental validation of the pressure drops across the models presented in this study would go further to confirm this theory.

The 48yo and 60yo models were the smallest in size and this resulted in higher wall shear stress magnitudes. Based on the geometry analysis, peak velocities and wall shear stresses should occur at the minimal cross-sections, and this corroborated well with the highest values indeed found in the 48yo and 60yo models. Locally, the highest wall shear stress concentrations occurred around the internal nasal valve, and this is consistent across all the models. Interestingly the nasopharynx also exhibited a significant increase in wall shear stress, and this was due to the cross-sectional area decreasing at the nasopharynx.
The influence of nasal valve geometry on flow development is clearly seen, particularly in NC09 which exhibits an expanded nasal valve. As a result the peak velocity in the turbinal region is lower in comparison to the sections cross sectional area when compared with the other models presented here.
The wall shear stress profiles along cross-sectional slices showed peak wall shear stresses produced inferiorly on the internal nasal valve while there was very low wall shear stresses superiorly. For the turbinate slices, the velocity contours showed the bulk flow and peak velocity were offset from the nasal floor. This is the continuation of the fluid flow that was rising from the nasal valve region. The main nasal passage expands but the flow is unable to reach the new extremities.

The static pressure variation shown in Figure \ref{fig:stpr} merits some discussion. Although it would seem logical that an increased cavity volume would have the effect of reducing pressure drop, this is not the result that has been recorded by previous researchers using rhinomanometry \cite{Lindemann2008}. This discrepancy is possibly due to the flexibility of the nasopharynx, varying the diameter of the exit in ways that are not replicated in the ct models.

The more even distribution of wall shear stress - both sagittally, as seen in Figures \ref{fig:wcont} and \ref{fig:wax}; and coronally, as seen in Figure \ref{fig:wcs} - is indicative of variations in airflow structure and therefore performance of the nasal cavity models. In particular the variation in the local maxima for WSS found in the region of the nasal valve is indicative of varying air conditioning capacity.

It seems in particular that it is only in the more severely enlarged cavities that drastically increased vorticity is seen in the air flow structures. The volume of the NC07 model of 55.07 $cm^3$.

It seems probable that these marked increases in vorticity are linked to the observed reductions in air-conditioning functionality in specimens from this age group \cite{Lindemann2009a}. This is in accordance with the findings of \cite{Garcia2007} in relation to atrophic rhinitis patients. The model in their study had a recorded volume of 34.5 $cm^3$, however this was only measured to the end of the septum, and a comparison with the models from this study, as seen in Figure ~\ref{fig:area} shows that the volume through the turbinal region of the atrophic rhinitis model was significantly larger than that of the largest model from this study, NC07. Although a quantity is not given for the volume of the nasopharynx in the atrophic rhinitis model, qualitative comparison between the  images of the model and those of the models from this study seem to show that the significant expansion of the nasopharynx which is present in all of the elderly models was not present in the model of the younger (26 yo) atrophic rhinitis sufferer. 

Previous research has suggested that the presence of significant vortices such as those seen in the larger of the models presented in this study is related to the impairment of air conditioning functionality. The specifics of this mechanism , however, are yet to be investigated, and this is a piece of work which should be completed in future. In addition, the impact of the observed variations in airflow structure on the ability of the nasal cavity to filter particles from the air has not been investigated, and this is also an area which could be of some significance.

The heat and vapour transfer figures show what seems to be close to an inverse linear relationship between the voluminousness of the cavities and their heat and vapour transfer efficacy. This is particularly pronounced in the nasal valve. This relationship agrees well with the data from Garcia et al \cite{Garcia2007} taken from their atrophic rhinitis patient. This suggests that the mechanisms causing this reduced efficacy in the atrophic rhinitis patient are the same as those in elderly patients; this may imply that, for more extreme cases, that the surgical procedures implemented for atrophic rhinitis patients could also be effective for the elderly. It has been suggested by \cite{Lindemann2008} that heat and vapour transfer are significantly effected by the minimal cross sectional area. The current data doesn't show anything to support or disprove this suggestion, as in general the models presenting higher MCAs are also presenting higher voluminousness. In fact, the model that shows the highest MCA relative to its voluminousness [NC09], shows very good heat and vapour transfer.

Detailed understanding of the associations among variabilities in nasal anatomy, age, and airflow dynamics will greatly improve current knowledge regarding the influence of conductive mechanisms on olfactory ability. Variabilities in nasal 
anatomy have traditionally been understood to influence olfaction \cite{Eiting2015, Craven2007}. Nasal profiles that have been postulated to be associated with improved olfaction include the dorsal conduit, which delivers inhaled odorant-laden air to the olfaction recess through enhanced olfactory airflow, and an enlarged olfactory recess at the posterior end of the nasal cavity \cite{Eiting2015, Craven2007, Eiting2014}. It has also been reported that anatomical changes in the olfactory cleft or the nasal valve region alter odorant transport to the olfactory epithelium \cite{Zhao2004a}. Furthermore, the incidence of olfactory dysfunction in the general population is unclear, but has been estimated to be between 1\% and 3\%, with an exponential increase of 24.5\% among individuals aged 53 to 97 years \cite{Landis2004, Braemerson2004, Wysocki1989, Hoffman1998, Murphy2002}. The present study provides preliminary understanding connecting nasal anatomy, age and airflow dynamics together.
