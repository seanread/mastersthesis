
The relationships between age and geometry have been investigated thoroughly through significant samples in previous studies using acoustic rhinometry\cite{Kalmovich2005, WhanKim2007, Edelstein1996}. What can be seen in the results from this study is the impact of the variation of geometry, in particular volume, on the air flow patterns within the cavity. 

What can be observed here in the geometries that cannot be observed through acoustic rhinometry is the surface area of the models. This allows the effective diameter to be calculated.  It can be seen that since the increase in the nasal volume is derived from a widening of the cavities, the surface area is effected significantly less than the volume, leading to increased effective diameter, which has been suggested to cause to increase vorticity in the cavities\cite{Garcia2007}.

The static pressure variation shown in figure \ref{fig:stpr} merits some discussion. Although it would seem logical that an increased cavity volume would have the effect of reducing the pressure drop seen in figure \ref{fig:stpr}, this is not the result that has been recorded by previous researchers using rhinomanometry\cite{Lindemann2008}.

It seems in particular that it is only in the more severely enlarged cavities that drastically increased vorticity is seen in the air flow structures. The volume of the NC07 model of 55.07 $cm^3$, however is not a-typical of specimens of this age range, with a study of 81 men reporting an average volume for males between 65 and 79 of 56.7 $cm^3$\cite{Kalmovich2005}.

It seems probable that these marked increases in vorticity are linked to the observed reductions in air-conditioning functionality in specimens from this age group\cite{Lindemann2009a}. This is in accordance with the findings of\cite{Garcia2007} in relation to atrophic rhinitis patients. The model in their study had a recorded volume of 34.5 $cm^3$, however this was only measured to the end of the septum, and a comparison with the models from this study, as seen in figure ~\ref{fig:area} shows that the volume through the turbinal region of the atrophic rhinitis model was significantly larger than that of the largest model from this study, NC07. Although a quantity is not given for the volume of the nasopharynx in the atrophic rhinitis model, qualitative comparison between the  images of the model and those of the models from this study seem to show that the significant expansion of the nasopharynx which is present in all of the elderly models was not present in the model of the younger (26 yo) atrophic rhinitis sufferer. 

previous research has suggested that the presence of significant vortices such as those seen in the larger of the models presented in this study is related to the impairment of air conditioning functionality. The specifics of this mechanism , however, are yet to be investigated, and this is a piece of work which should be completed in future. In addition, the impact of the observed variations in airflow structure on the ability of the nasal cavity to filter particles from the air has not been investigated, and this is also an area which could be of some significance.
