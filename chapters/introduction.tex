\section{Background}

The ageing global population increases the need for investment in geriatrics. One area of particular significance within this field is that of rhinology. A range of dysfunctions and aberrations from the normal functioning of nasal cavities in healthy adults have been observed in the elderly population\cite{Edelstein1996, Lindemann2008}. For example, the occurrence of respiratory diseases in the elderly is markedly higher than that found in younger populations\cite{HO2001, Edelstein1996}. It has been suggested that these higher recorded rates is due in part to the impaired air conditioning functionality\cite{Lindemann2008}.


The nasal cavities of elderly citizens have been shown by previous researchers to exhibit increased volume\cite{Kalmovich2005}. Alterations in histiological function have also been shown\cite{HO2001}. The extent to which functional aberrations are caused by geometric (as opposed to histiological) variations remains unclear\cite{Varga-Huettner2013}. These changes in nasal physiology - and their impact on nasal functionality - need to be investigated further.


Computational Fluid Dynamics (CFD) numerically approximate flow field characteristics for a given domain. CFD  simulations give highly detailed results with information covering a range of areas for a given fluid system at a minimal cost. One area in which CFD simulations have been gaining credence is that of biomechanics; many biological fluid systems can be modelled effectively through CFD simulations, allowing for an unprecedented insight into their function.

The use of 3D medical imageing techniques such as computed tomography (CT) scans in collaboration with CFD has facilitated the numerical approximation of fluid mechanism parameters in anatomically accurate models. This allows for more detailed analysis of the effects of topological variations on fluid mechanisms. This capacity for detailed comparison makes CFD simulation an ideal tool for analysing the impact of physiological discrepencies on nasal patency.


To date numerous inter-demographic studies have been carried out using CFD analysis of 3D models reconstructed from CT scan data\cite{Xi2012, Garcia2007, Zhu2011}. These studies have indicated that factors such as ethnicity and gender are liable to impact on nasal physiology. Previous studies have focused on age\cite{Xi2012}, however to date these age related studies have all focused on children. The use of CFD to examine functional variations in older models - accounting for potential interdemographic variations such as gender or ethnicity - is thus an important step that needs to be taken in order to achieve a more comprehensive understanding of the impact of ageing on nasal patency.


\section{Research questions and objectives}

In light of the information posed above, the following questions become pertinent:

\begin{itemize}

  \item How do variations in nasal cavity geometries in older Chinese Males influence the inhaled airflow mechanisms, contributing to respiratory ailments?

  \item How do the geometry variations impact on heat and vapour transfer within the cavity, contributing to respiratory ailments?

\end{itemize}

To address these issues, the following objectives are outlined:

\begin{itemize}

  \item Reconstruct a series of nasal cavity geometrie s from medical scans that represent a spread of geometric characteristics [such as volume and surface area] across the norm. The existing literature shows a clearly defined relationship between age and volume: these models will serve as a representative sample of the older population to be analysed computationally.

  \item Model airflow across the series of reconstructed nasal cavities using CFD with a steady state assumption; defining inlet conditions to approximate a resting rate of respiration. 

  \item Compare the simulation results between geometries. A variety of post processing methods are available to compare various aspects of fluid mechanic functionaility of nasal cavity models. The literature has shown clear discrepencies in the functionality of nasal cavities as a function of age; it is our intention through these measurements to examine in more detail the relationship between these variations and geometry.

  \item  Compare with results from the literature. 
\end{itemize}
 
\section{Research overview}

Medical image reconstruction technologies now allow researchers to reconstruct highly detailed, digital 3D representations of various anatomical structures from ct scans. When coupled with CFD simulations, this presents an unprecedented capacity for in depth analysis of physiological fluid flow mechanisms.

This study aims to use CFD analysis of CT scan data from the nasal cavities of a range of Asian males to investigate the impact of geometric variations between older nasal cavities on the airflow structures and air-conditioning capacity of the nasal cavity. Air flow mechanisms, heat transfer rates and humidification efficacy are analysed in order to arrive at a more precise understanding of the role of nasal geometry in the presentation of respiratory ailments.

This study represents, to the best of our knowledge, the first in depth, mechanistic, computational study undertaken into the role of nasal geometry in common rhinological symptoms observed in older patients.

\section{Thesis outline}

Herein a concise description is given of the thesis chapters' contents

\begin{description}
  \item{Chapter 2} reviews current literature related to the topic at hand. This is divided into sections such as nasal anatomy, rhinology and fluid dynamics. On the basis of this review clear research questions are determined and salient research methods developed.

  \item{Chapter 3} provides a detailed description of the process by which anatomical structures are modeled for CFD analysis and the technologies involved. The process by which the models are then approximated by computational meshing - as to make them solvable by CFD methods - is then described

  \item{Chapter 4} describes the basic principles of CFD, and how they are implemented for the solution of fluid systems defined by 3D models of nasal cavity geometries.

  \item{Chapter 5} presents the results of numerical simulations carried out for a series of older nasal geometries. The results show relationships between cavity geometry and wall shear stress as well as pressure drop. Heat and vapour transfer capacities are also presented.

  \item{Chapter 6} discusses the significance of the results presented in the previous chapter. It is postulated that the preliminary understanding of the relationhsip between age, geometry and airflow dynamics presented in this thesis could help to further our understanding of  olfactory mechanisms in the human nasal cavity.

  \item{Chapter 7} Summarises the key findings gleaned from this thesis and discusses possible future extensions and applications.

\end{description}
