\section{Background}

The ageing global population behooves a greater interest and investment in research and innovation in the area of geriatrics. One area of particular significance within the field of geriatrics is that of geriatric rhinology. A range of dysfunctions and aberrations from the normal functioning of healthy adults have been observed in the nasal cavities of elderly populations\cite{Edelstein1996, Lindemann2008}. These aberrations are liable to impact significantly on the quality of life of the sufferers. The nasal cavities of elderly citizens have been shown by previous researchers to show increased volume\cite{Kalmovich2005}. Alterations in histiological function have also been shown\cite{HO2001}. The extent to which functional aberrations are caused by geometric variations remains unclear\cite{Varga-Huettner2013}. The occurrence of respiratory diseases in the elderly is markedly higher than that found in younger populations\cite{HO2001, Edelstein1996}. It has been suggested that these higher recorded rates could be due in part to the impaired air conditioning functionality\cite{Lindemann2008}.

particle toxicology is an area which has been receiving increasing interest in recent decades. The potential health issues related to the inhalation of environmental hazards are multifarious and often life threatening. In order to minimise the physical cost to society of both man made and natural environmental toxins an understanding of the mechanisms by which the contaminants are being introduced in to the human body is imperative.

The use of computational fluid dynamics (CFD) to analyse nasal cavity flow dynamics is an area which has been receiving significant research attention in recent years. CFD simulations allow the achievement of highly detailed results with information covering a range of areas for a given fluid system at a minimal cost. Some of the more significant areas The use of 3d medical imaging techniques such as computed tomography (CT) scans in collaboration with CFD has facilitated the use CFD modeling techniques to approximate numerically fluid mechanism parameters of anatomically accurate models taken from models produced in vivo. This allows for more detailed comparisons of the effects of topological variations on the relevant fluid mechanisms.


The analysis of highly accurate models facilitated by the use of ct scans presents an opportunity for the analysis of inter-demographic variations in nasal cavity functionality. To date numerous inter-demographic studies have been carried out using CFD analysis of 3d models reconstructed from ct scan data. These demographic studies have included several focusing on age, however these age related studies have all focused on the variations between children and adults.


\section{Research questions and objectives}

In light of the information posed above, the following questions seem pertinent:

\begin{itemize}

  \item How do changes in nasal cavity geometry - caused by age - influence its airflow mechanisms?

  \item How do the same changes impact on heat and vapour transfer within the cavity?

\end{itemize}

To answer the aforementioned questions, the following objectives have been outlined:

\begin{itemize}

  \item reconstruct a series of nasal cavity geometries from medical scans that represent a spread of geometric characteristics [such as volume and surface area] across the norm. The existing literature shows a clearly defined relationship between age and volume: these models will serve as a representation of the aging population to be analysed compuationally.

  \item Model airflow across the series of reconstructed nasal cavities using computational fluid dynamic using a steady state assumption; defining inlet conditions to approximate a resting rate of respiration. 

  \item Compare the simulation results between geometries. A variety of post processing methods have been employed to compare various aspects of fluid mechanic functionaility of the nasal cavity models. The literature has shown clear discrepencies in the functionality of nasal cavities as a function of age; it is our intention through these measurements to examine in more detail the extent of these variations.

  \item  Compare results with existing results from various experimental methods from the previously extant literature. The results of the post processing Will be used to explain discrepancies in nasal functionality observed by previous researchers.

\end{itemize}
 
\section{Research outline}

Developments medical image reconstruction technologies now allow researchers to reconstruct highly detailed, digital 3d representations of various anatomical structures from ct scans. When coupled with CFD simulations, this presents an unprecedented capacity for in depth analysis of physiological fluid flow mechanisms.

This study aims to use CFD analysis of CT scan data from the nasal cavities of a range of Asian males to investigate the impact of the age induced expansion of the nasal cavity on the air-conditioning functionality of the nasal cavity. Air flow mechanisms, heat transfer rates and humidification efficacy is analysed. The Data is then discussed - in the context of the rhinological symptoms attributed by previous researches to elderly patients - in order to arrive at a more precise understanding of the role of nasal geometry in the presentation of said symptoms.

This study represents, to the best of our knowledge, the first in depth, mechanistic, computational study undertaken into the role of nasal geometry in common rhinological symptoms associated with the aging process.
